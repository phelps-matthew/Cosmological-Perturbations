\documentclass[10pt,letterpaper]{article}
\usepackage{mymacros}

\title{SVT Decomposition from Orthogonal Projections}
\author{}
\date{}

\begin{document}
\maketitle
Fundamental observers are locally at rest with respect to the matter fluid. Motivated by these ``preferred'' frames, we seek to split a given rank 2 tensor $T_{ab}$ into components parallel and orthogonal to a velocity vector $u_\mu$. The rest frames locally define surfaces of constant $t$. The induced metric for the surfaces of simultaneity is
\[
	h_{ab} = g_{ab} + u_a u_b.
\]
That this acts like a 3-space metric can verified by
\[
	h^a{}_b h^b{}_c =  h^a{}_c,\qquad h^c{}_c = 3, \qquad h^a{}_b u^b = 0.
\]
Note that the last relation uses $h^a{}_b$ to project the components orthogonal to $u^a$.  Likewise $U^a{}_b \equiv -u^au_b$ projects components parallel to $u_a$. \\ \\
We can use these projectors to decompose a tensor into components parallel and orthogonal to the local velocity. Take arbitrary symmetric rank 2 tensor $T_{ab}$
\begin{align}
	T_{ab} &= g^c{}_a g^d{}_b T_{cd}\\
	&= (h^c{}_a + U^c{}_a{})(h^d{}_b + U^d{}_b{})T_{cd}\\
	&= h^c{}_a h^d{}_b T_{cd} - u_a(u^c h^d{}_b T_{cd}) - u_b (u^d h^c{}_a T_{cd}) + u_a u_b (u^c u^d T_{cd}) .
\end{align}
This can be re-expressed as:
\begin{align}
	T_{ab}&= \frac13 h_{ab}h^{cd}T_{cd} + \bigg[ \frac12 h^c{}_a h^d{}_b+\frac12 h^c{}_b h^d{}_a - \frac13 h_{ab}h^{cd}\bigg] T_{cd}
+\bigg[\frac12 h^c{}_a h^d{}_{b}-\frac12 h^d{}_a h^c{}_{b}\bigg]T_{cd} \\
	&\quad - u_a(h^d{}_b T_{cd}u^c ) - u_b ( h^c{}_a T_{cd}u^d) + u_a u_b (u^c u^d T_{cd}).
\end{align}
In this form, we note the quantity
\[
	  \bigg[ \frac12 h^c{}_a h^d{}_b+\frac12 h^c{}_b h^d{}_a - \frac13 h_{ab}h^{cd}\bigg] T_{cd}
\]
is symmetric, trace-free, and orthogonal to $u^a$ and $u^b$, i.e. the projected symmetric tracefree (PTSF) part. 
We also note that the quantity
\begin{equation}
	\bigg[\frac12 h^c{}_a h^d{}_{b}-\frac12 h^d{}_a h^c{}_{b}\bigg]T_{cd}
\end{equation}
vanishes given a symmetric $T_{cd}$. 
Let us relabel the following quantities:
\begin{align}
	\rho &= u^cu^dT_{cd}\\
	p &= \frac13 h^{cd}T_{cd}\\
	q_a &  = -h^b{}_a{} u^c  T_{bc} \\
	\pi_{ab} &=  \bigg[ \frac12 h^c{}_a h^d{}_b+\frac12 h^c{}_b h^d{}_a - \frac13 h_{ab}h^{cd}\bigg] T_{cd}.
\end{align}
Now the energy momentum tensor may be expressed as 
\begin{align}
	T_{ab} &= u_a u_b \rho + h_{ab}p + u_a q_b + u_b q_a + \pi_{ab}\\
	&= (\rho + p)u_au_b + pg_{ab} + u_a q_b + u_b q_a + \pi_{ab}
\end{align} 
Note that $u^a q_a = 0$, and that $\pi_{ab}$ (projected symmetric traceless) has 5 components. Thus 2 scalars, 3 vector components, and 5 from the tensor give us 10 in total. \\ \\
In comoving coordinates in FLRW space, the velocity vector is
\[
	u^a = \delta^a_0,\qquad u_a = -\delta_a^0
\]
and thus the only non-zero components of $\pi_{ab}$ are $\pi_{ij}$ (spatial). According to York (1973) we may decompose a symmetric tensor on a positive definite Riemannian space as
\[
	\pi_{ab} = \pi_{ab}^{TT} + \pi_{ab}^L + \pi_{ab}^{Tr}
\]
where 
\[
	\pi_{ab}^{Tr} = \frac13 g_{ab}g^{cd}\pi_{cd}
\]
and (restricting to 3 dimensions)
\[
	\pi_{ab}^{L} = \nabla_a W_b + \nabla_b W_a -\frac23 g_{ab}\nabla_c W^c.
\]
By construction 
\[
	g^{ab}\pi_{ab}^{TT} = 0.
\]
The transverse requirement leads to an equation for the vector $W_a$
\[
	-\nabla_b \pi^{ab(L)} = - \nabla_b (\pi^{ab} -\frac13 g^{ab} g_{cd}\pi^{cd}).
\]
York shows that such a vector $W_a$ must exists and is unique, up to conformal Killing vectors. Moreover, he also shows that decomposition actually holds its form under conformal transformation on the metric. \\ \\
Going back to the symmetric traceless tensor $\pi_{ab}$, we may write this as
\begin{align}
	\pi_{ab} &= \pi_{ab}^{TT} + \pi_{ab}^L\\
	&=\pi_{ab}^{TT} +  \nabla_a W_b + \nabla_b W_a -\frac23 g_{ab}\nabla_c W^c.
\end{align}
\\ \\
Going back to the general symmetric tensor, if we instead make the substitutions
\begin{align}
	\rho &= -2\phi\\
	p &= \psi\\
	q_a &= -(B_a+\nabla_a B);\qquad \nabla_a B^a =0\\
	W_a &= (E_a +\nabla_a E);\qquad \nabla_a E^a \\
	\pi^{TT}_{ab} &= E_{ab}
\end{align}
then
\begin{align}
	\pi_{ab} &=\pi_{ab}^{TT} +  \nabla_a W_b + \nabla_b W_a -\frac23 g_{ab}\nabla_c W^c\\
	&= E_{ab} + \nabla_a E_b + \nabla_b E_a +2 \nabla_a\del_b E - \frac23h_{ab} \nabla^2 E.
\end{align}
It follow that we end up with the same form of the perturbation metric as given in the standard SVT decomposition:
\[
	T_{ab} = -2\phi u_a u_b - (B_b + \nabla_b B)u_a - (B_a+\nabla_a B)u_b - 2\gamma_{ab} \psi + \nabla_a E_b + \nabla _b E_a +
	+2E_{ab}.
\]
In flat space the spacetime interval is 
\[
	ds^2 =  -(1+2\phi)dt^2 + 2(B_i+\nabla_i B)dx^i dt + \blr{-2 \delta_{ij} \psi + (\nabla_i E_j + \nabla_j E_i) + 2\del_i \nabla_j E + 2E_{ij}}dx^i dx^j.
\]
Thus the SVT decomposition can be achieved first by orthogonal decomposition of a symmetric tensor relative to the four velocity, and then decomposing the projected symmetric trace-free portion into transverse and longitudinal components. 
\end{document}