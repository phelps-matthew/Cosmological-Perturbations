\documentclass[10pt,letterpaper]{article}
\usepackage[textwidth=7in, top=1in,textheight=9in]{geometry}
\usepackage[fleqn]{mathtools} 
\usepackage{amssymb}
\allowdisplaybreaks[0]
\title{Special Gauge Matthew v8}
\date{}
\begin{document}
\maketitle
\noindent
\section*{Setup}
Metric decomposed to first order:
\begin{equation}
g_{\mu\nu} = g^{(0)}_{\mu\nu} + g^{(1)}_{\mu\nu} = \Omega^2(\tau)(\eta_{\mu\nu}+h_{\mu\nu}).
\end{equation}
We then split $h_{\mu\nu}$ into its traceless and trace components, i.e.
\begin{equation}
	h_{\mu\nu} = K_{\mu\nu} + \frac 14 \eta_{\mu\nu}h
\end{equation}
where $h = \eta^{\mu\nu}h_{\mu\nu}$. 
We impose a generalized gauge of the form
\begin{equation}
	\eta^{\alpha\beta}\partial_{\alpha}K_{\beta\nu} = \Omega^{-1} J \eta^{\alpha\beta}K_{\nu\alpha}\partial_\beta \Omega + P \partial_\nu h + R \Omega^{-1} h \partial_\nu \Omega.
\end{equation}
With $J=-3$, the trace for $\Omega(\tau) = \frac{1}{H\tau}$ becomes
\begin{equation}
	\eta^{\mu\nu}\delta G_{\mu\nu} = 
	(-4 R \tau^{-2} -  \tfrac{3}{4} \eta^{\mu \nu} \partial_{\mu} \partial_{\nu} + P \eta^{\mu \nu} \partial_{\mu} \partial_{\nu} -  \tfrac{3}{2} \tau^{-1} \partial_{0} + 3 P \tau^{-1} \partial_{0} + R \tau^{-1} \partial_{0}) h
\end{equation}
With $J=-4$, the trace for $\Omega(\tau) = \frac{1}{H\tau}$ becomes
\begin{equation}
	\eta^{\mu\nu}\delta G_{\mu\nu} = 
(-3 R \tau^{-2} -  \tfrac{3}{4} \eta^{\mu \nu} \partial_{\mu} \partial_{\nu} + P \eta^{\mu \nu} \partial_{\mu} \partial_{\nu} -  \tfrac{3}{2} \tau^{-1} \partial_{0} + 2 P \tau^{-1} \partial_{0} + R \tau^{-1} \partial_{0}) h
\end{equation}
We seek to see if it is possible to find coefficients of $P$ and $R$ such that this reduces to the box operator onto a factor of $\Omega(\tau)$. There are two possible forms
\begin{equation}
	C\Omega^{-2}\eta^{\alpha\beta}\partial_\alpha\partial_\beta (\Omega^2 h)
	= C(\eta^{\mu \nu} \partial_{\mu} \partial_{\nu} - 6 \tau^{-2} + 4 \partial_{0} \tau^{-1}) h
\end{equation}
and
\begin{equation}
	C\Omega^{2}\eta^{\alpha\beta}\partial_\alpha\partial_\beta (\Omega^{-2} h)
	=C(\eta^{\mu \nu} \partial_{\mu} \partial_{\nu} - 2 \tau^{-2} - 4 \partial_{0} \tau^{-1}) h
\end{equation}
where $C$ is just an overall coefficient. 
\\ 
\section*{ $J=-3$, \ $\Omega^{-2}\eta^{\alpha\beta}\partial_\alpha\partial_\beta (\Omega^2 h)$}
Matching coefficients from (4) with (6) we find
\begin{align}
	C &= -\frac34 + P\\
	4C &= -\frac32 + 3P+R\\
	-6C &= -4R.
\end{align}
These three linearly independent equations will uniquely specify $C$, $P$ and $R$. Their solution is
\begin{equation}
	C = -\frac32,\qquad J=-3,\qquad P = -\frac34,\qquad R =- \frac94.
\end{equation}
With these constants, the fluctuation equations become:
\begin{equation}
\eta^{\mu\nu}\delta G_{\mu\nu} = -\frac32 \Omega^{-2}\eta^{\alpha\beta}\partial_\alpha\partial_\beta (\Omega^2 h)
\end{equation}
\begin{align}
\delta G_{00}={}&(\tfrac{1}{2} \eta^{\mu \nu} \partial_{\mu} \partial_{\nu}
 + 2 \tau^{-1} \partial_{0}) K_{00}
 + (\tfrac{5}{8} \eta^{\mu \nu} \partial_{\mu} \partial_{\nu}
 + \partial_{0} \partial_{0}) h.
\end{align}
\begin{align}
\delta G_{01}={}&\tfrac{1}{2} \tau^{-1} \partial_{1} K_{00}
 + (\tfrac{3}{2} \tau^{-2}
 + \tfrac{1}{2} \eta^{\mu \nu} \partial_{\mu} \partial_{\nu}
 + \tfrac{3}{2} \tau^{-1} \partial_{0}) K_{01}
 + (- \tfrac{7}{8} \tau^{-1} \partial_{1}
 + \partial_{1} \partial_{0}) h.
\end{align}
\begin{align}
\delta G_{11}={}&\tau^{-1} \partial_{1} K_{01}
 + (3 \tau^{-2}
 + \tfrac{1}{2} \eta^{\mu \nu} \partial_{\mu} \partial_{\nu}
 + \tau^{-1} \partial_{0}) K_{11}
 + (\tfrac{9}{4} \tau^{-2}
 -  \tfrac{5}{8} \eta^{\mu \nu} \partial_{\mu} \partial_{\nu}
 -  \tfrac{7}{4} \tau^{-1} \partial_{0}\nonumber\\
& + \partial_{1} \partial_{1}) h.
\end{align}
\begin{align}
\delta G_{12}={}&\tfrac{1}{2} \tau^{-1} \partial_{2} K_{01}
 + \tfrac{1}{2} \tau^{-1} \partial_{1} K_{02}
 + (3 \tau^{-2}
 + \tfrac{1}{2} \eta^{\mu \nu} \partial_{\mu} \partial_{\nu}
 + \tau^{-1} \partial_{0}) K_{12}
 + \partial_{2} \partial_{1} h.
\end{align}
\\
\section*{ $J=-3$, \ $\Omega^{2}\eta^{\alpha\beta}\partial_\alpha\partial_\beta (\Omega^{-2} h)$}
Matching coefficients from (4) with (7) we find
\begin{align}
	C &= -\frac34 + P\\
	-4C &= -\frac32 + 3P+R\\
	-2C &= -4R.
\end{align}
 Their solution is
\begin{equation}
	C = -\frac{1}{10},\qquad J=-3,\qquad P = \frac{13}{20},\qquad R =-\frac{1}{20}.
\end{equation}
\begin{equation}
\eta^{\mu\nu}\delta G_{\mu\nu} = -\frac{1}{10} \Omega^{2}\eta^{\alpha\beta}\partial_\alpha\partial_\beta (\Omega^{-2} h)
\end{equation}
\begin{align}
\delta G_{00}={}&(\tfrac{1}{2} \eta^{\mu \nu} \partial_{\mu} \partial_{\nu}
 + 2 \tau^{-1} \partial_{0}) K_{00}
 + (- \tfrac{3}{40} \eta^{\mu \nu} \partial_{\mu} \partial_{\nu}
 + \tfrac{2}{5} \tau^{-1} \partial_{0}
 -  \tfrac{2}{5} \partial_{0} \partial_{0}) h.
\end{align}
\begin{align}
\delta G_{01}={}&\tfrac{1}{2} \tau^{-1} \partial_{1} K_{00}
 + (\tfrac{3}{2} \tau^{-2}
 + \tfrac{1}{2} \eta^{\mu \nu} \partial_{\mu} \partial_{\nu}
 + \tfrac{3}{2} \tau^{-1} \partial_{0}) K_{01}
 + (\tfrac{9}{40} \tau^{-1} \partial_{1}
 -  \tfrac{2}{5} \partial_{1} \partial_{0}) h.
\end{align}
\begin{align}
\delta G_{11}={}&\tau^{-1} \partial_{1} K_{01}
 + (3 \tau^{-2}
 + \tfrac{1}{2} \eta^{\mu \nu} \partial_{\mu} \partial_{\nu}
 + \tau^{-1} \partial_{0}) K_{11}
 + (\tfrac{1}{20} \tau^{-2}
 + \tfrac{3}{40} \eta^{\mu \nu} \partial_{\mu} \partial_{\nu}
 + \tfrac{1}{20} \tau^{-1} \partial_{0}\nonumber\\
& -  \tfrac{2}{5} \partial_{1} \partial_{1}) h.
\end{align}
\begin{align}
\delta G_{12}={}&\tfrac{1}{2} \tau^{-1} \partial_{2} K_{01}
 + \tfrac{1}{2} \tau^{-1} \partial_{1} K_{02}
 + (3 \tau^{-2}
 + \tfrac{1}{2} \eta^{\mu \nu} \partial_{\mu} \partial_{\nu}
 + \tau^{-1} \partial_{0}) K_{12}
 -  \tfrac{2}{5} \partial_{2} \partial_{1} h.
\end{align}
\section*{ $J=-4$, \ $\Omega^{-2}\eta^{\alpha\beta}\partial_\alpha\partial_\beta (\Omega^2 h)$}
Matching coefficients from (5) with (6) we find
\begin{align}
	C &= -\frac34 + P\\
	4C &= -\frac32 + 2P+R\\
	-6C &= -3R.
\end{align}
 Since two equations are linearly dependent, their solution is
\begin{equation}
	R=2C,\qquad P=\frac34+C,\qquad\to\qquad R = 2P-\frac32.
\end{equation}
Hence we may vary $P$ such that the equations simplify the most (note that we could also have $C=0$ which is explored below). For $R = 2P-\tfrac32$ the fluctuation equations take the form
\begin{equation}
\eta^{\mu\nu}\delta G_{\mu\nu} =(P-\tfrac34) \Omega^{-2}\eta^{\alpha\beta}\partial_\alpha\partial_\beta (\Omega^{2} h)
\end{equation}
\begin{align}
\delta G_{00}={}&(-2 \tau^{-2}
 + \tfrac{1}{2} \eta^{\mu \nu} \partial_{\mu} \partial_{\nu}
 + 3 \tau^{-1} \partial_{0}) K_{00}
 + (\tfrac{3}{4} \tau^{-2}
 -  P \tau^{-2}
 + \tfrac{1}{4} \eta^{\mu \nu} \partial_{\mu} \partial_{\nu}
 -  \tfrac{1}{2} P \eta^{\mu \nu} \partial_{\mu} \partial_{\nu}\nonumber\\
& + P \tau^{-1} \partial_{0}
 + \tfrac{1}{4} \partial_{0} \partial_{0}
 -  P \partial_{0} \partial_{0}) h.
\end{align}
\begin{align}
\delta G_{01}={}&\tau^{-1} \partial_{1} K_{00}
 + (\tau^{-2}
 + \tfrac{1}{2} \eta^{\mu \nu} \partial_{\mu} \partial_{\nu}
 + 2 \tau^{-1} \partial_{0}) K_{01}
 + (- \tfrac{1}{2} \tau^{-1} \partial_{1}
 + P \tau^{-1} \partial_{1}
 + \tfrac{1}{4} \partial_{1} \partial_{0}\nonumber\\
& -  P \partial_{1} \partial_{0}) h.
\end{align}
\begin{align}
\delta G_{11}={}&\tau^{-2} K_{00}
 + 2 \tau^{-1} \partial_{1} K_{01}
 + (3 \tau^{-2}
 + \tfrac{1}{2} \eta^{\mu \nu} \partial_{\mu} \partial_{\nu}
 + \tau^{-1} \partial_{0}) K_{11}
 + (\tfrac{3}{4} \tau^{-2}
 -  P \tau^{-2}\nonumber\\
& -  \tfrac{1}{4} \eta^{\mu \nu} \partial_{\mu} \partial_{\nu}
 + \tfrac{1}{2} P \eta^{\mu \nu} \partial_{\mu} \partial_{\nu}
 -  \tau^{-1} \partial_{0}
 + P \tau^{-1} \partial_{0}
 + \tfrac{1}{4} \partial_{1} \partial_{1}
 -  P \partial_{1} \partial_{1}) h.
\end{align}
\begin{align}
\delta G_{12}={}&\tau^{-1} \partial_{2} K_{01}
 + \tau^{-1} \partial_{1} K_{02}
 + (3 \tau^{-2}
 + \tfrac{1}{2} \eta^{\mu \nu} \partial_{\mu} \partial_{\nu}
 + \tau^{-1} \partial_{0}) K_{12}
 + (\tfrac{1}{4} \partial_{2} \partial_{1}
 -  P \partial_{2} \partial_{1}) h.
\end{align}
Possible choices that allow simplification are $P\in (0,1,\tfrac14, \tfrac12,\tfrac34)$. For $P=\tfrac14$ the fluctuation equations reduce to
\begin{equation}
\eta^{\mu\nu}\delta G_{\mu\nu} =-\frac12 \Omega^{-2}\eta^{\alpha\beta}\partial_\alpha\partial_\beta (\Omega^{2} h)
\end{equation}
\begin{align}
\delta G_{00}={}&(-2 \tau^{-2}
 + \tfrac{1}{2} \eta^{\mu \nu} \partial_{\mu} \partial_{\nu}
 + 3 \tau^{-1} \partial_{0}) K_{00}
 + (\tfrac{1}{2} \tau^{-2}
 + \tfrac{1}{8} \eta^{\mu \nu} \partial_{\mu} \partial_{\nu}
 + \tfrac{1}{4} \tau^{-1} \partial_{0}) h.
\end{align}
\begin{align}
\delta G_{01}={}&\tau^{-1} \partial_{1} K_{00}
 + (\tau^{-2}
 + \tfrac{1}{2} \eta^{\mu \nu} \partial_{\mu} \partial_{\nu}
 + 2 \tau^{-1} \partial_{0}) K_{01}
 -  \tfrac{1}{4} \tau^{-1} \partial_{1} h.
\end{align}
\begin{align}
\delta G_{11}={}&\tau^{-2} K_{00}
 + 2 \tau^{-1} \partial_{1} K_{01}
 + (3 \tau^{-2}
 + \tfrac{1}{2} \eta^{\mu \nu} \partial_{\mu} \partial_{\nu}
 + \tau^{-1} \partial_{0}) K_{11}
 + (\tfrac{1}{2} \tau^{-2}
 -  \tfrac{1}{8} \eta^{\mu \nu} \partial_{\mu} \partial_{\nu}\nonumber\\
& -  \tfrac{3}{4} \tau^{-1} \partial_{0}) h.
\end{align}
\begin{align}
\delta G_{12}={}&\tau^{-1} \partial_{2} K_{01}
 + \tau^{-1} \partial_{1} K_{02}
 + (3 \tau^{-2}
 + \tfrac{1}{2} \eta^{\mu \nu} \partial_{\mu} \partial_{\nu}
 + \tau^{-1} \partial_{0}) K_{12}.
\end{align}
\\
\section*{ $J=-4$, \ $\Omega^{2}\eta^{\alpha\beta}\partial_\alpha\partial_\beta (\Omega^{-2} h)$}
Matching coefficients from (5) with (7) we find
\begin{align}
	C &= -\frac34 + P\\
	-4C &= -\frac32 + 2P+R\\
	-2C &= -3R.
\end{align}
 Here there is no solution for $C\ne 0$. However, if we do take
 \begin{equation}
	C =0,\qquad J=-3,\qquad P =\frac34,\qquad R =0,
\end{equation}
then we have a very simple equation for the trace, i.e.
\begin{equation}
\eta^{\mu\nu}\delta G_{\mu\nu} = \frac{3}{4\tau} \partial_0 h
\end{equation}
and thus
\begin{equation}
	h =\frac43\int d\tau\ \tau (\eta^{\mu\nu}\delta G_{\mu\nu}).
\end{equation}
The rest of the fluctuation equations take the form
\begin{align}
\delta G_{00}={}&(\tfrac{1}{2} \eta^{\mu \nu} \partial_{\mu} \partial_{\nu}
 + 2 \tau^{-1} \partial_{0}) K_{00}
 + (- \tfrac{1}{8} \eta^{\mu \nu} \partial_{\mu} \partial_{\nu}
 + \tfrac{3}{8} \tau^{-1} \partial_{0}
 -  \tfrac{1}{2} \partial_{0} \partial_{0}) h.
\end{align}
\begin{align}
\delta G_{01}={}&\tfrac{1}{2} \tau^{-1} \partial_{1} K_{00}
 + (\tfrac{3}{2} \tau^{-2}
 + \tfrac{1}{2} \eta^{\mu \nu} \partial_{\mu} \partial_{\nu}
 + \tfrac{3}{2} \tau^{-1} \partial_{0}) K_{01}
 + (\tfrac{1}{4} \tau^{-1} \partial_{1}
 -  \tfrac{1}{2} \partial_{1} \partial_{0}) h.
\end{align}
\begin{align}
\delta G_{11}={}&\tau^{-1} \partial_{1} K_{01}
 + (3 \tau^{-2}
 + \tfrac{1}{2} \eta^{\mu \nu} \partial_{\mu} \partial_{\nu}
 + \tau^{-1} \partial_{0}) K_{11}
 + (\tfrac{1}{8} \eta^{\mu \nu} \partial_{\mu} \partial_{\nu}
 + \tfrac{1}{8} \tau^{-1} \partial_{0}
 -  \tfrac{1}{2} \partial_{1} \partial_{1}) h.
\end{align}
\begin{align}
\delta G_{12}={}&\tfrac{1}{2} \tau^{-1} \partial_{2} K_{01}
 + \tfrac{1}{2} \tau^{-1} \partial_{1} K_{02}
 + (3 \tau^{-2}
 + \tfrac{1}{2} \eta^{\mu \nu} \partial_{\mu} \partial_{\nu}
 + \tau^{-1} \partial_{0}) K_{12}
 -  \tfrac{1}{2} \partial_{2} \partial_{1} h.
\end{align}
\end{document}
