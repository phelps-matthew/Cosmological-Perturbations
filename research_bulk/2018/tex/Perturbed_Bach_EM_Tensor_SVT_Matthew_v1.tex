\documentclass[10pt,letterpaper]{article}
\usepackage[textwidth=7in, top=1in,textheight=9in]{geometry}
\usepackage[fleqn]{mathtools} 
\usepackage{amssymb}

\title{$\delta W_{\mu\nu} = \delta T_{\mu\nu}$ (SVT) Matthew v1}
\date{}
\begin{document}
\maketitle
\noindent 
\section*{Perturbing $T_{\mu\nu}$}
We work in the perturbed geometry:
\begin{align}
ds^2 =&{} - g_{\mu\nu}dx^\mu dx^\nu= -\Omega^2(\eta_{\mu\nu}+f_{\mu\nu})dx^\mu dx^\nu \nonumber\\
=\ &{} \Omega^2(x)\bigg[ (1+2\phi)dt^2 - 2(\tilde\nabla_i B+ B_i)dtdx^i - [(1-2\psi)\delta_{ij} + 2\tilde\nabla_i\tilde\nabla_j E + \tilde\nabla_i E_j + \tilde\nabla_j E_i 
+ 2 E_{ij}]dx^idx^j\bigg].
\end{align}
According to (E6), via orthogonal projection to the four velocity $U^\mu$, we may decompose the rank 2 $T_{\mu\nu}$ as
\begin{equation}
T_{\mu\nu} = (\rho+p)U_\mu U_\nu + p g_{\mu\nu} + U_\mu q_\nu + U_\nu q_\mu + \pi_{\mu\nu}
\end{equation}
where
\begin{equation}
	U^\mu q_{\mu} = 0,\qquad U^\nu \pi_{\mu\nu} = 0,\qquad \pi_{\mu\nu} = \pi_{\nu\mu},\qquad g^{\mu\nu}\pi_{\mu\nu} =U^\mu U^\nu \pi_{\mu\nu} = 0.
\end{equation}
We will expand the above $T_{\mu\nu}$ up to first order as
\begin{equation}
	T_{\mu\nu} = T_{\mu\nu}^{(0)} + \delta T_{\mu\nu}.
\end{equation}
Since the background $g_{\mu\nu}^{(0)} = \Omega^2 \eta_{\mu\nu}$ is conformal to flat, it follows that $W_{\mu\nu}^{(0)} = T_{\mu\nu}^{(0)} = 0$. Hence the full $T_{\mu\nu}$ of (3) will be entirely first order. The first order quantities will be defined according to the convention
\begin{equation}
\rho^{(1)} = \delta \rho,\qquad p^{(1)} = \delta p,\qquad U^{(1)} = \delta U,\qquad q_\mu^{(1)} = q_\mu,\qquad \pi_{\mu\nu}^{(1)} = \pi_{\mu\nu}.
\end{equation}
Now (4) is evaluated as
\begin{equation}
	\delta T_{\mu\nu} = (\delta \rho + \delta p)U^{(0)}_{\mu}U^{(0)}_{\nu} + g_{\mu\nu}^{(0)}\delta p  + U_\mu^{(0)}q_\nu + U_\nu^{(0)}q_\mu + \pi_{\mu\nu}
\end{equation}
As goes the four velocity, we see we only need the background component. Given $-1 =  g_{\mu\nu}U^\mu U^\nu$, let us decompose the metric and $U^\mu$ in which the four velocity condition becomes
\begin{align}
	-1 =&{} g_{\mu\nu}U^\mu U^\nu\nonumber\\
-1=&{} (g_{\mu\nu}^{(0)} + h_{\mu\nu})(U_{(0)}^\mu+\delta U^\mu)(U_{(0)}^\nu +\delta U^\nu)\nonumber\\
-1 = &{} g_{\mu\nu}^{(0)} U^\mu_{(0)} U^\nu_{(0)} + g_{\mu\nu}^{(0)} U^\mu \delta U^\nu+ g_{\mu\nu}^{(0)} \delta U^\mu U^\nu
+ h_{\mu\nu} U^\mu_{(0)}U^\nu_{(0)}.
\end{align}
To zeroth order, we expect the background four velocity to remain unit time-like, viz.
\begin{align}
-1 =&{}  g_{\mu\nu}^{(0)} U^\mu_{(0)} U^\nu_{(0)}\nonumber \\
-1 =&{} \Omega^2(x) \eta_{\mu\nu}U^\mu_{(0)} U^\nu_{(0)}.
\end{align}
(We could also have noted that $\delta (g_{\mu\nu}U^\mu U^\nu) = 0$ to arrive at the same result). 
If we start with the background line element, denoted here as $ds_{(0)}^2 \equiv d\tau^2$, it follows
\begin{align}
	d\tau^2 =&{} - \Omega^2(x) \eta_{\mu\nu}dx^\mu dx^\nu\nonumber\\
= &{} \Omega^2(x) \big[ dt^2 - \delta_{ij} dx^i dx^j \big]\nonumber\\
= &{} \Omega^2(x) dt^2\bigg[1- \delta_{ij} \frac{dx^i}{dt} \frac{dx^j}{dt} \bigg]\nonumber\\
= &{} \Omega^2(x) dt^2(1-\mathbf{v}^2).
\end{align}
Defining the canonical time dilation factor as 
\begin{equation}
	\gamma^2 = \frac{1}{1-\mathbf v^2}
\end{equation}
we have
\begin{equation}
	d\tau^2 = \Omega^2 (x) \gamma^{-2} dt^2.
\end{equation}
To relate this to the four velocity, we note from (8) and (9) that
\begin{equation}
	U^{\mu}_{(0)} = \frac{dx^\mu}{d\tau} = \frac{dx^\mu}{dt}\bigg(\frac{dt}{d\tau}\bigg) = \frac{d x^\mu}{dt} \frac{\gamma}{\Omega(x)}.
\end{equation}
In the Roberston Walker metric, the spatial coordinates $x^i$ are comoving, i.e. $\frac{dx^i}{dt} = 0$, thus $\gamma = 1$ and the four velocity takes the form
\begin{equation}
	U_{(0)}^\mu = \Omega^{-1}(x) \delta^\mu_0,\qquad U_\mu^{(0)} = - \Omega(x) \delta_\mu^0.
\end{equation}
Given this four velocity, we now evaluate $\delta T_{\mu\nu}$ given in (6)
\begin{equation}
	\delta T_{\mu\nu} = \Omega^2(\delta \rho + \delta p)\delta^0_\mu \delta^0_\nu + \Omega^2  \eta_{\mu\nu}\delta p - \Omega \delta^0_\mu q_\nu -\Omega\delta^0_\nu q_\mu + \pi_{\mu\nu}.
\end{equation}
The components are:
\begin{equation}
\delta T_{00} = \Omega(x)^2 \delta \rho
\end{equation}
\begin{equation}
\delta T_{0i} = -\Omega(x) q_i
\end{equation}
\begin{equation}
\delta T_{ij} = \Omega^2(x) \delta_{ij} \delta p + \pi_{ij}
\end{equation}
Under infinitesmial coordinate tranformation $x^\mu \to x'^\mu = x^\mu - \epsilon^\mu$, a general vector $S_{\mu\nu}$ transforms as
\begin{equation}
	S_{\mu\nu}' = S_{\mu\nu} +S^{\lambda}{}_{\mu}\nabla_\nu \epsilon_\lambda +S^\lambda{}_{\nu}\nabla_\mu\epsilon_\lambda
+ \nabla^{\lambda}S_{\mu\nu}\epsilon_\lambda.
\end{equation}
We see that any $S_{\mu\nu}$ that is strictly first order will automatically be gauge invariant, and hence our $\delta T_{\mu\nu}$ is gauge invariant. 
\section*{Conformal Transformations}
Comparing $\delta W_{\mu\nu}$ to $\delta T_{\mu\nu}$, we note that $\delta T_{\mu\nu}$ does not have the same form with respect to the overall factor of $\Omega$. This is because a quantity such as $\delta \rho$, which we recall is defined in terms of projectors as
\begin{equation}
	\delta \rho = U_{(0)}^\sigma U_{(0)}^\tau \delta T_{\sigma\tau} = \Omega^{-2} \delta^\sigma_0 \delta^\tau_0 \delta T_{\sigma\tau},
\end{equation}
already has the conformal factors absorbed into its definition. This suggests that perhaps a more suitable forumulation may be to start with the flat space projected $\delta T_{\mu\nu}$ and then afterward peform a conformal transformation $g_{\mu\nu} \to \Omega^2 g_{\mu\nu}$. In this perscription, we would start by defining $\delta \rho$ within the geometry
\begin{equation}
ds^2 = \bigg[ (1+2\phi)dt^2 - 2(\tilde\nabla_i B+ B_i)dtdx^i - [(1-2\psi)\delta_{ij} + 2\tilde\nabla_i\tilde\nabla_j E + \tilde\nabla_i E_j + \tilde\nabla_j E_i 
+ 2 E_{ij}]dx^idx^j\bigg]
\end{equation}
as 
\begin{equation}
	\delta \rho = U_{(0)}^\sigma U_{(0)}^\tau \delta T_{\sigma\tau} =  \delta^\sigma_0 \delta^\tau_0 \delta T_{\sigma\tau}= \delta T_{00}.
\end{equation}
To see how $\rho$ changes, we note that via $W_{\mu\nu} = T_{\mu\nu}$, it must follow that when we perform a conformal transformation we have
\begin{equation}
T_{\mu\nu} \to T'_{\mu\nu} = \Omega^{-2}T_{\mu\nu},\qquad \delta T_{\mu\nu} \to \Omega^{-2}\delta T_{\mu\nu}
\end{equation}
and thereby
\begin{equation}
	\delta \rho \to \Omega^{-2}\delta \rho.
\end{equation}
By conformally transforming, we can now express $\delta T_{\mu\nu}$ within the full geometry of (1) as
\begin{equation}
\delta T_{00} = \Omega^{-2} \delta \rho
\end{equation}
\begin{equation}
\delta T_{0i} = -\Omega^{-2}q_i
\end{equation}
\begin{equation}
\delta T_{ij} = \Omega^{-2}\big( \delta_{ij} \delta p + \pi_{ij}\big )
\end{equation}
where the scalars, vectors, and tensors are defined in terms of \emph{flat} projectors and a flat $\delta T_{\mu\nu}^{(f)}$ as
\begin{align}
	\delta \rho =&{}  U_{(0)}^\sigma U_{(0)}^\tau \delta T^{(f)}_{\sigma\tau},\qquad \delta p =  \frac{1}{3} P_{(0)}^{\sigma\tau} \delta T^{(f)}_{\sigma\tau},\qquad
	q_\mu = -P_\mu{}^\sigma U_{(0)}^{\tau}\delta T^{(f)}_{\sigma\tau}\nonumber\\
	\pi_{\mu\nu} =&{} \bigg[ \frac12 P_\mu{}^\sigma P_\nu{}^\tau + \frac12 P_{\nu}{}^\sigma  P_\mu{}^\tau- \frac13 P_{\mu\nu}^{(0)}P_{(0)}^{\sigma\tau}\bigg]\delta 
T_{\sigma\tau}^{(f)}.
\end{align}
 
\section*{SVT Decomposition}
To bring $\delta T_{\mu\nu}$ closer to form of $\delta W_{\mu\nu}$ in the SVT basis, we follow appendix E and introduce
\begin{equation}
	Q = \int d^3y D^3(x-y) \tilde\nabla^i_y q_i
\end{equation}
such that
\begin{equation}
	 q_i = Q_i + \tilde\nabla_i Q,\qquad \tilde\nabla^i Q_i = 0.
\end{equation}
For $\pi_{\mu\nu}$, we recall that (evalauted in the geoemetry of (20)) it obeys 
\begin{equation}
	g^{\mu\nu}\pi_{\mu\nu} = U^\mu U^\nu \pi_{\mu\nu} = 0.
\end{equation} 
Via (E21), we may decompose the five component $\pi_{\mu\nu}$ into a transverse traceless $\pi_{ij}$, a divergenceless $\pi_i$, and a scalar $\pi$ as
\begin{equation}
	\pi_{ij} = -\frac{2}{3} \delta_{ij}\tilde\nabla^k \tilde\nabla_k \pi  + 2\tilde\nabla_i\tilde\nabla_j \pi + \tilde\nabla_i \pi_j + \tilde\nabla_j \pi_i + \pi_{ij}^{T\theta},
\end{equation}
where we have restricted to $D=3$ according to $U^\mu U^\nu \pi_{\mu\nu} = 0$. Now (24-26) can be expressed in the SVT form as
\begin{align}
\delta T_{00}  &= \Omega^{-2} \delta \rho,
\nonumber\\	
\delta T_{0i} &= -\Omega^{-2} ( Q_i + \tilde\nabla_i Q),
\nonumber\\	
\delta T_{ij}  &= \Omega^{-2}\bigg[ \delta_{ij} \delta p -\frac{2}{3} \delta_{ij}\tilde\nabla^k \tilde\nabla_k \pi + 2\tilde\nabla_i\tilde\nabla_j \pi + \tilde\nabla_i \pi_j + \tilde\nabla_j \pi_i + \pi_{ij}^{T\theta}\bigg]
\end{align}
\section*{Kinematics}
From (24-26), we see that we have two one component scalars $\delta \rho$ and $\delta p$, a three component $q_i$, and a five component traceless $\pi_{ij}$. This adds up to 10 as it should, however, we note from the SVT decomposition of $\delta W_{\mu\nu}$ that we only have five gauge invariant components. Thus there must be some remaining degrees of freedom that need to be reduced. While the quantity $\delta T_{\mu\nu}$ is gauge invariant, it still must obey the same kinematic relations as that of $\delta W_{\mu\nu}$ - namely it must be traceless and covariantly conserved. This amounts to five additional equations to fix the remaining gauge freedom, thus allowing reduction to five total components as expected. 
\\ \\
The trace condition is the same in either $\Omega(x)$ or $\Omega(x) = 1$ and is given as
\begin{equation}
	\eta^{\mu\nu}\delta T_{\mu\nu} = -\delta\rho + 3\delta p=0\qquad \Rightarrow\qquad   \delta \rho = 3\delta p.
\end{equation}
As for the covariant conservation, we have
\begin{equation}
	\nabla_\nu T^{\mu\nu} = \partial_\nu T^{\mu\nu} + T^{\nu\sigma}g^{\mu\rho}\partial_\nu g_{\rho\sigma} - \frac12 T^{\nu\sigma} g^{\mu\rho}\partial_\rho
g_{\nu\sigma} + \frac12 T^{\mu\sigma} g^{\nu\rho} \partial_\sigma g_{\nu\rho}.
\end{equation}
Since under a conformal transformation $T^{\mu\nu} \to \Omega^{-6}T^{\mu\nu}$, it can be verified that the covariant conservation law is also preserved
\begin{equation}
\nabla_\nu T^{\mu\nu} \to \Omega^{-6} \nabla_\nu T^{\mu\nu}.
\end{equation}
Therefore, we will work in the flat geometry of (20) to find the relations between S.V.T. components. First, we express the flat (32) in contravariant components:
\begin{align}
\delta T^{00}  &= \delta \rho,
\nonumber\\	
\delta T^{0i} &=  Q^i + \tilde\nabla^i Q,
\nonumber\\	
\delta T^{ij}  &=  \delta^{ij} \delta p -\frac{2}{3} \delta^{ij}\tilde\nabla^k \tilde\nabla_k \pi + 2\tilde\nabla^i\tilde\nabla^j \pi + \tilde\nabla^i \pi^j + \tilde\nabla^j \pi^i + \pi^{ij}_{T\theta}
\end{align}
Evaluating $\nabla_\nu T^{\mu\nu}$ yields one time equation
\begin{align}
0=&{}\partial_t \delta T^{00} + \tilde\nabla_i \delta T^{0i} \nonumber\\
-\partial_t\rho = &{} \tilde\nabla_i \tilde\nabla^i Q
\end{align}
and three space equations
\begin{align}
0=&{}\partial_t \delta T^{0i} + \tilde\nabla_j \delta T^{ij} \nonumber\\
0 = &{} \partial_t (Q^i + \tilde\nabla^i Q) + \tilde\nabla^i \delta p +\frac43 \tilde\nabla^i \tilde\nabla^k \tilde\nabla_k \pi + \tilde\nabla_k \tilde\nabla^k \pi^i.
\end{align}
Gathering the trace and conservation conditions, we have altogether
\begin{align}
 \delta \rho =&{} 3\delta p\\
-\partial_t\rho = &{} \tilde\nabla_i \tilde\nabla^i Q\\
0 = &{} \partial_t (Q^i + \tilde\nabla^i Q) + \tilde\nabla^i \delta p +\frac43 \tilde\nabla^i \tilde\nabla^k \tilde\nabla_k \pi + \tilde\nabla_k \tilde\nabla^k \pi^i.
\end{align}
From (40), it follows
\begin{equation}
Q = -\int d^3y D^3(\mathbf x-\mathbf y) \partial_t \delta \rho.
\end{equation}
Applying $\tilde\nabla_i$ to (41) and inserting (39-40) yields
\begin{equation}
	0 = -\partial_t^2 \delta \rho + \frac13 \tilde\nabla_k\tilde\nabla^k \delta\rho + \frac43 \tilde\nabla^l \tilde\nabla_l \tilde\nabla^k \tilde\nabla_k \pi
\end{equation}
in which we may solve for $\pi$ as
\begin{align}
\pi =&{} \frac34 \int d^3y D^3(\mathbf x-\mathbf y) \int d^3z D^3(\mathbf y-\mathbf z) \bigg[  \partial_t^2 \delta \rho - \frac13 \tilde\nabla_k^{(z)}\tilde\nabla^k_{(z)} \delta\rho
\bigg]\nonumber\\
=&{}  \frac34 \int d^3y D^3(\mathbf x-\mathbf y) \bigg[ \int d^3z D^3(\mathbf y-\mathbf z) \partial_t^2 \delta \rho - \frac13\delta\rho\bigg]
\end{align}
Now we insert $Q$ and $\pi$ back into (41) and solve for $Q_i$ and $\pi_i$
\begin{equation}
Q_i = - \tilde\nabla_k \tilde\nabla^k \int dt\  \pi_i 
\end{equation}
\begin{equation}
\pi_i = - \int d^3y D^3(\mathbf x-\mathbf y) \partial_t Q_i.
\end{equation}
Finally, we can express $\delta T_{\mu\nu}$ in terms of 5 components consisting of $\delta \rho$, $\pi_i$ and $\pi_{ij}^{T\theta}$ as
\begin{align}
\delta T_{00}  &= \Omega^{-2} \delta \rho,
\nonumber\\	
\delta T_{0i} &= \Omega^{-2} \bigg[ \tilde\nabla_k \tilde\nabla^k \int dt\  \pi_i  + \tilde\nabla_i  \int d^3y D^3(\mathbf x-\mathbf y) \partial_t \delta \rho\bigg],
\nonumber\\	
\delta T_{ij}  &= \Omega^{-2}\bigg[ 
\frac12 \delta_{ij} \delta\rho + \frac12 \delta_{ij} \int d^3y D^3(\mathbf x-\mathbf y) \partial_t^2 \delta \rho \nonumber\\
&\quad -\frac32 \tilde\nabla_i\tilde\nabla_j \int d^3y D^3(\mathbf x-\mathbf y) \bigg( \int d^3z D^3(\mathbf y-\mathbf z) \partial_t^2 \delta \rho + \frac13\delta\rho\bigg)
+ \tilde\nabla_i \pi_j + \tilde\nabla_j \pi_i + \pi_{ij}^{T\theta}\bigg].
\end{align}
This represents the general form for the perturbed energy momentum tensor within conformal gravity with vanishing background geometry. 
\section*{Fluctuation Equations}
\begin{align}
\delta W_{00}  &= -\frac{2}{3\Omega^2} \delta^{mn}\delta^{\ell k}\tilde{\nabla}_m\tilde{\nabla}_n\tilde{\nabla}_{\ell}\tilde{\nabla}_k (\phi + \psi +\dot{B}-\ddot{E}),
\nonumber\\	
\delta W_{0i} &=  -\frac{2}{3\Omega^2} \delta^{mn}\tilde{\nabla}_i\tilde{\nabla}_m\tilde{\nabla}_n\partial_t(\phi +\psi +\dot{B}-\ddot{E})
	+\frac{1}{2\Omega^2}\left[\delta^{mn}\delta^{\ell k}\tilde{\nabla}_m\tilde{\nabla}_n\tilde{\nabla}_{\ell}\tilde{\nabla}_k(B_i - \dot{E}_i) -  \delta^{\ell k}\tilde{\nabla}_{\ell}\tilde{\nabla}_k \partial_t^2(B_i - \dot{E}_i)\right],
\nonumber\\	
\delta W_{ij}  &= \frac{1}{3\Omega^2}\bigg{[} \delta_{ij}\delta^{\ell k}\tilde{\nabla}_{\ell}\tilde{\nabla}_k  \partial_t^2(\phi+ \psi+\dot{B}-\ddot{E}) + \delta^{\ell k}\tilde{\nabla}_{\ell}\tilde{\nabla}_k \tilde{\nabla}_i\tilde{\nabla}_j (\phi + \psi +\dot{B}-\ddot{E}) 
\nonumber\\
&- \delta_{ij} \delta^{mn}\delta^{\ell k}\tilde{\nabla}_m\tilde{\nabla}_n\tilde{\nabla}_{\ell}\tilde{\nabla}_k(\phi + \psi +\dot{B}-\ddot{E}) -3\tilde{\nabla}_i\tilde{\nabla}_j \partial_t^2(\phi + \psi +\dot{B}-\ddot{E})\bigg{] }
\nonumber\\
&+\frac{1}{2\Omega^2}\left[ \delta^{\ell k}\tilde{\nabla}_{\ell}\tilde{\nabla}_k \tilde{\nabla}_i   \partial_t(B_j - \dot{E}_j)+ \delta^{\ell k}\tilde{\nabla}_{\ell}\tilde{\nabla}_k \tilde{\nabla}_j \partial_t(B_i - \dot{E}_i) - \tilde{\nabla}_i\partial_t^3(B_j - \dot{E}_j)-\tilde{\nabla}_j\partial_t^3(B_i - \dot{E}_i)\right]
\nonumber\\
&+\frac{1}{\Omega^2}\left[\delta^{mn}\tilde{\nabla}_m\tilde{\nabla}_n-\partial_t^2\right]^2E_{ij}.
\label{AP75}
\end{align}
\begin{align}
\delta T_{00}  &= \Omega^{-2} \delta \rho,
\nonumber\\	
\delta T_{0i} &= \Omega^{-2} \bigg[ \tilde\nabla_k \tilde\nabla^k \int dt\  \pi_i  + \tilde\nabla_i  \int d^3y D^3(\mathbf x-\mathbf y) \partial_t \delta \rho\bigg],
\nonumber\\	
\delta T_{ij}  &= \Omega^{-2}\bigg[ 
\frac12 \delta_{ij} \delta\rho + \frac12 \delta_{ij} \int d^3y D^3(\mathbf x-\mathbf y) \partial_t^2 \delta \rho \nonumber\\
&\quad -\frac32 \tilde\nabla_i\tilde\nabla_j \int d^3y D^3(\mathbf x-\mathbf y) \bigg( \int d^3z D^3(\mathbf y-\mathbf z) \partial_t^2 \delta \rho + \frac13\delta\rho\bigg)
+ \tilde\nabla_i \pi_j + \tilde\nabla_j \pi_i + \pi_{ij}^{T\theta}\bigg].
\end{align}
\end{document}
