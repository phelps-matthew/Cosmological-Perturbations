\documentclass[10pt,letterpaper]{article}
\usepackage[textwidth=7in, top=1in,textheight=9in]{geometry}
\usepackage[fleqn]{mathtools} 
\usepackage{amssymb,hyperref,xcolor}
\usepackage{xcolor}
\hypersetup{
    colorlinks,
    linkcolor={blue!50!black},
    citecolor={red!50!black},
    urlcolor={blue!80!black}
}
\setlength{\parindent}{0pt}
\numberwithin{equation}{subsection}

\title{Caprini Notes}
\author{}
\date{Last Updated: \today}

\begin{document}
\maketitle
\tableofcontents
\newpage
%%%%%%%%%%%%%%%%%%%%%%%%%%%%%%%%%
\section{Matter Action for $S(x)$ and $\psi_{\frac12}(x)$}

The most general curved space conformally invariant matter action for a fermion $\psi(x)$ and here a real spin-zero scalar field $S(x)$ is
\begin{eqnarray}
	I_{\rm M} = -\int d^4x(-g)^{1/2} \left[ \xi \left(\frac12 \nabla^\mu S \nabla_\mu S - \frac{1}{12} S^2 R^{\mu}{}_{\mu}\right) + \lambda S^4 
+ i\bar\psi\gamma^\mu(x)[\partial_\mu + \Gamma_\mu(x)]\psi -hS\bar\psi\psi\right]
\label{matteraction}
\end{eqnarray}
We will designate $\xi = 1$ for CG and $\xi = -1$ for MCG. The author instead uses an $\epsilon$ with $\epsilon = -\xi$. 
\\ \\
Equations of motion:
\begin{eqnarray}
i\gamma^\mu(x)[\partial_\mu + \Gamma^\mu(x)]\psi - hS\psi &=& 0
\label{psieom}
\\
\xi \left( \nabla_\alpha\nabla^\alpha S + \frac16 S R^\alpha{}_\alpha \right) - 4\lambda S^3 + h\bar\psi\psi &=& 0
\label{seom}
\end{eqnarray}

Energy momentum tensor (without use of equations of motion)
\begin{eqnarray}
T_{\mu\nu} &=& i\bar \psi \gamma_\mu(x)[\partial_\nu +\Gamma_\nu(x)]\psi 
- g_{\mu\nu}\left[ \lambda S^4 + i\bar\psi\gamma^\alpha(x)[\partial_\alpha + \Gamma^\alpha(x)]\psi - hS\bar\psi \psi\right]
\nonumber\\
&&
+\xi \left[ \frac23 \nabla_\mu S\nabla_\nu S - \frac 16 g_{\mu\nu} \nabla_\alpha S\nabla^\alpha S 
- \frac13 S\nabla_\mu\nabla_\nu S + \frac13 g_{\mu\nu} S \nabla_\alpha\nabla^\alpha S  
- \frac16 S^2\left( R_{\mu\nu} - \frac12 g_{\mu\nu} R^\alpha{}_\alpha\right)\right] 
\label{em1}
\end{eqnarray}

Multiplying \eqref{psieom} by the left with $\bar\psi$, inserting into \eqref{em1} and defining
\begin{eqnarray}
T_{\mu\nu}^f = i\bar \psi \gamma_\mu(x)[\partial_\nu +\Gamma_\nu(x)]\psi ,
\end{eqnarray}
we see that \eqref{em1} takes the form

\begin{eqnarray}
T_{\mu\nu} &=&T_{\mu\nu}^f - g_{\mu\nu} \lambda S^4
+ \xi \bigg[ \frac23 \nabla_\mu S\nabla_\nu S - \frac 16 g_{\mu\nu} \nabla_\alpha S\nabla^\alpha S 
- \frac13 S\nabla_\mu\nabla_\nu S + \frac13 g_{\mu\nu} S \nabla_\alpha\nabla^\alpha S  
\nonumber\\
&& - \frac16 S^2\left( R_{\mu\nu} - \frac12 g_{\mu\nu} R^\alpha{}_\alpha\right)\bigg] 
\label{em2}
\end{eqnarray}
Taking the trace of the above we find

\begin{eqnarray}
g^{\mu\nu}T_{\mu\nu} &=& i\bar\psi \gamma^\mu(x)[\partial_\mu + \Gamma^\mu(x)]\psi -4\lambda S^4
+ \xi \bigg( S\nabla_\alpha\nabla^\alpha S + \frac16 S^2 R\bigg).
\label{em2tr}
\end{eqnarray}

Straightfoward use of the equations of motion \eqref{psieom} and \eqref{seom} show that $T_{\mu\nu}$ is traceless. 

%%%%%%%%%%%%%%%%%%%%%%%%

\subsection{Spontaneously Broken $T_{\mu\nu}$}
When $S(x)$ acquires a non-zero constant vacuum expectation value $S_0$, \eqref{em2} reduces to 
\begin{eqnarray}
T_{\mu\nu} &=&T_{\mu\nu}^f - g_{\mu\nu} \lambda S_0^4
 - \frac16 \xi S_0^2\left( R_{\mu\nu} - \frac12 g_{\mu\nu} R^\alpha{}_\alpha\right)
\label{em3}
\end{eqnarray}
with equations of motion
\begin{eqnarray}
i\gamma^\mu(x)[\partial_\mu + \Gamma^\mu(x)]\psi - hS_0\psi &=& 0
\label{psieom2}
\\
\frac16 \xi S_0 R^\alpha{}_\alpha - 4\lambda S_0^3 + h\bar\psi\psi &=& 0
\label{seom2}
\end{eqnarray}

%%%%%%%%%%%%%%%%%%%%%%%%%%%%%%%%%%%
\section{Einstein Form of $4\alpha_g W_{\mu\nu} = T_{\mu\nu}$}
Following Caprini, let us define
\begin{eqnarray}
8\pi\tilde G \equiv \frac{6}{S_0^2},\qquad \Lambda \equiv 6\lambda S_0^2.
\end{eqnarray}
The energy momentum tensor \eqref{em3} becomes
\begin{eqnarray}
 T_{\mu\nu} = T_{\mu\nu}^f  - \frac{1}{8\pi\tilde G} \left( \Lambda g_{\mu\nu} + \xi G_{\mu\nu} \right),
\label{em4}
\end{eqnarray}
with equations of motion
\begin{eqnarray}
i\gamma^\mu(x)[\partial_\mu + \Gamma^\mu(x)]\psi - hS_0\psi &=& 0
\\
\label{psieom3}
\frac{1}{8\pi\tilde G} ( \xi R-4\Lambda ) + hS_0 \bar\psi\psi &=& 0
\label{seom3}
\end{eqnarray}
The above equations of motion may be combined into
\begin{eqnarray}
8\pi \tilde GT_f = 4\Lambda -\xi R
\label{sphieom4}
\end{eqnarray}
where $T_f = g^{\mu\nu}T_{\mu\nu}^f$. 
\\ \\
Using \eqref{em4} the gravitational equation of motion may be expressed as
\begin{eqnarray}
4\alpha_g W_{\mu\nu} &=&T_{\mu\nu}
\nonumber\\
\to\quad \xi G_{\mu\nu} +\Lambda g_{\mu\nu}+ 32\pi\tilde G \alpha_g W_{\mu\nu} 
&=& 8\pi\tilde G T_{\mu\nu}^f.
\end{eqnarray}
The authors then introduce a ``graviton'' mass $m_g^2 \equiv (32\tilde G\alpha_g)^{-1}$ such that the gravitational field equations may finally be expressed as
\begin{eqnarray}
\xi G_{\mu\nu} +\Lambda g_{\mu\nu}+  \frac{1}{m_g^2} W_{\mu\nu} 
&=& 8\pi\tilde G T_{\mu\nu}^f.
\label{geom}
\end{eqnarray}

%%%%%%%%%%%%%%%%%%%
\section{Gauge Choice and Linearized Solutions}
In solving the linearization of \eqref{geom}, the author uses 1) Teyssandier gauge and 2) makes an ansatz to reduce the four order equations of motion into two second order equations. 
%%%%%%%%%%%%%%%%%%
\subsection{Reduction to Second Order}
The author selects to reduce the order of the gravitational equations by defining $h_{\mu\nu}$ as
\begin{eqnarray}
h_{\mu\nu} =  -\xi \left( H_{\mu\nu} + \Psi_{\mu\nu}\right),
\end{eqnarray}
with the \emph{ansatz}
\begin{eqnarray}
\Psi_{\mu\nu} = \frac{1}{m^2_g}\left( \partial_\alpha \partial^\alpha h_{\mu\nu} - \frac13 \eta_{\mu\nu}\delta R\right).
\end{eqnarray}
The author refers to $H_{\mu\nu}$ as the ``massless" mode that parallels GR and $\Psi_{\mu\nu}$ as the ``massive'' mode which represents the GR deviation. 
\\ \\
The authors choice of ansatz implies an expression of $h_{\mu\nu}$ as
\begin{eqnarray}
h_{\mu\nu} = \xi\left[H_{\mu\nu} + \frac{1}{m_g^2} \left( \partial_\alpha \partial^\alpha h_{\mu\nu} - \frac13 \eta_{\mu\nu}\delta R\right)\right].
\end{eqnarray}


%%%%%%%%%%%%%%%%%%%
\subsection{Teyssandier Gauge}
Let us define the vector
\begin{eqnarray}
Z_\mu = -\frac{1}{m_g^2} \left[ (\partial_\alpha\partial^\alpha + \xi m_g^2) \partial^\rho \bar h_{\rho\mu} + \frac13 \partial_\mu \delta R\right],
\end{eqnarray}
where the author uses the trace reversed $\bar h_{\mu\nu} = h_{\mu\nu}- \frac12 g_{\mu\nu}g^{\rho\sigma}h_{\rho\sigma}$.
The Teyssandier gauge is chosen such that $Z_\mu = 0$. In appendix $B$, the author claims that a proper choice of residual gauge conditions can reduce the $h_{\mu\nu} = -\xi(H_{\mu\nu} - \Psi_{\mu\nu})$ into two gauge invariant quantities for $H_{\mu\nu}$ and five gauge invariant quantities for $\Psi_{\mu\nu}$.
\\ \\
This contrasts with section \ref{resgauge} where we  show a total of six gauge invariant quantities; two for the $A_{\mu\nu}$ modes and four for the $B_{\mu\nu} t$ modes. Moreover, one must be careful in showing that the residual conditions must obey solutions to the equation of motion itself, i.e. $\nabla^4 K_{\mu\nu} =$ and will not generally hold for $\nabla^4 K_{\mu\nu} = \delta T_{\mu\nu}$. 
\\ \\
However, we must also note that the gravitational side used by the author is no longer $4\alpha_g W_{\mu\nu}$ but rather
the LHS of \eqref{geom}. Residual gauge symmetries for the homogenous case must obey this particular equation of motion. 






\newpage
%%%%%%%%%%%%%%%%%%%%%%%%%%%%%%%%
%%%%%%%%%%%%%%%%%%%%%%%%%%%%%%%%
\appendix
\section{Appendix}
\subsection{Curvature Tensors Under Conformal Transformation}
Curvature tensors (in Weinberg convention) transform under conformal transformation $g_{\mu\nu}\to \Omega^2(x)g_{\mu\nu} = e^{2\alpha(x)}g_{\mu\nu}$  as 
\begin{align}
R_{\lambda\mu\nu\kappa} &\to \Omega^2 R_{\lambda\mu\nu\kappa} + \Omega\left ( -g_{\mu\nu}\nabla_\lambda \nabla_\kappa \Omega
+ g_{\lambda\nu}\nabla_\mu\nabla_\kappa \Omega + g_{\mu\kappa} \nabla_\nu\nabla_\lambda \Omega - g_{\lambda\kappa} \nabla_\mu\nabla_\nu \Omega \right)
\nonumber\\
&\qquad+ 2g_{\mu\nu} \nabla_\kappa\Omega \nabla_\lambda\Omega - 2g_{\lambda\nu} \nabla_\kappa\Omega \nabla_\mu\Omega - 2g_{\mu\kappa}
\nabla_\lambda\Omega \nabla_\nu\Omega + 2g_{\lambda\kappa} \nabla_\mu \Omega \nabla_\nu\Omega
\nonumber\\
&\qquad + (g_{\lambda\nu} g_{\mu\kappa}-g_{\lambda\kappa}g_{\mu\nu})\nabla^\rho \Omega \nabla_\rho \Omega
\nonumber\\
&= e^{2\alpha}\bigg[ R_{\lambda\mu\nu\kappa} + (g_{\mu\kappa}g_{\lambda\nu} - g_{\lambda\kappa}g_{\mu\nu})\nabla_\rho\alpha \nabla^\rho\alpha+
g_{\mu\nu} \nabla_\kappa\alpha \nabla_\lambda\alpha - g_{\lambda\nu} \nabla_\kappa\alpha \nabla_\mu\alpha - g_{\mu\kappa} \nabla_\lambda\alpha
\nabla_\nu\alpha 
\nonumber\\
&\qquad + g_{\kappa\lambda}\nabla_\mu\alpha\nabla_\nu\alpha - g_{\mu\nu}\nabla_\lambda\nabla_\kappa \alpha + g_{\lambda\nu} \nabla_\mu\nabla_\kappa\alpha + g_{\mu\kappa}\nabla_\nu\nabla_\lambda \alpha -
g_{\kappa\lambda}\nabla_\mu\nabla_\nu\alpha\bigg]
 \label{GRA11} \\
{}
\nonumber\\
R_{\mu\nu} &\to R_{\mu\nu} + \Omega^{-2} g_{\mu\nu}\nabla_\lambda \Omega \nabla^\lambda \Omega
	-4 \Omega^{-2} \nabla_\mu \Omega \nabla_\nu \Omega + \Omega^{-1} g_{\mu\nu}\nabla_\lambda \nabla^\lambda \Omega + 2\Omega^{-1}
	\nabla_\mu \nabla_\nu \Omega
\nonumber\\
&= R_{\mu\nu} + 2 g_{\mu\nu}\nabla_\lambda \alpha \nabla^\lambda\alpha - 2 \nabla_\mu\alpha \nabla_\nu \alpha +  g_{\mu\nu} \nabla_\lambda \nabla^\lambda
\alpha + 2\nabla_\mu \nabla_\nu \alpha 
\label{GRA12} \\ 
{}
\nonumber\\
R^\alpha{}_\alpha &\to \Omega^{-2}R^{\alpha}{}_\alpha + 6\Omega^{-3}\nabla_\lambda \nabla^\lambda \Omega
\nonumber\\
&= e^{-2\alpha} R^{\alpha}{}_\alpha + 6 e^{-2\alpha}\nabla_\lambda\alpha \nabla^\lambda\alpha + 6 e^{-2\alpha}\nabla_\lambda \nabla^\lambda \alpha.
\label{GRA13}
\end{align}
Using the curvature tensors we may form the transformation of the Einstein tensor
\begin{equation}
G_{\mu\nu} \to G_{\mu\nu} + \Omega^{-1}\left( -2g_{\mu\nu}\nabla^\lambda \nabla_\lambda \Omega + 2\nabla_\mu \nabla_\nu \Omega\right) +
\Omega^{-2}\left( g_{\mu\nu} \nabla_\lambda \Omega \nabla^\lambda \Omega - 4 \nabla_\mu \Omega \nabla_\nu \Omega\right)\label{GRA14}
\end{equation}
\subsection{Fields Under Conformal Transformation}\label{A2}
Under local conformal transformation $g_{\mu\nu} \to e^{2\alpha(x)}g_{\mu\nu}$, the infinitesimal distance between two points also transforms as
\begin{equation}
	ds^2 = g_{\mu\nu}dx^\mu dx^\nu \to e^{2\alpha(x)} ds^2,
\end{equation}
and hence the unit of length $L$ scales as $L \to e^{\alpha(x)}L$ (note that scale transformation can be achieved from Weyl rescaling or from coordinate conformal transformations - the resulting transformation on length $L$ is the same). Therefore, determination of the length dimensions of our fields will specify their conformal weight. Noting that the canonical conjugate momentum $\pi$ of a field $\phi$ is 
\begin{equation}
\pi = \frac{\partial \mathcal L}{\partial \dot \phi},
\end{equation}
a conveneint method to finding the length dimension can be obtained from the quantized canonical commutation relations
\begin{equation}
	[\phi(\mathbf x),\pi(\mathbf x')] = i \delta^{(3)}(\mathbf x - \mathbf x'),
\end{equation}
and hence $\phi \pi \sim L^{-3}$. For the scalar field $S(x)$, the relevant conjugate momentum is 
\begin{equation}
	\pi = \frac{\partial}{\partial \dot S} \left(-\frac12 \eta^{\mu\nu} \partial_\mu S \partial_\nu S \right) = 
\frac{\partial}{\partial \dot S}\left(\frac12\dot S^2- \frac12 \delta^{ij} \partial_i S \partial_j S\right) = \dot S.
\end{equation}
Therefore, we find $S\sim L^{-1}$ and hence
\begin{equation}
	S(x)\to e^{-\alpha(x)} S(x).
\end{equation} 
For the Dirac spinor $\psi(x)$ we note the relevant piece
\begin{equation}
	i\bar \psi\gamma^\mu(x) \partial_\mu \psi \propto \bar \psi \gamma^0(x) \dot \psi.
\end{equation}
Recalling $\bar\psi = \psi^\dagger \gamma^0$ and $(\gamma^0)^2=-1$ we have
\begin{equation}
	\pi = -i \psi^\dagger,
\end{equation}
and hence $\psi \psi^\dagger \sim L^{-3}$. Therefore $\psi \sim L^{-3/2}$ and 
\begin{equation}
\psi(x) \to e^{-3 \alpha(x)/2} \psi(x).
\end{equation}

%%%%%%%%%%%%%%%%

\subsection{Matter Action for Conformally Coupled Scalar Field}
\begin{equation}
	I_{\rm M} = -\int d^4x(-g)^{1/2} \left[ \frac12 \nabla^\mu S \nabla_\mu S - \frac{1}{12} S^2 R^{\mu}{}_{\mu} + \lambda S^4 
+ i\bar\psi\gamma^\mu(x)[\partial_\mu + \Gamma_\mu(x)]\psi -hS\bar\psi\psi\right]\label{GRW1}.
\end{equation}

%%%%%%%%%%%%%%%%%%

\subsubsection{Definitions}
Definitions:
\begin{align}
\gamma^\mu(x) &= V^\mu_a(x)\hat \gamma ^a
\nonumber\\
\Gamma_\mu(x) &= \frac{1}{8}\left( [\gamma^\nu(x),\partial_\mu\gamma_\nu(x)] -[\gamma^\nu(x),\gamma_\sigma(x)]\Gamma^\sigma_{\mu\nu} \right)
\nonumber\\
ds^2 &= -(dx^0)^2 + \delta_{ij}dx^idx^j = \eta_{ab}dx^a dx^b
\nonumber\\
-2\eta^{ab} &= \hat \gamma^a \hat \gamma^b + \hat \gamma^b \hat\gamma^a 
\nonumber\\
\bar\psi &= \psi^\dagger \hat D
\nonumber\\
\hat \gamma^{a\dagger} &= \hat D \hat\gamma^a \hat D^{-1}
\end{align}
Hermiticty is implied, in that
\begin{align}
i\bar\psi \gamma^\mu(x)[\partial_\mu+ \Gamma_\mu(x)]\psi &= \frac{i}{2} \bar \psi\gamma^\mu(x)[\partial_\mu + \Gamma_{\mu}(x)]\psi - \frac{i}{2}
\bar\psi[\overset{\leftarrow}\partial_\mu +\Gamma_\mu(x)]\gamma^\mu(x)\psi
\nonumber\\
&= \frac{i}{2}\bar\psi\gamma^\mu(x)[\partial_\mu+\Gamma_\mu(x)]\psi + \rm{h.c.} 
\end{align}
and
\begin{align}
i\bar\psi\gamma_\mu(x)[\partial_\nu+\Gamma_\nu(x)]\psi &= \frac{i}{4}\bar\psi \gamma_\mu(x)[\partial_\nu + \Gamma_\nu(x)]\psi +
\frac{i}{4} \bar\psi \gamma_\nu(x)[\partial_\mu + \Gamma_\mu(x)]\psi + \rm{h.c.}
\end{align}

%%%%%%%%%%%%%%%

\subsubsection{Conformal Invariance}
Note that under $g_{\mu\nu}\to e^{2\alpha(x)}g_{\mu\nu}$ we have (see \ref{A2})
\begin{equation}
S(x) \to e^{-\alpha(x)}S(x),\qquad \psi(x)\to e^{-3\alpha(x)/2}\psi(x),\qquad \bar \psi(x)\to e^{-3\alpha(x)/2}\bar\psi(x).
\end{equation}
As for the determinant in $D$ dimensions,
\begin{equation}
	\rm{det}[g_{\mu\nu}]\to \rm{det}[e^{2\alpha}g_{\mu\nu}] = e^{2D\alpha}det[g_{\mu\nu}]
\end{equation}
whereby
\begin{equation}
	(-g^{1/2}) \to e^{D\alpha(x)}(-g^{1/2}),
\end{equation}
and thus for $D=4$ it will suffice to show that each term in 
\begin{equation}
\mathcal L =  \frac12 \nabla^\mu S \nabla_\mu S - \frac{1}{12} S^2 R^{\mu}{}_{\mu} + \lambda S^4 
+ i\bar\psi\gamma^\mu(x)[\partial_\mu + \Gamma_\mu(x)]\psi -hS\bar\psi\psi
\end{equation}
must transform as $e^{-4\alpha(x)}$. Looking at the first two terms, we have
\begin{align}
\frac12 g^{\mu\nu}\nabla_\mu S\nabla_\nu S &\to   \frac12 g^{\mu\nu}e^{-2\alpha} \left( e^{-\alpha} \nabla_\mu S - S e^{-\alpha}\nabla_\mu \alpha \right)\left( e^{-\alpha} \nabla_\nu S - S e^{-\alpha}\nabla_\nu \alpha \right)\\
&=  \frac12 g^{\mu\nu}e^{-4\alpha} \left( \nabla_\mu S\nabla_\nu S + S^2 \nabla_\mu\alpha\nabla_\nu\alpha -S \nabla_\nu S \nabla_\mu \alpha - S\nabla_\nu S\nabla_\mu \alpha \right)
\nonumber\\
&= e^{-4\alpha}\left( \frac{1}{2} \nabla_\mu S\nabla^\mu S + \frac{1}{2} S^2 \nabla_\lambda \alpha \nabla^\lambda\alpha - S \nabla_\lambda S\nabla^\lambda \alpha \right) 
\end{align}
and (see \ref{GRA13})
\begin{equation}
-\frac{1}{12} S^2 R^\mu{}_\mu \to e^{-4\alpha}\left(-\frac{1}{12} S^2 R^\mu{}_\mu -\frac12 S^2 \nabla_\lambda\alpha \nabla^\lambda \alpha -\frac12 S^2 \nabla_\lambda
\nabla^\lambda \alpha \right).
\end{equation}
What remains from these two is infact a total derivative, noting that
\begin{align}
-\frac12 \nabla_\lambda ( S^2 \nabla^\lambda \alpha ) &= -\frac12 S^2 \nabla_\lambda \nabla^\lambda \alpha - S\nabla_\lambda S\nabla^\alpha \alpha
\nonumber\\
&= -\frac{1}{2}(-g)^{1/2} \partial_\lambda \left[ (-g)^{-1/2}S^2\partial^\lambda \alpha \right].
\end{align}
Variation of this term with respect to the relevent fields ($\delta g_{\mu\nu}$ and $\delta S$ here) allow it to vanish when evaluated on the boundary. Hence, under conformal transformations the contribution of this term will not affect the equations of motion (as with any total divergence). 
\\ \\
\noindent
For the fermion kinetic energy term, hermiticity has been implied with the full expression being
\begin{align}
i\bar\psi \gamma^\mu(x)[\partial_\mu+ \Gamma_\mu(x)]\psi &\equiv \frac{i}{2} \bar \psi\gamma^\mu(x)[\partial_\mu + \Gamma_{\mu}(x)]\psi - \frac{i}{2}
\bar\psi[\overset{\leftarrow}\partial_\mu +\Gamma_\mu(x)]\gamma^\mu(x)\psi.
\end{align}
Under conformal transformation we have 
\begin{align}
g_{\mu\nu} = V^{a}_\mu V^b_\nu \eta_{ab} &\to \Omega^2 g_{\mu\nu}
\nonumber\\
&= 
\end{align}

%%%%%%%%%%%%%%

\subsubsection{Trace}
Allowing the parameter $\epsilon \in (-1,1)$ to represent conformal and massive conformal gravity respectively, the energy momentum tensor evaluates to
\begin{align}
	T_{\mu\nu} =&{}
	 \epsilon\bigg[ -\frac23 \nabla_\mu S\nabla_\nu S + \frac16 g_{\mu\nu} \nabla_\alpha S\nabla^\alpha S + \frac13
	S \nabla_\mu \nabla_\nu S -\frac13 g_{\mu\nu} S\nabla_\alpha\nabla^\alpha S+ \frac16 S^2\left( R_{\mu\nu} - \frac12 g_{\mu\nu}R\right)\bigg] - g_{\mu\nu}\lambda S^4
\nonumber\\
&{} +\frac12 \left[ i\bar\psi \gamma_\mu(\partial_\nu +\Gamma_\nu)\psi +  i\bar\psi \gamma_\nu(\partial_\mu +\Gamma_\mu)\psi\right].
\end{align}
The trace of this for arbitrary $S(x)$ is
\begin{equation}
	g^{\mu\nu} T_{\mu\nu} = \epsilon\left( - S \nabla_\alpha \nabla^\alpha S - \frac16 S^2 R\right) - 4\lambda S^4 + i \bar\psi \gamma^\mu(\partial_\mu
	+\Gamma_\mu)\psi.
\end{equation}
The equations of motion for the fields are
\begin{equation}
	\epsilon\left( -\nabla_\alpha \nabla^\alpha S -\frac16 S R\right) - 4\lambda S^3 + \xi \bar\psi\psi = 0
\end{equation}
\begin{equation}
i\gamma^\mu(\partial_\mu +\Gamma_\mu)\psi - \xi S\psi = 0.
\end{equation}
Substituting these into (13) we find that it is traceless.

%%%%%%%%%%%%%%%%%%%%%%%%%%%%%
\subsection{Massive Conformal Gravity}
The action used in MCG is 
\begin{equation}
I = I_G+I_M
\end{equation}
where
\begin{equation}
I_{\rm G} = \frac{c^3}{16\pi G} \int d^4x(-g)^{1/2} \left[ \phi^2 R^{\alpha}{}_\alpha + 6 \nabla_\mu \phi \nabla^\mu \phi - 2\Lambda_G \phi^4 - \frac{\alpha^2}{2}C^{\lambda\mu\nu\kappa}C_{\lambda\mu\nu\kappa}\right]
\end{equation}
\begin{equation}
I_{\rm M} = -\frac{1}{c}\int d^4x (-g)^{1/2} \left[ \frac12 \nabla^\mu S \nabla_\mu S + \frac{1}{12} S^2 R^{\mu}{}_{\mu} + \lambda S^4 
+ i\bar\psi\gamma^\mu(x)[\partial_\mu + \Gamma_\mu(x)]\psi +hS\bar\psi\psi\right].
\end{equation}
Compare this to CG where
\begin{equation}
I_{\rm G} = -\alpha_g \int d^4x (-g)^{1/2} C^{\lambda\mu\nu\kappa}C_{\lambda\mu\nu\kappa}
\end{equation}
\begin{equation}
	I_{\rm M} = -\int d^4x(-g)^{1/2} \left[ \frac12 \nabla^\mu S \nabla_\mu S - \frac{1}{12} S^2 R^{\mu}{}_{\mu} + \lambda S^4 
+ i\bar\psi\gamma^\mu(x)[\partial_\mu + \Gamma_\mu(x)]\psi -hS\bar\psi\psi\right].
\end{equation}

%%%%%%%%%%%%%%%%%%%%%%%%%%%
\subsection{Conformal Symmetry and Trace}
Consider an arbitrary action
\begin{equation}
	I = \int d^4x (-g)^{1/2} C(x),
\end{equation}
where $C(x)$ is a general coordinate scalar. Variation of this action with respect to the metric yields the tensor $C_{\mu\nu}$, defined as
\begin{equation} 
\frac{\delta I}{\delta g^{\mu\nu}} = \int d^4x (-g)^{1/2} C_{\mu\nu} \delta g^{\mu\nu}.
\end{equation} 
Under conformal transformation, 
\begin{equation}
\delta g^{\mu\nu} \to e^{-2\alpha}\delta g^{\mu\nu}\qquad (-g)^{1/2} \to e^{4\alpha}(-g)^{1/2},
\end{equation}
and hence, to retain conformal invariance it must follow that
\begin{equation}
C_{\mu\nu} \to e^{-2\alpha}C_{\mu\nu}.
\end{equation}
In maintaining the generality of $C(x)$, $C_{\mu\nu}$ here could represent the energy momentum tensor due to curvature or matter. Now, let us decompose the general $C_{\mu\nu}$ into a trace-free and traceless component via
\begin{equation}
C_{\mu\nu} = C^{\theta}_{\mu\nu} + \frac14 g_{\mu\nu}\left( g^{\alpha\beta} C_{\alpha\beta}\right).
\end{equation}
Under conformal transformation, denoting transformed quantities with bars and $C = g^{\alpha\beta}C_{\alpha\beta}$, we find the traceless sector transforms into
\begin{equation}
\bar C_{\mu\nu}^\theta = \bar C_{\mu\nu}- \frac14 \bar g_{\mu\nu} \bar C = e^{-2\alpha}C_{\mu\nu} - \frac14 e^{2\alpha} g_{\mu\nu} C
\end{equation}
in which it is apparent that $\bar g^{\mu\nu} \bar C^\theta_{\mu\nu} = e^{-2\alpha}g^{\mu\nu}C_{\mu\nu}= 0$, i.e. tracelessness is preserved as expected.
\begin{equation}
	\bar g^{\mu\nu} \bar C_{\mu\nu}^\theta = 
\end{equation}
\begin{equation}
	\bar C_{\mu\nu}^\theta = e^{-2\alpha} C_{\mu\nu}^\theta + \frac14 (e^{2\alpha} - e^{-2\alpha})g_{\mu\nu} C
\end{equation}

%%%%%%%%%%%%%%%%%%%%%%%%%%%%%%%
\section{Residual Gauge for Flat Transverse Traceless $\Box^2 K_{\mu\nu}=0$}
\label{resgauge}
In the transverse gauge $\partial_\nu K^{\mu\nu} = 0$ in the Minkowski background the vacuum equation of motion for the traceless $K_{\mu\nu}$ is
\begin{equation}
\delta W_{\mu\nu} = \eta^{\alpha\beta} \eta^{\sigma\rho}\partial_\alpha\partial_\beta\partial_\sigma\partial_\rho K_{\mu\nu} =0.
\end{equation}
The momentum eigenstate solutions take the form 
\begin{equation}
K_{\mu\nu} = A_{\mu\nu}e^{ikx} + n_\alpha x^\alpha B_{\mu\nu} e^{ikx} +\text{c.c.}\label{202}
\end{equation}
where $n_\alpha = (1,0,0,0)$ and $k^\mu k_{\mu} = 0$. Following the transverse condition, the solution must obey
\begin{equation}
0=\left(ik^\nu A_{\mu\nu} + n^\nu B_{\mu\nu}\right)e^{ikx}
+ \left( ik^\nu B_{\mu\nu}\right) n_\alpha x^\alpha  e^{ikx}+ \text{c.c.}\label{203}
\end{equation}
In addition to the tracelessness condition, to satisfy all $x$ (noting that $e^{ikx}$, $e^{-ikx}$, $te^{ikx}$ and $te^{-ikx}$ are linearly independent), we set in \eqref{203} each coefficient preceding the space-time dependent function to zero, viz.
\begin{equation}
A^\mu{}_\mu = 0,\qquad B^\mu{}_\mu=0,\qquad ik^\nu A_{\mu\nu} + n^\nu B_{\mu\nu}= 0,\qquad i k^\nu B_{\mu\nu} = 0.
\end{equation}
We have a total of $10$ conditions upon the 20 total components of $A_{\mu\nu}$ and $B_{\mu\nu}$. 
It is easy to check that these conditions (and also their implied conjugate expressions) satisfy our choice of transverse coordinate system and retain the tracelessness of $K_{\mu\nu}$. 
Under infinitesimal coordinate transformation $x^\mu \to x^\mu + \epsilon^\mu(x)$, $K_{\mu\nu}$ transforms as
\begin{equation}
	K_{\mu\nu}' = K_{\mu\nu} - \partial_\mu \epsilon_\nu - \partial_\nu\epsilon_\mu + \frac12 g_{\mu\nu} \partial_\rho \epsilon^\rho.
\end{equation}
We denote the change in $K_{\mu\nu}$ (Lie derivative) as the tensor
\begin{equation}
\Delta K_{\mu\nu} = - \partial_\mu \epsilon_\nu - \partial_\nu\epsilon_\mu + \frac12 g_{\mu\nu} \partial_\rho \epsilon^\rho.\label{207}
\end{equation}
Noting that $\Delta K_{\mu\nu}$ is manifestly traceless, in order to preserve the tranverse gauge condition $\partial_\mu K^{\mu\nu} = 0$,  $\Delta K^{\mu\nu}$ must obey $\partial_\nu \Delta K^{\mu\nu}=0$, viz.
\begin{equation}
	0 =
	-\partial_\nu \partial^\nu \epsilon^\mu - \frac12 \partial^\mu \partial_\nu \epsilon^\nu .\label{208}
\end{equation}
We take the $\epsilon^\mu(x)$ to be of the plane wave form,
\begin{equation}
\epsilon^\mu(x) = i A^\mu e^{ikx} + iB^\mu n_\alpha x^\alpha e^{ikx} + \text{c.c.},
\end{equation}
which obeys the following relations:
\begin{equation}
\partial^\nu \epsilon^\mu = - k^\nu\left(A^\mu e^{ikx} + B^\mu n_\alpha x^\alpha e^{ikx}\right)+ 
i n^\nu \left(  B^\mu  e^{ikx} \right)+ \text{c.c.}
\end{equation}
\begin{equation}
\partial_\nu \partial^\nu \epsilon^\mu = -2k_\nu n^\nu \left(B^\mu  e^{ikx}\right)+\text{c.c.},
\end{equation}
\begin{equation}
\partial_\mu \partial^\nu \epsilon^\mu = -i k_\mu k^\nu\left(A^\mu e^{ikx} +  B^\mu n_\alpha x^\alpha e^{ikx}\right)
- (k^\nu n_\mu+k_\mu n^\nu)\left[ B^\mu e^{ikx}\right] + \text{c.c.},
\end{equation}
where for reference we also have the relation
\begin{equation}
\partial_\beta \partial^\beta (n_\alpha x^\alpha e^{ikx}) = 2i n_\alpha k^\alpha e^{ikx}.
\end{equation}
The transverse condition per \eqref{208} then takes the form
\begin{align}
0 {}&= 2k_\nu n^\nu \left(B^\mu  e^{ikx}\right)+\frac12 i k_\nu k^\mu\left(A^\nu e^{ikx} +  B^\nu n_\alpha x^\alpha e^{ikx}\right)
+\frac12 (k^\mu n_\nu+k_\nu n^\mu)\left[ B^\nu e^{ikx}\right]+ \text{c.c.}\ .
\end{align}
To hold for arbitrary $x$, we have the two separate conditions,
\begin{equation}
2k_\nu n^\nu B^\mu +\frac12 ik_\nu k^\mu A^\nu + \frac12 (k^\mu n_\nu+k_\nu n^\mu)B^\nu=0,\qquad \frac12 ik_\nu k^\mu B^\nu=0.\label{2019}
\end{equation}
For arbitrary $k^\mu$, the second condition in \ref{2019} implies $k_\nu B^\nu = 0$. As such, the remaining condition is
\begin{equation}
2k_\nu n^\nu B^\mu + \frac12 k^\mu n_\nu B^\nu + \frac12 i k_\nu k^\mu A^\nu = 0.
\end{equation}
Let us now take a wave propagating in the $z$ direction, with wavevector
\begin{equation}
k^\mu = (k,0,0,k),\qquad k_\mu = (-k,0,0,k).
\end{equation}
The transverse condition $\partial^\mu \Delta K_{\mu\nu}$ then entails
\begin{equation}
B_0 = -B_3,\qquad B_0 = \frac{i}{5}k(A_0+A_3),\qquad B_1 = B_2 = 0.
\end{equation}
%We see that the specific form of $\epsilon^\mu(x)$ comprises of four independent components, here chosen as $B_0$, $A_0$, $A_1$, and $A_2$. The dependencies are:
%\begin{equation}
%B_{1} = B_{2} = 0,\qquad B_3 = -B_0,\qquad A_3 = -A_0 - \frac{5i}{k} B_0.
%\end{equation}
For the tensor polarizations $A_{\mu\nu}$ and $B_{\mu\nu}$ the transverse relations take the form
\begin{equation}
B^\mu{}_\mu = A^\mu{}_\mu = 0,\qquad B_{0\mu} = -B_{3\mu},\qquad ik(A_{\mu 0}+ A_{\mu 3}) = B_{0\mu}.
\end{equation}
Although this would appear to be 10 total constraints, the condition $B_{00} = -B_{30}$ reduces the equation
\begin{equation}
ik(A_{\mu0}+A_{\mu 3}) = B_{0\mu},
\end{equation}
from 4 to 3 conditions, namely
\begin{equation}
ik(A_{10}+A_{13}) = B_{01},\qquad ik(A_{20}+A_{23}) = B_{02},\qquad A_{00} + 2A_{03} + A_{33} = 0.
\end{equation}
% We will take 11 the independent components as 
%\begin{equation}
%B_{00}, B_{01},B_{02},B_{11}, B_{12}, A_{00},A_{01},A_{02},A_{11},A_{22},A_{12}.
%\end{equation}
%In order to arrive at the following choice of independent components for $B_{\mu\nu}$, we utliize the gauge conditions which lead us to following dependencies:
%\begin{equation}
%B_{33} = -B_{03} = B_{00},\qquad B_{23} = -B_{02},\qquad B_{13} = -B_{01},\qquad B_{22} = -B_{11}.
%\end{equation}
%As for $A_{\mu\nu}$, the dependencies are:
%\begin{equation}
%A_{13} = -\frac{i}{k} B_{01} - A_{01},\quad
%A_{23} = -\frac{i}{k}B_{02} - A_{02},\quad
%A_{33} = A_{00} - A_{11} - A_{22},\quad
%A_{03} = -A_{00} + \frac12 (A_{11}+A_{22}).
%\end{equation}
The form for the transformation (Lie derivative) onto $K_{\mu\nu}$ is
\begin{align}
\Delta K_{\mu\nu} &= \left[ k_\nu A_\mu + k_\mu A_\nu - i \left( n_\nu B_\mu + n_\mu B_\nu\right) 
-\frac12 g_{\mu\nu} A^\alpha k_\alpha + \frac{i}{2} g_{\mu\nu}n_\alpha B^\alpha \right]e^{ikx}
\nonumber\\
&\qquad + \bigg[ k_\nu B_\mu + k_\mu B_\nu \bigg] n_\alpha x^\alpha e^{ikx}.
\end{align}
It will be useful to evaluate this for different components:
\begin{align}
\Delta K_{00} &=\left[ -2 kA_0 +\frac12 k(A_0 + A_3) - \frac{3i}{2} B_0\right]e^{ikx} - \bigg[2kB_0\bigg] n_\alpha x^\alpha e^{ikx}
\nonumber\\
\Delta K_{01} &=  -k A_1 e^{ikx},\qquad \Delta K_{02} =  -k A_2 e^{ikx}
\nonumber\\
\Delta K_{03} &= \left[ -kA_3 +kA_0 -i B_3\right]e^{ikx} - \left[2kB_3\right] n_\alpha x^\alpha e^{ikx}
\nonumber\\
\Delta K_{11} &= \Delta K_{22} =  \left[ - \frac12 k(A_0 + A_3) - \frac{i}{2}B_0 \right]  e^{ikx},\qquad \Delta K_{12} = 0
\nonumber\\
\Delta K_{13} &= [kA_1]e^{ikx},\qquad \Delta K_{23} = [kA_2]e^{ikx}
\nonumber\\
\Delta K_{33} &= \left[ 2kA_3 -\frac12 k(A_0+A_3) - \frac{i}{2}B_0\right]e^{ikx} + \bigg[2kB_3\bigg]n_\alpha x^\alpha e^{ikx}.
\end{align}
The total transformation on each polarization tensor, for $A_{\mu\nu} \to A_{\mu\nu}'$ and $B_{\mu\nu} \to B'_{\mu\nu}$, is 
\begin{align}
A_{00}' &= A_{00} -2kA_0 - 4i B_0 			& B_{00}'&= B_{00} -2k B_0
\nonumber\\
A_{01}' &= A_{01} - kA_1							& B_{01}'&= B_{01} 
\nonumber\\
A_{02}' &= A_{02} - kA_2							&B_{02}'&= B_{02}
\nonumber\\
A_{03}' &= A_{03} +2kA_0 + 6i B_0			&B_{03}'&= B_{03} + 2kB_0
\nonumber \\
A_{11}' &= A_{11} + 2i B_0						&B_{11}'&= B_{11}
\nonumber\\
A_{22}' & = A_{22} +2iB_0						&B_{22}'&= B_{22}
\nonumber\\
A_{33}' &= A_{33}-2k A_0 - 8i B_0				&B_{33}'&= B_{33} - 2kB_0
\nonumber\\
A_{12}' &= A_{12}									&B_{12}'&=B_{12}
\nonumber\\
A_{13}' &= A_{13} + kA_1						&B_{13}'&=B_{13}
\nonumber\\
A_{23}' &= A_{23} + k A_2						&B_{23}'&=B_{23}.
\end{align}
Neither the polarizations nor the gauge terms $A_\mu$ and $B_\mu$ are all independent. Their dependencies are:
\begin{align}
-A_{00} + A_{11}+A_{22} + A_{33} = 0&\quad&A_3 = -A_0 - \frac{5i}{k} B_0\nonumber\\
ik(A_{\mu 0} + A_{\mu 3}) = B_{0\mu} &\quad &B_3 = -B_0 \nonumber\\
B_{0\mu} = -B_{3\mu}&\quad& B_{1}=B_2 = 0\nonumber\\
-B_{00} + B_{11}+B_{22} + B_{33} = 0&\quad&
\end{align}
Looking more closely at these dependencies amongst $B_{\mu\nu}$, we note:
\begin{equation}
B_{33} = -B_{03} = B_{00},\qquad B_{23} = -B_{02},\qquad B_{13} = -B_{01},\qquad B_{22} = -B_{11}.
\end{equation}
This leaves $B_{\mu\nu}$ with 5 total independent components, chosen as: $B_{33}$, $B_{12}$, $B_{11}$, $B_{01}$ and $B_{02}$. 
\\ \\
As for the $A_{\mu\nu}$, we note:
\begin{equation}
A_{13} = -\frac{i}{k} B_{01} - A_{01},\quad
A_{23} = -\frac{i}{k}B_{02} - A_{02},\quad
A_{22} = A_{00} - A_{11} - A_{33},\quad
A_{03} = - \frac12 (A_{00}+A_{33}).
\end{equation}
$A_{\mu\nu}$ thus has a total of 6 independent components, here chosen as: $A_{00}$, $A_{01}$, $A_{02}$, $A_{11}$, $A_{33}$, and $A_{12}$. 
%B_{1} = B_{2} = 0,\qquad B_3 = -B_0,\qquad A_3 = -A_0 - \frac{5i}{k} B_0
Regarding the gauge invariance, we may choose to set 
\begin{equation}
B_0 = \frac{B_{33}}{2k}
\end{equation}
which eliminates $B'_{33}$, $B'_{00}$, and $B'_{03}$. This leaves $B_{\mu\nu}$ with four total independent gauge invariant quantities that cannot be eliminated: $B_{12}$, $B_{11}$, $B_{01}$ and $B_{02}$. 
\\ \\
As for $A_{\mu\nu}$, we first set 
\begin{equation}
A_1 = \frac{A_{01}}{k},\quad A_2 = \frac{A_{02}}{k}.
\end{equation}
This eliminates $A'_{01}$ and $A'_{02}$. Through some various manipulation of the dependencies, we also see that if we set
\begin{equation}
A_0 = \frac{2A_{00} - A_{33}}{2k},
\end{equation}
this will eliminate $A_{00}$ and $A_{33}$. This leaves $A_{\mu\nu}$ with two total independent gauge invariant quantities which cannot be eliminated: $A_{12}$ and $A_{11}$. 
\\ \\
In summary we are left with gauge invariant wave solutions of the form
\begin{equation}
K_{\mu\nu} = \begin{pmatrix}0&0&0&0\\0&A_{11}&A_{12}&0\\0&A_{12}&-A_{11}&0\\0&0&0&0\end{pmatrix}e^{ikx} + \begin{pmatrix}
0&B_{01}&B_{02}&0\\B_{01}&B_{11}&B_{12}&0\\B_{02}&B_{12}&-B_{11}&0\\0&0&0&0  \end{pmatrix} n_\alpha x^\alpha e^{ikx}
\end{equation}

\end{document}