\documentclass[10pt,letterpaper]{article}
\usepackage[textwidth=7in, top=1in,textheight=9in]{geometry}
\usepackage[fleqn]{mathtools} 
\usepackage{amssymb,braket,hyperref,xcolor}
\hypersetup{colorlinks, linkcolor={blue!50!black}, citecolor={red!50!black}, urlcolor={blue!80!black}}
\usepackage[title]{appendix}
\usepackage[sorting=none]{biblatex}
\numberwithin{equation}{section}
\setlength{\parindent}{0pt}
\title{Coordinate Transformation RW SVT3 v2}
\date{}
\allowdisplaybreaks
\begin{document} 
\maketitle
\noindent 
%%%%%%%%%%%%%%%%%%%%%%%%%%%%%%%%
\section{RW $\Omega(\tau)$}
%%%%%%%%%%%%%%%%%%%%%%%%%%%%%%%%%%
\begin{eqnarray}
ds^2 &=& (g_{\mu\nu} + h_{\mu\nu})dx^\mu dx^\nu = \Omega^2(\tau)(\tilde g_{\mu\nu} + f_{\mu\nu})dx^\mu dx^\nu
\\
\tilde g_{\mu\nu} &=& \text{diag}\left(-1,\frac{1}{1-kr^2},r^2,r^2\sin^2\theta\right)\qquad \tilde \Gamma^{\lambda}_{\alpha\beta} = \delta^\lambda_i \delta^j_\alpha \delta^k_\beta \tilde \Gamma^{i}_{jk}
\end{eqnarray}
%
%%%%%%%%%%%%%%%%%%%%%%%%%%%%%%%%%%%%
\subsection{$f_{\mu\nu}(SVT3)$}
%%%%%%%%%%%%%%%%%%%%%%%%%%%%%%%%%%%%
\begin{eqnarray}
f_{00} &=& -2\phi
\nonumber\\
f_{0i} &=& B_i + \tilde\nabla_i B
\nonumber\\
f_{ij} &=& -2\tilde g_{ij} \psi + 2\tilde\nabla_i\tilde\nabla_j E + \tilde\nabla_i E_j + \tilde\nabla_j E_i + 2E_{ij}
\nonumber\\
\tilde g^{ij} f_{ij} &=& -6\psi + 2\tilde\nabla^k\tilde\nabla_k E
\nonumber\\
\tilde g^{\mu\nu} f_{\mu\nu} &=& 2\phi-6\psi + 2\tilde\nabla^k\tilde\nabla_k E
\\ \nonumber\\
-\tilde\nabla^a \tilde\nabla^\alpha \Omega f_{a\alpha} &=& \dot\Omega \tilde\nabla_a\tilde\nabla^a B
\end{eqnarray}
%%%%%%%%%%%%%%%%
%
%%%%%%%%%%%%%%%%%%%%%%%%%%%%%%%%%%%%
\subsection{$SVT3(f_{\mu\nu})$}
%%%%%%%%%%%%%%%%%%%%%%%%%%%%%%%%%%%%
\begin{eqnarray}
\phi &=& -\tfrac{1}{2} f_{00}
\label{phif}
\\ \nonumber\\
\tilde\nabla_a\tilde\nabla^a B &=& \tilde\nabla^a f_{0a}
\\ \nonumber\\
(\tilde\nabla_a\tilde\nabla^a -2k)B_i &=& (\tilde\nabla_a\tilde\nabla^a -2k)f_{0i} -\tilde\nabla_i \tilde\nabla^a f_{0a}
\\ \nonumber\\
(\tilde\nabla_a\tilde\nabla^a +3 k)\psi &=& \frac{1}{4}\left[\tilde\nabla^a\tilde\nabla^b f_{ab}-(\tilde\nabla_a\tilde\nabla^a+2k) (\tilde g^{bc}f_{bc})\right]
\\ \nonumber\\
(\tilde\nabla_a\tilde\nabla^a +3 k)\tilde\nabla_b\tilde\nabla^b E &=& \frac{3}{4}\left[ \tilde\nabla^a\tilde\nabla^b f_{ab} -\frac{1}{3}\tilde\nabla_a\tilde\nabla^a (\tilde g^{bc}f_{bc})\right]
\\ \nonumber\\
(\tilde\nabla_a\tilde\nabla^a +2 k)(\tilde\nabla_b\tilde\nabla^b-2k) E_i &=&
(\tilde\nabla_a\tilde\nabla^a -2k)\tilde\nabla^b f_{ib} - \tilde\nabla_i \tilde\nabla^a\tilde\nabla^b f_{ab}
\\ \nonumber\\
(\tilde\nabla_a\tilde\nabla^a-2k)(\tilde\nabla_b\tilde\nabla^b -3k)(2E_{ij})
&=& 
(\tilde\nabla_a\tilde\nabla^a-2k)(\tilde\nabla_b\tilde\nabla^b-3k)f_{ij}
+ \tfrac12 \tilde\nabla_i\tilde\nabla_j\big[ \tilde\nabla^a\tilde\nabla^b f_{ab} + (\tilde\nabla_a\tilde\nabla^a +4k)(\tilde g^{bc}f_{bc})\big]
\nonumber\\
&&
+\tfrac12 \tilde g_{ij} \big[ (\tilde\nabla_a\tilde\nabla^a-4k)\tilde\nabla^b\tilde\nabla^c f_{bc}-(\tilde\nabla_a\tilde\nabla^a\tilde\nabla_b\tilde\nabla^b -2k \tilde\nabla_a\tilde\nabla^a +4k^2)(\tilde g^{bc}f_{bc})\big]
\nonumber\\
&&
-(\tilde\nabla_a\tilde\nabla^a -3k)(\tilde\nabla_i\tilde\nabla^b f_{jb} + \tilde\nabla_j \tilde\nabla^b f_{ib})
\label{ftt}
\end{eqnarray}
%%%%%%%%%%%%%%%
\subsection{$\Delta_\epsilon[SVT3]$}
%%%%%%%%%%%%%%%%%%%%%%%%%%%%%%%
\begin{eqnarray}
\bar x^\mu =  x^\mu - \epsilon^\mu(x) &\implies&  \bar h_{\mu\nu} = h_{\mu\nu} + \nabla_\mu \epsilon_\nu + \nabla_\nu \epsilon_\mu
\\ \nonumber\\
\Delta_\epsilon\left[ \phi \right]&=&  \dot{\Omega} \Omega^{-1} (\tilde{\nabla}_{a}\tilde{\nabla}^{a}+3k)f_{0}{}
\\ \nonumber\\
\Delta_\epsilon\left[ \tilde\nabla_a\tilde\nabla^a B \right] &=& \tilde{\nabla}_{a}\dot{f}^{a} + \tilde{\nabla}_{a}\tilde{\nabla}^{a}f_{0}{}
\\ \nonumber\\
\Delta_\epsilon\left[ (\tilde\nabla_a\tilde\nabla^a -2k)B_i \right] &=& (\tilde{\nabla}_{a}\tilde{\nabla}^{a}-2k)\dot{f}_{i} -  \tilde{\nabla}_{i}\tilde{\nabla}_{a}\dot{f}^{a}
\\ \nonumber\\
\Delta_\epsilon\left[(\tilde\nabla_a\tilde\nabla^a +3 k)\psi \right] &=& - \dot{f}_{0}{} -  \dot{\Omega} f_{0}{} \Omega^{-1}
\\ \nonumber\\
\Delta_\epsilon\left[ (\tilde\nabla_a\tilde\nabla^a +3 k)\tilde\nabla_b\tilde\nabla^b E \right] &=&(\tilde{\nabla}_{b}\tilde{\nabla}^{b}+3k)\tilde{\nabla}_{a}f^{a}
\\ \nonumber\\
\Delta_\epsilon\left[(\tilde\nabla_a\tilde\nabla^a +2 k)(\tilde\nabla_b\tilde\nabla^b-2k) E_i \right] &=& (\tilde\nabla_a\tilde\nabla^a +2 k)(\tilde\nabla_b\tilde\nabla^b-2k)f_{i} -  \tilde{\nabla}_{i}(\tilde{\nabla}_{b}\tilde{\nabla}^{b}+4k)\tilde{\nabla}_{a}f^{a}
\\ \nonumber\\
\Delta_\epsilon\left[(\tilde\nabla_a\tilde\nabla^a-2k)(\tilde\nabla_b\tilde\nabla^b -3k)(2E_{ij}) \right] &=& 0
\end{eqnarray}
%%%%%%%%%%%%%%%%%%%%%%%%%%%%%%%
\subsection{Gauge Invariants}
%%%%%%%%%%%%%%%%%%%%%%%%%%%%%%%
We mix time derivative notation a bit, using $\partial_0$ upon $f_{\mu\nu}$ and dot upon $\Omega$ and SVT3 quantities. 
\begin{eqnarray}
(\tilde\nabla_a\tilde\nabla^a + 3k)\tilde\nabla_b\tilde\nabla^b[ \phi +\psi + \dot B - \ddot E] &=& (\tilde\nabla_a\tilde\nabla^a + 3k)\tilde\nabla^b (\partial_0 f_{0b})
-\tfrac14 (\tilde\nabla_a\tilde\nabla^a +2k-\partial_0^2)\tilde\nabla_b\tilde\nabla^b(\tilde g^{cd}f_{cd})
\nonumber\\
&&+\tfrac14 (\tilde\nabla_a\tilde\nabla^a-3\partial_0^2)\tilde\nabla^b\tilde\nabla^c f_{bc}
-\tfrac12 (\tilde\nabla_a\tilde\nabla^a + 3k)\tilde\nabla_b\tilde\nabla^b f_{00}
\\ \nonumber\\
%
(\tilde\nabla_a\tilde\nabla^a + 3k)\tilde\nabla_b\tilde\nabla^b[ \psi-\dot \Omega \Omega^{-1}(B-\dot E)] &=& 
-\dot\Omega \Omega^{-1}(\tilde\nabla_a\tilde\nabla^a + 3k)\tilde\nabla^b f_{0b}
+\tfrac14 (\tilde\nabla_a\tilde\nabla^a+3\dot\Omega \Omega^{-1}\partial_0)\tilde\nabla^b\tilde\nabla^c f_{bc}
\nonumber\\
&& -\tfrac14 (\tilde\nabla_a\tilde\nabla^a +2k + \dot\Omega \Omega^{-1} \partial_0)\tilde\nabla_b\tilde\nabla^b (\tilde g^{cd}f_{cd})
\\ \nonumber\\
%
(\tilde\nabla_a\tilde\nabla^a +2 k)(\tilde\nabla_b\tilde\nabla^b-2k)[B_i -\dot E_i] &=& (\tilde\nabla_a\tilde\nabla^a +2 k)(\tilde\nabla_b\tilde\nabla^b-2k)f_{0i}
-(\tilde\nabla_a\tilde\nabla^a-2k)\tilde\nabla^b (\partial_0 f_{ib})
\nonumber\\
&&
-\tilde\nabla_i (\tilde\nabla_a\tilde\nabla^a+4k)\tilde\nabla^b f_{0b}
+\tilde\nabla_i \tilde\nabla^a \tilde\nabla^b (\partial_0 f_{ab})
\\ \nonumber\\
%
(\tilde\nabla_a\tilde\nabla^a-2k)(\tilde\nabla_b\tilde\nabla^b -3k)[2E_{ij}]
&=& 
(\tilde\nabla_a\tilde\nabla^a-2k)(\tilde\nabla_b\tilde\nabla^b-3k)f_{ij}
+ \tfrac12 \tilde\nabla_i\tilde\nabla_j\big[ \tilde\nabla^a\tilde\nabla^b f_{ab} + (\tilde\nabla_a\tilde\nabla^a +4k)(\tilde g^{bc}f_{bc})\big]
\nonumber\\
&&
+\tfrac12 \tilde g_{ij} \big[ (\tilde\nabla_a\tilde\nabla^a-4k)\tilde\nabla^b\tilde\nabla^c f_{bc}-(\tilde\nabla_a\tilde\nabla^a\tilde\nabla_b\tilde\nabla^b -2k \tilde\nabla_a\tilde\nabla^a +4k^2)(\tilde g^{bc}f_{bc})\big]
\nonumber\\
&&
-(\tilde\nabla_a\tilde\nabla^a -3k)(\tilde\nabla_i\tilde\nabla^b f_{jb} + \tilde\nabla_j \tilde\nabla^b f_{ib})
\end{eqnarray}
\\ \\
---------------------------------------------------------------------------------------------------------------------------------------------------------
---------------------------------------------------------------------------------------------------------------------------------------------------------
%%%%%%%%%%%%%%%%%%%%%%%%%%%%%%%%
\section{RW $\Omega(T,R)$}
%%%%%%%%%%%%%%%%%%%%%%%%%%%%%%%%%%
\begin{eqnarray}
ds^2 &=& (g'_{\mu\nu} + h'_{\mu\nu})dx'^\mu dx'^\nu = \Omega^2(T,R)(\tilde g'_{\mu\nu} + f_{\mu\nu})dx'^\mu dx'^\nu
\\ \nonumber\\
\tilde g'_{\mu\nu} &=& \text{diag}\left(-1,1,R^2,R^2\sin^2\theta\right)
\end{eqnarray}
%
%%%%%%%%%%%%%%%%%%%%%%%%%%%%%%%%%%%%
\subsection{$f_{\mu\nu}(SVT3)$}
%%%%%%%%%%%%%%%%%%%%%%%%%%%%%%%%%%%%
\begin{eqnarray}
f_{00} &=& -2\phi
\nonumber\\
f_{0i} &=& B_i + \tilde\nabla_i B
\nonumber\\
f_{ij} &=& -2\tilde g'_{ij} \psi + 2\tilde\nabla_i\tilde\nabla_j E + \tilde\nabla_i E_j + \tilde\nabla_j E_i + 2E_{ij}
\nonumber\\
\tilde g'^{ij} f_{ij} &=& -6\psi + 2\tilde\nabla^k\tilde\nabla_k E
\nonumber\\
\tilde g'^{\mu\nu} f_{\mu\nu} &=& 2\phi-6\psi + 2\tilde\nabla^k\tilde\nabla_k E
\end{eqnarray}
%%%%%%%%%%%%%%%%
%
%%%%%%%%%%%%%%%%%%%%%%%%%%%%%%%%%%%%
\subsection{$SVT3(f_{\mu\nu})$}
%%%%%%%%%%%%%%%%%%%%%%%%%%%%%%%%%%%%
These quantities mimic \eqref{phif}-\eqref{ftt} with $k=0$.
\begin{eqnarray}
\phi &=& -\tfrac{1}{2} f_{00}
\label{phifp}
\\ \nonumber\\
\tilde\nabla_a\tilde\nabla^a B &=& \tilde\nabla^a f_{0a}
\\ \nonumber\\
\tilde\nabla_a\tilde\nabla^a B_i &=& \tilde\nabla_a\tilde\nabla^a f_{0i} -\tilde\nabla_i \tilde\nabla^a f_{0a}
\\ \nonumber\\
\tilde\nabla_a\tilde\nabla^a\psi &=& \frac{1}{4}\left[\tilde\nabla^a\tilde\nabla^b f_{ab}-\tilde\nabla_a\tilde\nabla^a (\tilde g'^{bc}f_{bc})\right]
\\ \nonumber\\
\tilde\nabla_a\tilde\nabla^a\tilde\nabla_b\tilde\nabla^b E &=& \frac{3}{4}\left[ \tilde\nabla^a\tilde\nabla^b f_{ab} -\frac{1}{3}\tilde\nabla_a\tilde\nabla^a (\tilde g'^{bc}f_{bc})\right]
\\ \nonumber\\
\tilde\nabla_a\tilde\nabla^a \tilde\nabla_b\tilde\nabla^b E_i &=&
\tilde\nabla_a\tilde\nabla^a \tilde\nabla^b f_{ib} - \tilde\nabla_i \tilde\nabla^a\tilde\nabla^b f_{ab}
\label{Ep}
\\ \nonumber\\
\tilde\nabla_a\tilde\nabla^a\tilde\nabla_b\tilde\nabla^b (2E_{ij})
&=& 
\tilde\nabla_a\tilde\nabla^a\tilde\nabla_b\tilde\nabla^bf_{ij}
+ \tfrac12 \tilde\nabla_i\tilde\nabla_j\big[ \tilde\nabla^a\tilde\nabla^b f_{ab} + \tilde\nabla_a\tilde\nabla^a (\tilde g'^{bc}f_{bc})\big]
\nonumber\\
&&
+\tfrac12 \tilde g'_{ij} \big[ \tilde\nabla_a\tilde\nabla^a\tilde\nabla^b\tilde\nabla^c f_{bc}-\tilde\nabla_a\tilde\nabla^a\tilde\nabla_b\tilde\nabla^b(\tilde g'^{bc}f_{bc})\big]
\nonumber\\
&&
-\tilde\nabla_a\tilde\nabla^a (\tilde\nabla_i\tilde\nabla^b f_{jb} + \tilde\nabla_j \tilde\nabla^b f_{ib})
\label{fttp}
\end{eqnarray}

%%%%%%%%%%%%%%%%%%%%%%%%%%%%%%%%1
\subsection{$\Delta_{\epsilon}[f_{\mu\nu}]$}
%%%%%%%%%%%%%%%%%%%%%%%%%%%%%%%
\begin{eqnarray}
\bar x^\mu &=&  x'^\mu - \epsilon^\mu(x) \implies \Delta_\epsilon \left[ h_{\mu\nu}\right] = \nabla_\mu \epsilon_\nu +\nabla_\nu \epsilon_\mu
\label{gaugetrans2}
\\ \nonumber\\
f_{\mu} &=& \Omega^2 \epsilon_\mu,\qquad f^\mu = \epsilon^\mu
\\ \nonumber\\
f_0 &=& -T,\qquad f_a = \tilde\nabla_a L + L_a,\qquad \tilde\nabla^a L_a =0
\\ \nonumber\\
\Delta_\epsilon\left[ f_{\mu\nu} \right]  &=& \tilde\nabla_{\mu }f_{\nu} + \tilde\nabla_{\nu}f_{\mu} + 2 f^{\gamma } \tilde g'_{\mu\nu} \Omega^{-1} \tilde\nabla_{\gamma }\Omega
\\ \nonumber\\
\Delta_\epsilon\left[ \tilde f_{00} \right] &=&  -2\dot T - 2\Omega^{-1} \left[ T \dot \Omega + (\tilde\nabla^a L + L^a)\tilde\nabla_a \Omega\right]
\\ \nonumber\\
\Delta_\epsilon\left[ \tilde f_{0i} \right]  &=&   \tilde\nabla_i \dot L +\dot L_i - \tilde\nabla_i T
\\ \nonumber\\
\Delta_\epsilon\left[ \tilde f_{ij}  \right]  &=& 2\tilde\nabla_i\tilde\nabla_j L + \tilde\nabla_{i }L_{j } + \tilde\nabla_{j}L_{i} + 2\Omega^{-1} \tilde g_{ij} \left[ T \dot \Omega + (\tilde\nabla^a L + L^a)\tilde\nabla_a \Omega\right]
\\ \nonumber\\
\Delta_\epsilon\left[ \tilde g'^{ab} f_{ab} \right]  &=& 2\tilde\nabla_a\tilde\nabla^a L + 6\Omega^{-1}\left[ T \dot \Omega + (\tilde\nabla^a L + L^a)\tilde\nabla_a \Omega\right]
\\ \nonumber\\
\Delta_\epsilon\left[ \tilde g'^{\alpha\beta} f_{\alpha\beta} \right]  &=& 2\dot T+ 2\tilde\nabla_a\tilde\nabla^a L + 8\Omega^{-1}\left[ T \dot \Omega + (\tilde\nabla^a L + L^a)\tilde\nabla_a \Omega\right]
\\ \nonumber\\
\Delta_\epsilon\left[ \tilde\nabla^a f_{0a} \right] &=& \tilde\nabla_a\tilde\nabla^a ( \dot L- T)
\\ \nonumber\\
\Delta_\epsilon\left[ \tilde\nabla^b f_{ab}\right] &=& 2 \dot{\Omega} \Omega^{-1} \tilde{\nabla}_{a}T + T (2 \Omega^{-1} \tilde{\nabla}_{a}\dot{\Omega} - 2 \dot{\Omega} \Omega^{-2} \tilde{\nabla}_{a}\Omega) + 2 \Omega^{-1} \tilde{\nabla}_{a}L^{b} \tilde{\nabla}_{b}\Omega \nonumber \\ 
&& + L^{b} (-2 \Omega^{-2} \tilde{\nabla}_{a}\Omega \tilde{\nabla}_{b}\Omega + 2 \Omega^{-1} \tilde{\nabla}_{b}\tilde{\nabla}_{a}\Omega) + \tilde{\nabla}_{b}\tilde{\nabla}^{b}L_{a} + 2 \tilde{\nabla}_{b}\tilde{\nabla}^{b}\tilde{\nabla}_{a}L \nonumber \\ 
&& + (-2 \Omega^{-2} \tilde{\nabla}_{a}\Omega \tilde{\nabla}_{b}\Omega + 2 \Omega^{-1} \tilde{\nabla}_{b}\tilde{\nabla}_{a}\Omega) \tilde{\nabla}^{b}L + 2 \Omega^{-1} \tilde{\nabla}_{b}\tilde{\nabla}_{a}L \tilde{\nabla}^{b}\Omega 
\end{eqnarray}

%%%%%%%%%%%%%%%%%%%%%%%%%%%%%%%
\section{Integral Relations}
%%%%%%%%%%%%%%%%%%%%%%%%%%%%%%%
%
%
\begin{eqnarray}
\Delta_\epsilon[\phi] &=& \dot{T} + \Omega^{-1}\left[ T \dot \Omega + (\tilde\nabla^a L + L^a)\tilde\nabla_a \Omega\right]
\nonumber\\
%
\Delta_\epsilon[B] &=& \int D \nabla^2(\dot L-T)
\nonumber\\
%
\Delta_\epsilon[B_i] &=&\dot L_i + \nabla_i(\dot L -T) -\nabla_i \int D \nabla^2(\dot L-T)
\nonumber\\
%
\Delta_\epsilon[\psi] &=&  -\tfrac12 \tilde\nabla_a\tilde\nabla^a L - \tfrac32 \Omega^{-1}\left[ T \dot \Omega + (\tilde\nabla^a L + L^a)\tilde\nabla_a \Omega\right]
\nonumber\\
&& + \tilde\nabla^a \bigg[ \int D \bigg( 
\tfrac{1}{2} \dot{\Omega} \Omega^{-1} \tilde{\nabla}_{a}T + T (\tfrac{1}{2} \Omega^{-1} \tilde{\nabla}_{a}\dot{\Omega} -  \tfrac{1}{2} \dot{\Omega} \Omega^{-2} \tilde{\nabla}_{a}\Omega) + \tfrac{1}{2} \Omega^{-1} \tilde{\nabla}_{a}L^{b} \tilde{\nabla}_{b}\Omega \nonumber \\ 
&& + L^{b} (- \tfrac{1}{2} \Omega^{-2} \tilde{\nabla}_{a}\Omega \tilde{\nabla}_{b}\Omega + \tfrac{1}{2} \Omega^{-1} \tilde{\nabla}_{b}\tilde{\nabla}_{a}\Omega) + \tfrac{1}{4} \tilde{\nabla}_{b}\tilde{\nabla}^{b}L_{a} + \tfrac{1}{2} \tilde{\nabla}_{b}\tilde{\nabla}^{b}\tilde{\nabla}_{a}L \nonumber \\ 
&& + (- \tfrac{1}{2} \Omega^{-2} \tilde{\nabla}_{a}\Omega \tilde{\nabla}_{b}\Omega + \tfrac{1}{2} \Omega^{-1} \tilde{\nabla}_{b}\tilde{\nabla}_{a}\Omega) \tilde{\nabla}^{b}L + \tfrac{1}{2} \Omega^{-1} \tilde{\nabla}_{b}\tilde{\nabla}_{a}L \tilde{\nabla}^{b}\Omega 
\bigg)\bigg]
\nonumber\\
%
\Delta_\epsilon[E] &=& \int D \Bigg[ -\tfrac12 \tilde\nabla_a\tilde\nabla^a L - \tfrac32 \Omega^{-1}\left[ T \dot \Omega + (\tilde\nabla^a L + L^a)\tilde\nabla_a \Omega\right]
\nonumber\\
&& + \tilde\nabla^a \bigg[ \int D \bigg(  \tfrac{3}{2} \dot{\Omega} \Omega^{-1} \tilde{\nabla}_{a}T + T (\tfrac{3}{2} \Omega^{-1} \tilde{\nabla}_{a}\dot{\Omega} -  \tfrac{3}{2} \dot{\Omega} \Omega^{-2} \tilde{\nabla}_{a}\Omega) + \tfrac{3}{2} \Omega^{-1} \tilde{\nabla}_{a}L^{b} \tilde{\nabla}_{b}\Omega \nonumber \\ 
&& + L^{b} (- \tfrac{3}{2} \Omega^{-2} \tilde{\nabla}_{a}\Omega \tilde{\nabla}_{b}\Omega + \tfrac{3}{2} \Omega^{-1} \tilde{\nabla}_{b}\tilde{\nabla}_{a}\Omega) + \tfrac{3}{4} \tilde{\nabla}_{b}\tilde{\nabla}^{b}L_{a} + \tfrac{3}{2} \tilde{\nabla}_{b}\tilde{\nabla}^{b}\tilde{\nabla}_{a}L \nonumber \\ 
&& + (- \tfrac{3}{2} \Omega^{-2} \tilde{\nabla}_{a}\Omega \tilde{\nabla}_{b}\Omega + \tfrac{3}{2} \Omega^{-1} \tilde{\nabla}_{b}\tilde{\nabla}_{a}\Omega) \tilde{\nabla}^{b}L + \tfrac{3}{2} \Omega^{-1} \tilde{\nabla}_{b}\tilde{\nabla}_{a}L \tilde{\nabla}^{b}\Omega 
\bigg)\bigg]\Bigg]
\nonumber\\
%
\Delta_\epsilon[E_i] &=& \int D \bigg[ \tilde{\nabla}_{a}\tilde{\nabla}^{a}L_{i} + 2 \Omega^{-1} \tilde{\nabla}_{a}\Omega \tilde{\nabla}_{i}L^{a} + 2 \dot{\Omega} \Omega^{-1} \tilde{\nabla}_{i}T + T (2 \Omega^{-1} \tilde{\nabla}_{i}\dot{\Omega} - 2 \dot{\Omega} \Omega^{-2} \tilde{\nabla}_{i}\Omega) + 2 \Omega^{-1} \tilde{\nabla}^{a}\Omega \tilde{\nabla}_{i}\tilde{\nabla}_{a}L \nonumber \\ 
&& + L^{a} (-2 \Omega^{-2} \tilde{\nabla}_{a}\Omega \tilde{\nabla}_{i}\Omega + 2 \Omega^{-1} \tilde{\nabla}_{i}\tilde{\nabla}_{a}\Omega) + \tilde{\nabla}^{a}L (-2 \Omega^{-2} \tilde{\nabla}_{a}\Omega \tilde{\nabla}_{i}\Omega + 2 \Omega^{-1} \tilde{\nabla}_{i}\tilde{\nabla}_{a}\Omega) \nonumber \\ 
&& + 2 \tilde{\nabla}_{i}\tilde{\nabla}_{a}\tilde{\nabla}^{a}L \bigg]
- \tilde\nabla_i \int D \tilde\nabla^b \int D \bigg[ 
 \tilde{\nabla}_{a}\tilde{\nabla}^{a}L_{b} + 2 \Omega^{-1} \tilde{\nabla}_{a}\Omega \tilde{\nabla}_{b}L^{a} + 2 \dot{\Omega} \Omega^{-1} \tilde{\nabla}_{b}T + T (2 \Omega^{-1} \tilde{\nabla}_{b}\dot{\Omega} - 2 \dot{\Omega} \Omega^{-2} \tilde{\nabla}_{b}\Omega) \nonumber\\
&& + 2 \Omega^{-1} \tilde{\nabla}^{a}\Omega \tilde{\nabla}_{b}\tilde{\nabla}_{a}L  + L^{a} (-2 \Omega^{-2} \tilde{\nabla}_{a}\Omega \tilde{\nabla}_{b}\Omega + 2 \Omega^{-1} \tilde{\nabla}_{b}\tilde{\nabla}_{a}\Omega) + \tilde{\nabla}^{a}L (-2 \Omega^{-2} \tilde{\nabla}_{a}\Omega \tilde{\nabla}_{b}\Omega + 2 \Omega^{-1} \tilde{\nabla}_{b}\tilde{\nabla}_{a}\Omega) \nonumber \\ 
&& + 2 \tilde{\nabla}_{b}\tilde{\nabla}_{a}\tilde{\nabla}^{a}L
\bigg]
\end{eqnarray}
We may also include the trace condition
\begin{eqnarray}
-6\bar \psi + 2\nabla^2 \bar E &=& -6 \psi + 2\nabla^2 E +2 \nabla^2 L.
\end{eqnarray}

%%%%%%%%%%%%%%%%%%%%%%%%%%%%%%%%
\subsection{$\Delta_\epsilon[SVT3]$}
%%%%%%%%%%%%%%%%%%%%%%%%%%%%%%%
\begin{eqnarray}
\Delta_\epsilon\left[ \phi \right]&=& - \dot{f}_{0}{} -  \dot{\Omega} f_{0}{} \Omega^{-1} + f^{a} \Omega^{-1} \tilde{\nabla}_{a}\Omega
\\ \nonumber\\
\Delta_\epsilon\left[ \tilde\nabla_a\tilde\nabla^a B \right] &=&\tilde{\nabla}'_{a}\dot{f}^{a} + \tilde{\nabla}'_{a}\tilde{\nabla}'^{a}f_{0}{}
\\ \nonumber\\
\Delta_\epsilon\left[ \tilde\nabla_a\tilde\nabla^a B_i \right] &=&\tilde{\nabla}'_{a}\tilde{\nabla}'^{a}\dot{f}_{i} -  \tilde{\nabla}'_{a}\tilde{\nabla}'_{i}\dot{f}^{a}
\\ \nonumber\\
\Delta_{\epsilon}\left[ \tilde\nabla_a\tilde\nabla^a \psi \right]&=& f_{0}{} \Omega^{-1} \tilde{\nabla}'_{a}\tilde{\nabla}'^{a}\dot{\Omega} + \dot{\Omega} \Omega^{-1} \tilde{\nabla}'_{a}\tilde{\nabla}'^{a}f_{0}{} -  \dot{\Omega} f_{0}{} \Omega^{-2} \tilde{\nabla}'_{a}\tilde{\nabla}'^{a}\Omega + 2 \Omega^{-1} \tilde{\nabla}'_{a}f_{0}{} \tilde{\nabla}'^{a}\dot{\Omega} - 2 f_{0}{} \Omega^{-2} \tilde{\nabla}'_{a}\Omega \tilde{\nabla}'^{a}\dot{\Omega} \nonumber \\ 
&& - 2 \dot{\Omega} \Omega^{-2} \tilde{\nabla}'_{a}\Omega \tilde{\nabla}'^{a}f_{0}{} + 2 \dot{\Omega} f_{0}{} \Omega^{-3} \tilde{\nabla}'_{a}\Omega \tilde{\nabla}'^{a}\Omega -  \Omega^{-1} \tilde{\nabla}'^{a}\Omega \tilde{\nabla}'_{b}\tilde{\nabla}'^{b}f_{a} + f^{a} \Omega^{-2} \tilde{\nabla}'_{a}\Omega \tilde{\nabla}'_{b}\tilde{\nabla}'^{b}\Omega \nonumber \\ 
&& -  f^{a} \Omega^{-1} \tilde{\nabla}'_{b}\tilde{\nabla}'^{b}\tilde{\nabla}'_{a}\Omega + 2 \Omega^{-2} \tilde{\nabla}'_{a}\Omega \tilde{\nabla}'_{b}\Omega \tilde{\nabla}'^{b}f^{a} - 2 \Omega^{-1} \tilde{\nabla}'_{b}\tilde{\nabla}'_{a}\Omega \tilde{\nabla}'^{b}f^{a} - 2 f^{a} \Omega^{-3} \tilde{\nabla}'_{a}\Omega \tilde{\nabla}'_{b}\Omega \tilde{\nabla}'^{b}\Omega \nonumber \\ 
&& + 2 f^{a} \Omega^{-2} \tilde{\nabla}'_{b}\tilde{\nabla}'_{a}\Omega \tilde{\nabla}'^{b}\Omega 
\nonumber \\ \\
\Delta_\epsilon\left[ \tilde\nabla_a\tilde\nabla^a\tilde\nabla_b\tilde\nabla^b E \right] &=&\tilde{\nabla}'_{b}\tilde{\nabla}'^{b}\tilde{\nabla}'_{a}f^{a}
\\ \nonumber\\
\Delta_\epsilon\left[\tilde\nabla_a\tilde\nabla^a\tilde\nabla_b\tilde\nabla^b E_i \right] &=&\tilde{\nabla}'_{b}\tilde{\nabla}'^{b}\tilde{\nabla}'_{a}\tilde{\nabla}'^{a}f_{i} -  \tilde{\nabla}'_{b}\tilde{\nabla}'^{b}\tilde{\nabla}'_{i}\tilde{\nabla}'_{a}f^{a}
\\ \nonumber\\
\Delta_\epsilon\left[\tilde\nabla_a\tilde\nabla^a\tilde\nabla_b\tilde\nabla^b(2E_{ij}) \right] &=& 0
\end{eqnarray}
%%%%%%%%%%%%%%%%%%%%%%%%%%%%%%%
\subsection{Gauge Invariants}
%%%%%%%%%%%%%%%%%%%%%%%%%%%%%%%
We mix time derivative notation a bit, using $\partial_0$ upon $f_{\mu\nu}$ and dot upon $\Omega$ and SVT3 quantities. 
\begin{eqnarray}
\tilde\nabla_a\tilde\nabla^a\tilde\nabla_b\tilde\nabla^b[ \phi +\psi + \dot B - \ddot E] &=& \tilde\nabla_a\tilde\nabla^a\tilde\nabla^b (\partial_0 f_{0b})
-\tfrac14 (\tilde\nabla_a\tilde\nabla^a-\partial_0^2)\tilde\nabla_b\tilde\nabla^b(\tilde g'^{cd}f_{cd})
\nonumber\\
&&+\tfrac14 (\tilde\nabla_a\tilde\nabla^a-3\partial_0^2)\tilde\nabla^b\tilde\nabla^c f_{bc}
-\tfrac12 \tilde\nabla_a\tilde\nabla^a\tilde\nabla_b\tilde\nabla^b f_{00}
\\ \nonumber\\
%
\tilde\nabla_a\tilde\nabla^a\tilde\nabla_b\tilde\nabla^b\times
\nonumber\\
\bigg[\psi- \Omega^{-1}[(B-\dot E)\dot\Omega- (\tilde\nabla_a E + E_a)\tilde\nabla^a\Omega]\bigg] &=& ?
\\ \nonumber\\
%
\tilde\nabla_a\tilde\nabla^a\tilde\nabla_b\tilde\nabla^b[B'_i -\dot E'_i] &=& \tilde\nabla_a\tilde\nabla^a\tilde\nabla_b\tilde\nabla^b f_{0i}
-\tilde\nabla_a\tilde\nabla^a\tilde\nabla^b (\partial_0 f_{ib})
-\tilde\nabla_i \tilde\nabla_a\tilde\nabla^a\tilde\nabla^b f_{0b}
+\tilde\nabla_i \tilde\nabla^a \tilde\nabla^b (\partial_0 f_{ab})
\nonumber\\ \\ \nonumber\\
%
\tilde\nabla_a\tilde\nabla^a\tilde\nabla_b\tilde\nabla^b [2E_{ij}]
&=& 
\tilde\nabla_a\tilde\nabla^a\tilde\nabla_b\tilde\nabla^bf_{ij}
+ \tfrac12 \tilde\nabla_i\tilde\nabla_j\big[ \tilde\nabla^a\tilde\nabla^b f_{ab} + \tilde\nabla_a\tilde\nabla^a (\tilde g'^{bc}f_{bc})\big]
\nonumber\\
&&
+\tfrac12 \tilde g'_{ij} \big[ \tilde\nabla_a\tilde\nabla^a\tilde\nabla^b\tilde\nabla^c f_{bc}-\tilde\nabla_a\tilde\nabla^a\tilde\nabla_b\tilde\nabla^b(\tilde g'^{bc}f_{bc})\big]
\nonumber\\
&&
-\tilde\nabla_a\tilde\nabla^a (\tilde\nabla_i\tilde\nabla^b f_{jb} + \tilde\nabla_j \tilde\nabla^b f_{ib})
\end{eqnarray}
%
%



%%%%%%%%%%%%%%%%%%%%%%%%%%%%%%%
\subsection{On the G.I. of $\psi- \Omega^{-1}[(B-\dot E)\dot\Omega- (\tilde\nabla_a E + E_a)\tilde\nabla^a\Omega]$ }
%%%%%%%%%%%%%%%%%%%%%%%%%%%%%%%
In the conformal to flat decomposition, $E_i$ is given by the integral
\begin{eqnarray}
E_i &=& \int D \tilde\nabla^k f_{ik} - \tilde\nabla_i \int D \tilde\nabla^k \tilde\nabla^l f_{kl},\qquad \tilde\nabla_a\tilde\nabla^a D(x,x') = \delta(x-x').
\end{eqnarray}
As given in \eqref{Ep}, the lowest derivative relation in terms of $f_{\mu\nu}$ for $E_i$ is
\begin{eqnarray}
\tilde\nabla_a\tilde\nabla^a \tilde\nabla_b\tilde\nabla^b E_i &=&
\tilde\nabla_a\tilde\nabla^a \tilde\nabla^b f_{ib} - \tilde\nabla_i \tilde\nabla^a\tilde\nabla^b f_{ab}.
\end{eqnarray}
$E_i$ can also be found as a single derivative within $f_{ij}$
\begin{eqnarray}
f_{ij}&=& -2\tilde g_{ij}\psi + 2\tilde\nabla_i\tilde\nabla_j E + \tilde\nabla_i E_j + \tilde\nabla_j E_i + 2E_{ij}
\end{eqnarray}
When we take any derivative upon the gauge invariant
\begin{eqnarray}
\psi- \Omega^{-1}[(B-\dot E)\dot\Omega- (\tilde\nabla_a E + E_a)\tilde\nabla^a\Omega],
\end{eqnarray}
from the product rule we will necessarily generate terms that depend on $E_a$ alone; i.e. terms that could only be expressed as integrals over $f_{ij}$ and not derivatives of $f_{ij}$. Consequently, it would not seem possible to construct this gauge invariant based on any combination of $f_{\mu\nu}$ or derivatives thereof. 
\\ \\
It would then seem puzzling how we were able to express $\Delta_{\mu\nu}$ in terms of the gauge invariant $\gamma=\psi- \Omega^{-1}[(B-\dot E)\dot\Omega- (\tilde\nabla_a E + E_a)\tilde\nabla^a\Omega]$. Looking at \emph{RW\_Radiation\_SVT3\_Conformal\_Flat\_-k\_Cartesian\_v2.pdf}, it turns out that neither $\delta G_{\mu\nu}$ nor $\delta T_{\mu\nu}$ have any terms that depend on $E_i$ without derivatives. When forming the gauge invariant combinations, we made substitutions like
\begin{eqnarray}
\psi &=& \gamma + \Omega^{-1}[(B-\dot E)\dot\Omega- (\tilde\nabla_a E + E_a)\tilde\nabla^a\Omega].
\end{eqnarray}
All contributions of $E_a$ that we originally introduce end up canceling after simplifying all relevant terms.
\end{document}