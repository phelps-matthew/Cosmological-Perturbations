\documentclass[10pt,letterpaper]{article}
\usepackage[textwidth=7in, top=1in,textheight=9in]{geometry}
\usepackage[fleqn]{mathtools} 
\usepackage{amssymb}
\usepackage{hyperref}
\numberwithin{equation}{subsection}

\title{Helicity Decomposition v1}
\author{}
\date{}

\begin{document}
\maketitle
\tableofcontents
\newpage
\section{Motivation}
\subsection{Helicity}
There are a couple sources in literature that suggest the SVT decomposition is equivalent to a decomposition of $h_{\mu\nu}$ into its irreducible representations of SO(3) in which the equations of motion then split into equations of helicity 0, $\pm1$ and $\pm 2$. When we say scalar, vector, and tensor, we are really referring to SO(3) scalars, vectors, and tensors. For a scalar such as $h_{00} = -2\phi$ actually mixes its components under the full Lorentz transformation. But if we restrict ourselves to SO(3) transformations we see it indeed acts like a scalar. Though not stated explicitly in the literature, we may note that any first order tensor such as $h_{\mu\nu}$ or $\delta W_{\mu\nu}$ is invariant under infinitesimal coordinate translations, with the Lie derivative vanishing. This perhaps suggests that infinitesimal transformations upon our first order tensors may be affected via the Lorentz subgroup of the full Poincare group. Given that the underlying Lie algebra of the Lorentz group is $\mathfrak {su}(2)\times\mathfrak {su}(2)$, we know that a Lorentz four vector such as $A_{\mu}$ may be represented as a four component spinor of $(j_1=1/2,j_2 = 1/2)$. Addition of angular momentum then permits us to represent this as
\begin{equation}
\left( \frac12,\frac 12\right ) = 0 \oplus 1,
\end{equation}
i.e. a spin 0 scalar and a 3 component spin 1 object with helicity eigenvalues of $J_z$ of $0, \pm1$. If we perform a 3+1 splitting on such a vector such as $A_{\mu}$, followed by a decomposition into tranverse and orthogonal components, we will see that this suggets we are carrying out an irreducible decomposition under SO(3). As for the tensors, we note that their vector space may be formed as the direct product of two Lorentz four vectors (though not every vector space may be formulated as a direct product, here it may be plausible, as we can decompose our metric into two vierbeins that behave as Lorentz four vectors locally). Schematically, we have
\begin{equation}
\left( \frac12,\frac 12\right )\otimes \left( \frac12,\frac 12\right ) = (0,0)\oplus (0,1)\oplus(1,0) \oplus(1,1).
\end{equation}
The $(0,0)$ represents the spin 0 scalar, $(0,1)$ the spin 1 vector, and $(1,1)$ as the traceless symmetric spin 2 tensor (I believe the addition of the trace would bring this to a total of $10$ components, sufficient to describe a non-traceless $h_{\mu\nu}$, but this still needs to be worked out more carefully). For the case of $K_{\mu\nu}$, we note that 
\begin{equation}
(0,0)\simeq K_{00},\qquad (0,1) \simeq K_{0i},\qquad (1,1)\simeq K_{ij}.
\end{equation}
Therefore the procedure of the 3+1 splitting, followed by the SVT decomposition has a relation to a decomposition into irreducible SO(3) representations. After the SVT decomposition, if the equations are represented in terms of a helicity basis (or in other words the sphericial basis, i.e. $\mathbf e_\pm = -\frac{i}{\sqrt 2}(\mathbf e_1 \pm i \mathbf e_2), \mathbf e_0 = \mathbf e_3$), then the equations of motion decomple into modes with a given helicity. 
\\ \\
\subsection{Gauge Invariance}
Once our equations were decomposed via SVT, we noted that the gauge invariance was immediately manifest for an object like $\delta W_{\mu\nu}$ for example. If we perform an infinitesimal coordinate transformation $x^\mu \to x^\mu + \epsilon^\mu(x)$ for a general $\epsilon^{\mu}(x)$, the equations do not change their form. This implies that if we begin with a gauge invariant quantity, and then coordinate transform to coordinates where $\epsilon^{\mu}(x)$ satisfies a gauge criterion (such as $\nabla^\mu K_{\mu\nu} = 0$), then the equations will be unaffected. This means that if we choose to work in the tranverse gauge, it must be possible to use the gauge condition to re-express our equations of motion entirely in terms of gauge invariant quantities. The advantage of the gauge invariant formulation is that all the modes are physical - there does not exist a gauge which can eliminate any extra degrees of freedom. In Weinberg, we note that in finding plane wave solutions to $\Box h_{\mu\nu} =0$ in the tranverse gauge, we had to assess which modes were physical by finding those which were invariant under residual gauge transformations. This process is not necessary if we begin apriori with a gauge invariant formulation. Moreover, the irreducible decomposition greatly simplifies the equations of motion into more managable and natural subsets, where of which solutions are far more straightforward.
\subsection{Overview}
In the proceeding calculations, we first demonstrate for the simple Maxwell equations the equivalence between the transverse gauge and the gauge invariant SVT decomposition (and also show, for that matter that the SVT decomposition is well suited to electromagnetism). Then we show the equivalence between $\delta W_{\mu\nu}$ in the transverse gauge and the gauge invariant SVT decomposition. Lastly, we decompose our equations into their helicity components and find which spin  modes propogate as plane waves in source free regions. It remains to consider 1). Boundary Conditions, 2). the Einstein equations, and to 3). Recheck how I set longitudinal components to each other.
\section{$\nabla^\mu F_{\mu\nu}$ Decomposition}
\subsection{Gauge Invariant Formulation}
Take the Maxwell equations with source $J_{\mu}$
\begin{equation}
	\nabla^\nu F_{\mu\nu} = \nabla ^\nu\nabla_\nu A_\mu - \nabla^\nu \nabla_\mu A_\nu =  -J_\mu.
\end{equation}
Now decompose $J_{\mu}$, first via the 3+1 split, and then into its longitudinal and transverse components, 
\begin{equation}
J_{\mu} = (J^0, J_i^T+ \tilde\nabla_i J).
\end{equation}
This must be conserved as
\begin{equation}
 \nabla^\mu J_{\mu} = 0.
\end{equation}
Which means (in flat space)
\begin{align}
\dot J_0 &= \tilde\nabla^i J_i\\
\dot J_0&= \tilde\nabla_a \tilde\nabla^a J.
\end{align}
We may similarly decompose $A^\mu$ as
\begin{equation}
A_\mu = (A^0, A_i^T + \tilde\nabla_i A).
\end{equation}
For $\partial^\nu F_{\mu\nu}$ it follows
\begin{align}
J_\mu &= -\dot F_{\mu 0} + \tilde\nabla^i F_{\mu i}
\nonumber \\
&=  - \tilde\nabla_\mu \dot A_0 + \ddot A_\mu + \tilde\nabla^i \tilde\nabla_\mu A_i^T + \tilde\nabla_\mu \tilde\nabla_a\tilde\nabla^a A  - \tilde\nabla_a \tilde\nabla^a A_\mu
\nonumber \\
&= - \tilde\nabla_\mu \dot A_0 + \ddot A_\mu  + \tilde\nabla_\mu \tilde\nabla_a\tilde\nabla^a A  - \tilde\nabla_a \tilde\nabla^a A_\mu
\end{align}
For $\mu = 0$ we have
\begin{equation}
J_0 = \tilde\nabla_a \tilde\nabla^a\left( \dot A - A_0\right ).
\end{equation}
For $\mu = i$ it follows that
\begin{align}
J_i &= -\tilde\nabla_i \dot A_0 + \ddot A_i^T + \tilde\nabla_i \ddot A + \tilde\nabla_i \tilde\nabla_a\tilde\nabla^a A - \tilde\nabla_a\tilde\nabla^a A_i^T - \tilde\nabla_i \tilde\nabla_a\tilde\nabla^a A
\nonumber\\
J_i^T + \tilde\nabla_i J &= \tilde\nabla_i \left( -\dot A_0 + \ddot A\right) + \left(\partial_0^2 - \tilde\nabla_a\tilde\nabla^a\right)A_i^T.
\end{align}
It follows that
\begin{equation}
J = \ddot A - A_0 =\int d^3x \ D(\mathbf x - \mathbf y) \tilde\nabla_a^y \tilde\nabla^a_y \dot J_0,
\end{equation}
and 
\begin{equation}
J_i^T =  \left(\partial_0^2 - \tilde\nabla_a\tilde\nabla^a\right)A_i^T.
\end{equation}
Setting $J_\mu = 0$, this leaves us with the two equations
\begin{equation}
\boxed{
\tilde\nabla_a \tilde\nabla^a \left(\dot A - A_0\right) = 0,\qquad   \left(\partial_0^2 - \tilde\nabla_a\tilde\nabla^a\right)A_i^T=0.}
\end{equation}
With the allowed gauge transformation being of the form
\begin{equation}
A_{\mu} \to A_{\mu} + \nabla_\mu \chi,
\end{equation}
we decompose it as
\begin{equation}
A_0 \to A_0 + \dot \chi,\qquad A^T_i \to A^T_i,\qquad A \to A + \chi.
\end{equation}
Hence the combination $\dot A - A_0$ we found is in fact gauge invariant. Thus we have 3 physical components. 
\subsection{Tranvserse Gauge}
To reconcile this with setting a gauge explicitly, we calculate the decomposed EM equation of motion according to the condition
\begin{equation}
	\nabla^\mu A_\mu = 0.
\end{equation}
The above decomposes just like $\nabla^\mu J_{\mu}$, viz.
\begin{equation}
\dot A_0 = \tilde\nabla_a \tilde\nabla^a A.
\end{equation}
The equation of motion in this gauge is 
\begin{equation}
 \left(\partial_0^2 - \tilde\nabla_a\tilde\nabla^a\right)A_\mu=J_{\mu}
\end{equation}
which decomposes as
\begin{equation}
 \left(\partial_0^2 - \tilde\nabla_a\tilde\nabla^a\right)A_0=J_{0}
\end{equation}
and
\begin{equation}
 \left(\partial_0^2 - \tilde\nabla_a\tilde\nabla^a\right)A_i^T +  \tilde\nabla_i\left(\partial_0^2 - \tilde\nabla_a\tilde\nabla^a\right)A=J_i.
\end{equation}
Substituting the gauge condition $\ddot A_0 = \tilde\nabla_a\tilde\nabla^a \dot A$ into $J_0$, we recover the gauge invariant scalar equation
\begin{equation}
\tilde\nabla_a \tilde\nabla^a \left(\dot A - A_0\right) = J_0.
\end{equation}
We also see that we recover the gauge invariant transverse equation if we decompose the source as $J_i = J_i^T + \tilde\nabla_i J$. Hence, in this simple case we have used the gauge condition to reexpress the equations of motion in a gauge invariant manner, showing equivalence to the "gauge-free" SVT decomposition.


\section{$W_{\mu\nu}$ Decomposition}
\subsection{Gauge Invariant $\delta W_{\mu\nu} = \delta T_{\mu\nu}$}
Via the 3+1 projection followed by a helicity decomposition, we may express an arbitrary traceless, transverse, symmetric rank 2 tensor as
\begin{align}
\delta T_{00}  &= \rho,
\nonumber\\	
\delta T_{0i} &= -Q_i  + \tilde\nabla_i  \int d^3y D^3(\mathbf x-\mathbf y) \partial_t  \rho,
\nonumber\\	
\delta T_{ij}  &= 
\frac12 \delta_{ij} \rho - \frac12 \delta_{ij} \int d^3y D^3(\mathbf x-\mathbf y) \partial_t^2 \rho +\frac32 \tilde\nabla_i\tilde\nabla_j \int d^3y D^3(\mathbf x-\mathbf y) \bigg( \int d^3z D^3(\mathbf y-\mathbf z) \partial_t^2 \rho - \frac13\rho\bigg) 
\nonumber\\
&\quad -\tilde\nabla_i \int d^3y D^3(\mathbf x - \mathbf y) \partial_0 Q_j - \tilde\nabla_j \int d^3y D^3(\mathbf x - \mathbf y) \partial_0 Q_i + \pi_{ij}^{T\theta}.
\end{align}
We may equivalently express $\delta W_{\mu\nu}$ in terms of the analogous barred perturbation quantities ($\bar \rho$, $\bar Q_i$, $\bar E_{ij}$) as
\begin{align}
\delta W_{00}  &= \bar\rho,
\nonumber\\	
\delta W_{0i} &= -\bar Q_i  + \tilde\nabla_i  \int d^3y D^3(\mathbf x-\mathbf y) \partial_t  \bar\rho,
\nonumber\\	
\delta W_{ij}  &= 
\frac12 \delta_{ij} \bar\rho - \frac12 \delta_{ij} \int d^3y D^3(\mathbf x-\mathbf y) \partial_t^2 \bar\rho +\frac32 \tilde\nabla_i\tilde\nabla_j \int d^3y D^3(\mathbf x-\mathbf y) \bigg( \int d^3z D^3(\mathbf y-\mathbf z) \partial_t^2 \bar\rho - \frac13\bar\rho\bigg) 
\nonumber\\
&\quad -\tilde\nabla_i \int d^3y D^3(\mathbf x - \mathbf y) \partial_0  \bar Q_j - \tilde\nabla_j \int d^3y D^3(\mathbf x - \mathbf y) \partial_0 \bar Q_i + \bar \pi_{ij}^{T\theta}.
\end{align}
Then, the fluctuation equation $\delta W_{\mu\nu} = \delta T_{\mu\nu}$ then entails
\begin{align}
\bar \rho &= \rho
\nonumber\\
\bar Q_i &= Q_i
\nonumber\\
\bar \pi_{ij}^{T\theta} &= \pi_{ij}^{T\theta}.
\end{align}
The $\delta W_{00} = \delta T_{00}$ fixes $\rho$, allowing $\delta W_{0i} = \delta T_{0i}$ to fix $Q_i$, thereby leading to $\bar\pi_{ij}^{T\theta} = \pi_{ij}^{T\theta}$ without having to apply transverse projections or deal with additional homogenous solutions such as $\tilde\nabla_i\tilde\nabla_j \tilde\nabla_a\tilde\nabla^a \chi = 0$. This is also why the fluctuations equations have been expressed in terms of $Q_i$ rather than $\pi_i$, as the equation of $\pi_i$ necessarily leads to 
\begin{equation}
\tilde\nabla_a\tilde\nabla^a \bar \pi_i = \tilde\nabla_a \tilde\nabla^a \pi_i,
\end{equation}
which only permits equivalence of $\bar\pi_i = \pi_i$ under assumptions upon the boundary conditions of the perturbations (see A.1). 
\\ \\
Upon carrying through the same analogous helicity decomposition on $K_{\mu\nu}$, we find that the helicity components of $\delta W_{\mu\nu}$ take the form
\begin{align}
\bar \rho &= -\frac{2}{3} \tilde{\nabla}_a\tilde{\nabla}^a\tilde{\nabla}_b\tilde{\nabla}^b (\phi + \psi +\partial_0{B}-\partial_0^2{E}) 
\nonumber\\
\bar Q_i &= -\frac{1}{2} \tilde{\nabla}_a\tilde{\nabla}^a\left(-\partial_0^2+\tilde{\nabla}_b\tilde{\nabla}^b\right)(B_i - \partial_0{E}_i)
\nonumber \\
\bar \pi_{ij}^{T\theta} &= \left(-\partial_0^2 + \tilde\nabla_a\tilde\nabla^a\right)^2 E_{ij}.
\end{align}
\subsection{Transverse Gauge}
Here we will analyze the equations for $\delta W_{\mu\nu}$ within the transverse gauge, now with respect to the helicity decomposition. Results are calculated within the Minkowski background $g_{\mu\nu}^{(0)} = \eta_{\mu\nu}$. The traceless $K_{\mu\nu}$ is given as
\begin{equation}
K_{\mu\nu} = h_{\mu\nu} - \frac14 \eta_{\mu\nu} h,
\end{equation}
where 
\begin{equation}
h = -h_{00}+ \delta^{ij}h_{ij} = 2\phi - 6\psi + 2\tilde\nabla_a \tilde\nabla^a E.
\end{equation}
Imposing the transverse gauge 
\begin{equation}
\partial^\nu K_{\mu\nu} = 0
\end{equation}
leads to the simplified fluctuation equation
\begin{equation}
\delta W_{\mu\nu} = \frac12 \left( - \partial_0^2 +  \tilde\nabla_a\tilde\nabla^a\right)^2 K_{\mu\nu} .
\end{equation}
Evaluated in terms of the helicity components, we have
\begin{align}
\delta W_{00}&{}=\frac12\left( - \partial_0^2 +  \tilde\nabla_a\tilde\nabla^a\right)^2 \left[ -\frac32 \phi - \frac32 \psi + \frac12  \tilde\nabla_b\tilde\nabla^b  E\right]
\nonumber\\
\delta W_{0i}&{} = \frac12\left( - \partial_0^2 +  \tilde\nabla_a\tilde\nabla^a\right)^2 \left[ \tilde\nabla_i B + B_i\right]
\nonumber\\
\delta W_{ij}&{} = \frac12 \left( - \partial_0^2 +  \tilde\nabla_a\tilde\nabla^a\right)^2 \left[ \delta_{ij}\left( - \frac12 \phi - \frac12 \psi -\frac12  \tilde\nabla_b\tilde\nabla^b E \right)
+ 2 \tilde\nabla_i \tilde\nabla_j E + \tilde\nabla_i E_j + \tilde\nabla_j E_i + 2E_{ij}\right].
\end{align}
Inspection of the transverse condition yields the four conditions
\begin{equation}
\partial^0K_{00} + \tilde\nabla^i K_{0i}=0,\qquad \partial^0K_{0i} + \tilde\nabla^j K_{ij} = 0.
\end{equation}
The first condition evaluates to (noting $\partial^0 K_{00} = -\dot K_{00}$),
\begin{align}
0=&{} 2\dot\phi - \frac14 \dot h +  \tilde\nabla_a\tilde\nabla^a B
\nonumber\\
=&\frac32 \dot\phi + \frac32 \dot \psi +  \tilde\nabla_a\tilde\nabla^a B - \frac12  \tilde\nabla_a\tilde\nabla^a \dot E
\end{align}
The remaining spatial transverse condition takes the form
\begin{align}
0 =&{}- \dot B_i - \tilde\nabla_i \dot B -2\tilde\nabla_i \psi + 2\tilde\nabla_i  \tilde\nabla_a\tilde\nabla^a E +  \tilde\nabla_a\tilde\nabla^a E_i - \frac14 \tilde\nabla_i h
\nonumber\\
=& \tilde\nabla_i\left( -\frac12 \phi - \frac12 \psi - \dot B + \frac32  \tilde\nabla_a\tilde\nabla^a E\right) - \dot B_i +  \tilde\nabla_a\tilde\nabla^a E_i.
\end{align}
Let us denote the two simplified scalar conditions as
\begin{equation}
 S_1 \equiv \dot\phi + \dot\psi + \frac23 \tilde\nabla_a\tilde\nabla^a B - \frac13 \tilde\nabla_a\tilde\nabla^a \dot E =0,
\qquad
S_2 \equiv \tilde\nabla_a\tilde\nabla^a \left( \phi + \psi + 2\dot B - 3\tilde\nabla_b\tilde\nabla^b E\right) =0.
\end{equation}
We are free to form combinations of $S_1$ and $S_2$ that yield quantities that are gauge invariant. Such a gauge invariant quantity will be equivalent to that found from the usual "gauge-free" S.V.T. decomposition. To show this, take the explicit relation:
\begin{equation}
0=\frac98 \partial_0^3 S_1 - \frac{15}{8} \tilde\nabla_a\tilde\nabla^a\partial_0 S_1+\frac{1}{8} \tilde\nabla_a\tilde\nabla^a S_2    -\frac38 \partial_0^2 S_2 .
\end{equation}
Substitution of $S_1$ and $S_2$ into the above yields
\begin{align}
0={}& \left(\frac98 \partial_0^4 \phi - \frac94  \tilde\nabla_a\tilde\nabla^a \partial_0^2 \phi + \frac18  \tilde\nabla_a\tilde\nabla^a \tilde\nabla_b\tilde\nabla^b \phi\right) + \left(\frac98 \partial_0^4 \psi - \frac94  \tilde\nabla_a\tilde\nabla^a \partial_0^2 \psi + \frac18  \tilde\nabla_a\tilde\nabla^a \tilde\nabla_b\tilde\nabla^b \psi\right)
\nonumber\\
& - \tilde\nabla_a\tilde\nabla^a \tilde\nabla_b\tilde\nabla^b \partial_0 B 
+\left( -\frac38 \tilde\nabla_a\tilde\nabla^a \partial_0^4 E - \frac38 \tilde\nabla_a\tilde\nabla^a \tilde\nabla_b\tilde\nabla^b\tilde\nabla_c\tilde\nabla^c E + \frac74
\tilde\nabla_a\tilde\nabla^a \tilde\nabla_b\tilde\nabla^b\partial_0^2 E\right)
\nonumber\\
={}&\left(-\partial_0^2+\tilde\nabla_a\tilde\nabla^a\right)^2
\left[ \frac98 \phi + \frac98\psi  -\frac38 \tilde\nabla_b\tilde\nabla^b E \right]
 -\tilde\nabla_a\tilde\nabla^a \tilde\nabla_b\tilde\nabla^b\left( \phi + \psi +\partial_0 B - \partial_0^2 E\right).
\end{align}
Hence we arrive at
\begin{equation}
\boxed{
\frac12 \left(-\partial_0^2+\tilde\nabla_a\tilde\nabla^a\right)^2
\left[ -\frac32 \phi - \frac32\psi  +\frac12 \tilde\nabla_b\tilde\nabla^b E \right]=-\frac{2}{3}\tilde\nabla_a\tilde\nabla^a \tilde\nabla_b\tilde\nabla^b\left( \phi + \psi +\partial_0 B - \partial_0^2 E\right)}
\end{equation}
For the vector component, we again look at the spatial piece of the transverse gauge condition
\begin{equation}
V_i \equiv  \tilde\nabla_i\left( -\frac12 \phi - \frac12 \psi - \dot B + \frac32  \tilde\nabla_a\tilde\nabla^a E\right) - \dot B_i +  \tilde\nabla_a\tilde\nabla^a E_i = 0.
\end{equation}
The longitudinal component of $V_i$ is defined as $\tilde\nabla_i V$, where
\begin{equation}
V = \int d^3y\ D^{(3)}(\mathbf x - \mathbf y)\tilde\nabla_y^i V_i = -\frac12 \phi - \frac12 \psi - \dot B + \frac32  \tilde\nabla_a\tilde\nabla^a E.
\end{equation}
In the above we assumed that (see A.1)
\begin{align}
0=&{}\int d^3y \tilde\nabla_i^y \tilde\nabla^i_y \left[ D^{(3)}(\mathbf x - \mathbf y) \left( -\frac12 \phi - \frac12 \psi - \dot B + \frac32  \tilde\nabla_a\tilde\nabla^a E\right)\right] 
\nonumber\\
=&{} \int dS_i \tilde\nabla^i_y \left[ D^{(3)}(\mathbf x - \mathbf y) \left( -\frac12 \phi - \frac12 \psi - \dot B + \frac32  \tilde\nabla_a\tilde\nabla^a E\right) \right]. 
\end{align}
Since $V_i$ is to be identically zero, it follows from the definition of $V$ that that $V$ itself should also vanish. This leads to a gauge condition on the tranverse vectors of the form
\begin{equation}
\dot B_i = \tilde\nabla_a \tilde\nabla^a E_i.
\end{equation}
With this gauge condition in hand, we look at the tranverse component of $\delta W_{0i}$,
\begin{equation}
\delta W_{0i}^T= \frac12\left( - \partial_0^2 +  \tilde\nabla_a\tilde\nabla^a\right)^2 B_i = \frac{1}{2}\left( \partial_0^4 - \tilde\nabla_a\tilde\nabla^a \partial_0^2
+ \tilde\nabla_a\tilde\nabla^a\tilde\nabla_b\tilde\nabla^b\right) B_i.
\end{equation}
Substitution of the vector gauge condition $\ddot B_i =\tilde\nabla_a \tilde\nabla^a\dot E_i$ then yields
\begin{equation}
\boxed{
\bar Q_i=-\frac{1}{2} \tilde{\nabla}_a\tilde{\nabla}^a\left(-\partial_0^2+\tilde{\nabla}_b\tilde{\nabla}^b\right)(B_i - \partial_0{E}_i)}
\end{equation}
Lastly, we see that the tensor mode already obeys the appropriate gauge invariant SVT equations, with
\begin{equation}
\boxed{
\bar\pi_{ij} =  \left(-\partial_0^2 + \tilde\nabla_a\tilde\nabla^a\right)^2 E_{ij}}.
\end{equation}
Through use of the gauge conditions, we have brought the tranverse $\nabla^\mu K_{\mu\nu}$ into the equivalent gauge invariant form as from the SVT decomposition.
\section{Helicity Components}
Here we will decompose the spatial part of the dynamical fields according to the $e^{i\mathbf k \mathbf x}$ basis (Fourier transform). Such a basis representation assumes the inverse Fourier transform exists, which may be true given certain conditions on our functions (such as belonging to $L^2[-\infty,\infty]$ and $\lim_{\mathbf x\to\infty}f(\mathbf x,t) = 0$). We need to explicitly show whether or not it is reasonable to expect these conditions to hold physically, but for the proceeding calculations we assume our functions are well behaved enough. In the Fourier basis, we take the direction of spatial propogation along the $z$ axis, i.e. $\mathbf k = (0,0,k_3)$
\subsection{$\nabla^\mu F_{\mu\nu} = 0$}
In the gauge-invariant formulation, we decomposed $A_{\mu}$ as
\begin{equation}
A_\mu = (A_0, A_i^T + \tilde\nabla_i A).
\end{equation}
Fourier transforming, this becomes
\begin{equation}
\tilde A_{\mu}(\mathbf k,t) = 
\begin{pmatrix}
\tilde A_0 \\ \tilde A^T_1 \\ \tilde A^T_2 \\ -i \mathbf k \tilde A
\end{pmatrix},
\end{equation}
where the transverse condition $\tilde\nabla^i  A_i^T$ leads to $i \mathbf k \tilde A_3 = 0$. Now we apply a rotation matrix about the direction of propagation:
\begin{equation}
\begin{pmatrix}
1&0&0&0\\
0&\cos\theta&-\sin\theta&0\\
0&\sin\theta&\cos\theta&0\\
0&0&0&1
\end{pmatrix}
\begin{pmatrix}
\tilde A_0 \\ \tilde A^T_1 \\ \tilde A^T_2 \\ -i \mathbf k \tilde A
\end{pmatrix}
=
\begin{pmatrix}
\tilde A_0 \\ \cos\theta \tilde A^T_1 -\sin\theta \tilde A^T_2\\ \sin\theta \tilde A_1^T +\cos\theta \tilde A^T_2 \\ -i \mathbf k \tilde A
\end{pmatrix}.
\end{equation}
As expected, the scalars $\tilde A_0$ and $\tilde A$ transform as objects with helicity 0. However, the tranverse components transform in the helicity basis as
\begin{equation}
A_+ = \left(\tilde A_1^T + i\tilde A_2^T\right) \to e^{i\theta} A_+,\qquad A_- = \left(\tilde A_1^T - i\tilde A_2^T\right)\to e^{-i\theta} A_-,
\end{equation}
i.e. as objects with helicity $\pm1$. We recall that in the SVT decomposition, the equations of motion take the form
\begin{equation}
\tilde\nabla_a \tilde\nabla^a \left(\dot A - A_0\right) = 0,\qquad   \left(\partial_0^2 - \tilde\nabla_a\tilde\nabla^a\right)A_i^T=0.
\end{equation}
The scalar equation is Laplace's equation $\nabla^2\phi$ for the electric potential $\phi$ within a source free region. Whereas the solutions to the transverse equation consist of massless, spin 1, left and right handed circularly polarized photons propogating along the $k_\mu = (-k,0,0, k)$ direction
\begin{equation}
\boxed{
A_i^T = C\begin{pmatrix}1\\i\\0\end{pmatrix} e^{ikx} + C^*\begin{pmatrix}1\\-i\\0\end{pmatrix} e^{-ikx}}
\end{equation}
where $k_\mu k^\mu = 0$.
\\
\subsection{$\delta W_{\mu\nu} = 0$}
The source free equations of motion in the gauge-invariant SVT formulation are
\begin{align}
0=\bar \rho &= -\frac{2}{3} \tilde{\nabla}_a\tilde{\nabla}^a\tilde{\nabla}_b\tilde{\nabla}^b (\phi + \psi +\partial_0{B}-\partial_0^2{E}) 
\nonumber\\
0=\bar Q_i &= -\frac{1}{2} \tilde{\nabla}_a\tilde{\nabla}^a\left(-\partial_0^2+\tilde{\nabla}_b\tilde{\nabla}^b\right)(B_i - \partial_0{E}_i)
\nonumber \\
0=\bar \pi_{ij}^{T\theta} &= \left(-\partial_0^2 + \tilde\nabla_a\tilde\nabla^a\right)^2 E_{ij},
\end{align}
Note that $\bar \rho = \delta W_{00}$, which transforms as an SO(3) scalar, whereas $\bar Q_i  = \delta W_{0i}^T$ transforms as an SO(3) 3 vector, and lastly $\bar \pi_{ij}^{T\theta}$ transforms as an SO(3) tensor. It follows then that $\bar \rho$ transforms as a spin 0 object with helicity 0, an object which follows the fourth order source free Laplace equation
\begin{equation}
\boxed{
 -\frac{2}{3} \tilde{\nabla}_a\tilde{\nabla}^a\tilde{\nabla}_b\tilde{\nabla}^b (\phi + \psi +\dot{B}-\ddot{E}) =0}.
\end{equation}
As for the tranverse $\bar Q_i$, we note that application of our rotation matrix to $\delta W_{0i}^T$ proceeds in the same manner as the $A_i^T$ vector for the source free Maxwell equation. Thus, we will have (omitting the overbars)
\begin{equation}
 Q_+ = \left(\tilde Q_1^T + i\tilde Q_2^T\right) \to e^{i\theta}  Q_+,\qquad  Q_- = \left(\tilde Q_1^T - i\tilde Q_2^T\right)\to e^{-i\theta}  Q_-.
\end{equation}
The spin 1, helicity $\pm 1$ vector components admit plane wave solutions of the form
\begin{equation}
\boxed{\left(B_i - \dot E_i\right) = C\begin{pmatrix}1\\i\\0\end{pmatrix} e^{ikx} + C^*\begin{pmatrix}1\\-i\\0\end{pmatrix} e^{-ikx}},
\end{equation} 
again with $k_\mu = (-k,0,0,k)$, $k_\mu k^\mu = 0$. While plane waves do satisfy the equation of motion, the presence of the extra $\tilde\nabla_a\tilde\nabla^a$ term  will yield more general solutions. 
\\ \\
For the tensor component, the tranverse condition yields $\tilde \pi_{3i} = 0$. A rotation along the $z$ axis is effectively applied as $R_{i}{}^{k}\pi_{kl} R^{l}{}_{j} = \pi_{ij}'$, 
\begin{equation}
\begin{pmatrix}
\cos\theta&-\sin\theta\\
\sin\theta&\cos\theta\\
\end{pmatrix}
\begin{pmatrix}
\tilde\pi_{11}& \tilde\pi_{12}\\
\tilde\pi_{12}&-\tilde\pi_{11}\\
\end{pmatrix}
\begin{pmatrix}
\cos\theta&\sin\theta\\
-\sin\theta&\cos\theta\\
\end{pmatrix}
= 
\begin{pmatrix}
\tilde\pi_{11}\cos(2\theta) - \tilde\pi_{12}\sin(2\theta) & \tilde\pi_{11}\sin(2\theta) + \tilde\pi_{12}\cos(2\theta)
\\
\tilde\pi_{11}\sin(2\theta) + \tilde\pi_{12}\cos(2\theta) & - \tilde\pi_{11}\cos(2\theta) + \tilde\pi_{12}\sin(2\theta).
\end{pmatrix}
\end{equation}
The transformations are
\begin{equation}
\tilde\pi_{11}' = \tilde\pi_{11}\cos(2\theta) - \tilde\pi_{12}\sin(2\theta),\qquad \tilde\pi_{12}' =  \tilde\pi_{11}\sin(2\theta) + \tilde\pi_{12}\cos(2\theta).
\end{equation}
In the helicity basis it follows that
\begin{equation}
\pi_+ = \tilde\pi_{11} + i \tilde\pi_{12} \to e^{i2\theta}\pi_+,\qquad 
\pi_- = \tilde\pi_{11} - i \tilde\pi_{12} \to e^{-i2\theta}\pi_-.
\end{equation}
This transformation, along with the equation of motion
\begin{equation}
\left(-\partial_0^2 + \tilde\nabla_a\tilde\nabla^a\right)^2 E_{ij}=0,
\end{equation}
indicate that the tranverse $E_{ij}$ represent massless spin 2, helicity $\pm 2$ waves. The solution to the $\Box^2$ wave equation for a given $k$ is 
\begin{equation}
\boxed{
E_{ij} = C\begin{pmatrix} 1&i\\i&-1\end{pmatrix} e^{ikx} + C\begin{pmatrix} 1&i\\i&-1\end{pmatrix}  n_\alpha x^\alpha e^{ikx}
+C^*\begin{pmatrix} 1&-i\\-i&-1\end{pmatrix} e^{-ikx} + C^*\begin{pmatrix} 1&-i\\-i&-1\end{pmatrix}  n_\alpha x^\alpha e^{-ikx}},
\end{equation}
where $n_{\alpha} = (1,0,0,0)$ and $k_\mu k^\mu  =0$.
\newpage
\appendix
\section{Appendix}
\subsection{Boundary Conditions}
Under infinitesimal coordinate transformation $x^\mu \to \bar x^\mu = x^\mu + \epsilon^\mu(x)$
where
\[
	\epsilon^0 = T,\qquad \epsilon^i = \tilde\nabla^i L + L^i,\qquad \tilde\nabla^i L_i = 0,
\]
it follows that $h_{0i}$ transforms as 
\begin{align}
 \bar h_{0i} &=  h_{0i} -  (\tilde\nabla_i \dot L + L_i) +  \partial_i T
\end{align}
which evaluates to
\begin{equation}
	\tilde \nabla_i \bar B + \bar B_i = \tilde\nabla_i B + B_i - \tilde\nabla_i \dot L - \dot L_i + \tilde\nabla_i T.
\end{equation}
or
\begin{equation}
\tilde\nabla_i \bar B + \bar B_i = \tilde\nabla_i(B - \dot L + T) + B_i.
\end{equation}
Since an arbitrary gradient of a scalar such as $\tilde\nabla_i T$ could in fact be transverse, we cannot immediately separate scalars to scalars and vectors to vectors. If we take the divergence, we arrive at
\begin{equation}
\tilde\nabla_a \tilde\nabla^a \bar B = \tilde\nabla_a \tilde\nabla^a (B-\dot L + T),
\end{equation}
in which we may define $\bar B$ as
\begin{align}
\bar B&= \int d^3y\ D^3(\mathbf x - \mathbf y)\tilde\nabla_a^y \tilde\nabla^a_y(B-\dot L + T)
\nonumber\\
&= \int d^3y\  \tilde\nabla_a^y \tilde\nabla^a_y\left[ D^3(\mathbf x - \mathbf y)(B-\dot L + T)\right] - \int d^3y \  \tilde\nabla_a^y \tilde\nabla^a_y D^3(\mathbf x - \mathbf y)(B-\dot L + T)
\nonumber\\
&= B-\dot L + T + \int dS_a\  \tilde\nabla^a_y\left[ D^3(\mathbf x - \mathbf y)(B-\dot L + T)\right]
\nonumber\\
&= B - \dot L + T + \chi.
\end{align}
The surface term takes the form
\begin{align}
\chi &=  \int dS_a\   \tilde\nabla^a_yD^3(\mathbf x - \mathbf y)(B-\dot L + T) + \int dS_a\  D^3(\mathbf x - \mathbf y)\tilde\nabla^a_y(B-\dot L + T).
\end{align}
The discussion in Jackson Electrodynamics pg. 39 suggests that a given Green's function $D(\mathbf x, \mathbf y)$, may be defined up to an arbitrary function 
$F(\mathbf x, \mathbf y)$ which satisfies $\nabla^2 F(\mathbf x, \mathbf y) = 0$. It is then suggested that the freedom in $F(\mathbf x, \mathbf y)$ may be used to formulate the solution for $\bar B$ in terms of either Dirichlet or Neumann boundary conditions by finding an $F(\mathbf x, \mathbf y)$ such that
\begin{equation}
D(\mathbf x, \mathbf y) = 0\quad\text{for}\quad \mathbf x\ \text{on}\ S,\qquad \text{or}\qquad \tilde\nabla_a D(\mathbf x, \mathbf y) = 0\quad\text{for}\quad \mathbf x\ \text{on}\ S.
\end{equation}
Let us assume we were able to find an $F(\mathbf x,\mathbf y)$ that allows for Dirichlet boundary conditions, i.e.
\begin{equation}
D(\mathbf x, \mathbf y) = 0\quad\text{for}\quad \mathbf x\ \text{on}\ S,
\end{equation}
then in order to arrive at the desired equation of
\begin{equation}
\bar B = B - \dot L + T
\end{equation}
we must require that 
\begin{equation}
B - \dot L + T = 0\quad\text{for}\quad \mathbf x\ \text{on}\ S,
\end{equation}
with $S$ being the asymptotic boundary surface at infinity. Imposing such a boundary condition would seem to allow better constraints when expanding the perturbation functions in momentum space viz.
\begin{equation}
B(t,x) = \int d^3k\ e^{ikx} \tilde B(t,k).
\end{equation}
For example, an equation such as
\begin{equation}
\tilde\nabla_a \tilde\nabla^a (B-E) = 0,
\end{equation}
leads to
\begin{equation}
\int d^3k\ e^{ikx} k^2 [-\tilde B(t,k)+\tilde E(t,k)] = 0.
\end{equation}
Without boundary conditions, either $\tilde B(t,k) = \tilde E(t,k)$ or $\tilde B(t,k)=\tilde E(t,k)+\delta(k)$ (or perhaps $k^n \delta(k)$ for $n>-2$). However, the requirement that $B(t,x)$ and $E(t,x)$ vanish at spatial infinity excludes the possible $\delta(k)$ solutions and thus yields $\tilde B(t,k) = \tilde E(t,k)$ and consequently $B(t,x) = E(t,x)$.
\\ \\
As an aside, we take the Laplacian of the boundary term $\chi$, which evaluates to
\begin{align}
\tilde\nabla_b^x\tilde\nabla^b_x \chi &=  \int dS_a\   \tilde\nabla^a_y \delta^3(\mathbf x - \mathbf y)(B-\dot L + T) + \int dS_a\  \delta^3(\mathbf x - \mathbf y)\tilde\nabla^a_y(B-\dot L + T)
\nonumber \\
&= -\tilde\nabla^a_x \int dS_a\ \delta^3(\mathbf x- \mathbf y)(B-\dot L + T)+ \int dS_a\  \delta^3(\mathbf x - \mathbf y)\tilde\nabla^a_y(B-\dot L + T)
\end{align}
The quantity $\nabla^2 \chi$ is only supported asymptotically, but even if $\mathbf x$ is evaluated at a point on the infinite surface, the two surface terms will mutually cancel. Therfore, for all $\mathbf x$ such a $\chi$ obeys 
\begin{equation}
\tilde\nabla_a\tilde\nabla^a \chi = 0.
\end{equation}
\textbf{Still need to consider freedom up to arbitrary functions of time $f(t)$ and boundary condition at $t=\infty$}.
\end{document}