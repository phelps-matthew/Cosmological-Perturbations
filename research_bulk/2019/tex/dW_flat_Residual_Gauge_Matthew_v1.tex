\documentclass[10pt,letterpaper]{article}
\usepackage[textwidth=7in, top=1in,textheight=9in]{geometry}
\usepackage[fleqn]{mathtools} 
\usepackage{amssymb}
\usepackage{hyperref}
\title{$\delta W_{\mu\nu}$ Residual Gauge v2}
\author{}
\date{}

\begin{document}
\maketitle
In the transverse gauge $\partial_\nu K^{\mu\nu} = 0$ in the Minkowski background the vacuum equation of motion for the traceless $K_{\mu\nu}$ is
\begin{equation}
\delta W_{\mu\nu} = \eta^{\alpha\beta} \eta^{\sigma\rho}\partial_\alpha\partial_\beta\partial_\sigma\partial_\rho K_{\mu\nu} =0.
\end{equation}
The momentum eigenstate solutions take the form 
\begin{equation}
K_{\mu\nu} = A_{\mu\nu}e^{ikx} + n_\alpha x^\alpha B_{\mu\nu} e^{ikx} +\text{c.c.}\label{202}
\end{equation}
where $n_\alpha = (1,0,0,0)$ and $k^\mu k_{\mu} = 0$. Following the transverse condition, the solution must obey
\begin{equation}
0=\left(ik^\nu A_{\mu\nu} + n^\nu B_{\mu\nu}\right)e^{ikx}
+ \left( ik^\nu B_{\mu\nu}\right) n_\alpha x^\alpha  e^{ikx}+ \text{c.c.}\label{203}
\end{equation}
In addition to the tracelessness condition, to satisfy all $x$ (noting that $e^{ikx}$, $e^{-ikx}$, $te^{ikx}$ and $te^{-ikx}$ are linearly independent), we set in \eqref{203} each coefficient preceding the space-time dependent function to zero, viz.
\begin{equation}
A^\mu{}_\mu = 0,\qquad B^\mu{}_\mu=0,\qquad ik^\nu A_{\mu\nu} + n^\nu B_{\mu\nu}= 0,\qquad i k^\nu B_{\mu\nu} = 0.
\end{equation}
We have a total of $10$ conditions upon the 20 total components of $A_{\mu\nu}$ and $B_{\mu\nu}$. 
It is easy to check that these conditions (and also their implied conjugate expressions) satisfy our choice of transverse coordinate system and retain the tracelessness of $K_{\mu\nu}$. 
Under infinitesimal coordinate transformation $x^\mu \to x^\mu + \epsilon^\mu(x)$, $K_{\mu\nu}$ transforms as
\begin{equation}
	K_{\mu\nu}' = K_{\mu\nu} - \partial_\mu \epsilon_\nu - \partial_\nu\epsilon_\mu + \frac12 g_{\mu\nu} \partial_\rho \epsilon^\rho.
\end{equation}
We denote the change in $K_{\mu\nu}$ (Lie derivative) as the tensor
\begin{equation}
\Delta K_{\mu\nu} = - \partial_\mu \epsilon_\nu - \partial_\nu\epsilon_\mu + \frac12 g_{\mu\nu} \partial_\rho \epsilon^\rho.\label{207}
\end{equation}
Noting that $\Delta K_{\mu\nu}$ is manifestly traceless, in order to preserve the tranverse gauge condition $\partial_\mu K^{\mu\nu} = 0$,  $\Delta K^{\mu\nu}$ must obey $\partial_\nu \Delta K^{\mu\nu}=0$, viz.
\begin{equation}
	0 =
	-\partial_\nu \partial^\nu \epsilon^\mu - \frac12 \partial^\mu \partial_\nu \epsilon^\nu .\label{208}
\end{equation}
We take the $\epsilon^\mu(x)$ to be of the plane wave form,
\begin{equation}
\epsilon^\mu(x) = i A^\mu e^{ikx} + iB^\mu n_\alpha x^\alpha e^{ikx} + \text{c.c.},
\end{equation}
which obeys the following relations:
\begin{equation}
\partial^\nu \epsilon^\mu = - k^\nu\left(A^\mu e^{ikx} + B^\mu n_\alpha x^\alpha e^{ikx}\right)+ 
i n^\nu \left(  B^\mu  e^{ikx} \right)+ \text{c.c.}
\end{equation}
\begin{equation}
\partial_\nu \partial^\nu \epsilon^\mu = -2k_\nu n^\nu \left(B^\mu  e^{ikx}\right)+\text{c.c.},
\end{equation}
\begin{equation}
\partial_\mu \partial^\nu \epsilon^\mu = -i k_\mu k^\nu\left(A^\mu e^{ikx} +  B^\mu n_\alpha x^\alpha e^{ikx}\right)
- (k^\nu n_\mu+k_\mu n^\nu)\left[ B^\mu e^{ikx}\right] + \text{c.c.},
\end{equation}
where for reference we also have the relation
\begin{equation}
\partial_\beta \partial^\beta (n_\alpha x^\alpha e^{ikx}) = 2i n_\alpha k^\alpha e^{ikx}.
\end{equation}
The transverse condition per \eqref{208} then takes the form
\begin{align}
0 {}&= 2k_\nu n^\nu \left(B^\mu  e^{ikx}\right)+\frac12 i k_\nu k^\mu\left(A^\nu e^{ikx} +  B^\nu n_\alpha x^\alpha e^{ikx}\right)
+\frac12 (k^\mu n_\nu+k_\nu n^\mu)\left[ B^\nu e^{ikx}\right]+ \text{c.c.}\ .
\end{align}
To hold for arbitrary $x$, we have the two separate conditions,
\begin{equation}
2k_\nu n^\nu B^\mu +\frac12 ik_\nu k^\mu A^\nu + \frac12 (k^\mu n_\nu+k_\nu n^\mu)B^\nu=0,\qquad \frac12 ik_\nu k^\mu B^\nu=0.\label{2019}
\end{equation}
For arbitrary $k^\mu$, the second condition in \ref{2019} implies $k_\nu B^\nu = 0$. As such, the remaining condition is
\begin{equation}
2k_\nu n^\nu B^\mu + \frac12 k^\mu n_\nu B^\nu + \frac12 i k_\nu k^\mu A^\nu = 0.
\end{equation}
Let us now take a wave propagating in the $z$ direction, with wavevector
\begin{equation}
k^\mu = (k,0,0,k),\qquad k_\mu = (-k,0,0,k).
\end{equation}
The transverse condition $\partial^\mu \Delta K_{\mu\nu}$ then entails
\begin{equation}
B_0 = -B_3,\qquad B_0 = \frac{i}{5}k(A_0+A_3),\qquad B_1 = B_2 = 0.
\end{equation}
We see that the specific form of $\epsilon^\mu(x)$ comprises of four independent components, here chosen as $B_0$, $A_0$, $A_1$, and $A_2$. The dependencies are:
\begin{equation}
B_{1} = B_{2} = 0,\qquad B_3 = -B_0,\qquad A_3 = -A_0 - \frac{5i}{k} B_0.
\end{equation}
For the tensor polarizations $A_{\mu\nu}$ and $B_{\mu\nu}$ the transverse relations take the form
\begin{equation}
B^\mu{}_\mu = A^\mu{}_\mu = 0,\qquad B_{0\mu} = -B_{3\mu},\qquad ik(A_{\mu 0}+ A_{\mu 3}) = B_{0\mu}.
\end{equation}
Although this would appear to be 10 total constraints, the condition $B_{00} = -B_{30}$ reduces the equation
\begin{equation}
ik(A_{\mu0}+A_{\mu 3}) = B_{0\mu},
\end{equation}
from 4 to 3 conditions, namely
\begin{equation}
ik(A_{10}+A_{13}) = B_{01},\qquad ik(A_{20}+A_{23}) = B_{02},\qquad A_{00} + 2A_{03} + A_{33} = 0.
\end{equation}
 We will take 11 the independent components as 
\begin{equation}
B_{00}, B_{01},B_{02},B_{11}, B_{12}, A_{00},A_{01},A_{02},A_{11},A_{22},A_{12}.
\end{equation}
In order to arrive at the following choice of independent components for $B_{\mu\nu}$, we utliize the gauge conditions which lead us to following dependencies:
\begin{equation}
B_{33} = -B_{03} = B_{00},\qquad B_{23} = -B_{02},\qquad B_{13} = -B_{01},\qquad B_{22} = -B_{11}.
\end{equation}
As for $A_{\mu\nu}$, the dependencies are:
\begin{equation}
A_{13} = -\frac{i}{k} B_{01} - A_{01},\quad
A_{23} = -\frac{i}{k}B_{02} - A_{02},\quad
A_{33} = A_{00} - A_{11} - A_{22},\quad
A_{03} = -A_{00} + \frac12 (A_{11}+A_{22}).
\end{equation}
The form for the transformation (Lie derivative) onto $K_{\mu\nu}$ is
\begin{align}
\Delta K_{\mu\nu} &= \left[ k_\nu A_\mu + k_\mu A_\nu - i \left( n_\nu B_\mu + n_\mu B_\nu\right) 
-\frac12 g_{\mu\nu} A^\alpha k_\alpha + \frac{i}{2} g_{\mu\nu}n_\alpha B^\alpha \right]e^{ikx}
\nonumber\\
&\qquad + \bigg[ k_\nu B_\mu + k_\mu B_\nu \bigg] n_\alpha x^\alpha e^{ikx}.
\end{align}
It will be useful to evaluate this for different components:
\begin{align}
\Delta K_{00} &=\left[ -2 kA_0 +\frac12 k(A_0 + A_3) - \frac{3i}{2} B_0\right]e^{ikx} - \bigg[2kB_0\bigg] n_\alpha x^\alpha e^{ikx}
\nonumber\\
\Delta K_{01} &=  -k A_1 e^{ikx},\qquad \Delta K_{02} =  -k A_2 e^{ikx}
\nonumber\\
\Delta K_{03} &= \left[ -kA_3 +kA_0 -i B_3\right]e^{ikx} - \left[2kB_3\right] n_\alpha x^\alpha e^{ikx}
\nonumber\\
\Delta K_{11} &= \Delta K_{22} =  \left[ - \frac12 k(A_0 + A_3) - \frac{i}{2}B_0 \right]  e^{ikx},\qquad \Delta K_{12} = 0
\nonumber\\
\Delta K_{13} &= [kA_1]e^{ikx},\qquad \Delta K_{23} = [kA_2]e^{ikx}
\nonumber\\
\Delta K_{33} &= \left[ 2kA_3 -\frac12 k(A_0+A_3) - \frac{i}{2}B_0\right]e^{ikx} + \bigg[2kB_3\bigg]n_\alpha x^\alpha e^{ikx}.
\end{align}
We express the gauge variables ($A_{\mu}$ and $B_{\mu}$) in terms of the 4 independent components, and compute the total transformation on each polarization tensor. For $A_{\mu\nu} \to A_{\mu\nu}'$ and $B_{\mu\nu} \to B'_{\mu\nu}$, we have
\begin{align}
A_{00}' &= A_{00} -2kA_0 - 4i B_0 			& B_{00}'&= B_{00} -2k B_0
\nonumber\\
A_{01}' &= A_{01} - kA_1							& B_{01}'&= B_{01} 
\nonumber\\
A_{02}' &= A_{02} - kA_2							&B_{02}'&= B_{02}
\nonumber\\
A_{03}' &= A_{03} +2kA_0 + 6i B_0			&B_{03}'&= B_{03} + 2kB_0
\nonumber \\
A_{11}' &= A_{11} + 2i B_0						&B_{11}'&= B_{11}
\nonumber\\
A_{22}' & = A_{22} +2iB_0						&B_{22}'&= B_{22}
\nonumber\\
A_{33}' &= A_{33}-2k A_0 - 8i B_0				&B_{33}'&= B_{33} - 2kB_0
\nonumber\\
A_{12}' &= A_{12}									&B_{12}'&=B_{12}
\nonumber\\
A_{13}' &= A_{13} + kA_1						&B_{13}'&=B_{13}
\nonumber\\
A_{23}' &= A_{23} + k A_2						&B_{23}'&=B_{23}.
\end{align}
If we filter the above such that we are only looking at the independent components, this becomes
\begin{align}
A_{00}' &= A_{00} -2kA_0 - 4i B_0 			& B_{00}'&= B_{00} -2k B_0
\nonumber\\
A_{01}' &= A_{01} - kA_1							& B_{01}'&= B_{01} 
\nonumber\\
A_{02}' &= A_{02} - kA_2							&B_{02}'&= B_{02}
\nonumber \\
A_{11}' &= A_{11} + 2i B_0						&B_{11}'&= B_{11}
\nonumber\\
A_{22}' & = A_{22} +2iB_0						&&
\nonumber\\
A_{12}' &= A_{12}									&B_{12}'&=B_{12}.
\end{align}
We see that of the 11 independent components, $A_{12}$, $B_{01}$, $B_{02}$, and $B_{12}$ are gauge invariant. For the $B_{\mu\nu}$ modes, this corresponds to a helicity $+2$ tensor mode and helicity $+1$ vector mode. However, it appears there is only one gauge invariant quantity, $A_{12}$, which grows as $e^{ikx}$. 
\end{document}