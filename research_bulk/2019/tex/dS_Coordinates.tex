\documentclass[10pt,letterpaper]{article}
\usepackage[textwidth=7in, top=1in,textheight=9in]{geometry}
\usepackage[fleqn]{mathtools} 
\usepackage{amssymb,braket,hyperref,xcolor}
\hypersetup{colorlinks, linkcolor={blue!50!black}, citecolor={red!50!black}, urlcolor={blue!80!black}}
\usepackage[title]{appendix}
%\numberwithin{equation}{section}
\setlength{\parindent}{0pt}
\title{de Sitter Geometries}
\date{}
\begin{document} 
	\maketitle
	\noindent 
%%%%%%%%%%%%%%%%%%%%%%%%%%%%%%
de Sitter space can be described as a submanifold embedded in a higher dimension Minskowski space. Working in $D=4$, take the $D+1$ Minkowski space defined as
\begin{eqnarray}
ds^2 = -dx_0^2 + dx_1^2 + dx_2^2 + dx_3^2 + dx_4^2.
\label{min5}
\end{eqnarray}
Now let us constrain our coordinates to a hyperboloid
\begin{eqnarray}
-x_0^2 + x_1^2 + x_2^2 + x_3^2 + x_4^2 = \alpha^2.
\label{hyper}
\end{eqnarray}
Taking the differential of \eqref{hyper},  we may relate $dx_4$ to the remaining coordinates
\begin{eqnarray}
f(x_0,x_1,x_2,x_3,x_4) &=& \alpha^2
\nonumber\\
df &=& \frac{\partial f}{\partial x_\mu}dx^\mu =0
\nonumber\\
\Rightarrow\quad x_4 dx_4 &=&x_0 dx_0-\mathbf x\cdot d\mathbf x 
\nonumber\\
\Rightarrow\quad dx_4^2 &=& \frac{(x_0 dx_0-\mathbf x\cdot d\mathbf x)^2}{x_4^2}
\nonumber\\
\Rightarrow\quad dx_4^2 &=& \frac{(x_0 dx_0 -\mathbf x\cdot d\mathbf x)^2}{\alpha^2+x_0^2 -\mathbf x^2}.
\end{eqnarray}
Hence we may express \eqref{min5} in terms of four coordinates
\begin{eqnarray}
ds^2 &=& -dx_0^2 + (d\mathbf x)^2 + \frac{x_0^2 dx_0^2 + (\mathbf x\cdot d\mathbf x)^2 - 2x_0 dx_0 (\mathbf x\cdot d\mathbf x)}{\alpha^2 + x_0^2 -\mathbf x^2}
\nonumber\\
&=& \frac{1}{\alpha^2 + x_0^2-\mathbf x^2}\left[ 
-dx_0^2(\alpha^2-\mathbf x^2) + dx_1^2(\alpha^2 +x_0^2 + x_1^2-\mathbf x^2) + ... -2 x_0 dx_0(\mathbf x\cdot d\mathbf x)\right].
\label{ds2}
\end{eqnarray}
Before proceeding, it is also worth noting how the curvature tensors are related to $\alpha^2$ in a $D=4$ maximally symmetric space
\begin{eqnarray}
R_{\lambda\mu\nu\kappa} &=& \frac{1}{\alpha^2}( g_{\mu\nu}g_{\lambda\kappa}-g_{\lambda\nu}g_{\mu\kappa})
\nonumber\\
R_{\mu\kappa}&=& -3/\alpha^2 g_{\mu\kappa}
\nonumber\\
R &=& -12/\alpha^2
\label{curverel}
\end{eqnarray}
Going back to the rather complicated line element, we may choose coordinates
\begin{eqnarray}
x_0 &=& \alpha \sinh(t/\alpha)
\nonumber\\
x_1 &=& \alpha\cosh(t/\alpha)\cos\chi
\nonumber\\
x_2 &=& \alpha\cosh(t/\alpha)\sin\chi\cos\theta
\nonumber\\
x_3&=& \alpha\cosh(t/\alpha)\sin\chi\sin\theta\cos\phi,
\end{eqnarray}
which brings \eqref{ds2} to
\begin{eqnarray}
ds^2 &=& -dt^2 + \alpha^2 \cosh^2(t/\alpha) d\chi^2 + \alpha^2 \cosh^2(t/\alpha)\sin^2\chi (d\theta^2 + \sin^2d\phi^2).
\end{eqnarray}
We may bring the de Sitter line element into a conformal flat form by first choosing coordinates
\begin{eqnarray}
x_0 &=& \alpha\sinh(t/\alpha)+ e^{t/\alpha}\mathbf x\cdot\mathbf x/2\alpha
\nonumber\\
x_1 &=& \alpha\cosh(t/\alpha)- e^{t/\alpha}\mathbf x\cdot\mathbf x/2\alpha
\nonumber\\
x_2 &=& e^{t/\alpha}X_1
\nonumber\\
x_3&=& e^{t/\alpha}X_2,
\end{eqnarray}
in which the line element becomes
\begin{eqnarray}
ds^2 &=& -dt^2 + e^{2t/\alpha} (d\mathbf X)^2.
\label{ds1}
\end{eqnarray}
We see that this looks like $k=0$ RW with $a(t) = e^{2t/\alpha}$. To bring to conformal form, take
\begin{eqnarray}
dt &=& e^{t/\alpha}d\tau
\nonumber\\
d\tau &=& \int dt e^{-t/\alpha}
\nonumber\\
\tau&=& -\alpha e^{-t/\alpha} + C
\nonumber\\
\tau &=& -\alpha e^{-t/\alpha} + \tau_{\infty}
\nonumber\\
\frac{\alpha^2}{(\tau-\tau_{\infty})^2} &=& e^{2t/\alpha}.
\label{coord1}
\end{eqnarray}
If time in both the $\tau$ and $t$ coordinates has the same infinitesimal direction (i.e positive $dt \Rightarrow +d\tau$) then coordinate time $\tau$ will be negative for $t>0$. 
\\ \\
From \eqref{coord1}, we may express \eqref{ds1} as 
\begin{eqnarray}
ds^2 &=& \frac{\alpha^2}{(\tau-\tau_{\infty})^2} \left[ -d\tau^2 + d\mathbf X^2 \right].
\end{eqnarray}
If we further define time coordinate $p = \tau-\tau_{\infty}$, then we may write the line element more conveniently as
\begin{eqnarray}
ds^2 = \frac{\alpha^2}{p^2}\left[ -dp^2 + d\mathbf X^2 \right].
\end{eqnarray}
Using curvature relations \eqref{curverel}, we may express the Einstein tensor as 
\begin{eqnarray}
G_{\mu\nu} &=& R_{\mu\nu} - \frac{1}{2}g_{\mu\nu} R
\nonumber\\
&=& \frac{3}{\alpha^2}g_{\mu\nu}.
\end{eqnarray}
Hence if we define a background $T_{\mu\nu}$ according to $-\kappa^2_4 T_{\mu\nu} =\frac{3}{\alpha^2} g_{\mu\nu}$, then it follows that the perturbation of the background $T_{\mu\nu}$ yields
\begin{eqnarray}
-\kappa^2_4\delta T_{\mu\nu} &=& \frac{3}{\alpha^2} h_{\mu\nu}
\end{eqnarray}
\begin{eqnarray}
-3(\dot\beta -\alpha) &=& \tau \nabla^2 \beta
\nonumber\\
-3\tau (\ddot \beta - \dot\alpha) + 12(\dot\beta-\alpha) &=& \tau^2 \nabla^2\alpha - 3\tau \nabla^2\beta
\end{eqnarray}
\end{document}