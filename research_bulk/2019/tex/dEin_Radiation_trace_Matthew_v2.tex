\documentclass[10pt,letterpaper]{article}
\usepackage[textwidth=7in, top=1in,textheight=9in]{geometry}
\usepackage[fleqn]{mathtools} 
\usepackage{amssymb,braket,hyperref,xcolor}
\hypersetup{colorlinks, linkcolor={blue!50!black}, citecolor={red!50!black}, urlcolor={blue!80!black}}
\usepackage[title]{appendix}
\usepackage[sorting=none]{biblatex}
\numberwithin{equation}{section}
\setlength{\parindent}{0pt}
\title{$g^{\mu\nu}\delta G_{\mu\nu}$ Radiation v2}
\date{}
\allowdisplaybreaks
\begin{document} 
\maketitle
\noindent 
%%%%%%%%%%%%%%%%%%%%%%%%%%%%%%%%
\section{Conformal Flat $\Omega(x)$}
%%%%%%%%%%%%%%%%%%%%%%%%%%%%%%%%%%
\begin{eqnarray}
ds^2 &=& (g_{\mu\nu} + h_{\mu\nu})dx^\mu dx^\nu = \Omega^2(x)(\tilde g_{\mu\nu} + f_{\mu\nu})dx^\mu dx^\nu
\label{geom1}\\
\tilde g_{\mu\nu} &=& \text{diag}\left(-1,1,r^2,r^2\sin^2\theta\right)\qquad \tilde \Gamma^{\lambda}_{\alpha\beta} = \delta^\lambda_i \delta^j_\alpha \delta^k_\beta \tilde \Gamma^{i}_{jk}
\end{eqnarray}
All subsequent equations hold for any flat $\tilde g_{\mu\nu}$, i.e. any $\tilde g_{\mu\nu}$ such that the corresponding curvature tensors vanish. 
%
%%%%%%%%%%%%%%%%%%%%%%%%%%%%%%%%%%%%
\subsection{$G_{\mu\nu}$}
%%%%%%%%%%%%%%%%%%%%%%%%%%%%%%%%%%%%
\begin{eqnarray}
G_{\mu\nu} &=& R_{\mu\nu} - \tfrac12 g_{\mu\nu} R
\\ \nonumber\\
g^{\mu\nu}G_{\mu\nu} &=& -R
\nonumber\\
&=& - \tilde{R} \Omega^{-2} - 6 \Omega^{-3} \tilde{\nabla}_{\alpha }\tilde{\nabla}^{\alpha }\Omega 
\nonumber\\
&=& 6 \overset{..}{\Omega} \Omega^{-3} - 6 \Omega^{-3} \tilde{\nabla}_{a}\tilde{\nabla}^{a}\Omega 
\label{Gtr}
\end{eqnarray}
%%%%%%%%%%%%%%%%%%%%%%%%%%%%%%%%%%%%
\subsection{$\delta(g^{\mu\nu}G_{\mu\nu})$}
%%%%%%%%%%%%%%%%%%%%%%%%%%%%%%%%%%%%
We calculate $\delta (g^{\mu\nu}G_{\mu\nu}) = -h^{\mu\nu}G_{\mu\nu}^{(0)} + g^{\mu\nu}\delta G_{\mu\nu}$ as this is the perturbed equation that follows directly from \eqref{Gtr}. Additional remarks on the trace are in \hyperref[sec:Remark on Trace]{Trace Gauge Invariance}.
\begin{eqnarray}
\delta(g^{\mu\nu} G_{\mu\nu})&=& -6 \dot{\phi} \dot{\Omega} \Omega^{-3} - 18 \dot{\psi} \dot{\Omega} \Omega^{-3} - 12 \overset{..}{\Omega} \phi \Omega^{-3} - 6 \overset{..}{\psi} \Omega^{-2} - 6 \dot{\Omega} \Omega^{-3} \tilde{\nabla}_{a}\tilde{\nabla}^{a}B - 2 \Omega^{-2} \tilde{\nabla}_{a}\tilde{\nabla}^{a}\dot{B} \nonumber \\ 
&& + 2 \Omega^{-2} \tilde{\nabla}_{a}\tilde{\nabla}^{a}\overset{..}{E} + 6 \dot{\Omega} \Omega^{-3} \tilde{\nabla}_{a}\tilde{\nabla}^{a}\dot{E} - 2 \Omega^{-2} \tilde{\nabla}_{a}\tilde{\nabla}^{a}\phi + 4 \Omega^{-2} \tilde{\nabla}_{a}\tilde{\nabla}^{a}\psi - 12 \psi \Omega^{-3} \tilde{\nabla}_{a}\tilde{\nabla}^{a}\Omega \nonumber \\ 
&& - 12 \Omega^{-3} \tilde{\nabla}_{a}\dot{\Omega} \tilde{\nabla}^{a}B - 6 \Omega^{-3} \tilde{\nabla}_{a}\Omega \tilde{\nabla}^{a}\dot{B} - 6 \Omega^{-3} \tilde{\nabla}_{a}\Omega \tilde{\nabla}^{a}\phi + 6 \Omega^{-3} \tilde{\nabla}_{a}\Omega \tilde{\nabla}^{a}\psi \nonumber \\ 
&& + 6 \Omega^{-3} \tilde{\nabla}^{a}\Omega \tilde{\nabla}_{b}\tilde{\nabla}^{b}\tilde{\nabla}_{a}E + 12 \Omega^{-3} \tilde{\nabla}_{b}\tilde{\nabla}_{a}\Omega \tilde{\nabla}^{b}\tilde{\nabla}^{a}E-12 B^{a} \Omega^{-3} \tilde{\nabla}_{a}\dot{\Omega} - 6 \dot{B}^{a} \Omega^{-3} \tilde{\nabla}_{a}\Omega \nonumber \\ 
&& + 6 \Omega^{-3} \tilde{\nabla}^{a}\Omega \tilde{\nabla}_{b}\tilde{\nabla}^{b}E_{a} + 12 \Omega^{-3} \tilde{\nabla}_{b}\tilde{\nabla}_{a}\Omega \tilde{\nabla}^{b}E^{a}+12 E^{ab} \Omega^{-3} \tilde{\nabla}_{b}\tilde{\nabla}_{a}\Omega 
\label{dG}
\end{eqnarray}
We substitute the gauge invariants and null trace condition ($g^{\mu\nu} G_{\mu\nu}=0\implies \tilde\nabla_a\tilde\nabla^a \Omega = \ddot\Omega$),
\begin{eqnarray}
\alpha &=& \phi + \psi +\dot B - \ddot E,\qquad \gamma = \psi -\Omega^{-1}[(B-\dot E)\dot\Omega - (\tilde\nabla_a E + E_a)\tilde\nabla^a \Omega],
\nonumber\\
Q_i&=& B_i - \dot E_i,\qquad E_{ij}.
\end{eqnarray}
The perturbed trace $\delta(g^{\mu\nu} G_{\mu\nu})$ is then expressed entirely in terms of the gauge invariants as
\begin{eqnarray}
\delta(g^{\mu\nu} G_{\mu\nu})&=& -6 \dot{\alpha} \dot{\Omega} \Omega^{-3} - 12 \dot{\gamma} \dot{\Omega} \Omega^{-3} - 12 \overset{..}{\Omega} \alpha \Omega^{-3} - 6 \overset{..}{\gamma} \Omega^{-2} - 2 \Omega^{-2} \tilde{\nabla}_{a}\tilde{\nabla}^{a}\alpha + 6 \Omega^{-2} \tilde{\nabla}_{a}\tilde{\nabla}^{a}\gamma \nonumber \\ 
&& - 6 \Omega^{-3} \tilde{\nabla}_{a}\Omega \tilde{\nabla}^{a}\alpha + 12 \Omega^{-3} \tilde{\nabla}_{a}\Omega \tilde{\nabla}^{a}\gamma -12 Q^{a} \Omega^{-3} \tilde{\nabla}_{a}\dot{\Omega} - 6 \dot{Q}^{a} \Omega^{-3} \tilde{\nabla}_{a}\Omega +12 E^{ab} \Omega^{-3} \tilde{\nabla}_{b}\tilde{\nabla}_{a}\Omega.
\label{dGtr1}
\end{eqnarray}
Defining the gauge invariants instead as
\begin{eqnarray}
\alpha &=& \phi + \psi +\dot B - \ddot E,\qquad \gamma =\phi- \psi+\dot B - \ddot E + 2\Omega^{-1}[(B-\dot E)\dot\Omega - (\tilde\nabla_a E + E_a)\tilde\nabla^a \Omega],
\nonumber\\
Q_i&=& B_i - \dot E_i,\qquad E_{ij}.
\end{eqnarray}
then \eqref{dG} becomes
\begin{eqnarray}
\delta(g^{\mu\nu} G_{\mu\nu})&=& -12 \dot{\alpha} \dot{\Omega} \Omega^{-3} + 6 \dot{\gamma} \dot{\Omega} \Omega^{-3} - 12 \overset{..}{\Omega} \alpha \Omega^{-3} - 3 \overset{..}{\alpha} \Omega^{-2} + 3 \overset{..}{\gamma} \Omega^{-2} + \Omega^{-2} \tilde{\nabla}_{a}\tilde{\nabla}^{a}\alpha - 3 \Omega^{-2} \tilde{\nabla}_{a}\tilde{\nabla}^{a}\gamma \nonumber \\ 
&& - 6 \Omega^{-3} \tilde{\nabla}_{a}\Omega \tilde{\nabla}^{a}\gamma -12 Q^{a} \Omega^{-3} \tilde{\nabla}_{a}\dot{\Omega} - 6 \dot{Q}^{a} \Omega^{-3} \tilde{\nabla}_{a}\Omega +12 E^{ab} \Omega^{-3} \tilde{\nabla}_{b}\tilde{\nabla}_{a}\Omega .
\end{eqnarray}

%%%%%%%%%%%%%%%%%%%%%%%%%%%%%%%%%
\subsection{Trace Gauge Invariance}
\label{sec:Remark on Trace}
%%%%%%%%%%%%%%%%%%%%%%%%%%%%%%%%%
With the transformation of the first order $\delta G_{\mu\nu}$ behaving as
\begin{eqnarray}
\Delta_\epsilon[\delta G_{\mu\nu}] &=& G^\lambda{}_\mu \nabla_\nu \epsilon_\lambda + G^\lambda{}_\nu \nabla_\mu \epsilon_\lambda + \nabla_\lambda G_{\mu\nu} \epsilon^\lambda,
\label{eptr1}
\end{eqnarray}
upon taking the trace, we have
\begin{eqnarray}
g^{\mu\nu}\Delta_{\epsilon}[\delta G_{\mu\nu}] &=& 2 G^{\lambda\mu} \nabla_\mu \epsilon_\lambda + \nabla_\lambda G^\mu{}_\mu \epsilon^\lambda. 
\end{eqnarray}
The above indicates that a vanishing $G^\mu{}_\mu$ alone does not ensure $g^{\mu\nu}\delta G_{\mu\nu}$ is gauge invariant. However, we may subtract from \eqref{eptr1} the contribution $h^{\mu\nu}G_{\mu\nu}$, which transforms as
\begin{eqnarray}
G_{\mu\nu}\Delta_\epsilon[h^{\mu\nu}] &=& 2 G^{\lambda\mu} \nabla_\mu \epsilon_\lambda.
\end{eqnarray}
Thus the quantity that is invariant under gauge transformation (assuming only that $G^\lambda{}_\lambda = 0$) is in fact
\begin{eqnarray}
\Delta_\epsilon[ \delta(g^{\mu\nu} G_{\mu\nu})]=\Delta_\epsilon[ -h^{\mu\nu}G_{\mu\nu} + g^{\mu\nu}\delta G_{\mu\nu}] = 0. 
\end{eqnarray}
If the background trace vanishes, it must be the perturbation of the full trace (and not just $g^{\mu\nu} \delta G_{\mu\nu}$) that is then gauge invariant. 
\end{document}