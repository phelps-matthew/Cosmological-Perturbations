\documentclass[10pt,letterpaper]{article}
\usepackage[textwidth=7in, top=1in,textheight=9in]{geometry}
\usepackage[fleqn]{mathtools} 
\usepackage{amssymb,braket,hyperref,xcolor}
\hypersetup{colorlinks, linkcolor={blue!50!black}, citecolor={red!50!black}, urlcolor={blue!80!black}}
\usepackage[title]{appendix}
\usepackage[sorting=none]{biblatex}
\numberwithin{equation}{section}
\setlength{\parindent}{0pt}
\title{4D SVT dS${}_4$ Einstein v3}
\date{}
\begin{document} 
\maketitle
\noindent 
%%%%%%%%%%%%%%%%%%%%%%%%%%%%%%%%%%
\section{$h_{\mu\nu}$ Decomposition}
%%%%%%%%%%%%%%%%%%%%%%%%%%%%%%%%%%
Curvature Tensors:
\begin{eqnarray}
R_{\lambda\mu\nu\kappa} &=& k(g_{\mu\nu}g_{\lambda\kappa}-g_{\lambda\nu}g_{\mu\kappa})
\nonumber\\
R_{\mu\kappa} &=& k(1-D)g_{\mu\kappa} = \frac{R}{D}g_{\mu\kappa}
\nonumber\\
R&=& kD(1-D) 
\end{eqnarray}
Covariant Commutation:
\begin{eqnarray}
[\nabla^\sigma \nabla_\nu] W_\sigma &=& -R_{\nu}{}^\sigma W_\sigma = -\frac{R}{D}W_\nu
\nonumber\\
{[}\nabla^\mu \nabla_\mu, \nabla_\nu] V &=& -R_{\nu}{}^\mu \nabla_\mu V = -\frac{R}{D}\nabla_\nu V
\nonumber\\
{[}\nabla^2,\nabla_\mu\nabla_\nu]V &=& \frac{2 g_{\mu\nu}R}{D(D-1)}\nabla^2 V - \frac{2R}{D-1}\nabla_\mu\nabla_\nu V
\label{comms}
\end{eqnarray}
Decomposition:
\begin{eqnarray}
h_{\mu\nu} &=& h_{\mu\nu}^{T\theta} + \nabla_\mu W_\nu + \nabla_\nu W_\mu - \frac{g_{\mu\nu}}{D-1}(\nabla^\sigma W_\sigma - h)
\nonumber\\
&& +\frac{2-D}{D-1}\left( \nabla_\mu\nabla_\nu -\frac{ g_{\mu\nu}R}{D(D-1)}\right) \int D(x,x') \nabla^\sigma W_\sigma
-\frac{1}{D-1}\left( \nabla_\mu\nabla_\nu -\frac{g_{\mu\nu}R}{D(D-1)}\right) \int D(x,x') h
\label{decomph}
\end{eqnarray}
\begin{eqnarray}
\left( \nabla_\alpha \nabla^\alpha - \frac{R}{D-1}\right)D(x,x') &=& g^{-1/2}\delta^4 (x-x')
\nonumber\\ \nonumber\\
\nabla^\mu h_{\mu\nu} &=& \left( \nabla_\alpha\nabla^\alpha-\frac{R}{D} \right) W_\nu
\end{eqnarray}
With the box-like operator mixing indices of $W_\nu$, the particular integral solution for $W_\nu$ involves a bi-tensor Green's function $F_{\sigma\rho'}$ which obeys
\begin{eqnarray}
\left( \nabla^\alpha\nabla_\alpha -\frac{R}{D}\right) F_{\sigma\rho'}(x,x') &=& g_{\sigma\rho'}g^{-1/2} \delta^4(x-x').
\label{fgreen}
\end{eqnarray}
Here $g_{\sigma\rho'}$ represents a parallel propagator, defined in terms of Vierbeins $e_\mu^a$:
\begin{eqnarray}
g^{\alpha'}{}_{\beta}(x,x') &=& e^{\alpha'}_a(x') e_{\beta}^a(x),\qquad g_{\mu\nu} = \eta_{ab}e_{\mu}^ae_\nu^b.
\end{eqnarray}
In terms of \eqref{fgreen}, $W_\nu$ has particular solution
\begin{eqnarray}
W_\nu &=& \int F_\nu{}^{\rho'}(x,x')\nabla^{\sigma'}h_{\rho'\sigma'}.
\end{eqnarray}
To construct a transverse vector $E_{\mu}$, split $W_{\mu}$
\begin{eqnarray}
W_{\mu} &=& \underbrace{W_\mu - \nabla_\mu \int A(x,x')\nabla^\sigma W_\sigma}_{E_\mu} +
\nabla_\mu \int A(x,x')\nabla^\sigma W_\sigma
\nonumber\\
\nabla_\alpha\nabla^\alpha A(x,x') &=& g^{-1/2} \delta^4(x-x')
\end{eqnarray}
With $h_{\mu\nu}^{T\theta} = 2E_{\mu\nu}$, \eqref{decomph} may be expressed as
\begin{eqnarray}
h_{\mu\nu} &=& 2E_{\mu\nu}^{T\theta} + \nabla_\mu E_\nu + \nabla_\nu E_\mu - \frac{g_{\mu\nu}}{D-1}(\nabla^\sigma W_\sigma -h) + 2\nabla_\mu\nabla_\nu \int A(x,x')\nabla^\sigma W_\sigma
\nonumber\\
&&+ \frac{1}{D-1}\left(\nabla_\mu\nabla_\nu - \frac{g_{\mu\nu}R}{D(D-1)}\right)\int D(x,x')\left[(2-D)\nabla^\sigma W_\sigma - h\right].
\end{eqnarray}
We may simplify this to
\begin{eqnarray}
h_{\mu\nu} &=& 2E_{\mu\nu}^{T\theta} + \nabla_\mu E_\nu + \nabla_\nu E_\mu 
\nonumber\\
&&+2\nabla_\mu \nabla_\nu \left( \int A(x,x')\nabla^\sigma W_\sigma + \frac{1}{2(D-1)}\int D(x,x')
[(2-D)\nabla^\sigma W_\sigma -h]\right)
\nonumber\\
&&-\frac{2g_{\mu\nu}}{2(D-1)}\left( \nabla^\sigma W_\sigma -h+ \frac{R}{D(D-1)}\int D(x,x')[(2-D)\nabla^\sigma W_\sigma -h]\right).
\end{eqnarray}
SVT Definitions:
\begin{eqnarray}
2E_{\mu\nu}^{T\theta} &=& h_{\mu\nu}^{T\theta}
\nonumber\\
E_\mu &=& W_\mu - \nabla_\mu\int A(x,x')\nabla^\sigma W_\sigma
\nonumber\\
E &=&  \int A(x,x')\nabla^\sigma W_\sigma + \frac{1}{2(D-1)}\int D(x,x')
[(2-D)\nabla^\sigma W_\sigma -h]
\nonumber\\ 
\psi &=& \frac{1}{2(D-1)}\left( \nabla^\sigma W_\sigma -h + \frac{R}{D(D-1)}\int D(x,x')[(2-D)\nabla^\sigma W_\sigma -h]\right)
\end{eqnarray}
In the flat space limit, $A(x,x')=D(x,x')$ and we have
\begin{eqnarray}
2E_{\mu\nu}^{T\theta} &=& h_{\mu\nu}^{T\theta}
\nonumber\\
E_\mu &=& W_\mu - \nabla_\mu\int D(x,x')\nabla^\sigma W_\sigma
\nonumber\\
E &=& \frac{1}{2(D-1)}\int D(x,x') [D\nabla^\sigma W_\sigma -h]
\nonumber\\ 
\psi &=& \frac{1}{2(D-1)}\left( \nabla^\sigma W_\sigma -h \right),
\end{eqnarray}
a form that coincides with \emph{Localization\_Condition\_Matthew} (2.1). 
%%%%%%%%%%%%%%%%%%%%%%%%%%%%%%%%%%
\section{$\delta T_{\mu\nu}$ Decomposition}
%%%%%%%%%%%%%%%%%%%%%%%%%%%%%%%%%%
For a conserved $\delta T_{\mu\nu}$ we take $W_\mu=0$.
\begin{eqnarray}
\delta T_{\mu\nu} &=& \delta T_{\mu\nu}^{T\theta} + \frac{g_{\mu\nu}}{D-1} \delta T
-\frac{1}{D-1}\left(\nabla_\mu\nabla_\nu - \frac{g_{\mu\nu}R}{D(D-1)}\right)\int D(x,x') \delta T 
\nonumber\\ \nonumber\\
6\bar\chi &=&\int D(x,x')\delta T
\nonumber\\ \nonumber\\
2\bar E_{\mu\nu} &=& \delta T_{\mu\nu}^{T\theta}
\nonumber\\ \nonumber\\
6(\nabla_\alpha\nabla^\alpha + 4k)\bar\chi &=& \delta T
\nonumber\\ \nonumber\\
\delta T_{\mu\nu} &=& 
2\left( \nabla_\alpha\nabla^\alpha g_{\mu\nu} +3k g_{\mu\nu} - \nabla_\mu\nabla_\nu\right)\bar\chi + 2\bar E_{\mu\nu}
\end{eqnarray}

%%%%%%%%%%%%%%%%%%%%%%%%%%%%%%%%%%
\section{dS$_4$ Background and Fluctuations}
%%%%%%%%%%%%%%%%%%%%%%%%%%%%%%%%%%
\begin{eqnarray}
G^{(0)}_{\mu\nu} &=& 3kg_{\mu\nu}
\nonumber\\
R^{(0)}_{\lambda\mu\nu\kappa} &=& k(g_{\mu\nu}g_{\lambda\kappa}-g_{\lambda\nu}g_{\mu\kappa})
\nonumber\\
R^{(0)}_{\mu\kappa} &=& -3k g_{\mu\kappa} = \frac{R}{D}g_{\mu\kappa}
\nonumber\\
R^{(0)}&=& -12 k
\nonumber\\ \nonumber\\
ds^2 &=& (g_{\mu\nu} + h_{\mu\nu})dx^\mu dx^\nu
\nonumber\\ \nonumber\\
\delta G_{\mu\nu}&=& 2 k h_{\mu \nu}
-  \tfrac{1}{2} k g_{\mu \nu} h
+ \tfrac{1}{2} \nabla_{\alpha}\nabla^{\alpha}h_{\mu \nu}
-  \tfrac{1}{2} g_{\mu \nu} \nabla_{\alpha}\nabla^{\alpha}h
+ \tfrac{1}{2} g_{\mu \nu} \nabla_{\beta}\nabla_{\alpha}h^{\alpha \beta}
-  \tfrac{1}{2} \nabla_{\mu}\nabla_{\alpha}h_{\nu}{}^{\alpha}\nonumber\\
&& -  \tfrac{1}{2} \nabla_{\nu}\nabla_{\alpha}h_{\mu}{}^{\alpha}
+ \tfrac{1}{2} \nabla_{\nu}\nabla_{\mu}h
\nonumber\\ \nonumber\\
\delta G &=&  \nabla^\alpha \nabla^\beta h_{\alpha\beta} - \nabla_\alpha\nabla^\alpha h
\nonumber\\ \nonumber\\
h_{\mu\nu} &=& -2g_{\mu\nu}\chi +2\nabla_\mu\nabla_\nu F + \nabla_\mu E_\nu + \nabla_\nu E_\mu + 2E_{\mu\nu}
\nonumber\\ \nonumber\\
\delta G_{\mu\nu} &=& 4 k E_{\mu \nu} + \nabla_{\alpha}\nabla^{\alpha}E_{\mu \nu} + 2 g_{\mu \nu} \nabla_{\alpha}\nabla^{\alpha}\chi + 3 k \nabla_{\mu}E_{\nu} + 3 k \nabla_{\nu}E_{\mu} + 6 k \nabla_{\nu}\nabla_{\mu}F - 2 \nabla_{\nu}\nabla_{\mu}\chi
\nonumber\\ \nonumber\\
\nabla^\mu \delta G_{\mu\nu} &=& 3k\nabla^\mu h_{\mu\nu} 
\nonumber\\
&=&  - 6 k \nabla_{\nu}\chi+ 18 k^2 \nabla_{\nu}F + 6 k \nabla_{\nu}\nabla_{\alpha}\nabla^{\alpha}F+ 9 k^2 E_{\nu} + 3 k \nabla_{\alpha}\nabla^{\alpha}E_{\nu}
\nonumber\\ \nonumber\\
-\kappa^2_4 T^{(0)}_{\mu\nu} &=& 3k g_{\mu\nu}
\nonumber\\ \nonumber\\
-\kappa^2_4 \delta T^{(b)}_{\mu\nu} &=& 3k h_{\mu\nu}
\nonumber\\ 
&=& -6kg_{\mu\nu}\chi +6k\nabla_\mu\nabla_\nu F + 3k\nabla_\mu E_\nu + 3k\nabla_\nu E_\mu + 6kE_{\mu\nu}
\nonumber\\ \nonumber\\
\nabla^\mu ( \delta G_{\mu\nu} + \kappa^2_4 \delta T_{\mu\nu}^{(b)}) &=& 0
\nonumber\\ \nonumber\\
\delta T_{\mu\nu} &=& \delta T_{\mu\nu}^{(b)}+\delta T_{\mu\nu}^{(s)}
\nonumber\\ \nonumber\\
-\kappa^2_4 \delta T_{\mu\nu}^{(s)} &=& 2\left( \nabla_\alpha\nabla^\alpha g_{\mu\nu} +3k g_{\mu\nu} - \nabla_\mu\nabla_\nu\right)\bar\chi + 2\bar E_{\mu\nu},\qquad -\kappa^2_4\nabla^\mu \delta T_{\mu\nu}^{(s)}=0
\end{eqnarray}

%%%%%%%%%%%%%%%%%%%%%%%%%%%%%%%%%%
\section{SVT Separation}
%%%%%%%%%%%%%%%%%%%%%%%%%%%%%%%%%%
\begin{eqnarray}
\delta G_{\mu\nu}+\kappa^2_4 \delta T_{\mu\nu}^{(b)}  &=& -\kappa^2_4 \delta T_{\mu\nu}^{(s)}
\nonumber\\
2\left( \nabla_\alpha\nabla^\alpha g_{\mu\nu} +3k g_{\mu\nu} - \nabla_\mu\nabla_\nu\right)\chi + (\nabla_\alpha\nabla^\alpha -2k) E_{\mu\nu}&=& 
2\left( \nabla_\alpha\nabla^\alpha g_{\mu\nu} +3k g_{\mu\nu} - \nabla_\mu\nabla_\nu\right)\bar\chi + 2\bar E_{\mu\nu}
\label{dgdt1}
\end{eqnarray}

Trace \eqref{dgdt1}:
\begin{eqnarray}
6(\nabla_\alpha\nabla^\alpha+4k)\chi &=& 6(\nabla_\alpha\nabla^\alpha+4k)\bar\chi
\end{eqnarray}
For $\chi=\bar\chi$, \eqref{dgdt1} becomes
\begin{eqnarray}
(\nabla_\alpha\nabla^\alpha -2k)E_{\mu\nu} &=& 2\bar E_{\mu\nu} 
\end{eqnarray}

%%%%%%%%%%%%%%%%%%%%%%%%%%%%%%%%%%
\section{SVT4 Covariant vs. Conformal}
%%%%%%%%%%%%%%%%%%%%%%%%%%%%%%%%%%
Perturbing on the background source, we have in SV4 covariant (barred quantities):
\begin{eqnarray}
\Delta &=& 6(\nabla_\alpha\nabla^\alpha+4k)\bar \chi
\label{Deltacov}
\\ \nonumber\\
\Delta_{\mu\nu} &=& 2\left( \nabla_\alpha\nabla^\alpha g_{\mu\nu} +3k g_{\mu\nu} - \nabla_\mu\nabla_\nu\right)\bar \chi + (\nabla_\alpha\nabla^\alpha -2k) \bar E_{\mu\nu}
\label{Deltauvcov}
\end{eqnarray}
This may be compared to SVT4 conformal (unbarred quantities):
\begin{eqnarray}
\Delta_{\mu\nu} &=& \tau^{-2}\big( 6 \ddot{F} \tilde{g}_{\mu \nu}
+ 2 \dot{\chi} \tilde{g}_{\mu \nu} \tau
+ 6 \tilde{g}_{\mu \nu} \chi
+ 2 \tilde{g}_{\mu \nu} \tau \tilde{\nabla}_{\alpha}\tilde{\nabla}^{\alpha}\dot{F}
+ 2 \tilde{g}_{\mu \nu} \tau^2 \tilde{\nabla}_{\alpha}\tilde{\nabla}^{\alpha}\chi
+ 2 \tau U_{\nu} \tilde{\nabla}_{\mu}\chi
+ 2 \tau U_{\mu} \tilde{\nabla}_{\nu}\chi\nonumber\\
&& - 2 \tau \tilde{\nabla}_{\nu}\tilde{\nabla}_{\mu}\dot{F}
- 2 \tau^2 \tilde{\nabla}_{\nu}\tilde{\nabla}_{\mu}\chi
+6 \dot{E}^{\alpha} \tilde{g}_{\mu \nu} U_{\alpha}
+ 2 \tilde{g}_{\mu \nu} \tau U^{\alpha} \tilde{\nabla}_{\beta}\tilde{\nabla}^{\beta}E_{\alpha}
- 2 \tau U^{\alpha} \tilde{\nabla}_{\nu}\tilde{\nabla}_{\mu}E_{\alpha}
\nonumber\\
&&+2 \dot{E}_{\mu \nu} \tau
+ 6 E_{\alpha \beta} \tilde{g}_{\mu \nu} U^{\alpha} U^{\beta}
+ \tau^2 \tilde{\nabla}_{\alpha}\tilde{\nabla}^{\alpha}E_{\mu \nu}
- 2 \tau U^{\alpha} \tilde{\nabla}_{\mu}E_{\nu \alpha}
- 2 \tau U^{\alpha} \tilde{\nabla}_{\nu}E_{\mu \alpha} \big)
\label{Deltauvcon}
\\ \nonumber\\
\Delta &=&k( 24 \ddot{F} + 12 \dot{\chi} \tau + 24 \chi + 6 \tau \tilde{\nabla}_{\alpha}\tilde{\nabla}^{\alpha}\dot{F} + 6 \tau^2 \tilde{\nabla}_{\alpha}\tilde{\nabla}^{\alpha}\chi
+24 \dot{E}^{\alpha} U_{\alpha} + 6 \tau U^{\alpha} \tilde{\nabla}_{\beta}\tilde{\nabla}^{\beta}E_{\alpha}
+24 E_{\alpha \beta} U^{\alpha} U^{\beta} )
\label{Deltacon}
\end{eqnarray}
Evaluating \eqref{Deltacon} in conformal flat coordinates
\begin{eqnarray}
 6(\nabla_\alpha\nabla^\alpha+4k)\bar \chi &=& 6k ( 2  \dot{\bar \chi} \tau + 4 \bar \chi +  \tau^2 \tilde{\nabla}_{\alpha }\tilde{\nabla}^{\alpha }\bar \chi )
\end{eqnarray}
Equating \eqref{Deltacov} to \eqref{Deltacon}
\begin{eqnarray}
12  \dot{\bar \chi} \tau + 24 \bar \chi + 6 \tau^2 \tilde{\nabla}_{\alpha }\tilde{\nabla}^{\alpha }\bar \chi
&=&  24 \ddot{F} + 12 \dot{\chi} \tau + 24 \chi + 6 \tau \tilde{\nabla}_{\alpha}\tilde{\nabla}^{\alpha}\dot{F} + 6 \tau^2 \tilde{\nabla}_{\alpha}\tilde{\nabla}^{\alpha}\chi
\nonumber\\ 
&&+24 \dot{E}^{\alpha} U_{\alpha} + 6 \tau U^{\alpha} \tilde{\nabla}_{\beta}\tilde{\nabla}^{\beta}E_{\alpha}
+24 E_{\alpha \beta} U^{\alpha} U^{\beta}
\end{eqnarray}
\begin{eqnarray}
6(\nabla_\alpha \nabla^\alpha + 4k)\bar\chi &=& \nabla^\alpha\nabla^\beta h_{\alpha\beta} - (\nabla_\alpha\nabla^\alpha  + 3k) h
\end{eqnarray}
Various Relations:
\begin{eqnarray}
\bar\chi &=& \frac{1}{6}\int g^{1/2} D(x,x') \left[  \nabla^\alpha\nabla^\beta h_{\alpha\beta} - (\nabla_\alpha\nabla^\alpha  + 3k) h\right]
\\ \nonumber\\
\left(\nabla_\alpha\nabla^\alpha -\frac{R}{D-1}\right)D(x,x') &=& g^{-1/2} \delta^4(x-x')
\\ \nonumber\\
\Omega^4k ( 2  \tau \partial_0  + 4 +  \tau^2 \tilde{\nabla}_{\alpha }\tilde{\nabla}^{\alpha } )D(x,x') &=& \delta^4(x-x') = \tilde\nabla_\alpha\tilde\nabla^\alpha F(x,x') 
\\ \nonumber\\
\nabla_\alpha\nabla^\alpha \bar\chi &=& k( 
2  \dot{\chi} \tau +  \tau^2 \tilde{\nabla}_{\alpha }\tilde{\nabla}^{\alpha }\bar \chi )
\\ \nonumber\\
\nabla^\alpha\nabla^\beta h_{\alpha\beta} - (\nabla_\alpha\nabla^\alpha  + 3k) h
&=&- \Omega^{-2} \tilde{\nabla}_{\alpha }\tilde{\nabla}^{\alpha }f
-  f \Omega^{-3} \tilde{\nabla}_{\alpha }\tilde{\nabla}^{\alpha }\Omega
- 3 \Omega^{-3} \tilde{\nabla}_{\alpha }\Omega \tilde{\nabla}^{\alpha }f
+ 2 f \Omega^{-4} \tilde{\nabla}_{\alpha }\Omega \tilde{\nabla}^{\alpha }\Omega
\nonumber\\
&& + 6 \Omega^{-3} \tilde{\nabla}^{\alpha }\Omega \tilde{\nabla}_{\beta }f_{\alpha }{}^{\beta }
+ \Omega^{-2} \tilde{\nabla}_{\beta }\tilde{\nabla}_{\alpha }f^{\alpha \beta }
+ 4 f^{\alpha \beta } \Omega^{-3} \tilde{\nabla}_{\beta }\tilde{\nabla}_{\alpha }\Omega
+ 4 f_{\alpha \beta } \Omega^{-4} \tilde{\nabla}^{\alpha }\Omega \tilde{\nabla}^{\beta }\Omega
\nonumber\\ \nonumber\\
 &=&-3 k \dot{f} \tau + 12 k f_{\alpha \beta } U^{\alpha } U^{\beta } -  k \tau^2 \tilde \nabla_{\alpha }\tilde \nabla^{\alpha }f
\nonumber\\
&& + 6 k \tau U^{\alpha } \tilde\nabla_{\beta }f_{\alpha }{}^{\beta } + k \tau^2 \tilde \nabla_{\beta }\tilde \nabla_{\alpha }f^{\alpha \beta }
\end{eqnarray}

%%%%%%%%%%%%%%%%%%%%%%%%%%%%%%%%%%
\end{document}