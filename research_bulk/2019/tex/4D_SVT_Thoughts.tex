\documentclass[10pt,letterpaper]{article}
\usepackage[textwidth=7in, top=1in,textheight=9in]{geometry}
\usepackage[fleqn]{mathtools} 
\usepackage{amssymb,braket,hyperref,xcolor}
\hypersetup{colorlinks, linkcolor={blue!50!black}, citecolor={red!50!black}, urlcolor={blue!80!black}}
\usepackage[title]{appendix}
%\numberwithin{equation}{section}
\setlength{\parindent}{0pt}
\title{4D SVT Thoughts}
\date{}
\begin{document} 
	\maketitle
	\noindent 
%%%%%%%%%%%%%%%%%%%%%%%%%%%%%%%%%%%
\section*{Green's Identity}
%%%%%%%%%%%%%%%%%%%%%%%%%%%%%%%%%%%
For any function $\phi$, we may always represent it as
\begin{eqnarray}
\phi = \int D \nabla^2 \phi + \int \nabla^\rho[\nabla_\rho D\phi - D\nabla_\rho \phi].
\end{eqnarray}
Given $\nabla^2\phi =\rho$, this leads us to the fundamental solution to Laplace's equation,
\begin{eqnarray}
\phi = \int D \rho + \int dS^\nu[\nabla_\rho D\phi - D\nabla_\rho \phi].
\end{eqnarray}
In flat space, the same identity holds for vectors, i.e.
\begin{eqnarray}
A_\mu = \int D \nabla^2 A_\mu + \int \nabla^\rho[\nabla_\rho DA_\mu - D\nabla_\rho A_\mu].
\end{eqnarray}
Given $\nabla^2 A_\mu = J_\mu$, the fundamental solution is then
\begin{eqnarray}
A_\mu = \int D J_\mu + \int \nabla^\rho[\nabla_\rho DA_\mu - D\nabla_\rho A_\mu].
\label{asol1}
\end{eqnarray}
It is of interest to ask what happens when we take the divergence:
\begin{eqnarray}
\nabla^\mu A_\mu &=& \nabla^\mu \int D J_\mu + \nabla^\mu \int \nabla^\rho[\nabla_\rho DA_\mu - D\nabla_\rho A_\mu]
\nonumber\\
&=& \int D \nabla^\mu J_\mu - \int \nabla^\mu (DJ_\mu) + \int \nabla^\rho[\nabla^\mu D\nabla_\rho A_\mu- \nabla^\mu \nabla_\rho DA_\mu ]
\nonumber\\
&=& \int D \nabla^\mu J_\mu + \int \nabla^\rho[\nabla^\mu D \nabla_\rho A_\mu - \nabla^\mu \nabla_\rho D A_\mu - D\nabla^2 A_\rho ]
\label{Adiv1}
\end{eqnarray}
Using the delta function relation
\begin{eqnarray}
\int \nabla^2 D \nabla^\mu A_\mu &=& - \int \nabla^\mu \nabla^2 D A_\mu
\end{eqnarray}
we may express \eqref{Adiv1} as 
\begin{eqnarray}
\nabla^\mu A_\mu &=& \int D \nabla^\mu J_\mu + \int \nabla^\rho[ \nabla_\rho D \nabla^\mu A_\mu - D\nabla_\rho \nabla^\mu A_\mu],
\label{adiv}
\end{eqnarray}
which we may recognize as the fundamental solution to $\nabla^2 (\nabla^\mu A_\mu) = \nabla^\mu J_\mu$, where we treat $\nabla^\mu A_\mu$ as a scalar with no apriori assumptions on the form of $A_\mu$. 
\\ \\
Hence we have shown that the divergence of $A_\mu$ as defined by \eqref{asol1} is consistent with the fundamental solution of $\nabla^2 \phi = \rho$ where $\phi = \nabla^\mu A_\mu$ and $\rho = \nabla^\mu J_\mu$. In order to construct an $A_\mu$ that obeys $\nabla^\mu A_\mu = \int D \nabla^\mu J_\mu$ we require $\nabla^\mu A_\mu$ and $D$ to vanish on the surface.
%
%
%
%%%%%%%%%%%%%%%%%%%%%%%%%%%%%%%%%%%
\section*{4D SVT}
%%%%%%%%%%%%%%%%%%%%%%%%%%%%%%%%%%%
In reference to your email, according to decomposition (C), $W_\mu$ is fixed by condition (D)
\begin{eqnarray}
\nabla^2 W_\mu = \nabla^\alpha h_{\alpha\mu}. 
\end{eqnarray}
The fundamental solution to (D) is
\begin{eqnarray}
W_\mu = \int D \nabla^\alpha h_{\alpha\mu} + \int \nabla^\alpha[\nabla_\alpha D W_\mu - D\nabla_\alpha W_\mu].
\label{Wfun}
\end{eqnarray}
Taking the divergence, we have from \eqref{adiv} 
\begin{eqnarray}
\nabla^\alpha W_\alpha &=& \int D \nabla^\alpha\nabla^\beta h_{\alpha\beta} + \int \nabla^\rho[ \nabla_\rho D \nabla^\alpha W_\alpha - D\nabla_\rho \nabla^\alpha W_\alpha]
\label{divW}
\end{eqnarray}
From (D) we may construct (G) as
\begin{eqnarray}
\nabla^2[\nabla^\alpha W_\alpha - h] = \nabla^\alpha\nabla^\beta h_{\alpha\beta} - \nabla^2 h,
\end{eqnarray}
which is equivalent to
\begin{eqnarray}
\nabla^2 \nabla^\alpha W_\alpha = \nabla^\alpha\nabla^\beta h_{\alpha\beta}.
\label{G2}
\end{eqnarray}
Based on the Green's identities, we have shown that fundamental solution to the above $\nabla^2 \nabla^\alpha W_\alpha = \nabla^\alpha\nabla^\beta h_{\alpha\beta}$ is again just \eqref{divW}. Decomposing $h$ into its harmonic and non-harmonic components via
\begin{eqnarray}
h = \int D \nabla^2 h + \int \nabla^\alpha [\nabla_\alpha D h - D\nabla_\alpha h],
\end{eqnarray}
the most general combination $\nabla^\alpha W_\alpha - h$ that satisfies the condition $\nabla^2 W_\mu = \nabla^\alpha h_{\mu\alpha}$
must be
\begin{eqnarray}
\nabla^\alpha W_\alpha - h &=& \int D [ \nabla^\alpha\nabla^\beta h_{\alpha\beta} -\nabla^2 h] + \underbrace{\oint dS^\rho[ \nabla_\rho D(\nabla^\alpha W_\alpha - h) - D \nabla_\rho (\nabla^\alpha W_\alpha - h)]}_{A}.
\end{eqnarray}
The harmonic surface term $A$ may vanish given that $D$, $h$, and $\nabla^\alpha W_\alpha$ vanish on the surface. Such constraints would appear to correspond to the freedom to perform integration by parts. Hence it would appear we cannot construct a $\psi$ or $\nabla^2 E_{\mu\nu}$ that is gauge invariant under large spatial gauge transformations.
\\ \\
Nonetheless, in defining 
\begin{eqnarray}
\psi &=& \nabla^\alpha W_\alpha - h,
\end{eqnarray}
it holds that $\nabla^2 \psi$, $\nabla^2 E_{\mu\nu} + (D-2)\nabla_\mu \nabla_\nu \psi$ and $\nabla^4 E_{\mu\nu}$ are gauge invariant. 
\\ \\
Though we can reframe the decomposition entirely in terms of a $W_\mu$ which must obey $\nabla^2 W_\mu = \nabla^\alpha h_{\mu\alpha}$, the most general decomposition that is gauge invariant under all transformations would appear to require a $W_\mu$ with the form given in \eqref{Wfun}. 






\end{document}