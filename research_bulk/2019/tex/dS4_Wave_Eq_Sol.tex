\documentclass[10pt,letterpaper]{article}
\usepackage[textwidth=7in, top=1in,textheight=9in]{geometry}
\usepackage[fleqn]{mathtools} 
\usepackage{amssymb,braket,hyperref,xcolor}
\hypersetup{colorlinks, linkcolor={blue!50!black}, citecolor={red!50!black}, urlcolor={blue!80!black}}
\usepackage[title]{appendix}
%\numberwithin{equation}{section}
\setlength{\parindent}{0pt}
\title{de Sitter Geometries}
\date{}
\begin{document} 
	\maketitle
	\noindent 
%%%%%%%%%%%%%%%%%%%%%%%%%%%%%%
de Sitter space can be described as a submanifold embedded in a higher dimension Minskowski space. Working in $D=4$, take the $D+1$ Minkowski space defined as
\begin{eqnarray}
ds^2 = -dx_0^2 + dx_1^2 + dx_2^2 + dx_3^2 + dx_4^2.
\end{eqnarray}
Now let us constrain our coordinates to a hyperboloid
\begin{eqnarray}
-x_0^2 + x_1^2 + x_2^2 + x_3^2 + x_4^2 = C^2
\end{eqnarray}



\end{document}