\documentclass[10pt,letterpaper]{article}
\usepackage{mymacros}

\title{Gauge Invariant $\delta G_{\mu\nu} = \delta T_{\mu\nu}$}
\author{}
\date{}
\newcommand{\hu}{\mathcal H}
\begin{document}
\maketitle 
\noindent Perturbed metric:
\be
	ds^2 = \Omega^2 \clr{ -(1+2\phi) d\tau^2 + 2(\del_i B + B_i)d\tau dx^i + \blr{ (1-2\psi)\delta_{ij} + 2\del_i\del_j E + \del_i E_j + \del_j E_i + 2E_{ij}}dx^i dx^j}
\ee
where
\[
	\del^i B_i = 0,\ \del^i E_i = 0,\ \del^i E_{ij} = 0,\ \delta^{ij}E_{ij} = 0.
\]
Under coordinate transformation
\be
	x^\mu \to \tilde x^\mu = x^\mu + \ep^\mu
\ee
where 
\[
	\ep^\mu = (\ep^0, \del^i \ep + \ep^i),\qquad \del^i \ep_i = 0
\]
the components of the metric transform as
\begin{align}
	\tilde \phi &= \phi - H \ep^0 - \dot \ep^0\\
	\tilde \psi &= \psi + H \ep^0\\
	\tilde B &=B + \ep^0 - \dot \ep\\
	\tilde E &= E-\ep \\
	\tilde E_{i}&= E_i - \ep_i\\
	\tilde B_i &= B_i - \dot \ep_i\\
	\tilde E_{ij} &= E_{ij}
\end{align}
From the above, we may form gauge invariant combinations (adding to 6 DOF):
\begin{align}
	\Phi &= \phi - H(\dot E - B) - (\ddot E - \dot B)\\
	\Psi &= \psi + H(\dot E-B)\\
	\mathcal Q_i &= B_i - \dot E_i\\
	E_{ij} &= E_{ij}
\end{align}
By orthogonal and parallel projections to the four velocity $u^\mu$, a generic symmetric $T_{\mu\nu}$ may be decomposed as
\be
	T_{\mu\nu} = (\rho + p)u_\mu u_\nu + p g_{\mu\nu} + u_\nu q_\mu + u_\mu q_\nu + \pi_{\mu\nu}
\ee
where
\[
	u^\mu q_\mu = 0,\ g^{\mu\nu} \pi_{\mu\nu} = 0, \ u^\mu u_\nu u^\rho u_\sigma \pi_{\nu\sigma} = 0.
\]
The conditions on $\pi_{\mu\nu}$ specify that it is traceless and orthogonal to the four velocity $u^\mu$, i.e. $\pi_{\mu\nu} = \pi_{ij}$. We may further decompose $\pi_{ij}$ as
\be
	\pi_{ij} = \del_i \del_j \Pi - \frac13 \del^2 \Pi \delta_{ij} + \frac12 \del_i \Pi_j + \frac12 \del_j \Pi_i + \Pi_{ij}
\ee
where as expected,
\[
	\del^i \Pi_i = 0,\  \del^i\Pi_{ij} = 0,\ \delta^{ij}\Pi_{ij} = 0.
\]
We have 2 degrees of freedom from $\rho$ and $p$, 3 from $q_\mu$, and 5 from $\pi_{\mu\nu}$ adding to 10 in total. We decompose $T_{\mu\nu}$ into a background piece and first order fluctuations (mixed tensor to match Ellis):
\[
	T_{\mu\nu} = {}^{(0)}T_{\mu\nu} + \delta T_{\mu\nu}.
\]
We start by separating scalars, where according to homogeneity and isotropy, the background may only depend on $\tau$, 
\begin{align}
	\rho(x^\mu) &= \bar \rho(\tau) + \delta \rho(x^\mu)\\
	p(x^\mu) &= \bar p(\tau) + \delta p(x^\mu).
\end{align}
The four velocity is also perturbed
\be
	u^\mu = \frac{1}{a}\frac{dx^\mu}{d\tau}= \bar u^\mu + \delta u^\mu
\ee
where $\bar u^\mu = a^{-1} \delta^\mu{}_0$ and $\delta u^i = \del^i v + v^i$ with $\del_i v^i = 0$. By normalization of the four velocity $-1 = g_{\mu\nu}u^\mu u^\nu$, we may derive the background and perturbed components of $u^\mu$:
\be
	u^\mu = \frac{1}{\Omega}\plr{ 1-\phi,\ \del^i v + v^i},\qquad u_\mu = \Omega\plr{ -1-\phi,\ \del_i v + v_i + \del_i B - B_i}.
\ee
Since the background of interest (FLRW) is homogeneous and isotropic, there is no anisotropic stress $\pi_{\mu\nu}$ at zeroth order and so $\pi_{ij}$ itself is first order. We may now form the perturbed E-M tensor:
\[
	\delta T_{\mu\nu} = (\delta\rho + \delta p)u_{\mu}u_\nu + (\rho+p)\delta u_\mu u_\nu + (\rho+p)u_\mu \delta u_\nu + \delta p g_{\mu\nu} + ph_{\mu\nu} +  \pi_{\mu\nu}
\]
\begin{align}
	\delta T_{00} &= \Omega^2(2\rho \phi + \delta \rho)\\
	\delta T_{0i} &= -\Omega^2\rho(\del_i v + \del_i B + v_i + B_i)-\Omega^2 p(\del_i v + v_i)\\
	\delta T_{ij} &= \Omega^2ph_{ij} + \Omega^2 \delta_{ij} \delta p + \pi_{ij}
\end{align}
Under gauge transformation (2), scalars transform as (see A.1)
\begin{align}
	\delta \tilde \rho &= \delta \rho - \ep^0 \dot{\bar\rho}\\
\delta \tilde p &= \delta p - \ep^0 \dot{\bar p}
\end{align}
and the velocity transforms as (see A.2)
\be
	\tilde v = v + \dot \ep,\quad \tilde v^i = v^i + \dot \ep^i.
\ee
The components of $\pi_{ij}$, that is $\Pi$, $\Pi_i$ and $\Pi_{ij}$ are all gauge invariant since they vanish in the background (A.3). From these transformation laws, we may form many gauge invariant quantities (omitting the bars on all background quantities now and denoting $\sigma \equiv \dot E-B$):
\begin{align}
	\delta \rho_\sigma &= \delta \rho - \dot\rho\sigma \\
	\delta p_\sigma &= \delta p - \dot p\sigma\\
	\mathcal V &= v+\dot E\\
	\mathcal B_i &= B_i + v_i\\
	\pi_{ij} &= \pi_{ij}
\end{align}
and $\Pi, \Pi_i, \Pi_{ij}$. 
\\ \\
\section*{Einstein equations }
$\delta G_{\mu\nu} = -8\pi G \delta T_{\mu\nu}$
\\
\\
\textbf{Scalars:}
\\
\\
\underline{$\delta G_{00} = -8\pi G\delta T_{00}$:}
\ba
	\delta G_{00} & = -2\del^2 \psi - 2\mathcal H\del^2 \sigma + 6\mathcal H\dot\psi\\
		&= -2\del^2 \Psi + 6\mathcal H \dot \Psi - 6\mathcal H\dot{\mathcal H}\sigma - 6{\mathcal H}^2\dot\sigma
\ea
\ba
	-8\pi G\delta T_{00} & =-8\pi G\Omega^2(2\rho \phi + \delta \rho) \\
		&=-6\mathcal H^2\Phi - 6\mathcal H^2 \dot\sigma - 6\mathcal H \dot{\mathcal H}\sigma - 8\pi G\Omega^2 \delta \rho_\sigma
\ea
This leads to field equation
\[
	\boxed{ \del^2 \Psi - 3\mathcal H \dot\Psi - 3\mathcal H^2 \Phi - 4\pi G \Omega^2 \delta\rho_\sigma = 0}
\]
\\
\\
\underline{$\delta G_{0i} = -8\pi G\delta T_{0i}$:}
\ba
	\delta G_{0i} & = \del_i\plr{ -2\dot \psi -2 \mathcal H \phi - \mathcal H^2 B + 2\frac{\ddot \Omega}{\Omega} B}\\
		&= \del_i\plr{ 2\dot \Psi + 2\dot{\mathcal H}\sigma +2\dot{\mathcal H}B + \mathcal H^2 B - 2\mathcal H \Phi - 2\mathcal H^2 \sigma}
\ea
\ba
	-8\pi G\delta T_{0i} & =-8\pi G\Omega^2\del_i(\rho(v+B)+pv) \\
		&=\del_i\plr{-v(2\dot{\mathcal H}+\mathcal H^2) + 3\mathcal H^2 (v+B)}
\ea
This leads to field equation
\\
\\
\underline{$\delta G_{ij} = -8\pi G\delta T_{ij},\quad i\ne j$:}
\ba
	\delta G_{ij} & = \del_i \del_j\plr{-\dot \sigma + \phi - \psi -2\hu^2E - 2\hu \sigma +4\dot\hu E+4\hu^2 E}\\
		&= \del_i\del_j\plr{ \Phi - \Psi +2\hu^2 E +4\dot\hu E}
\ea
\ba
	-8\pi G\delta T_{0i} & =-8\pi G\Omega^2\del_i\del_j (2pE) \\
		&=\del_i\del_j(2(2\dot\hu + \hu^2)E)
\ea
This leads to field equation
\[
	\boxed{\Psi-\Phi = 0}
\]
\\
\\
\underline{$\delta^{ij}\delta G_{ij} = -8\pi G\delta^{ij}\delta T_{ij},$:}
\ba
	\delta^{ij}\delta G_{ij} & =2\del^2\psi - 2\del^2 \hu + 2\delta \dot\sigma - 2\hu^2 \del^2E -6\ddot\psi -6\hu\dot\phi+ 6\hu^2 (\phi+\psi) +4\hu\del^2\sigma \\
&\qquad\qquad + 4(\hu^2+\dot\hu)\del^2 E-12(\hu^2 +\dot\hu)(\phi+\psi) - 12\hu\dot\psi \\
		&= 2\del^2 (\Psi-\Phi) -6\ddot\Psi - 6\hu^2(\Psi+\Phi) + 2\hu^2 \del^2 E + 6\ddot \hu\sigma + 6\hu\dot\hu \sigma + 4\dot\hu \del^2 E - 6\hu\dot\Phi \\
&\qquad\qquad - 12\dot \hu(\Psi+\Phi) - 12\hu \dot \Psi
\ea
\ba
	-8\pi G\delta^{ij}\delta T_{ij} & =-8\pi G\Omega^2(-6\psi p+2p\del^2E+3\delta p)\\
		&=2(2\dot\hu +\hu^2)\del^2 E + 6\hu\dot\hu \sigma + 6\ddot\hu \sigma -12 \dot \hu\Psi - 12\hu^2 \psi - 24\pi G\Omega^2 \delta\rho_\sigma
\ea
This leads to field equation
\[
	\boxed{3\ddot\Psi + 6\dot\hu \Psi + 6i = 0}
\]
\textbf{Vectors:}
\\
\\
\underline{$\delta G_{0i} = -8\pi G\delta T_{0i}$:}
\ba
	\delta G_{0i} & =-\frac{1}{2}\del^2(\dot E_i - B_i)-\hu^2 B_i + 2B_i(\dot \hu + \hu^2) \\
&= \frac{1}{2}\del^2\mathcal Q_i+ B_i(2\dot \hu + \hu^2) 
\ea
\ba
	-8\pi G\delta T_{0i} & =-8\pi G\Omega^2(-\rho B_i - (\rho+p)v_i)\\
		&=3\hu^2(v_i + B_i)-(2\dot \hu +\hu^2)v_i\\
&= 3\hu^2\mathcal B_i - 2(\dot\hu +\hu^2)v_i
\ea
This leads to field equation
\[
	\boxed{ \frac12 \del^2 \mathcal Q_i - 3\hu^2 \mathcal B_i + (2\dot\hu +\hu^2)\mathcal B_i = 0}
\]
\\
\\
\underline{$\delta G_{ii} = -8\pi G\delta T_{ii}$:}
\ba
	\delta G_{ii} & = \del_i \blr{\dot{\mathcal Q_i} + 2\hu^2 E_i + 2\hu\mathcal Q_i + 4\dot\hu E_i}\\
&= \frac{1}{2}\del^2\mathcal Q_i+ B_i(2\dot \hu + \hu^2) 
\ea
\ba
	-8\pi G\delta T_{ii} & =-8\pi G\Omega^2\del_i(2E_ip)\\
		&=(4\dot\hu + 2\hu^2)\del_i E_i
\ea
This leads to field equation
\[
	\boxed{ \frac12 \del^2 \mathcal Q_i - 3\hu^2 \mathcal B_i + (2\dot\hu +\hu^2)\mathcal B_i = 0}
\]
\textbf{Tensors:}
\\
\\
\underline{$\delta G_{ij} = -8\pi G\delta T_{ij}$:}
\ba
	\delta G_{ij} & =\del^2 E_{ij} - \ddot E_{ij} -2\hu^2 E_{ij} -2\hu\dot E_{ij} + 4(\dot\hu +\hu^2)E_{ij} 
\ea
\ba
	-8\pi G\delta T_{ij} & =-8\pi G\Omega^2(pE_{ij} + \pi_{ij})\\
		&=(2\dot\hu + \hu^2)(E_{ij}+\pi_{ij})
\ea
This leads to field equation
\[
	\boxed{ \del^2 E_{ij} - \ddot E_{ij} -2\hu \dot E_{ij} + (2\dot\hu +\hu^2)(E_{ij} - \pi_{ij})}
\]
\section*{Weyl equations }
In conformal gravity, there are only 5 independent degrees of freedom (via the traceless $K_{\mu\nu}$):
\ba
	\Sigma &= \Psi + \Psi = \psi + \phi - \dot\sigma\\
	\mathcal Q_i &= B_i - \dot E_i\\
	E_{ij} &= E_{ij} .
\ea
The gauge invariance of the SVT decomposition $\delta W_{\mu\nu}$ is immediate, and here we write $\delta W_{\mu\nu}$ in terms of the gauge invariant quanties for arbitrary $\Omega$.
\\ \\
\textbf{Scalars:}
\ba
	\delta W_{00} & = -\frac{2}{3\Omega^2}\del^4 \Sigma\\
	\delta W_{0i} &=  -\frac{2}{3\Omega^2}\del^4\dot\Sigma\\
	\delta W_{ij} & = \frac{1}{3\Omega^2}\plr{ g_{ij}\del^2 \ddot \Sigma + \del^2 \del_i\del_j \Sigma - g_{ij}\del^4\Sigma -3\del_i\del_j \ddot \Sigma}
\ea
\textbf{Vectors:}
\ba
	\delta W_{0i} &=  \frac{1}{2\Omega^2}\plr{ \del^4 \mathcal Q_i - \del^2 \ddot{\mathcal Q_i}} \\
	\delta W_{ij} & = \frac{1}{2\Omega^2}\plr{ \del^2 \del_i \dot {\mathcal Q_j}+ \del^2 \del_j\dot{\mathcal Q_i} - \del_i \ddot{\mathcal Q_j}-\del_j\ddot{\mathcal Q_i}}
\ea
\textbf{Tensors:}
\ba
	\delta W_{ij} & = \frac{1}{\Omega^2}\plr{E_{ij}-2\del^2\ddot E_{ij}+\del^4E_{ij}}
\ea
\\ \\ \\ \\
\section*{Appendix}
Useful forms of the Friedman equations required for $\delta G_{\mu\nu} = \delta T_{\mu\nu}$ :
\[
	\rho\frac{8\pi G\Omega}{3} = \hu^2
\]
\[
	8\pi G \Omega^2 \dot\rho = 6(\dot \hu \hu -\hu^3)
\]
\[
	\frac{4\pi G\Omega}{3} (\rho-3p) = \frac{\ddot \Omega}{\Omega}
\]
\[
	-(2\dot\hu + \hu^2) = 8\pi G\Omega^2p
\]
\[
	8\pi G\Omega^2 \dot p = 2\hu (2\dot\hu + \hu^2)-(2\ddot\hu+2\hu\dot\hu)
\]
%Friedmann Equations:  From the $00$ component we have
%\be
%	\pfrac{\dot \Omega}{\Omega}^2 =\Omega^2 \frac83 \pi G\rho.
%\ee
%From the trace we have
%\be
%	\frac{\ddot \Omega}{\Omega^3} = \frac43 \pi G (\rho-3p)
%\ee
%From conservation of energy $\del^\mu T_{\mu\nu} = 0$ we have
%\be
%	\dot \rho + 3\frac{\dot\Omega}{\Omega}(\rho+p) = 0
%\ee
%\textbf{Scalars:}
%\\
%\\
%\underline{$\delta G^0{}_0 = \delta T^0{}_0$:}
%\be
%	 \del^2 \psi - 3H(\dot \psi + H\phi) + H\del^2(\dot E-B) = 4\pi G \Omega^2 \delta \rho 
%\ee
%If we use the Freidman (background) equation $G_{00} = -8\pi G T_{00}$, which implies
%\[
%	3H^2 = 8\pi G \rho \Omega^2
%\]
%then we may express (34) in terms of gauge invariant variables
%\be
%	\boxed{-\del^2 \Psi + 3H(\dot\Psi + H\Phi) = -4\pi G\Omega^2 \delta \rho_\sigma}
%\ee
%\\
%\underline{$\delta G^0{}_i = \delta T^0{}_i$:}
%\\ \\
%From the Mathematica result, we get
%\be
%	\del_i ( \dot \psi + H\phi) = -4\pi G (\rho+p)\del_i(v+B).
%\ee
%Ellis drops the $\del_i$ common to both sides, though it seems we may add an arbitrary function of time. Ellis's result is then
%\be
%	\dot \psi + H\phi = -4\pi G \Omega^2 (\rho+p)(v+B).
%\ee
%If we use the Freidman trace equation for the background, which implies
%\[
%	3\frac{\ddot \Omega}{\Omega} = 4\pi G(\rho+3p),
%\]
%then we can express (37) as the gauge invariant equation:
%\be
%	\boxed{\dot \Psi + H \Phi =-4\pi G \Omega^2(\rho+p)\mathcal V}
%\ee
%\\
%\underline{$\delta G^i{}_j = \delta T^i{}_j\quad i\ne j$:}
%\\ \\
%From the Mathematica result, we get
%\be
%	\del_i \del_j\blr{ (\ddot E-\dot B)+2H(\dot E-B)-\phi+\psi} = 8\pi G\Omega^2 \del_i \del_j \Pi.
%\ee
%Ellis again drops the $\del_i \del_j$ to obtain
%\be
%	 (\ddot E-\dot B)+2H(\dot E-B)-\phi+\psi= 8\pi G\Omega^2 \Pi.
%\ee
%This one may be expressed easily in gauge invariant form
%\be
%	\boxed{\Psi - \Phi = 8\pi G\Omega^2 \Pi}
%\ee
%\\
%\underline{$\delta^j{}_i\delta G^i{}_j = \delta^j{}_i\delta T^i{}_j$:}
%\\ \\
%From the Mathematica result of the spatial trace, we get
%\be
%	\ddot \psi + H(\dot\phi + 2\dot\psi) +(2\dot H+H^2)\phi +\frac13\del^2[\phi - \psi +\dot B- \ddot E +2H(B-\dot E)] = \frac43\pi G \Omega^2 \delta p.
%\ee
%Substituting the Laplacian of (40) into the above, we get 
%\be
%	\ddot \psi + H(\dot\phi + 2\dot\psi) +(2\dot H+H^2)\phi = 4\pi G\Omega^2\plr{ \delta p + \frac23 \del^2\Pi}.
%\ee
%The gauge invariant form given in Ellis for the spatial trace needs further inspection.
%\\ \\ \\
%\textbf{Vectors:}
%\\
%\\
%\underline{$\delta G^0{}_i = \delta T^0{}_i$:}\\
%From the Mathematica result, we get
%\be
%	\del^2(B_i+\dot E_i) = -16\pi G \Omega^2(\rho+p)(v_i-B_i)
%\ee 
%which is easily expressed in gauge invariant form
%\be
%	\boxed{ \del^2 Q_i = -16\pi G\Omega^2 q_i}
%\ee
%\\
%\\
%\underline{$\delta G^i{}_j = \delta T^i{}_j\quad i\ne j$:}\\
%From the Mathematica result, we get
%\be
%	\del_i (\dot B_j +\ddot E_j)+\del_j (\dot B_i+\ddot E_i)+2H\del_i(B_j+\dot E_j)+2H\del_j(B_i+\dot E_i)=8\pi G \Omega^2 (\del_i \Pi_j+\del_j \Pi_i).
%\ee
%Ellis equates the $\del_i$ and $\del_j$ quantities with each other as in
%\[
%	\dot B_i+\ddot E_i + 2H B_i+\dot E_i = 8\pi G\Omega^2 \Pi_i
%\]
%in which the gauge invariance manifests as
%\be
%	\boxed{\dot Q_i + 2HQ_i = 8\pi G \Omega^2 \Pi_i}.
%\ee
%\\ \\ \\
%\textbf{Tensors:}
%\\
%\\
%\underline{$\delta G^i{}_j = \delta T^i{}_j$:}\\
%From the Mathematica result, we get a result that is already gauge invariant
%\be
%	\boxed{2H\dot E_{ij} -\Box E_{ij} = 8\pi G \Omega^2 \Pi_{ij}}
%\ee
\end{document}