\documentclass[10pt,letterpaper]{article}
\usepackage{mymacros}

\title{SVT Decomposition of $\delta W_{\mu\nu}$}
\author{}
\date{}

\begin{document}
\maketitle
Under conformal transformation $g_{\mu\nu} \to \bar{g}_{\mu\nu} = \Omega^2 g_{\mu\nu}$, $W_{\mu\nu}$ transforms as
\[
	\bar W_{\mu\nu}(\bar g_{\mu\nu}) =  \Omega^{-2}W_{\mu\nu}(g_{\mu\nu}).
\]
Perturbing the metric, 
\[
	\bar g_{\mu\nu} = \bar g^{(0)}_{\mu\nu} + \bar h_{\mu\nu} = \Omega^2 g^{(0)}_{\mu\nu} + \Omega^2 h_{\mu\nu}
\]
it follows that to first order
\be
	\delta \bar W_{\mu\nu}(\bar h_{\mu\nu}) = \Omega^{-2} \delta W_{\mu\nu}(h_{\mu\nu}).
\ee
Under an infinitesimal oordinate transformation $x^\mu \to x'^\mu = x^\mu + \ep^\mu(x)$, the perturbed tensor $\delta W_{\mu\nu}$ transforms as
\[
	\delta W_{\mu\nu}(h_{\mu\nu}) \to \delta W'_{\mu\nu}(h'_{\mu\nu}) = \delta W_{\mu\nu}(h_{\mu\nu}) - \delta W_{\mu\nu}(\ep_{\mu;\nu}+\ep_{\nu;\mu})
\]
At the same time, we also consider the transformation of the entire $W_{\mu\nu}$ under the infinitesimal coordinate transformation
\be
	W_{\mu\nu} \to W_{\mu\nu}' = W_{\mu\nu} + \mathcal{L}_e( W_{\mu\nu})
\ee
where the Lie derivative $\mathcal L_e$ for the rank 2 tensor is
\[
	 \mathcal{L}_e( W_{\mu\nu}) = W^{\lambda}{}_\mu \ep_{\lambda;\nu} + W^{\lambda}{}_\nu \ep_{\lambda;\mu} + W_{\mu\nu;\lambda}\ep^\lambda.
\]
Defining $\delta W_{\mu\nu}(\ep_{\mu;\nu}+\ep_{\nu;\mu}) \equiv \delta W_{\mu\nu}(\ep)$, if we expand eq (2) to first order, we conclude that
\[
	 \delta W_{\mu\nu}(\ep) =  W^{\lambda}{}_\mu \ep_{\lambda;\nu} + W^{\lambda}_\nu \ep_{\lambda;\mu} + W_{\mu\nu;\lambda}\ep^\lambda.
\]
Hence, in any background that is conformal to flat, the Lie derivative vanishes and thus $\delta W_{\mu\nu}$ must be gauge invariant. As such, it must always be possible to express $\delta W_{\mu\nu}$ in terms of 5 gauge invariant quantities (10 symmetric components - 4 coordinate transformation - 1 traceless condition = 5). This is shown below. Alternatively, we may also fix the gauge, as we have done to make $\delta W_{\mu\nu}$ diagonal in its indicies. 
\\ \\
Now decomposing $h_{\mu\nu}$ according to 
\[
	ds^2 = \Omega^2\clr{ -(1+2\phi)d\tau^2 + (\pd_i B-B_i)dx^id\tau + [(1-2\psi)\delta_{ij}+2\pd_i\pd_j E + \pd_i E_j+\pd_j E_i + 2E_{ij}]dx^i dx^j}
\]
(same as Ellis when tensor mode $E_{ij}$ is doubled), we have in flat space $\delta W_{\mu\nu}(h_{\mu\nu})$:
\figg[width=180mm]{svt_pic_1.pdf}
\noindent According to eq. (1), we may find $\delta W_{\mu\nu}$ based on a conformal to flat background by simply multiplying the above by a factor of $\Omega^{-2}$. 
\\ \\
The gauge invariant SVT quantities  in the RW $K=0$ space are
\be
	\bar \phi = \phi - \frac{\dot \Omega}{\Omega}(\dot E-B) - (\ddot E - \dot B)
\ee
\be
	\bar \psi = \psi + \frac{\dot \Omega}{\Omega}(\dot E - B)
\ee
\be
	F_i = \dot E_i + B_i
\ee
\be
	E_{ij} = E_{ij}.
\ee
In the flat space all $\dot \Omega$ gauge quantities vanish and we see immediately $\delta W_{\mu\nu}$ can be expressed solely in terms of $\bar \phi, \bar \psi, F_i,$ and $E_{ij}$. In fact, we see that $\psi$ and $\phi$ are on equal footing everywhere, and thus we may combine the two invariant scalars
\[
	\bar \xi = \bar \phi + \bar \psi =  \phi + \psi - (\ddot E - \dot B).
\]
so that we are now exactly at 5 independent degrees of freedom. Now, in the conformal to flat background, we must use the gauge invariant quantities eq. (3-6). However we note that the quantity $\bar \xi$ remains unchanged, as the $\dot \Omega$ terms cancel. 
We see now that even in a conformal to flat background, we are able to retain the same gauge invariant quantities whilst preseving the conformal symmetry, i.e. 
\[
	\delta \bar W_{\mu\nu}(\bar h_{\mu\nu}) = \Omega^{-2} \delta W_{\mu\nu}(h_{\mu\nu}).
\]
\end{document}