\documentclass[10pt,letterpaper]{article}
\usepackage{macroshw}

\title{Weinberg Gravitation Notes}
\author{}
\date{}

\begin{document}
\maketitle
\section{Special Relativity}
Principle of special relativity states that the laws of nature must be invariant under Lorentz transformations. A Lorentz
transformation is such a transformation that leaves the spacetime interval invariant:
\[
	d\tau^2 = -\eta_{\alpha\beta}dx^\alpha dx^\beta
\]
If we perform a general coordinate transformation,
we can in fact show (2.1.8) that the transformation must be linear 
\[
	x'^\alpha = \Lambda^\alpha{}_\beta x^\beta + a^\alpha
\]
and satisfies
\[
	\Lambda^\alpha{}_\gamma \Lambda^\beta{}_\delta \eta_{\alpha\beta} = \eta_{\gamma\delta}.
\]
(2.6.5) $ \det \Lambda = 1$. Assume this arises from delta function $\delta(ax) = \frac{\delta(x)}{|a|}$, as in magnitude of 
$\Lambda$. Not sure. 
\\ \\
(2.6.7) Either must be a closed system such that the current flux through the surface is zero, or must be integrated
over all space such that the current flux vanishes at the (spatial) infinite boundary.
\\ \\
(2.9.5) Note that 
\ba
	\ep_{\alpha \beta \gamma \delta}(a^\beta p^\gamma p^\delta - a^\gamma p^\beta p^\delta) &=
	2\ep_{\alpha \beta \gamma \delta}a^\beta p^\gamma p^\delta\\
	& = -2\ep_{\alpha \beta \delta \gamma}(a^\beta p^\delta p^\gamma) \\
	&\Rightarrow 0.
\ea
It seems that given a fully antisymmetric tensor $\ep$ that is contracted with a tensor $R$ in which any two indices are
symmetric, then it follows that the contraction of $\ep R = 0$. 
\\ \\
\[
	(\pd^\alpha)( \pd_\alpha - ieA_\alpha(x))\psi(x) = \frac{\pd^2}{\pd^2x^\alpha} \psi(x) - ie\pdiff[A_\alpha]{x^\alpha}\psi(x)
	-ieA_\alpha\pdiff{\psi(x)}{x^\alpha}
\]
\[
	\frac{\pd^2 x'^\tau}{\pd x^\mu \pd x^\nu} = \pdiff[x'^\tau]{x^\lambda}\Gamma^{\lambda}_{\mu \nu}
	- \pdiff[x'^\rho]{x^\mu}\pdiff[x'^\sigma]{x^\nu}\Gamma'^\tau_{\rho\sigma}
\]
\[
	\frac{\pd^2 x'^\tau}{\pd x^\kappa \pd x^\lambda} = \pdiff[x'^\tau]{x^\eta}\Gamma^{\eta}_{\kappa \lambda}
	- \pdiff[x'^\rho]{x^\kappa}\pdiff[x'^\sigma]{x^\lambda}\Gamma'^\tau_{\rho\sigma}
\]
\[
	\frac{\pd^2 x'^\tau}{\pd x^\kappa \pd x^\lambda}\Gamma^\lambda_{\mu\nu} = \Gamma^\lambda_{\mu\nu} 
	\plr{\pdiff[x'^\tau]{x^\eta}\Gamma^{\eta}_{\kappa \lambda}
	- \pdiff[x'^\rho]{x^\kappa}\pdiff[x'^\sigma]{x^\lambda}\Gamma'^\tau_{\rho\sigma}}
\]
\[
	-\frac{\pd^2 x'^\sigma}{\pd x^\kappa \pd x^\nu}\frac{\pd x'^\rho}{\pd x^\mu}
	\Gamma'^\tau_{\rho\sigma} = -\frac{\pd x'^\rho}{\pd x^\mu}
	\Gamma'^\tau_{\rho\sigma}
	\plr{\pdiff[x'^\sigma]{x^\eta}\Gamma^{\eta}_{\kappa \nu}
	- \pdiff[x'^\eta]{x^\kappa}\pdiff[x'^\xi]{x^\nu}\Gamma'^\sigma_{\eta\xi}}
\]
We seek to find some sort of field equation to describe gravitation. Gravitation should be uniquely defined by something,
and so we are looking for some quantity that includes gravitation and is valid in all ref. frames. Clearly this must be 
a tensor. So let us form a tensor with $g_{\mu \nu}$ and its first derivatives, i.e. $\Gamma^\lambda_{\mu\nu}$. Well,
in our local inertial frame, $\Gamma^\lambda_{\mu\nu} = 0$ and thus
\[
	T \propto g_{\mu\nu}.
\]
But under general coordinate transformations, $g_{\mu\nu}$ transforms as a tensor and thus this equality is true
in all coordinate systems. Rather, we need to go up a power and look at second derivatives. 
\\ \\
Starting with
\[
	\frac{DA^\mu}{d\tau} = \frac{dA^\mu}{d\tau} + \Gamma^\mu_{\nu\lambda}\frac{dx^\lambda}{d\tau}A^\nu
\]
we form the second covariant derivative along a curve $x(\tau)$
\[
	\frac{ D^2 A^\mu}{d\tau^2} = \frac{d^2 A^\mu}{d\tau^2} + 2\Gamma^\mu_{\lambda\nu}\frac{dx^\lambda}{d\tau}
	\frac{dA^\nu}{d\tau}+\frac{\pd\Gamma^\mu_{\lambda\nu}}{\pd x^\rho}\frac{dx^\rho}{d\tau}\frac{dx^\lambda}{d\tau}
	A^\nu + \Gamma^\mu_{\rho\sigma}\Gamma^\rho_{\lambda\nu} \frac{dx^\sigma}{d\tau}\frac{dx^\lambda}{d\tau}
	A^\lambda
\]
	


\end{document}