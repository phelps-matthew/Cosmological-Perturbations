\documentclass[8pt]{beamer}
\usepackage{units,multicol,booktabs,tikz,xparse,tikz-3dplot,animate,mathtools}
\usetikzlibrary{arrows.meta,calc,shapes,angles,quotes,patterns}

\graphicspath{{Images/}}

%Display framenumber in footer
\newcommand*\oldmacro{}%
\let\oldmacro\insertshorttitle%
\renewcommand*\insertshorttitle{%
  \oldmacro\hfill%
  \insertframenumber\,/\,\inserttotalframenumber}

%Choose colors for themes
\definecolor{myblue}{rgb}{0.2,0.2,0.7} %Default Warsaw
\definecolor{myred}{RGB}{165,8,8}
\definecolor{myviolet}{RGB}{100,8,165}
\definecolor{mygreen}{RGB}{0,132,15}

%Tikz colors
\definecolor{bluegrid}{RGB}{197,207,224}

% Initialize default color theme
\setbeamercolor{author in head/foot}{bg=black} 		%1501 foot
\setbeamercolor{subsection in head/foot}{bg=myblue} 	% lecture title foot
\setbeamercolor{section in head/foot}{bg=black} 		% frametitle gradient
\setbeamercolor{frametitle}{bg=myblue} 				%frametitle primary color
\setbeamercolor{local structure}{fg=black}				%Necessary for enumerate
\setbeamercolor{item projected}{bg=myblue}			%itemize bullets

% iClicker enumerate box drawing
\newcommand{\tikzmark}[1]{\tikz[overlay,remember picture] \node (#1) {};}

% Macros
\newcommand{\vect}[1]{\mathbf{#1}}
\newcommand{\vecth}[1]{\hat{\mathbf{#1}}}
\def\flb#1\fle{\begin{flalign*}#1\end{flalign*}} % begin / end full left align

% More Stuff
\usefonttheme[onlymath]{serif}
\setbeamercovered{invisible}
\hfuzz=60pt %Surpresses hbox full warnings by extending to 50pt
\newdimen\hfuzz
\vbadness = 10000 %Similar to hbox warning supression

% Mode
\mode<presentation>{\usetheme{Warsaw}}


% Generate Credentials
\title[Cosmological Fluctuations]{Cosmological Fluctuations in Standard and Conformal Gravity}
\author[Matthew Phelps]{Matthew Phelps}
\institute{\normalsize{Doctoral Degree Final Examination}
\and \includegraphics[width=0.8 in]{uconn-logo.png} }
\date{June 02, 2020}


%%%%%%%%%%%%%%%%%%%%%%%%%%%%%%%%%%%%%%%%%%%%%%%%%%%%%%%%%%%%%%
% Begin Document
%%%%%%%%%%%%%%%%%%%%%%%%%%%%%%%%%%%%%%%%%%%%%%%%%%%%%%%%%%%%%%
%\includeonlyframes{current}
\begin{document}
\beamertemplatenavigationsymbolsempty %Remove navigation icons
\setbeamertemplate{enumerate item}{(\Alph{enumi})} %Set default enumeration style


%%%%%%%%%%%%%%%%%%%%%%%%%%%%%%%%%%%%%%%%%%%%%%%%%%%%%%%%%%%%%%
% Slide  - Title Page
%%%%%%%%%%%%%%%%%%%%%%%%%%%%%%%%%%%%%%%%%%%%%%%%%%%%%%%%%%%%%%

\begingroup
\Large
\begin{frame}
\setcounter{framenumber}{0}
	\titlepage
\end{frame}
\endgroup

%%%%%%%%%%%%%%%%%%%%%%%%%%%%%%%%%%%%%%%%%%%%%%%%%%%%%%%%%%%%%%
% Slide  - TOC
%%%%%%%%%%%%%%%%%%%%%%%%%%%%%%%%%%%%%%%%%%%%%%%%%%%%%%%%%%%%%%

\begingroup
\Large
\begin{frame}{Overview}
	\begin{itemize}
		\item Introduction and Formalism
		\item Conformal Gravity
	\end{itemize}
\end{frame}
\endgroup

%%%%%%%%%%%%%%%%%%%%%%%%%%%%%%%%%%%%%%%%%%%%%%%%%%%%%%%%%%%%%%
% Slide  - Introduction and Formalism
%%%%%%%%%%%%%%%%%%%%%%%%%%%%%%%%%%%%%%%%%%%%%%%%%%%%%%%%%%%%%%

\begingroup
\Large
\begin{frame}{Overview}
	\begin{itemize}
		\item Introduction and Formalism
		\begin{itemize}
			%\large
			\item Cosmological Geometries
			\item Einstein Gravity
			\item Perturbation Theory
			\item Gauge Transformations
		\end{itemize}
	\end{itemize}
\end{frame}
\endgroup

%%%%%%%%%%%%%%%%%%%%%%%%%%%%%%%%%%%%%%%%%%%%%%%%%%%%%%%%%%%%%%
% Slide  - Cosmological Geometries 
%%%%%%%%%%%%%%%%%%%%%%%%%%%%%%%%%%%%%%%%%%%%%%%%%%%%%%%%%%%%%%

\begin{frame}{Cosmological Geometries}
	\begin{itemize}
		\item Cosmological Principle: Structure of spacetime is homoegenous and isotropic at large scales
		\item Geometries: Robertson Walker (flat, spherical, hyperbolic), de Sitter ($dS_4 \subset \rm{RW}$)
		\item All background geometries relevant to cosmology can be expressed as conformal to flat
	\end{itemize}
	\begin{eqnarray*}
		ds^2 = \Omega(x)^2\big(-dt^2 + dx^2 + dy^2 + dz^2\big)
	\end{eqnarray*}
\end{frame}

%%%%%%%%%%%%%%%%%%%%%%%%%%%%%%%%%%%%%%%%%%%%%%%%%%%%%%%%%%%%%%
% Slide  - Cosmological Geometries R.W. k=0
%%%%%%%%%%%%%%%%%%%%%%%%%%%%%%%%%%%%%%%%%%%%%%%%%%%%%%%%%%%%%%

\begin{frame}{Cosmological Geometries R.W.}
	Comoving Robertson Walker geometry:
	\begin{eqnarray*}
	ds^2 &=& -dt^2 + a(t)^2 \tilde g_{ij}dx^i dx^j
	\nonumber\\
	&=& -dt^2 + a(t)^2\bigg[\frac{dr^2}{1-kr^2} + r^2 d\theta^2 + r^2 \sin^2\theta d\phi^2\bigg]
	\end{eqnarray*}
	3-Space Curvature Tensors,
	\begin{eqnarray*}
	R_{ijkl} = k(\tilde g_{jk}\tilde g_{il} - \tilde g_{ik}\tilde g_{jl}), \qquad R_{ij} = -3k\tilde g_{ij}, \qquad R = -6k
	\end{eqnarray*}
	with $k \in \{-1,0,1\}$. Define the conformal time
	\begin{eqnarray*}
		\tau = \int \frac{dt}{a(t)},
	\end{eqnarray*}
	\only<1>{
	\begin{eqnarray*}
		ds^2 = a(\tau)^2\bigg[-d\tau^2 + \frac{dr^2}{1-kr^2} + r^2 d\theta^2 + r^2 \sin^2\theta d\phi^2\bigg]
	\end{eqnarray*}
	}
	\only<2>{
	set $k=0$ (flat), simple conformal to flat form
	\begin{eqnarray*}
		ds^2 = a(\tau)^2\bigg[-d\tau^2 + dr^2 + r^2 d\theta^2 + r^2 \sin^2\theta d\phi^2\bigg]
	\end{eqnarray*}		
	}
\end{frame}

%%%%%%%%%%%%%%%%%%%%%%%%%%%%%%%%%%%%%%%%%%%%%%%%%%%%%%%%%%%%%%
% Slide  - Cosmological Geometries R.W. k=1
%%%%%%%%%%%%%%%%%%%%%%%%%%%%%%%%%%%%%%%%%%%%%%%%%%%%%%%%%%%%%%

\begin{frame}{Cosmological Geometries R.W. $k=1$}
	$k=1$ (spherical)
	\begin{eqnarray*}
		ds^2 = a(\tau)^2\bigg[-d\tau^2 + \frac{dr^2}{1-r^2} + r^2 d\theta^2 + r^2 \sin^2\theta d\phi^2\bigg]
	\end{eqnarray*}	
	Set $\sin\chi = r$, $p = \tau$, 
	\begin{eqnarray*}
		ds^2 = a(p)^2\bigg[-dp^2 + d\chi^2 + \sin^2\chi d\theta^2 + \sin^2\chi \sin^2\theta d\phi^2\bigg]
	\end{eqnarray*}
	Introduce coordinates
	\begin{eqnarray*}
			p' + r' &=& \tan[(p+\chi)/2],\quad p'-r'=\tan[(p-\chi)/2]
			\nonumber\\
			p' &=& \frac{\sin p}{\cos p + \cos \chi}, \quad r' = \frac{\sin\chi}{\cos p + \cos\chi}
	\end{eqnarray*}
	\begin{eqnarray*}
	\implies ds^2 = \frac{4a^2(p)}{[1+(p'+r')^2][1+(p'-r')^2]}[-dp'^2 + dr'^2 +r'^2 d\theta^2 + r'^2\sin^2\theta d\phi^2]
	\end{eqnarray*}
\end{frame}

%%%%%%%%%%%%%%%%%%%%%%%%%%%%%%%%%%%%%%%%%%%%%%%%%%%%%%%%%%%%%%
% Slide  - Cosmological Geometries R.W. k=-1
%%%%%%%%%%%%%%%%%%%%%%%%%%%%%%%%%%%%%%%%%%%%%%%%%%%%%%%%%%%%%%

\begin{frame}{Cosmological Geometries R.W. $k=-1$}
	$k=-1$ (hyperbolic)
	\begin{eqnarray*}
		ds^2 = a(\tau)^2\bigg[-d\tau^2 + \frac{dr^2}{1+r^2}  + r^2 d\theta^2 + r^2 \sin^2\theta d\phi^2\bigg]
	\end{eqnarray*}	
	Set $\sin\chi = r$, $p = \tau$, 
	\begin{eqnarray*}
		ds^2 = a(p)^2\bigg[-dp^2 + d\chi^2 + \sin^2\chi d\theta^2 + \sin^2\chi \sin^2\theta d\phi^2\bigg]
	\end{eqnarray*}
	Introduce coordinates
	\begin{eqnarray*}
		p' + r' &=& \tanh[(p+\chi)/2],\quad p'-r'=\tanh[(p-\chi)/2]
		\nonumber\\
		p' &=& \frac{\sinh p}{\cosh p + \cosh \chi}, \quad r' = \frac{\sinh\chi}{\cosh p + \cosh\chi}
	\end{eqnarray*}
	\begin{eqnarray*}
	\implies ds^2 = \frac{4a^2(p)}{[1-(p'+r')^2][1-(p'-r')^2]}[-dp'^2 + dr'^2 +r'^2 d\theta^2 + r'^2\sin^2\theta d\phi^2]
	\end{eqnarray*}
\end{frame}

%%%%%%%%%%%%%%%%%%%%%%%%%%%%%%%%%%%%%%%%%%%%%%%%%%%%%%%%%%%%%%
% Slide  - Einstein Gravity
%%%%%%%%%%%%%%%%%%%%%%%%%%%%%%%%%%%%%%%%%%%%%%%%%%%%%%%%%%%%%%

\begin{frame}{Einstein Gravity}
	Einstein Hilbert action
	\begin{eqnarray*}
	I_{\text{EH}} = -\frac{1}{16\pi G} \int d^4x (-g)^{1/2}  g^{\mu\nu}R_{\mu\nu}.
	\end{eqnarray*}
	Functional variation w.r.t $g_{\mu\nu}$ yields Einstein tensor,
	\begin{eqnarray*}
	\frac{16\pi G}{(-g)^{1/2}} \frac{\delta I_{\text{EH}}}{\delta g_{\mu\nu}}= G^{\mu\nu} = R^{\mu\nu} - \frac{1}{2}g^{\mu\nu}R^\alpha{}_\alpha,
	\end{eqnarray*}
	likewise, variation of matter action $I_{\rm M}$ w.r.t $g_{\mu\nu}$ yields Energy Momentum tensor
	\begin{eqnarray*}
	\frac{2}{(-g)^{1/2}} \frac{ \delta I_\text{M}}{\delta g_{\mu\nu}} = T_{\mu\nu}. 
	\end{eqnarray*}
	Requiring sum of actions to be stationary gives us Einstein field equations
	\begin{eqnarray*}
	R^{\mu\nu} - \frac{1}{2}g^{\mu\nu}R^\alpha{}_\alpha = -8\pi G T^{\mu\nu},
	\label{EinEOM}
	\end{eqnarray*}
	subject to Bianchi identity
	\begin{eqnarray*}
	\nabla_\mu R^{\mu\nu} = \frac{1}{2}\nabla^\nu R^\mu{}_\mu \implies \nabla_\mu G^{\mu\nu} = 0.
	\end{eqnarray*}
\end{frame}

%%%%%%%%%%%%%%%%%%%%%%%%%%%%%%%%%%%%%%%%%%%%%%%%%%%%%%%%%%%%%%
% Slide  - Perturbation Theory
%%%%%%%%%%%%%%%%%%%%%%%%%%%%%%%%%%%%%%%%%%%%%%%%%%%%%%%%%%%%%%

\begin{frame}{Cosmological Perturbation Theory}
	Decompose metric into background and fluctuation, truncating at linear order
	\begin{columns}
		\begin{column}{0.7\linewidth}
			\begin{eqnarray*}
				g_{\mu\nu}(x) &=& g_{\mu\nu}^{(0)}(x) + h_{\mu\nu}(x),\qquad g^{\mu\nu}_{(0)}h_{\mu\nu} \equiv h
				\\ \\
				G_{\mu\nu} &=& G_{\mu\nu}(g_{\mu\nu}^{(0)}) + \delta G_{\mu\nu}(h_{\mu\nu})
				\\ \\
				G_{\mu\nu}^{(0)} &=& R_{\mu\nu}^{(0)} -\frac{1}{2} g_{\mu\nu}^{(0)} R_\alpha^{(0)\alpha}
				\label{Einzero}
				\\ \\
				\delta G_{\mu\nu} &=& \delta R_{\mu\nu} - \frac{1}{2} h_{\mu\nu} R_\alpha^{(0)\alpha} -\frac{1}{2}g_{\mu\nu}\delta R^\alpha{}_\alpha.
			\end{eqnarray*}
			\hspace{.04\linewidth}Likewise perturb $T_{\mu\nu}$ around background
			\begin{eqnarray*}
				T_{\mu\nu} &=& T_{\mu\nu}(g_{\mu\nu}^{(0)}) + \delta T_{\mu\nu}(h_{\mu\nu})
			\end{eqnarray*}
		\end{column}
		\begin{column}{0.3\linewidth}
			\begin{figure}[t]
				\includegraphics[width=\linewidth]{sphere_perturb.png}
			\end{figure}
			\footnotemark
		\end{column}
	\end{columns}
	\vspace{1em}
	Form background and first order equations of motion (upon setting $8\pi G=1$)
	\begin{eqnarray*}
		\Delta_{\mu\nu}^{0} &=& G_{\mu\nu}^{(0)} + T_{\mu\nu}^{(0)} =0
		\\
		\Delta_{\mu\nu} &=& \delta G_{\mu\nu}^{(0)} + \delta T_{\mu\nu}^{(0)}=0
	\end{eqnarray*}
	\footnotetext{Walter, U. (2019). Correction to: Astronautics. In Astronautics (pp. C1–C1). Springer International Publishing.}
\end{frame}

%%%%%%%%%%%%%%%%%%%%%%%%%%%%%%%%%%%%%%%%%%%%%%%%%%%%%%%%%%%%%%
% Slide  - Gauge Transformations
%%%%%%%%%%%%%%%%%%%%%%%%%%%%%%%%%%%%%%%%%%%%%%%%%%%%%%%%%%%%%%

\begin{frame}{Gauge Transformations}
	\begin{itemize}
		\item 	Under coordinate transformation $x^\mu \to x^\mu - \epsilon^\mu(x)$, with $\epsilon^\mu \sim \mathcal O(h)$, the peturbed metric transforms as
			\begin{eqnarray*}
				h_{\mu\nu} \to h_{\mu\nu} + \nabla_\mu \epsilon_\nu + \nabla_\nu \epsilon_\mu
			\end{eqnarray*}
		\item 	For every solution $h_{\mu\nu}$ to $\delta G_{\mu\nu} + \delta T_{\mu\nu} = 0$, a transformed $h'_{\mu\nu} = h_{\mu\nu} + \nabla_\mu \epsilon_\nu + \nabla_\nu \epsilon_\mu$ will also serve as a solution
		\item Set of four $\epsilon^\mu(x)$ define gauge freedom under coordinate transformation
		\item 10 components in $h_{\mu\nu}$, 4 coordinate transformations, leads to 6 independent degrees of freedom
		\item Under $x^\mu \to x^\mu - \epsilon^\mu(x)$, the perturbed tensors transform as
			\begin{eqnarray*}
				\delta G_{\mu\nu} \to \delta G_{\mu\nu} + {}^{(0)}G^\lambda{}_\mu \nabla_\nu \epsilon_\lambda +  {}^{(0)}G^{\lambda}{}_{\nu}\nabla_\mu \epsilon_\mu + \nabla_\lambda  G^{(0)}_{\mu\nu} \epsilon^\lambda
				\nonumber\\
				\delta T_{\mu\nu} \to \delta T_{\mu\nu} + {}^{(0)}T^\lambda{}_\mu \nabla_\nu \epsilon_\lambda +  {}^{(0)}T^{\lambda}{}_{\nu}\nabla_\mu \epsilon_\mu + \nabla_\lambda  T^{(0)}_{\mu\nu} \epsilon^\lambda.
			\end{eqnarray*}
		\item If background $G_{\mu\nu}^{(0)} = 0$ , then $\delta G_{\mu\nu}$ separately gauge invariant; likewise for vanishing background energy momentum tensor
		\item If $G_{\mu\nu}^{(0)} \ne 0$, then only the entire $\Delta_{\mu\nu} = \delta G_{\mu\nu} + T_{\mu\nu}$ gauge invariant
	\end{itemize}


\end{frame}

%%%%%%%%%%%%%%%%%%%%%%%%%%%%%%%%%%%%%%%%%%%%%%%%%%%%%%%%%%%%%%
% Slide  - Solving Equations of Motion
%%%%%%%%%%%%%%%%%%%%%%%%%%%%%%%%%%%%%%%%%%%%%%%%%%%%%%%%%%%%%%

\begin{frame}{}
	Perturbed equations of motion $\delta G_{\mu\nu} + \delta T_{\mu\nu} = 0$ form a rather complex and extensive set of coupled non-linear tensor PDE's
	\\ \vspace{2mm}
	Much effort involved in simplifying, decoupling, and solving them
	\\ \vspace{2mm}
	Two main approaches
	\begin{itemize}
		\item Fix the gauge by constraining $h_{\mu\nu}$, e.g. transverse gauge $\nabla^\mu h_{\mu\nu} = 0$, then solve fluctuation equations directly in terms of $h_{\mu\nu}$
		\begin{itemize}
			\item Simplification usually not effective in more general curved backgrounds
		\end{itemize}
		\item Decompose $h_{\mu\nu}$ into a basis of scalars, vectors, and tensors, express in terms of gauge invariant combinations, and solve fluctuation equations with possible decoupling between modes 
		\begin{itemize}
			\item SVT Decomposition, de facto approach in modern cosmology
		\end{itemize}
	\end{itemize}
	\tiny
	\begin{eqnarray}
		\delta G_{ij}&=& - \tfrac{1}{2} \overset{..}{f}_{ij} + \tfrac{1}{2} \overset{..}{f}_{00}{} \tilde{g}_{ij} + \tfrac{1}{2} \overset{..}{f} \tilde{g}_{ij} -  k \tilde{g}^{ba} \tilde{g}_{ij} f_{ab} + 3 k f_{ij} -  \dot{\Omega}^2 f_{ij} \Omega^{-2} -  \dot{\Omega}^2 \tilde{g}_{ij} f_{00}{} \Omega^{-2} 
		\nonumber\\
		&& -  \dot{f}_{ij} \dot{\Omega} \Omega^{-1}  + 2 \dot{f}_{00}{} \dot{\Omega} \tilde{g}_{ij} \Omega^{-1} + \dot{f} \dot{\Omega} \tilde{g}_{ij} \Omega^{-1} + 2 \overset{..}{\Omega} f_{ij} \Omega^{-1} + 2 \overset{..}{\Omega} \tilde{g}_{ij} f_{00}{} \Omega^{-1} 
		\nonumber\\
		&& + 2 \dot{\Omega} \tilde{g}^{ba} \tilde{g}_{ij} f_{0}{}_{b} \Omega^{-2} \tilde{\nabla}_{a}\Omega  - 2 \dot{f}_{0}{}_{b} \tilde{g}^{ba} \tilde{g}_{ij} \Omega^{-1} \tilde{\nabla}_{a}\Omega -  \tilde{g}^{ba} \tilde{g}_{ij} \tilde{\nabla}_{b}\dot{f}_{0}{}_{a} 
		\nonumber\\
		&& - 4 \tilde{g}^{ba} \tilde{g}_{ij} f_{0}{}_{a} \Omega^{-1} \tilde{\nabla}_{b}\dot{\Omega} + \tilde{g}^{ba} \Omega^{-1} \tilde{\nabla}_{a}\Omega \tilde{\nabla}_{b}f_{ij} \nonumber \\ 
		&& - 2 \dot{\Omega} \tilde{g}^{ba} \tilde{g}_{ij} \Omega^{-1} \tilde{\nabla}_{b}f_{0}{}_{a} -  \tilde{g}^{ba} \tilde{g}_{ij} \Omega^{-1} \tilde{\nabla}_{a}f \tilde{\nabla}_{b}\Omega -  \tilde{g}^{ca} \tilde{g}^{db} \tilde{g}_{ij} f_{cd} \Omega^{-2} \tilde{\nabla}_{a}\Omega \tilde{\nabla}_{b}\Omega \nonumber \\ 
		&& + \tilde{g}^{ba} f_{ij} \Omega^{-2} \tilde{\nabla}_{a}\Omega \tilde{\nabla}_{b}\Omega + \tfrac{1}{2} \tilde{g}^{ba} \tilde{\nabla}_{b}\tilde{\nabla}_{a}f_{ij} -  \tfrac{1}{2} \tilde{g}^{ba} \tilde{g}_{ij} \tilde{\nabla}_{b}\tilde{\nabla}_{a}f - 2 \tilde{g}^{ba} f_{ij} \Omega^{-1} \tilde{\nabla}_{b}\tilde{\nabla}_{a}\Omega \nonumber \\ 
		&& -  \tfrac{1}{2} \tilde{g}^{ba} \tilde{\nabla}_{b}\tilde{\nabla}_{i}f_{ja} -  \tfrac{1}{2} \tilde{g}^{ba} \tilde{\nabla}_{b}\tilde{\nabla}_{j}f_{ia} + 2 \tilde{g}^{ca} \tilde{g}^{db} \tilde{g}_{ij} \Omega^{-1} \tilde{\nabla}_{a}\Omega \tilde{\nabla}_{d}f_{cb} 
		\nonumber\\
		&& + \tfrac{1}{2} \tilde{g}^{ca} \tilde{g}^{db} \tilde{g}_{ij} \tilde{\nabla}_{d}\tilde{\nabla}_{c}f_{ab}  + 2 \tilde{g}^{ca} \tilde{g}^{db} \tilde{g}_{ij} f_{ab} \Omega^{-1} \tilde{\nabla}_{d}\tilde{\nabla}_{c}\Omega + \tfrac{1}{2} \tilde{\nabla}_{i}\dot{f}_{0}{}_{j} 
		\nonumber\\
		&& -  \tilde{g}^{ba} \Omega^{-1} \tilde{\nabla}_{a}\Omega \tilde{\nabla}_{i}f_{jb} + \dot{\Omega} \Omega^{-1} \tilde{\nabla}_{i}f_{0}{}_{j} + \tfrac{1}{2} \tilde{\nabla}_{j}\dot{f}_{0}{}_{i} -  \tilde{g}^{ba} \Omega^{-1} \tilde{\nabla}_{a}\Omega \tilde{\nabla}_{j}f_{ib}
		\nonumber\\
		&&  + \dot{\Omega} \Omega^{-1} \tilde{\nabla}_{j}f_{0}{}_{i} + \tfrac{1}{2} \tilde{\nabla}_{j}\tilde{\nabla}_{i}f,
		\label{dg_conf31}
	\end{eqnarray}
\end{frame}

%%%%%%%%%%%%%%%%%%%%%%%%%%%%%%%%%%%%%%%%%%%%%%%%%%%%%%%%%%%%%%
% Slide  - 
%%%%%%%%%%%%%%%%%%%%%%%%%%%%%%%%%%%%%%%%%%%%%%%%%%%%%%%%%%%%%%

\begin{frame}{}
	
\end{frame}

%%%%%%%%%%%%%%%%%%%%%%%%%%%%%%%%%%%%%%%%%%%%%%%%%%%%%%%%%%%%%%
% Slide  - 
%%%%%%%%%%%%%%%%%%%%%%%%%%%%%%%%%%%%%%%%%%%%%%%%%%%%%%%%%%%%%%

\begin{frame}{}
	
\end{frame}

%%%%%%%%%%%%%%%%%%%%%%%%%%%%%%%%%%%%%%%%%%%%%%%%%%%%%%%%%%%%%%
% Slide  - 
%%%%%%%%%%%%%%%%%%%%%%%%%%%%%%%%%%%%%%%%%%%%%%%%%%%%%%%%%%%%%%

\begin{frame}{}
	
\end{frame}

%%%%%%%%%%%%%%%%%%%%%%%%%%%%%%%%%%%%%%%%%%%%%%%%%%%%%%%%%%%%%%
% Slide  - 
%%%%%%%%%%%%%%%%%%%%%%%%%%%%%%%%%%%%%%%%%%%%%%%%%%%%%%%%%%%%%%

\begin{frame}{}
	
\end{frame}

%%%%%%%%%%%%%%%%%%%%%%%%%%%%%%%%%%%%%%%%%%%%%%%%%%%%%%%%%%%%%%
% End Document
%%%%%%%%%%%%%%%%%%%%%%%%%%%%%%%%%%%%%%%%%%%%%%%%%%%%%%%%%%%%%%

\end{document}