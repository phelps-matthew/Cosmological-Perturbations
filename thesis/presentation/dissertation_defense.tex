\documentclass[8pt,draft]{beamer}
\usepackage{units,multicol,booktabs,tikz,xparse,tikz-3dplot,animate,mathtools}
\usetikzlibrary{arrows.meta,calc,shapes,angles,quotes,patterns}

\graphicspath{{Images/}}

%Display framenumber in footer
\newcommand*\oldmacro{}%
\let\oldmacro\insertshorttitle%
\renewcommand*\insertshorttitle{%
  \oldmacro\hfill%
  \insertframenumber\,/\,\inserttotalframenumber}

%Choose colors for themes
\definecolor{myblue}{rgb}{0.2,0.2,0.7} %Default Warsaw
\definecolor{myred}{RGB}{165,8,8}
\definecolor{myviolet}{RGB}{100,8,165}
\definecolor{mygreen}{RGB}{0,132,15}

%Tikz colors
\definecolor{bluegrid}{RGB}{197,207,224}

% Initialize default color theme
\setbeamercolor{author in head/foot}{bg=black} 		%1501 foot
\setbeamercolor{subsection in head/foot}{bg=myblue} 	% lecture title foot
\setbeamercolor{section in head/foot}{bg=black} 		% frametitle gradient
\setbeamercolor{frametitle}{bg=myblue} 				%frametitle primary color
\setbeamercolor{local structure}{fg=black}				%Necessary for enumerate
\setbeamercolor{item projected}{bg=myblue}			%itemize bullets

% iClicker enumerate box drawing
\newcommand{\tikzmark}[1]{\tikz[overlay,remember picture] \node (#1) {};}

% Macros
\newcommand{\vect}[1]{\mathbf{#1}}
\newcommand{\vecth}[1]{\hat{\mathbf{#1}}}
\def\flb#1\fle{\begin{flalign*}#1\end{flalign*}} % begin / end full left align

% More Stuff
\usefonttheme[onlymath]{serif}
\setbeamercovered{invisible}
\hfuzz=60pt %Surpresses hbox full warnings by extending to 50pt
\newdimen\hfuzz
\vbadness = 10000 %Similar to hbox warning supression

% Mode
\mode<presentation>{\usetheme{Warsaw}}


% Generate Credentials
\title[Cosmological Fluctuations]{Cosmological Fluctuations in Standard and Conformal Gravity}
\author[Matthew Phelps]{Matthew Phelps}
\institute{\normalsize{Doctoral Degree Final Examination}
\and \includegraphics[width=0.8 in]{uconn-logo.png} }
\date{June 02, 2020}


%%%%%%%%%%%%%%%%%%%%%%%%%%%%%%%%%%%%%%%%%%%%%%%%%%%%%%%%%%%%%%
% Begin Document
%%%%%%%%%%%%%%%%%%%%%%%%%%%%%%%%%%%%%%%%%%%%%%%%%%%%%%%%%%%%%%
%\includeonlyframes{current}
\begin{document}
\beamertemplatenavigationsymbolsempty %Remove navigation icons
\setbeamertemplate{enumerate item}{(\Alph{enumi})} %Set default enumeration style

%%%%%%%%%%%%%%%%%%%%%%%%%%%%%%%%%%%%%%%%%%%%%%%%%%%%%%%%%%%%%%
% Slide 1 - Title Page
%%%%%%%%%%%%%%%%%%%%%%%%%%%%%%%%%%%%%%%%%%%%%%%%%%%%%%%%%%%%%%

\begingroup
\Large
\begin{frame}
\setcounter{framenumber}{0}
	\titlepage
\end{frame}
\endgroup

%%%%%%%%%%%%%%%%%%%%%%%%%%%%%%%%%%%%%%%%%%%%%%%%%%%%%%%%%%%%%%
% Slide 2 - TOC
%%%%%%%%%%%%%%%%%%%%%%%%%%%%%%%%%%%%%%%%%%%%%%%%%%%%%%%%%%%%%%

\begingroup
\Large
\begin{frame}{Overview}
	\begin{itemize}
		\item Introduction and Formalism
		\begin{itemize}
		\normalsize
			\item Cosmological Geometries
			\item Einstein Gravity
			\item Perturbation Theory
			\item Gauge Transformations
		\end{itemize}
		\item Conformal Gravity
	\end{itemize}
\end{frame}
\endgroup

%%%%%%%%%%%%%%%%%%%%%%%%%%%%%%%%%%%%%%%%%%%%%%%%%%%%%%%%%%%%%%
% Slide 3 - Cosmological Geometries 
%%%%%%%%%%%%%%%%%%%%%%%%%%%%%%%%%%%%%%%%%%%%%%%%%%%%%%%%%%%%%%

\begin{frame}{Cosmological Geometries}
	\begin{itemize}
		\item Cosmological Principle: Structure of spacetime is homoegenous and isotropic at large scales
		\item Geometries: Robertson Walker (flat, spherical, hyperbolic), de Sitter
		\item All background geometries relevant to cosmology can be expressed as conformal to flat
	\end{itemize}
	\begin{eqnarray*}
		ds^2 = \Omega(x)^2\big(-dt^2 + dx^2 + dy^2 + dz^2\big)
	\end{eqnarray*}
\end{frame}

%%%%%%%%%%%%%%%%%%%%%%%%%%%%%%%%%%%%%%%%%%%%%%%%%%%%%%%%%%%%%%
% Slide 4 - Cosmological Geometries R.W. k=0
%%%%%%%%%%%%%%%%%%%%%%%%%%%%%%%%%%%%%%%%%%%%%%%%%%%%%%%%%%%%%%

\begin{frame}{Cosmological Geometries R.W.}
	Comoving Robertson Walker geometry:
	\begin{eqnarray*}
	ds^2 &=& -dt^2 + a(t)^2 \tilde g_{ij}dx^i dx^j
	\nonumber\\
	&=& -dt^2 + a(t)^2\bigg[\frac{dr^2}{1-kr^2} + r^2 d\theta^2 + r^2 \sin^2\theta d\phi^2\bigg]
	\end{eqnarray*}
	3-Space Curvature Tensors,
	\begin{eqnarray*}
	R_{ijkl} = k(\tilde g_{jk}\tilde g_{il} - \tilde g_{ik}\tilde g_{jl}), \qquad R_{ij} = -3k\tilde g_{ij}, \qquad R = -6k
	\end{eqnarray*}
	with $k \in \{-1,0,1\}$. Define the conformal time
	\begin{eqnarray*}
		\tau = \int \frac{dt}{a(t)},
	\end{eqnarray*}
	\only<1>{
	\begin{eqnarray*}
		ds^2 = a(\tau)^2\bigg[-d\tau^2 + \frac{dr^2}{1-kr^2} + r^2 d\theta^2 + r^2 \sin^2\theta d\phi^2\bigg]
	\end{eqnarray*}
	}
	\only<2>{
	set $k=0$ (flat), simple conformal to flat form
	\begin{eqnarray*}
		ds^2 = a(\tau)^2\bigg[-d\tau^2 + dr^2 + r^2 d\theta^2 + r^2 \sin^2\theta d\phi^2\bigg]
	\end{eqnarray*}		
	}
\end{frame}

%%%%%%%%%%%%%%%%%%%%%%%%%%%%%%%%%%%%%%%%%%%%%%%%%%%%%%%%%%%%%%
% Slide 5 - Cosmological Geometries R.W. k=1
%%%%%%%%%%%%%%%%%%%%%%%%%%%%%%%%%%%%%%%%%%%%%%%%%%%%%%%%%%%%%%

\begin{frame}{Cosmological Geometries R.W. $k=1$}
	$k=1$ (spherical)
	\begin{eqnarray*}
		ds^2 = a(\tau)^2\bigg[-d\tau^2 + dr^2 + r^2 d\theta^2 + r^2 \sin^2\theta d\phi^2\bigg]
	\end{eqnarray*}	
	Set $\sin\chi = r$, $p = \tau$, 
	\begin{eqnarray*}
		ds^2 = a(p)^2\bigg[-dp^2 + d\chi^2 + \sin^2\chi d\theta^2 + \sin^2\chi \sin^2\theta d\phi^2\bigg]
	\end{eqnarray*}
	Introduce coordinates
	\begin{eqnarray*}
			p' + r' &=& \tan[(p+\chi)/2],\quad p'-r'=\tan[(p-\chi)/2]
			\nonumber\\
			p' &=& \frac{\sin p}{\cos p + \cos \chi}, \quad r' = \frac{\sin\chi}{\cos p + \cos\chi}
	\end{eqnarray*}
	\begin{eqnarray*}
	\implies ds^2 = \frac{4a^2(p)}{[1+(p'+r')^2][1+(p'-r')^2]}[-dp'^2 + dr'^2 +r'^2 d\theta^2 + r'^2\sin^2\theta d\phi^2]
	\end{eqnarray*}
\end{frame}

%%%%%%%%%%%%%%%%%%%%%%%%%%%%%%%%%%%%%%%%%%%%%%%%%%%%%%%%%%%%%%
% Slide 6 - Cosmological Geometries R.W. k=-1
%%%%%%%%%%%%%%%%%%%%%%%%%%%%%%%%%%%%%%%%%%%%%%%%%%%%%%%%%%%%%%

\begin{frame}{Cosmological Geometries R.W. $k=-1$}
	$k=-1$ (hyperbolic)
	\begin{eqnarray*}
		ds^2 = a(\tau)^2\bigg[-d\tau^2 + dr^2 + r^2 d\theta^2 + r^2 \sin^2\theta d\phi^2\bigg]
	\end{eqnarray*}	
	Set $\sin\chi = r$, $p = \tau$, 
	\begin{eqnarray*}
		ds^2 = a(p)^2\bigg[-dp^2 + d\chi^2 + \sin^2\chi d\theta^2 + \sin^2\chi \sin^2\theta d\phi^2\bigg]
	\end{eqnarray*}
	Introduce coordinates
	\begin{eqnarray*}
		p' + r' &=& \tanh[(p+\chi)/2],\quad p'-r'=\tanh[(p-\chi)/2]
		\nonumber\\
		p' &=& \frac{\sinh p}{\cosh p + \cosh \chi}, \quad r' = \frac{\sinh\chi}{\cosh p + \cosh\chi}
	\end{eqnarray*}
	\begin{eqnarray*}
	\implies ds^2 = \frac{4a^2(p)}{[1-(p'+r')^2][1-(p'-r')^2]}[-dp'^2 + dr'^2 +r'^2 d\theta^2 + r'^2\sin^2\theta d\phi^2]
	\end{eqnarray*}
\end{frame}

%%%%%%%%%%%%%%%%%%%%%%%%%%%%%%%%%%%%%%%%%%%%%%%%%%%%%%%%%%%%%%
% End Document
%%%%%%%%%%%%%%%%%%%%%%%%%%%%%%%%%%%%%%%%%%%%%%%%%%%%%%%%%%%%%%

\end{document}