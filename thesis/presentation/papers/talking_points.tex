\documentclass[10pt,letterpaper]{article}
\usepackage[textwidth=7in, top=1in,textheight=9in]{geometry}
\usepackage[fleqn]{mathtools} 
\usepackage{amssymb,braket,hyperref,xcolor}
\hypersetup{colorlinks, linkcolor={blue!50!black}, citecolor={red!50!black}, urlcolor={blue!80!black}}
%\usepackage[sorting=none]{biblatex}
\numberwithin{equation}{section}
\setlength{\parindent}{0pt}
\title{Dissertation Talking Points}
\date{}
\allowdisplaybreaks
\begin{document} 
\maketitle
\noindent 

%%%%%%%%%%%%%%%%%%%%%%%%%%%%%%%%%%%%%%%%%%%%%%%%%%%%%%%%%%%%%%
% Slide  - Title Page
%%%%%%%%%%%%%%%%%%%%%%%%%%%%%%%%%%%%%%%%%%%%%%%%%%%%%%%%%%%%%%

\section{Title Page}
\begin{itemize}
	\item Thank you everyone for coming. Welcome to my dissertation defense on the topic Cosmological Perturbations as applied to both standard Einstein Gravity and Conformal gravity
\end{itemize}


%%%%%%%%%%%%%%%%%%%%%%%%%%%%%%%%%%%%%%%%%%%%%%%%%%%%%%%%%%%%%%
% Slide  - TOC
%%%%%%%%%%%%%%%%%%%%%%%%%%%%%%%%%%%%%%%%%%%%%%%%%%%%%%%%%%%%%%

\section{Overview}
\begin{itemize}
	\item Alright so, to give an overview of what will be covered,
	\item we will first form the necessary background of cosmological perturbation theory. Talking about the geometry of the universe, introduce Einstein gravity, perturbations, and coordinate invariance
	\item As an approach to simplify and solve the equations arising in cosmology, we're going to analyze something called the SVT decomp. in 3 dimensions and then perform some generalizations to four dimensions, demonstrating specific applications of both within a de Sitter background geometry
	\item Then we'll cover a discussion of cosmological perturbations in conformal gravity, and demonstrate a calculation of the fluctuations in the early universe radiation era
	\item Finally we'll wrap it up with overall conclusions and I'll demonstrate some of the computations involved in order to do the calculations that we'll see through this presentation
\end{itemize}

%%%%%%%%%%%%%%%%%%%%%%%%%%%%%%%%%%%%%%%%%%%%%%%%%%%%%%%%%%%%%%
% Slide  - Cosmological Geometries 
%%%%%%%%%%%%%%%%%%%%%%%%%%%%%%%%%%%%%%%%%%%%%%%%%%%%%%%%%%%%%%

\section{Cosmological Geometries}
\begin{itemize}
	\item Let's first discuss the geometry that is relevant to cosmology. If we take a look at the distribution of matter in the universe, we will see something like this image taken from the hubble telescope (which I'll add in served as my desktop background for quite a number of years). 
	\item What we observe is that no matter where you are in the universe and no matter what direction you look, the universe is the same on large scales. More specifically, while the structure of universe may be in-homogeneous on small scales, on a large scale the universe is statistically homogeneous and isotropic. These two features are embodied in what is called the cosmological principle. 
	\item Based solely on arguments of homogeneity and isotropy, the large scale geometry of such a universe is described through the RW metric and de Sitter metric (of which one can show that de Sitter space is actually a subset of the RW geometry).
	\item It of interest to note that with a proper choice of coordinates, the roberston walker geometries can all be expressed in a conformal to flat form
	\item with this equation here showing the spacetime line element being expressed as a minkowski spacetime multiplied by an overall conformal factor
\end{itemize}

%%%%%%%%%%%%%%%%%%%%%%%%%%%%%%%%%%%%%%%%%%%%%%%%%%%%%%%%%%%%%%
% Slide  - Cosmological Geometries - Robertson Walker
%%%%%%%%%%%%%%%%%%%%%%%%%%%%%%%%%%%%%%%%%%%%%%%%%%%%%%%%%%%%%%

\section{Cosmological Geometries - Robertson Walker}
\begin{itemize}
	\item To see what the RW geometry entails, we have an expression for the spacetime line element in eq. (1), here in comoving coordinates. The geometry describes the expansion of space over time as characterized by the functional form of the scale factor $a(t)$
	\item The comoving coordinates are at rest with respect to the hubble flow, and if we look at the figure here we can see that the grid here represents the comoving coordinates. And if we assume these galaxies don't have any peculiar velocity, we can see that the comoving distance between the two galaxies remains constant, where as the proper distance increases as space expands according to $a(t)$
	\item The space itself that is expanding, referred to as the 3-space, is a space of uniform curvature, which can be represented by the curvature constant $k$, taking the values of $-1, 0, 1 $. These values correspond to hyperbolic, flat, or spherical space respectively.
	\item Spaces of constant curvature are called maximally symmetric, where the spatial components of the curvature tensors take the specific form in eq(2). So here we have the Riemann tensor, which we might recall is the unique tensor composed of second order derivatives of the metric which measures the local curvature. We also have its contractions the Ricci tensor, and the Ricci scalar and we see that the Ricci scalar is a constant in a max. symm. 3-space.
	\item As mentioned before, with a proper choice of coordinates, the RW metric can be cast into a conformal to flat form
	\item The simplest case is for $k=0$, in which if we define the conformal time $\tau$ and set $k=0$, then the line element take the form of eq (4), which we can recognize as conformal to a spherical polar flat metric
	\item For $k=\pm 1$, we have to perform additional coord. transformations, and we just note that for these the conformal factor is a function of both space and time.
	\item So this describes the geometry of the large scale universe, but in order to discuss the interaction of gravity and matter, one needs to introduce Einstein field equations, which we do now
\end{itemize}

%%%%%%%%%%%%%%%%%%%%%%%%%%%%%%%%%%%%%%%%%%%%%%%%%%%%%%%%%%%%%%
% Slide  - Einstein Gravity
%%%%%%%%%%%%%%%%%%%%%%%%%%%%%%%%%%%%%%%%%%%%%%%%%%%%%%%%%%%%%%

\section{Einstein Gravity}
\begin{itemize}
	\item One starts with the Einstein Hilbert action defined as the coordinate invariant integral over the Ricci scalar
	\item Functional variation w.r.t. the metric yields the Einstein tensor $G_{\mu\nu}$, and likewise upon specification of a matter action, one obtains the energy momentum tensor
	\item In requiring the sum of both the E.H. action and the matter action to be stationary with respect to arbitrary variations in the metric, we obtain the EFE's.
	\item With (10) showing an identity that relates the derivative of the ricci tensor to its contraction, we can see that the Einstein tensor is conserved
	\item In the EFE's the interaction of gravity and matter can be seen via the LHS being a pure function of the metric representing the curvature of space while the RHS defines the source of matter and energy
	\item Now one of the main difficulties in GR is that the FE's form a set of tightly coupled are non-linear PDE's where exact solutions are typically only found in very symmetric geometries. 
	\item In order to be able to solve the FE's that describe the effects of cosmology, we are going to look at the first order perturbations of the Einstein field equations
\end{itemize}

%%%%%%%%%%%%%%%%%%%%%%%%%%%%%%%%%%%%%%%%%%%%%%%%%%%%%%%%%%%%%%
% Slide  - Perturbation Theory
%%%%%%%%%%%%%%%%%%%%%%%%%%%%%%%%%%%%%%%%%%%%%%%%%%%%%%%%%%%%%%

\section{Cosmological Perturbation Theory}
\begin{itemize}
	\item So as discussed prior, on a large scale the universe is homogeneous and isotropic which we can think of as the smooth surface of this sphere.
	\item Now in order to capture the departures from homogeneity and isotropy, things that are necessary in order to form localized structures in spacetime, we introduce small fluctuations on top of the otherwise smooth background. Thus we define the metric according to a background contribution and first order perturbation $h_{\mu\nu}$ 
	\item If we then substitute the metric into $G_{\mu\nu}$, it then can be split into a background piece and fluctuation tensor
	\item Upon similarly perturbing $T_{\mu\nu}$, we can then form the entire background field equations and first order fluctuation equations, where here we've combined them into the tensor $\Delta_{\mu\nu}$
	\item The background equations serve to define the rate of expansion of space given the source, whereas the fluctuation equations describe the evolution of metric perturbations due to things like over densities arising from source
\end{itemize}

%%%%%%%%%%%%%%%%%%%%%%%%%%%%%%%%%%%%%%%%%%%%%%%%%%%%%%%%%%%%%%
% Slide  - Gauge Transformations
%%%%%%%%%%%%%%%%%%%%%%%%%%%%%%%%%%%%%%%%%%%%%%%%%%%%%%%%%%%%%%

\section{Coordinate Transformations}
\begin{itemize}	
	\item So upon perturbing the EFE's, we will need to consider the effect of coordinate transformations
	\item The field equations are covariant w.r.t. general coordinate transformations, with the metric transforming as in (18)
	\item If we now consider an infinitesimal coordinate transformation with the vector field $\epsilon$ small in the same sense that $h$ is small, then it is convenient to attribute the whole change in $g_{\mu\nu}$ to a change in the perturbation $h_{\mu\nu}$ 
	\item The fluctuation eqns are then to be invariant under the so called gauge transformation of eq (20), where $\Delta h_{\mu\nu}$ is given by this symmetric sum of derivatives onto epsilon
	\item Thus if $h_{\mu\nu}$ serves as a solution to the EFE's, then and $h'_{\mu\nu}$ defined by (20) will also serve as a solution
	\item Now since $h_{\mu\nu}$ is a 4x4 symmetric rank 2 tensor it has 10 components, and with the four coordinate functions that define the vector field $\epsilon$, one can then use the coordinate freedom to reduce $h_{\mu\nu}$ to six independent components
	\item Its also quite instructive to look at the transformation of the fluctuation tensors themselves
	\item Here we note that if the background tensor vanishes, then the fluctuation tensors themselves are separately gauge invariant
	\item However, if the background does not vanish, then it only the entire sum of $\delta G_{\mu\nu} + \delta T_{\mu\nu}$ that is gauge invariant. In Einstein gravity the background only vanishes in spaces where the Ricci tensor itself vanishes, and thus for non-flat cosmological geometries, its only $\Delta_{\mu\nu}$ that is gauge invariant
\end{itemize}

%%%%%%%%%%%%%%%%%%%%%%%%%%%%%%%%%%%%%%%%%%%%%%%%%%%%%%%%%%%%%%
% Slide  - Solving Equations of Motion
%%%%%%%%%%%%%%%%%%%%%%%%%%%%%%%%%%%%%%%%%%%%%%%%%%%%%%%%%%%%%%

\section{Solution Methods}
\begin{itemize}
	\item Now lets take a look at what the perturbed Einstein tensor looks like in a non-flat geometry
	\item Here we have an expression for the spatial components of the Einstein tensor, and we can get a sense for 1) how many terms there are and 2) how tightly these terms are coupled together
	\item In fact, if one were to expand out the contractions over dummy indices, and try to look at a single spatial component, like the radial component $\delta G_{rr}$, one would get an expression with about 3-5 times as many terms
	\item So even after linearizing we require additional methods to simplify and decouple the fluctuation equations
\end{itemize}


%%%%%%%%%%%%%%%%%%%%%%%%%%%%%%%%%%%%%%%%%%%%%%%%%%%%%%%%%%%%%%
% Slide  - SVT3 Decomposition
%%%%%%%%%%%%%%%%%%%%%%%%%%%%%%%%%%%%%%%%%%%%%%%%%%%%%%%%%%%%%%

\section{SVT3 Decomposition}
\begin{itemize}
	\item One such method, called the SVT3 decomposition, is to take $h_{\mu\nu}$ and decompose it into a basis of scalars, vectors, and tensors defined according to their transformation behavior under 3D rotations
	\item To show this we are going to keep things generic and first factor out a conformal factor from $h_{\mu\nu}$ and express it in terms of the perturbation $f_{\mu\nu}$
	\item Then we form the line element, here's our background and perturbation, and then here we do a 3+1 splitting to separate the time and spatial components
	\item So now we want to define the time and space components of $f_{\mu\nu}$ in terms of scalars, vectors, and tensors
	\item Hence we view $f_{00}$ as a 3-scalar, $f_{0i}$ as a 3-vector and $f_{ij}$ and a 3-tensor. $f_{00}$ just being a scalar, we redefine it in terms of a $\phi$, here we break up the 3 vector in to a transverse vector $B_i$ and the derivatives of a scalar $B$, and here we have two scalars $\psi$ and $E$, a transverse vector $E_i$ and a TT tensor $E_{ij}$. 
	\item If we count up the components, we have 4 scalars, two two component transverse vectors $E_i$ and $B_i$, and one 2 component TT tensor $E_{ij}$, adding up to 10 in total
	\item finally here the total line element in the SVT3 basis
\end{itemize}

%%%%%%%%%%%%%%%%%%%%%%%%%%%%%%%%%%%%%%%%%%%%%%%%%%%%%%%%%%%%%%
% Slide  - SVT3 $\delta G_{\mu\nu}$ de Sitter 1/4
%%%%%%%%%%%%%%%%%%%%%%%%%%%%%%%%%%%%%%%%%%%%%%%%%%%%%%%%%%%%%%

\section{SVT3 $\delta G_{\mu\nu}$ in a de Sitter Background 1/4}
\begin{itemize}
	\item To see how the SVT3 decomposition may be helpful in solving the perturbation equations, we are going to evaluate the EFE's in the de Sitter background
	\item As mentioned earlier, the de Sitter background can be expressed as a special case of the RW metric which here corresponds to choosing the scalar factor to have the form of $1/H\tau$
	\item While the RW metric consists of a 3-space that is max. symm.,  the de Sitter space is actually maximally symmetric w.r.t. the full 4D spacetime and so its curvature tensors take this form
	\item The E.M. tensor that gives rise to the desitter space is that which consists of just a cosmological constant - a simple constant background energy that drives the expansion of space
	\item de Sitter is chosen here just to keep things relatively simple, but the same SVT3 decomposition can be carried out in a more generic RW space where one considers an energy momentum tensor consisting of a perfect fluid
	\item So in perturbing the energy momentum tensor, we simply get a fluctuation proportional to $f_{\mu\nu}$
\end{itemize}

%%%%%%%%%%%%%%%%%%%%%%%%%%%%%%%%%%%%%%%%%%%%%%%%%%%%%%%%%%%%%%
% Slide  - SVT3 $\delta G_{\mu\nu}$ de Sitter 2/4
%%%%%%%%%%%%%%%%%%%%%%%%%%%%%%%%%%%%%%%%%%%%%%%%%%%%%%%%%%%%%%

\section{SVT3 $\delta G_{\mu\nu}$ in a de Sitter Background 2/4}
\begin{itemize}
	\item Now we are going to insert the SVT3 decomposed $f_{\mu\nu}$ into $\delta G_{\mu\nu}$ to obtain eq 30
	\item We see that in the SVT3 basis the number of terms we have to deal with has been reduced a little bit, but still quite a few remain
\end{itemize}

%%%%%%%%%%%%%%%%%%%%%%%%%%%%%%%%%%%%%%%%%%%%%%%%%%%%%%%%%%%%%%
% Slide  - SVT3 $\delta G_{\mu\nu}$ de Sitter 3/4
%%%%%%%%%%%%%%%%%%%%%%%%%%%%%%%%%%%%%%%%%%%%%%%%%%%%%%%%%%%%%%

\section{SVT3 $\delta G_{\mu\nu}$ in a de Sitter Background 3/4}
\begin{itemize}
	\item To form the fluctuation equations, we need to add in $\delta T_{\mu\nu}$ to form the full $\Delta_{\mu\nu}$ here
	\item A couple things to note: 
	1) if we look at the various components, we can see for example that scalars are coupled to vectors here (in $\Delta_{00}$) and scalars, vectors, and tensors are still coupled together here ($\Delta_{ij}$)
	2) We recall that $\Delta_{\mu\nu}$ must be entirely gauge invariant, and thus the specific combinations of the SVT3 quantities that appear in each component of $\Delta_{\mu\nu}$ must themselves be gauge invariant
	\item Here we've identified the appropriate gauge invariant combinations, and if one counts the DOF, we see that there are two scalars, one 2-component transverse vector, and one 2 component TT tensor, totaling a set of 6 gauge invariants as expected from the gauge freedom from $\epsilon$ that we mentioned earlier
	\item Now in order to actually solve these, we will need to decouple them, and we find that by applying appropriate higher derivatives, we can do so
\end{itemize}

%%%%%%%%%%%%%%%%%%%%%%%%%%%%%%%%%%%%%%%%%%%%%%%%%%%%%%%%%%%%%%
% Slide  - SVT3 $\delta G_{\mu\nu}$ de Sitter 4/4
%%%%%%%%%%%%%%%%%%%%%%%%%%%%%%%%%%%%%%%%%%%%%%%%%%%%%%%%%%%%%%

\section{SVT3 $\delta G_{\mu\nu}$ in a de Sitter Background 4/4}
\begin{itemize}
	\item Here we have a set of 6 decoupled equations in the 6 gauge invariants, in which we can solve 
	\item So to give quick recap here, we first perturbed the Einstein and E.M. around a de Sitter background
	\item Then we decomposed $h_{\mu\nu}$ into a basis of 3-scalars, 3-vectors, and 3-tensors
	\item Inserted that $h_{\mu\nu}$ in the fluctuation tensors
	\item We formed the perturbed field equations, noting that they are entirely gauge invariant
	\item We express the FE's in terms of the gauge invariant SVT3 combinations
	\item Finally applied derivatives to decouple the SVT3 modes
\end{itemize}
  
%%%%%%%%%%%%%%%%%%%%%%%%%%%%%%%%%%%%%%%%%%%%%%%%%%%%%%%%%%%%%%
% Slide  - SVT3 Integral Formulation 1/4
%%%%%%%%%%%%%%%%%%%%%%%%%%%%%%%%%%%%%%%%%%%%%%%%%%%%%%%%%%%%%%

\section{SVT3 Integral Formulation 1/4}
\begin{itemize}
	\item Prior, I simply stated that this is the decomposition of $h_{\mu\nu}$ in the SVT3 basis. However, one might ask, is such a decomposition always possible? Or even further one might ask how does a quantity like $\phi$ transform under gauge transformations?
	\item To properly address the SVT3 decomposition, we need to effectively invert eq 34 so as to express each SVT3 quantity directly in terms of $h_{\mu\nu}$ itself.
	\item To keep things clear and simple, we are going to analyze the following SVT3 decomposition around a flat minkowski background and we are going to start with the decomposition of the 3-vector $h_{0i}$ into the transverse $B_i$ and the scalar $B$
\end{itemize}

%%%%%%%%%%%%%%%%%%%%%%%%%%%%%%%%%%%%%%%%%%%%%%%%%%%%%%%%%%%%%%
% Slide  - SVT3 Integral Formulation 2/4
%%%%%%%%%%%%%%%%%%%%%%%%%%%%%%%%%%%%%%%%%%%%%%%%%%%%%%%%%%%%%%

\section{SVT3 Integral Formulation 2/4}
\begin{itemize}
	\item So we seek to decompose a general vector into two parts, a transverse part and a longitudinal part, written as the derivative of a scalar
	\item However, by itself the derivative of a scalar is not necessarily longitudinal; if we take the divergence of a general vector, we see that any scalar that vanishes under the laplacian will be transverse
	\item So to form the longitudinal part, we have to find the derivative of a scalar which could never be transverse
	\item To accomplish, we first introduce a flat spatial greens function and an identity, which here is just a particular ordering of derivatives and the product rule
	\item Upon integrating (36), we arrive at the decomposition of a generic scalar into its non-harmonic and harmonic components; here when I say harmonic, it is to mean any function that vanishes under the generalized laplacian
	\item Upon applying the laplacian to (37), one can show that the harmonic surface integral vanishes identically and so we are left only with the non-harmonic contribution
	\item So in order form the derivative of a scalar which could never be transverse, we require the scalar to be non-harmonic, and thus we require the surface contribution to vanish; 
	\item This means that the scalar $V$ must itself vanish asymptotically or decay sufficiently fast
	\item So with this being the definition of $V$, we can now form the longitudinal component and finally arrive at this definition for the entire decomposition
	\item Here we see we have defined the transverse and longitudinal components in terms of the vector field $V_i$ itself
\end{itemize}

%%%%%%%%%%%%%%%%%%%%%%%%%%%%%%%%%%%%%%%%%%%%%%%%%%%%%%%%%%%%%%
% Slide  - SVT3 Integral Formulation 3/4
%%%%%%%%%%%%%%%%%%%%%%%%%%%%%%%%%%%%%%%%%%%%%%%%%%%%%%%%%%%%%%

\section{SVT3 Integral Formulation 3/4}
\begin{itemize}
	\item We can also take the decomposition we just did and cast it into the form a transverse vector project, here given by this quantity $\Pi_{ij}$. Applying $\Pi_{ij}$ to a generic vector projects out its transverse component, and it obeys the expected projected algebra given in (42)
	\item So getting back to the decomposition of $h_{0i}$, we can now express $B_i$ and $B$ in terms of $h_{0i}$, and we note that the these SVT3 quantities are defined as non-local integrals over the metric perturbation, and we additionally require that the scalar $B$ vanish asymptotically or decay sufficiently fast
\end{itemize}

%%%%%%%%%%%%%%%%%%%%%%%%%%%%%%%%%%%%%%%%%%%%%%%%%%%%%%%%%%%%%%
% Slide  - SVT3 Integral Formulation 4/4
%%%%%%%%%%%%%%%%%%%%%%%%%%%%%%%%%%%%%%%%%%%%%%%%%%%%%%%%%%%%%%

\section{SVT3 Integral Formulation 4/4}
\begin{itemize}
	\item So that covers the vector decomposition, but what about the tensor decomposition of $h_{ij}$? 
	\item Using a similar procedure now with tensors, we introduce a vector W and present eq (45) as the transverse traceless component of $h_{ij}$
	\item Eq. (45) is automatically traceless, and to be transverse we require the vector $W$ to obey eq. (46) here, which thus serves to define $W$
	\item Eq (45) thus represents the representation of the TT tensor $E_{ij}$, however we still need to obtain representations for the vectors $E_i$ and scalars $E$ and $\psi$
	\item To form a transverse vector, we note that we can further decompose $W$ intro its transverse and longitudinal parts
	\item So finally putting everything together, we arrive at the following decomposition of $h_{ij}$ in terms of all the SVT3 quantities
	\item Again one can readily check that this quantity is TT, this is transverse, and the remaining scalars are defined by the metric or derivative prefactors to match the form in (44)
	\item So this completes the inversion of the SVT3 basis entirely in terms $h_{\mu\nu}$
\end{itemize}

%%%%%%%%%%%%%%%%%%%%%%%%%%%%%%%%%%%%%%%%%%%%%%%%%%%%%%%%%%%%%%
% Slide  - SVT4 Setup
%%%%%%%%%%%%%%%%%%%%%%%%%%%%%%%%%%%%%%%%%%%%%%%%%%%%%%%%%%%%%%

\section{SVT4 Setup}
\begin{itemize}
	\item There are some shortcomings to this basis though. Recall that in the SVT3 basis, we had to do a 3+1 decomposition on both $h_{\mu\nu}$ and the fluctuation tensors themselves to give us a larger set of equations to solve. And since the full transformations of GR consist of general 4D coordinate transformations, something like this quantity $\phi$ which is proportional to $h_{00}$ will transform into linear combinations of any other SVT3 quantities if we do coordinate transformations in both time and space
	\item It would be more natural to use a basis that is covariant with respect to the underlying transformation group of GR and it may lead to a simpler set of equations to solve. To construct such a basis, we seek to generalize the existing SVT3 basis in two ways
	\item A) Generalize to $D=4$ to match GR transformations, in fact there is no additional overhead to generalize this to arbitrary dimension $D$, so we we'll do that \vspace{1mm}
	\item And B) Generalize to curved space backgrounds beyond the Minkowski treatment that we showed for SVT3
	\item When going to curved space, matters are a little more complicated as we cannot simply take the partial derivatives and carry them into covariant derivatives since curved space covariant derivatives no longer commute
	\item From (50) we see that the commutation is sensitive to the Riemann curvature tensor which is non-zero in the cosmological backgrounds, so the overall decomposition is going to have to take this into account
	\item In four dimensions here is what the covariant decomposition would look like
	\item We have two scalars, one 3 component transverse vector, and one 5 component TT tensor, adding up to 10 components in total
	\item To address A) and B) here, we are going to show the covariant decomposition around a maximally symmetric de Sitter background. I'll add in that in our paper we've been able to formulate the decomposition around arbitrary $D$ dimensional curved spaces as well 
\end{itemize}

%%%%%%%%%%%%%%%%%%%%%%%%%%%%%%%%%%%%%%%%%%%%%%%%%%%%%%%%%%%%%%
% Slide - SVTD Integral Formulaion - Max. Sym. Space 1/2
%%%%%%%%%%%%%%%%%%%%%%%%%%%%%%%%%%%%%%%%%%%%%%%%%%%%%%%%%%%%%%

\section{SVTD Integral Formulation - Maximally Symmetric Space 1/2}
\begin{itemize}
	\item Ok so again when we say a maximally symmetric space, we mean a space of constant curvature, where the curvature tensors take the form of eq. (53), and we see that the Ricci scalar is a constant
	\item To form the TT projection, we introduce the curved space Green's function of (54), and with it one can then check that (55) is traceless and transverse where the vector $W$ is defined in terms of $h$ within (55)
	\item To show transversness, one has to make use of all these commutation relations, which are proportional to the Ricci scalar
\end{itemize}

%%%%%%%%%%%%%%%%%%%%%%%%%%%%%%%%%%%%%%%%%%%%%%%%%%%%%%%%%%%%%%
% Slide - SVTD Integral Formulaion - Max. Sym. Space 2/2
%%%%%%%%%%%%%%%%%%%%%%%%%%%%%%%%%%%%%%%%%%%%%%%%%%%%%%%%%%%%%%

\section{SVTD Integral Formulation - Maximally Symmetric Space 2/2}
\begin{itemize}
	\item Now in order to define the transverse vectors $F_{\mu}$ and the scalars $F$ and $\chi$, similar to before we can break up the vector $W$ into its transverse and longitudinal components, where introduce a different green's function here
	\item Finally, we make the appropriate definition, and complete the SVTD representation to be able to form (61)
	\item If we take $D=3$ and take the background to the flat Kronecker delta, the two green's function become equivalent and we recover the SVT3 decomposition
\end{itemize}

%%%%%%%%%%%%%%%%%%%%%%%%%%%%%%%%%%%%%%%%%%%%%%%%%%%%%%%%%%%%%%
% Slide  - SVT4 $\delta G_{\mu\nu}$ de Sitter 1/2
%%%%%%%%%%%%%%%%%%%%%%%%%%%%%%%%%%%%%%%%%%%%%%%%%%%%%%%%%%%%%%

\section{SVT4 $\delta G_{\mu\nu}$ in a de Sitter Background 1/2}
\begin{itemize}
	\item Ok, now that we've formed a new covariant decomposition, let's go ahead an apply it to the same deSitter background case that we saw earlier in the SVT3 basis
	\item Here we note that the covariant derivatives are with respect to the full background, and in forming the full fluctuation equations, we get a nice compact form for $\Delta_{\mu\nu}$ given here
	\item If we recall that $\Delta_{\mu\nu}$ is entirely gauge invariant, we see that is it is $F_{\mu\nu}$ and $\chi$ that are the SVT4 gauge invariant quantities, and if we count the components, we see we have one scalar, one five component TT tensor, summing to 6 independent components in total
	\item To separate the SVT modes we can apply the trace and use this scalar commutation relation and then applying higher derivatives we get the decoupled scalar and tensor modes
\end{itemize}

%%%%%%%%%%%%%%%%%%%%%%%%%%%%%%%%%%%%%%%%%%%%%%%%%%%%%%%%%%%%%%
% Slide  - SVT3 vs. SVT4 $\delta G_{\mu\nu}$ de Sitter
%%%%%%%%%%%%%%%%%%%%%%%%%%%%%%%%%%%%%%%%%%%%%%%%%%%%%%%%%%%%%%

\section{SVT3 vs. SVT4 $\delta G_{\mu\nu}$ de Sitter}
\begin{itemize}
	\item So here we can compare the SVT4 basis to the SVT3 basis and find that not only is the SVT4 basis covariant w.r.t. the transformations of GR, it also provides a much more convenient and simpler formalism
\end{itemize}

%%%%%%%%%%%%%%%%%%%%%%%%%%%%%%%%%%%%%%%%%%%%%%%%%%%%%%%%%%%%%%
% Slide  - Conformal Gravity Intro
%%%%%%%%%%%%%%%%%%%%%%%%%%%%%%%%%%%%%%%%%%%%%%%%%%%%%%%%%%%%%%

\section{Conformal Gravity Introduction}
\begin{itemize}
	\item So all the field equations presented so far have been in the context of standard Einstein gravity and we are now going to discuss the evolution of cosmological perturbations in an alternative theory of gravitation called conformal gravity
	\item The basis of conformal gravity is that we require the gravitational action to be locally conformal invariant, meaning that it is to be invariant under rescalings of the metric with arbitrary $\Omega$ here.
	\item The Weyl action here is the single unique action composed purely of the metric that is locally conformally invariant, and unlike the EH action where we had a coordinate invariant integral over the contraction of the Ricci tensor, here we have a contraction over the square of the Weyl tensor C and $\alpha_g$ is a dimensionless coupling constant
	\item So $C_{\lambda\mu\nu\kappa}$ here is in fact the traceless component of the Riemann tensor and has the unique property that it is invariant under conformal transformations
	\item With this theory having been advanced by my advisor Philip Mannheim, astrophysical and cosmological support for conformal gravity is motivated by the excellent agreement between fits of the accelerating universe Hubble plot data as well as galactic rotation curves of over 100 spiral galaxies, work done with Mannheim and Obrien, all without the imposition of dark energy or dark matter
	\item Alright so to obtain the FE's, as before, we vary the action w.r.t. the metric tensor, to obtain what is called the Bach tensor $W_{\mu\nu}$, which serves as the analog to the Einstein tensor $G$. $W_{\mu\nu}$ can be expressed in two ways here, firstly  as derivatives onto the Weyl tensor or secondly as this entire combination $W_2 - W_1$ where these are expressed in terms of the Ricci tensor.
	\item We can see that vacuum solutions to the conformal gravity correspond to either a) a vanishing ricci tensor, or b), a vanishing Weyl tensor
	\item With the ricci tensor vanishing, this means that all vacuum solutions to Einstein gravity and also vacuum solutions to conformal gravity
	\item With the weyl tensor being conformally invariant, it therefore vanishes in any geometry that is conformal to flat
	\item and as we have seen all the RW geometries can be cast into the conformal to flat form, so therefore in cosmology, the background Bach tensor vanishes identically
	\item Now the field equations here are fourth order and include quite a large number of terms esepcially in comparison to the Einstein field equations here
	\item However, we will see that the conformal properties inherited in conformal gravity provide some very simplifications in cosmology that are not shared with einstein gravity
\end{itemize}

%%%%%%%%%%%%%%%%%%%%%%%%%%%%%%%%%%%%%%%%%%%%%%%%%%%%%%%%%%%%%%
% Slide  - Conformal Invariance in Conformal Gravity
%%%%%%%%%%%%%%%%%%%%%%%%%%%%%%%%%%%%%%%%%%%%%%%%%%%%%%%%%%%%%%

\section{Conformal Invariance in Conformal Gravity}
\begin{itemize}
	\item So to discuss these conformal properties, we first note that the Bach tensor is traceless and conserved and that under conformal transformation, its covariant form transforms as $\Omega^{-2}$
	\item As with before, we can decompose this into a background contribution and a first order fluctuation, to obtain the background and perturbative field equations
\end{itemize}

%%%%%%%%%%%%%%%%%%%%%%%%%%%%%%%%%%%%%%%%%%%%%%%%%%%%%%%%%%%%%%
% Slide  - Trace Properties in Conformal Gravity
%%%%%%%%%%%%%%%%%%%%%%%%%%%%%%%%%%%%%%%%%%%%%%%%%%%%%%%%%%%%%%

\section{Trace Properties in Conformal Gravity}
\begin{itemize}
	\item In conformal gravity, its convenient to cast things in terms of $K_{\mu\nu}$, which is simply the traceless component of $h_{\mu\nu}$
	\item Now if we generically express the Bach fluctuation in terms of $h_{\mu\nu}$, we get one contribution from $K_{\mu\nu}$ and one contribution from the trace of $h_{\mu\nu}$
	\item Using the properties of the Bach tensor under conformal transformations, one can show that the perturbation as a function of the trace and the trace of the perturbation itself are both proportional to the background Bach tensor
	\item What this means is that in geometries where the background $W_{\mu\nu}$ vanishes, the Bach fluctuation is both traceless and can be written entirely in terms of the 9 component $K_{\mu\nu}$
	\item And with our freedom of four coordinate transformations, we see that in conformal cosmology $h_{\mu\nu}$ consists of 5 independent physical components
\end{itemize}

%%%%%%%%%%%%%%%%%%%%%%%%%%%%%%%%%%%%%%%%%%%%%%%%%%%%%%%%%%%%%%
% Slide  - $\delta W_{\mu\nu}$ General
%%%%%%%%%%%%%%%%%%%%%%%%%%%%%%%%%%%%%%%%%%%%%%%%%%%%%%%%%%%%%%

\section{$\delta W_{\mu\nu}$}
\begin{itemize}
	\item So we are now going to construct the cosmological field equations in conformal gravity
	\item We start off completely generically, not having yet specified a background, and we find a pretty extensive expression here for $\delta W_{\mu\nu}$, with 52 terms in the $K_{\mu\nu}$ sector and 19 in the trace sector
\end{itemize}

%%%%%%%%%%%%%%%%%%%%%%%%%%%%%%%%%%%%%%%%%%%%%%%%%%%%%%%%%%%%%%
% Slide  - $\delta G_{\mu\nu}$ General
%%%%%%%%%%%%%%%%%%%%%%%%%%%%%%%%%%%%%%%%%%%%%%%%%%%%%%%%%%%%%%

\section{$\delta G_{\mu\nu}$}
\begin{itemize}
	\item It is especially complex if we compare this to a general standard second order Einstein fluctuation
\end{itemize}


%%%%%%%%%%%%%%%%%%%%%%%%%%%%%%%%%%%%%%%%%%%%%%%%%%%%%%%%%%%%%%
% Slide  - $\delta W_{\mu\nu}$ General
%%%%%%%%%%%%%%%%%%%%%%%%%%%%%%%%%%%%%%%%%%%%%%%%%%%%%%%%%%%%%%

\section{$\delta W_{\mu\nu}$}
\begin{itemize}
	\item However as mentioned earlier, in geometries where the background Bach tensor vanishes, this trace dependent term entirely vanishes, so we are left with the fluctuation being a function only of $K_{\mu\nu}$ here
\end{itemize}

%%%%%%%%%%%%%%%%%%%%%%%%%%%%%%%%%%%%%%%%%%%%%%%%%%%%%%%%%%%%%%
% Slide  - $\delta W_{\mu\nu}$ in Conformal to Flat Backgrounds 2/3
%%%%%%%%%%%%%%%%%%%%%%%%%%%%%%%%%%%%%%%%%%%%%%%%%%%%%%%%%%%%%%

\section{$\delta W_{\mu\nu}$ in Conformal to Flat Backgrounds 2/3}
\begin{itemize}
	\item Now we are going to evaluate these expression in a conformal to Minkowski background, with an arbitrary conformal factor
	\item and we get a large number of terms here, 151 in total
\end{itemize}

%%%%%%%%%%%%%%%%%%%%%%%%%%%%%%%%%%%%%%%%%%%%%%%%%%%%%%%%%%%%%%
% Slide  -$\delta W_{\mu\nu}$ in Conformal to Flat Backgrounds 3/3
%%%%%%%%%%%%%%%%%%%%%%%%%%%%%%%%%%%%%%%%%%%%%%%%%%%%%%%%%%%%%%

\section{SVT4 $\delta W_{\mu\nu}$ in Conformal to Flat Backgrounds 3/3}
\begin{itemize}
	\item Despite its formidable form, we can express this set of terms in a rather compact form as a sequence of derivatives onto the conformal factor
	\item However, even with this immense simplification, we see that the various components of $K_{\mu\nu}$ are still tightly coupled to each other and the solution is not immediately obvious
	\item However, if we use what we have learned about the transverse traceless tensor projection, and form the integral expression in terms of $K_{\mu\nu}$, we see that these sequences of derivatives precisely get rid of all the non-local integrals here and allows us to express the fluctuations as just one single term here
	\item So this equation here represents the fluctuation FE of conformal gravity and we that not only is it expressed a function of a 5 component TT tensor, but that each component of the TT tensor is completely decoupled from one another, and so the solution is readily obtainable
\end{itemize}

%%%%%%%%%%%%%%%%%%%%%%%%%%%%%%%%%%%%%%%%%%%%%%%%%%%%%%%%%%%%%%
% Slide  - SVT4 $\delta W_{\mu\nu}$ in Conformal to Flat Backgrounds 3/3
%%%%%%%%%%%%%%%%%%%%%%%%%%%%%%%%%%%%%%%%%%%%%%%%%%%%%%%%%%%%%%

\section{SVT4 $\delta W_{\mu\nu}$ in Conformal to Flat Backgrounds 3/3}
\begin{itemize}
	\item To touch basis on our earlier SVT4 decomposition, we can evaluate the conformal fluctuations in a conformal to flat background, where we then obtain the fluctuation equations here
	\item With $F_{\mu\nu}$ being the TT component of $h_{\mu\nu}$, we see that the SVT4 basis nicely reconciles with our previous result
\end{itemize}


%%%%%%%%%%%%%%%%%%%%%%%%%%%%%%%%%%%%%%%%%%%%%%%%%%%%%%%%%%%%%%
% Slide  - Conformal Gravity Robertson-Walker Radiation Era
%%%%%%%%%%%%%%%%%%%%%%%%%%%%%%%%%%%%%%%%%%%%%%%%%%%%%%%%%%%%%%

\section{Conformal Gravity Robertson-Walker Radiation Era}
\begin{itemize}
	\item To illustrate a solution of these equations in conformal gravity, we are going look at fluctuations in the early universe described by a radiation dominated perfect fluid. In the observational fits to the spiral galaxies and hubble data plots I mentioned earlier, one finds that phenomenologically, the $k=-1$ hyperbolic RW geometry is preferred in conformal gravity, and here is its expression in comoving coordinates
	\item and here is the same geometry as expressed in conformal to flat coordinates
	\item The plane wave solutions to this fourth order equation are given by a sum of polarization tensors $A_{\mu\nu}$ and $B_{\mu\nu}$, where here the $B_{\mu\nu}$ term has a prefactor $n\dot x$ which is proportional to the p' time coordinate in the conformal to flat coordinate basis
	\item By doing appropriate coordinate transformations, one can show that the leading order time behavior of $h_{\mu\nu}^{TT}$ resides in the product of $\Omega^2$ time $p'$, and after carrying out all the necessary coordinate transformations to convert this to the comoving time $t$, we find that fluctuations grow as $t^4$. This can be contrasted with the einstein radiation era perfect fluid solutions where fluctuations grow as $t^{1/2}$
	\item more comments on size of growth? 
\end{itemize}

%%%%%%%%%%%%%%%%%%%%%%%%%%%%%%%%%%%%%%%%%%%%%%%%%%%%%%%%%%%%%%
% Slide  - Conclusions
%%%%%%%%%%%%%%%%%%%%%%%%%%%%%%%%%%%%%%%%%%%%%%%%%%%%%%%%%%%%%%

\section{Conclusions}
\begin{itemize}
	\item Alright so that completes the discussion of cosmological fluctuations
	\item To summarize what we've covered here, uh
	\item We presented an integral formalism to represent the SVT3 basis components in terms of $h_{\mu\nu}$, find that the SVT components are composed of nonlocal integrals and automatically incorporate asymptotic boundary conditions
	\item After forming the fluctuation equations, we could effectively decouple the SVT modes by applying higher derivatives
	\item We then generalized the SVT decomposition to arbitrary dimension D and in a maximally symmetric curved space, finding that in $D=4$, the fluctuations are both simpler and covariant
	\item And finally we formulated the cosmological fluctuations as applied to conformal gravity
	\item Here we saw that despite the rather formidable fourth order nature of the theory, in conformal to flat backgrounds the equations take a remarkably simple form, whereby we demonstrated the growth of fluctuations going as $t^4$ in a radiation era early universe RW geometry
\end{itemize}

%%%%%%%%%%%%%%%%%%%%%%%%%%%%%%%%%%%%%%%%%%%%%%%%%%%%%%%%%%%%%%
% Slide  - Computational Methods
%%%%%%%%%%%%%%%%%%%%%%%%%%%%%%%%%%%%%%%%%%%%%%%%%%%%%%%%%%%%%%

\section{Computational Methods}
\begin{itemize}
	\item So the vast majority of equations I presented had to be generated computationally, so I need to give a brief overview of how these calculation were performed
	\item So in composing the fluctuations equations in cosmology, there are numerous steps that pose a lot of difficulty if one were to do the calculations non-computationally, mostly due to the sheer number of terms and manipulations involved. 
	\item To do them programatically, what one needs is a language capable of doing symbolic tensor calculus on manifolds. 
	\item In terms of abstract symbolic computation, Mathematica is one of the most effective frameworks, however it doesn't natively include support for GR specific applications.
	\item To remedy this, in the late 2000's, a physicist named Jose Martin Garcia developed a package to be used within Mathematica called xAct, which is composed of a suite of free GR specific packages that implement state-of-the-art algorithms for fast manipulation of indices in a highly programmable and configurable form
	\item All the calculations I've done using this package and additional programs I've coded are available at this github repository, where anyone is free to use or modify them
	\item I'm going to very quickly show you one of the notebooks that I use extensively and some of the calculations involved
\end{itemize}

%%%%%%%%%%%%%%%%%%%%%%%%%%%%%%%%%%%%%%%%%%%%%%%%%%%%%%%%%%%%%%
% Slide  - References
%%%%%%%%%%%%%%%%%%%%%%%%%%%%%%%%%%%%%%%%%%%%%%%%%%%%%%%%%%%%%%

\section{References}
\begin{itemize}
	\item Here are the two papers this presentation is based on. Um, I remember that before i started doing anything in GR I would hear that calculations in GR could be quite the task of labor, and now with these two papers combined totaling 100 pages, I definitely think I appreciate that sentiment a lot more
\end{itemize}

%%%%%%%%%%%%%%%%%%%%%%%%%%%%%%%%%%%%%%%%%%%%%%%%%%%%%%%%%%%%%%
% Slide  - Acknowledgments
%%%%%%%%%%%%%%%%%%%%%%%%%%%%%%%%%%%%%%%%%%%%%%%%%%%%%%%%%%%%%%

\section{Acknowledgments}
\begin{itemize}
	\item So finally, to those who I've been involved with during my PhD I'd like to give some acknowledgments.
	\item Foremost I want to thank my advisor Philip Mannheim for giving me this opportunity. Its been an honor for me to work for you and I'm grateful that there are so lessons I've learned under you that will always be with me with me
	\item Thank you to my co-advisors for your flexibility and giving me direction from time to time
	\item To my friend afar Ray Retherford, for helping me through difficult times, I hope to see you soon
	\item To my partner Michaela Poppick and her continual support
	\item and to my friends I've made while in Connecticut, its immensely difficult to imagine how I'd survive without you, special thanks to Candost, H Perry, Lukasz, Chen unit,
	\item I also want to give thanks to Jason Hancock for having faith in me and giving me some degree of autonomy in developing the studio physics labs, Diego as well. Its honestly been a pleasure working with you
	\item and to my family, your encouragement has meant a lot to me, and finally to all other friends and faculty that I've had the pleasure to get to know, thank you all. 
\end{itemize}

%%%%%%%%%%%%%%%%%%%%%%%%%%%%%%%%%%%%%%%%%%%%%%%%%%%%%%%%%%%%%%
% Slide  - End
%%%%%%%%%%%%%%%%%%%%%%%%%%%%%%%%%%%%%%%%%%%%%%%%%%%%%%%%%%%%%%

\section{The End}

%%%%%%%%%%%%%%%%%%%%%%%%%%%%%%%%%%%%%%%%%%%%%%%%%%%%%%%%%%%%%%
%%%%%%%%%%%%%%%%%%%%%%%%%%%%%%%%%%%%%%%%%%%%%%%%%%%%%%%%%%%%%%
%%%%%%%%%%%%%%%%%%%%%%%%%%%%%%%%%%%%%%%%%%%%%%%%%%%%%%%%%%%%%%
\end{document}