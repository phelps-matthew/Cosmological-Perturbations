\documentclass[10pt,letterpaper]{article}
\usepackage[textwidth=7in, top=1in,textheight=9in]{geometry}
\usepackage[fleqn]{mathtools} 
\usepackage{amssymb,braket,hyperref,xcolor}
\hypersetup{colorlinks, linkcolor={blue!50!black}, citecolor={red!50!black}, urlcolor={blue!80!black}}
\usepackage[title]{appendix}
%\usepackage[sorting=none]{biblatex}
\numberwithin{equation}{section}
\setlength{\parindent}{0pt}
\title{Dissertation Notes}
\date{}
\allowdisplaybreaks
\begin{document} 
\maketitle
\noindent 

%%%%%%%%%%%%%%%%%%%%%%%%%%%%%%%%%%
\section*{Cosmological Geometries}
%%%%%%%%%%%%%%%%%%%%%%%%%%%%%%%%%%
\begin{itemize}
	\item $k\in \{-1,0,1\}$ topological?
	\item General form derived from homogeneity and isotropy, Einstein equations only serve to define  $a(t)$.
	\item 3D space of uniform curvature. Almost homogeneous and isotropic when averaged over a large scale
\end{itemize}
%%%%%%%%%%%%%%%%%%%%%%%%%%%%%%%%%%
\section*{Einstein Gravity}
%%%%%%%%%%%%%%%%%%%%%%%%%%%%%%%%%%
\begin{itemize}
	\item Total action $I$ is stationary with respect to arbitrary variation in $g_{\mu\nu}$ if and only if $R_{\mu\nu} - \frac{1}{2}g_{\mu\nu}R = T_{\mu\nu}$ (Weinberg pg. 364)
	\item Can show conservation of $\nabla^\mu T_{\mu\nu}$ by imposing field equation (only holds for stationary paths), or by imposing (scalar) field equations of motion to $T_{\mu\nu}$ directly (holds for stationary paths beyond those satisfying field equations)
\end{itemize}

%%%%%%%%%%%%%%%%%%%%%%%%%%%%%%%%%%
\section*{Perturbations}
%%%%%%%%%%%%%%%%%%%%%%%%%%%%%%%%%%
\begin{itemize}
	\item Small compared to what? ``We are helped in this task by the fact that we expect such inhomogeneities to be of very small amplitude early on so we can adopt a kind of perturbative approach, at least for the early stages of the problem. If the length scale of the perturbations is smaller than the effective cosmological horizon $d_H = c / H_0$, a Newtonian treatment of the subject is expected to be valid."
	\item ``Cosmological   perturbation   theory   applies   to   largescales, up to and beyond the particle horizon of the ob-servable Universe.  Such length scales are, by definition,comparable  to  the  characteristic  wavelength, $\lambda_c \sim L_C$, where $L_C$ the typical length scale associated with the regime of cosmological perturbation theory. "
	\item ``The most natural explanation for the large-scale structures seen in galaxy surveys (e.g. superclusters, walls, and filaments) is that they are the result of gravitational amplification of small primordial fluctuations due to the gravitational interaction of collisionless cold dark matter (CDM) particles in an expanding universe"
	\item Fluctuations capture departure from homogeneity and isotropy
	\item ``As an essential feature of the analysis presented here, we assume that during most of the history of the universe all departures from homogeneity and isotropy have been small, so that they can be treated as first-order perturbations."
	\item First given by Lifshitz 1946, created notation $\delta g_{\mu\nu} = h_{\mu\nu}$
	\end{itemize}
	
%%%%%%%%%%%%%%%%%%%%%%%%%%%%%%%%%%
\section*{Gauge Transformations}
%%%%%%%%%%%%%%%%%%%%%%%%%%%%%%%%%%
\begin{itemize}	
	\item Weinberg G\&C 10.9, 15.10 580, Cosmology 235
	\item Maggiore, $x^\mu \to x'^\mu$. $x'^\mu$ must be invertible, differentiable, and with a differential inverse (i.e. an arbitrary diffeomorphism)
	\item Brane Localized Mannheim pg. 82 (shows change of entire $g_{\mu\nu} = g_{\mu\nu}^{(0)}+h_{\mu\nu}$ and collects with $h_{\mu\nu}$)
	\item Attribute the whole change in $g_{\mu\nu}$ to change in perturbation $h_{\mu\nu}$
\end{itemize}

%%%%%%%%%%%%%%%%%%%%%%%%%%%%%%%%%%
\section*{SVT3 $\delta G_{\mu\nu}$ in a de Sitter Background}
%%%%%%%%%%%%%%%%%%%%%%%%%%%%%%%%%%
\begin{itemize}
	\item How to address gauge invariants?
\end{itemize}

%%%%%%%%%%%%%%%%%%%%%%%%%%%%%%%%%%
\section*{SVT3 Integral Formulation}
%%%%%%%%%%%%%%%%%%%%%%%%%%%%%%%%%%
\begin{itemize}
	\item Show identical vanishing of surface term upon application of $\partial_i \partial^i$
	\item Wording: Harmonic function, generalized  Laplacian. divergence of gradient. 
	\item Think about conditions requires to vanish
\end{itemize}
\end{document}