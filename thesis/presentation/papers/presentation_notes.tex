\documentclass[10pt,letterpaper]{article}
\usepackage[textwidth=7in, top=1in,textheight=9in]{geometry}
\usepackage[fleqn]{mathtools} 
\usepackage{amssymb,braket,hyperref,xcolor}
\hypersetup{colorlinks, linkcolor={blue!50!black}, citecolor={red!50!black}, urlcolor={blue!80!black}}
\usepackage[title]{appendix}
%\usepackage[sorting=none]{biblatex}
\numberwithin{equation}{section}
\setlength{\parindent}{0pt}
\title{Dissertation Notes}
\date{}
\allowdisplaybreaks
\begin{document} 
\maketitle
\noindent 
%%%%%%%%%%%%%%%%%%%%%%%%%%%%%%%%%%
\section{Perturbations}
%%%%%%%%%%%%%%%%%%%%%%%%%%%%%%%%%%
\begin{itemize}
	\item Fluctuations capture departure from homogeneity and isotropy
	\item ``As an essential feature of the analysis presented here, we assume that during most of the history of the universe all departures from homogeneity and isotropy have been small, so that they can be treated as first-order perturbations."
	\item First given by Lifshitz 1946, created notation $\delta g_{\mu\nu} = h_{\mu\nu}$
	\item Weinberg G\&C 10.9, 15.10 580, Cosmology 235
	\item Maggiore, $x^\mu \to x'^\mu$. $x'^\mu$ must be invertible, differentiable, and with a differential inverse (i.e. an arbitrary diffeomorphism)
\end{itemize}
\end{document}