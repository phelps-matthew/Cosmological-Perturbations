\documentclass[10pt,letterpaper]{article}
\usepackage[textwidth=7in, top=1in,textheight=9in]{geometry}
\usepackage[fleqn]{mathtools} 
\usepackage{amssymb,braket,hyperref,xcolor}
\hypersetup{colorlinks, linkcolor={blue!50!black}, citecolor={red!50!black}, urlcolor={blue!80!black}}
\usepackage[title]{appendix}
%\usepackage[sorting=none]{biblatex}
\numberwithin{equation}{section}
\setlength{\parindent}{0pt}
\title{Dissertation Notes}
\date{}
\allowdisplaybreaks
\begin{document} 
\maketitle
\noindent 

%%%%%%%%%%%%%%%%%%%%%%%%%%%%%%%%%%
\section*{Cosmological Geometries}
%%%%%%%%%%%%%%%%%%%%%%%%%%%%%%%%%%
\begin{itemize}
	\item $k\in \{-1,0,1\}$ topological?
\end{itemize}
%%%%%%%%%%%%%%%%%%%%%%%%%%%%%%%%%%
\section*{Einstein Gravity}
%%%%%%%%%%%%%%%%%%%%%%%%%%%%%%%%%%
\begin{itemize}
	\item Total action $I$ is stationary with respect to arbitrary variation in $g_{\mu\nu}$ if and only if $R_{\mu\nu} - \frac{1}{2}g_{\mu\nu}R = T_{\mu\nu}$ (Weinberg pg. 364)
	\item Can show conservation of $\nabla^\mu T_{\mu\nu}$ by imposing field equation (only holds for stationary paths), or by imposing (scalar) field equations of motion to $T_{\mu\nu}$ directly (holds for stationary paths beyond those satisfying field equations)
\end{itemize}

%%%%%%%%%%%%%%%%%%%%%%%%%%%%%%%%%%
\section*{Perturbations and Gauge Transformations}
%%%%%%%%%%%%%%%%%%%%%%%%%%%%%%%%%%
\begin{itemize}
	\item Small compared to what?
	\item Fluctuations capture departure from homogeneity and isotropy
	\item ``As an essential feature of the analysis presented here, we assume that during most of the history of the universe all departures from homogeneity and isotropy have been small, so that they can be treated as first-order perturbations."
	\item First given by Lifshitz 1946, created notation $\delta g_{\mu\nu} = h_{\mu\nu}$
	\item Weinberg G\&C 10.9, 15.10 580, Cosmology 235
	\item Maggiore, $x^\mu \to x'^\mu$. $x'^\mu$ must be invertible, differentiable, and with a differential inverse (i.e. an arbitrary diffeomorphism)
	\item Brane Localized Mannheim pg. 82
\end{itemize}

%%%%%%%%%%%%%%%%%%%%%%%%%%%%%%%%%%
\section*{SVT3 $\delta G_{\mu\nu}$ in a de Sitter Background}
%%%%%%%%%%%%%%%%%%%%%%%%%%%%%%%%%%
\begin{itemize}
	\item How to address gauge invariants?
\end{itemize}

%%%%%%%%%%%%%%%%%%%%%%%%%%%%%%%%%%
\section*{SVT3 Integral Formulation}
%%%%%%%%%%%%%%%%%%%%%%%%%%%%%%%%%%
\begin{itemize}
	\item Show identical vanishing of surface term upon application of $\partial_i \partial^i$
	\item Wording: harmonic, Laplacian?
	\item Think about conditions requires to vanish
\end{itemize}
\end{document}