\documentclass[10pt,letterpaper]{article}
\usepackage[textwidth=7in, top=1in,textheight=9in]{geometry}
\usepackage[fleqn]{mathtools} 
\usepackage{amssymb,braket,hyperref,xcolor}
\hypersetup{colorlinks, linkcolor={blue!50!black}, citecolor={red!50!black}, urlcolor={blue!80!black}}
\usepackage[title]{appendix}
\usepackage{cite}
\allowdisplaybreaks
\numberwithin{equation}{section}
%\setlength{\parindent}{0pt}
\title{Dissertation Proposal}
\date{November 7, 2019}
\author{Matthew Phelps}
\begin{document} 
\maketitle
\newpage
\tableofcontents
\newpage
%
\begin{abstract}
In the theory of cosmological perturbations, extensive methods of simplifying the equations of motion and eliminating non-physical gauge modes are required in order to construct the perturbative solutions. One approach is to fix the gauge freedom by imposing coordinate constraints. In the context of conformal gravity, we aim to continue work done in obtaining solutions to the cosmological fluctuation equations \cite{Mannheim2012a} by constructing a gauge condition that is conformally invariant. Determination of the appropriate gauge would permit the full set of exact solutions to the fluctuations equations to be obtained. Another method used  extensively in cosmology is the scalar, vector, tensor (SVT) decomposition\cite{Ellis2012}. Here the perturbed metric is decomposed in terms of SO(3) representations whereby it is asserted that the scalars, vectors, and tensors decouple within the equations of motion. We aim to investigate issues of gauge invariance and asymptotic behavior in an integral formalism of the SVT decomposition. Moreover, we propose to characterize the role of boundary conditions in the SVT separation of the equations of motion.
\end{abstract}

%%%%%%%%%%%%%%%%%%%%%%%%%%%%%%%%%%%
\section{Introduction}
%%%%%%%%%%%%%%%%%%%%%%%%%%%%%%%%%%%
To motivate the discussion of the research proposed - the SVT decomposition and choice of gauge - we first will introduce the equations of motion in Einstein gravity, followed by a decomposition of the metric into its background and first order perturbation. We then calculate the linearized Einstein tensor in an arbitrary background and discuss properties of gauge invariance. This sequence is repeated for conformal gravity, along with an outline of particular properties associated with conformal invariance. As an illustrative model we take the background geometry to be Minkowski and introduce the SVT decomposition of the metric perturbation and the linearized Einstein tensor, where discussion of gauge invariance and equation decomposition is continued. Finally, we outline the cosmological geometries of interest and show that each may be expressed in conformal to flat coordinates. 
%
%%%%%%%%%%%%%%%%%%%%%%%%%%%%%%%%%%%
\subsection{Einstein Gravity}
\label{sec:Einstein Gravity}
%%%%%%%%%%%%%%%%%%%%%%%%%%%%%%%%%%%
\indent The formulation of the Einstein field equations first begins by introducing the Einstein-Hilbert action \cite{Weinberg1972}
\begin{eqnarray}
I_{\text{EH}} = -\frac{1}{16\pi G} \int d^4x (-g)^{1/2}  g^{\mu\nu}R_{\mu\nu}.
\end{eqnarray}
Variation of this action with respect to $g_{\mu\nu}$ yields the Einstein tensor
\begin{eqnarray}
\frac{16\pi G}{(-g)^{1/2}} \frac{\delta I_{\text{EH}}}{\delta g_{\mu\nu}}= G^{\mu\nu} = R^{\mu\nu} - \frac{1}{2}g^{\mu\nu}R^\alpha{}_\alpha.
\label{Eintensor}
\end{eqnarray}
Upon specification of a matter action, $I_\text{M}$, an energy momentum tensor may likewise be constructed by variation with respect to the metric,
\begin{eqnarray}
\frac{2}{(-g)^{1/2}} \frac{ \delta I_\text{M}}{\delta g_{\mu\nu}} = T_{\mu\nu}. 
\end{eqnarray}
Requiring the total gravitational + matter action $I_{\text{EH}}+I_\text{M}$ to be stationary under variation of $g_{\mu\nu}$ then yields the Einstein equations of motion
\begin{eqnarray}
R^{\mu\nu} - \frac{1}{2}g^{\mu\nu}R^\alpha{}_\alpha = -8\pi G T^{\mu\nu}.
\label{EinEOM}
\end{eqnarray}
The Einstein tensor itself is covariantly conserved via the Bianchi identities,
\begin{eqnarray}
\nabla_\mu R^{\mu\nu} = \frac{1}{2}\nabla^\nu R^\mu{}_\mu \implies \nabla_\mu G^{\mu\nu} = 0.
\end{eqnarray}

\indent As a first step towards describing fluctuations in the universe, we may decompose the metric $g_{\mu\nu}(x)$ into a background metric and a first order perturbation according to
\begin{eqnarray}
g_{\mu\nu}(x) = g_{\mu\nu}^{(0)}(x) + h_{\mu\nu}(x),\qquad g^{\mu\nu}h_{\mu\nu} \equiv h,
\end{eqnarray}
whereby $G_{\mu\nu}$ can be decomposed as
\begin{eqnarray}
G_{\mu\nu} = G_{\mu\nu}(g_{\mu\nu}^{(0)}) + \delta G_{\mu\nu}(h_{\mu\nu}).
\end{eqnarray}
By virtue of the Bianchi identities, the 10 components of the symmetric $G_{\mu\nu}$ can be reduced to 6 independent components in total. In terms of perturbations of the curvature tensors, the decomposition of $G_{\mu\nu}$ takes the form
\begin{eqnarray}
G_{\mu\nu}^{(0)} &=& R_{\mu\nu}^{(0)} -\frac{1}{2} g_{\mu\nu}^{(0)} R_\alpha^{(0)\alpha}
\label{Einzero}
\\
\delta G_{\mu\nu} &=& \delta R_{\mu\nu} - \frac{1}{2} h_{\mu\nu} R_\alpha^{(0)\alpha} -\frac{1}{2}g_{\mu\nu}\delta R^\alpha{}_\alpha.
\label{Einone}
\end{eqnarray}
Under a coordinate transformation of the form $x^\mu \to x^\mu - \epsilon^\mu(x)$, with $\epsilon^\mu$ of $\mathcal O(h)$, the perturbed metric transforms as
\begin{eqnarray}
h_{\mu\nu} \to h_{\mu\nu} + \nabla_\mu \epsilon_\mu + \nabla_\mu \epsilon_\nu. 
\label{gaugeh}
\end{eqnarray}
For every solution $h_{\mu\nu}$ to $\delta G_{\mu\nu}+8\pi G \delta T_{\mu\nu}$, a transformed $h'_{\mu\nu}=h_{\mu\nu} + \nabla_\mu \epsilon_\mu + \nabla_\mu \epsilon_\nu$ will also serve as a solution - hence the set of four functions $\epsilon^\mu$ serve to define the gauge freedom under coordinate transformation. If the gauge is fixed, the initial 10 components of the symmetric $h_{\mu\nu}$ are reduced to 6 individual components. As will be discussed later, one can also construct gauge invariant quantities as functions of the $h_{\mu\nu}$ and express the field equations entirely in terms of 6 gauge invariant functions (see \hyperref[sec:Scalar, vector, Tensor Decomposition]{Scalar, Vector, Tensor Decomposition}).

\indent As regards the perturbed gravitational and energy momentum tensors, under $x^\mu \to x^\mu - \epsilon^\mu(x)$ they transform as
\begin{eqnarray}
\delta G_{\mu\nu} \to \delta G_{\mu\nu} + {}^{(0)}G^\lambda{}_\mu \nabla_\nu \epsilon_\lambda +  {}^{(0)}G^{\lambda}{}_{\nu}\nabla_\mu \epsilon_\mu + \nabla_\lambda  G^{(0)}_{\mu\nu} \epsilon^\lambda
\nonumber\\
\delta T_{\mu\nu} \to \delta T_{\mu\nu} + {}^{(0)}T^\lambda{}_\mu \nabla_\nu \epsilon_\lambda +  {}^{(0)}T^{\lambda}{}_{\nu}\nabla_\mu \epsilon_\mu + \nabla_\lambda  T^{(0)}_{\mu\nu} \epsilon^\lambda.
\label{gaugetranstensor}
\end{eqnarray}
If $G_{\mu\nu}^{(0)}=0$, then $\delta G_{\mu\nu}$ will be separately gauge invariant. On the other hand, if $ G_{\mu\nu}^{(0)} \ne 0$, then it is only $\delta G_{\mu\nu} + 8\pi G \delta T_{\mu\nu}$ that is gauge invariant by virtue of the background equations of motion. (The transformation behavior of tensors under the gauge transformation as given in \eqref{gaugetranstensor}, otherwise known as the Lie derivative, is in fact generic to all tensors defined on the manifold. One can check that it readily holds for \eqref{gaugeh}). 

With aim towards a description of fluctuations in the universe, let us perturb the metric around an arbitrary background according to
\begin{eqnarray}
ds^2=g_{\mu\nu}dx^\mu dx^\nu = (g^{(0)}_{\mu\nu} + h_{\mu\nu} + \mathcal O(h^2)+ ...)dx^\mu dx^\nu.
\end{eqnarray}
The zeroth order $G_{\mu\nu}^{(0)}$ is given as \eqref{Einzero} and the first order fluctuation evaluates to 
\begin{eqnarray}
\delta G_{\mu\nu} &=& 
- \tfrac{1}{2} h_{\mu \nu } R + \tfrac{1}{2} g_{\mu \nu } h^{\alpha \beta } R_{\alpha \beta } + \tfrac{1}{2} \nabla_{\alpha }\nabla^{\alpha }h_{\mu \nu } -  \tfrac{1}{2} g_{\mu \nu } \nabla_{\alpha }\nabla^{\alpha }h \nonumber\\
&& -  \tfrac{1}{2} \nabla_{\alpha }\nabla_{\mu }h_{\nu }{}^{\alpha } -  \tfrac{1}{2} \nabla_{\alpha }\nabla_{\nu }h_{\mu }{}^{\alpha } + \tfrac{1}{2} g_{\mu \nu } \nabla_{\beta }\nabla_{\alpha }h^{\alpha \beta } + \tfrac{1}{2} \nabla_{\mu }\nabla_{\nu }h.
\label{dEin}
\end{eqnarray}
(Here the covariant derivatives $\nabla$ are defined with respect to the background $g_{\mu\nu}^{(0)}$ and all curvature tensors are taken as zeroth order). For the purpose of illustrating gauge fixing and, later, the SVT decomposition, we evaluate \eqref{dEin} within a Minkowski background viz.
\begin{eqnarray}
ds^2 &=& (\eta_{\mu\nu} + h_{\mu\nu})dx^\mu dx^\nu, \qquad \eta_{\mu\nu} = \text{diag}(-1,1,1,1),\qquad G^{(0)}_{\mu\nu} = 0
\nonumber\\
\delta G_{\mu\nu} &=& \tfrac{1}{2} \nabla_{\alpha }\nabla^{\alpha }h_{\mu \nu } -  \tfrac{1}{2} g_{\mu \nu } \nabla_{\alpha }\nabla^{\alpha }h -  \tfrac{1}{2} \nabla_{\alpha }\nabla_{\mu }h_{\nu }{}^{\alpha } -  \tfrac{1}{2} \nabla_{\alpha }\nabla_{\nu }h_{\mu }{}^{\alpha } + \tfrac{1}{2} g_{\mu \nu } \nabla_{\beta }\nabla_{\alpha }h^{\alpha \beta } + \tfrac{1}{2} \nabla_{\mu }\nabla_{\nu }h.
\label{dEinflat}
\end{eqnarray}
In then taking $\delta T_{\mu\nu}=0$, the equations of motion describing the gravitational fluctuations in an empty universe (vacuum) are given by 
\begin{eqnarray}
0=\tfrac{1}{2} \nabla_{\alpha }\nabla^{\alpha }h_{\mu \nu } -  \tfrac{1}{2} g_{\mu \nu } \nabla_{\alpha }\nabla^{\alpha }h + \tfrac{1}{2} g_{\mu \nu } \nabla_{\beta }\nabla_{\alpha }h^{\alpha \beta } -  \tfrac{1}{2} \nabla_{\mu }\nabla_{\alpha }h_{\nu }{}^{\alpha } -  \tfrac{1}{2} \nabla_{\nu }\nabla_{\alpha }h_{\mu }{}^{\alpha } + \tfrac{1}{2} \nabla_{\nu }\nabla_{\mu }h.
\label{dEinflatEOM}
\end{eqnarray}
\indent In order to solve \eqref{dEinflatEOM}, we have two general approaches: a). Use the coordinate freedom in $h_{\mu\nu}$ to impose a specific gauge, typically one in which the equations of motion are most simplified
b). Determine gauge invariant quantities as functions of $h_{\mu\nu}$ and express the equation of motion entirely in terms of the gauge invariants. 

\indent As an example of using method a) to solve \eqref{dEinflatEOM}, it is common to choose coordinates that satisfy the harmonic gauge condition
\begin{eqnarray}
\nabla^\mu h_{\mu\nu} - \tfrac{1}{2}\nabla_\nu h = 0,
\end{eqnarray} 
whereby \eqref{dEinflatEOM} reduces to 
\begin{eqnarray}
0= \tfrac{1}{2} \nabla_{\alpha }\nabla^{\alpha } \left(h_{\mu\nu} - \tfrac{1}{2} g_{\mu\nu} h\right).
\end{eqnarray}
The trace defines a solution for $h$ whereafter $h_{\mu\nu}$ can be solved as $\nabla_\alpha\nabla^\alpha h_{\mu\nu} = 0$. 

\indent Method b) is facilitated by the use of what is called the Scalar, Vector, Tensor decomposition first introduced in \cite{Lifshitz1946} and later formulated in terms of gauge invariants in \cite{Bardeen1980}. The flat space $\delta G_{\mu\nu}$ of \eqref{dEinflat} given in terms of gauge invariant scalars, vectors, and tensors is discussed in section \hyperref[sec:Scalar, vector, Tensor Decomposition]{Scalar, Vector, Tensor Decomposition}. 
%
%%%%%%%%%%%%%%%%%%%%%%%%%%%%%%%%%%%
\subsection{Conformal Gravity}
\label{sec:Conformal Gravity}
%%%%%%%%%%%%%%%%%%%%%%%%%%%%%%%%%%%

\indent Conformal gravity is a candidate theory of gravitation based on a pure metric action that is not only invariant under local Lorentz transformations (to thus possess the properties of general coordinate invariance and adherenace to the equivalence principle as found in Einstein gravity) but also invariant under local conformal transformations (i.e. transformations of the form $g_{\mu\nu}(x) \to e^{2\alpha(x)}g_{\mu\nu}(x)$ with $\alpha(x)$ arbitrary). Such a metric action that obeys these symmetries is uniquely prescribed and is given by a polynomial function of the Riemann tensor viz.\cite{Mannheim2012a}
\begin{eqnarray}
I_\text{W} &=& -\alpha_g \int d^4x (-g)^{1/2}C_{\lambda\mu\nu\kappa}C^{\lambda\mu\nu\kappa}
\nonumber\\
&\equiv& -2\alpha_g \int d^4x(-g)^{1/2}\left[R_{\mu\nu}R^{\mu\nu} - \tfrac{1}{3}(R^\alpha{}_\alpha)^2\right],
\label{IW}
\end{eqnarray}
where
\begin{eqnarray}
C_{\lambda\mu\nu\kappa}= R_{\lambda\mu\nu\kappa}
-\frac{1}{2}\left(g_{\lambda\nu}R_{\mu\kappa}-
g_{\lambda\kappa}R_{\mu\nu}-
g_{\mu\nu}R_{\lambda\kappa}+
g_{\mu\kappa}R_{\lambda\nu}\right)
+\frac{1}{6}R^{\alpha}_{\phantom{\alpha}\alpha}\left(
g_{\lambda\nu}g_{\mu\kappa}-
g_{\lambda\kappa}g_{\mu\nu}\right)
\end{eqnarray}
is the Weyl tensor and $\alpha_g$ is a dimensionless gravitational coupling constant. Under conformal transformation $g_{\mu\nu}(x) \to e^{2\alpha(x)}g_{\mu\nu}(x)$, the Weyl tensor \cite{Bach1921} transforms as $C^\lambda{}_{\mu\nu\kappa} \to C^\lambda{}_{\mu\nu\kappa}$. Variation of $I_W$ with respect to the metric yields the Bach tensor $W_{\mu\nu}$ according to
\begin{eqnarray}
-\frac{2}{(-g)^{1/2}} \frac{\delta I_{\text{W}}}{\delta g_{\mu\nu}}= 4\alpha_gW^{\mu\nu} = 4\alpha_g \left[2\nabla_\kappa\nabla_\lambda C^{\mu\lambda\nu\kappa} - R_{\kappa\lambda}C^{\mu\lambda\nu\kappa}\right].
\label{Bachtensor}
\end{eqnarray}
In addition to being covariantly conserved $\nabla_\mu W^{\mu\nu} = 0$, the Bach tensor is traceless $g_{\mu\nu}W^{\mu\nu} = 0$ and possesses conformal covariance under conformal transformation viz.
\begin{eqnarray}
g_{\mu\nu} \to \Omega^2(x)g_{\mu\nu} &\implies& W_{\mu\nu} \to \Omega^{-2}W_{\mu\nu}
\nonumber\\
&\implies& W^{\mu\nu} \to \Omega^{-6}W^{\mu\nu}.
\end{eqnarray}
Upon specification of the energy momentum tensor, conformal gravity obeys the equations of motion
\begin{eqnarray}
4\alpha_g W_{\mu\nu} &=& T_{\mu\nu}. 
\end{eqnarray}
The fluctuation $\delta W_{\mu\nu}$ may be evaluated by perturbing around an arbitrary background geometry $ds^2 = (g_{\mu\nu}^{(0)}+h_{\mu\nu})dx^\mu dx^\nu$, which yields
\begin{eqnarray}
\delta W_{\mu\nu} &=&  - \tfrac{1}{6} h_{\mu \nu } R^2 + \tfrac{1}{3} g_{\mu \nu } h^{\alpha \beta } R R_{\alpha \beta } + \tfrac{1}{2} h_{\mu \nu } R_{\alpha \beta } R^{\alpha \beta } -  g_{\mu \nu } h^{\alpha \beta } R_{\alpha }{}^{\gamma } R_{\beta \gamma } + \tfrac{1}{3} h_{\nu }{}^{\alpha } R R_{\mu \alpha } \nonumber \\ 
&& + \tfrac{1}{2} h_{\nu }{}^{\alpha } R_{\alpha \beta } R_{\mu }{}^{\beta } -  \tfrac{2}{3} h^{\alpha \beta } R_{\alpha \beta } R_{\mu \nu } + \tfrac{1}{3} h_{\mu }{}^{\alpha } R R_{\nu \alpha } + h^{\alpha \beta } R_{\mu \alpha } R_{\nu \beta } + \tfrac{1}{2} h_{\mu }{}^{\alpha } R_{\alpha \beta } R_{\nu }{}^{\beta } \nonumber \\ 
&& + h^{\alpha \beta } R_{\nu }{}^{\gamma } R_{\mu \alpha \beta \gamma } -  \tfrac{2}{3} h^{\alpha \beta } R R_{\mu \alpha \nu \beta } -  h_{\nu }{}^{\alpha } R^{\beta \gamma } R_{\mu \beta \alpha \gamma } + 2 h^{\alpha \beta } R_{\alpha }{}^{\gamma } R_{\mu \beta \nu \gamma } + 2 h^{\alpha \beta } R_{\alpha \gamma \beta \zeta } R_{\mu }{}^{\gamma }{}_{\nu }{}^{\zeta } \nonumber \\ 
&& - 2 h^{\alpha \beta } R_{\alpha }{}^{\gamma } R_{\mu \nu \beta \gamma } + h^{\alpha \beta } R_{\mu }{}^{\gamma } R_{\nu \alpha \beta \gamma } -  h_{\mu }{}^{\alpha } R^{\beta \gamma } R_{\nu \beta \alpha \gamma } + \tfrac{1}{3} R \nabla_{\alpha }\nabla^{\alpha }h_{\mu \nu } -  \tfrac{1}{3} g_{\mu \nu } R \nabla_{\alpha }\nabla^{\alpha }h \nonumber \\ 
&& + \tfrac{2}{3} R_{\mu \nu } \nabla_{\alpha }\nabla^{\alpha }h -  \tfrac{1}{6} h_{\mu \nu } \nabla_{\alpha }\nabla^{\alpha }R -  \tfrac{1}{12} g_{\mu \nu } \nabla_{\alpha }R \nabla^{\alpha }h + \nabla_{\alpha }R_{\mu \nu } \nabla^{\alpha }h -  \tfrac{1}{6} \nabla_{\alpha }h_{\mu \nu } \nabla^{\alpha }R \nonumber \\ 
&& + \tfrac{1}{6} g_{\mu \nu } \nabla^{\alpha }R \nabla_{\beta }h_{\alpha }{}^{\beta } - 2 \nabla_{\alpha }h^{\alpha \beta } \nabla_{\beta }R_{\mu \nu } + \tfrac{1}{3} g_{\mu \nu } R \nabla_{\beta }\nabla_{\alpha }h^{\alpha \beta } -  \tfrac{2}{3} R_{\mu \nu } \nabla_{\beta }\nabla_{\alpha }h^{\alpha \beta } \nonumber \\ 
&& + R_{\nu }{}^{\alpha } \nabla_{\beta }\nabla_{\alpha }h_{\mu }{}^{\beta } -  R^{\alpha \beta } \nabla_{\beta }\nabla_{\alpha }h_{\mu \nu } + R_{\mu }{}^{\alpha } \nabla_{\beta }\nabla_{\alpha }h_{\nu }{}^{\beta } + \tfrac{1}{2} g_{\mu \nu } R^{\alpha \beta } \nabla_{\beta }\nabla_{\alpha }h + \tfrac{1}{6} g_{\mu \nu } h^{\alpha \beta } \nabla_{\beta }\nabla_{\alpha }R \nonumber \\ 
&& -  h^{\alpha \beta } \nabla_{\beta }\nabla_{\alpha }R_{\mu \nu } + \tfrac{1}{2} h_{\nu }{}^{\alpha } \nabla_{\beta }\nabla^{\beta }R_{\mu \alpha } + \tfrac{1}{2} h_{\mu }{}^{\alpha } \nabla_{\beta }\nabla^{\beta }R_{\nu \alpha } + \tfrac{1}{2} \nabla_{\beta }\nabla^{\beta }\nabla_{\alpha }\nabla^{\alpha }h_{\mu \nu } \nonumber \\ 
&& -  \tfrac{1}{6} g_{\mu \nu } \nabla_{\beta }\nabla^{\beta }\nabla_{\alpha }\nabla^{\alpha }h + \nabla_{\alpha }R_{\nu \beta } \nabla^{\beta }h_{\mu }{}^{\alpha } + \nabla_{\alpha }R_{\mu \beta } \nabla^{\beta }h_{\nu }{}^{\alpha } -  g_{\mu \nu } R^{\alpha \beta } \nabla_{\gamma }\nabla_{\beta }h_{\alpha }{}^{\gamma } \nonumber \\ 
&& + \tfrac{2}{3} g_{\mu \nu } R^{\alpha \beta } \nabla_{\gamma }\nabla^{\gamma }h_{\alpha \beta } - 2 R_{\mu \alpha \nu \beta } \nabla_{\gamma }\nabla^{\gamma }h^{\alpha \beta } + \tfrac{1}{6} g_{\mu \nu } h^{\alpha \beta } \nabla_{\gamma }\nabla^{\gamma }R_{\alpha \beta } -  h^{\alpha \beta } \nabla_{\gamma }\nabla^{\gamma }R_{\mu \alpha \nu \beta } \nonumber \\ 
&& + \tfrac{1}{6} g_{\mu \nu } \nabla_{\gamma }\nabla^{\gamma }\nabla_{\beta }\nabla_{\alpha }h^{\alpha \beta } + \tfrac{1}{3} g_{\mu \nu } \nabla_{\gamma }R_{\alpha \beta } \nabla^{\gamma }h^{\alpha \beta } - 2 \nabla_{\gamma }R_{\mu \alpha \nu \beta } \nabla^{\gamma }h^{\alpha \beta } -  \nabla_{\beta }R_{\nu \alpha } \nabla_{\mu }h^{\alpha \beta } \nonumber \\ 
&& + \tfrac{1}{6} \nabla^{\alpha }R \nabla_{\mu }h_{\nu \alpha } -  \tfrac{1}{2} \nabla^{\alpha }h \nabla_{\mu }R_{\nu \alpha } + \tfrac{1}{2} \nabla_{\alpha }h^{\alpha \beta } \nabla_{\mu }R_{\nu \beta } -  \tfrac{1}{3} R \nabla_{\mu }\nabla_{\alpha }h_{\nu }{}^{\alpha } -  \tfrac{1}{2} R_{\nu }{}^{\alpha } \nabla_{\mu }\nabla_{\alpha }h \nonumber \\ 
&& -  \tfrac{1}{2} R_{\nu }{}^{\alpha } \nabla_{\mu }\nabla_{\beta }h_{\alpha }{}^{\beta } + R^{\alpha \beta } \nabla_{\mu }\nabla_{\beta }h_{\nu \alpha } -  \tfrac{1}{2} \nabla_{\mu }\nabla_{\beta }\nabla^{\beta }\nabla_{\alpha }h_{\nu }{}^{\alpha } -  \nabla_{\beta }R_{\mu \alpha } \nabla_{\nu }h^{\alpha \beta } + \tfrac{1}{3} \nabla_{\mu }R_{\alpha \beta } \nabla_{\nu }h^{\alpha \beta } \nonumber \\ 
&& + \tfrac{1}{6} \nabla^{\alpha }R \nabla_{\nu }h_{\mu \alpha } + \tfrac{1}{3} \nabla_{\mu }h^{\alpha \beta } \nabla_{\nu }R_{\alpha \beta } -  \tfrac{1}{2} \nabla^{\alpha }h \nabla_{\nu }R_{\mu \alpha } + \tfrac{1}{2} \nabla_{\alpha }h^{\alpha \beta } \nabla_{\nu }R_{\mu \beta } -  \tfrac{1}{3} R \nabla_{\nu }\nabla_{\alpha }h_{\mu }{}^{\alpha } \nonumber \\ 
&& -  \tfrac{1}{2} R_{\mu }{}^{\alpha } \nabla_{\nu }\nabla_{\alpha }h -  \tfrac{1}{2} R_{\mu }{}^{\alpha } \nabla_{\nu }\nabla_{\beta }h_{\alpha }{}^{\beta } + R^{\alpha \beta } \nabla_{\nu }\nabla_{\beta }h_{\mu \alpha } -  \tfrac{1}{2} \nabla_{\nu }\nabla_{\beta }\nabla^{\beta }\nabla_{\alpha }h_{\mu }{}^{\alpha } -  \tfrac{2}{3} R^{\alpha \beta } \nabla_{\nu }\nabla_{\mu }h_{\alpha \beta } \nonumber \\ 
&& + \tfrac{1}{3} R \nabla_{\nu }\nabla_{\mu }h + \tfrac{1}{3} h^{\alpha \beta } \nabla_{\nu }\nabla_{\mu }R_{\alpha \beta } + \tfrac{1}{6} \nabla_{\nu }\nabla_{\mu }\nabla_{\alpha }\nabla^{\alpha }h + \tfrac{1}{3} \nabla_{\nu }\nabla_{\mu }\nabla_{\beta }\nabla_{\alpha }h^{\alpha \beta }.
\label{dWgen}
\end{eqnarray}
Despite the equation of motion being fourth order and containing many more terms as compared to $\delta G_{\mu\nu}$, the trace and conformal properties of $W_{\mu\nu}$ as inherited by $\delta W_{\mu\nu}$ provide immense simplification of the fluctuation equations within particular background geometries (these geometries are detailed in \hyperref[sec:Cosmological Geometries]{Cosmological Geometries}). 

\indent In particular, we may first note that from the tracelessness of $W_{\mu\nu}$ it follows
\begin{eqnarray}
 \delta (g^{\mu\nu}W_{\mu\nu}) = 0\implies g^{\mu\nu}\delta W_{\mu\nu} = h^{\mu\nu}W^{(0)}_{\mu\nu},
 \label{dWtrace}
\end{eqnarray}
and from the conformal covariance of $W_{\mu\nu}$ we have 
\begin{eqnarray}
g_{\mu\nu} \to \Omega^2(x)g_{\mu\nu} &\implies& \delta W_{\mu\nu} \to \Omega^{-2}\delta W_{\mu\nu}
\nonumber\\
&\implies& \delta W^{\mu\nu} \to \Omega^{-6}\delta W^{\mu\nu}.
\label{dWct}
\end{eqnarray}
With \eqref{dWtrace} and \eqref{dWct} it can then be shown that
\begin{eqnarray}
-\tfrac{1}{4}h W^{(0)}_{\mu\nu} = \delta W_{\mu\nu}\left( \tfrac{1}{4} h g_{\mu\nu}\right). 
\end{eqnarray}
From the above properties we deduce that if we select background geometries in which $W^{(0)}_{\mu\nu} =0$, then $g^{\mu\nu}\delta W_{\mu\nu}=0$ and the trace of the metric perturbation, $h$, is decoupled from the equations of motion. In these geometries, $\delta W_{\mu\nu}$ can thus be defined entirely in terms of a 9 component $K_{\mu\nu}$ according to
\begin{eqnarray}
K_{\mu\nu} = h_{\mu\nu} - \tfrac{1}{4}g_{\mu\nu} h.
\end{eqnarray}
Moreover, if the background geometry is conformal to flat
\begin{eqnarray}
ds^2 = \Omega^2(x)(\eta_{\mu\nu} + h_{\mu\nu})dx^\mu dx^\nu,
\end{eqnarray}
then 
\begin{eqnarray}
\delta W_{\mu\nu} = \Omega^{-2} \delta \tilde W_{\mu\nu}.
\end{eqnarray}
Thus in conformal to flat geometries, we may remarkably reduce the general 175 term \eqref{dWgen} to 
\begin{eqnarray}
\delta \tilde W_{\mu\nu} &=& \Omega^{-2} \big[ \tfrac{1}{2}\tilde\nabla_{\beta }\tilde\nabla^{\beta }\tilde\nabla_{\alpha }\tilde\nabla^{\alpha }\tilde K_{\mu \nu } + \tfrac{1}{6} g_{\mu \nu } \tilde\nabla_{\gamma }\tilde\nabla^{\gamma }\tilde\nabla_{\beta }\tilde\nabla_{\alpha }\tilde K^{\alpha \beta } -  \tfrac{1}{2} \tilde\nabla_{\mu }\tilde\nabla_{\beta }\tilde\nabla^{\beta }\tilde\nabla_{\alpha }\tilde K_{\nu }{}^{\alpha }\nonumber\\
&& -  \tfrac{1}{2} \tilde\nabla_{\nu }\tilde\nabla_{\beta }\tilde\nabla^{\beta }\tilde\nabla_{\alpha }\tilde K_{\mu }{}^{\alpha } + \tfrac{1}{3} \tilde\nabla_{\nu }\tilde\nabla_{\mu }\tilde\nabla_{\beta }\tilde\nabla_{\alpha }\tilde K^{\alpha \beta }\big].
\label{dWflat}
\end{eqnarray}
Analogous to our treatment in Einstein gravity, we may take the vacuum equations in flat space ($\Omega=1$ within \eqref{dWflat}) as an example and proceed to solve them by either choosing a particular gauge or decomposing the metric perturbation into scalars, vectors, and tensors in order to form gauge invariant combinations. 

\indent By method a), imposing a gauge, we may solve $\delta \tilde W_{\mu\nu} = 0$ by choosing the transverse gauge according to
\begin{eqnarray}
\tilde\nabla^\mu \tilde K_{\mu\nu} = 0,
\end{eqnarray}
whereby \eqref{dWflat} simplfies to
\begin{eqnarray}
0 = \tfrac{1}{2} \tilde\nabla_{\beta }\tilde\nabla^{\beta }\tilde\nabla_{\alpha }\tilde\nabla^{\alpha }\tilde K_{\mu \nu }.
\end{eqnarray}
%%%%%%%%%%%%%%%%%%%%%%%%%%%%%%%%%%%
\subsection{Cosmological Geometries}
\label{sec:Cosmological Geometries}
%%%%%%%%%%%%%%%%%%%%%%%%%%%%%%%%%%%
The cosmological principle asserts that on a large enough scale, the structure of spacetime is homogeneous and isotropic. Allowing for expansion or contraction of the universe over time, the generic metric that satisfies these criteria is the FLRW \cite{Kodama1984} metric
\begin{eqnarray}
ds^2 = -dt^2 + a(t)^2\left[ \frac{dr^2}{1-kr^2} + r^2d\theta^2 + r^2\sin^2\theta d\theta^2\right]. 
\label{FRLW}
\end{eqnarray}
Here the scale factor $a(t)$ describes the expansion of space in the universe and $k \in \{-1,0,1\}$ describes the global geometry of the universe, being spatially hyperbolic, flat, or spherical respectively. 

\indent In a universe dominated by a cosmological constant (as relevant to inflation), one may solve the Einstein equations $G_{\mu\nu} = -8\pi G \Lambda g_{\mu\nu}$ to determine the requisite metric. The solution is the deSitter geometry, which describes a maximally symmetric space with curvature tensors of the form
\begin{eqnarray}
R_{\lambda\mu\nu\kappa} = \Lambda (g_{\lambda\nu}g_{\nu\kappa}-g_{\lambda\kappa}g_{\mu\nu}),
\qquad R_{\mu\kappa} = -3\Lambda g_{\mu\kappa},\qquad R=-12\Lambda.
\end{eqnarray} 
With deSitter space possessing higher symmetry than the general FLRW space, it is in fact a special case of FRLW as can be seen in the choice of coordinates 
\begin{eqnarray}
ds^2 = -dt^2 + e^{2\Lambda t} (dr^2 + r^2d\theta^2 + r^2\sin^2\theta d\phi^2),
\end{eqnarray}
which corresponds to $k=0$, $a(t) = e^{2\Lambda t}$ within \eqref{FRLW}. 

\indent A remarkable aspect about the FRLW geometry (and dS4 by extension) is that in each global geometry (hyperbolic, flat, spherical) the space can be written in conformal to flat form. As a simple example, if we take the $k=0$ (flat 3-space) metric of \eqref{FRLW} according to
\begin{eqnarray}
ds^2 = -dt^2 + a(t)^2\left[ dr^2 + r^2d\theta^2 + r^2\sin^2\theta d\theta^2\right],
\end{eqnarray}
then in transforming the time coordinate via $\tau = \int \frac{dt}{a(t)}$, the geometry may be written in the form
\begin{eqnarray}
ds^2 = a(\tau^2)( -d\tau^2 + dr^2 + r^2d\theta^2 + r^2\sin^2\theta d\phi^2). 
\end{eqnarray}
If we are interested in the $k=-1/L^2$ hyperbolic space, a proper choice of coordinates \cite{Mannheim2012a} allows the FRLW to be expressed as
\begin{eqnarray}
ds^2=\frac{4L^2 a^2(p',r')}{[1-(p'+r')^2][1-(p'-r')^2]} \left[ -dp'^2 + dr'^2+r'^2 d\theta^2 + r'^2 \sin^2\theta d\phi^2\right]
\end{eqnarray}
whereas for the $k=1/L^2$ spherical 3-space \eqref{FRLW} takes the form
\begin{eqnarray}
ds^2=\frac{4L^2 a^2(p',r')}{[1+(p'+r')^2][1+(p'-r')^2]} \left[ -dp'^2 + dr'^2+r'^2 d\theta^2 + r'^2 \sin^2\theta d\phi^2\right].
\end{eqnarray}
Since all the cosmological geometries of interest possess a coordinate expression where the space is conformal to flat, within such geometries the background Bach tensor vanishes $W^{(0)}_{\mu\nu} = 0$ to thus render $\delta W_{\mu\nu}$ to independently be a gauge invariant tensor, i.e. without reference to the equation of motion. 
%
%%%%%%%%%%%%%%%%%%%%%%%%%%%%%%%%%%%
\subsection{Scalar, Vector, Tensor Decomposition}
\label{sec:Scalar, vector, Tensor Decomposition}
%%%%%%%%%%%%%%%%%%%%%%%%%%%%%%%%%%%
In the field of perturbative cosmology, it is standard to first introduce a 3+1 decomposition of the metric perturbation followed by a decomposition into SO(3) scalars, vectors, and tensors (the SVT decomposition)\cite{Ellis2012}. For fluctuations around a Minkowski background the decomposition takes the form
\begin{eqnarray}
ds^2 =  (\eta_{\mu\nu} + h_{\mu\nu})dx^\mu dx^\nu
\end{eqnarray}
where
\begin{eqnarray}
h_{00} = -2\phi,\quad h_{0i} = \nabla_i B + B_i,\qquad h_{ij} = -2\delta_{ij}\psi + 2\nabla_i\nabla_j E + \nabla_i E_j + \nabla_j E_i + 2 E_{ij}. 
\label{SVTflat}
\end{eqnarray}
The SVT elements are required to obey 
\begin{eqnarray}
\nabla^a B_a =0,\qquad \nabla^a E_a = 0,\qquad \nabla^a E_{ab} = 0,\qquad \delta^{ab}E_{ab} = 0.
\end{eqnarray}
Counting the degrees of freedom, we have four scalars ($\phi$, $\psi$, $E$, $B$), two 2-component transverse vectors ($B_i$, $E_i$), and one transverse traceless 2-component tensor ($E_{ij}$), totaling 10 degrees of freedom. The decomposition of $h_{\mu\nu}$ into irreducible SO(3) components was first shown by Lifshitz \cite{Lifshitz1946} and much later Bardeen \cite{Bardeen1980} then formulated the set of gauge invariant SVT quantities, given in flat space as
\begin{eqnarray}
\psi,\qquad \phi + \dot B-\ddot E,\qquad B_i-\dot E_i,\qquad E_{ij}. 
\label{flatgaugeinvs}
\end{eqnarray}
Expressing $\delta G_{\mu\nu}$ as given in \eqref{dEinflat} according to \eqref{SVTflat} indeed illustrates that the perturbed Einstein tensor (a quantity that is itself gauge invariant in a Minkowski background) can be expressed entirely in terms of the gauge invariants \eqref{flatgaugeinvs}
\begin{eqnarray}
\delta G_{00}&=&- 2 \delta^{ab} {\nabla}_{b}{\nabla}_{a}\psi,
\nonumber\\
\delta G_{0i}&=&- 2 {\nabla}_{i}\dot{\psi}+ \tfrac{1}{2} \delta^{ab} {\nabla}_{b}{\nabla}_{a}(B_{i} -  \dot{E}_{i}),
\nonumber\\
\delta G_{ij}&=&- 2 \delta_{ij} \ddot{\psi} -  \delta^{ab} \delta_{ij} {\nabla}_{b}{\nabla}_{a}(\phi+\dot{B}  -\ddot{E})+ \delta^{ab} \delta_{ij} {\nabla}_{b}{\nabla}_{a}\psi 
+ {\nabla}_{j}{\nabla}_{i}(\phi+\dot{B} -  \ddot{E})  -  {\nabla}_{j}{\nabla}_{i}\psi
\nonumber\\
&+& \tfrac{1}{2} {\nabla}_{i}(\dot{B}_{j} - \ddot{E}_{j}) + \tfrac{1}{2} {\nabla}_{j}(\dot{B}_{i}  
- \ddot{E}_{i})- \ddot{E}_{ij} + \delta^{ab} {\nabla}_{b}{\nabla}_{a}E_{ij},
\nonumber\\
g^{\mu\nu}\delta G_{\mu\nu}&=&-\delta G_{00}+\delta^{ij}\delta G_{ij}=4 \delta^{ab} {\nabla}_{b}{\nabla}_{a}\psi -6\ddot{\psi}-2 \delta^{ab} {\nabla}_{b}{\nabla}_{a}(\phi+\dot{B}  -\ddot{E}).
\label{deinSVTflat}
\end{eqnarray}
While the expression of the field equations in terms of gauge invariant quantities proves very beneficial in extracting the desired physical modes, the great efficacy of the SVT decomposition theorem is that the equations of motion themselves are taken to decouple into equations composed separately of scalars, vectors, and tensors. Such a decomposition at the level of the equations of motion is a primary topic of proposed research and is discussed more detail in \hyperref[sec:Equation of Motion Decomposition]{Equation of Motion Decomposition}. 
%
%%%%%%%%%%%%%%%%%%%%%%%%%%%%%%%%%%%
\section{Proposed Research}
\label{sec:Proposed Research}
%%%%%%%%%%%%%%%%%%%%%%%%%%%%%%%%%%%
%
%%%%%%%%%%%%%%%%%%%%%%%%%%%%%%%%%%%
\subsection{Conformal Gauge}
\label{sec:Conformal Gauge}
%%%%%%%%%%%%%%%%%%%%%%%%%%%%%%%%%%%
In the work of \cite{Mannheim2012a}, the perturbed Bach tensor was first evaluated in a deSitter geometry according to
\begin{eqnarray}
ds^2 = (g_{\mu\nu}+h_{\mu\nu})dx^\mu dx^\nu,
\end{eqnarray}
with background metric $g_{\mu\nu}$ obeying
\begin{eqnarray}
R_{\lambda\mu\nu\kappa} = H^2 (g_{\lambda\nu}g_{\nu\kappa}-g_{\lambda\kappa}g_{\mu\nu}),
\qquad R_{\mu\kappa} = -3H^2 g_{\mu\kappa},\qquad R=-12H^2.
\end{eqnarray} 
Prior to choosing a gauge or making a conformal transformation, the author \cite{Mannheim2012a} was able to show that the trace of the fluctuation drops out and $\delta W_{\mu\nu}$ can be expressed entirely in terms of $K_{\mu\nu}$ as the compact form
\begin{eqnarray}
\delta W_{\mu\nu} &=& \tfrac{1}{2}\left[\nabla_\alpha\nabla^\alpha - 4H^2\right]\left[ \nabla_\beta\nabla^\beta-2H^2\right]K_{\mu\nu} - \tfrac{1}{2}\left[\nabla_\beta\nabla^\beta-4H^2\right]\left[\nabla_\mu\nabla_\lambda K^\lambda{}_\nu + \nabla_\nu\nabla_\lambda K^\lambda{}_\mu\right]
\nonumber\\
&&+\tfrac{1}{6} \left[g_{\mu\nu} \nabla_\alpha\nabla^\alpha + 2\nabla_\mu\nabla_\nu - 6H^2g_{\mu\nu}\right]\nabla_\kappa\nabla_\lambda K^{\kappa\lambda}.
\label{dwdS41}
\end{eqnarray}
Upon imposing the transverse gauge, $\delta W_{\mu\nu}$ then simplifies to 
\begin{eqnarray}
\delta W_{\mu\nu} = \tfrac{1}{2}\left[\nabla_\alpha\nabla^\alpha - 4H^2\right]\left[ \nabla_\beta\nabla^\beta-2H^2\right]K_{\mu\nu}.
\label{dwdS42}
\end{eqnarray}
Given that the covariant box operator $\nabla_\alpha\nabla^\alpha K_{\mu\nu}$ mixes the indices of $K_{\mu\nu}$, decoupling the components to provide a set of solutions to \eqref{dwdS42} required the use of an additional set of constraints, namely the synchronous gauge condition
\begin{eqnarray}
K_{0\mu} = 0.
\end{eqnarray}
In evaluating \eqref{dwdS42} in the deSitter background according to
\begin{eqnarray}
ds^2 = \left( \frac{1}{H^2\tau^2}\right) (-d\tau^2 + dr^2 + r^2d\theta^2 + r^2\sin^2\theta d\phi^2),
\end{eqnarray}
for fluctuations that obey the synchronous gauge the author \cite{Mannheim2012a} found solutions as
\begin{eqnarray}
K_{\mu\nu}(\tau,\mathbf x) = \frac{1}{\tau^2 H^2} A_{\mu\nu} e^{i\mathbf{q}\cdot\mathbf{x}-iq\tau}+\frac{1}{\tau H^2}B_{\mu\nu}  e^{i\mathbf{q}\cdot\mathbf{x}-iq\tau}
\end{eqnarray}
The first solution proportional to $A_{\mu\nu}$ is in fact the same solution generated in standard Einstein gravity, while the second solution proportional to $B_{\mu\nu}$ is a result unique to the fourth order equations of motion. 

\indent In addition to the deSitter case, the  author \cite{Mannheim2012a} also evaluates $\delta W_{\mu\nu}$ in the FRLW background geometries of \eqref{FRLW} for $k\in\{-1,0,1\}$. Here, the author \cite{Mannheim2012a} first evaluates $\delta W_{\mu\nu}$ as given in an arbitrary flat curvilinear coordinate system via taking \eqref{dwdS42} with $H=0$. Upon then imposing the transverse gauge in flat coordiantes $\tilde\nabla^\mu \tilde K_{\mu\nu} = 0$, $\delta \tilde W_{\mu\nu}$ reduces to 
\begin{eqnarray}
\delta \tilde W_{\mu\nu} &=& \tfrac{1}{2}\tilde\nabla_\alpha\tilde\nabla^\alpha\tilde\nabla_\beta\nabla^\beta \tilde K_{\mu\nu},
\end{eqnarray}
where $\tilde\nabla$ is defined with respect to the flat curvlinear coordinate system
\begin{eqnarray}
ds^2 = -dp^2 + dr^2 +r^2d\theta^2 + r^2\sin^2\theta d\phi^2.
\end{eqnarray}
The author \cite{Mannheim2012a} then constrains the fluctuations according to 
\begin{eqnarray}
K_{p\mu} = 0,\qquad K_{r\mu} = 0.
\label{dscurvflat}
\end{eqnarray}
Finally, upon performing a conformal transformation to bring \eqref{dscurvflat} to the RW forms given in \hyperref[sec:Cosmological Geometries]{Cosmological Geometries}, the components of $\tilde K_{\mu\nu}$ are used to generate the desired solutions of the angular components of the full $K_{\mu\nu}$. As an example, the positive frequency $K_{\theta\theta}$ solution to the $k=-1$ RW geometry takes the form
\begin{eqnarray}
K_{\theta\theta}(p,r,\theta,\phi)&=& L^2 a(p,r)^2 \cosh^2[(p+\chi)/2]\cosh^2[(p-\chi)/2][2\alpha_{\theta\theta} + \beta_{\theta\theta}\tanh[(p+\chi)/2]+\beta_{\theta\theta}\tanh[(p-\chi)/2]]
\nonumber\\
&&\times[\tanh[(p+\chi)/2]-\tanh[(p-\chi)/2]]\times \frac{\exp[-iq\tanh[(p-\chi)/2]]}{\sin^2\theta},
\end{eqnarray}
where $r=\sinh \chi$. 

\indent The aim of proposed research regarding this topic is to find the full set of solutions to the perturbed Bach tensor in both deSitter and FRLW geometries. As mentioned, in both geometries the author \cite{Mannheim2012a} found restricted sets of solutions; synchronous solutions in the case of deSitter and solutions obeying a conformal analog of $K_{r\mu} = K_{p\mu} = 0$ in the FRLW geometry. We anticipate that the simplification of the equation of motion for $\delta W_{\mu\nu}$ may be facilitated by a choice of gauge condition that is conformal invariant. It can be shown that the transverse gauge $\nabla^\mu K_{\mu\nu} = 0$ does not transform conformally and thus does not serve to diagonalize the equations of motion (i.e. a decoupling of the components of $K_{\mu\nu}$). However, a choice of gauge that serves to diagonalize the equations of motion in flat space and retains its form under conformal transformation may allow the equations in the full conformal to flat background to preserve the diagonalization. In such a scenario, we anticipate the solution of the metric perturbation can be solved in full and exactly, without recourse to auxiliary constraints. 
%%%%%%%%%%%%%%%%%%%%%%%%%%%%%%%%%%%
\subsection{S.V.T. Formalism}
\label{sec:S.V.T. Formalism}
%%%%%%%%%%%%%%%%%%%%%%%%%%%%%%%%%%%
%
%%%%%%%%%%%%%%%%%%%%%%%%%%%%%%%%%%%
\subsubsection{Integral Form and Generalization to Curved Geometries}
\label{sec:Integral Form and Generalization to Curved Geometries}
%%%%%%%%%%%%%%%%%%%%%%%%%%%%%%%%%%%
One initial aim of the proposed research regarding the SVT formalism is to invert the expressions of the SVT decomposition of $h_{\mu\nu}$
\begin{eqnarray}
h_{00} = -2\phi,\quad h_{0i} = \nabla_i B + B_i,\qquad h_{ij} = -2\delta_{ij}\psi + 2\nabla_i\nabla_j E + \nabla_i E_j + \nabla_j E_i + 2 E_{ij},
\end{eqnarray}
such that we may express each SVT quantity in terms of $h_{\mu\nu}$. Towards this end, we first note that the $h_{00}$ term is trivial as it may be expressed as $\phi = -\tfrac{1}{2}h_{00}$. The next term, $B$, can be expressed in terms of $h_{0i}$ according to
\begin{eqnarray}
\nabla^i h_{0i} = \nabla_a\nabla^a B.
\end{eqnarray}
From here, we take $B$ as the solution
\begin{eqnarray}
B = \int d^3x' D(x,x') \nabla^a h_{0a},
\label{Bsol}
\end{eqnarray}
where we have introduced the flat space Greens function $D(x,x')$ which obeys
\begin{eqnarray}
\nabla_a\nabla^a D(x,x') = \delta (x-x').
\end{eqnarray}
Using \eqref{Bsol}, we may then express $B_i$ as
\begin{eqnarray}
B_i = h_{0i} - \nabla_i B = h_{0i} - \nabla_i \int d^3x' D(x,x') \nabla^a h_{0a}. 
\label{Bisol}
\end{eqnarray}
Thus for the $B$ and $B_i$ term, the above decomposition serves to define both SVT quantities as functions of $h_{\mu\nu}$ - non-local integrals in this case - whereby we may confirm that $B_i$ is indeed transverse. Is this decomposition of $h_{0i}$ into longitudinal and transverse components unique? It is in fact so, since we have first defined $B$ as a scalar that can never be transverse (the only $\nabla_i B$ that is transverse is one in which $h_{0i}$ itself vanishes) and then have taken $h_{0i} - \nabla_i B$ to define the complement, i.e. the transverse component. 

\indent The above approach may in fact be used to decompose any vector into its transverse and longitudinal components. More formally, we may cast the method into the language of non-local projectors. Let us define the transverse projector
\begin{eqnarray}
\hat\Pi_{ij}  = \delta_{ij} - \nabla_i \int d^3x' D(x,x')\nabla_j,
\label{vectorproj}
\end{eqnarray}
in which \eqref{Bisol} can be expressed as
\begin{eqnarray}
B_i = \hat\Pi_{ij}h^{0j} = h_{0i}^T. 
\end{eqnarray}
We may check that the transverse vector projector \eqref{vectorproj} obeys the projector algebra, namely
\begin{eqnarray}
\hat\Pi_{ij}\hat\Pi^{j}{}_k = \hat\Pi_{ik},\qquad \hat\Pi_{ij}A^{Tj} = A_i^T,\qquad \hat\Pi_{ij}A^{Lj} = 0. 
\end{eqnarray}
As given in \cite{Mannheim2005}, it can be shown that in flat space transverse and longitudinal tensor projectors may be defined as
\begin{eqnarray}
T_{ijkl} &=& \delta_{ik}\delta_{jl} - \nabla_i \int d^3x' D(x,x') \delta_{jl}\nabla_k - \nabla_j \int d^3x' D(x,x')\delta_{ik}\nabla_l
\nonumber\\
&&+\nabla_i\nabla_j \int d^3x' D(x,x')\nabla_k \int d^3x'' D(x',x'')\nabla_l
\\ 
L_{ijkl} &=&   \nabla_i \int d^3x' D(x,x') \delta_{jl}\nabla_k +  \nabla_j \int d^3x' D(x,x')\delta_{ik}\nabla_l\nonumber\\
&&-\nabla_i\nabla_j \int d^3x' D(x,x')\nabla_k \int d^3x'' D(x',x'')\nabla_l,
\end{eqnarray}
which are implemented as
\begin{eqnarray}
L_{ijkl}h^{kl} = h_{ij}^L,\qquad T_{ijkl}h^{kl} = h^T_{ij}
\end{eqnarray}
and obey the projector algebra
\begin{eqnarray}
T_{ijkl}T^{kl}{}_{mn} = T_{ijmn},\qquad L_{ijkl}L^{kl}{}_{mn} = L_{ijmn},
\qquad T_{ijkl}L^{kl}{}_{mn} = 0,\qquad L_{ijkl} + T_{ijkl}= \delta_{ik}\delta_{jl}.
\end{eqnarray}
As defined, we may express $h^T_{ij}$ as
\begin{eqnarray}
h_{ij}^T &=& h_{ij} - \nabla_i \int d^3x' D(x,x') \nabla^k h_{jk} - \nabla_j \int d^3x' D(x,x')\nabla^l h_{il}
\nonumber\\
&&+\nabla_i\nabla_j \int d^3x' D(x,x')\nabla_k \int d^3x' D(x',x'')\nabla^l h_{kl}.
\end{eqnarray}
Given the present decompositions, we have been able to express $\phi$, $B$, and $B_i$ in terms of $h_{\mu\nu}$ but it remains to invert the remaining SVT quantities ($\psi$, $E$, $E_i$ and $E_{ij}$). With $E_{ij}=h_{ij}^{T\theta}$ being the transverse and traceless component of $h_{ij}$, we anticipate another projector similar to $T_{ijkl}$ is required - one which projects the traceless component of a transverse tensor. 

\indent Presuming that all SVT components can be expressed in terms of integrals of $h_{\mu\nu}$, one still has the limitation that the above decomposition is only defined in flat space. As we move to the curved space geometries of $\text{dS}_4$ and FRLW, covariant derivatives obey the commutation relation
\begin{eqnarray}
[ \nabla_\mu\nabla^\mu ,\nabla_\nu] A = R_{\nu\mu}\nabla^\mu  A.
\end{eqnarray}
As such, we can see that the covariant extension of $T_{ijkl}$ will not suffice to serve as a transverse projector. Thus the generalization of the integral SVT formulation in maximally symmetric and conformal to flat geometries will be required. 
%
%%%%%%%%%%%%%%%%%%%%%%%%%%%%%%%%%%%
\subsubsection{SVT Gauge Invariance}
\label{sec:SVT Gauge Invariance}
%%%%%%%%%%%%%%%%%%%%%%%%%%%%%%%%%%%
It of also of interest to inquire what conditions permit the SVT decomposition in the first place. Returning to the decomposition of $h_{0i}$ we have
\begin{eqnarray}
h_{0i} &=& B_i + \nabla_i B,\qquad B=\int d^3x' D(x,x')\nabla^a h_{0a},\qquad B_i = h_{0i} - \nabla_i \int d^3x' D(x,x')\nabla^a h_{0a}.
\end{eqnarray}
As a step towards analyzing the scalar $B$, we integrate over Greens identity
\begin{eqnarray}
(\nabla_a\nabla^a D(x,x'))B &=& D(x,x')\nabla_a\nabla^a B + \nabla^a[(\nabla_a D(x,x'))B-D(x,x')\nabla_a B]
\nonumber\\
B &=& \underbrace{\int d^3x' D(x,x')\nabla_a\nabla^a B}_{B^{NH}} + \underbrace{\oint dS^a [(\nabla_a D(x,x'))B-D(x,x')\nabla_a B]}_{B^{H}}
\label{Bdecomp}
\end{eqnarray}
With $\nabla_a\nabla^a$ annihilating the surface term, the integral expression represents the decomposition of $B$ into its harmonic and non-harmonic parts (harmonic according to math convention, meaning $\nabla^2 f(x) =0$). If we insert the relation $\nabla^a h_{0a} = \nabla_a\nabla^a B$ into \eqref{Bdecomp}, it follows
\begin{eqnarray}
B &=& B + \oint dS^a [(\nabla_a D(x,x'))B-D(x,x')\nabla_a B] \implies \oint dS^a [(\nabla_a D(x,x'))B-D(x,x')\nabla_a B] =0.
\end{eqnarray}
Consequently, our definition of $B$ in terms of $h_{0i}$ yields a $B$ whose asymptotic behavior is constrained. We propose to determine the requisite asymptotic behavior of all SVT quantities in order to properly characterize the structure of the SVT decomposition.

\indent In addition to its asymptotic behavior, it is of interest to analyze the behavior of the SVT quantities under gauge transformations. Continuing in a Minkowski background, we utilize \eqref{gaugeh} to yield the transformation
\begin{eqnarray}
h_{0i} \to h_{0i} + \nabla_i \epsilon_0  + \dot \epsilon_i,
\end{eqnarray}
where the overdot denotes the time derivative $\nabla_0 = \partial_0$. It follows that $B$ transforms under gauge transformations as
\begin{eqnarray}
B \to B + \int d^3x' D(x,x')(\nabla_i \epsilon_0  + \dot \epsilon_i). 
\end{eqnarray}
Hence, in order to form gauge invariants constructed from combinations of SVT quantities (cf. \eqref{flatgaugeinvs}) we are presented with a few possible scenarios: a) Appropriate combinations of SVT quantities will permit all gauge terms to naturally cancel at the integral level, b) We must assume conditions that allow SVT defintions to be integrable by parts (i.e. conditions on their asymptotic behavior) and explore the resultant gauge behavior, or c) Gauge invariants must in fact be expressed as derivative relations of SVT variables. It is our aim to explore each of these scenarios in order to properly determine and characterize the gauge invariants composed from the SVT decomposition.

%%%%%%%%%%%%%%%%%%%%%%%%%%%%%%%%%%%
\subsubsection{$D=4$ SVT Decomposition}
\label{sec:SVT Decomposition}
%%%%%%%%%%%%%%%%%%%%%%%%%%%%%%%%%%%
In defining the SVT decomposition thus far, we have first performed a 3+1 splitting upon the components of $h_{\mu\nu}$, from which it follows that the equations of motion themselves are also 3+1 decomposed (cf. \eqref{deinSVTflat}). It is of interest to inquire what form the SVT decomposition would take if we decomposed $h_{\mu\nu}$ into scalar, vectors, and tensors that transform under the full Poincare group. One can deduce that such a decomposition would take the form
\begin{eqnarray}
h_{\mu\nu} &=& -2g_{\mu\nu} \chi + 2\nabla_\mu\nabla_\nu F + \nabla_\mu F_\nu + \nabla_\nu F_{\mu} + 2 F_{\mu\nu} 
\end{eqnarray}
where 
\begin{eqnarray}
\nabla^\mu F_{\mu} = 0,\qquad \nabla^\mu F_{\mu\nu} = 0,\qquad g^{\mu\nu} F_{\mu\nu} = 0. 
\end{eqnarray}
Counting the degrees of freedom, we have two scalars ($\chi$, $F$), one 3-component transverse vector $F_\mu$, and one 5-component transverse traceless tensor $F_{\mu\nu}$, totaling 10 individual components. 

\indent In this four dimensional decomposition, it remains to determine a) which combinations of the SVT components are gauge invariant, b) the analogous integral relations of the SVT quantities in terms of $h_{\mu\nu}$, and c) the form of the equations of motion for $\delta G_{\mu\nu}$ and $\delta W_{\mu\nu}$. Given that $\delta W_{\mu\nu}$ can be expressed as the 5 component $K_{\mu\nu}$ in conformal flat geometries, we anticipate that $\delta W_{\mu\nu}$ may be expressed entirely in terms of $F_{\mu\nu}$, with $F_{\mu\nu}$ serving as the gauge invariant. It is possible that four dimensional formalism may provide additional simplification to the equations of motion, most especially for spaces of maximal symmetry (Minkowski and $\text{dS}_4$). 
%
%%%%%%%%%%%%%%%%%%%%%%%%%%%%%%%%%%%
\subsubsection{Equation of Motion Decomposition}
\label{sec:Equation of Motion Decomposition}
%%%%%%%%%%%%%%%%%%%%%%%%%%%%%%%%%%%
As stated in \hyperref[sec:Scalar, vector, Tensor Decomposition]{Scalar, vector, Tensor Decomposition}, a primary aim of the proposed research is to assess whether the equations of motion themselves separate into equations involving only scalars, vectors, or tensors. As a schematic example, we recall the SVT form of $\delta G_{\mu\nu}$ within a Minkowski background 
\begin{eqnarray}
\delta G_{00}&=&- 2 \delta^{ab} {\nabla}_{b}{\nabla}_{a}\psi,
\nonumber\\
\delta G_{0i}&=&- 2 {\nabla}_{i}\dot{\psi}+ \tfrac{1}{2} \delta^{ab} {\nabla}_{b}{\nabla}_{a}(B_{i} -  \dot{E}_{i}),
\nonumber\\
\delta G_{ij}&=&- 2 \delta_{ij} \ddot{\psi} -  \delta^{ab} \delta_{ij} {\nabla}_{b}{\nabla}_{a}(\phi+\dot{B}  -\ddot{E})+ \delta^{ab} \delta_{ij} {\nabla}_{b}{\nabla}_{a}\psi 
+ {\nabla}_{j}{\nabla}_{i}(\phi+\dot{B} -  \ddot{E})  -  {\nabla}_{j}{\nabla}_{i}\psi
\nonumber\\
&+& \tfrac{1}{2} {\nabla}_{i}(\dot{B}_{j} - \ddot{E}_{j}) + \tfrac{1}{2} {\nabla}_{j}(\dot{B}_{i}  
- \ddot{E}_{i})- \ddot{E}_{ij} + \delta^{ab} {\nabla}_{b}{\nabla}_{a}E_{ij},
\nonumber\\
g^{\mu\nu}\delta G_{\mu\nu}&=&-\delta G_{00}+\delta^{ij}\delta G_{ij}=4 \delta^{ab} {\nabla}_{b}{\nabla}_{a}\psi -6\ddot{\psi}-2 \delta^{ab} {\nabla}_{b}{\nabla}_{a}(\phi+\dot{B}  -\ddot{E}).
\end{eqnarray}
Taking the equations of motion in vacuum $\delta G_{\mu\nu} = 0$, the SVT decomposition theorem asserts that we may separate the equations into scalars, vectors, and tensors according to
\begin{eqnarray}
&&- 2 \delta^{ab} {\nabla}_{b}{\nabla}_{a}\psi=0,
\nonumber\\
&&- 2 {\nabla}_{i}\dot{\psi}=0,\quad \tfrac{1}{2} \delta^{ab} {\nabla}_{b}{\nabla}_{a}(B_{i} -  \dot{E}_{i})=0,
\nonumber\\
&& -2 \delta_{ij} \ddot{\psi} -  \delta^{ab} \delta_{ij} {\nabla}_{b}{\nabla}_{a}(\phi+\dot{B}  -\ddot{E})+ \delta^{ab} \delta_{ij} {\nabla}_{b}{\nabla}_{a}\psi +{\nabla}_{j}{\nabla}_{i}(\phi+\dot{B} -  \ddot{E})  -  {\nabla}_{j}{\nabla}_{i}\psi=0.
\nonumber\\
&& \tfrac{1}{2} {\nabla}_{i}(\dot{B}_{j} - \ddot{E}_{j}) + \tfrac{1}{2} {\nabla}_{j}(\dot{B}_{i} 
- \ddot{E}_{i})=0,
\nonumber\\
&&- \ddot{E}_{ij} + \delta^{ab} {\nabla}_{b}{\nabla}_{a}E_{ij}=0. 
\label{deineomdecomp}
\end{eqnarray}
On the other hand, one may take derivatives of $\delta G_{\mu\nu}$ to obtain 
\begin{eqnarray}
&&\delta^{ab} {\nabla}_{b}{\nabla}_{a}\psi=0,
\nonumber\\
&&\delta^{ab} {\nabla}_{b}{\nabla}_{a} \delta^{cd} {\nabla}_{c}{\nabla}_{d}(\phi+\dot{B}  -\ddot{E})=0,
\nonumber\\
&&\delta^{ab} {\nabla}_{b}{\nabla}_{a} \delta^{cd} {\nabla}_{c}{\nabla}_{d}(B_i-\dot{E}_i)=0,
\nonumber\\
&&\delta^{ab} {\nabla}_{b}{\nabla}_{a} \delta^{cd} {\nabla}_{c}{\nabla}_{d}(-\ddot{E}_{ij}+\delta^{ef} {\nabla}_{e}{\nabla}_{f}E_{ij})=0. 
\label{deineomdecomp2}
\end{eqnarray}
Without recourse to the decomposition theorem, we are required to go to fourth order derivatives in order to separate the equations of motion. The bulk aim of the proposed research is to determine under what conditions does an analogous \eqref{deineomdecomp2} imply an analogous \eqref{deineomdecomp}. We anticipate the SVT decomposition of the equations of motion is strictly dependent on boundary conditions, and to this end we will investigate the separation of the equations of motion in the following geometries: i) Minkowski, ii) deSitter, iii) $k=0$ FLRW, iv) $k=-1$ FLRW, v) $k=1$ FLRW. In addition, we will repeat the aforementioned analysis in the proposed $D=4$ SVT basis, where we anticipate new constraints as the boundary conditions depend on both space and time. 
\newpage
%
%%%%%%%%%%%%%%%%%%%%%%%%%%%%%%%%%%%
\section*{Summary}
\label{sec:Summary}
%%%%%%%%%%%%%%%%%%%%%%%%%%%%%%%%%%%
\indent Given the nonlinear nature of the differential equations governing cosmological fluctuations in the universe, we have seen that extracting exact physical solutions requires considerable simplification of the equations of motion. In the context of cosmology, this simplification procedure is primarily carried out via two methods: finding a suitable gauge or expressing the equations of motion entirely in terms of gauge invariant quantities. 

\indent As to the choice of gauge, one goal of the proposed research is to continue the study of cosmological perturbations in conformal gravity \cite{Mannheim2012a} where the author has determined a restricted set of solutions of the perturbed metric. By imposing a suitable gauge condition that adheres to the symmetry of the theory (conformal invariance), we anticipate that the full set of exact solutions to the fluctuation equations may be obtained. Thereafter, the solutions may be contrasted with solutions obtained in Einstein gravity to possibly provide a diagnostic between the two gravitational theories. 

\indent In regards to the expression of gauge invariant quantities, the SVT decomposition has been widely adopted in cosmology as it provides a convenient means of determining gauge invariants \cite{Kodama1984}\cite{Ellis2012} and, more importantly, permits the equations of motion themselves to decouple into scalar, vector, and tensor equations. Our research goals broadly entail a detailed study into the SVT decomposition. In particular, we aim to i) provide an integral formulation of the decomposition, ii) generalize this formulation to curved space and to four dimensions, iii) determine the gauge dependent relationships of the SVT components to thereby characterize the form and behavior of gauge invariant quantities, and lastly iv) investigate the interplay of boundary conditions within the separation of the equations of motion as applied to Einstein and conformal gravity for all cosmological geometries of interest. 

\begin{thebibliography}{99}
	
	\bibitem{Lifshitz1946} \href{https://doi.org/10.1007/s10714-016-2165-8}{E. M. Lifshitz, J. Phys. (USSR) {\bf 10}, 116 (1946) (republished as Gen. Rel. Gravit. \textbf{49}, 18  (2017)).}
	
	\bibitem{Bardeen1980} \href{https://doi.org/10.1103/PhysRevD.22.1882}{J. M. Bardeen, Phys. Rev. D \textbf{22}, 1882 (1980).}
	
	\bibitem{Kodama1984} \href{https://doi.org/10.1143/PTPS.78.1}{H. Kodama and M. Sasaki, Prog. Theo. Phys. Suppl. {\bf 78}, 1 (1984).}
	
	\bibitem{Ellis2012} \href{https://doi.org/10.1017/CBO9781139014403}{G. F. R. Ellis, R. Maartens and  M. A. H. MacCallum, {\it Relativistic Cosmology} (Cambridge University Press,  Cambridge U. K. 2012).}
	
	\bibitem{Weinberg1972} S. Weinberg, {\it Gravitation and Cosmology:
		Principles  and Applications of the General Theory of Relativity} (Wiley, New York, 1972).
	
	\bibitem{Bach1921} \href{https://doi.org/10.1007/BF01378338}{R. Bach, Math. Z.  \textbf{9}, 110 (1921).}
	
	
	\bibitem{Mannheim2012a}  \href{https://doi.org/10.1103/PhysRevD.85.124008}{P. D. Mannheim, Phys. Rev. D {\bf 85}, 124008 (2012).}
	
	\bibitem{Mannheim2005}  \href{https://doi.org/10.1142/5975}{P. D. Mannheim, \textit{Brane-Localized Gravity} (World Scientific Publishing Company, Singapore, 2005).}
	
\end{thebibliography} 

\end{document}