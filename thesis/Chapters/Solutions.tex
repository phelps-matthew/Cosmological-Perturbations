
\chapter{Construction and Solution of SVT Fluctuation Equations}
\label{c:construction_and_solution_of_svt}

%%%%%%%%%%%%%%%%%%%%%%%%%%%%%%%%%%%%%%%%%%%%
\section{SVT3}
\label{s:svt3_construction}
%%%%%%%%%%%%%%%%%%%%%%%%%%%%%%%%%%%%%%%%%%%%

%%%%%%%%%%%%%%%%%%%%%%%%%%%%%%%%%%%%%%%%%%%%
\subsection{$dS_4$}
\label{ss:ds4_svt3}
%%%%%%%%%%%%%%%%%%%%%%%%%%%%%%%%%%%%%%%%%%%%
In the SVT3 formulation within a de Sitter background, the background and fluctuation metric can be expressed in the conformal to flat form
%
\begin{eqnarray}
ds^2 &=&\frac{1}{H^2\tau^2}\bigg{[}(1+2\phi) d\tau^2 -2(\tilde{\nabla}_i B +B_i)d\tau dx^i - [(1-2\psi)\delta_{ij} +2\tilde{\nabla}_i\tilde{\nabla}_j E 
\nonumber\\
&&+ \tilde{\nabla}_i E_j + \tilde{\nabla}_j E_i + 2E_{ij}]dx^i dx^j\bigg{]},
\label{7.1}
\end{eqnarray}
%
where the $\tilde{\nabla}_{i}$ denote derivatives with respect to the flat 3-space $\delta_{ij}dx^idx^j$ metric.
In terms of the SVT3 form for the fluctuations the components of the perturbed $\delta G_{\mu\nu}$ are given by \cite{amarasinghe_2019}
%
\begin{eqnarray}
\delta G_{00}&=&-\frac{6}{\tau}\dot{\psi}-\frac{2}{\tau}\tilde{\nabla}^2(\tau \psi +B-\dot{E}),
\nonumber\\
\delta G_{0i}&=&\frac{1}{2}\tilde{\nabla}^2(B_i-\dot{E}_i)+\frac{1}{\tau^2}\tilde{\nabla}_i(3B-2\tau^2\dot{\psi}+2\tau \phi)+\frac{3}{\tau^2}B_i,
\nonumber\\
\delta G_{ij}&=&\frac{\delta_{ij}}{\tau^2}\bigg[-2\tau^2\ddot{\psi}+2\tau\dot{\phi}+4\tau\dot{\psi}-6\phi-6\psi
\nonumber\\
&&+\tilde{\nabla}^2\left(2\tau B-\tau^2\dot{B}+\tau^2\ddot{E}-2\tau\dot{E}-\tau^2\phi+\tau^2\psi\right)\bigg]
\nonumber\\
&&+\frac{1}{\tau^2}\tilde{\nabla}_i\tilde{\nabla}_j\left[-2\tau B +\tau^2\dot{B}-\tau^2\ddot{E}+2\tau\dot{E}+6E+\tau^2\phi-\tau^2\psi\right]
\nonumber\\
&&+\frac{1}{2\tau^2}\tilde{\nabla}_i\left[-2\tau B_j+2\tau\dot{E}_j+\tau^2\dot{B}_j-\tau^2\ddot{E}_j+6E_j\right]
\nonumber\\
&&+\frac{1}{2\tau^2}\tilde{\nabla}_j\left[-2\tau B_i+2\tau\dot{E}_i+\tau^2\dot{B}_i-\tau^2\ddot{E}_i+6E_i\right]
\nonumber\\
&&-\ddot{E}_{ij}+\frac{6}{\tau^2}E_{ij}+\frac{2}{\tau}\dot{E}_{ij}+\tilde{\nabla}^2E_{ij},
\label{7.2}
\end{eqnarray}
%
where the dot denotes the derivative with respect to the conformal time $\tau$ and $\tilde{\nabla}^2=\delta^{ij}\tilde{\nabla}_i\tilde{\nabla}_j$. 
For the de Sitter SVT3 metric the gauge-invariant metric combinations are (see e.g. \cite{amarasinghe_2019})
%
\begin{eqnarray}
\alpha=\phi+\psi+\dot{B}-\ddot{E} ,\quad \beta=\tau\psi+B-\dot{E}, \quad B_i-\dot{E}_i,\quad E_{ij}.
\label{7.3}
\end{eqnarray}
%
(For a generic SVT3 metric with a general conformal factor $\Omega(\tau)$ the quantity $-(\Omega/\dot{\Omega})\psi+B-\dot{E}$ is gauge invariant, to thus become $\beta$ when $\Omega(\tau)=1/H\tau$, with the other gauge invariants being independent of $\Omega(\tau)$.)
In terms of the gauge-invariant combinations the fluctuation equations $\Delta_{\mu\nu}=\delta G_{\mu\nu}+\delta T_{\mu\nu}=0$ take the form
%
\begin{eqnarray}
\Delta_{00}&=&-\frac{6}{\tau^2}(\dot{\beta}-\alpha)-\frac{2}{\tau}\tilde{\nabla}^2\beta=0,
\label{7.4}
\end{eqnarray}
%
\begin{eqnarray}
\Delta_{0i}&=&\frac{1}{2}\tilde{\nabla}^2(B_i-\dot{E}_i)-\frac{2}{\tau}\tilde{\nabla}_i(\dot{\beta}-\alpha)=0,
\label{7.5}
\end{eqnarray}
%
\begin{eqnarray}
\Delta_{ij}&=&\frac{\delta_{ij}}{\tau^2}\left[-2\tau(\ddot{\beta}-\dot{\alpha})+6(\dot{\beta}-\alpha)+\tau \tilde{\nabla}^2(2\beta-\tau \alpha)\right]
+\frac{1}{\tau}\tilde{\nabla}_i\tilde{\nabla}_j(-2 \beta +\tau\alpha)
\nonumber\\
&+&\frac{1}{2\tau}\tilde{\nabla}_i[-2(B_j-\dot{E}_j)+\tau(\dot{B}_j-\ddot{E}_j)]
+\frac{1}{2\tau}\tilde{\nabla}_j[-2(B_i-\dot{E}_i)+\tau(\dot{B}_i-\ddot{E}_i)]
\nonumber\\
&-&\ddot{E}_{ij}+\frac{2}{\tau}\dot{E}_{ij}+\tilde{\nabla}^2E_{ij}=0,
\label{7.6}
\end{eqnarray}
%
\begin{eqnarray}
g^{\mu\nu}\Delta_{\mu\nu}&=&H^2[-6\tau(\ddot{\beta}-\dot{\alpha})+24(\dot{\beta}-\alpha)
+6\tau \tilde{\nabla}^2\beta-2\tau^2\tilde{\nabla}^2\alpha]=0,
\label{7.7}
\end{eqnarray}
%
to thus be manifestly gauge invariant.

If there is to be a decomposition theorem the S, V and T components of $\Delta_{\mu\nu}$ will satisfy $\Delta_{\mu\nu}=0$ independently, to thus be required to obey
%
\begin{eqnarray}
&&-\frac{6}{\tau^2}(\dot{\beta}-\alpha)-\frac{2}{\tau}\tilde{\nabla}^2\beta=0,\quad \frac{1}{2}\tilde{\nabla}^2(B_i-\dot{E}_i)=0, \quad \frac{2}{\tau}\tilde{\nabla}_i(\dot{\beta}-\alpha)=0,
\nonumber\\
&&\frac{\delta_{ij}}{\tau^2}\left[-2\tau(\ddot{\beta}-\dot{\alpha})+6(\dot{\beta}-\alpha)+\tau \tilde{\nabla}^2(2\beta-\tau\alpha)\right]+ \frac{1}{\tau^2}\tilde{\nabla}_i\tilde{\nabla}_j(-2\tau \beta +\tau^2\alpha)=0,
\nonumber\\
&&\frac{1}{2\tau^2}\tilde{\nabla}_i[-2\tau (B_j-\dot{E}_j)+\tau^2(\dot{B}_j-\ddot{E}_j)]
\nonumber\\
&&\qquad\qquad\qquad\qquad+\frac{1}{2\tau^2}\tilde{\nabla}_j[-2\tau (B_i-\dot{E}_i)+\tau^2(\dot{B}_i-\ddot{E}_i)]=0,
\nonumber\\
&&-\ddot{E}_{ij}+\frac{2}{\tau}\dot{E}_{ij}+\tilde{\nabla}^2E_{ij}=0.
\label{7.8}
\end{eqnarray}
%

To determine whether or not these conditions might hold we need to solve the fluctuation equations $\Delta_{\mu\nu}=0$ directly, to see what the structure of the solutions might look like.  To this end we first apply $\tau\partial_{\tau}-1$ to $-\tau^2\Delta_{00}/2$,  to obtain
%
\begin{eqnarray}
\tau^2\tilde{\nabla}^2\dot{\beta}+3\tau(\ddot{\beta}-\dot{\alpha})-3(\dot{\beta}-\alpha)=0,
\label{7.9}
\end{eqnarray}
%
and then add $3\tau^2\Delta_{00}$ to  $g^{\mu\nu}\Delta_{\mu\nu}/H^2$ to obtain
%
\begin{eqnarray}
\tau^2\tilde{\nabla}^2\alpha+3\tau(\ddot{\beta}-\dot{\alpha})-3(\dot{\beta}-\alpha)=0.
\label{7.10}
\end{eqnarray}
%
Combining these equations and using $\Delta_{00}=0$ we thus obtain
%
\begin{eqnarray}
\tilde{\nabla}^2(\alpha-\dot{\beta})=0,\quad \tilde{\nabla}^2\beta=0,
\label{7.11}
\end{eqnarray}
%
and 
%
\begin{eqnarray}
\tau^2\tilde{\nabla}^2(\alpha+\dot{\beta})+6\tau(\ddot{\beta}-\dot{\alpha})-6(\dot{\beta}-\alpha)=0.
\label{7.12}
\end{eqnarray}
%
Applying $\tilde{\nabla}^2$ then gives
%
\begin{eqnarray}
\tilde{\nabla}^4(\alpha+\dot{\beta})=0,\quad \tilde{\nabla}^4(\alpha-\dot{\beta})=0.
\label{7.13}
\end{eqnarray}
%
Applying $\tilde{\nabla}^2$ to $\Delta_{0i}$ in turn then gives
%
\begin{eqnarray}
\tilde{\nabla}^4(B_i-\dot{E}_i)=0,
\label{7.14}
\end{eqnarray}
%
while applying $\epsilon^{ijk}\tilde{\nabla}_j$ to $\Delta_{0k}$ gives
%
\begin{eqnarray}
\frac{1}{2}\epsilon^{ijk}\tilde{\nabla}_j\tilde{\nabla}^2(B_k-\dot{E}_k)=0.
\label{7.15}
\end{eqnarray}
%
Finally, to obtain an equation that only involves $E_{ij}$ we apply $\tilde{\nabla}^4$ to $\Delta_{ij}$, to obtain
%
\begin{eqnarray}
\tilde{\nabla}^4\left(-\ddot{E}_{ij}+\frac{2}{\tau}\dot{E}_{ij}+\tilde{\nabla}^2E_{ij}\right)=0.
\label{7.16}
\end{eqnarray}
%
As we see, we can isolate all the individual S, V and T gauge-invariant combinations, to thus give decomposition for the individual SVT3 components. However, the relations we obtain look nothing like the relations that a decomposition theorem would require, and thus without some further input we do not obtain a decomposition theorem.

To provide some further input we impose some asymptotic boundary conditions. To this end  we recall from Sec. \ref{s:decomposition_theorem} that for any spatially asymptotically bounded function $A$ that obeys $\tilde{\nabla}^2A=0$, the only solution is $A=0$. If $A$ obeys $\tilde{\nabla}^4A=0$, we must first set $\tilde{\nabla}^2A=C$, so that $\tilde{\nabla}^2C=0$. Imposing boundary conditions for $C$ enables us to set $C$=0. In such a case we can then set $\tilde{\nabla}^2A=0$, and with sufficient asymptotic convergence can then set $A=0$. Now a function could obey $\tilde{\nabla}^2A=0$ trivially by being independent of the spatial coordinates altogether, and only depend on $\tau$. However, it then would not vanish at spatial infinity, and we can thus exclude this possibility. With such spatial convergence for all of the S, V and T components we can then set 
%
\begin{eqnarray}
\alpha=0,\quad \dot{\beta}=0,\quad \beta =0,\quad B_i-\dot{E}_i=0,\quad -\tau\ddot{E}_{ij}+2\dot{E}_{ij}+\tau\tilde{\nabla}^2E_{ij}=0.
\label{7.17}
\end{eqnarray}
%
Since this solution coincides with the solution that would be obtained to (\ref{7.8}) under the same boundary conditions, we see that under these asymptotic boundary conditions we have a decomposition theorem.


In this solution all components of the SVT3 decomposition vanish identically except the rank two tensor $E_{ij}$. Taking $E_{ij}$ to behave as $\epsilon_{ij}\tau^2f(\tau)g({\bf x})$ where $\epsilon_{ij}$ is a polarization tensor, we find that the solution obeys
%
\begin{eqnarray}
\frac{\tau^2 \ddot{f}+2\tau \dot{f}-2f}{\tau^2f}=\frac{\tilde{\nabla}^2g}{g}=-k^2,
\label{7.18}
\end{eqnarray}
%
where $k^2$ is a separation constant. Consequently $E_{ij}$ is given as
%
\begin{eqnarray}
E_{ij}=\epsilon_{ij}({\bf k})\tau^2[a_1({\bf k})j_1(k\tau)+b_1({\bf k})y_1(k\tau)]e^{i{\bf k}\cdot{\bf x}},
\label{7.19}
\end{eqnarray}
%
where ${\bf k}\cdot{\bf k}=k^2$, $j_1$ and $y_1$ are spherical Bessel functions, and $a_1({\bf k})$ and $b_1({\bf k})$ are spacetime independent constants. For $E_{ij}$ to obey the transverse and traceless conditions $\delta^{ij}E_{ij}=0$, $\tilde{\nabla}^jE_{ij}=0$ the polarization tensor $\epsilon_{ij}({\bf k})$ must obey $\delta^{ij}\epsilon_{ij}=0$, ${\bf k}^{j}\epsilon_{ij}({\bf k})=0$.
Then, by taking a family of separation constants we can form a transverse-traceless wave packet
%
\begin{eqnarray}
E_{ij}&=&\sum_{\bf k}\epsilon_{ij}({\bf k})\tau^2[a_1({\bf k})j_1(k\tau)+b_1({\bf k})y_1(k\tau)]e^{i{\bf k}\cdot{\bf x}}\nonumber\\
&=&
\sum_{\bf k}\epsilon_{ij}({\bf k})\bigg[a_1({\bf k})\left(\frac{\sin(k\tau)}{k^2}-\frac{\tau\cos(k\tau)}{k}\right)
\nonumber\\
&&+b_1({\bf k})\left(\frac{\cos(k\tau)}{k^2}+\frac{\tau\sin(k\tau)}{k}\right)\bigg],
\label{7.20}
\end{eqnarray}
%
and can choose the $a_1({\bf k})$ and $b_1({\bf k})$ coefficients to make the packet be as well-behaved at spatial infinity as desired. Finally, since according to (\ref{7.1}) the full fluctuation is given not by $E_{ij}$ but by $2E_{ij}/H^2\tau^2$, then with $\tau=e^{-Ht}/H$, through the $\cos(k\tau)/k^2$ term we find that at large comoving time  $E_{ij}/\tau^2$ behaves as $e^{2Ht}$, viz. the standard de Sitter fluctuation exponential growth.

%%%%%%%%%%%%%%%%%%%%%%%%%%%%%%%%%%%%%%%%%%%%
\subsection{Robertson Walker $k=0$ Radiation Era}
\label{ss:rw_k=0_radiation_svt3}
%%%%%%%%%%%%%%%%%%%%%%%%%%%%%%%%%%%%%%%%%%%%
\label{S8}
In comoving coordinates a spatially flat Robertson-Walker background metric takes the form $ds^2=dt^2-a^2(t)\delta_{ij}dx^idx^j$. In the radiation era where a perfect fluid pressure $p$ and energy density $\rho$ are related by $\rho=3p$, the background energy-momentum tensor is given by the traceless
%
\begin{eqnarray}
T_{\mu\nu}=p(4U_{\mu}U_{\nu}+g_{\mu\nu}),
\label{8.1}
\end{eqnarray}
%
where $g^{\mu\nu}U_{\mu}U_{\nu}=-1$, $U^{0}=1$, $U_0=-1$, $U^{i}=0$, $U_i=0$. With this source the background Einstein equations $G_{\mu\nu}=-T_{\mu\nu}$ with $8\pi G=1$ fix $a(t)$ to be $a(t)=t^{1/2}$. In conformal to flat coordinates we set $\tau=\int dt/t^{1/2}=2t^{1/2}$, with the conformal factor being given by $\Omega(\tau)=\tau/2$. In conformal to flat coordinates the background pressure is of the form $p=4/\tau^4$ while $U^{0}=2/\tau$, $U_0=-\tau/2$. In this coordinate system the SVT3 fluctuation metric as written with an explicit conformal factor is of the form 
%
\begin{eqnarray}
ds^2 &=&\frac{\tau^2}{4}\bigg{[}(1+2\phi) d\tau^2 -2(\tilde{\nabla}_i B +B_i)d\tau dx^i - [(1-2\psi)\delta_{ij} +2\tilde{\nabla}_i\tilde{\nabla}_j E
\nonumber\\
&& + \tilde{\nabla}_i E_j + \tilde{\nabla}_j E_i + 2E_{ij}]dx^i dx^j\bigg{]},
\label{8.2}
\end{eqnarray}
%
and the fluctuation energy-momentum tensor is of the form
%
\begin{eqnarray}
\delta T_{\mu\nu}=\delta p(4U_{\mu}U_{\nu}+g_{\mu\nu})+p(4\delta U_{\mu}U_{\nu}+4U_{\mu}\delta U_{\nu}+h_{\mu\nu}).
\label{8.3}
\end{eqnarray}
%
As written, we might initially expect there to be five fluctuation variables in the fluctuation energy-momentum tensor: $p$ and the four components of $\delta U_{\mu}$. However, varying  $g^{\mu\nu}U_{\mu}U_{\nu}=-1$ gives 
%
\begin{eqnarray}
\delta g^{00}U_{0}U_{0}+2g^{00}U_{0}\delta U_{0}=0,
\label{8.4}
\end{eqnarray}
%
i.e. 
%
\begin{eqnarray}
\delta U_{0}=-\frac{1}{2}(g^{00})^{-1}(-g^{00}g^{00}\delta g_{00})U_{0}=-\frac{\tau\phi}{2}.
\label{8.5}
\end{eqnarray}
%
Thus $\delta U_{0}$ is not an independent of the metric fluctuations, and we need the fluctuation equations to only fix six (viz. ten minus four) independent gauge-invariant metric fluctuations and the  four $\delta p$ and $\delta U_i$ (counting all components). With ten $\Delta_{\mu\nu}=0$ equations, we can nicely determine all of them. 


To this end we evaluate $\delta G_{\mu\nu}$, to obtain 
%
\begin{eqnarray}
\delta G_{00}&=&\frac{6}{\tau}\dot{\psi}+\frac{2}{\tau}\tilde{\nabla}^2(-\tau \psi +B-\dot{E}),
\nonumber\\
\delta G_{0i}&=&\frac{1}{2}\tilde{\nabla}^2(B_i-\dot{E}_i)+\frac{1}{\tau^2}\tilde{\nabla}_i(-B-2\tau^2\dot{\psi}-2\tau \phi)-\frac{1}{\tau^2}B_i,
\nonumber\\
\delta G_{ij}&=&\frac{\delta_{ij}}{\tau^2}\bigg[-2\tau^2\ddot{\psi}-2\tau\dot{\phi}-4\tau\dot{\psi}+2\phi+2\psi
\nonumber\\
&&+\tilde{\nabla}^2\left(-2\tau B-\tau^2\dot{B}+\tau^2\ddot{E}+2\tau\dot{E}-\tau^2\phi+\tau^2\psi\right)\bigg]
\nonumber\\
&&+\frac{1}{\tau^2}\tilde{\nabla}_i\tilde{\nabla}_j\left[2\tau B +\tau^2\dot{B}-\tau^2\ddot{E}-2\tau\dot{E}-2E+\tau^2\phi-\tau^2\psi\right]
\nonumber\\
&&+\frac{1}{2\tau^2}\tilde{\nabla}_i\left[2\tau B_j-2\tau\dot{E}_j+\tau^2\dot{B}_j-\tau^2\ddot{E}_j-2E_j\right]
\nonumber\\
&&+\frac{1}{2\tau^2}\tilde{\nabla}_j\left[2\tau B_i-2\tau\dot{E}_i+\tau^2\dot{B}_i-\tau^2\ddot{E}_i-2E_i\right]
\nonumber\\
&&-\ddot{E}_{ij}-\frac{2}{\tau^2}E_{ij}-\frac{2}{\tau}\dot{E}_{ij}+\tilde{\nabla}^2E_{ij},
\label{8.6}
\end{eqnarray}
%
where the dot denotes $\partial_\tau$.
For a spatially flat Robertson-Walker  metric the gauge-invariant metric combinations are (see e.g. \cite{amarasinghe_2019})
%
\begin{eqnarray}
\alpha=\phi+\psi+\dot{B}-\ddot{E} ,\quad \gamma=-\tau\psi+B-\dot{E}, \quad B_i-\dot{E}_i,\quad E_{ij}.
\label{8.7}
\end{eqnarray}
%
(For a generic (\ref{8.2}) with conformal factor $\Omega(\tau)$, the quantity $-(\Omega/\dot{\Omega})\psi+B-\dot{E}$ becomes $\gamma$ when $\Omega(\tau)=\tau/2$, with the other gauge invariants being independent of $\Omega(\tau)$.)
In terms of the gauge-invariant combinations the fluctuation equations $\Delta_{\mu\nu}=\delta G_{\mu\nu}+\delta T_{\mu\nu}=0$ take the form
%
\begin{eqnarray}
\Delta_{00}=-\frac{16}{\tau^3}\delta U_{0}-\frac{8}{\tau^2}\phi+\frac{3\tau^2}{4}\left(\delta p -\frac{16}{\tau^4}\psi\right)
+\frac{6}{\tau^2}(\alpha-\dot{\gamma})+\frac{2}{\tau}\tilde{\nabla}^2\gamma=0,
\label{8.8}
\end{eqnarray}
%
\begin{eqnarray}
\Delta_{0i}&=&-\frac{8}{\tau^3}\delta U_{i}+\frac{4}{\tau}\tilde{\nabla}_i\psi
+\frac{1}{2}\tilde{\nabla}^2(B_i-\dot{E}_i)-\frac{2}{\tau}\tilde{\nabla}_i(\alpha-\dot{\gamma})=0,
\label{8.9}
\end{eqnarray}
%
\begin{eqnarray}
\Delta_{ij}&=&\frac{\delta_{ij}}{4\tau^2}\left[\tau^4\delta p-16\psi-8\tau(\dot{\alpha}-\ddot{\gamma})+8(\alpha-\dot{\gamma})-4\tau \tilde{\nabla}^2(\tau\alpha+2\gamma)\right]
\nonumber\\
&&+\frac{1}{\tau}\tilde{\nabla}_i\tilde{\nabla}_j(\tau\alpha+2\gamma) +\frac{1}{2\tau}\tilde{\nabla}_i[2(B_j-\dot{E}_j)+\tau(\dot{B}_j-\ddot{E}_j)]
\nonumber\\
&&+\frac{1}{2\tau}\tilde{\nabla}_j[2(B_i-\dot{E}_i)+\tau(\dot{B}_i-\ddot{E}_i)]
-\ddot{E}_{ij}-\frac{2}{\tau}\dot{E}_{ij}+\tilde{\nabla}^2E_{ij}=0,
\label{8.10}
\end{eqnarray}
%
\begin{eqnarray}
g^{\mu\nu}\Delta_{\mu\nu}&=&\frac{64}{\tau^5}\delta U_0+\frac{32}{\tau^4}\phi -\frac{24}{\tau^3}(\dot{\alpha}-\ddot{\gamma})-\frac{8}{\tau^2}\tilde{\nabla}^2\alpha-\frac{24}{\tau^3}\tilde{\nabla}^2\gamma=0.
\label{8.11}
\end{eqnarray}
%
Since $\Delta_{\mu\nu}$ is gauge invariant, we see that it is not $\delta U_0$, $\delta U_i$ and $\delta p$ themselves that are gauge invariant. Rather it is the combinations $\delta U_0+\tau\phi/2$, $\delta p-16\psi/\tau^4$, and $\delta U_i-\tau^2\tilde{\nabla}_i\psi/2$ that are gauge invariant instead. Since we have shown above that $\delta U_0+\tau\phi/2$ is actually zero, confirming that it is equal to a gauge-invariant quantity provides a nice check on our calculation. However, since $\delta U_0+\tau\phi/2$ is zero, we can replace the $\Delta_{00}$ and $g^{\mu\nu}\Delta_{\mu\nu}$ equations by 
%
\begin{eqnarray}
\Delta_{00}&=&\frac{3\tau^2}{4}\left(\delta p -\frac{16}{\tau^4}\psi\right)
+\frac{6}{\tau^2}(\alpha-\dot{\gamma})+\frac{2}{\tau}\tilde{\nabla}^2\gamma=0,
\label{8.12}
\end{eqnarray}
%
%
\begin{eqnarray}
g^{\mu\nu}\Delta_{\mu\nu}&=& -\frac{24}{\tau^3}(\dot{\alpha}-\ddot{\gamma})-\frac{8}{\tau^2}\tilde{\nabla}^2\alpha-\frac{24}{\tau^3}\tilde{\nabla}^2\gamma=0.
\label{8.13}
\end{eqnarray}
%

To put $\delta U_i$ in a more convenient form we decompose it into transverse and longitudinal components  as $\delta U_i=V_i+\tilde{\nabla}_iV$ where 
%
\begin{eqnarray}
\tilde{\nabla}^iV_i=0,\quad \tilde{\nabla}^2V=\tilde{\nabla}^i\delta U_i,\quad V({\bf x},\tau)=\int d^3{\bf y}D^{(3)}({\bf x}-{\bf y})\tilde{\nabla}^i_y\delta U_i({\bf y}, \tau).
\label{8.14}
\end{eqnarray}
%
In terms of these components the $\Delta_{0i}=0$ equation takes the form
%
\begin{eqnarray}
\Delta_{0i}=-\frac{8}{\tau^3}V_{i}+\frac{1}{2}\tilde{\nabla}^2(B_i-\dot{E}_i)-\frac{8}{\tau^3}\tilde{\nabla}_iV+\frac{4}{\tau}\tilde{\nabla}_i\psi
-\frac{2}{\tau}\tilde{\nabla}_i(\alpha-\dot{\gamma})=0.
\label{8.15}
\end{eqnarray}
%
Applying $\tilde{\nabla}^i$ to $\Delta_{0i}$ and then $\tilde{\nabla}^2$ in turn yields
%
\begin{eqnarray}
\tilde{\nabla}^i\Delta_{0i}&=&\tilde{\nabla}^2\left[-\frac{8}{\tau^3}V+\frac{4}{\tau}\psi
-\frac{2}{\tau}(\alpha-\dot{\gamma})\right]=0,
\nonumber\\
&& \tilde{\nabla}^2\left[-\frac{8}{\tau^3}V_{i}+\frac{1}{2}\tilde{\nabla}^2(B_i-\dot{E}_i)\right]=0,\label{8.16}
\end{eqnarray}
%
while applying $\epsilon^{ijk}\tilde{\nabla}_j$ to $\Delta_{0k}$ yields
%
\begin{eqnarray}
\epsilon^{ijk}\tilde{\nabla}_j\Delta_{0k}&=&\epsilon^{ijk}\tilde{\nabla}_j\left[-\frac{8}{\tau^3}V_{k}+\frac{1}{2}\tilde{\nabla}^2(B_k-\dot{E}_k)\right]=0.
\label{8.17}
\end{eqnarray}
%
We thus identify $V-\tau^2\psi/2$ and $V_i$ as being gauge invariant.

To determine more of the structure of the solutions to the fluctuation equations we apply $\tilde{\nabla}^i\tilde{\nabla}^j$ to the $\Delta_{ij}=0$ equation and then use the $\Delta_{00}=0$ equation to eliminate $\delta p -16\psi/\tau^4$ from $\Delta_{ij}$, to obtain
%
\begin{eqnarray}
\tilde{\nabla}^2\bigg{[}\frac{\tau^2}{4}\delta p-\frac{4}{\tau^2}\psi-\frac{2}{\tau}(\dot{\alpha}-\ddot{\gamma})+\frac{2}{\tau^2}(\alpha-\dot{\gamma})\bigg{]}
\nonumber\\
=-\frac{2}{3\tau}\left[
\tilde{\nabla}^4\gamma+3\tilde{\nabla}^2(\dot{\alpha}-\ddot{\gamma})\right]=0.
\label{8.18}
\end{eqnarray}
% 
With the application of $\tilde{\nabla}^2$ to the $g^{\mu\nu}\Delta_{\mu\nu}=0$ equation yielding
%
\begin{eqnarray}
3\tilde{\nabla}^2(\dot{\alpha}-\ddot{\gamma})+\tau \tilde{\nabla}^4\alpha+3\tilde{\nabla}^4\gamma=0,
\label{8.19}
\end{eqnarray}
% 
we obtain
%
\begin{eqnarray}
\tilde{\nabla}^4(\tau \alpha+2\gamma)=0.
\label{8.20}
\end{eqnarray}
% 
On applying $\partial_{\tau}$ to (\ref{8.20}) and using (\ref{8.20}) and (\ref{8.18}) we obtain
%
\begin{eqnarray}
3\tau\tilde{\nabla}^4\dot{\alpha}=-3\tilde{\nabla}^4\alpha-6\tilde{\nabla}^4\dot{\gamma}
=\frac{6}{\tau}\tilde{\nabla}^4\gamma-6\tilde{\nabla}^4\dot{\gamma}
=3\tau\tilde{\nabla}^4\ddot{\gamma}-\tau\tilde{\nabla}^6\gamma.
\label{8.21}
\end{eqnarray}
%
The parameter $\gamma$ thus obeys
%
\begin{eqnarray}
\tilde{\nabla}^4\left(\tilde{\nabla}^2\gamma-3\ddot{\gamma}-\frac{6}{\tau}\dot{\gamma}+\frac{6}{\tau^2}\gamma\right)=0.
\label{8.22}
\end{eqnarray}
%
We can treat (\ref{8.22})  as a second-order differential equation for $\tilde{\nabla}^4\gamma$.
On setting $\tilde{\nabla}^4\gamma=f_k(\tau)g_k({\bf x})$ for a single mode, on separating  with a separation constant $-k^2$ according to
%
\begin{eqnarray}
\frac{\tilde{\nabla}^2g_k}{g_k}=\frac{3\ddot{f}_k+6\dot{f}_k/\tau-6f_k/\tau^2}{f_k}=-k^2,
\label{8.23}
\end{eqnarray}
%
we find that the $\tau$ dependence of $f_k(\tau)$ is given by $j_1(k\tau/\surd{3})$ and $y_1(k\tau/\surd{3})$, while the spatial dependence is given by plane waves. Since the set of plane waves is complete, the general solution to (\ref{8.22}) can be written as
%
\begin{eqnarray}
\gamma&=&\sum_{\bf k}[a_1({\bf k})j_1(k\tau/\surd{3})+b_1({\bf k})y_1(k\tau/\surd{3})]e^{i{\bf k}\cdot{\bf x}} +{\rm delta~function~terms},
\label{8.24}
\end{eqnarray}
%
where the delta function terms are solutions to $\tilde{\nabla}^4\gamma=0$, solutions that, in analog to  (\ref{1.8}),  are generically of the form $\delta(k)$, $\delta(k)/k$, $\delta(k)/k^2$, $\delta(k)/k^3$.


Proceeding the same way for $\alpha$  we obtain 
%
\begin{eqnarray}
3\tilde{\nabla}^4\dot{\alpha}=3\tilde{\nabla}^4\ddot{\gamma}-\tilde{\nabla}^6\gamma
=-\frac{3}{2}\tilde{\nabla}^4(\tau\ddot{\alpha}+2\dot{\alpha})+\frac{\tau}{2}\tilde{\nabla}^6\alpha.
\label{8.25}
\end{eqnarray}
%
The parameter $\alpha$ thus obeys
%
\begin{eqnarray}
\tilde{\nabla}^4\left(\tilde{\nabla}^2\alpha-3\ddot{\alpha}-\frac{12}{\tau}\dot{\alpha}\right)=0.
\label{8.26}
\end{eqnarray}
%
On setting $\tilde{\nabla}^4\alpha=d_k(\tau)e_k({\bf x})$ for a single mode, we can separate with a separation constant $-k^2$ according to
%
\begin{eqnarray}
\frac{\tilde{\nabla}^2e_k}{e}=\frac{3\ddot{d}_k+12\dot{d}_k/\tau}{d_k}=-k^2.
\label{8.27}
\end{eqnarray}
%
The $\tau$ dependence of $d_k(\tau)$ is thus given by $j_1(k\tau/\surd{3})/\tau$ and  $y_1(k\tau/\surd{3})/\tau$, while the spatial dependence is given by plane waves. The general solution to (\ref{8.26}) is thus given by 
%
\begin{eqnarray}
\alpha&=&\frac{1}{\tau}\sum_{\bf k}[m_1({\bf k})j_1(k\tau/\surd{3})+n_1({\bf k})y_1(k\tau/\surd{3})]e^{i{\bf k}\cdot{\bf x}}
\nonumber\\
&& +{\rm delta~function~terms},
\label{8.28}
\end{eqnarray}
%
where the delta function terms are solutions to $\tilde{\nabla}^4\alpha=0$. Finally, we recall that $\alpha$ and $\gamma$ are related through $\tilde{\nabla}^4(\tau\alpha+2\gamma)=0$,  with the coefficients  thus obeying
%
\begin{eqnarray}
m_1({\bf k})+2a_1({\bf k})=0,\quad n_1({\bf k})+2b_1({\bf k})=0.
\label{8.29}
\end{eqnarray}
%


Having determined $\alpha$ and $\gamma$, we can now determine $\delta p-16\psi/\tau^4$ from the $\Delta_{00}=0$ equation, and obtain
%
\begin{eqnarray}
\nonumber\\
&&\delta p -\frac{16}{\tau^4}\psi=-\frac{8}{\tau^4}(\alpha-\dot{\gamma})-\frac{8}{3\tau^3}\tilde{\nabla}^2\gamma
\nonumber\\
&=&\sum_{\bf k}\left[\frac{8}{\tau^4}\left(\frac{2}{\tau}+\frac{\partial}{\partial \tau}\right)
+\frac{8k^2}{3\tau^3}\right]\left[a_1({\bf k})j_1(k\tau/\surd{3})+b_1({\bf k})y_1(k\tau/\surd{3})\right]e^{i{\bf k}\cdot{\bf x}}
\nonumber\\
&& +{\rm delta~function~terms}
\nonumber\\
&=&\sum_{\bf k}a_1({\bf k})\left[\frac{8k}{\tau^4\surd{3}}j_0(k\tau/\surd{3})+\frac{8k^2}{3\tau^3}j_1(k\tau/\surd{3})\right]
\nonumber\\
&&+\sum_{\bf k}b_1({\bf k})\left[\frac{8k}{\tau^4\surd{3}}y_0(k\tau/\surd{3})+\frac{8k^2}{3\tau^3}y_1(k\tau/\surd{3})\right]
\nonumber\\
&&+{\rm delta~function~terms}.
\label{8.30}
\end{eqnarray}
%
To determine $B_i-\dot{E}_i$ we apply $\tilde{\nabla}^j$ to $\Delta_{ij}=0$, to obtain
%
\begin{eqnarray}
\tilde{\nabla}^2\left[\frac{1}{\tau}(B_i-\dot{E}_i)+\frac{1}{2}(\dot{B}_i-\ddot{E}_i)\right]=\tilde{\nabla}_i\left[\frac{2}{\tau}(\dot{\alpha}-\ddot{\gamma})+\frac{2}{3\tau}\tilde{\nabla}^2\gamma\right],
\label{8.31}
\end{eqnarray}
%
from which it follows that
%
\begin{eqnarray}
\tilde{\nabla}^i\tilde{\nabla}^2\left[\frac{1}{\tau}(B_i-\dot{E}_i)+\frac{1}{2}(\dot{B}_i-\ddot{E}_i)\right]=\tilde{\nabla}^2\left[\frac{2}{\tau}(\dot{\alpha}-\ddot{\gamma})+\frac{2}{3\tau}\tilde{\nabla}^2\gamma\right]=0.
\label{8.32}
\end{eqnarray}
%
On now applying $\tilde{\nabla}^2$ to (\ref{8.31}) we  obtain
%
\begin{eqnarray}
\tilde{\nabla}^4\left[\frac{1}{\tau}(B_i-\dot{E}_i)+\frac{1}{2}(\dot{B}_i-\ddot{E}_i)\right]=0,
\label{8.33}
\end{eqnarray}
%
just as required for consistency with (\ref{8.18}).
Equation (\ref{8.33}) can be satisfied through a $1/\tau^2$ conformal time dependence, and while it could also be satisfied via a spatial dependence that satisfies $\tilde{\nabla}^4(B_i-\dot{E}_i)=0$, viz. the above delta function terms. With plane waves being complete we can thus set
%
\begin{eqnarray}
B_i-\dot{E}_i=\frac{1}{\tau^2}\sum_{\bf k}a_i({\bf k})e^{i{\bf k}\cdot{\bf x}}
+F(\tau)\times {\rm delta~function~terms},
\label{8.34}
\end{eqnarray}
%
where the $a_i({\bf k})$ are transverse vectors that obeys $k^ia_i({\bf k})=0$, and where $F(\tau)$ is an arbitrary function of $\tau$.

After solving (\ref{8.31}) and (\ref{8.32}), from the second equation in (\ref{8.16}) we can then determine $V_i$ as it obeys 
%
\begin{eqnarray}
\frac{8}{\tau^3}\tilde{\nabla}^2V_{i}=\frac{1}{2}\tilde{\nabla}^4(B_i-\dot{E}_i)=\frac{1}{2\tau^2}\sum_{\bf k}k^4a_i({\bf k})e^{i{\bf k}\cdot{\bf x}}.
\label{8.35}
\end{eqnarray}
%
From (\ref{8.17}) we can infer that 
%
\begin{eqnarray}
-\frac{8}{\tau^3}V_{i}+\frac{1}{2}\tilde{\nabla}^2(B_i-\dot{E}_i)=\tilde{\nabla}_iA,
\label{8.36}
\end{eqnarray}
%
where $A$ is a scalar function that obeys $\tilde{\nabla}^2A=0$, with $\tilde{\nabla}_iA$ being curl free. We recognize $A$ as an integration constant for the integration of (\ref{8.35}). From the first equation in (\ref{8.16}) we additionally obtain
%
\begin{eqnarray}
&&\tilde{\nabla}^2(-\frac{8}{\tau^3}V+\frac{4}{\tau}\psi)
=\frac{2}{\tau}\tilde{\nabla}^2(\alpha-\dot{\gamma})
\nonumber\\
&=&\frac{2}{\tau\surd{3}}\sum_{\bf k}k^3[a_1({\bf k})j_0(k\tau/\surd{3})+b_1({\bf k})y_0(k\tau/\surd{3})]e^{i{\bf k}\cdot{\bf x}}
\nonumber\\
&&+ {\rm delta~function~terms.}
\label{8.37}
\end{eqnarray}
%

To determine an equation for $E_{ij}$ we note that the $\delta_{ij}$ term in $\Delta _{ij}$ can be written as 
$(\delta_{ij}/4\tau^2)[-8\tau(\dot{\alpha}-\ddot{\gamma})-4\tau^2\tilde{\nabla}^2\alpha-(32/3)\tau\tilde{\nabla}^2\gamma]$. Through use of (\ref{8.18}), (\ref{8.19}) and (\ref{8.20}), we can show that $\tilde{\nabla}^2$ applied to this term gives zero. Then given (\ref{8.33}) and (\ref{8.20}), from (\ref{8.10}) it follows that $E_{ij}$ obeys 
%
\begin{eqnarray}
\tilde{\nabla}^4\left(-\ddot{E}_{ij}-\frac{2}{\tau}\dot{E}_{ij}+\tilde{\nabla}^2E_{ij}\right)=0.
\label{8.38}
\end{eqnarray}
%
Setting $\tilde{\nabla}^4E_{ij}=\epsilon_{ij}({\bf k})f_k(\tau)g_k({\bf x})$ for a momentum mode, the $\tau$ dependence  is given as $j_0(k\tau)$ and $y_0(k\tau)$, with the general solution being of the form
%
\begin{eqnarray}
E_{ij}&=&\sum_{\bf k}[a^0_{ij}({\bf k})j_0(k\tau)+b^0_{ij}({\bf k})y_0(k\tau)]e^{i{\bf k}\cdot{\bf x}}
+ {\rm delta~function~terms}.
\label{8.39}
\end{eqnarray}
%
Since according to (\ref{8.2}) the full tensor fluctuation is given not by $E_{ij}$ but by $\tau^2E_{ij}/2$, then with $\tau=2t^{1/2}$, we find that at large comoving time  $\tau^2 E_{ij}$ behaves as $t^{1/2}$. Thus to summarize,  we have constructed the exact and general solution to the SVT3 $k=0$ radiation era Robertson-Walker fluctuation equations for all of the dynamical degrees of freedom $\alpha$, $\beta$, $B_i-\dot{E}_i$, $E_{ij}$, $\delta p-16\psi/\tau^4$, $V-\tau^2\psi/2$ and $V_i$. 

For a decomposition theorem to hold the condition $\Delta_{\mu\nu}=0$ would need to decompose into 

%
\begin{eqnarray}
&&\frac{3\tau^2}{4}\left(\delta p -\frac{16}{\tau^4}\psi\right)
+\frac{6}{\tau^2}(\alpha-\dot{\gamma})+\frac{2}{\tau}\tilde{\nabla}^2\gamma=0,
\label{8.40}
\end{eqnarray}
%
\begin{eqnarray}
&&-\frac{8}{\tau^3}V_{i}+\frac{1}{2}\tilde{\nabla}^2(B_i-\dot{E}_i)=0,
\label{8.41}
\end{eqnarray}
%
%
\begin{eqnarray}
&&-\frac{8}{\tau^3}\tilde{\nabla}_iV+\frac{4}{\tau}\tilde{\nabla}_i\psi
-\frac{2}{\tau}\tilde{\nabla}_i(\alpha-\dot{\gamma})=0,
\label{8.42}
\end{eqnarray}
%
\begin{eqnarray}
&&\frac{\delta_{ij}}{4\tau^2}\left[\tau^4\delta p-16\psi-8\tau(\dot{\alpha}-\ddot{\gamma})+8(\alpha-\dot{\gamma})-4\tau \tilde{\nabla}^2(\tau\alpha+2\gamma)\right]
\nonumber\\
&&+\frac{1}{\tau}\tilde{\nabla}_i\tilde{\nabla}_j(\tau\alpha+2\gamma)=0,
\label{8.43}
\end{eqnarray}
%
\begin{eqnarray}
\frac{1}{2\tau}\tilde{\nabla}_i[2(B_j-\dot{E}_j)+\tau(\dot{B}_j-\ddot{E}_j)]
+\frac{1}{2\tau}\tilde{\nabla}_j[2(B_i-\dot{E}_i)+\tau(\dot{B}_i-\ddot{E}_i)]=0,
\label{8.44}
\end{eqnarray}
%
\begin{eqnarray}
&&-\ddot{E}_{ij}-\frac{2}{\tau}\dot{E}_{ij}+\tilde{\nabla}^2E_{ij}=0,
\label{8.45}
\end{eqnarray}
%
\begin{eqnarray}
&& -\frac{24}{\tau^3}(\dot{\alpha}-\ddot{\gamma})-\frac{8}{\tau^2}\tilde{\nabla}^2\alpha-\frac{24}{\tau^3}\tilde{\nabla}^2\gamma=0.
\label{8.46}
\end{eqnarray}
%

To determine whether these conditions might hold, we note that in the $\alpha$, $\gamma$ sector the (\ref{8.40}) and (\ref{8.46})  equations are the same as in the general $\Delta_{\mu\nu}=0$ case (viz. (\ref{8.12}) and (\ref{8.13})), but (\ref{8.43}) is different. If we use (\ref{8.12}) to substitute for $\delta p$ in (\ref{8.43}) we obtain 
%
\begin{eqnarray}
&&\frac{\delta_{ij}}{4\tau^2}\left[-8\tau(\dot{\alpha}-\ddot{\gamma})-
4\tau^2\tilde{\nabla}^2\alpha-\frac{32\tau}{3} \tilde{\nabla}^2\gamma\right]+\frac{1}{\tau}\tilde{\nabla}_i\tilde{\nabla}_j(\tau\alpha+2\gamma)=0.
\label{8.47}
\end{eqnarray}
%
Given the differing behaviors of $\delta_{ij}$ and $\tilde{\nabla}_i\tilde{\nabla}_j$ it follows that the terms that they multiply  in (\ref{8.47}) must each vanish, and thus we can set
%
\begin{eqnarray}
&&-8\tau(\dot{\alpha}-\ddot{\gamma})-
4\tau^2\tilde{\nabla}^2\alpha-\frac{32\tau}{3} \tilde{\nabla}^2\gamma=0,
\label{8.48}
\end{eqnarray}
%
\begin{eqnarray}
\tau\alpha+2\gamma=0.
\label{8.49}
\end{eqnarray}
%
Combining these equations then gives
%
\begin{eqnarray}
&&3(\dot{\alpha}-\ddot{\gamma})+ \tilde{\nabla}^2\gamma=0.
\label{8.50}
\end{eqnarray}
%
We recognize (\ref{8.18}) as the $\tilde{\nabla}^2$ derivative of (\ref{8.50}) and recognize (\ref{8.20}) as the $\tilde{\nabla}^4$ derivative of (\ref{8.49}).

Similarly, in the $V$,$V_i$ sector we recognize the two equations that appear in (\ref{8.16}) as the $\nabla^2$ derivative of (\ref{8.41}) and the $\tilde{\nabla}^i$ derivative of (\ref{8.42}), with (\ref{8.17}) being the curl of (\ref{8.41}). In the $B_i-\dot{E}_i$ sector we recognize (\ref{8.33}) as the $\tilde{\nabla}^j\tilde{\nabla}^2$ derivative of (\ref{8.44}), and in the $E_{ij}$ sector we recognize (\ref{8.38}) as the $\tilde{\nabla}^4$ derivative of  (\ref{8.45}). Consequently we see that if spatially asymptotic boundary conditions are such that the only solutions to $\Delta_{\mu\nu}=0$ are also solutions to (\ref{8.40}) to (\ref{8.46}) (i.e. vanishing of all delta function terms and integration constants that would lead to non-vanishing asymptotics), then the decomposition theorem follows. Otherwise it does not. Finally, we should note that, as constructed, in the matter sector we have found solutions for the gauge-invariant quantities $\delta p-16\psi/\tau^4$, and $V-\tau^2\psi/2$. However since $\psi$ is not gauge invariant, by choosing a gauge in which $\psi=0$, we would then have solutions for $\delta p$ and $V$ alone.
%%%%%%%%%%%%%%%%%%%%%%%%%%%%%%%%%%%%%%%%%%%%
\subsection{General Roberston Walker}
\label{ss:general_rw_svt3}
%%%%%%%%%%%%%%%%%%%%%%%%%%%%%%%%%%%%%%%%%%%%
\subsubsection{Constructing the Fluctuations}
\label{sss:setting_up_the_equations}


Having seen how things work in a particular background Robertson-Walker case (radiation era with $k=0$, Sec. \ref{ss:rw_k=0_radiation_svt3}), we now present a general analysis that can be applied to any background Robertson-Walker geometry with any background perfect fluid equation of state. To characterize a general Robertson-Walker background there are two straightforward options. One is to write the background metric in a conformal to flat form $ds^2=\Omega^2(\tau,x^i)(d\tau^2-\delta_{ij}dx^idx^j)$ with $\Omega(\tau,x^i)$ depending on both the conformal time $\tau=\int dt/a(t)$ and the spatial coordinates. The other is to write the background geometry as conformal to a static Robertson-Walker geometry: 
%
\begin{eqnarray}
ds^2&=&\Omega^2(\tau)[d\tau^2-\tilde{\gamma}_{ij}dx^idx^j]
\nonumber\\
&=&\Omega^2(\tau)\left[ d\tau^2-\frac{dr^2}{1-kr^2}-r^2d\theta^2-r^2\sin\theta^2d\phi^2\right],
\label{9.1}
\end{eqnarray}
%
with $\Omega(\tau)$ depending only on $\tau$, and with $\tilde{\gamma}_{ij}dx^idx^j$ denoting the spatial sector of the metric. These two formulations of the background metric are coordinate equivalent as one can transform one into the other by a general coordinate transformation without any need to make a conformal transformation on the background metric (see Appendix \ref{ab:cosmologies} and Sec. \ref{ss:deltaW_conformal_flat_SVT3}). For our purposes in this section we shall take (\ref{9.1}) to be the background metric, and shall take the fluctuation metric to be of the form
%
\begin{eqnarray}
ds^2&=&\Omega^2(\tau)\bigg[2\phi d\tau^2 -2(\tilde{\nabla}_i B +B_i)d\tau dx^i - [-2\psi\tilde{\gamma}_{ij} +2\tilde{\nabla}_i\tilde{\nabla}_j E
\nonumber\\
&& + \tilde{\nabla}_i E_j + \tilde{\nabla}_j E_i + 2E_{ij}]dx^i dx^j\bigg].
\label{9.2}
\end{eqnarray}
%
In (\ref{9.2})  $\tilde{\nabla}_i=\partial/\partial x^i$ and  $\tilde{\nabla}^i=\tilde{\gamma}^{ij}\tilde{\nabla}_j$  (with Latin indices) are defined with respect to the background three-space metric $\tilde{\gamma}_{ij}$. And with
%
\begin{eqnarray}
\tilde{\gamma}^{ij}\tilde{\nabla}_j V_i=\tilde{\gamma}^{ij}[\partial_j V_i-\tilde{\Gamma}^{k}_{ij}V_k]
\label{9.3}
\end{eqnarray}
%
for any 3-vector $V_i$ in a 3-space with 3-space connection $\tilde{\Gamma}^{k}_{ij}$, the elements of (\ref{9.2}) are required to obey
%
\begin{eqnarray}
\tilde{\gamma}^{ij}\tilde{\nabla}_j B_i = 0,\quad \tilde{\gamma}^{ij}\tilde{\nabla}_j E_i = 0, \quad E_{ij}=E_{ji},\quad \tilde{\gamma}^{jk}\tilde{\nabla}_kE_{ij} = 0, \quad\tilde{\gamma}^{ij}E_{ij} = 0.
\label{9.4}
\end{eqnarray}
%
With the  3-space sector of the background geometry being maximally 3-symmetric, it is described by a Riemann tensor of the form
%
\begin{eqnarray}
\tilde{R}_{ijk\ell}=k[\tilde{\gamma}_{jk}\tilde{\gamma}_{i\ell}-\tilde{\gamma}_{ik}\tilde{\gamma}_{j\ell}].
\label{9.5}
\end{eqnarray}
%
In \cite{amarasinghe_2019} (and as discussed in (\ref{9.43a}) to (\ref{9.47a}) below), it was shown that for the fluctuation metric given in (\ref{9.2}) with $\Omega(\tau)$ being arbitrary function of $\tau$, the gauge-invariant metric combinations are $\phi + \psi + \dot B - \ddot E$, $ - \dot\Omega^{-1}\Omega \psi + B - \dot E$, $B_i-\dot{E}_i$, and $E_{ij}$. As we shall see, the fluctuation equations will explicitly depend on these specific combinations.



We take the background $T_{\mu\nu}$ to be of the perfect fluid form
%
\begin{eqnarray}
T_{\mu\nu}=(\rho+p)U_{\mu}U_{\nu}+pg_{\mu\nu},
\label{9.6}
\end{eqnarray}
%
with fluctuation
%
\begin{eqnarray}
\delta T_{\mu\nu}=(\delta\rho+\delta p)U_{\mu}U_{\nu}+\delta pg_{\mu\nu}+(\rho+p)(\delta U_{\mu}U_{\nu}+U_{\mu}\delta U_{\nu})+ph_{\mu\nu}.
\label{9.7}
\end{eqnarray}
%
Here  $g^{\mu\nu}U_{\mu}U_{\nu}=-1$, $U^{0}=\Omega^{-1}(\tau)$, $U_0=-\Omega(\tau)$, $U^{i}=0$, $U_i=0$ for the background, while for the fluctuation we have 
%
\begin{eqnarray}
\delta g^{00}U_{0}U_{0}+2g^{00}U_{0}\delta U_{0}=0,
\label{9.8}
\end{eqnarray}
%
i.e. 
%
\begin{eqnarray}
\delta U_{0}=-\frac{1}{2}(g^{00})^{-1}(-g^{00}g^{00}\delta g_{00})U_{0}=-\Omega(\tau)\phi.
\label{9.9}
\end{eqnarray}
%
Thus just as in Sec. \ref{S8} we see that $\delta U_0$ is not an independent degree of freedom. As in Sec. \ref{S8} we shall set $\delta U_i=V_i+\tilde{\nabla}_iV$, where now $\tilde{\gamma}^{ij}\tilde{\nabla}_j V_i=\tilde{\gamma}^{ij}[\partial_j V_i-\tilde{\Gamma}^{k}_{ij}V_k]=0$. As constructed, in general we have 11 fluctuation variables, the six from the metric together with the three $\delta U_i$, and $\delta\rho$ and $\delta p$. But we only have ten fluctuation equations. Thus to solve the theory when there is both a $\delta \rho$ and a $\delta p$ we will need some constraint between $\delta p$ and $\delta \rho$, a point we return to below.

For the background Einstein equations we have
%
\begin{eqnarray}
G_{00}&=& -3k - 3 \dot{\Omega}^2\Omega^{-2},\quad G_{0i} =0,
\quad G_{ij} = \tilde{\gamma}_{ij}\left[k - \dot\Omega^2\Omega^{-2}+ 2\ddot\Omega \Omega^{-1}\right],
\nonumber\\
G_{00}+8\pi G T_{00} &=& -3k - 3 \dot\Omega^2\Omega^{-2} + \Omega^2 \rho=0,
\nonumber\\
G_{ij}+8\pi G T_{ij}&=& \tilde{\gamma}_{ij}\left[k - \dot\Omega^2\Omega^{-2} + 2\ddot\Omega \Omega^{-1}  + \Omega^2 p\right]=0,
\nonumber\\
\rho &=& 3k\Omega^{-2}+3\dot\Omega^2 \Omega^{-4},\quad p = -k\Omega^{-2} + \dot\Omega^2\Omega^{-4} -2\ddot\Omega \Omega^{-3},
\nonumber\\
 p &=& -\rho -\frac{1}{3} \frac{\Omega}{\dot\Omega}\dot\rho,
\label{9.10}
\end{eqnarray}
%
(after setting $8\pi G=1$), with the last relation following from $\nabla_{\nu}T^{\mu\nu}=0$, viz. conservation of the energy-momentum tensor in the full 4-space. To solve these equations we would need an equation of state that would relate $\rho$ and $p$. However we do not need to impose one just yet, as we shall generate the fluctuation equations as subject to (\ref{9.10}) but without needing to specify the form of $\Omega(\tau)$ or a relation  between $\rho$ and $p$.

For $\delta G_{\mu\nu}$ we have
%
\begin{eqnarray}
\delta G_{00}&=& -6 k \phi - 6 k \psi + 6 \dot{\psi} \dot{\Omega} \Omega^{-1} + 2 \dot{\Omega} \Omega^{-1} \tilde{\nabla}_{a}\tilde{\nabla}^{a}B - 2 \dot{\Omega} \Omega^{-1} \tilde{\nabla}_{a}\tilde{\nabla}^{a}\dot{E}
\nonumber\\
&& - 2 \tilde{\nabla}_{a}\tilde{\nabla}^{a}\psi, 
\nonumber\\ 
\delta G_{0i}&=& 3 k \tilde{\nabla}_{i}B -  \dot{\Omega}^2 \Omega^{-2} \tilde{\nabla}_{i}B + 2 \overset{..}{\Omega} \Omega^{-1} \tilde{\nabla}_{i}B - 2 k \tilde{\nabla}_{i}\dot{E} - 2 \tilde{\nabla}_{i}\dot{\psi} - 2 \dot{\Omega} \Omega^{-1} \tilde{\nabla}_{i}\phi\nonumber\\
&& +2 k B_{i} -  k \dot{E}_{i} -  B_{i} \dot{\Omega}^2 \Omega^{-2} + 2 B_{i} \overset{..}{\Omega} \Omega^{-1} + \tfrac{1}{2} \tilde{\nabla}_{a}\tilde{\nabla}^{a}B_{i} -  \tfrac{1}{2} \tilde{\nabla}_{a}\tilde{\nabla}^{a}\dot{E}_{i},
\nonumber\\ 
\delta G_{ij}&=& -2 \overset{..}{\psi}\tilde{\gamma}_{ij} + 2 \dot{\Omega}^2\tilde{\gamma}_{ij} \phi \Omega^{-2} + 2 \dot{\Omega}^2\tilde{\gamma}_{ij} \psi \Omega^{-2} - 2 \dot{\phi} \dot{\Omega}\tilde{\gamma}_{ij} \Omega^{-1} - 4 \dot{\psi} \dot{\Omega}\tilde{\gamma}_{ij} \Omega^{-1}\nonumber\\
&& - 4 \overset{..}{\Omega}\tilde{\gamma}_{ij} \phi \Omega^{-1} - 4 \overset{..}{\Omega}\tilde{\gamma}_{ij} \psi \Omega^{-1} - 2 \dot{\Omega}\tilde{\gamma}_{ij} \Omega^{-1} \tilde{\nabla}_{a}\tilde{\nabla}^{a}B - \tilde{\gamma}_{ij} \tilde{\nabla}_{a}\tilde{\nabla}^{a}\dot{B} 
\nonumber\\
&&+\tilde{\gamma}_{ij} \tilde{\nabla}_{a}\tilde{\nabla}^{a}\overset{..}{E} + 2 \dot{\Omega}\tilde{\gamma}_{ij} \Omega^{-1} \tilde{\nabla}_{a}\tilde{\nabla}^{a}\dot{E} 
 - \tilde{\gamma}_{ij} \tilde{\nabla}_{a}\tilde{\nabla}^{a}\phi +\tilde{\gamma}_{ij} \tilde{\nabla}_{a}\tilde{\nabla}^{a}\psi 
 \nonumber\\
 &&+ 2 \dot{\Omega} \Omega^{-1} \tilde{\nabla}_{j}\tilde{\nabla}_{i}B + \tilde{\nabla}_{j}\tilde{\nabla}_{i}\dot{B} -  \tilde{\nabla}_{j}\tilde{\nabla}_{i}\overset{..}{E} - 2 \dot{\Omega} \Omega^{-1} \tilde{\nabla}_{j}\tilde{\nabla}_{i}\dot{E}  + 2 k \tilde{\nabla}_{j}\tilde{\nabla}_{i}E
 \nonumber\\
 && - 2 \dot{\Omega}^2 \Omega^{-2} \tilde{\nabla}_{j}\tilde{\nabla}_{i}E + 4 \overset{..}{\Omega} \Omega^{-1} \tilde{\nabla}_{j}\tilde{\nabla}_{i}E + \tilde{\nabla}_{j}\tilde{\nabla}_{i}\phi -  \tilde{\nabla}_{j}\tilde{\nabla}_{i}\psi +\dot{\Omega} \Omega^{-1} \tilde{\nabla}_{i}B_{j}
 \nonumber\\
 && + \tfrac{1}{2} \tilde{\nabla}_{i}\dot{B}_{j}
 -  \tfrac{1}{2} \tilde{\nabla}_{i}\overset{..}{E}_{j} -  \dot{\Omega} \Omega^{-1} \tilde{\nabla}_{i}\dot{E}_{j} + k \tilde{\nabla}_{i}E_{j} -  \dot{\Omega}^2 \Omega^{-2} \tilde{\nabla}_{i}E_{j}
 \nonumber\\
 && + 2 \overset{..}{\Omega} \Omega^{-1} \tilde{\nabla}_{i}E_{j} + \dot{\Omega} \Omega^{-1} \tilde{\nabla}_{j}B_{i} + \tfrac{1}{2} \tilde{\nabla}_{j}\dot{B}_{i}  -  \tfrac{1}{2} \tilde{\nabla}_{j}\overset{..}{E}_{i} -  \dot{\Omega} \Omega^{-1} \tilde{\nabla}_{j}\dot{E}_{i} 
 \nonumber\\
 &&+ k \tilde{\nabla}_{j}E_{i} -  \dot{\Omega}^2 \Omega^{-2} \tilde{\nabla}_{j}E_{i} + 2 \overset{..}{\Omega} \Omega^{-1} \tilde{\nabla}_{j}E_{i}- \overset{..}{E}_{ij} - 2 \dot{\Omega}^2 E_{ij} \Omega^{-2} \nonumber \\ 
&& - 2 \dot{E}_{ij} \dot{\Omega} \Omega^{-1} + 4 \overset{..}{\Omega} E_{ij} \Omega^{-1} + \tilde{\nabla}_{a}\tilde{\nabla}^{a}E_{ij},
\nonumber\\
g^{\mu\nu}\delta G_{\mu\nu} &=& 6 \dot{\Omega}^2 \phi \Omega^{-4} + 6 \dot{\Omega}^2 \psi \Omega^{-4} - 6 \dot{\phi} \dot{\Omega} \Omega^{-3} - 18 \dot{\psi} \dot{\Omega} \Omega^{-3} - 12 \overset{..}{\Omega} \phi \Omega^{-3}
\nonumber\\
&& - 12 \overset{..}{\Omega} \psi \Omega^{-3} - 6 \overset{..}{\psi} \Omega^{-2} + 6 k \phi \Omega^{-2}  + 6 k \psi \Omega^{-2} - 6 \dot{\Omega} \Omega^{-3} \tilde{\nabla}_{a}\tilde{\nabla}^{a}B
\nonumber\\
&& - 2 \Omega^{-2} \tilde{\nabla}_{a}\tilde{\nabla}^{a}\dot{B} + 2 \Omega^{-2} \tilde{\nabla}_{a}\tilde{\nabla}^{a}\overset{..}{E} + 6 \dot{\Omega} \Omega^{-3} \tilde{\nabla}_{a}\tilde{\nabla}^{a}\dot{E} \nonumber \\ 
&& - 2 \dot{\Omega}^2 \Omega^{-4} \tilde{\nabla}_{a}\tilde{\nabla}^{a}E + 4 \overset{..}{\Omega} \Omega^{-3} \tilde{\nabla}_{a}\tilde{\nabla}^{a}E + 2 k \Omega^{-2} \tilde{\nabla}_{a}\tilde{\nabla}^{a}E
\nonumber\\
&& - 2 \Omega^{-2} \tilde{\nabla}_{a}\tilde{\nabla}^{a}\phi + 4 \Omega^{-2} \tilde{\nabla}_{a}\tilde{\nabla}^{a}\psi. 
\label{9.11}
\end{eqnarray}
%
We introduce
%
\begin{eqnarray}
\alpha  &=& \phi + \psi + \dot B - \ddot E,\quad \gamma = - \dot\Omega^{-1}\Omega \psi + B - \dot E,\quad \hat{V} = V-\Omega^2 \dot \Omega^{-1}\psi,
\nonumber\\
\delta \hat{\rho}&=&\delta \rho - 12 \dot{\Omega}^2 \psi \Omega^{-4} + 6 \overset{..}{\Omega} \psi \Omega^{-3} - 6 k \psi \Omega^{-2}=\delta\rho +\frac{\Omega}{\dot{\Omega}}\dot{\rho}\psi=\delta \rho-3(\rho+p)\psi,
\nonumber\\
\delta \hat{p}&=&\delta p - 4 \dot{\Omega}^2 \psi \Omega^{-4} + 8 \overset{..}{\Omega} \psi \Omega^{-3} + 2 k \psi \Omega^{-2} - 2 \overset{...}{\Omega} \dot{\Omega}^{-1} \psi \Omega^{-2}=\delta p +\frac{\Omega}{\dot{\Omega}}\dot{p}\psi,
\label{9.12}
\end{eqnarray}
%
(in (\ref{9.12}) we used (\ref{9.10})), where, as we show below, the functions $\delta \hat{\rho}$, $\delta \hat{p}$ and $\hat{V} $ are gauge invariant. (The $\gamma$ introduced in (\ref{8.7}) is a special case of the $\gamma$ introduced in (\ref{9.12}).) Given (\ref{9.12}) we can express the components of $\Delta_{\mu\nu}=\delta G_{\mu\nu}+8\pi G\delta T_{\mu\nu}$ quite compactly. Specifically,  on using (\ref{9.10}) for the background but without imposing any relation between the background $\rho$ and $p$, we obtain evolution equations of the form 
%
\begin{eqnarray}
\Delta_{00}&=& 6 \dot{\Omega}^2 \Omega^{-2}(\alpha-\dot\gamma) + \delta \hat{\rho} \Omega^2 + 2 \dot{\Omega} \Omega^{-1} \tilde{\nabla}_{a}\tilde{\nabla}^{a}\gamma=0, 
\label{9.13}
\end{eqnarray}
%
%
\begin{eqnarray}
\Delta_{0i}&=& -2 \dot{\Omega} \Omega^{-1} \tilde{\nabla}_{i}(\alpha - \dot\gamma) + 2 k \tilde{\nabla}_{i}\gamma 
+(-4 \dot{\Omega}^2 \Omega^{-3}  + 2 \overset{..}{\Omega} \Omega^{-2}  - 2 k \Omega^{-1}) \tilde{\nabla}_{i}\hat{V}
\nonumber\\
&& +k(B_i-\dot E_i)+ \tfrac{1}{2} \tilde{\nabla}_{a}\tilde{\nabla}^{a}(B_{i} - \dot{E}_{i})
+ (-4 \dot{\Omega}^2 \Omega^{-3} 
\nonumber\\
&&+ 2 \overset{..}{\Omega} \Omega^{-2} - 2 k \Omega^{-1})V_{i}=0,
\label{9.14}
\end{eqnarray}
%
%
\begin{eqnarray}
\Delta_{ij}&=& \tilde{\gamma}_{ij}\big[ 2 \dot{\Omega}^2 \Omega^{-2}(\alpha-\dot\gamma)
-2  \dot{\Omega} \Omega^{-1}(\dot\alpha -\ddot\gamma)-4\ddot\Omega\Omega^{-1}(\alpha-\dot\gamma)
+ \Omega^2 \delta \hat{p}
\nonumber\\
&&-\tilde\nabla_a\tilde\nabla^a( \alpha + 2\dot\Omega \Omega^{-1}\gamma) \big] 
+\tilde\nabla_i\tilde\nabla_j( \alpha + 2\dot\Omega \Omega^{-1}\gamma)
+\dot{\Omega} \Omega^{-1} \tilde{\nabla}_{i}(B_{j}-\dot E_j)
\nonumber\\
&&+\tfrac{1}{2} \tilde{\nabla}_{i}(\dot{B}_{j}-\ddot{E}_j)
+\dot{\Omega} \Omega^{-1} \tilde{\nabla}_{j}(B_{i}-\dot E_i)+\tfrac{1}{2} \tilde{\nabla}_{j}(\dot{B}_{i}-\ddot{E}_i)
\nonumber\\
&&- \overset{..}{E}_{ij} - 2 k E_{ij} - 2 \dot{E}_{ij} \dot{\Omega} \Omega^{-1} + \tilde{\nabla}_{a}\tilde{\nabla}^{a}E_{ij}=0,
\label{9.15}
\end{eqnarray}
%
%
\begin{eqnarray}
\tilde{\gamma}^{ij}\Delta_{ij} &=&  6 \dot{\Omega}^2 \Omega^{-2}(\alpha-\dot\gamma)
-6  \dot{\Omega} \Omega^{-1}(\dot\alpha -\ddot\gamma)-12\ddot\Omega\Omega^{-1}(\alpha-\dot\gamma)+ 3\Omega^2 \delta \hat{p}
\nonumber\\
&&-2\tilde\nabla_a\tilde\nabla^a( \alpha + 2\dot\Omega \Omega^{-1}\gamma)=0,
\label{9.16}
\end{eqnarray}
%
%
\begin{eqnarray}
g^{\mu\nu}\Delta_{\mu\nu}&=& 3 \delta \hat{p} -  \delta \hat{\rho}
-12 \overset{..}{\Omega}  \Omega^{-3}(\alpha - \dot\gamma) -6 \dot{\Omega} \Omega^{-3}(\dot{\alpha} -\ddot\gamma)
\nonumber\\
&&
-2 \Omega^{-2} \tilde{\nabla}_{a}\tilde{\nabla}^{a}(\alpha +3\dot\Omega\Omega^{-1}\gamma)=0.
\label{9.17}
\end{eqnarray}
%

Starting from the general identities
%
\begin{eqnarray}
\nabla_{k}\nabla_{n}T_{\ell m}-\nabla_{n}\nabla_{k}T_{\ell m}&=&T^{s}_{\phantom{s}m}R_{\ell s n k}+T_{\ell}^{\phantom{\ell}s}R_{ms n k},
\nonumber\\
\nabla_{k}\nabla_{n}A_{m}-\nabla_{n}\nabla_{k}A_{m}&=&A^{s}R_{ms n k}
\label{9.18}
\end{eqnarray}
%
that hold for any rank two tensor or vector in any geometry, for the 3-space Robertson-Walker geometry where $\tilde{R}_{msnk}=k(\tilde{\gamma}_{sn}\tilde{\gamma}_{mk}-\tilde{\gamma}_{mn}\tilde{\gamma}_{sk})$ 
we obtain
%
\begin{eqnarray}
&&\tilde\nabla_i\tilde\nabla_a\tilde\nabla^aA_j-\tilde\nabla_a\tilde\nabla^a\tilde\nabla_iA_j 
= 2k\tilde{\gamma}_{ij}\tilde{\nabla}_aA^a-2k(\tilde\nabla_i A_j + \tilde\nabla_j A_i),
\nonumber\\
&&\tilde\nabla^j\tilde\nabla_a\tilde\nabla^aA_j=
(\tilde\nabla_a\tilde\nabla^a+2k)\tilde\nabla^j A_j,\quad \tilde{\nabla}^j\tilde{\nabla}_iA_j=\tilde{\nabla}_i\tilde{\nabla}^jA_j+2kA_i
\label{9.19}
\end{eqnarray}
%
for any 3-vector $A_i$ in a maximally symmetric 3-geometry with 3-curvature $k$. Similarly, noting that for any scalar $S$ in any geometry we have
%
\begin{eqnarray}
&&\nabla_a\nabla_b\nabla_iS=\nabla_a\nabla_i\nabla_bS=\nabla_i\nabla_a\nabla_bS+\nabla^sSR_{bsia},
\nonumber\\
&&\nabla_{\ell}\nabla_k\nabla_n\nabla_{m}S=\nabla_n\nabla_{m}\nabla_{\ell}\nabla_kS
+\nabla_{n}[\nabla^sSR_{ksm\ell}]
+\nabla^s\nabla_kSR_{msn\ell}
\nonumber\\
&&\qquad\qquad+\nabla_m\nabla^sSR_{ksn\ell}
+\nabla_{\ell}[\nabla^sSR_{msnk}],
\label{9.20}
\end{eqnarray}
%
in a Robertson-Walker 3-geometry background we obtain 
%
\begin{eqnarray}
\tilde{\nabla}_a\tilde{\nabla}^a\tilde{\nabla}_iS&=&\tilde{\nabla}_i\tilde{\nabla}_a\tilde{\nabla}^aS+2k\tilde{\nabla}_iS,
\nonumber\\ \tilde{\nabla}_a\tilde{\nabla}^a\tilde{\nabla}_i\tilde{\nabla}_jS&=&\tilde{\nabla}_i\tilde{\nabla}_j\tilde{\nabla}_a\tilde{\nabla}^aS+6k(\tilde{\nabla}_i\tilde{\nabla}_j-\tfrac{1}{3}\tilde{\gamma}_{ij}\tilde{\nabla}_a\tilde{\nabla}^a)S,
\nonumber\\
\tilde{\nabla}_a\tilde{\nabla}^a\tilde{\nabla}_i\tilde{\nabla}_{j}S&=&\tilde{\nabla}_i\tilde{\nabla}_{j}\tilde{\nabla}_a\tilde{\nabla}^aS
+6k\tilde{\nabla}_i\tilde{\nabla}_{j}S-2k\tilde{\gamma}_{ij}\tilde{\nabla}_a\tilde{\nabla}^aS.
\label{9.21}
\end{eqnarray}
%
Thus we find that 
%
\begin{eqnarray}
\tilde\nabla^i \Delta_{0i} &=& 
\tilde\nabla_a\tilde\nabla^a\big[ -2 \dot{\Omega} \Omega^{-1} (\alpha - \dot\gamma) + 2 k \gamma 
+(-4 \dot{\Omega}^2 \Omega^{-3}  + 2 \overset{..}{\Omega} \Omega^{-2}  - 2 k \Omega^{-1}) \hat{V}\big]
\nonumber\\
&=&0,
\label{9.22}
\end{eqnarray}
%
and thus
%
\begin{eqnarray}
(\tilde{\nabla}_k\tilde\nabla^k -2k)\Delta_{0i} &=& (\tilde{\nabla}_k\tilde\nabla^k-2k)\bigg[k(B_i-\dot E_i)+ \tfrac{1}{2} \tilde{\nabla}_{a}\tilde{\nabla}^{a}(B_{i} - \dot{E}_{i})
\nonumber\\
&&
+ (-4 \dot{\Omega}^2 \Omega^{-3} + 2 \overset{..}{\Omega} \Omega^{-2} - 2 k \Omega^{-1})V_{i}\bigg]
=0.~~~~~~
\label{9.23}
\end{eqnarray}
%
Also we obtain
%
\begin{eqnarray}
\epsilon^{ij\ell}\tilde{\nabla}_j\Delta_{0i}&=&\epsilon^{ij\ell}\tilde{\nabla}_j\bigg[k(B_i-\dot E_i)+ \tfrac{1}{2} \tilde{\nabla}_{a}\tilde{\nabla}^{a}(B_{i} - \dot{E}_{i})
\nonumber\\
&&
+ (-4 \dot{\Omega}^2 \Omega^{-3} + 2 \overset{..}{\Omega} \Omega^{-2} - 2 k \Omega^{-1})V_{i}\bigg]=0.
\label{9.24}
\end{eqnarray}
%
Conferring Sec. \ref{ss:ds4_svt4}, we have noted that in a de Sitter space if a tensor $A^{P}_{\phantom{P}M}$ is transverse and traceless then so is $\nabla_L\nabla^LA^{P}_{\phantom{P}M}$ Since this holds in any maximally symmetric space the quantity  $\tilde{\nabla}_a\tilde{\nabla}^aE_{ij}$ is transverse and traceless too. Thus given (\ref{9.19}) we  obtain
%
\begin{eqnarray}
\tilde\nabla^j\Delta_{ij}&=& \tilde{\nabla}_{i}[ 2 \dot{\Omega}^2 \Omega^{-2}(\alpha-\dot\gamma)
-2  \dot{\Omega} \Omega^{-1}(\dot\alpha -\ddot\gamma)-4\ddot\Omega\Omega^{-1}(\alpha-\dot\gamma)+ \Omega^2 \delta \hat{p}
\nonumber\\
&&+ 2 k(\alpha + 2 \dot{\Omega}  \Omega^{-1} \gamma)]
+[ \tilde{\nabla}_{a}\tilde{\nabla}^{a}+2k][\tfrac{1}{2}(\dot{B}_i-\ddot{E}_i)+\dot{\Omega}\Omega^{-1}(B_i-\dot{E}_i)]=0,
\nonumber\\
\label{9.25}
\end{eqnarray}
%
%
\begin{eqnarray}
\tilde\nabla^i\tilde\nabla^j\Delta_{ij}&=& \tilde{\nabla}_{a}\tilde{\nabla}^{a}[2 \dot{\Omega}^2 \Omega^{-2}(\alpha-\dot\gamma)
-2  \dot{\Omega} \Omega^{-1}(\dot\alpha -\ddot\gamma)-4\ddot\Omega\Omega^{-1}(\alpha-\dot\gamma)+ \Omega^2 \delta \hat{p}
\nonumber \\ 
&& + 2 k(\alpha + 2 \dot{\Omega}  \Omega^{-1} \gamma)]=0.
\label{9.26}
\end{eqnarray}
%
Thus we obtain
%
\begin{eqnarray}
3\tilde\nabla^i\tilde\nabla^j\Delta_{ij}-\tilde\nabla_a\tilde\nabla^a(\tilde{\gamma}^{ij}\Delta_{ij})
=2\tilde{\nabla}^2[\tilde{\nabla}^2+3k](\alpha+2\dot{\Omega}\Omega^{-1}\gamma)=0,
\label{9.27}
\end{eqnarray}
%
%
\begin{eqnarray}
\tilde\nabla^i\tilde\nabla^j\Delta_{ij}+k\tilde{\gamma}^{ij}\Delta_{ij}
&=&[\tilde{\nabla}^2+3k][2 \dot{\Omega}^2 \Omega^{-2}(\alpha-\dot\gamma)
-2  \dot{\Omega} \Omega^{-1}(\dot\alpha -\ddot\gamma)
\nonumber\\
&&-4\ddot\Omega\Omega^{-1}(\alpha-\dot\gamma)+ \Omega^2 \delta \hat{p}]=0.
\label{9.28}
\end{eqnarray}
%
We now define $A=2 \dot{\Omega}^2 \Omega^{-2}(\alpha-\dot\gamma)-2  \dot{\Omega} \Omega^{-1}(\dot\alpha -\ddot\gamma)-4\ddot\Omega\Omega^{-1}(\alpha-\dot\gamma)+ \Omega^2 \delta \hat{p}$ and $C=\alpha+2\dot{\Omega}\Omega^{-1}\gamma$. And using (\ref{9.21}) obtain
%
\begin{eqnarray}
(\tilde{\nabla}_a\tilde{\nabla}^a+k)\tilde{\nabla}_i(A+2kC)=\tilde{\nabla}_i(\tilde{\nabla}_a\tilde{\nabla}^a+3k)(A+2kC),
\label{9.29}
\end{eqnarray}
%
and thus with  (\ref{9.27}) and (\ref{9.28}) obtain
%
\begin{eqnarray}
(\tilde{\nabla}_a\tilde{\nabla}^a-2k)(\tilde{\nabla}_a\tilde{\nabla}^a+k)\tilde{\nabla}_i(A+2kC)&=&
\tilde{\nabla}_i\tilde{\nabla}_a\tilde{\nabla}^a(\tilde{\nabla}_b\tilde{\nabla}^b+3k)(A+2kC)
\nonumber\\
&=&0.
\label{9.30}
\end{eqnarray}
%
Consequently, on comparing with (\ref{9.25})  we obtain
%
\begin{eqnarray}
(\tilde{\nabla}_a\tilde{\nabla}^a-2k)(\tilde{\nabla}_b\tilde{\nabla}^b+k)\tilde\nabla^j\Delta_{ij}&=&
(\tilde{\nabla}_a\tilde{\nabla}^a-2k)(\tilde{\nabla}_b\tilde{\nabla}^b+k)[\tilde{\nabla}_{c}\tilde{\nabla}^{c}+2k]\times
\nonumber\\
&&[\tfrac{1}{2}(\dot{B}_i-\ddot{E}_i)+\dot{\Omega}\Omega^{-1}(B_i-\dot{E}_i)]=0,
\label{9.31}
\end{eqnarray}
%
to give a relation that only involves $B_i-\dot{E}_{i}$.

To obtain a relation that involves $E_{ij}$ we proceed as follows. We note that sector of $\Delta_{ij}$ that contains the above $A$ and $C$ can be written as 
%
\begin{eqnarray}
D_{ij}=\tilde{\gamma}_{ij}(A-\tilde{\nabla}_a\tilde{\nabla}^aC)+\tilde{\nabla}_i\tilde{\nabla}_jC.
\label{9.32}
\end{eqnarray}
%
We thus introduce 
%
\begin{eqnarray}
A_{ij}&=&D_{ij}-\frac{1}{3}\tilde{\gamma}_{ij}\tilde{\gamma}^{ab}D_{ab}=(\tilde{\nabla}_i\tilde{\nabla}_j-\tfrac{1}{3} \tilde{\gamma}_{ij}\tilde{\nabla}_a\tilde{\nabla}^a)C,
\nonumber\\
B_{ij}&=&\Delta_{ij}-\frac{1}{3}\tilde{\gamma}_{ij}\tilde{\gamma}^{ab}\Delta_{ab}=(\tilde{\nabla}_i\tilde{\nabla}_j-\tfrac{1}{3} \tilde{\gamma}_{ij}\tilde{\nabla}_a\tilde{\nabla}^a)C
\nonumber\\
&+&\dot{\Omega} \Omega^{-1} \tilde{\nabla}_{i}(B_{j}-\dot E_j)+\tfrac{1}{2} \tilde{\nabla}_{i}(\dot{B}_{j}-\ddot{E}_j)
+\dot{\Omega} \Omega^{-1} \tilde{\nabla}_{j}(B_{i}-\dot E_i)+\tfrac{1}{2} \tilde{\nabla}_{j}(\dot{B}_{i}-\ddot{E}_i)
\nonumber\\
&-& \overset{..}{E}_{ij} - 2 k E_{ij} - 2 \dot{E}_{ij} \dot{\Omega} \Omega^{-1} + \tilde{\nabla}_{a}\tilde{\nabla}^{a}E_{ij}=0,
\label{9.33}
\end{eqnarray}
%
with (\ref{9.33}) defining $A_{ij}$ and $B_{ij}$, and with $A$ dropping out. Using (\ref{9.19}) and the third relation in (\ref{9.21}) we obtain
%
\begin{eqnarray}
(\tilde{\nabla}_b\tilde{\nabla}^b-3k)A_{ij}=
(\tilde{\nabla}_i\tilde{\nabla}_j-\tfrac{1}{3} \tilde{\gamma}_{ij}\tilde{\nabla}_a\tilde{\nabla}^a)(\tilde{\nabla}_b\tilde{\nabla}^b+3k)C,
\label{9.34}
\end{eqnarray}
%
and via (\ref{9.21}) and (\ref{9.22}) thus obtain 
%
\begin{eqnarray}
(\tilde{\nabla}_a\tilde{\nabla}^a-6k)(\tilde{\nabla}_b\tilde{\nabla}^b-3k)A_{ij}&=&
(\tilde{\nabla}_i\tilde{\nabla}_j-\tfrac{1}{3} \tilde{\gamma}_{ij}\tilde{\nabla}_a\tilde{\nabla}^a)\tilde{\nabla}_b\tilde{\nabla}^b(\tilde{\nabla}_c\tilde{\nabla}^c+3k)C
\nonumber\\
&=&0.
\label{9.35}
\end{eqnarray}
%
Comparing with the structure of $\Delta_{ij}$ and $\tilde{\gamma}^{ij}\Delta_{ij}$, we thus obtain
%
\begin{eqnarray}
&&(\tilde{\nabla}_a\tilde{\nabla}^a-6k)(\tilde{\nabla}_b\tilde{\nabla}^b-3k)[B_{ij}-A_{ij}]
=(\tilde{\nabla}_a\tilde{\nabla}^a-6k)(\tilde{\nabla}_b\tilde{\nabla}^b-3k)
\nonumber\\
&&\times
\big{[}\dot{\Omega} \Omega^{-1} \tilde{\nabla}_{i}(B_{j}-\dot E_j)+\tfrac{1}{2} \tilde{\nabla}_{i}(\dot{B}_{j}-\ddot{E}_j)
+\dot{\Omega} \Omega^{-1} \tilde{\nabla}_{j}(B_{i}-\dot E_i)+\tfrac{1}{2} \tilde{\nabla}_{j}(\dot{B}_{i}-\ddot{E}_i)
\nonumber\\
&& -\overset{..}{E}_{ij} - 2 k E_{ij} - 2  \dot{\Omega} \Omega^{-1}\dot{E}_{ij} + \tilde{\nabla}_{a}\tilde{\nabla}^{a}E_{ij}\big{]}=0.
\label{9.36}
\end{eqnarray}
%
We now note that for any vector $A_i$ that obeys $\tilde{\nabla}^iA_i=0$, through repeated use of the first relation in (\ref{9.19}) we obtain 
%
\begin{eqnarray}
&&(\tilde{\nabla}_b\tilde{\nabla}^b-3k)(\tilde{\nabla}_iA_j+\tilde{\nabla}_jA_i)=
\tilde{\nabla}_i(\tilde{\nabla}_b\tilde{\nabla}^b+k)A_j+
\tilde{\nabla}_j(\tilde{\nabla}_b\tilde{\nabla}^b+k)A_i,
\nonumber\\
&&(\tilde{\nabla}_a\tilde{\nabla}^a-6k)(\tilde{\nabla}_b\tilde{\nabla}^b-3k)(\tilde{\nabla}_iA_j+\tilde{\nabla}_jA_i)
=\tilde{\nabla}_i(\tilde{\nabla}_a\tilde{\nabla}^a-2k)(\tilde{\nabla}_b\tilde{\nabla}^b+k)A_j
\nonumber\\
&&\qquad\qquad+
\tilde{\nabla}_j(\tilde{\nabla}_a\tilde{\nabla}^a-2k)(\tilde{\nabla}_b\tilde{\nabla}^b+k)A_i.
\label{9.37}
\end{eqnarray}
%
On using the first relation in (\ref{9.19}) again,  it follows that 
%
\begin{eqnarray}
&&(\tilde{\nabla}_c\tilde{\nabla}^c-2k)(\tilde{\nabla}_a\tilde{\nabla}^a-6k)(\tilde{\nabla}_b\tilde{\nabla}^b-3k)(\tilde{\nabla}_iA_j+\tilde{\nabla}_jA_i)
\nonumber\\
&=&\tilde{\nabla}_i(\tilde{\nabla}_c\tilde{\nabla}^c+2k)(\tilde{\nabla}_a\tilde{\nabla}^a-2k)(\tilde{\nabla}_b\tilde{\nabla}^b+k)A_j
\nonumber\\
&&+
\tilde{\nabla}_j(\tilde{\nabla}_c\tilde{\nabla}^c+2k)(\tilde{\nabla}_a\tilde{\nabla}^a-2k)(\tilde{\nabla}_b\tilde{\nabla}^b+k)A_i.
\label{9.38}
\end{eqnarray}
%
On setting $A_i=\tfrac{1}{2}(\dot{B}_i-\ddot{E}_i)+\dot{\Omega}\Omega^{-1}(B_i-\dot{E}_i)$ (so that $A_i$ is such that $\tilde{\nabla}^iA_i=0$), and recalling (\ref{9.31}) we obtain
%
\begin{eqnarray}
&&(\tilde{\nabla}_c\tilde{\nabla}^c-2k)(\tilde{\nabla}_a\tilde{\nabla}^a-6k)(\tilde{\nabla}_b\tilde{\nabla}^b-3k)
\nonumber\\
&&\times\bigg{[}
\tilde{\nabla}_i[\tfrac{1}{2}(\dot{B}_j-\ddot{E}_j)+\dot{\Omega}\Omega^{-1}(B_j-\dot{E}_j)]+\tilde{\nabla}_j[\tfrac{1}{2}(\dot{B}_i-\ddot{E}_i)+\dot{\Omega}\Omega^{-1}(B_i-\dot{E}_i)]\bigg{]}\nonumber\\
&&=0.
\label{9.39}
\end{eqnarray}
%
Thus finally from (\ref{9.36}) we obtain
%
\begin{eqnarray}
&&(\tilde{\nabla}_c\tilde{\nabla}^c-2k)(\tilde{\nabla}_a\tilde{\nabla}^a-6k)(\tilde{\nabla}_b\tilde{\nabla}^b-3k)\times
\nonumber\\
&&
\big{[}
- \overset{..}{E}_{ij} - 2 k E_{ij} - 2  \dot{\Omega} \Omega^{-1}\dot{E}_{ij} + \tilde{\nabla}_{a}\tilde{\nabla}^{a}E_{ij}\big{]}=0.
\label{9.40}
\end{eqnarray}
%
Thus with (\ref{9.27}), (\ref{9.28}), (\ref{9.31}), (\ref{9.40}) together with (\ref{9.13}), (\ref{9.22}) and (\ref{9.23}) we have succeeded in decomposing the fluctuation equations for the components, with the various components obeying derivative equations that are higher than second order.

With (\ref{9.27}) only involving $\alpha +2\dot{\Omega}\Omega^{-1}\gamma$, with (\ref{9.31}) only involving 
$B_i-\dot{E}_i$, and with (\ref{9.40}) only involving $E_{ij}$, and with all components of $\Delta_{\mu\nu}$ being gauge invariant, we recognize $C=\alpha +2\dot{\Omega}\Omega^{-1}\gamma$, $B_i-\dot{E}_i$ and $E_{ij}$ as being gauge invariant. With $B_i-\dot{E}_i$ being gauge invariant, from (\ref{9.23}) we recognize $V_i$ as being gauge invariant too. While we have identified some gauge-invariant quantities we note that by manipulating $\Delta_{\mu\nu}$ so as to obtain derivative expressions in which each of these quantities appears on its own, we cannot establish the gauge invariance of all 11 of the fluctuation variables this way since $\Delta_{\mu\nu}$ only has 10 components. However, just as with fluctuations around flat spacetime, in analog to (\ref{2.6}) below we shall obtain derivative relations between the SVT3 fluctuations and the $h_{\mu\nu}$ fluctuations by manipulating  (\ref{9.2}). As we show below, this will enable us  to establish the gauge invariance of the remaining fluctuation quantities.

\subsubsection{Decomposition Theorem Requirements}
\label{sss:what_is_needed_decomposition_theorem_svt3}

To get a decomposition theorem for $\Delta_{\mu\nu}=0$ we would require 
%
\begin{eqnarray}
0&=&6 \dot{\Omega}^2 \Omega^{-2}(\alpha-\dot\gamma) + \delta \hat{\rho}{} \Omega^2 + 2 \dot{\Omega} \Omega^{-1} \tilde{\nabla}_{a}\tilde{\nabla}^{a}\gamma, 
\nonumber\\
0&=&-2 \dot{\Omega} \Omega^{-1} \tilde{\nabla}_{i}(\alpha - \dot\gamma) + 2 k \tilde{\nabla}_{i}\gamma 
+(-4 \dot{\Omega}^2 \Omega^{-3}  + 2 \overset{..}{\Omega} \Omega^{-2}  - 2 k \Omega^{-1}) \tilde{\nabla}_{i}\hat{V},
\nonumber\\
0&=&k(B_i-\dot E_i)+ \tfrac{1}{2} \tilde{\nabla}_{a}\tilde{\nabla}^{a}(B_{i} - \dot{E}_{i})
+ (-4 \dot{\Omega}^2 \Omega^{-3} + 2 \overset{..}{\Omega} \Omega^{-2} - 2 k \Omega^{-1})V_{i},
\nonumber\\
0&=&\tilde{\gamma}_{ij}\big[ 2 \dot{\Omega}^2 \Omega^{-2}(\alpha-\dot\gamma)
-2  \dot{\Omega} \Omega^{-1}(\dot\alpha -\ddot\gamma)-4\ddot\Omega\Omega^{-1}(\alpha-\dot\gamma)+ \Omega^2 \delta \hat{p}
\nonumber\\
&&-\tilde\nabla_a\tilde\nabla^a( \alpha + 2\dot\Omega \Omega^{-1}\gamma) \big] 
+\tilde\nabla_i\tilde\nabla_j( \alpha + 2\dot\Omega \Omega^{-1}\gamma),
\nonumber\\
0&=&\dot{\Omega} \Omega^{-1} \tilde{\nabla}_{i}(B_{j}-\dot E_j)+\tfrac{1}{2} \tilde{\nabla}_{i}(\dot{B}_{j}-\ddot{E}_j)
+\dot{\Omega} \Omega^{-1} \tilde{\nabla}_{j}(B_{i}-\dot E_i)+\tfrac{1}{2} \tilde{\nabla}_{j}(\dot{B}_{i}-\ddot{E}_i),
\nonumber\\
0&=&- \overset{..}{E}_{ij} - 2 k E_{ij} - 2 \dot{E}_{ij} \dot{\Omega} \Omega^{-1} + \tilde{\nabla}_{a}\tilde{\nabla}^{a}E_{ij},
\nonumber\\
0&=&6 \dot{\Omega}^2 \Omega^{-2}(\alpha-\dot\gamma)
-6  \dot{\Omega} \Omega^{-1}(\dot\alpha -\ddot\gamma)-12\ddot\Omega\Omega^{-1}(\alpha-\dot\gamma)+ 3\Omega^2 \delta \hat{p}
\nonumber\\
&&-2\tilde\nabla_a\tilde\nabla^a( \alpha + 2\dot\Omega \Omega^{-1}\gamma),
\nonumber\\
0&=&3 \delta \hat{p}{} -  \delta \hat{\rho}
-12 \overset{..}{\Omega}  \Omega^{-3}(\alpha - \dot\gamma) -6 \dot{\Omega} \Omega^{-3}(\dot{\alpha} -\ddot\gamma)
-2 \Omega^{-2} \tilde{\nabla}_{a}\tilde{\nabla}^{a}(\alpha +3\dot\Omega\Omega^{-1}\gamma).
\nonumber\\
\label{9.41}
\end{eqnarray}
%
With $\tilde{\gamma}_{ij}$ and $\tilde{\nabla}_{i}\tilde{\nabla}_j$ not being equal to each other, we would immediately obtain 
%
\begin{eqnarray}
2 \dot{\Omega}^2 \Omega^{-2}(\alpha-\dot\gamma)
-2  \dot{\Omega} \Omega^{-1}(\dot\alpha -\ddot\gamma)-4\ddot\Omega\Omega^{-1}(\alpha-\dot\gamma)+ \Omega^2 \delta \hat{p}  &=&0,
\nonumber\\
\alpha + 2\dot\Omega \Omega^{-1}\gamma&=&0.
\label{9.42}
\end{eqnarray}
%
We recognize the equations for the components of the fluctuations as being derivatives of the relations that are required of the decomposition theorem. We thus need to see if we can find boundary conditions that would force the solutions to the higher-derivative fluctuation equations to obey (\ref{9.41}) and (\ref{9.42}).

\subsubsection{Gauge Invariance}
\label{sss:establishing_gauge_invariance}

Starting with (\ref{9.2}), setting $h_{\mu\nu}=\Omega^2(\tau)f_{\mu\nu}$,  $f=\tilde{\gamma}^{ij}f_{ij}=-6\psi+2\tilde{\nabla}_i\tilde{\nabla}^iE$ and taking appropriate derivatives, then following quite a bit of algebra we obtain
%
\begin{eqnarray}
(3k+\tilde{\nabla}^b\tilde{\nabla}_b)\tilde{\nabla}^a\tilde{\nabla}_a\alpha&=&-\frac{1}{2}(3k+\tilde{\nabla}^b\tilde{\nabla}_b)\tilde{\nabla}^i\tilde{\nabla}_if_{00}
\nonumber\\
&&
+\frac{1}{4}\tilde{\nabla}^a\tilde{\nabla}_a\left(-2kf-\tilde{\nabla}^b\tilde{\nabla}_bf+\tilde{\nabla}^m\tilde{\nabla}^nf_{mn}\right)
\nonumber\\
&&
+\partial_0(3k+\tilde{\nabla}^b\tilde{\nabla}_b)\tilde{\nabla}^if_{0i}
-\frac{1}{4}\partial^2_0\left(3\tilde{\nabla}^m\tilde{\nabla}^nf_{mn}-\tilde{\nabla}^a\tilde{\nabla}_af\right),
\nonumber\\
\label{9.43a}
\end{eqnarray}
%
%
\begin{eqnarray}
(3k+\tilde{\nabla}^b\tilde{\nabla}_b)\tilde{\nabla}^a\tilde{\nabla}_a\gamma
&=&-\frac{1}{4}\Omega\dot{\Omega}^{-1}\tilde{\nabla}^a\tilde{\nabla}_a\left(-2kf-\tilde{\nabla}^b\tilde{\nabla}_bf+\tilde{\nabla}^m\tilde{\nabla}^nf_{mn}\right)
\nonumber\\
&&
+\bigg[(3k+\tilde{\nabla}^b\tilde{\nabla}_b)\tilde{\nabla}^if_{0i}-\frac{1}{4}\partial_0\big(3\tilde{\nabla}^m\tilde{\nabla}^nf_{mn}
\nonumber\\
&&-\tilde{\nabla}^a\tilde{\nabla}_af\big)\bigg],
\label{9.44a}
\end{eqnarray}
%
%
\begin{eqnarray}
(\tilde{\nabla}^a\tilde{\nabla}_a-2k)(\tilde{\nabla}^i\tilde{\nabla}_i +2k)(B_j-\dot{E_j})&=&(\tilde{\nabla}^i\tilde{\nabla}_i +2k)(\tilde{\nabla}^a\tilde{\nabla}_af_{0j}-2kf_{0j}
\nonumber\\
&&
-\tilde{\nabla}_j\tilde{\nabla}^af_{0a})
-\partial_0\tilde{\nabla}^a\tilde{\nabla}_a\tilde{\nabla}^if_{ij}
\nonumber\\
&&
+\partial_0\tilde{\nabla}_j\tilde{\nabla}^a\tilde{\nabla}^bf_{ab}
+2k\partial_0\tilde{\nabla}^if_{ij},
\label{9.45a}
\end{eqnarray}
%


%
\begin{eqnarray}
&&2(3k+\tilde{\nabla}^c\tilde{\nabla}_c)(-3k+\tilde{\nabla}^b\tilde{\nabla}_b)(\tilde{\nabla}^a\tilde{\nabla}_a-6k)(\tilde{\nabla}^d\tilde{\nabla}_d-2k)E_{ij}
\nonumber
\\
&&=(3k+\tilde{\nabla}^c\tilde{\nabla}_c)(-3k+\tilde{\nabla}^b\tilde{\nabla}_b)(\tilde{\nabla}^a\tilde{\nabla}_a-6k)(\tilde{\nabla}^d\tilde{\nabla}_d-2k)f_{ij}
\nonumber
\\
&&+\frac{1}{2}(-3k+\tilde{\nabla}^c\tilde{\nabla}_c)(\tilde{\nabla}^b\tilde{\nabla}_b-6k)(\tilde{\nabla}^a\tilde{\nabla}_a-2k)(-2kf-\tilde{\nabla}^d\tilde{\nabla}_df+\tilde{\nabla}^m\tilde{\nabla}^nf_{mn}) \tilde{\gamma}_{ij}
\nonumber
\\
&&-2(\tilde{\nabla}^c\tilde{\nabla}_c-2k)\Bigg[\frac{1}{4}(3k+\tilde{\nabla}^b\tilde{\nabla}_b)\tilde{\nabla}_i\tilde{\nabla}_j\left(3\tilde{\nabla}^m\tilde{\nabla}^nf_{mn}-\tilde{\nabla}^a\tilde{\nabla}_af\right)
\nonumber
\\
&&-k\tilde{\gamma}_{ij}\tilde{\nabla}_d\tilde{\nabla}^d((3k+\tilde{\nabla}^e\tilde{\nabla}_e)f+\frac{3}{2}(-2kf-\tilde{\nabla}^g\tilde{\nabla}_gf+\tilde{\nabla}^m\tilde{\nabla}^nf_{mn}))
\nonumber
\\
&&-k\tilde{\gamma}_{ij}(-3k+\tilde{\nabla}^h\tilde{\nabla}_h)((3k+\tilde{\nabla}^m\tilde{\nabla}_m)f+\frac{3}{2}(-2kf-\tilde{\nabla}^n\tilde{\nabla}_nf+\tilde{\nabla}^m\tilde{\nabla}^nf_{mn}))\Bigg]
\nonumber
\\
&&-\frac{1}{3}(-3k+\tilde{\nabla}^c\tilde{\nabla}_c)(3k+\tilde{\nabla}^b\tilde{\nabla}_b)\Bigg(\tilde{\nabla}_i\tilde{\nabla}^a\tilde{\nabla}_a(3\tilde{\nabla}^df_{dj}-\tilde{\nabla}_jf)
\nonumber
\\
&&+\tilde{\nabla}_j\tilde{\nabla}^d\tilde{\nabla}_d(3\tilde{\nabla}^bf_{bi}-\tilde{\nabla}_if)-2\tilde{\nabla}_i\tilde{\nabla}_j(3\tilde{\nabla}^b\tilde{\nabla}^cf_{bc}-\tilde{\nabla}^e\tilde{\nabla}_ef)
\nonumber
\\
&&-2k\tilde{\nabla}_i(3\tilde{\nabla}^af_{aj}-\tilde{\nabla}_jf)-2k\tilde{\nabla}_j(3\tilde{\nabla}^af_{ai}-\tilde{\nabla}_if)\Bigg).
\label{9.46a}
\end{eqnarray}
%
Despite its somewhat forbidding appearance (\ref{9.46a}) is actually a derivative of 
%
\begin{align}
&2(\tilde{\nabla}^a\tilde{\nabla}_a-2k)(\tilde{\nabla}^b\tilde{\nabla}_b-3k)E_{ij}
=(\tilde{\nabla}^a\tilde{\nabla}_a-2k)(\tilde{\nabla}^b\tilde{\nabla}_b-3k)f_{ij}
\nonumber\\
&+\tfrac{1}{2}\tilde{\nabla}_i\tilde{\nabla}_j\left[\tilde{\nabla}^a\tilde{\nabla}^bf_{ab}+(\tilde{\nabla}^a\tilde{\nabla}_a+4k)f\right]-(\tilde{\nabla}^a\tilde{\nabla}_a-3k)(\tilde{\nabla}_i\tilde{\nabla}^bf_{jb}+\tilde{\nabla}_j\tilde{\nabla}^bf_{ib})
\nonumber\\
&+\tfrac{1}{2}\tilde{\gamma}_{ij}\left[(\tilde{\nabla}^a\tilde{\nabla}_a-4k)\tilde{\nabla}^b\tilde{\nabla}^cf_{bc}
-(\tilde{\nabla}_a\tilde{\nabla}^a\tilde{\nabla}_b\tilde{\nabla}^b-2k\tilde{\nabla}^a\tilde{\nabla}^a+4k^2)f\right],
\label{9.47a}
\end{align}
%
a relation that itself can be  derived from the $D=3$ version of (\ref{6.4}) with $H^2=k$ by application  of $(\tilde{\nabla}^a\tilde{\nabla}_a-2k)(\tilde{\nabla}^a\tilde{\nabla}_a-3k)$ to (\ref{6.4}). One can check the validity of these relations by inserting 
%
\begin{eqnarray}
h_{0i}&=&\Omega^2(\tau)f_{0i}=\Omega^2(\tau)[\tilde{\nabla}_iB+B_i],
\\
h_{ij}&=&\Omega^2(\tau)f_{ij}=\Omega^2(\tau)[-2\psi\tilde{\gamma}_{ij} +2\tilde{\nabla}_i\tilde{\nabla}_j E + \tilde{\nabla}_i E_j + \tilde{\nabla}_j E_i + 2E_{ij}]
\nonumber
\label{9.48a}
\end{eqnarray}
%
into them. And one can check their gauge invariance by inserting $h_{\mu\nu}\rightarrow h_{\mu\nu}-\nabla_{\mu}\epsilon_{\nu}-\nabla_{\mu}\epsilon_{\mu}$ into them. We thus establish that the metric fluctuations $\alpha$, $\gamma$, $B_i-\dot{E}_i$ and $E_{ij}$ are gauge invariant. And from  (\ref{9.13}), (\ref{9.22}), (\ref{9.23}) and (\ref{9.28}) can thus establish that the matter fluctuations $\delta \hat{\rho}$, $\hat{V}$, $V_i$ and $\delta \hat{p}$ are gauge invariant too. Interestingly, we see that in going from fluctuations around flat to fluctuations around Robertson-Walker with arbitrary $k$ and arbitrary dependence of $\Omega(\tau)$ on $\tau$ the gauge-invariant metric fluctuation combinations $\alpha$, $\gamma$, $B_i-\dot{E}_i$ and $E_{ij}$  remain the same, though $\gamma$ does depend generically on $\Omega(\tau)$. 


\subsubsection{Background Solution}
\label{sss:solving_the_background_svt3}

In order to actually solve the fluctuation equations we will need to determine the appropriate background $\Omega(\tau)$, and we will also need to deal with the fact that, as noted above,  the fluctuation equations contain more degrees of freedom (11) than there are evolution equations (10). For the background first we note that no matter what the value of $k$, from (\ref{9.10}) we see that if $\rho=3p$ then $\rho=3/\Omega^4$,  as written in a convenient normalization (one which differs from the one used in Sec. \ref{S8}), while if $p=0$ we have $\rho=3/\Omega^3$. Once we specify a background equation of state that relates $\rho$ and $p$ we can solve for $\Omega (\tau)$ and $t=\int \Omega(\tau)d\tau$. We thus obtain 
%
\begin{eqnarray}
p=\rho/3,~k=0:&&\quad \Omega=\tau,\quad p=1/\tau^4,\quad \rho=3/\tau^4,\quad t=\tau^2/2,
\nonumber\\
&&\qquad a(t)=\Omega(\tau)=(2t)^{1/2},
\nonumber\\
p=\rho/3,~k=-1:&&\quad \Omega=\sinh\tau,\quad p=1/\sinh^4\tau,\quad \rho=3/\sinh^4\tau,
\nonumber\\
&& \qquad t=\cosh\tau,\quad a(t)=\Omega(\tau)=(t^2-1)^{1/2},
\nonumber\\
p=\rho/3,~k=+1:&&\quad \Omega=\sin\tau,\quad p=1/\sin^4\tau,\quad \rho=3/\sin^4\tau,\quad t=-\cos\tau,
\nonumber\\
&& \qquad a(t)=\Omega(\tau)=(1-t^2)^{1/2},
\nonumber\\
p=0,~k=0:&&\quad \Omega=\tau^2/4,\quad p=0,\quad \rho=192/\tau^6,\quad t=\tau^{3}/12,
\nonumber\\
&&\qquad a(t)=\Omega(\tau)=(3t/2)^{2/3},
\nonumber\\
p=0,~k=-1:&&\quad \Omega=\sinh^2(\tau/2),\quad p=0,\quad \rho=3/\sinh^6(\tau/2),
\nonumber\\
&& \qquad t=\tfrac{1}{2}[\sinh\tau-\tau],\quad a(t)=\Omega(\tau),
\nonumber\\
p=0,~k=1:&&\quad \Omega=\sin^2(\tau/2),\quad p=0,\quad \rho=3/\sin^6(\tau/2),
\nonumber\\
&& \qquad t=\tfrac{1}{2}[\tau-\sin\tau],\quad a(t)=\Omega(\tau).
\label{9.49}
\end{eqnarray}
%
For $p=0$ and $k=\pm 1$ we cannot obtain $a(t)$ in a closed form. Consequently one ordinarily only determines $a(t)$ in parametric form. As we see, the conformal time $\tau$ can serve as the appropriate parameter.

\subsubsection{Relation between $\delta\rho$ and $\delta p$}
\label{sss:relating_dp_drho_svt3}

To reduce the number of fluctuation variables from 11 to 10 we follow kinetic theory, and first consider a relativistic flat spacetime ideal $N$ particle classical gas of spinless particles each of mass $m$ in a volume $V$ at a temperature $T$. As discussed for instance in \cite{mannheim_2006}, for this
system one can use a basis of momentum eigenmodes, with the Helmholtz free energy $A(V,T)$ being given as 
%                                                                               
\begin{equation}
e^{-A(V,T)/NkT}=V\int
d^3pe^{-(p^2+m^2)^{1/2}/kT}, 
\label{9.50}
\end{equation}                                 
%  
so that the pressure takes the simple form 
%                                                                               
\begin{equation}
p=-\left(\frac{\partial A}{ \partial
	V}\right)_T=\frac{NkT}{V},
\label{9.51}
\end{equation}                                 
% 
while the internal energy $U=\rho V$ evaluates in terms of Bessel
functions as  
%                                                                               
\begin{equation}
U=A-T\left(\frac{\partial A}{ \partial
	T}\right)_V=3NkT+Nm\frac{K_1(m/kT)}{K_2(m/kT)}.
\label{9.52}
\end{equation}                                 
% 
In the high and low temperature limits (the radiation and matter eras)
we then find that the expression for $U$ simplifies to
%                                                                               
\begin{eqnarray}
\rho=\frac{U}{V}\rightarrow
\frac{3NkT}{V}=3p,&&\quad \frac{m}{kT}\rightarrow 0,
\nonumber \\
\rho=\frac{U}{V} \rightarrow
\frac{Nm}{V}+\frac{3NkT}{2V}=\frac{Nm}{V}+\frac{3p}{2} \approx
\frac{Nm}{V},&&\quad \frac{m}{kT}
\rightarrow \infty.
\label{9.53}
\end{eqnarray}                                 
% 
Consequently, while $p$ and $\rho$ are nicely proportional to each other ($p=w\rho$)
in the high temperature radiation and the low temperature matter eras
(where $w(T\rightarrow\infty)=1/3$ and $w(T\rightarrow 0)=0$), we also
see that in transition region between the two eras their relationship is
altogether more complicated. Since such a transition era occurs fairly close to recombination, it is this complicated relation that should be used there. Use of a $p=w\rho$ equation of state would at best only be valid at temperatures which are very different from
those of order $m/K$, though for massless particles it would be of
course be valid to use $p=\rho/3$ at all temperatures. 

As derived, these expressions only hold in a flat  Minkowski spacetime. However, $A(V,T)$ only involves  an integration over the spatial 3-momentum. Thus for the spatially flat $k=0$ Robertson-Walker metric $ds^2=dt^2-a^2(t)(dr^2+r^2d\theta^2+r^2\sin^2\theta d\phi^2)$, all of these kinetic theory relations will continue to hold with $T$ taken to depend on the comoving time $t$. (Typically $T\sim 1/a(t)$.) Suppose we now perturb the system and obtain a perturbed $\delta T$ that now depends on both $t$ and $r,\theta,\phi$. In the radiation era where $p=\rho/3$ we would obtain $\delta p =\delta \rho/3$ (and thus $3\delta\hat{p}=\delta\hat{\rho}$). In the matter era where $p=0$, from (\ref{9.53}) we would obtain $\delta p=2\delta \rho/3$. In the intermediate region the relation would be much more complicated. Nonetheless in all cases we would have reduced the number of independent fluctuation variables, though we note that not just in the radiation and matter eras but even in the intermediate region, it is standard in cosmological perturbation theory to use $\delta p/\delta \rho=v^2$ where $v^2$ is taken to be a spacetime-independent constant.

While we can use the above $A(V,T)$ for spatially flat cosmologies with $k=0$, for spatially curved cosmologies with non-zero $k$ we cannot use a mode basis made out of 3-momentum eigenstates at all. One has to adapt the basis to a curved 3-space by replacing
$(p^2+m^2)^{1/2}/kT$ by $(dx^{\mu}/d\tau)U^{\nu}g_{\mu\nu}/kT$ (see e.g. \cite{mannheim_2006}), while replacing
$\int d^3p$ by a sum over a complete set of basis modes associated with
the propagation of a spinless massive particle in the chosen
$g_{\mu\nu}$ background, and then follow the steps above to see what
generalization of (\ref{9.53}) might then ensue. While tractable in principle it is not straightforward to do this in practice, and we will not do it here. While one would need to do this in order to obtain a $k\neq 0$ generalization of the $k=0$  $\delta p/\delta \rho=v^2$ relation, and while such a generalization would be needed in order to solve the $k\neq 0$ fluctuation equations completely, since our purpose here is only to test for the validity of the decomposition theorem, we will not actually need to find a relation between $\delta p$ and $\delta \rho$, since as we now see, we will be able to test for the validity of the decomposition theorem without actually needing to know the specific form of such a relation at all, or even needing to specify any particular form for the background $\Omega(\tau)$ either for that matter.
%%%%%%%%%%%%%%%%%%%%%%%%%%%%%%%%%%%%%%%%%%%%
\subsection{Robertson Walker $k=-1$}
\label{ss:rw_k=-1_svt3}
%%%%%%%%%%%%%%%%%%%%%%%%%%%%%%%%%%%%%%%%%%%%

\subsubsection{The Scalar Sector}
\label{sss:scalar_sector}

We have seen that the scalar sector evolution equations (\ref{9.22}), (\ref{9.26}), (\ref{9.27}) and (\ref{9.28}) involve derivatives of the form $\tilde{\nabla}^2$, $\tilde{\nabla}^2+3k$ where the coefficient of $k$ is either zero or positive, while the vector and tensor sectors equations (\ref{9.23}), (\ref{9.31}) and (\ref{9.40}) also involve derivatives such as $\tilde{\nabla}^2-2k$, $\tilde{\nabla}^2-3k$, and $\tilde{\nabla}^2-6k$ in which the coefficient of $k$ is negative. As the implications of boundary conditions are very sensitive to the sign of the coefficient of $k$, and we will need to monitor both positive and negative coefficient cases below. In implementing evolution equations that involve products of derivative operators such as the generic $(\tilde{\nabla}^2+\alpha)(\tilde{\nabla}^2+\beta)F=0$ ($F$ denotes scalar, vector or tensor), we can satisfy these relations by $(\tilde{\nabla}^2+\alpha)F=0$,  by $(\tilde{\nabla}^2+\beta)F=0$, or by $F=0$. The decomposition theorem will only follow if boundary conditions prevent us from satisfying $(\tilde{\nabla}^2+\alpha)F=0$ or   $(\tilde{\nabla}^2+\beta)F=0$ with non-zero $F$, leaving $F=0$ as the only remaining possibility. It is the purpose of this section to explore whether or not boundary conditions do force us to $F=0$ in any of the scalar, vector or tensor sectors. While a decomposition theorem would immediately hold if they do, as we will show in Sec. \ref{ss:recovering_decomposition_theorem} the interplay of the vector and tensor sectors in the $\Delta_{ij}=0$ relation given in (\ref{9.15}) will still force us to a decomposition theorem even if they do not.


To illustrate what is involved it is sufficient to restrict $k$ to $k=-1$, and to take the metric to be of the form 
%
\begin{eqnarray}
ds^2=\Omega^2(\tau)\left[ d\tau^2-d\chi^2-\sinh^2\chi d\theta^2-\sinh^2\chi\sin^2\theta d\phi^2\right],
\label{10.1b}
\end{eqnarray}
%
where $r=\sinh \chi$. Since the analysis leading to the structure for $\Delta_{\mu\nu}$ given in (\ref{9.13}) to (\ref{9.17}) is completely covariant these equations equally hold if we represent the spatial sector of the metric as given in (\ref{10.1b}). With $k=-1$ the scalar sector evolution equations involving the $\tilde{\nabla}^2$ and $\tilde{\nabla}^2-3$ operators take the form
%
\begin{eqnarray}
(\tilde{\nabla}_a\tilde{\nabla}^a+A_S)S=0.
\label{10.2b}
\end{eqnarray}
%
(Here $S$ is to denote the full combinations of scalar sector  components that appear in  (\ref{9.22}), (\ref{9.26}), (\ref{9.27}) and (\ref{9.28}).)  In (\ref{10.2b}) we have introduced a generic scalar sector constant $A_S$, whose values in  (\ref{9.22}), (\ref{9.26}), (\ref{9.27}) and (\ref{9.28})  are $(0,-3)$. On setting $S(\chi,\theta,\phi)=S_{\ell}(\chi)Y^m_{\ell}(\theta,\phi)$ (\ref{10.2b}) reduces to 
%  
\begin{eqnarray}
\left[\frac{d^2}{d\chi^2}+2\frac{\cosh\chi }{\sinh\chi}\frac{d }{ d\chi}
-\frac{\ell(\ell+1)}{ \sinh^2\chi}+A_S\right]S_{\ell}=0.
\label{10.3b}
\end{eqnarray}
%
In the  $\chi\rightarrow \infty$ and $\chi\rightarrow 0$ limits  we take the solution to behave as $e^{\lambda \chi}$ (times an irrelevant polynomial in $\chi$), and as $\chi^n$, to thus obtain
%
\begin{eqnarray}
&&\lambda^2+2\lambda+A_S=0,\quad \lambda=-1\pm(1-A_S)^{1/2},
\nonumber\\
&&\lambda(A_S=0)=(-2,~0),\quad \lambda(A_S=-3)=(-3,~1),
\nonumber\\
&&n(n-1)+2n-\ell(\ell+1)=0,\quad n=\ell,-\ell-1.
\label{10.4b}
\end{eqnarray}
%
For each of $A_S=0$ and  $A_S=-3$ one solution converges at $\chi=\infty$ and the other diverges at $\chi=\infty$. Thus we need to see how they match up with the solutions at $\chi=0$, where one solution is well-behaved and the other is not. 

So to this end we look for exact solutions to (\ref{10.3b}). Thus,  as discussed for instance in \cite{bander_itzykson_1966,mannheim_kazanas_1988}, and as appropriately generalized here,  (\ref{10.3b}) 
admits of solutions (known as associated Legendre functions) of the form 
%
\begin{eqnarray}
S_{\ell}=\sinh^{\ell}\chi\left(\frac{1}{ \sinh\chi} \frac{d }{ d\chi}\right)^{\ell+1}f(\chi),
\label{10.5b}
\end{eqnarray}
%
where $f(\chi)$ obeys
%
\begin{eqnarray}
\left[\frac{d^3}{d\chi^3}+\nu^2\frac{d}{d\chi}\right]f(\chi)=0,\quad \nu^2=A_S-1,
\label{10.6b}
\end{eqnarray}
%
with $f(\chi)$ thus obeying 
%
\begin{eqnarray}
f(\nu^2>0)&=&\cos\nu\chi,~\sin\nu\chi,\quad f(\nu^2=-\mu^2<0)=\cosh\mu\chi,~\sinh\mu\chi,
\nonumber\\
f(\nu^2=0)&=&\chi,~\chi^2.
\label{10.7b}
\end{eqnarray}
%
For each $f(\chi)$ this would lead to solutions of the form
%
\begin{eqnarray}
\hat{S}_0&=&\frac{1}{\sinh\chi}\frac{df}{d\chi},\quad \hat{S}_1=\frac{d\hat{S}_0}{d\chi},
\nonumber\\ \hat{S}_2&=&\sinh\chi\frac{d}{d\chi}\left[\frac{\hat{S}_1}{\sinh\chi}\right],\quad \hat{S}_3=\sinh^2\chi\frac{d}{d\chi}\left[\frac{\hat{S}_2}{\sinh^2\chi}\right],....
\label{10.8b}
\end{eqnarray}
%
However, on evaluating these expressions it can happen that some of these solutions vanish. Thus for $A_S=0$ for instance where $f(\chi)=(\sinh\chi,\cosh\chi)$ the two solutions with $\ell=0$ are $\cosh\chi/\sinh\chi$ and $1$. However this would lead to the two solutions with $\ell=1$ being $1/\sinh^2\chi$ and $0$. To address this point we note that suppose we have obtained some non-zero solution $\hat{S}_{\ell}$. Then, a second solution of the form $\hat{f}_{\ell}(\chi)\hat{S}_{\ell}(\chi)$ may be found by inserting $\hat{f}_{\ell}(\chi)\hat{S}_{\ell}(\chi)$ into (\ref{10.3b}), to yield
%
\begin{eqnarray}
\hat{S}_{\ell}\frac{d^2 \hat{f}_{\ell}}{ d\chi^2}+2\hat{S}_{\ell}\frac{\cosh\chi }{ \sinh\chi}\frac{d \hat{f}_{\ell}}{ d\chi}+2\frac{d \hat{S}_{\ell}}{ d\chi}\frac{d \hat{f}_{\ell}}{ d\chi}=0,
\label{10.9b}
\end{eqnarray}
%
which integrates to
%
\begin{eqnarray}
\frac{d \hat{f}_{\ell}}{ d\chi}=\frac{1}{\sinh^2\chi\hat{S}_{\ell}^2},~~~~~\hat{f}_{\ell}\hat{S}_{\ell}=\hat{S}_{\ell}\int \frac{d\chi }{\sinh^2\chi\hat{S}_{\ell}^2}.
\label{10.10b}
\end{eqnarray}
%
Thus for $\ell=1$, from the non-trivial $A_S=0$ solution $\hat{S}_{1}=1/\sinh^2\chi$ we obtain a second solution of the form $\hat{f}_{\ell}\hat{S}_{\ell}=\cosh\chi/\sinh\chi-\chi/\sinh^2\chi$. However, once we have this second solution we can then return to (\ref{10.8b}) and use it to obtain the subsequent solutions associated with higher $\ell$ values, since use of the chain in (\ref{10.8b}) only requires that at any point the elements in it are solutions regardless of how they may or may not have been found.

Having the form given in (\ref{10.10b}) is  useful for another purpose, as it allows us to relate the behaviors of the solutions in the $\chi\rightarrow \infty$ and $\chi\rightarrow 0$ limits. Thus suppose that $\hat{S}_{\ell}$ behaves as $e^{\lambda\chi}$ and as $\chi^{\ell}$ in these two limits. Then $\hat{f}_{\ell}\hat{S}_{\ell}$ must behave as $e^{-(\lambda+2)\chi}$ and $\chi^{-\ell-1}$ in the two limits. Alternatively, if $\hat{S}_{\ell}$ behaves as $e^{\lambda\chi}$ and as $\chi^{-\ell-1}$ in these two limits, then $\hat{f}_{\ell}\hat{S}_{\ell}$ must behave as $e^{-(\lambda+2)\chi}$ and $\chi^{\ell}$ in the two limits. Comparing with (\ref{10.4b}), we note that if we set $\lambda=-1\pm(1-A_S)^{1/2}$ then consistently we find that $-(\lambda+2)=-1\mp(1-A_S)^{1/2}$. However, this analysis shows that we cannot directly identify which  $\chi\rightarrow \infty$ behavior is associated with which $\chi\rightarrow 0$ behavior (the insertion of either $\chi^{\ell}$ or $\chi^{-\ell -1}$ into (\ref{10.10b}) generates the other, with both behaviors thus being required in any $\hat{S}_{\ell}$, $\hat{f}_{\ell}\hat{S}_{\ell}$ pair), and to determine which is which we thus need to construct the asymptotic solutions directly.

For $A_S=0$, $\nu=i$,  the relevant $f(\nu^2)$ given in (\ref{10.7b}) are $\cosh \chi$ and $\sinh \chi$. 
Consequently, we find the first few $S^{(i)}_{\ell}$, $i=1,2$ solutions to $\tilde{\nabla}_a\tilde{\nabla}^aS=0$ to be of the form 
%
\begin{align}
&\hat{S}^{(1)}_{0}(A_S=0)=\frac{\cosh\chi}{\sinh\chi},\quad \hat{S}^{(2)}_{0}(A_S=0)=1,
\nonumber\\
&\hat{S}^{(1)}_{1}(A_S=0)=\frac{1}{\sinh^2\chi},\quad \hat{S}^{(2)}_{1}(A_S=0)=\frac{\cosh\chi}{\sinh\chi}-\frac{\chi}{\sinh^2\chi},
\nonumber\\
&\hat{S}^{(1)}_{2}(A_S=0)=\frac{\cosh\chi}{\sinh^3\chi},\quad \hat{S}^{(2)}_{2}(A_S=0)=1+\frac{3}{\sinh^2\chi}-\frac{3\chi\cosh\chi}{\sinh^3\chi},
\nonumber\\
&\hat{S}^{(1)}_{3}(A_S=0)=\frac{4}{\sinh^2\chi}+\frac{5}{\sinh^4\chi},
\nonumber\\
& \hat{S}^{(2)}_{3}(A_S=0)=
\frac{2\cosh\chi}{\sinh\chi}+\frac{15\cosh\chi}{\sinh^3\chi}-\frac{12\chi}{\sinh^2\chi}-\frac{15\chi}{\sinh^4\chi}.
\label{10.11b}
\end{align}
%
From this pattern we see that the solutions that are bounded at $\chi=\infty$ are badly-behaved at $\chi=0$, while the solutions that are  well-behaved at $\chi=0$ are unbounded at $\chi=\infty$. Thus all of these $A_S=0$ solutions are excluded by a requirement that solutions be  bounded at $\chi=\infty$ and be well-behaved at $\chi=0$.

For $A_S=-3$, $\nu=2i$, the relevant $f(\nu^2)$ given in (\ref{10.7b}) are $\cosh 2\chi$ and $\sinh 2\chi$. Consequently, the first few solutions  to $(\tilde{\nabla}_a\tilde{\nabla}^a-3)S=0$ are of the form
%
\begin{eqnarray}
&&\hat{S}^{(1)}_0(A_S=-3)=\cosh\chi,\quad \hat{S}^{(2)}_0(A_S=-3)=2\sinh\chi+\frac{1}{\sinh\chi},
\nonumber\\
&&\hat{S}^{(1)}_1(A_S=-3)=\sinh\chi,\quad \hat{S}^{(2)}_1(A_S=-3)=2\cosh\chi-\frac{\cosh\chi}{\sinh^2\chi},
\nonumber\\
&&\hat{S}^{(1)}_2(A_S=-3)=2\cosh\chi-\frac{3\cosh\chi}{\sinh^2\chi}+\frac{3\chi}{\sinh^3\chi},\quad \hat{S}^{(2)}_2(A_S=-3)=\frac{1}{\sinh^3\chi},
\nonumber\\
&&\hat{S}^{(1)}_3(A_S=-3)=2\sinh\chi-\frac{5}{\sinh\chi}-\frac{15}{\sinh^3\chi}+\frac{15\chi\cosh\chi}{\sinh^4\chi},
\nonumber\\
&& \hat{S}^{(2)}_3(A_S=-3)=\frac{\cosh\chi}{\sinh^4\chi}.
\label{10.12b}
\end{eqnarray}
%
From this pattern we again see that the solutions that are bounded at $\chi=\infty$ are badly-behaved at $\chi=0$, while the solutions that are  well-behaved at $\chi=0$ are unbounded at $\chi=\infty$. Thus all of these $A_S=-3$ solutions are also excluded by a requirement that solutions be  bounded at $\chi=\infty$ and be well-behaved at $\chi=0$.

With all of these $A_S=0$, $A_S=-3$ solutions being excluded we must realize (\ref{9.22}), (\ref{9.26}), (\ref{9.27}) and (\ref{9.28}) by 
%
\begin{eqnarray}
-2 \dot{\Omega} \Omega^{-1} (\alpha - \dot\gamma) + 2 k \gamma 
+(-4 \dot{\Omega}^2 \Omega^{-3}  + 2 \overset{..}{\Omega} \Omega^{-2}  - 2 k \Omega^{-1}) \hat{V}=0,
\label{10.13b}
\end{eqnarray}
%
%
\begin{eqnarray}
&&2 \dot{\Omega}^2 \Omega^{-2}(\alpha-\dot\gamma)
-2  \dot{\Omega} \Omega^{-1}(\dot\alpha -\ddot\gamma)-4\ddot\Omega\Omega^{-1}(\alpha-\dot\gamma)+ \Omega^2 \delta \hat{p}
+ 2 k(\alpha + 2 \dot{\Omega}  \Omega^{-1} \gamma)]
\nonumber\\
&&=0,
\label{10.14b}
\end{eqnarray}
%
%
\begin{eqnarray}
\alpha+2\dot{\Omega}\Omega^{-1}\gamma=0,
\label{10.15b}
\end{eqnarray}
%
%
\begin{eqnarray}
2 \dot{\Omega}^2 \Omega^{-2}(\alpha-\dot\gamma)
-2  \dot{\Omega} \Omega^{-1}(\dot\alpha -\ddot\gamma)-4\ddot\Omega\Omega^{-1}(\alpha-\dot\gamma)+ \Omega^2 \delta \hat{p}=0.
\label{10.16b}
\end{eqnarray}
%
%
These equations are augmented by (\ref{9.13}), (\ref{9.16}) and (\ref{9.17})
%
\begin{eqnarray}
6 \dot{\Omega}^2 \Omega^{-2}(\alpha-\dot\gamma) + \delta \hat{\rho} \Omega^2 + 2 \dot{\Omega} \Omega^{-1} \tilde{\nabla}_{a}\tilde{\nabla}^{a}\gamma=0, 
\label{10.17b}
\end{eqnarray}
%
%
\begin{eqnarray}
&&6 \dot{\Omega}^2 \Omega^{-2}(\alpha-\dot\gamma)
-6  \dot{\Omega} \Omega^{-1}(\dot\alpha -\ddot\gamma)-12\ddot\Omega\Omega^{-1}(\alpha-\dot\gamma)+ 3\Omega^2 \delta \hat{p}
\nonumber\\
&&-2\tilde\nabla_a\tilde\nabla^a(\alpha + 2\dot\Omega \Omega^{-1}\gamma)=0,
\label{10.18b}
\end{eqnarray}
%
%
\begin{eqnarray}
&&3 \delta \hat{p}-  \delta \hat{\rho}
-12 \overset{..}{\Omega}  \Omega^{-3}(\alpha - \dot\gamma) -6 \dot{\Omega} \Omega^{-3}(\dot{\alpha} -\ddot\gamma)
\nonumber\\
&&-2 \Omega^{-2} \tilde{\nabla}_{a}\tilde{\nabla}^{a}(\alpha +3\dot\Omega\Omega^{-1}\gamma)=0.
\label{10.19b}
\end{eqnarray}
%
On taking the $\tilde{\nabla}_i$ derivative of (\ref{10.13b}), we recognize (\ref{10.13b}) to (\ref{10.19b})  as precisely being the scalar sector ones given in (\ref{9.41}) and (\ref{9.42}). We thus establish the decomposition theorem in the scalar sector. 

\subsubsection{The Vector Sector}
\label{sss:vector_sector}

To determine the structure of $k=-1$ solutions to the vector sector  (\ref{9.23}) and (\ref{9.31}), we first need to evaluate the quantity $\tilde{\nabla}_a\tilde{\nabla}^aV^i$, where $V^i$ obeys the transverse condition 
%
\begin{eqnarray}
\tilde\nabla_a V^a&=& \frac{V_{2} \cos\theta}{\sin\theta \sinh^2\chi} + \frac{2 V_{1} \cosh\chi}{\sinh\chi} + \partial_{1}V_{1} + \frac{\partial_{2}V_{2}}{\sinh^2\chi} + \frac{\partial_{3}V_{3}}{\sin^2\theta \sinh^2\chi}
\nonumber\\
&=&0.
\label{10.20b}
\end{eqnarray}
%
On implementing this condition, the $(\chi,\theta,\phi) \equiv (1,2,3)$ components of $\tilde{\nabla}_a\tilde{\nabla}^aV^i$ take the form
% 
\begin{eqnarray}
\tilde{\nabla}_a\tilde{\nabla}^aV^1&=&V_{1} \left(2 + \frac{2}{\sinh^2\chi}\right) + \frac{4 \cosh\chi \partial_{1}V_{1}}{\sinh\chi} + \partial_{1}\partial_{1}V_{1} + \frac{\cos\theta \partial_{2}V_{1}}{\sin\theta \sinh^2\chi} + \frac{\partial_{2}\partial_{2}V_{1}}{\sinh^2\chi}
\nonumber\\
&& + \frac{\partial_{3}\partial_{3}V_{1}}{\sin^2\theta \sinh^2\chi},
\nonumber\\ 
\tilde{\nabla}_a\tilde{\nabla}^aV^2&=& V_{2} \left(- \frac{2}{\sinh^4\chi} + \frac{1}{\sin^2\theta \sinh^4\chi} -  \frac{2}{\sinh^2\chi}\right) + \frac{4 V_{1} \cos\theta \cosh\chi}{\sin\theta \sinh^3\chi}
\nonumber\\
&& + \frac{2 \cos\theta \partial_{1}V_{1}}{\sin\theta \sinh^2\chi} + \frac{\partial_{1}\partial_{1}V_{2}}{\sinh^2\chi}  + \frac{2 \cosh\chi \partial_{2}V_{1}}{\sinh^3\chi} + \frac{3 \cos\theta \partial_{2}V_{2}}{\sin\theta \sinh^4\chi}
\nonumber\\
&& + \frac{\partial_{2}\partial_{2}V_{2}}{\sinh^4\chi} + \frac{\partial_{3}\partial_{3}V_{2}}{\sin^2\theta \sinh^4\chi},
\nonumber\\ 
\tilde{\nabla}_a\tilde{\nabla}^aV^3&=& - \frac{2 V_{3}}{\sin^2\theta \sinh^2\chi} + \frac{\partial_{1}\partial_{1}V_{3}}{\sin^2\theta \sinh^2\chi} -  \frac{\cos\theta \partial_{2}V_{3}}{\sin^3\theta \sinh^4\chi} + \frac{\partial_{2}\partial_{2}V_{3}}{\sin^2\theta \sinh^4\chi}
\nonumber\\
&& + \frac{2 \cosh\chi \partial_{3}V_{1}}{\sin^2\theta \sinh^3\chi} + \frac{2 \cos\theta \partial_{3}V_{2}}{\sin^3\theta \sinh^4\chi} + \frac{\partial_{3}\partial_{3}V_{3}}{\sin^4\theta \sinh^4\chi}.
\label{10.21b}
\end{eqnarray}
% 
To explore the structure of the $k=-1$ vector sector we seek solutions to
%
\begin{eqnarray}
(\tilde{\nabla}_a\tilde{\nabla}^a+A_V)V_i=0.
\label{10.22b}
\end{eqnarray}
%
(Here $V_i$ is to denote the full combinations of vector components that appear in (\ref{9.23}) and (\ref{9.31}).) In (\ref{10.22b}) we have introduced a generic vector sector constant $A_V$, whose values in (\ref{9.23}) and (\ref{9.31})  are $(2,-1,-2)$.

Conveniently, we find that the equation for $V_1$ involves no mixing with $V_2$ or $V_3$, and can thus be solved directly. On setting $V_1(\chi,\theta,\phi)=g_{1,\ell}(\chi)Y_{\ell}^m(\theta,\phi)$, the equation for $V_1$ reduces to 
%
\begin{eqnarray}
\left[\frac{d^2}{d\chi^2}+4\frac{\cosh\chi}{ \sinh\chi}\frac{d }{d\chi}
+2+A_V+\frac{2 }{ \sinh^2\chi}-\frac{\ell(\ell+1)}{ \sinh^2\chi}\right]g_{1,\ell}=0.
\label{10.23b}
\end{eqnarray}
%
To check the $\chi \rightarrow \infty$ and $\chi \rightarrow 0$ limits, we take the solutions to behave as $e^{\lambda\chi}$ (times an irrelevant polynomial in $\chi$) and $\chi^n$ in these two limits. For (\ref{10.23b}) the limits give
%
\begin{eqnarray}
&&\lambda^2+4\lambda+2+A_V=0,\quad\lambda=-2\pm (2-A_V)^{!/2},\quad
\lambda(A_V=2)=(-2,~-2),
\nonumber\\
&& \lambda(A_V=-1)=-2\pm \surd{3},\quad \lambda(A_V=-2)=(0,-4),
\nonumber\\
&&
n(n-1)+4n+2-\ell(\ell+1)=0,\quad n=\ell-1, -\ell-2.
\label{10.24b}
\end{eqnarray}
%
Thus for $A_V=2$ and $A_V=-1$ both solutions are bounded at infinity, while for $A_V=-2$ one solution is bounded at infinity. Moreover, for each value of $A_V$ one of the solutions will  be well-behaved as $\chi\rightarrow 0$ for any $\ell\geq 1$ while the other solution will not be.  Thus for $A_V=2$ there will always be one $\ell\geq 1$ solution that is bounded at $\chi=\infty$ and well-behaved at $\chi=0$. To determine whether we can obtain a solution that is bounded in both limits for $A_V=-1$, $A_V=-2$ we need to explicitly find the solutions in closed form. 


To this end we need to put (\ref{10.23b})  into the form of a differential equation whose solutions are known. We thus set $g_{1,\ell}=\alpha_{\ell}/\sinh\chi$, to find that (\ref{10.23b}) takes the form
%
\begin{eqnarray}
\left[\frac{d^2 }{d\chi^2}+2\frac{\cosh\chi}{ \sinh\chi}\frac{d }{d\chi}
-\frac{\ell(\ell+1) }{\sinh^2\chi}+A_V-1\right]\alpha_{\ell}=0.
\label{10.25b}
\end{eqnarray}
%
We recognize (\ref{10.25b}) as being in the form given in  (\ref{10.3b}), which we discussed above, with $\nu^2=A_V-2$.


Thus for $A_V=2$, viz. $\nu=0$  in (\ref{10.7b}) and $f(\nu^2=0)=\chi,~\chi^2$,  we find $V^{(i)}_{\ell}$, $i=1,2$ solutions to $(\tilde{\nabla}_a\tilde{\nabla}^a+2)V_1=0$ of the form 
%
\begin{eqnarray}
\hat{V}^{(1)}_0(A_V=2)&=&\frac{1}{ \sinh^2\chi},\quad \hat{V}^{(2)}_0(A_V=2)=\frac{\chi }{ \sinh^2\chi},
\nonumber\\
\hat{V}^{(1)}_1(A_V=2)&=&\frac{\cosh \chi }{ \sinh^3\chi},\quad \hat{V}^{(2)}_1(A_V=2)=\frac{1}{ \sinh^2\chi}-\frac{\chi\cosh\chi}{ \sinh^3\chi},
\nonumber\\
\hat{V}^{(1)}_2(A_V=2)&=&\frac{2}{ \sinh^2\chi}+\frac{3}{\sinh^4\chi},
\nonumber\\
\hat{V}^{(2)}_2(A_V=2)&=&\frac{3\cosh\chi}{\sinh^3\chi}-\frac{2\chi}{\sinh^2\chi}-\frac{3\chi }{\sinh^4\chi},
\nonumber\\
\hat{V}^{(1)}_3(A_V=2)&=&\frac{2\cosh\chi}{\sinh^3\chi}+\frac{5\cosh\chi}{\sinh^5\chi},
\nonumber\\
 \hat{V}^{(2)}_3(A_V=2)&=&\frac{11}{\sinh^2\chi}+\frac{15}{\sinh^4\chi}-\frac{6\chi\cosh\chi}{\sinh^3\chi}-\frac{15\chi\cosh\chi }{\sinh^5\chi}.~~~
\label{10.26b}
\end{eqnarray}
%
The just as required by (\ref{10.24b}), the $\hat{V}^{(2)}_{\ell}(A_V=2)$ solutions with $\ell \geq1$ are bounded at  $\chi=\infty$ and well-behaved at $\chi=0$. Since they can thus not be excluded by boundary conditions at $\chi=\infty$ and $\chi=0$ (though boundary conditions do exclude modes with $\ell=0$), solutions to (\ref{9.23}) and (\ref{9.31}) do not become the vector sector solutions associated with (\ref{9.41}). Thus if we implement (\ref{9.31}) by $(\tilde{\nabla}_a\tilde{\nabla}^a+2)V_i=0$, the  decomposition theorem will fail in the vector sector for modes with $\ell \geq 1$. Thus an equation such as (\ref{9.31}) 
will be solved by 
%
\begin{eqnarray}
(\tilde{\nabla}_a\tilde{\nabla}^a-1)(\tilde{\nabla}_b\tilde{\nabla}^b-2)\left[\tfrac{1}{2}(\dot{B}_i-\ddot{E}_i)+\dot{\Omega}\Omega^{-1}(B_i-\dot{E}_i)\right]=V_i,
\label{10.27b}
\end{eqnarray}
%
and not by
%
\begin{eqnarray}
\tfrac{1}{2}(\dot{B}_i-\ddot{E}_i)+\dot{\Omega}\Omega^{-1}(B_i-\dot{E}_i)=0.
\label{10.28b}
\end{eqnarray}
%
Thus (\ref{9.31}) is solved by the $\chi$ dependence of $B_i-\dot{E}_i$ and not by its $\tau$ dependence, i.e., not by the $B_i-\dot{E_i}=1/\Omega^2$ dependence on $\tau$ that one would have obtained from the decomposition-theorem-required (\ref{10.28b}). This then raises the question of what does fix the $\tau$ dependence in the vector sector. We will address this issue below.

For $A_V=-2$ we see that $\nu^2=-4$ and that $f(\nu^2)=\cosh 2\chi,\sinh 2\chi$. However in the scalar case discussed above where $\nu^2=A_S-1$, $\nu^2$ would also obey $\nu^2=-4$ if $A_S=-3$. Thus for $A_V=-2$ we can obtain the solutions to $(\tilde{\nabla}_a\tilde{\nabla}^a-2)V_1=0$ directly from (\ref{10.12b}), and after implementing $g_{1,\ell}=\alpha_{\ell}/\sinh\chi$ we  obtain 
%
\begin{eqnarray}
&&\hat{V}^{(1)}_0(A_V=-2)=\frac{\cosh\chi}{\sinh\chi},\quad \hat{V}^{(2)}_0(A_V=-2)=2+\frac{1}{\sinh^2\chi},
\nonumber\\
&&\hat{V}^{(1)}_1(A_V=-2)=1,\quad \hat{V}^{(2)}_1(A_V=-2)=2\frac{\cosh\chi}{\sinh\chi}-\frac{\cosh\chi}{\sinh^3\chi},
\nonumber\\
&&\hat{V}^{(1)}_2(A_V=-2)=2\frac{\cosh\chi}{\sinh\chi}-\frac{3\cosh\chi}{\sinh^3\chi}+\frac{3\chi}{\sinh^4\chi},\quad \hat{V}^{(2)}_2(A_V=-2)=\frac{1}{\sinh^4\chi},
\nonumber\\
&&\hat{V}^{(1)}_3(A_V=-2)=2-\frac{5}{\sinh^2\chi}-\frac{15}{\sinh^4\chi}+\frac{15\chi\cosh\chi}{\sinh^5\chi},
\nonumber\\
&& \hat{V}^{(2)}_3(A_V=-2)=\frac{\cosh\chi}{\sinh^5\chi}.
\label{10.29b}
\end{eqnarray}
%
As required by (\ref{10.24b}), the $\hat{V}^{(2)}_2(A_V=-2)$ and $\hat{V}^{(2)}_3(A_V=-2)$ solutions  are bounded at  $\chi=\infty$. However, they are not well-behaved at $\chi=0$. Since they thus can  be excluded by boundary conditions at $\chi=\infty$ and $\chi=0$, if we implement (\ref{9.31}) by $(\tilde{\nabla}_a\tilde{\nabla}^a-2)V_i=0$,  the only allowed solution will be $V_i=0$, and the decomposition theorem will then follow.

Finally, for $A_V=-1$, viz. $\nu=i\surd{3}$, $f(\nu^2)=e^{\chi\surd{3}},e^{-\chi\surd{3}}$, the solutions to $(\tilde{\nabla}_a\tilde{\nabla}^a-1)V_1=0$ are of the form
%
\begin{eqnarray}
&&\hat{V}^{(1)}_0(A_V=-1)=\frac{e^{\chi\surd{3}}}{\sinh^2\chi},\quad \hat{V}^{(2)}_0(A_V=-1)=\frac{e^{-\chi\surd{3}}}{\sinh^2\chi},
\nonumber\\
&&\hat{V}^{(1)}_1(A_V=-1)=\frac{e^{\chi\surd{3}}}{\sinh^3\chi}\left[\surd{3}\sinh\chi-\cosh\chi\right],
\nonumber\\
&& \hat{V}^{(2)}_1(A_V=-1)=\frac{e^{-\chi\surd{3}}}{\sinh^3\chi}\left[-\surd{3}\sinh\chi-\cosh\chi\right],
\nonumber\\
&&\hat{V}^{(1)}_2(A_V=-1)=\frac{e^{\chi\surd{3}}}{\sinh^4\chi}\left[3-3\surd{3}\cosh\chi\sinh\chi+5\sinh^2\chi
\right],
\nonumber\\
&&\hat{V}^{(2)}_2(A_V=-1)=\frac{e^{-\chi\surd{3}}}{\sinh^4\chi}\left[3+3\surd{3}\cosh\chi\sinh\chi+5\sinh^2\chi\right],
\nonumber\\
&&\hat{V}^{(1)}_3(A_V=-1)=\frac{e^{\chi\surd{3}}}{\sinh^5\chi}\bigg[15\surd{3}\sinh\chi+14\surd{3}\sinh^3\chi
-15\cosh\chi
\nonumber\\
&&\qquad-24\cosh\chi\sinh^2\chi\bigg],
\nonumber\\
&&\hat{V}^{(2)}_3(A_V=-1)=\frac{e^{-\chi\surd{3}}}{\sinh^5\chi}\bigg[-15\surd{3}\sinh\chi-14\surd{3}\sinh^3\chi
-15\cosh\chi
\nonumber\\
&&\qquad-24\cosh\chi\sinh^2\chi\bigg].
\label{10.30b}
\end{eqnarray}
%
All of these solutions are bounded at $\chi=\infty$ and all $\hat{V}^{(1)}_{\ell}(A_V=-1)-\hat{V}^{(2)}_{\ell}(A_V=-1)$ with $\ell\geq 1$ are well-behaved at $\chi=0$.  Thus if implement (\ref{9.31}) by $(\tilde{\nabla}_a\tilde{\nabla}^a-1)V_i=0$,  we are not forced to $V_i=0$, with the decomposition theorem not then following in this sector.



\subsubsection{The Tensor Sector}
\label{sss:tensor_sector}

For $k=-1$ the transverse-traceless tensor sector modes need to satisfy 
%
\begin{eqnarray}
\tilde{\gamma}^{ab}T_{ab}&=& T_{11} + \frac{T_{22}}{\sinh^2\chi} + \frac{T_{33}}{\sin^2\theta \sinh^2\chi} =0,
\nonumber\\
\tilde\nabla_a T^{a 1}&=& - \frac{\cosh\chi T_{22}}{\sinh^3\chi} -  \frac{\cosh\chi T_{33}}{\sin^2\theta \sinh^3\chi} + \frac{\cos\theta T_{12}}{\sin\theta \sinh^2\chi} + \frac{2 \cosh\chi T_{11}}{\sinh\chi} + \partial_{1}T_{11} 
\nonumber\\
&&+ \frac{\partial_{2}T_{12}}{\sinh^2\chi}  + \frac{\partial_{3}T_{13}}{\sin^2\theta \sinh^2\chi}=0, \nonumber\\
\tilde\nabla_a T^{a 2}&=& - \frac{\cos\theta T_{33}}{\sin^3\theta \sinh^4\chi} + \frac{\cos\theta T_{22}}{\sin\theta \sinh^4\chi} + \frac{2 \cosh\chi T_{12}}{\sinh^3\chi} + \frac{\partial_{1}T_{12}}{\sinh^2\chi} + \frac{\partial_{2}T_{22}}{\sinh^4\chi} 
\nonumber\\
&&+ \frac{\partial_{3}T_{23}}{\sin^2\theta \sinh^4\chi}=0,
\nonumber\\
\tilde\nabla_a T^{a 3}&=& \frac{\cos\theta T_{23}}{\sin^3\theta \sinh^4\chi} + \frac{2 \cosh\chi T_{13}}{\sin^2\theta \sinh^3\chi} + \frac{\partial_{1}T_{13}}{\sin^2\theta \sinh^2\chi} + \frac{\partial_{2}T_{23}}{\sin^2\theta \sinh^4\chi}
\nonumber\\
&& + \frac{\partial_{3}T_{33}}{\sin^4\theta \sinh^4\chi}=0.
\label{10.31b}
\end{eqnarray}
%
Under these conditions the components of $\tilde{\nabla}_a\tilde{\nabla}^aT^{ij}$ evaluate to
%
\begin{eqnarray}
\tilde{\nabla}_a\tilde{\nabla}^aT^{11}&=& T_{11} \left(6 + \frac{6}{\sinh^2\chi}\right) + \frac{6 \cosh\chi \partial_{1}T_{11}}{\sinh\chi} + \partial_{1}\partial_{1}T_{11} + \frac{\cos\theta \partial_{2}T_{11}}{\sin\theta \sinh^2\chi} 
\nonumber\\
&&+ \frac{\partial_{2}\partial_{2}T_{11}}{\sinh^2\chi} + \frac{\partial_{3}\partial_{3}T_{11}}{\sin^2\theta \sinh^2\chi},
\nonumber\\ 
\tilde{\nabla}_a\tilde{\nabla}^aT^{22}&=& \frac{4 T_{22}}{\sinh^6\chi} -  \frac{4 T_{22}}{\sin^2\theta \sinh^6\chi} + \frac{4 T_{11}}{\sinh^4\chi} -  \frac{2 T_{22}}{\sinh^4\chi} -  \frac{2 T_{11}}{\sin^2\theta \sinh^4\chi} 
\nonumber\\
&&+ \frac{2 T_{11}}{\sinh^2\chi} -  \frac{2 \cosh\chi \partial_{1}T_{22}}{\sinh^5\chi} + \frac{\partial_{1}\partial_{1}T_{22}}{\sinh^4\chi} + \frac{4 \cosh\chi \partial_{2}T_{12}}{\sinh^5\chi} 
\nonumber\\
&&+ \frac{\cos\theta \partial_{2}T_{22}}{\sin\theta \sinh^6\chi} 
+ \frac{\partial_{2}\partial_{2}T_{22}}{\sinh^6\chi} -  \frac{4 \cos\theta \partial_{3}T_{23}}{\sin^3\theta \sinh^6\chi} 
+ \frac{\partial_{3}\partial_{3}T_{22}}{\sin^2\theta \sinh^6\chi},
\nonumber\\ 
\tilde{\nabla}_a\tilde{\nabla}^aT^{33}&=& \frac{2T_{33}} {\sin^4\theta\sinh^6\chi}\left(1-{\sinh^2\chi}\right) + T_{11} \left(\frac{2}{\sin^4\theta \sinh^4\chi} + \frac{2}{\sin^2\theta \sinh^2\chi}\right) 
\nonumber\\
&&-  \frac{4 \cos\theta \cosh\chi T_{12}}{\sin^3\theta \sinh^5\chi} 
 -  \frac{4 \cos\theta \partial_{1}T_{12}}{\sin^3\theta \sinh^4\chi} -  \frac{2 \cosh\chi \partial_{1}T_{33}}{\sin^4\theta \sinh^5\chi} + \frac{\partial_{1}\partial_{1}T_{33}}{\sin^4\theta \sinh^4\chi} 
 \nonumber\\
 &&+ \frac{4 \cos\theta \partial_{2}T_{11}}{\sin^3\theta \sinh^4\chi} + \frac{\cos\theta \partial_{2}T_{33}}{\sin^5\theta \sinh^6\chi}  + \frac{\partial_{2}\partial_{2}T_{33}}{\sin^4\theta \sinh^6\chi} + \frac{4 \cosh\chi \partial_{3}T_{13}}{\sin^4\theta \sinh^5\chi} 
 \nonumber\\
 &&+ \frac{\partial_{3}\partial_{3}T_{33}}{\sin^6\theta \sinh^6\chi},
\nonumber\\ 
\tilde{\nabla}_a\tilde{\nabla}^aT^{12}&=& T_{12} \left(- \frac{1}{\sin^2\theta \sinh^4\chi} -  \frac{2}{\sinh^2\chi}\right)
 + \frac{2 \cosh\chi \partial_{1}T_{12}}{\sinh^3\chi} + \frac{\partial_{1}\partial_{1}T_{12}}{\sinh^2\chi} 
 \nonumber\\
 &&+ \frac{2 \cosh\chi \partial_{2}T_{11}}{\sinh^3\chi}  + \frac{\cos\theta \partial_{2}T_{12}}{\sin\theta \sinh^4\chi} + \frac{\partial_{2}\partial_{2}T_{12}}{\sinh^4\chi}
 -  \frac{2 \cos\theta \partial_{3}T_{13}}{\sin^3\theta \sinh^4\chi}
 \nonumber\\
 && + \frac{\partial_{3}\partial_{3}T_{12}}{\sin^2\theta \sinh^4\chi},
\nonumber\\ 
\tilde{\nabla}_a\tilde{\nabla}^aT^{13}&=& - \frac{2 T_{13}}{\sin^2\theta \sinh^2\chi} + \frac{2 \cosh\chi \partial_{1}T_{13}}{\sin^2\theta \sinh^3\chi} + \frac{\partial_{1}\partial_{1}T_{13}}{\sin^2\theta \sinh^2\chi} -  \frac{\cos\theta \partial_{2}T_{13}}{\sin^3\theta \sinh^4\chi}
\nonumber\\
&& + \frac{\partial_{2}\partial_{2}T_{13}}{\sin^2\theta \sinh^4\chi}  + \frac{2 \cosh\chi \partial_{3}T_{11}}{\sin^2\theta \sinh^3\chi} + \frac{2 \cos\theta \partial_{3}T_{12}}{\sin^3\theta \sinh^4\chi} + \frac{\partial_{3}\partial_{3}T_{13}}{\sin^4\theta \sinh^4\chi},
\nonumber\\ 
\tilde{\nabla}_a\tilde{\nabla}^aT^{23}&=& T_{23} \left(\frac{2(1-\sinh^2\chi)}{\sin^2\theta\sinh^6\chi} -  \frac{1}{\sin^4\theta \sinh^6\chi}\right) + \frac{2 \cos\theta \partial_{1}T_{13}}{\sin^3\theta \sinh^4\chi} 
\nonumber\\
&&-  \frac{2 \cosh\chi \partial_{1}T_{23}}{\sin^2\theta \sinh^5\chi} + \frac{\partial_{1}\partial_{1}T_{23}}{\sin^2\theta \sinh^4\chi} + \frac{2 \cosh\chi \partial_{2}T_{13}}{\sin^2\theta \sinh^5\chi} + \frac{\cos\theta \partial_{2}T_{23}}{\sin^3\theta \sinh^6\chi} 
\nonumber\\
&&+ \frac{\partial_{2}\partial_{2}T_{23}}{\sin^2\theta \sinh^6\chi} + \frac{2 \cosh\chi \partial_{3}T_{12}}{\sin^2\theta \sinh^5\chi} + \frac{2 \cos\theta \partial_{3}T_{22}}{\sin^3\theta \sinh^6\chi} 
 + \frac{\partial_{3}\partial_{3}T_{23}}{\sin^4\theta \sinh^6\chi}.
 \nonumber\\
\label{10.32b}
\end{eqnarray}
%

Following our analysis of the vector sector, in the $k=-1$ tensor sector we seek solutions to
%
\begin{eqnarray}
(\tilde{\nabla}_a\tilde{\nabla}^a+A_T)T_{ij}=0.
\label{10.33b}
\end{eqnarray}
%
(Here $T_{ij}$ is to denote the full combination of  tensor components that appears in (\ref{9.40}).)  In (\ref{10.33b}) we have introduced a generic tensor sector constant $A_T$, whose values in (\ref{9.40})  are $(2,3,6)$.
Conveniently, we find that the equation for $T_{11}$ involves no mixing with any other components of $T_{ij}$, and can thus be solved directly. On setting $T_{11}(\chi,\theta,\phi)=h_{11,\ell}(\chi)Y_{\ell}^m(\theta,\phi)$, the equation for $T_{11}$ reduces to 
%
\begin{eqnarray}
\left[\frac{d^2}{d\chi^2}+6\frac{\cosh\chi}{ \sinh\chi}\frac{d }{d\chi}
+6+\frac{6 }{ \sinh^2\chi}-\frac{\ell(\ell+1)}{ \sinh^2\chi}+A_T\right]h_{11,\ell}=0.
\label{10.34b}
\end{eqnarray}
%

To determine the $\chi \rightarrow \infty$ and $\chi \rightarrow 0$ limits, we take the solutions to behave as $e^{\lambda\chi}$ (times an irrelevant polynomial in $\chi$) and $\chi^n$ in these two limits. For (\ref{10.34b}) the limits give
%
\begin{eqnarray}
&&\lambda^2+6\lambda+6+A_T,\quad \lambda=-3\pm(3-A_T)^{1/2},
\nonumber\\
&&\lambda(A_T=2)=(-4,~-2),
\nonumber\\
&& \lambda(A_T=3)=(-3,~-3),\quad \lambda(A_T=6)=-3\pm i\surd{3},
\nonumber\\
&&n(n-1)+6n+6-\ell(\ell+1)=0,\quad n=\ell-2, -\ell-3.
\label{10.35b}
\end{eqnarray}
%
Thus for any allowed $A_T$, every solution to (\ref{10.34b}) is bounded at $\chi=\infty$, while for each $A_T$ one of the solutions will be well-behaved as $\chi\rightarrow 0$ for any $\ell\geq 2$.  Thus for $\ell=2,3,4,..$  there will always be one solution for any allowed $A_T$ that is bounded at $\chi=\infty$ and well-behaved at $\chi=0$, with all solutions with $\ell=0$ or $\ell=1$ being excluded.

To solve (\ref{10.34b}) we set $h_{11,\ell}=\gamma_{\ell}/\sinh^2\chi$ to obtain:
%
\begin{eqnarray}
\left[\frac{d^2}{d\chi^2}+2\frac{\cosh\chi}{\sinh\chi}\frac{d}{d\chi}
-\frac{\ell(\ell+1) }{ \sinh^2\chi}-2+A_T\right]\gamma_{\ell}=0.
\label{10.36b}
\end{eqnarray}
%
We recognize (\ref{10.36b}) as being (\ref{10.3b}), and can set $\nu^2=A_T-3$ in (\ref{10.7b}), viz. $\nu^2=(-1,0,3)$ for $A_T=2,3,6$. For $A_T=2$ we see that $\nu^2=-1$. However in the scalar case discussed above where $\nu^2=A_S-1$, $\nu^2$ would also obey $\nu^2=-1$ if $A_S=0$. Thus for $A_T=2$ we can obtain the solutions to $(\tilde{\nabla}_a\tilde{\nabla}^a+2)T_{11}=0$ directly from (\ref{10.11b}), and after implementing $h_{11,\ell}=\gamma_{\ell}/\sinh^2\chi$ we obtain $T^{(1)}_{\ell}$, $T^{(2)}_{\ell}$ solutions to (\ref{10.34b}) of the form 
%
\begin{align}
&\hat{T}^{(1)}_{0}(A_T=2)=\frac{\cosh\chi}{\sinh^3\chi},\quad \hat{T}^{(2)}_{0}(A_T=2)=\frac{1}{\sinh^2\chi},
\nonumber\\
&\hat{T}^{(1)}_{1}(A_T=2)=\frac{1}{\sinh^4\chi},\quad \hat{T}^{(2)}_{1}(A_T=2)=\frac{\cosh\chi}{\sinh^3\chi}-\frac{\chi}{\sinh^4\chi},
\nonumber\\
&\hat{T}^{(1)}_{2}(A_T=2)=\frac{\cosh\chi}{\sinh^5\chi},
\nonumber\\
& \hat{T}^{(2)}_{2}(A_T=2)=\frac{1}{\sinh^2\chi}+\frac{3}{\sinh^4\chi}-\frac{3\chi\cosh\chi}{\sinh^5\chi},
\nonumber\\
&\hat{T}^{(1)}_{3}(A_T=2)=\frac{4}{\sinh^4\chi}+\frac{5}{\sinh^6\chi},
\nonumber\\
&\hat{T}^{(2)}_{3}(A_T=2)=
\frac{2\cosh\chi}{\sinh^3\chi}+\frac{15\cosh\chi}{\sinh^5\chi}-\frac{12\chi}{\sinh^4\chi}-\frac{15\chi}{\sinh^6\chi}.
\label{10.37b}
\end{align}
%
All of these solutions are bounded at $\chi=\infty$ and all $\hat{T}^{(2)}_{\ell}(A_T=2)$ with $\ell\geq 2$ are well-behaved at $\chi=0$.  Thus if implement (\ref{9.40}) by $(\tilde{\nabla}_a\tilde{\nabla}^a+2)T_{ij}=0$,  we are not forced to $T_{ij}=0$, with the decomposition theorem not then following in the tensor sector.

For $A_T=3$ we see that $\nu^2=0$. However in the vector case discussed above where $\nu^2=A_V-2$, $\nu^2$ would also obey $\nu^2=0$ if $A_V=2$. Thus for $A_T=3$ we can obtain the solutions to $(\tilde{\nabla}_a\tilde{\nabla}^a+3)T_{11}=0$ directly from (\ref{10.26b}), and after implementing $h^{11}_{\ell}=\alpha_{\ell}/\sinh\chi$ we obtain 
%
\begin{eqnarray}
\hat{T}^{(1)}_0(A_T=3)&=&\frac{1}{ \sinh^3\chi},\quad \hat{T}^{(2)}_0(A_T=3)=\frac{\chi }{\sinh^3\chi},
\nonumber\\
\hat{T}^{(1)}_1(A_T=3)&=&\frac{\cosh \chi }{ \sinh^4\chi},\quad \hat{T}^{(2)}_1(A_T=3)=\frac{1}{ \sinh^3\chi}-\frac{\chi\cosh\chi}{\sinh^4\chi},
\nonumber\\
\hat{T}^{(1)}_2(A_T=3)&=&\frac{2}{ \sinh^3\chi}+\frac{3}{\sinh^5\chi},
\nonumber\\
 \hat{T}^{(2)}_2(A_T=3)&=&\frac{3\cosh\chi}{\sinh^4\chi}-\frac{2\chi}{\sinh^3\chi}-\frac{3\chi }{\sinh^5\chi},
\nonumber\\
\hat{T}^{(1)}_3(A_T=3)&=&\frac{2\cosh\chi}{\sinh^4\chi}+\frac{5\cosh\chi}{\sinh^6\chi},
\nonumber\\
\hat{T}^{(2)}_3(A_T=3)&=&\frac{11}{\sinh^3\chi}+\frac{15}{\sinh^5\chi}-\frac{6\chi\cosh\chi}{\sinh^4\chi}-\frac{15\chi\cosh\chi }{\sinh^6\chi}.~~~
\label{10.38b}
\end{eqnarray}
%
All of these solutions are bounded at $\chi=\infty$ and all $\hat{T}^{(2)}_{\ell}(A_T=3)$ with $\ell\geq 2$ are well-behaved at $\chi=0$.  Thus if implement (\ref{9.40}) by $(\tilde{\nabla}_a\tilde{\nabla}^a+3)T_{ij}=0$,  we are not forced to $T_{ij}=0$, with the decomposition theorem not then following.

A similar outcome occurs for $A_T=6$, and even though we do not evaluate the $A_T=6$ solutions explicitly, according to (\ref{10.35b}) all solutions to $(\tilde{\nabla}_a\tilde{\nabla}^a+6)T_{11}=0$ with $A_T=6$ are bounded at $\chi=\infty$ (behaving as $e^{-3\chi}\cos(\surd{3}\chi)$ and $e^{-3\chi}\sin(\surd{3}\chi)$), with one set of these solutions being well-behaved at $\chi=0$ for all $\ell \geq 2$.  Thus if implement (\ref{9.40}) by $(\tilde{\nabla}_a\tilde{\nabla}^a+6)T_{ij}=0$,  we are not forced to $T_{ij}=0$, with the decomposition theorem again not following in the tensor sector.
%%%%%%%%%%%%%%%%%%%%%%%%%%%%%%%%%%%%%%%%%%%%
\subsubsection{Decomposition Theorem Analysis}
\label{ss:recovering_decomposition_theorem}
%%%%%%%%%%%%%%%%%%%%%%%%%%%%%%%%%%%%%%%%%%%%
In Sec. \ref{ss:rw_k=-1_svt3} we have seen that there are realizations of the evolution equations in the scalar, vector, and tensor sectors that would not lead to a decomposition theorem in those sectors. However, equally there are other realizations that given the boundary conditions would lead to a decomposition theorem. Thus we need to determine which realizations are the relevant ones. To this end we look not at the individual higher-derivative equations obeyed by the separate scalar, vector, and tensor sectors, but at how these various sectors interface with each other in the original second-order $\Delta_{\mu\nu}=0$ equations themselves. Any successful such interface would require that all the terms in $\Delta_{\mu\nu}=0$ would have to have the same $\chi$ behavior. Noting that the scalar modes appear with two $\tilde{\nabla}$ derivatives in $\Delta_{ij}=0$, the vector sector appears with one $\tilde{\nabla}$ derivative and the tensor appears with none, we need to compare derivatives of scalars with vectors and derivatives of vectors with tensors. 

To see how to obtain such a needed common $\chi$ behavior we differentiate the scalar field (\ref{10.3b}) with respect to $\chi$, to obtain
%  
\begin{eqnarray}
&&\left[\frac{d^2}{d\chi^2}+4\frac{\cosh\chi}{\sinh\chi}\frac{d }{ d\chi}
+\frac{2}{\sinh^2\chi}-\frac{\ell(\ell+1)}{\sinh^2\chi}+4+A_S\right]\frac{d S_{\ell}}{d \chi}
+2A_S\frac{\cosh\chi}{\sinh\chi}S_{\ell}
\nonumber\\
&&=0.
\label{11.1}
\end{eqnarray}
%
Comparing with the vector (\ref{10.23b}) we see that up to an overall normalization we can identify $d S_{\ell}/d\chi$ with the vector $g_{1,\ell}$ for modes that obey $A_S=0$ and $A_V=2$, so that these particular scalar and vector modes can interface. As a check, with the vector sector needing $\ell \geq 1$ we differentiate $\hat{S}^{(2)}_1(A_S=0)$ to obtain
%  
\begin{eqnarray}
\frac{d}{d \chi}\hat{S}^{(2)}_1(A_S=0) &=&\frac{d}{d \chi}\left[\frac{\cosh\chi}{\sinh\chi}-\frac{\chi}{\sinh^2\chi}\right]
\nonumber\\
&=&-\frac{2}{ \sinh^2\chi}+\frac{2\chi\cosh\chi}{ \sinh^3\chi}=-2\hat{V}^{(2)}_1(A_V=2).
\label{11.2}
\end{eqnarray}
%

Similarly, if we differentiate the vector field (\ref{10.23b}) with respect to $\chi$ we  obtain
%
\begin{eqnarray}
&&\left[\frac{d^2}{d\chi^2}+6\frac{\cosh\chi}{ \sinh\chi}\frac{d }{d\chi}
+10+A_V+\frac{6 }{ \sinh^2\chi}-\frac{\ell(\ell+1)}{ \sinh^2\chi}\right]\frac{d g_{1,\ell}}{d \chi}
\nonumber\\
&&+2(2+A_V)\frac{\cosh\chi}{\sinh\chi}g_{1,\ell}=0.
\label{11.3}
\end{eqnarray}
%
Comparing with the tensor (\ref{10.34b}) we see that up to an overall normalization we can identify $d g_{1,\ell}/d\chi$ with the tensor $h_{11,\ell}$ for modes that obey $A_V=-2$ and $A_T=2$, so that these particular vector and tensor modes can interface. As a check, with the tensor sector needing $\ell \geq 2$ we differentiate $\hat{V}^{(1)}_2(A_V=-2)$ to obtain
%  
\begin{eqnarray}
\frac{d}{d \chi}\hat{V}^{(1)}_2(A_V=-2)&=& \frac{d}{d \chi}\left[\frac{2\cosh\chi}{\sinh\chi}-\frac{3\cosh\chi }{\sinh^3\chi}+\frac{3\chi}{\sinh^4\chi}\right]
\nonumber\\
&=&\frac{4}{\sinh^2\chi}+\frac{12}{\sinh^4\chi}-\frac{12\chi\cosh\chi}{\sinh^5\chi}
\nonumber\\
&=&4\hat{T}^{(2)}_2(A_T=2).~~
\label{11.4}
\end{eqnarray}
%

Thus while we can interface $A_S=0$ and $A_V=2$, we cannot interface $A_V=2$ with any of the tensor modes. Rather, we must interface the $A_V=-2$ vector modes with the  $A_T=2$ tensor modes. With none of the scalar sector modes meeting the boundary conditions at both $\chi=\infty$ and $\chi=0$ anyway, the scalar sector must satisfy  $\Delta_{\mu\nu}=0$ by itself, with the scalar term contribution to $\Delta_{\mu\nu}=0$ then having to vanish, just as required of the decomposition theorem. However, in the vector and tensor sectors we can achieve a common $\chi$ behavior if we set $B_1-\dot{E}_1=p_1(\tau)\hat{V}^{(1)}_2(A_V=-2)$, $E_{11}=q_{11}(\tau)\hat{T}^{(2)}_2(A_T=2)$, since then the $\Delta_{11}=0$ equation reduces to
%
\begin{eqnarray}
\Delta_{11}&=&\bigg{[}\frac{1}{\sinh^2\chi}+\frac{3}{\sinh^4\chi}-\frac{3\chi\cosh\chi}{\sinh^5\chi}\bigg{]}\times
\nonumber\\
&&\bigg{[} 8\dot{\Omega} \Omega^{-1}p_1(\tau)+4\dot{p}_1(\tau)- \ddot{q}_{11}(\tau) +2 q_{11}(\tau)  -2\dot{\Omega} \Omega^{-1}\dot{q}_{11}(\tau) -2q_{11}(\tau)\bigg{]}=0.
\nonumber\\
\label{11.5}
\end{eqnarray}
%
This relation has a non-trivial solution of the form
%
\begin{eqnarray}
4p_1(\tau)-\dot{q}_{11}(\tau)=\frac{1}{\Omega^2(\tau)},
\label{11.6}
\end{eqnarray}
%
to thereby relate the $\tau$ dependencies of the vector and tensor sectors. With the other components of $V_i$ and $T_{ij}$ being constructed in a similar manner, as such we have provided an exact interface solution in the vector and tensor sectors. However, it only falls short in one regard. Both of $\hat{V}^{(1)}_2(A_V=-2)$ and $\hat{T}^{(2)}_2(A_T=2)$ are well-behaved at $\chi=0$ and $\hat{T}^{(2)}_2(A_T=2)$ vanishes at $\chi=\infty$. However, $\hat{V}^{(1)}_2(A_V=-2)$ does not vanish at $\chi=\infty$, as it limits to a constant value. Imposing a boundary condition that the vector and tensor modes have to vanish at $\chi=\infty$ then excludes this solution, with the decomposition theorem then being recovered according to 
%
\begin{eqnarray}
\tfrac{1}{2}(\dot{B}_i-\ddot{E}_i)+\dot{\Omega}\Omega^{-1}(B_i-\dot{E}_i)&=&0,
\nonumber\\
- \overset{..}{E}_{ij} +2 E_{ij} - 2 \dot{E}_{ij} \dot{\Omega} \Omega^{-1} + \tilde{\nabla}_{a}\tilde{\nabla}^{a}E_{ij}&=&0,
\label{11.7}
\end{eqnarray}
%
with these being the equations that then serve to fix the $\tau$ dependencies in the vector and tensor sectors.  Consequently, we establish that the decomposition theorem does in fact hold for Robertson-Walker cosmologies with non-vanishing spatial 3-curvature after all. 

%%%%%%%%%%%%%%%%%%%%%%%%%%%%%%%%%%%%%%%%%%%%
\subsection{$\delta W_{\mu\nu}$ Conformal to Flat}
\label{ss:deltaW_conformal_flat_SVT3}
%%%%%%%%%%%%%%%%%%%%%%%%%%%%%%%%%%%%%%%%%%%%

Since the SVT3 and SVT4 formulations are not contingent on the choice of evolution equations, we continue our study of cosmological fluctuations by discussing how things work in an alternative to standard Einstein gravity, namely conformal gravity.  For SVT3 fluctuations around a Robertson-Walker background in the conformal gravity case  we have found it more convenient not to use the metric given in (\ref{9.1}), viz.
% 
\begin{eqnarray}
ds^2&=&a^2(\tau)\left[d\tau^2-\frac{dr^2}{1-kr^2}-r^2d\theta^2-r^2\sin^2\theta d\phi^2\right],
\label{13.1}
\end{eqnarray}
% 
but to instead take advantage of the fact that via a general coordinate transformation a non-zero $k$ Robertson-Walker metric can be brought into a form in which it is conformal to flat. (With $k=0$ the metric already is conformal to flat.) The needed transformations for $k<0$ and $k>0$ may for instance be found in \cite{amarasinghe_2019}.  
For the illustrative $k<0$ case for instance, it is convenient to set $k=-1/L^2$, and introduce ${\rm sinh} \chi=r/L$ and $p=\tau/L$, with the  metric given in (\ref{13.1}) then taking the form
%
\begin{eqnarray}
ds^2=L^2a^2(p)\left[dp^2-d\chi^2 -{\rm sinh}^2\chi d\theta^2-{\rm sinh}^2\chi \sin^2\theta d\phi^2\right].
\label{13.2}
\end{eqnarray}
%
Next we introduce
%
\begin{eqnarray}
&&p^{\prime}+r^{\prime}=\tanh[(p+\chi)/2],\quad p^{\prime}-r^{\prime}=\tanh[(p-\chi)/2],
\nonumber\\
&& p^{\prime}=\frac{\sinh p}{\cosh p+\cosh \chi},\quad r^{\prime}=\frac{\sinh \chi}{\cosh p+\cosh \chi},
\label{13.3}
\end{eqnarray}
%
so that
%
\begin{eqnarray}
dp^{\prime 2}-dr^{\prime 2}&=&\frac{1}{4}[dp^2-d\chi^2]{\rm sech}^2[(p+\chi)/2]{\rm sech}^2[(p-\chi)/2],
\nonumber\\
\frac{1}{4}(\cosh p+\cosh \chi)^2&=&{\rm \cosh}^2[(p+\chi)/2]{\rm \cosh}^2[(p-\chi)/2]
\nonumber\\
&=&\frac{1}{[1-(p^{\prime}+r^{\prime})^2][1-(p^{\prime}-r^{\prime})^2]}.
\label{13.4}
\end{eqnarray}
%
With these transformations the line element takes the conformal to flat form
%
\begin{eqnarray}
ds^2=\frac{4L^2a^2(p)}{[1-(p^{\prime}+r^{\prime})^2][1-(p^{\prime}-r^{\prime})^2]}\left[dp^{\prime 2}-dr^{\prime 2} -r^{\prime 2}d\theta^2-r^{\prime 2} \sin^2\theta d\phi^2\right].
\nonumber\\
\label{13.5}
\end{eqnarray}
%
The secpatial sector can then be written in Cartesian form
%
\begin{eqnarray}
ds^2=L^2a^2(p)(\cosh p+\cosh \chi)^2\left[dp^{\prime 2}-dx^{\prime 2} -dy^{\prime 2} -dz^{\prime 2}\right],
\label{13.6}
\end{eqnarray}
%
where $r^{\prime}=(x^{\prime 2}+ y^{\prime 2}+z^{\prime 2})^{1/2}$.  

We note that while our interest in this section is in discussing fluctuations in conformal gravity, a theory that actually has an underlying conformal symmetry, in transforming from (\ref{13.1})  to (\ref{13.6}) we have only made coordinate transformations and have not made any conformal transformation. However, since the $W_{\mu\nu}$ gravitational Bach tensor introduced in (\ref{AP3}) and (\ref{AP4}) above is associated with a conformal theory, under a conformal transformation of the form $g_{\mu\nu}\rightarrow \Omega^{2}(x)g_{\mu\nu}$, $W_{\mu\nu}$ and $\delta W_{\mu\nu}$ respectively transform into $\Omega^{-2}(x)W_{\mu\nu}$ and $\Omega^{-2}(x)\delta W_{\mu\nu}$. Moreover, since this is the case for any background metric that is conformal to flat we need not even restrict to Robertson-Walker or de Sitter, and can consider fluctuations around any background metric of the form 
%
\begin{eqnarray}
ds^2=\Omega^2(x)[dt^2-\delta_{ij}dx^idx^j],
\label{13.7}
\end{eqnarray}
%
where $\delta_{ij}$ is the Kronecker delta function and $\Omega(x)$ is a completely arbitrary function of the four $x^{\mu}$ coordinates.

In this background we take the SVT3  background plus fluctuation line element to be of the form
%
\begin{eqnarray}
ds^2 &=& \Omega^2(x) \bigg[ (1+2\phi) dt^2 -2(\tilde{\nabla}_i B +B_i)dt dx^i - [(1-2\psi)\delta_{ij} +2\tilde{\nabla}_i\tilde{\nabla}_j E
\nonumber\\
&& + \tilde{\nabla}_i E_j + \tilde{\nabla}_j E_i + 2E_{ij}]dx^i dx^j\bigg],
\label{13.8}
\end{eqnarray}
%
where $\Omega(x)$ is an arbitrary function of the coordinates, where $\tilde{\nabla}_i=\partial/\partial x^i$ (with Latin index) and  $\tilde{\nabla}^i=\delta^{ij}\tilde{\nabla}_j$ (i.e. not $\Omega^{-2}\delta^{ij}\tilde{\nabla}_j$) are defined with respect to the background 3-space metric $\delta_{ij}$, and where the elements of (\ref{13.8}) obey
%
\begin{eqnarray}
\delta^{ij}\tilde{\nabla}_j B_i = 0,\quad \delta^{ij}\tilde{\nabla}_j E_i = 0, \quad E_{ij}=E_{ji},\quad \delta^{jk}\tilde{\nabla}_kE_{ij} = 0, \quad \delta^{ij}E_{ij} = 0.
\label{13.9}
\end{eqnarray}
%
For these fluctuations $\delta W_{\mu\nu}$ is readily calculated, and it is found to have the form \cite{amarasinghe_2019} 
%
\begin{eqnarray}
\delta W_{00}  &=& -\frac{2}{3\Omega^2} \delta^{mn}\delta^{\ell k}\tilde{\nabla}_m\tilde{\nabla}_n\tilde{\nabla}_{\ell}\tilde{\nabla}_k \alpha,
\nonumber\\	
\delta W_{0i} &=&  -\frac{2}{3\Omega^2} \delta^{mn}\tilde{\nabla}_i\tilde{\nabla}_m\tilde{\nabla}_n\partial_0\alpha
+\frac{1}{2\Omega^2}\left[\delta^{\ell k}\tilde{\nabla}_{\ell}\tilde{\nabla}_k(\delta^{mn}\tilde{\nabla}_m\tilde{\nabla}_n-\partial_0^2)(B_i - \dot{E}_i)\right],
\nonumber\\	
\delta W_{ij}  &=& \frac{1}{3\Omega^2}\bigg{[} \delta_{ij}\delta^{\ell k}\tilde{\nabla}_{\ell}\tilde{\nabla}_k (\partial_0^2 - \delta^{mn}\tilde{\nabla}_m\tilde{\nabla}_n) 
+(\delta^{\ell k}\tilde{\nabla}_{\ell}\tilde{\nabla}_k -3\partial_0^2)\tilde{\nabla}_i\tilde{\nabla}_j  
\bigg{] }\alpha
\nonumber\\
&&+\frac{1}{2\Omega^2}\left[ \left[\delta^{\ell k}\tilde{\nabla}_{\ell}\tilde{\nabla}_k -\partial_0^2\right]\left[\tilde{\nabla}_i   \partial_0(B_j - \dot{E}_j)+ \tilde{\nabla}_j \partial_0(B_i - \dot{E}_i)\right] \right]
\nonumber\\
&&+\frac{1}{\Omega^2}\left[\delta^{mn}\tilde{\nabla}_m\tilde{\nabla}_n-\partial_0^2\right]^2E_{ij}.
\label{13.10}
\end{eqnarray}
%
where as before $\alpha=\phi + \psi +\dot{B}-\ddot{E}$. We note that the derivatives that appear in (\ref{13.10}) are conveniently with respect to the flat Minkowski metric and not with respect to the full background $ds^2=\Omega^2(x)[dt^2-\delta_{ij}dx^idx^j]$ metric, with the $\Omega(x)$ dependence only appearing as an overall factor. This must be the case since the $\delta W_{\mu\nu}$ given in (\ref{13.10}) is related to the $\delta W_{\mu\nu}$ given in (\ref{4.6}) by an $\Omega^{-2}(x)$  conformal transformation, and the $\delta W_{\mu\nu}$ given in (\ref{4.6}) is associated with fluctuations around flat spacetime.

Unlike the standard gravity case, in a background geometry that is conformal to flat, namely in a background geometry in which the Weyl tensor vanishes,  then according to the conformal transformation properties of Sec. \ref{ss:conformal_invariance} the background $W_{\mu\nu}$ will vanish as well. The background $T_{\mu\nu}$ thus vanishes also. Fluctuations are thus described by $\delta T_{\mu\nu}=0$, and thus by $\delta W_{\mu\nu}=0$, with $\delta W_{\mu\nu}$ as given in (\ref{13.10}), and thus $\alpha$, $B_i-\dot{E}_i$ and $E_{ij}$,  thus being gauge invariant. To check whether a decomposition theorem might hold we thus need to solve the equation $\delta W_{\mu\nu}=0$. To this end we note that since there are derivatives with respect to the purely spatial $\delta^{\ell k}\tilde{\nabla}_{\ell}\tilde{\nabla}_k$, on imposing spatial boundary conditions the relation $\delta W_{00}=0$ immediately sets $\alpha=0$.  With $\alpha=0$, applying the spatial boundary conditions to the relation $W_{0i}=0$ immediately sets $(\delta^{mn}\tilde{\nabla}_m\tilde{\nabla}_n-\partial_0^2)(B_i - \dot{E}_i)=0$, with $\delta W_{ij}=0$ then realizing $\left[\delta^{mn}\tilde{\nabla}_m\tilde{\nabla}_n-\partial_0^2\right]^2E_{ij}=0$. Thus with asymptotic boundary conditions the solution to $\delta W_{\mu\nu}=0$ is
%
\begin{eqnarray}
\alpha=0,\quad (\delta^{mn}\tilde{\nabla}_m\tilde{\nabla}_n-\partial_0^2)(B_i - \dot{E}_i)=0,\quad \left[\delta^{mn}\tilde{\nabla}_m\tilde{\nabla}_n-\partial_0^2\right]^2E_{ij}=0.
\label{13.11}
\end{eqnarray}
%
Since decomposition would require
%
\begin{align}
&\delta^{mn}\delta^{\ell k}\tilde{\nabla}_m\tilde{\nabla}_n\tilde{\nabla}_{\ell}\tilde{\nabla}_k \alpha=0, 
\nonumber\\
&\delta^{mn}\tilde{\nabla}_i\tilde{\nabla}_m\tilde{\nabla}_n\partial_0\alpha=0,
\nonumber\\
&\delta^{\ell k}\tilde{\nabla}_{\ell}\tilde{\nabla}_k(\delta^{mn}\tilde{\nabla}_m\tilde{\nabla}_n-\partial_0^2)(B_i - \dot{E}_i)=0,
\nonumber\\
&\bigg{[} \delta_{ij}\delta^{\ell k}\tilde{\nabla}_{\ell}\tilde{\nabla}_k (\partial_0^2 - \delta^{mn}\tilde{\nabla}_m\tilde{\nabla}_n) 
+(\delta^{\ell k}\tilde{\nabla}_{\ell}\tilde{\nabla}_k -3\partial_0^2)\tilde{\nabla}_i\tilde{\nabla}_j  
\bigg{] }\alpha=0,
\nonumber\\
&\left[\delta^{\ell k}\tilde{\nabla}_{\ell}\tilde{\nabla}_k -\partial_0^2\right]\left[\tilde{\nabla}_i   \partial_0(B_j - \dot{E}_j)+ \tilde{\nabla}_j \partial_0(B_i - \dot{E}_i)\right]=0,
\nonumber\\
& \left[\delta^{mn}\tilde{\nabla}_m\tilde{\nabla}_n-\partial_0^2\right]^2E_{ij}=0,
\label{13.12}
\end{align}
%
we see that the decomposition theorem is recovered.

To underscore this result we note that (\ref{13.10}) can be inverted so as to write each gauge-invariant combination as a separate combination of components of $\delta W_{\mu\nu}$, viz.
%
\begin{eqnarray}
\Omega^{2}\delta W_{00} &=&- \tfrac{2}{3} \tilde{\nabla}_{b}\tilde{\nabla}^{b}\tilde{\nabla}_{a}\tilde{\nabla}^{a}\alpha,
\label{13.13}
\end{eqnarray}
%
%
\begin{eqnarray}
\tilde\nabla_a\tilde\nabla^a(\Omega^2 \delta W_{0i}) - \tilde\nabla_i \tilde\nabla^a(\Omega^2  \delta W_{0a})&=&- \tfrac{1}{2} \tilde{\nabla}_{b}\tilde{\nabla}^{b}\tilde{\nabla}_{a}\tilde{\nabla}^{a}(\overset{..}{B}_{i}-\overset{...}{E}_{i})
\\
&& + \tfrac{1}{2} \tilde{\nabla}_{c}\tilde{\nabla}^{c}\tilde{\nabla}_{b}\tilde{\nabla}^{b}\tilde{\nabla}_{a}\tilde{\nabla}^{a}(B_{i} -\dot{E}_{i}),
\nonumber
\label{13.14}
\end{eqnarray}
%
%
\begin{eqnarray}
&&\tilde\nabla_a \tilde\nabla^a \big[ \tilde\nabla_b \tilde\nabla^b (\Omega^2 \delta W_{ij})- \tilde\nabla_i \tilde\nabla^l(\Omega^2  \delta W_{jl}) -  \tilde\nabla_j \tilde\nabla^l(\Omega^2  \delta W_{il})\big]
\nonumber\\
&&+\tfrac{1}{2}\delta_{ij}\tilde\nabla_a \tilde\nabla^a\big[  \tilde\nabla^k \tilde\nabla^l(\Omega^2  \delta W_{kl})-\tilde\nabla_b \tilde\nabla^b(\Omega^2\delta^{ab}\delta W_{ab})\big]
+\tfrac{1}{2} \tilde\nabla_i\tilde\nabla_j \big[ \tilde\nabla^k \tilde\nabla^l(\Omega^2  \delta W_{kl})
\nonumber\\
&& + \tilde\nabla_a \tilde\nabla^a(\Omega^2 \delta^{ab}\delta W_{ab})\big] 
=\tilde\nabla_a \tilde\nabla^a \tilde\nabla_b \tilde\nabla^b\left[\tilde\nabla_c \tilde\nabla^c - \partial_0^2\right]^2E_{ij}.
\label{13.15}
\end{eqnarray}
%
On setting $\delta W_{\mu\nu}=0$, each of these relations can now be integrated separately, with the decomposition theorem then following. 

For completeness we also note that for  SVT3 fluctuations around the (\ref{9.2}) background $k\neq 0$ Robertson-Walker metric given in (\ref{9.1}) with $\tilde{\gamma}_{ij}dx^idx^j=dr^2/(1-kr^2)+r^2d\theta^2+r^2\sin^2\theta d\phi^2$ and with $\Omega$ actually now more generally being an arbitrary function of $\tau$ and $x^i$, the conformal gravity $\delta W_{\mu\nu}$ is given by 
%
\begin{eqnarray}
\delta W_{00}&=& - \tfrac{2}{3} \Omega^{-2} (\tilde\nabla_a\tilde\nabla^a + 3k)\tilde\nabla_b\tilde\nabla^b \alpha,
\nonumber\\ 
\delta W_{0i}&=& -\tfrac{2}{3} \Omega^{-2}  \tilde\nabla_i (\tilde\nabla_a\tilde\nabla^a + 3k)\dot\alpha
+\tfrac12 \Omega^{-2}\bigg[ \tilde\nabla_a\tilde\nabla^a (\tilde\nabla_b \tilde\nabla^b-\partial_0^2)(B_i-\dot{E}_i)
\nonumber\\
&& -2k(2k+\partial_0^2)(B_i-\dot{E}_i)\bigg],
\nonumber\\ 
\delta W_{ij}&=& -\tfrac{1}{3} \Omega^{-2} \left[ \tilde{\gamma}_{ij} \tilde\nabla_a\tilde\nabla^a (\tilde\nabla_b \tilde\nabla^b +2k-\partial_0^2)\alpha - \tilde\nabla_i\tilde\nabla_j(\tilde\nabla_a\tilde\nabla^a - 3\partial_0^2)\alpha \right]
\nonumber\\
&& +\tfrac{1}{2} \Omega^{-2} \bigg[ \tilde\nabla_i ( \tilde\nabla_a\tilde\nabla^a -2k-\partial_0^2) \partial_0 (B_j-\dot{E}_j)
\nonumber\\
&& 
+  \tilde\nabla_j ( \tilde\nabla_a\tilde\nabla^a -2k-\partial_0^2) \partial_0 (B_i-\dot{E}_i)\bigg]
\nonumber\\
&&+ \Omega^{-2}\left[ (\tilde\nabla_a\tilde\nabla^a-\partial_0)^2 E_{ij} - 4k (\tilde\nabla_a\tilde\nabla^a - k-2\partial_0^2)E_{ij} \right],
\label{13.16}
\end{eqnarray}
%
where again $\alpha = \phi + \psi + \dot B - \ddot E$. These equations can be inverted, and yield
%
\begin{align}
&(\Omega^2\delta W_{00})= - \tfrac{2}{3}  (\tilde\nabla_a\tilde\nabla^a + 3k)\tilde\nabla_b\tilde\nabla^b \alpha,
\nonumber\\
\nonumber\\
&(\tilde\nabla_a\tilde\nabla^a-2k)(\Omega^2\delta W_{0i}) - \tilde\nabla_i \tilde\nabla^a (\Omega^2\delta W_{0a}) =
\tfrac{1}{2} (\tilde\nabla_a\tilde\nabla^a - 2k - \partial_0^2)\times
\nonumber\\
&\qquad(\tilde\nabla_b\tilde\nabla^b + 2k)
(\tilde\nabla_c\tilde\nabla^c -2k)(B_i-\dot{E}_i),
\nonumber\\
\nonumber\\
&(\tilde\nabla_a\tilde\nabla^a-2k)(\tilde\nabla_b\tilde\nabla^b-3k)(\Omega^2\delta W_{ij})
+ \tfrac{1}{2} \tilde\nabla_i\tilde\nabla_j\big[ \tilde\nabla^a\tilde\nabla^b (\Omega^2\delta W_{ab})
\nonumber\\
& + (\tilde\nabla_a\tilde\nabla^a +4k)(\tilde{\gamma}^{bc}(\Omega^2\delta W_{bc}))\big]
+\tfrac{1}{2} \tilde{\gamma}_{ij} \big[ (\tilde\nabla_a\tilde\nabla^a-4k)\tilde\nabla^b\tilde\nabla^c (\Omega^2\delta W_{bc})
\nonumber\\
&-(\tilde\nabla_a\tilde\nabla^a\tilde\nabla_b\tilde\nabla^b -2k \tilde\nabla_a\tilde\nabla^a +4k^2)(\tilde{\gamma}^{bc}(\Omega^2\delta W_{bc}))\big]
\nonumber\\
&-(\tilde\nabla_a\tilde\nabla^a -3k)(\tilde\nabla_i\tilde\nabla^b (\Omega^2\delta W_{jb}) + \tilde\nabla_j \tilde\nabla^b (\Omega^2\delta W_{ib}))
\nonumber\\
&=(\tilde\nabla_a\tilde\nabla^a-2k)(\tilde\nabla_b\tilde\nabla^b-3k)\left[ (\tilde\nabla_a\tilde\nabla^a-\partial_0)^2 E_{ij} - 4k (\tilde\nabla_a\tilde\nabla^a - k-2\partial_0^2)E_{ij} \right].
\label{13.17}
\end{align}
%
With this separation of the gauge-invariant combinations we again have the decomposition theorem. 

%%%%%%%%%%%%%%%%%%%%%%%%%%%%%%%%%%%%%%%%%%%%
\subsection{Gauge Invariants and the Decomposition Theorem}
\label{ss:gauge_invariants_decomp_theorem}
%%%%%%%%%%%%%%%%%%%%%%%%%%%%%%%%%%%%%%%%%%%%
In the SVT4 study of fluctuations around a de Sitter background that we presented in Sec. \ref{ss:ds4_svt4} we had found that one of the gauge-invariant combinations was given by $\alpha=\dot{F}+\tau \chi +F_0$ (see (\ref{6.54})). In this combination $F$ and $\chi$ are scalars while $F_0$ is the fourth component of the vector $F_{\mu}$. In solving the fluctuation equations in this case we actually solved for the gauge-invariant combinations and not for the individual scalar, vector and tensor sectors. In the solution we found that $\alpha=0$.  Thus would entail only that  $\dot{F}+\tau \chi =-F_0$. However, decomposition with respect to scalars, vectors and tensors would in addition entail that $\dot{F}+\tau \chi=0$, and $F_0=0$, something that would not be warranted as it is not required by the fluctuation equations, while moreover not being a gauge-invariant decomposition of the 
gauge-invariant $\alpha$. Thus given this example we in general see that one should only look for a decomposition theorem for gauge-invariant combinations and not look for one for the separate scalar, vector and tensor sectors as gauge invariance can in general intertwine them. Since it might perhaps be thought that this is an artifact of using SVT4 we now present two examples in which it occurs in SVT3. One is fluctuations around an anti de Sitter background, and the other is fluctuations around a completely general conformal to flat background.

\subsubsection{Fluctuations Around an Anti de Sitter Background}
\label{sss:fluctuations_around_ads4}
For an anti de Sitter background in four dimensions we have
%
\begin{eqnarray}
ds^2 &=& \Omega^2(z)\left[ dt^2 - dx^2-dy^2-dz^2\right]= -g_{\mu\nu}dx^{\mu} dx^{\nu},\quad
\Omega(z) = \frac{1}{Hz},
\nonumber\\
R_{\lambda\mu\nu\kappa} &=& -H^2(g_{\mu\nu}g_{\lambda\kappa} -g_{\lambda\nu}g_{\mu\kappa}),
\quad R_{\mu\kappa} =3H^2 g_{\mu\kappa},\quad R = 12H^2,
\nonumber\\
G_{\mu\nu} &=& -3H^2 g_{\mu\nu},\quad T_{\mu\nu} = 3H^2 g_{\mu\nu}.
\label{14.1}
\end{eqnarray}
%
We take the fluctuations to have the standard SVT3 form of
%
\begin{eqnarray}
ds^2 &=&- \Omega^2(z)\left( \eta_{\mu\nu}+ f_{\mu\nu}\right) dx^\mu dx^\nu,
\nonumber\\
f_{00} &=& -2 \phi,\quad f_{0i} = \tilde\nabla_i B + B_i,\quad \tilde{\nabla}^iB_i=0,
\nonumber\\
f_{ij} &=& -2 \psi \delta_{ij} + 2\tilde\nabla_i\tilde\nabla_j E + \tilde\nabla_i E_j
+ \tilde\nabla_i E_j + 2E_{ij},
\nonumber\\
\tilde{\nabla}^iE_i&=&0,\quad \tilde{\nabla}^iE_{ij}=0,\quad \delta^{ij}E_{ij}=0, 
\label{14.2}
\end{eqnarray}
%
where $\tilde{\nabla}_i$ and $\tilde{\nabla}^i=\delta^{ij}\tilde{\nabla}_j$ are defined with respect to a flat three-dimensional background $\eta_{ij}dx^idx^j=\delta_{ij}dx^idx^j$.

On defining
%
\begin{eqnarray}
\alpha &=& \phi +\psi+\dot B - \ddot E, \quad \delta = \phi -\psi + \dot B - \ddot E + \frac{2}{z}(\tilde\nabla_3 E + E_3),
\label{14.3}
\end{eqnarray}
%
following some algebra we find that the components of 
\begin{eqnarray}
\Delta_{\mu\nu}=\delta G_{\mu\nu}+\delta T_{\mu\nu}=\delta G_{\mu\nu}+ 3\Omega^2 H^2 f_{\mu\nu}
\end{eqnarray}
are given by 
\begin{eqnarray}
g^{\mu\nu}\Delta_{\mu\nu}&=& -12 H^2 \alpha - 3 H^2 z^2 \overset{..}{\alpha} + 3 H^2 z^2 \overset{..}{\delta} + 12 H^2 \delta + H^2 z^2 \tilde\nabla^{2}{}\alpha - 3 H^2 z^2 \tilde\nabla^{2}{}\delta 
\nonumber \\ 
&& + 6 H^2 z \tilde{\nabla}_{3}\delta +6 H^2 z (\dot{B}_3-\ddot{E}_3)+24 H^2 E_{33},
\nonumber\\ 
\delta^{ij}\Delta_{ij}&=& -9 z^{-2} \alpha - 3 \overset{..}{\alpha} + 3 \overset{..}{\delta} + 9 z^{-2} \delta - 2 \tilde\nabla^{2}{}\delta + z^{-1} \tilde{\nabla}_{3}\alpha + 5 z^{-1} \tilde{\nabla}_{3}\delta 
\nonumber\\
&&+6 z^{-1} (\dot{B}_3-\ddot{E}_3)+18 z^{-2} E_{33},
\nonumber\\ 
\Delta_{00}&=& 3 z^{-2} \alpha - 3 z^{-2} \delta -  \tilde\nabla^{2}{}\alpha + \tilde\nabla^{2}{}\delta + z^{-1} \tilde{\nabla}_{3}\alpha -  z^{-1} \tilde{\nabla}_{3}\delta -6 z^{-2} E_{33},
\nonumber\\ 
\Delta_{11}&=& -3 z^{-2} \alpha -  \overset{..}{\alpha} + \overset{..}{\delta} + 3 z^{-2} \delta -  \tilde\nabla^{2}{}\delta + \tilde{\nabla}_{1}\tilde{\nabla}_{1}\delta + z^{-1} \tilde{\nabla}_{3}\alpha + z^{-1} \tilde{\nabla}_{3}\delta 
\nonumber\\
&&+2 z^{-1} (\dot{B}_3-\ddot{E}_3) + \tilde{\nabla}_{1}(\dot{B}_1-\ddot{E}_1)
- \overset{..}{E}_{11} + 6 z^{-2} E_{33} + \tilde\nabla^{2}{}E_{11} 
\nonumber\\
&&+ 4 z^{-1} \tilde{\nabla}_{1}E_{13} - 2 z^{-1} \tilde{\nabla}_{3}E_{11},
\nonumber\\ 
\Delta_{22}&=& -3 z^{-2} \alpha -  \overset{..}{\alpha} + \overset{..}{\delta} + 3 z^{-2} \delta -  \tilde\nabla^{2}{}\delta + \tilde{\nabla}_{2}\tilde{\nabla}_{2}\delta + z^{-1} \tilde{\nabla}_{3}\alpha + z^{-1} \tilde{\nabla}_{3}\delta 
\nonumber\\
&&+2 z^{-1} (\dot{B}_3-\ddot{E}_3) + \tilde{\nabla}_{2}(\dot{B}_2-\ddot{E}_2)  - \overset{..}{E}_{22}+ 6 z^{-2} E_{33} + \tilde\nabla^{2}{}E_{22} 
\nonumber\\
&&+ 4 z^{-1} \tilde{\nabla}_{2}E_{23} - 2 z^{-1} \tilde{\nabla}_{3}E_{22},
\nonumber\\ 
\Delta_{33}&=& -3 z^{-2} \alpha -  \overset{..}{\alpha} + \overset{..}{\delta} + 3 z^{-2} \delta -  \tilde\nabla^{2}{}\delta -  z^{-1} \tilde{\nabla}_{3}\alpha + 3 z^{-1} \tilde{\nabla}_{3}\delta + \tilde{\nabla}_{3}\tilde{\nabla}_{3}\delta 
\nonumber\\
&&+2 z^{-1} (\dot{B}_3-\ddot{E}_3) + \tilde{\nabla}_{3}(\dot{B}_3-\ddot{E}_3)
\nonumber \\ 
&&  -\overset{..}{E}_{33}+ 6 z^{-2} E_{33} + \tilde\nabla^{2}{}E_{33} + 2 z^{-1} \tilde{\nabla}_{3}E_{33},
\nonumber\\ 
\Delta_{01}&=& - \tilde{\nabla}_{1}\dot{\alpha} + \tilde{\nabla}_{1}\dot{\delta}+\tfrac{1}{2} \tilde\nabla^{2}{}(B_1-\dot{E}_1) + z^{-1} \tilde{\nabla}_{1}(B_3-\dot{E}_3) -  z^{-1} \tilde{\nabla}_{3}(B_1-\dot{E}_1)
\nonumber\\
&&+2 z^{-1} \dot{E}_{13},
\nonumber\\ 
\Delta_{02}&=& - \tilde{\nabla}_{2}\dot{\alpha} + \tilde{\nabla}_{2}\dot{\delta}+\tfrac{1}{2} \tilde\nabla^{2}{}(B_2-\dot{E}_2) + z^{-1} \tilde{\nabla}_{2}(B_3-\dot{E}_3) -  z^{-1} \tilde{\nabla}_{3}(B_2-\dot{E}_2)
\nonumber\\
&&+2 z^{-1} \dot{E}_{23},
\nonumber\\ 
\Delta_{03}&=& - z^{-1} \dot{\alpha} + z^{-1} \dot{\delta} -  \tilde{\nabla}_{3}\dot{\alpha} + \tilde{\nabla}_{3}\dot{\delta}+\tfrac{1}{2} \tilde\nabla^{2}{}(B_3-\dot{E}_3)+2 z^{-1} \dot{E}_{33},
\nonumber\\ 
\Delta_{12}&=& \tilde{\nabla}_{2}\tilde{\nabla}_{1}\delta +\tfrac{1}{2} \tilde{\nabla}_{1}(\dot{B}_2-\ddot{E}_2) + \tfrac{1}{2} \tilde{\nabla}_{2}(\dot{B}_1-\ddot{E}_1)- \overset{..}{E}_{12} + \tilde\nabla^{2}{}E_{12}
\nonumber\\
&& + 2 z^{-1} \tilde{\nabla}_{1}E_{23} 
+ 2 z^{-1} \tilde{\nabla}_{2}E_{13} - 2 z^{-1} \tilde{\nabla}_{3}E_{12},
\nonumber\\ 
\Delta_{13}&=& - z^{-1} \tilde{\nabla}_{1}\alpha + z^{-1} \tilde{\nabla}_{1}\delta + \tilde{\nabla}_{3}\tilde{\nabla}_{1}\delta +\tfrac{1}{2} \tilde{\nabla}_{1}(\dot{B}_3-\ddot{E}_3) + \tfrac{1}{2} \tilde{\nabla}_{3}(\dot{B}_1-\ddot{E}_1)
\nonumber\\
&&- \overset{..}{E}_{13} 
+ \tilde\nabla^{2}{}E_{13} + 2 z^{-1} \tilde{\nabla}_{1}E_{33},
\nonumber\\ 
\Delta_{23}&=& - z^{-1} \tilde{\nabla}_{2}\alpha + z^{-1} \tilde{\nabla}_{2}\delta + \tilde{\nabla}_{3}\tilde{\nabla}_{2}\delta +\tfrac{1}{2} \tilde{\nabla}_{2}(\dot{B}_3-\ddot{E}_3) + \tfrac{1}{2} \tilde{\nabla}_{3}(\dot{B}_2-\ddot{E}_2)
\nonumber\\
&&- \overset{..}{E}_{23} 
+ \tilde\nabla^{2}{}E_{23} + 2 z^{-1} \tilde{\nabla}_{2}E_{33},
\label{14.4} 
\end{eqnarray}
%
where $\tilde\nabla^{2} = \delta^{ab} \tilde\nabla_a\tilde\nabla_b$. With $\Delta_{\mu\nu}$ being gauge invariant we recognize $\alpha$, $\delta$, $B_i-\dot{E}_i$ and $E_{ij}$ as being gauge invariant. We thus see that one of the gauge-invariant combinations, viz. $\delta$, depends on both scalars and vectors. Since our only purpose here is in establishing that one of the gauge-invariant SVT3 combinations does depend on both scalars and vectors, we shall not seek to solve $\Delta_{\mu\nu}=0$ in this particular case. Though if we were to we would only find expressions for $\alpha$, $\delta$, $B_i-\dot{E}_i$ and $E_{ij}$, and not for the separate scalar and vector components of $\delta$.

\subsubsection{Fluctuations Around a General Conformal to Flat Background}
\label{sss:fluctuations_around_general_conformal_flat}

In \cite{amarasinghe_2019} it was shown that for the arbitrary conformal to flat SVT3 fluctuations of the form
%
\begin{eqnarray}
ds^2 &=&\Omega^2({\bf x},t)\bigg[(1+2\phi) dt^2 -2(\partial_i B +B_i)dt dx^i - [(1-2\psi)\delta_{ij} +2\partial_i\partial_j E 
\nonumber\\
&&+ \partial_i E_j + \partial_j E_i + 2E_{ij}]dx^i dx^j\bigg]
\label{14.5}
\end{eqnarray}
%
with general $\Omega({\bf x},t)$, the metric sector gauge-invariant combinations are
%
\begin{eqnarray}
\alpha &=& \phi +\psi+\dot B - \ddot E, \quad \eta=\psi -\Omega^{-1}\dot{\Omega}(B-\dot E)+\Omega^{-1}\tilde\nabla^i\Omega(E_i+\tilde\nabla_i E),
\nonumber\\
&& B_i-\dot E_i,\quad E_{ij}.
\label{14.6}
\end{eqnarray}
%
Of these invariant combinations three are independent of $\Omega$ altogether and have been encountered frequently throughout this study, while only one, viz. $\eta$, actually depends on $\Omega$ at all. (For specific choices of $\Omega$ the quantity $-\Omega\dot{\Omega}^{-1}\eta$ reduces to the previously introduced $\beta$ in the de Sitter (\ref{7.3}) and to $\gamma$ in the Robertson-Walker (\ref{8.7}) and (\ref{9.2}), while $\alpha-2\eta$ reduces to the anti de Sitter $\delta$ given in (\ref{14.3}).) The invariant $\alpha$ involves scalars alone, the invariant $B_i-E_i$ involves vectors alone, the invariant $E_{ij}$  involves tensors alone, and only the invariant $\eta$ actually involves more than just one of the scalar, vector and tensor sets of modes, with it specifically involving both scalars and vectors. While $\eta$ must always involve scalars, if $\Omega$ has a spatial dependence $\eta$ will also involve the vector $E_i$. A spatial dependence for $\Omega$ is  encountered in our study of anti de Sitter fluctuations as shown in (\ref{14.3}), and is also encountered in SVT3 fluctuations around a Robertson-Walker background with $k\neq 0$,  where the background Robertson-Walker metric as shown in (\ref{13.6}) for $k<0$ (and in \cite{amarasinghe_2019} for $k>0$) is written in a conformal to flat form, with the conformal factor expressly being a function of both time and space coordinates. Thus in such cases we must expect $\Delta_{\mu\nu}$ to depend on $\eta$ itself and not be separable in separate scalar and vector sectors. While this issue is met for $k\neq 0$ Robertson-Walker fluctuations when the background metric is written in the conformal to flat form given in (\ref{13.6}), we note that it is not in fact met for fluctuations around a background Robertson-Walker geometry with metric $ds^2=\Omega^2(\tau)[d\tau^2-dr^2/(1-kr^2)-r^2d\theta^2-r^2\sin^2\theta d\phi^2]$ as given in (\ref{9.1}), since with $\Omega$ only depending on $\tau$ in that case, the gauge-invariant $\gamma = - \dot\Omega^{-1}\Omega \psi + B - \dot E$ as given in (\ref{9.12}) does not involve the vector sector modes. While one can of course transform the background metric given in (\ref{13.6}) into the background metric given in (\ref{9.1}) by a coordinate transformation with fluctuations around the two metrics thus describing the same physics, the very structure of (\ref{14.6}) shows that one cannot simply separate in scalar, vector tensor components at will. Rather one must first separate in gauge-invariant combinations, and only if these combinations turn out not to intertwine any of the scalar, vector and tensor sectors could one then separate in each of the scalar, vector and tensor sectors. Moreover, while one can find a form for the background metric in which there is no such intertwining in the $k\neq 0$ Robertson-Walker case (for $k=0$ $\Omega$ only depends on $\tau$ and so there is no intertwining), this only occurs because of the specific purely $\tau$-dependent form that  the $k\neq 0$ $\Omega$ just happens to take.  For more complicated $\Omega$ there would be no coordinate transformation that would eliminate the intertwining, and so it is of interest to study the spatially dependent $\Omega$ situation in and of itself.

While we of course do not need to explicitly solve for fluctuations around a $k\neq 0$ Robertson-Walker metric when written in a conformal to flat form since in Secs. \ref{ss:general_rw_svt3}, \ref{ss:rw_k=-1_svt3}, and \ref{ss:recovering_decomposition_theorem} we already have solved for fluctuations around the same geometry when written in the general coordinate equivalent form given in (\ref{9.1}), it is nonetheless of interest to explore the structure of fluctuations around the conformal to flat form for a $k\neq 0$ Robertson-Walker geometry. In particular it is of interest to show that the $\eta$ invariant given in (\ref{14.6}) actually behaves quite differently from all the other invariants.   In (\ref{9.43a}) to (\ref{9.47a}) we had obtained some kinematic relations (i.e., relations that do not involve the evolution equations) that express the gauge-invariant combinations in terms of the  $f_{\mu\nu}$ components of the fluctuation metric. Inspection of these relations and of (\ref{9.12}) shows there is  only one, viz. that for the relevant $\eta$ in that case, that depends on $\Omega$. Now the relations given  in (\ref{9.43a}) to (\ref{9.47a}) were derived for fluctuations around the (\ref{9.1}) metric. If we now set $k=0$ in these relations so that the $\tilde{\nabla}$ derivative now refers to flat spacetime, we would anticipate that for fluctuations around (\ref{13.6}) the relations for the gauge-invariant combinations that do not involve $\Omega$ might be replaced by 

%
\begin{align}
\tilde{\nabla}^b\tilde{\nabla}_b\tilde{\nabla}^a\tilde{\nabla}_a\alpha&=-\frac{1}{2}\tilde{\nabla}^b\tilde{\nabla}_b\tilde{\nabla}^i\tilde{\nabla}_if_{00}
+\frac{1}{4}\tilde{\nabla}^a\tilde{\nabla}_a\left(-\tilde{\nabla}^b\tilde{\nabla}_bf+\tilde{\nabla}^m\tilde{\nabla}^nf_{mn}\right)
\nonumber\\
&
+\partial_0\tilde{\nabla}^b\tilde{\nabla}_b\tilde{\nabla}^if_{0i}
-\frac{1}{4}\partial^2_0\left(3\tilde{\nabla}^m\tilde{\nabla}^nf_{mn}-\tilde{\nabla}^a\tilde{\nabla}_af\right),
\label{14.7}
\end{align}
%
%
\begin{align}
\tilde{\nabla}^a\tilde{\nabla}_a\tilde{\nabla}^i\tilde{\nabla}_i(B_j-\dot{E_j})&=\tilde{\nabla}^i\tilde{\nabla}_i (\tilde{\nabla}^a\tilde{\nabla}_af_{0j}-\tilde{\nabla}_j\tilde{\nabla}^af_{0a})
-\partial_0\tilde{\nabla}^a\tilde{\nabla}_a\tilde{\nabla}^if_{ij}
\nonumber\\
&
+\partial_0\tilde{\nabla}_j\tilde{\nabla}^a\tilde{\nabla}^bf_{ab},
\label{14.8}
\end{align}
%
%
\begin{align}
2\tilde{\nabla}^a\tilde{\nabla}_a\tilde{\nabla}^b\tilde{\nabla}_bE_{ij}
&=\tilde{\nabla}^a\tilde{\nabla}_a\tilde{\nabla}^b\tilde{\nabla}_bf_{ij}
+\tfrac{1}{2}\tilde{\nabla}_i\tilde{\nabla}_j\left[\tilde{\nabla}^a\tilde{\nabla}^bf_{ab}+\tilde{\nabla}^a\tilde{\nabla}_af\right]
\nonumber\\
&-\tilde{\nabla}^a\tilde{\nabla}_a(\tilde{\nabla}_i\tilde{\nabla}^bf_{jb}+\tilde{\nabla}_j\tilde{\nabla}^bf_{ib})
\nonumber\\
&
+\tfrac{1}{2}\tilde{\gamma}_{ij}\left[\tilde{\nabla}^a\tilde{\nabla}_a\tilde{\nabla}^b\tilde{\nabla}^cf_{bc}
-\tilde{\nabla}_a\tilde{\nabla}^a\tilde{\nabla}_b\tilde{\nabla}^bf\right],
\label{14.9}
\end{align}
%
(where $f=\delta^{ab}f_{ab}$), with $\alpha$ still being given by (\ref{14.6}). Explicit calculation shows that this anticipation is actually borne out, with (\ref{14.7}), (\ref{14.8}) and (\ref{14.9}) being found to hold for the fluctuations given in (\ref{14.5}), no matter how arbitrary $\Omega$ might be. 

Now in our discussion of the fluctuations associated with the conformal gravity $\delta W_{\mu\nu}$ we had obtained the relations given in (\ref{13.10}) and their inversion as given in (\ref{13.13}), (\ref{13.14}) and (\ref{13.15}). We now note that these relations involve the same gauge-invariant combinations as the ones that appear in (\ref{14.7}), (\ref{14.8}) and (\ref{14.9}), viz. $\alpha$, $B_i-\dot{E_i}$ and $E_{ij}$, with $\eta$ not appearing. That $\eta$ could not appear in $\delta W_{\mu\nu}$ is because $W_{\mu\nu}$ is traceless so that on allowing for four coordinate transformations $\delta W_{\mu\nu}$ can only involve five quantities (a one-component $\alpha$, a two-component $B_i-\dot{E}_i$, and a two-component $E_{ij}$). In this sense then we should think of $\alpha$, $B_i-\dot{E_i}$ and $E_{ij}$ as a unit, with $\eta$ needing to be treated independently.

Since $W_{\mu\nu}$ is zero in a conformal to flat background, it is associated with a background $T_{\mu\nu}$ that is zero, with $\delta W_{\mu\nu}$ then being gauge invariant on its own as $\delta T^{\mu\nu}$ is then zero. Thus to determine a gauge-invariant relation that does involve $\eta$ we should look for a purely geometric gauge-invariant fluctuation relation that does not involve $\delta T_{\mu\nu}$. However, none is immediately available as we have already used up $\delta W_{\mu\nu}$, and in general a fluctuation such as $\delta G_{\mu\nu}$ would not be gauge invariant on its own. However, there is one situation in which not $\delta G_{\mu\nu}$ but  $\delta(g^{\mu\nu}G_{\mu\nu})$ is gauge invariant on its own, namely in the radiation era where $T_{\mu\nu}$, and thus $G_{\mu\nu}$, are traceless, with $\delta (g^{\mu\nu}T_{\mu\nu})$ then being zero, and with the quantity $\delta(g^{\mu\nu}G_{\mu\nu})$ then being gauge invariant on its own.

Thus in a radiation era conformal to flat $k\neq 0$ Robertson-Walker background case as given by (\ref{14.5}) we evaluate 
%
\begin{eqnarray}
g^{\mu\nu}G_{\mu\nu}=6\ddot{\Omega}\Omega^{-3}-6\Omega^{-3}\tilde{\nabla}_a\tilde{\nabla}^a\Omega=0,
\label{14.10}
\end{eqnarray}
%
and on setting $\ddot{\Omega}=\tilde{\nabla}_a\tilde{\nabla}^a\Omega$ obtain 
%
\begin{eqnarray}
\delta(g^{\mu\nu} G_{\mu\nu})&=& -6 \dot{\alpha} \dot{\Omega} \Omega^{-3} - 12 \dot{\eta} \dot{\Omega} \Omega^{-3} - 12 \overset{..}{\Omega} \alpha \Omega^{-3} - 6 \overset{..}{\eta} \Omega^{-2} - 2 \Omega^{-2} \tilde{\nabla}_{a}\tilde{\nabla}^{a}\alpha 
\nonumber\\
&&+ 6 \Omega^{-2} \tilde{\nabla}_{a}\tilde{\nabla}^{a}\eta  - 6 \Omega^{-3} \tilde{\nabla}_{a}\Omega \tilde{\nabla}^{a}\alpha
 + 12 \Omega^{-3} \tilde{\nabla}_{a}\Omega \tilde{\nabla}^{a}\eta
\\
 && -12 (B^{a}-\dot{E}^a) \Omega^{-3} \tilde{\nabla}_{a}\dot{\Omega} - 6 (\dot{B}^{a}-\ddot{E}^a) \Omega^{-3} \tilde{\nabla}_{a}\Omega +12 E^{ab} \Omega^{-3} \tilde{\nabla}_{b}\tilde{\nabla}_{a}\Omega,
 \nonumber
\label{14.11}
\end{eqnarray}
%
where $\alpha$, $B_i-\dot{E}_i$ and $E_{ij}$ are as before, with $\eta$ now being given by the form given in (\ref{14.6}). We thus establish that in the conformal to flat case the appropriate $\eta$ is indeed given by the spatially-dependent $\eta=\psi -\Omega^{-1}\dot{\Omega}(B-\dot E)+\Omega^{-1}\tilde\nabla^i\Omega(E_i+\tilde\nabla_i E)$, just as required.

\subsubsection{Taking Advantage of the Gauge Freedom}
\label{sss:taking_advantage_gauge_freedom}

In \cite{amarasinghe_2019} infinitesimal gauge transformations of the form 
%
\begin{eqnarray}
\bar{h}_{\mu\nu}=h_{\mu\nu}-\nabla _{\mu}\epsilon_{\nu}-\nabla _{\nu}\epsilon_{\mu}
\label{14.12}
\end{eqnarray}
% 
acting on the conformal to flat (\ref{14.5}) with arbitrary $\Omega$ were considered. On introducing gauge parameters
%
\begin{eqnarray}
\epsilon_{\mu}=\Omega^2(x)f_{\mu},\quad f_{0}=-T,\quad f_i=L_i+\tilde{\nabla}_iL,\quad \delta^{ij}\tilde{\nabla}_jL_i=\tilde{\nabla}^iL_i=0,
\label{14.13}
\end{eqnarray}
%
the following transformation relations were found
%
\begin{eqnarray}
\bar{\phi}&=&\phi-\dot{T}-\Omega^{-1}[T\partial_0+(L_i+\tilde{\nabla}_iL)\delta^{ij}\partial_j]\Omega,\quad \bar{B}=B+T-\dot{L},
\nonumber\\
 \bar{\psi}&=&\psi+\Omega^{-1}[T\partial_0+(L_i+\tilde{\nabla}_iL)\delta^{ij}\partial_j]\Omega,
\nonumber\\
\bar{E}&=&E-L,\quad \bar{B}_i=B_i-\dot{L}_i,\quad \bar{E}_i=E_i-L_i, \quad \bar{E}_{ij}=E_{ij},
\label{14.14}
\end{eqnarray}
%
with the elimination of the gauge parameters leading directly to the gauge-invariant combinations shown in (\ref{14.6}). We now specialize to a particular gauge, and pick the gauge parameters so that 
%
\begin{eqnarray}
L_i=E_i,\quad L=E, \quad B+T-\dot{L}=0.
\label{14.15}
\end{eqnarray}
%
With this choice (\ref{14.6}) simplifies to
%
\begin{eqnarray}
\alpha &=& \phi +\psi, \quad \eta=\psi ,\quad B_i,\quad E_{ij},
\label{14.16}
\end{eqnarray}
%
and now $\eta$ only depends on scalars. The combinations given in (\ref{14.16}) constitute the longitudinal or conformal-Newtonian gauge, a gauge that is often considered in cosmological perturbation theory (see e.g. \cite{mukhanov_1992, bertschinger_2000}). We thus see that using the gauge freedom one can find gauges in which there is no intertwining of scalars and vectors, so that for them a decomposition theorem in the separate scalar, vector and tensor sectors is achievable.

%%%%%%%%%%%%%%%%%%%%%%%%%%%%%%%%%%%%%%%%%%%%
\section{SVT4}
\label{s:svt4}
%%%%%%%%%%%%%%%%%%%%%%%%%%%%%%%%%%%%%%%%%%%%
As well as discuss SVT3 fluctuations around general Robertson-Walker backgrounds in Einstein gravity,  it is of interest to discuss SVT4 fluctuations  as well.
%%%%%%%%%%%%%%%%%%%%%%%%%%%%%%%%%%%%%%%%%%%%
\subsection{Minkowski}
\label{ss:minkowski_svt4}
%%%%%%%%%%%%%%%%%%%%%%%%%%%%%%%%%%%%%%%%%%%%

In treating the fluctuation equations there are two types of perturbation that one needs to consider. If we start with the Einstein equations in the presence of some general non-zero background $T_{\mu\nu}$, viz. $G_{\mu\nu}+8\pi G T_{\mu\nu}=0$, the first type is to consider perturbations $\delta G_{\mu\nu}$ and $\delta T_{\mu\nu}$ to the background and look for solutions to  
%
\begin{eqnarray}
\delta G_{\mu\nu}+8\pi G \delta T_{\mu\nu}=0
\label{5.1}
\end{eqnarray}
%
in a background that obeys $G_{\mu\nu}+8\pi G T_{\mu\nu}=0$. If the background is not flat, the fluctuation $\delta G_{\mu\nu}$ will not be gauge invariant  but it will instead be the combination $\delta G_{\mu\nu}+8\pi G \delta T_{\mu\nu}$  that will be expressible in the gauge-invariant SVT3 or SVT4 bases as appropriately generalized to a non-flat background.

The second kind of perturbation is one in which we introduce some new perturbation $\delta \bar{T}_{\mu\nu}$ to a background that obeys $G_{\mu\nu}+8\pi G T_{\mu\nu}=0$. This $\delta \bar{T}_{\mu\nu}$ will modify both the background $G_{\mu\nu}$ and the background $T_{\mu\nu}$ and will lead to a fluctuation equation of the form 
%
\begin{eqnarray}
\delta G_{\mu\nu}+8\pi G \delta T_{\mu\nu}=-8 \pi G \delta \bar{T}_{\mu\nu}. 
\label{5.2}
\end{eqnarray}
%
In (\ref{5.2}) the combination $\delta G_{\mu\nu}+8\pi G \delta T_{\mu\nu}$ will be gauge invariant since structurally it will be of the same form as it would be in the absence of $\delta \bar{T}_{\mu\nu}$, and would thus be gauge invariant since it already was in the absence of $\delta \bar{T}_{\mu\nu}$. In consequence of this any $\delta \bar{T}_{\mu\nu}$ that we could introduce would have to be gauge invariant all on its own.

If there is no background $T_{\mu\nu}$ so that the background metric is flat, the only perturbation that one could consider is $\delta \bar{T}_{\mu\nu}$, with the fluctuation equation then being of the form 
%
\begin{eqnarray}
\delta G_{\mu\nu}=-8 \pi G \delta \bar{T}_{\mu\nu}. 
\label{5.3}
\end{eqnarray}
%
With  $\delta \bar{T}_{\mu\nu}$ obeying $\nabla^{\nu}\delta \bar{T}_{\mu\nu}=0$, in analog to (\ref{3.10}) in general in the SVT4 case $\delta \bar{T}_{\mu\nu}$ must be of the form $\nabla_{\alpha}\nabla^{\alpha}\bar{F}_{\mu\nu}+2(g_{\mu\nu}\nabla_{\alpha}\nabla^{\alpha}-\nabla_{\mu}\nabla_{\nu})\bar{\chi}$, with (\ref{5.3}) taking the form 
%
\begin{eqnarray}
\nabla_{\alpha}\nabla^{\alpha}F_{\mu\nu}+2(g_{\mu\nu}\nabla_{\alpha}\nabla^{\alpha}-\nabla_{\mu}\nabla_{\nu})\chi&=&-8 \pi G[\nabla_{\alpha}\nabla^{\alpha}\bar{F}_{\mu\nu}
\\
&&+2(g_{\mu\nu}\nabla_{\alpha}\nabla^{\alpha}-\nabla_{\mu}\nabla_{\nu})\bar{\chi}]. 
\nonumber
\label{5.4}
\end{eqnarray}
%
The idea behind the decomposition theorem is that the tensor and scalar sectors of (\ref{5.4}) satisfy (\ref{5.4}) independently, so that one can set 
%
\begin{eqnarray}
\nabla_{\alpha}\nabla^{\alpha}F_{\mu\nu}&=&-8 \pi G\nabla_{\alpha}\nabla^{\alpha}\bar{F}_{\mu\nu},
\nonumber\\
(g_{\mu\nu}\nabla_{\alpha}\nabla^{\alpha}-\nabla_{\mu}\nabla_{\nu})\chi&=&-8 \pi G(g_{\mu\nu}\nabla_{\alpha}\nabla^{\alpha}-\nabla_{\mu}\nabla_{\nu})\bar{\chi}. 
\label{5.5}
\end{eqnarray}
%
To see whether this is the case we take the trace of (\ref{5.4}), to obtain 
%
\begin{eqnarray}
\nabla_{\alpha}\nabla^{\alpha}\chi=-8 \pi G\nabla_{\alpha}\nabla^{\alpha}\bar{\chi}. 
\label{5.6}
\end{eqnarray}
%
If we now apply $\nabla_{\alpha}\nabla^{\alpha}$ to (\ref{5.4}), then given (\ref{5.6})  we obtain
%
\begin{eqnarray}
\nabla_{\alpha}\nabla^{\alpha}\nabla_{\beta}\nabla^{\beta}F_{\mu\nu}=-8\pi G \nabla_{\alpha}\nabla^{\alpha}\nabla_{\beta}\nabla^{\beta}\bar{F}_{\mu\nu}.
\label{5.7}
\end{eqnarray}
% 
Now while this does give us an equation that involves $F_{\mu\nu}$ alone, this equation is not the second-order derivative equation $\nabla_{\alpha}\nabla^{\alpha}F_{\mu\nu}=-8 \pi G\nabla_{\alpha}\nabla^{\alpha}\bar{F}_{\mu\nu}$ that one is looking for. Moreover, getting to (\ref{5.6}) and (\ref{5.7}) is initially as far as we can go, since according to  (\ref{3.10}) only $\nabla_{\alpha}\nabla^{\alpha}\nabla_{\beta}\nabla^{\beta}F_{\mu\nu}$ and $\nabla_{\alpha}\nabla^{\alpha}\chi$ are automatically gauge invariant.


Now initially (\ref{5.6})  does not imply that $\chi$ is necessarily equal to $-8\pi G\bar{\chi}$, since they could differ by any function $f$ that obeys $\nabla_{\alpha}\nabla^{\alpha}f=0$, i.e., by any harmonic function of the form $f(\mathbf{q}\cdot\mathbf{x}-q t)$. However, it is the very introduction of $\bar{\chi}$ that is causing $\chi$ to be non-zero in the first place, and thus $\chi$ must be proportional to $\bar{\chi}$. Hence harmonic functions can be ignored. Then with  $\chi=-8\pi G\bar{\chi}$, it follows from (\ref{5.4}) that $\nabla_{\alpha}\nabla^{\alpha}F_{\mu\nu}=-8 \pi G\nabla_{\alpha}\nabla^{\alpha}\bar{F}_{\mu\nu}$. And again, since it is the very introduction of $\bar{F}_{\mu\nu}$ that is causing $\delta G_{\mu\nu}$ to be non-zero in the first place, it must be the case that $F_{\mu\nu}=-8 \pi G\bar{F}_{\mu\nu}$. As we see, (\ref{5.5}) does hold, and thus for an external $\delta \bar{T}_{\mu\nu}$ perturbation to a flat background we  obtain the decomposition theorem.

However, in the absence of any explicit external $\delta \bar{T}_{\mu\nu}$ the discussion is different, and is only of relevance in those cases where there is a background $T_{\mu\nu}$, as otherwise $\delta T_{\mu\nu}$ would be zero. When the background $T_{\mu\nu}$ is non-zero and accordingly the background is not flat, the fluctuation quantity $\delta G_{\mu\nu}+8\pi G \delta T_{\mu\nu}$ can still only depend on six gauge-invariant SVT4 combinations, viz. the curved space generalizations of the above $F_{\mu\nu}$ and one combination of $\chi$, $F$ and $F_{\mu}$. Thus in the following we will explore the SVT4 formulation in some non-flat backgrounds that are of cosmological interest.
%%%%%%%%%%%%%%%%%%%%%%%%%%%%%%%%%%%%%%%%%%%%
\subsection{$dS_4$}
\label{ss:ds4_svt4}
%%%%%%%%%%%%%%%%%%%%%%%%%%%%%%%%%%%%%%%%%%%%

\subsubsection{SVT4 $dS_4$ Basis Without a Conformal Factor}
\label{sss:svt4_without_conformal_factor}

Since a background de Sitter metric can be written as a comoving coordinate system metric with no conformal prefactor [viz. $ds^2=dt^2-e^{2Ht}(dx^2+dy^2+dz^2)$],  or written with a conformal prefactor as  a conformal to flat Minkowski metric [$ds^2=(1/\tau H)^2(d\tau^2-dx^2-dy^2-dz^2)$ where $\tau=e^{-Ht}/H$],  in setting up the SVT4 description of fluctuations around a de Sitter background there are then two options. One is to define the fluctuations in terms of $\chi$, $F$, $F_{\mu}$ and $F_{\mu\nu}$ with no multiplying conformal prefactor so that
%
\begin{eqnarray}
h_{\mu\nu}=-2g_{\mu\nu}\chi+2\nabla_{\mu}\nabla_{\nu}F
+ \nabla_{\mu}F_{\nu}+\nabla_{\nu}F_{\mu}+2F_{\mu\nu},
\label{6.1}
\end{eqnarray}
%
with the $\nabla_{\mu}$ derivatives being fully covariant with respect to the de Sitter background so that $\nabla^{\mu}F_{\mu}=0$, $\nabla^{\nu}F_{\mu\nu}=0$. The second is to define the fluctuations with a conformal prefactor so that the fluctuation metric is written as conformal to a flat Minkowski metric according to
%
\begin{eqnarray}
h_{\mu\nu}=\frac{1}{(\tau H)^2}[-2\eta_{\mu\nu}\chi+2\tilde{\nabla}_{\mu}\tilde{\nabla}_{\nu}F
+ \tilde{\nabla}_{\mu}F_{\nu}+\tilde{\nabla}_{\nu}F_{\mu}+2F_{\mu\nu}],
\label{6.2}
\end{eqnarray}
%
with the $\tilde{\nabla}_{\mu}$ derivatives being with respect to flat Minkowski so that $\tilde{\nabla}^{\mu}F_{\mu}=0$, $\tilde{\nabla}^{\nu}F_{\mu\nu}=0$, i.e. $-\dot{F}_0+\tilde{\nabla}^jF_j=0$, $-\dot{F}_{00}+\tilde{\nabla}^jF_{0j}=0$, $-\dot{F}_{0i}+\tilde{\nabla}^jF_{ij}=0$. We shall discuss both options below starting with (\ref{6.1}). 

However, before doing so and in order to be as general as possible we shall initially work in $D$ dimensions where the de Sitter space Riemann tensor takes the form
%
\begin{eqnarray}
R_{\lambda\mu\nu\kappa}=H^2(g_{\mu\nu}g_{\lambda\kappa}-g_{\lambda\nu}g_{\mu\kappa}),
\quad R_{\mu\kappa}=H^2(1-D)g_{\mu\kappa},\quad R^{\alpha}_{\phantom{\alpha}\alpha}=H^2D(1-D).
\nonumber\\
\label{6.3}
\end{eqnarray}
% 
To construct fluctuations we have found it convenient to generalize (\ref{3.1}) to 
%
\begin{eqnarray}
h_{\mu\nu}&=&2F_{\mu\nu}+\nabla_{\nu}W_{\mu}+\nabla_{\mu}W_{\nu}+\frac{2-D}{D-1}\left[\nabla_{\mu}\nabla_{\nu}
+g_{\mu\nu}H^2\right]\times
\nonumber\\
&&\int d^Dy(-g)^{1/2}D^{(E)}(x,y)\nabla^{\alpha}W_{\alpha}
- \frac{g_{\mu\nu}}{D-1}(\nabla^{\alpha}W_{\alpha}-h)
\nonumber\\
&&-\frac{1}{D-1}\left[\nabla_{\mu}\nabla_{\nu}+g_{\mu\nu}H^2\right]\int d^Dy(-g)^{1/2}D^{(E)}(x,y)h,
\label{6.4}
\end{eqnarray}
%
where in the curved background the Green's function obeys
%
\begin{eqnarray}
\left(\nabla_{\nu}\nabla^{\nu}+H^2D\right)D^{(E)}(x,y)=(-g)^{-1/2}\delta^{(D)}(x-y).
\label{6.5}
\end{eqnarray}
%
With this definition $F_{\mu\nu}$ is automatically traceless. On applying $\nabla^{\nu}$ to (\ref{6.4}) and recalling that for any vector or scalar in a de Sitter space  we have
%
\begin{eqnarray}
(\nabla^{\nu}\nabla_{\mu}-\nabla_{\mu}\nabla^{\nu})W_{\nu}&=&H^2(D-1)W_{\mu},
\nonumber\\
(\nabla^{\nu}\nabla_{\mu}\nabla_{\nu}-\nabla_{\mu}\nabla^{\nu}\nabla_{\nu})V&=&H^2(D-1)\nabla_{\mu}V,
\label{6.6}
\end{eqnarray}
%
we obtain
%
\begin{eqnarray}
\nabla^{\nu}h_{\mu\nu}=\nabla_{\nu}\nabla^{\nu}W_{\mu}+H^2(D-1)W_{\mu},
\label{6.7}
\end{eqnarray}
%
with (\ref{6.7}) serving to define $W_{\mu}$. To decompose $W_{\mu}$ into transverse and longitudinal components we set $W_{\mu}=F_{\mu}+\nabla_{\mu}A$ where $\nabla^{\mu}F_{\mu}=0$, $\nabla^{\mu}W_{\mu}=\nabla^{\mu}\nabla_{\mu}A$, and thus set
%
\begin{eqnarray}
W_{\mu}=F_{\mu}+\nabla_{\mu}\int d^Dy(-g)^{1/2}D^{(D)}(x,y)\nabla^{\alpha}W_{\alpha},
\label{6.8}
\end{eqnarray}
%
where
%
\begin{eqnarray}
\nabla_{\nu}\nabla^{\nu}D^{(D)}(x,y)=(-g)^{-1/2}\delta^{(D)}(x-y).
\label{6.9}
\end{eqnarray}
%
Finally, with
%
\begin{eqnarray}
-2\chi&=&\frac{(2-D)H^2}{D-1}\int d^Dy(-g)^{1/2}D^{(E)}(x,y)\nabla^{\alpha}W_{\alpha}-\frac{\nabla^{\alpha}W_{\alpha}-h}{D-1}
\nonumber\\
&&-\frac{H^2}{D-1}\int d^Dy(-g)^{1/2}D^{(E)}(x,y)h,
\nonumber\\
2F&=&\frac{2-D}{D-1}\int d^Dy(-g)^{1/2}D^{(E)}(x,y)\nabla^{\alpha}W_{\alpha}
\nonumber\\
&&-\frac{1}{D-1}\int d^Dy(-g)^{1/2}D^{(E)}(x,y)h
+2\int d^Dy(-g)^{1/2}D^{(D)}(x,y)\nabla^{\alpha}W_{\alpha},
\nonumber\\
F_{\mu}&=&W_{\mu}-\nabla_{\mu}\int d^Dy(-g)^{1/2}D^{(D)}(x,y)\nabla^{\alpha}W_{\alpha},
\label{6.10}
\end{eqnarray}
%
we can now write $h_{\mu\nu}$ as given (\ref{6.1}), with $F_{\mu\nu}$ being given by the transverse-traceless
%
\begin{eqnarray}
2F_{\mu\nu}=h_{\mu\nu}+2g_{\mu\nu}\chi-2\nabla_{\mu}\nabla_{\nu}F
- \nabla_{\mu}F_{\nu}-\nabla_{\nu}F_{\mu},
\label{6.11}
\end{eqnarray}
%
where $W_{\mu}$ is determined from (\ref{6.7}). In this way then we can decompose $h_{\mu\nu}$ into a covariant SVTD in the de Sitter background case.

As well as the above formulation, which involves the Green's function $D^{(E)}(x,y)$, we should note that there is also an alternate formulation that does not involve it at all, one that implements the tracelessness of $F_{\mu\nu}$ using $D^{(D)}(x,y)$ alone, though it does so at the expense of leading to a more complicated expression for $W_{\mu}$. To this end we replace (\ref{6.4}) by 
%
\begin{eqnarray}
h_{\mu\nu}&=&2F_{\mu\nu}+\nabla_{\nu}W_{\mu}+\nabla_{\mu}W_{\nu}+\frac{2-D}{D-1}\nabla_{\mu}\nabla_{\nu}\int d^Dy(-g)^{1/2}D^{(D)}(x,y)\nabla^{\alpha}W_{\alpha}
\nonumber\\
&-&\frac{g_{\mu\nu}}{D-1}(\nabla^{\alpha}W_{\alpha}-h)-\frac{1}{D-1}\nabla_{\mu}\nabla_{\nu}\int d^Dy(-g)^{1/2}D^{(D)}(x,y)h,
\label{6.12}
\end{eqnarray}
%
with $F_{\mu\nu}$ automatically being traceless. To fix $W_{\mu}$ we evaluate 
%
\begin{eqnarray}
\nabla^{\nu}h_{\mu\nu}&=&\nabla_{\nu}\nabla^{\nu}W_{\mu}+H^2(D-1)W_{\mu}
+H^2(2-D)\times
\nonumber\\
&&\nabla_{\mu}\int d^Dy(-g)^{1/2}D^{(D)}(x,y)\nabla^{\alpha}W_{\alpha}
\nonumber\\
&&
-H^2\nabla_{\mu}\int d^Dy(-g)^{1/2}D^{(D)}(x,y)h,
\nonumber\\
\nabla^{\mu}\nabla^{\nu}h_{\mu\nu}&=&\nabla^{\mu}\nabla_{\nu}\nabla^{\nu}W_{\mu}+H^2(\nabla^{\nu}W_{\nu}-h).
\label{6.13}
\end{eqnarray}
%
In terms of (\ref{6.12}) and (\ref{6.8}) we can set 
%
\begin{eqnarray}
2\chi&=&\frac{1}{D-1}[\nabla^{\alpha}W_{\alpha}-h]
\nonumber\\
 2F&=&\frac{1}{D-1}\int d^Dy(-g)^{1/2}D^{(D)}(x,y)[D\nabla^{\alpha}W_{\alpha}-h],
\nonumber\\
F_{\mu}&=&W_{\mu}-\nabla_{\mu}\int d^Dy(-g)^{1/2}D^{(D)}(x,y)\nabla^{\alpha}W_{\alpha},
\nonumber\\
2F_{\mu\nu}&=&h_{\mu\nu}+2g_{\mu\nu}\chi-2\nabla_{\mu}\nabla_{\nu}F
- \nabla_{\mu}F_{\nu}-\nabla_{\nu}F_{\mu},
\label{6.14}
\end{eqnarray}
%
with $\nabla^{\mu}F_{\mu}=0$ as before, and with (\ref{6.1}) following. Thus either way we are led to (\ref{6.1}) and we now apply it to fluctuations around a background de Sitter geometry.

\subsubsection{Application of SVT4 to de Sitter Fluctuation Equations}
\label{sss:application_to_ds4_svt4}

We now restrict to  four dimensions where in a de Sitter geometry  the background Einstein equations are given by 
%
\begin{eqnarray}
G_{\mu\nu}=-8\pi G T_{\mu\nu}=3H^2g_{\mu\nu}.
\label{6.15}
\end{eqnarray}
% 
The fluctuating Einstein tensor is given by 
%
\begin{eqnarray}
\delta G_{\mu\nu}&=&\frac{1}{2}\left[\nabla_{\alpha}\nabla^{\alpha}h_{\mu\nu}-\nabla_{\nu}\nabla^{\alpha}h_{\alpha\mu}-\nabla_{\mu}\nabla^{\alpha}h_{\alpha\nu}+\nabla_{\mu}\nabla_{\nu}h\right]
\nonumber\\
&&
+\frac{g_{\mu\nu}}{2}\left[\nabla^{\alpha}\nabla^{\beta}h_{\alpha\beta}-\nabla_{\alpha}\nabla^{\alpha}h\right]
+\frac{H^2}{2}\left[4h_{\mu\nu}-g_{\mu\nu}h\right],
\label{6.16}
\end{eqnarray}
% 
while the perturbation in the background $T_{\mu\nu}$ is given by $\delta T_{\mu\nu}=-3H^2h_{\mu\nu}$ (we conveniently set $8\pi G=1$). If we now reexpress these fluctuations in  the SVT4 basis given in (\ref{6.1}) we obtain
%
\begin{eqnarray}
\delta G_{\mu\nu}&=&2g_{\mu\nu}\nabla_{\alpha}\nabla^{\alpha}\chi-2\nabla_{\mu}\nabla_{\nu}\chi
+6H^2\nabla_{\mu}\nabla_{\nu}F
+3H^2 \nabla_{\mu}F_{\nu}+3H^2\nabla_{\nu}F_{\mu}+
\nonumber\\
&&(\nabla_{\alpha}\nabla^{\alpha}+4H^2)F_{\mu\nu},
\label{6.17}
\end{eqnarray}
% 
%
\begin{eqnarray}
3H^2h_{\mu\nu}=3H^2\left[-2g_{\mu\nu}\chi+2\nabla_{\mu}\nabla_{\nu}F
+ \nabla_{\mu}F_{\nu}+\nabla_{\nu}F_{\mu}+2F_{\mu\nu}\right],
\label{6.18}
\end{eqnarray}
% 
and thus 
%
\begin{eqnarray}
\delta G_{\mu\nu}-3H^2h_{\mu\nu}&=&(\nabla_{\alpha}\nabla^{\alpha}-2H^2)F_{\mu\nu}
\nonumber\\
&&+2(g_{\mu\nu}\nabla_{\alpha}\nabla^{\alpha}-\nabla_{\mu}\nabla_{\nu}+3H^2g_{\mu\nu})\chi.
\label{6.19}
\end{eqnarray}
% 
As we see, $\delta G_{\mu\nu}+8\pi G\delta T_{\mu\nu}=\delta G_{\mu\nu}-3H^2h_{\mu\nu}$ only depends on $F_{\mu\nu}$ and $\chi$,  with it thus being these quantities that are  gauge invariant, with the thus non-gauge-invariant $F_{\mu}$ and $F$ dropping out.

In the event that there is an additional source term $\delta\bar{T}_{\mu\nu}$, it must be gauge invariant on its own, and must obey $\nabla^{\nu}\delta \bar{T}_{\mu\nu}=0$ in the de Sitter background, to thus be of the form 
%
\begin{eqnarray}
\delta \bar{T}_{\mu\nu}=\bar{F}_{\mu\nu}+2(g_{\mu\nu}\nabla_{\alpha}\nabla^{\alpha}-\nabla_{\mu}\nabla_{\nu}+3H^2g_{\mu\nu})\bar{\chi}.
\label{6.20}
\end{eqnarray}
% 
With this source the fluctuation equations take the form
%
\begin{eqnarray}
&&(\nabla_{\alpha}\nabla^{\alpha}-2H^2)F_{\mu\nu}+2(g_{\mu\nu}\nabla_{\alpha}\nabla^{\alpha}-\nabla_{\mu}\nabla_{\nu}+3H^2g_{\mu\nu})\chi=\bar{F}_{\mu\nu}
\nonumber\\
&&\qquad\qquad+2(g_{\mu\nu}\nabla_{\alpha}\nabla^{\alpha}-\nabla_{\mu}\nabla_{\nu}+3H^2g_{\mu\nu})\bar{\chi},
\label{6.21}
\end{eqnarray}
%
with trace
%
\begin{eqnarray}
6(\nabla_{\alpha}\nabla^{\alpha}+4H^2)\chi&=&6(\nabla_{\alpha}\nabla^{\alpha}+4H^2)\bar{\chi}.
\label{6.22}
\end{eqnarray}
% 

While the trace condition would only set $\chi=\bar{\chi}+f$ where $f$ obeys  $(\nabla_{\alpha}\nabla^{\alpha}+4H^2)f=0$, when there is a $\bar{\chi}$ source present then it is the cause of fluctuations in the background in the first place, and thus we can only have $\chi=\bar{\chi}$ with any possible $f$ being zero. Then, from  (\ref{6.21}) we obtain 
%
\begin{eqnarray}
(\nabla_{\alpha}\nabla^{\alpha}-2H^2)F_{\mu\nu}=\bar{F}_{\mu\nu},
\label{6.23}
\end{eqnarray}
%
and the decomposition theorem is achieved.

However, if there is no $\delta \bar{T}_{\mu\nu}$ source the fluctuation equations take the form 
%
\begin{eqnarray}
(\nabla_{\alpha}\nabla^{\alpha}-2H^2)F_{\mu\nu}+2(g_{\mu\nu}\nabla_{\alpha}\nabla^{\alpha}-\nabla_{\mu}\nabla_{\nu}+3H^2g_{\mu\nu})\chi=0,
\label{6.24}
\end{eqnarray}
%
with the trace condition  being given by 
%
\begin{eqnarray}
6(\nabla_{\alpha}\nabla^{\alpha}+4H^2)\chi=0,
\label{6.25a}
\end{eqnarray}
%
with spherical Bessel solution
%
\begin{eqnarray}
\chi=\sum_{\bf k} k^2\tau^2[a_2({\bf k})j_2(k\tau)+b_2 ({\bf k})y_2(k\tau)]e^{i{\bf k}\cdot {\bf x}}.
\label{6.26a}
\end{eqnarray}
%
(To obtain this solution for $\chi$ it is more straightforward to use $ds^2=(1/\tau^2 H^2)(d\tau^2-dx^2-dy^2-dz^2)$ as the background de Sitter metric, something we can do regardless of whether or not we include a conformal factor in the fluctuations.) Given the trace condition we can rewrite the evolution equation given in (\ref{6.24}) as 
%
\begin{eqnarray}
(\nabla_{\alpha}\nabla^{\alpha}-2H^2)F_{\mu\nu}-2(g_{\mu\nu}H^2+\nabla_{\mu}\nabla_{\nu})\chi=0.
\label{6.27a}
\end{eqnarray}
%


Since it is not automatic that $\chi$ would obey $(H^2g_{\mu\nu}+\nabla_{\mu}\nabla_{\nu})\chi=0$ even though it does obey  $g^{\mu\nu}(H^2g_{\mu\nu}+\nabla_{\mu}\nabla_{\nu})\chi=0$, it is thus not automatic that (\ref{6.24}) and (\ref{6.27a}) could be replaced by
%
\begin{eqnarray}
(\nabla_{\alpha}\nabla^{\alpha}-2H^2)F_{\mu\nu}=0,\quad (g_{\mu\nu}H^2+\nabla_{\mu}\nabla_{\nu})\chi=0,
\label{6.28a}
\end{eqnarray}
%
as would be required of a decomposition theorem. In fact, since $g_{\mu\nu}\chi$ and $\nabla_{\mu}\nabla_{\nu}\chi$ behave totally differently ($\nabla_{\mu}\nabla_{\nu}\chi$ is non-zero if $\mu\neq \nu$ while $g_{\mu\nu}\chi$ is not), the only way to get a decomposition theorem would be for $\chi$, and thus $a_2({\bf k})$ and $b_2({\bf k})$,  to be zero. As we now show, this can in fact be made to be the case, though it is only a particular solution to the full fluctuation equations.

To explore this possibility we need to obtain an expression that  only depends on $F_{\mu\nu}$, and we note that for any scalar in  $D=4$ de Sitter we  have \cite{mannheim_2012}
%
\begin{eqnarray}
\nabla_{\alpha}\nabla^{\alpha}\nabla_{\mu}\nabla_{\nu}\chi=\nabla_{\mu}\nabla_{\nu}\nabla_{\alpha}\nabla^{\alpha}\chi
-2H^2g_{\mu\nu}\nabla_{\alpha}\nabla^{\alpha}\chi
+8H^2\nabla_{\mu}\nabla_{\nu}\chi.
\label{6.29a}
\end{eqnarray}
% 
Given the trace condition shown in (\ref{6.25a})  we then find that
%
\begin{eqnarray}
(\nabla_{\alpha}\nabla^{\alpha}-4H^2)(g_{\mu\nu}H^2+\nabla_{\mu}\nabla_{\nu})\chi
=(\nabla_{\mu}\nabla_{\nu}-H^2g_{\mu\nu})(\nabla^{\alpha}\nabla^{\alpha}+4H^2)\chi=0,
\label{6.30a}
\end{eqnarray}
%
and from (\ref{6.27a}) we thus obtain the fourth-order derivative equation
%
\begin{eqnarray}
(\nabla_{\alpha}\nabla^{\alpha}-4H^2)(\nabla_{\alpha}\nabla^{\alpha}-2H^2)F_{\mu\nu}=0
\label{6.31a}
\end{eqnarray}
%
for $F_{\mu\nu}$, with a decomposition for the components of the fluctuations thus being found, only in the higher-derivative form given in (\ref{6.30a}) and (\ref{6.31a}) rather than in the second-derivative form given in (\ref{6.28a}). Now $(\nabla_{\alpha}\nabla^{\alpha}-2H^2)F_{\mu\nu}=0$ is a particular solution to (\ref{6.31a}), and for this particular solution it would follow that the only solution to (\ref{6.24}) would then be $\chi=0$, with both the $F_{\mu\nu}$ and $\chi$ sector equations given in  (\ref{6.28a}) then holding.

To determine the conditions under which $(\nabla_{\alpha}\nabla^{\alpha}-2H^2)F_{\mu\nu}=0$ might actually hold we need to look for the general solution to (\ref{6.31a}), and since (\ref{6.31a}) is a covariant equation we can evaluate it in any coordinate system, with conformal to flat Minkowski being the most convenient for the de Sitter background. To this end we recall that in any metric that is conformal to flat Minkowski ($ds^2=-g_{MN}dx^Mdx^N=-\Omega^2(x)\eta_{\mu\nu}dx^{\mu}dx^{\nu}$) one has the relation \cite{mannheim_2012}
%
\begin{eqnarray}
g^{LR}\nabla_{L}\nabla_{R}A_{MN}&=&
\eta^{LR}\Omega^{-2}\partial_{L}\partial_{R}A_{MN}
-2\Omega^{-4}\partial_{M}\Omega\partial_{N}\Omega \eta^{TQ}A_{TQ}
\nonumber\\
&&
-2\eta^{LR}\Omega^{-3}\partial_{L}\partial_{R}\Omega A_{MN}
-2\eta^{LR}\Omega^{-3}\partial_{R}\Omega \partial_{L}A_{MN}
\nonumber\\
&&
+2\Omega^{-4}\eta_{MN}\eta^{TX}\partial_{X}\Omega\eta^{QY} \partial_{Y}\Omega A_{TQ}
+2\eta^{KQ}\Omega^{-3}\partial_{Q}\Omega \partial_{N}A_{KM}
\nonumber\\
&&
+2\eta^{KQ}\Omega^{-3}\partial_{Q}\Omega \partial_{M}A_{KN}
-2\Omega^{-1}\partial_{N}\Omega \nabla_{L}A^{L}_{\phantom{L}M}
\nonumber\\
&&
-2\Omega^{-1}\partial_{M}\Omega \nabla_{L}A^{L}_{\phantom{L}N},
\label{6.32a}
\end{eqnarray}
%
for any rank two tensor $A_{MN}$, with the $\nabla_{L}$ referring to covariant derivatives in the $g_{MN}$ geometry. For an $A_{MN}$ that is transverse and traceless, and for $\Omega=1/\tau H$ (\ref{6.32a}) reduces to  (the dot denotes $\partial/\partial\tau$)
%
\begin{eqnarray}
g^{LR}\nabla_{L}\nabla_{R}A_{MN}&=&
\eta^{LR}\tau^2 H^2\partial_{L}\partial_{R}A_{MN}
+4H^2A_{MN}
-2\tau H^2\dot{A}_{MN}
+2H^2\eta_{MN}A_{00}
\nonumber \\
&+&2\tau H^2\partial_{N}A_{0M}
+2\tau H^2 \partial_{M}A_{0N}.
\label{6.33a}
\end{eqnarray}
%
While the general components of $A_{MN}$ are coupled in (\ref{6.33a}), this is not the case for  $A_{00}$, and so we look at $A_{00}$ and obtain 
%
\begin{eqnarray}
\nabla_{L}\nabla^{L}A_{00}&=&
\eta^{LR}\tau^2 H^2\partial_{L}\partial_{R}A_{00}
+2H^2A_{00}
+2\tau H^2\dot{A}_{00}.
\label{6.34a}
\end{eqnarray}
%
Now in a de Sitter background the identity 
%
\begin{eqnarray}
\nabla_{P}\nabla_{K}\nabla^{K}A^{P}_{\phantom{P}M}
=[\nabla_{K}\nabla^{K}+5H^2]\nabla_{P}A^{P}_{\phantom{P}M}
-2H^2\nabla_{M}A^{P}_{\phantom{P}P}
\end{eqnarray}
%
holds  \cite{mannheim_2012}. Thus if any $A_{MN}$ is transverse and traceless then so is $\nabla_{L}\nabla^{L}A_{MN}$. So let us define $A_{MN}=[\nabla_{L}\nabla^{L}-2H^2]F_{MN}$, with this $A_{MN}$ being traverse and traceless since $F_{MN}$ is. For this $A_{MN}$ (\ref{6.31a}) takes the form
%
\begin{eqnarray}
(\nabla_{\alpha}\nabla^{\alpha}-4H^2)A_{\mu\nu}=0.
\label{6.35a}
\end{eqnarray}
%
Thus for $A_{00}$ we have
%
\begin{eqnarray}
\eta^{LR}\tau^2 H^2\partial_{L}\partial_{R}A_{00}+2\tau H^2\dot{A}_{00}
-2H^2A_{00}=0.
\label{6.36a}
\end{eqnarray}
% 
In a plane wave mode $e^{i{\bf k}\cdot{\bf x}}$ the quantity $A_{00}$ thus obeys
%
\begin{eqnarray}
\ddot{A}_{00}-\frac{2}{\tau}\dot{A}_{00}+k^2A_{00}+\frac{2}{\tau^2}A_{00}=0.
\label{6.37a}
\end{eqnarray}
% 
The general solution to (\ref{6.36a}) is thus
%
\begin{eqnarray}
A_{00}&=&[\nabla_{L}\nabla^{L}-2H^2]F_{00}=\sum_{\bf k} k^4\tau^2[a_{00}({\bf k})j_0(k\tau)+b_{00} ({\bf k})y_0(k\tau)]e^{i{\bf k}\cdot {\bf x}}
\nonumber\\
&=&\sum_{\bf k} k^3\tau[a_{00}({\bf k})\sin(k\tau)+b_{00} ({\bf k})\cos(k\tau)]e^{i{\bf k}\cdot {\bf x}},
\label{6.38a}
\end{eqnarray}
% 
where $a_{00}({\bf k})$ and $b_{00}({\bf k})$ are polarization tensors. (Here and throughout we leave out the complex conjugate solution.)

To see if we can support this solution, or whether we are forced to (\ref{6.28a}), we need to see whether (\ref{6.38a}) is compatible  with  (\ref{6.27a}), and thus require that
%
\begin{eqnarray}
A_{00}&=&(\nabla_{\alpha}\nabla^{\alpha}-2H^2)F_{00}=2(g_{00}H^2+\nabla_{0}\nabla_{0})\chi
\nonumber\\
&=&2\left[-\frac{1}{\tau^2}+\frac{\partial^2}{\partial \tau^2}-\Gamma^{\alpha}_{00}\partial_{\alpha}\right]\chi,
\label{6.39}
\end{eqnarray}
%
when evaluated in the solution  for $\chi$ as given in (\ref{6.26a}). On noting that $\Gamma^{\alpha}_{00}=-\delta^{\alpha}_0/\tau$ in a background de Sitter geometry, we evaluate
%
\begin{eqnarray}
2\left[-\frac{1}{\tau^2}+\frac{\partial^2}{\partial \tau^2}-\Gamma^{\alpha}_{00}\partial_{\alpha}\right][k^2\tau^2j_2(k\tau)]&=&2\left[\frac{\partial^2}{\partial \tau^2}+\frac{1}{\tau}\frac{\partial}{\partial \tau}-\frac{1}{\tau^2}\right][k^2\tau^2j_2(k\tau)]
\nonumber\\
&=&2k^3\tau\sin(k\tau),
\label{6.40}
\end{eqnarray}
%
where we have utilized properties of Bessel functions in the last step. With an analogous expression holding for the $y_2(k\tau)$ term, we thus precisely do confirm (\ref{6.38a}), and on comparing (\ref{6.26a}) with (\ref{6.38a}) obtain
%
\begin{eqnarray}
a_{00}({\bf k})=2a_2({\bf k}), \quad b_{00}({\bf k})=2b_2({\bf k}).
\label{6.41}
\end{eqnarray}
%
As we see, in the general solution we are not at all forced to $\chi=0$ as would be required by the decomposition theorem.

For completeness we note that once we have determined $A_{00}$ we can use (\ref{6.33a}) and the $\nabla_{L}A^{L}_{\phantom{L}M}=0$ and $g^{MN}A_{MN}=0$ conditions to determine the other components of $A_{\mu\nu}$, and note only that they satisfy and behave as  
%
\begin{eqnarray}
&&
\eta^{\mu\nu}\partial_{\mu}A_{0\nu}+\frac{2}{\tau}A_{00}=0,\quad \eta^{\mu\nu}\partial_{\mu}A_{i\nu}+\frac{2}{\tau}A_{0i}=0,
\nonumber\\
&&\left[\frac{\partial^2}{\partial\tau^2}+k^2\right]A_{0i}=\frac{2}{\tau}\partial_iA_{00},
\nonumber\\
&&\left[\frac{\partial^2}{\partial\tau^2}+\frac{2}{\tau}\frac{\partial}{\partial \tau}+k^2\right]A_{ij} =\frac{2}{\tau^2}\delta_{ij}A_{00}+\frac{2}{\tau}\left(\partial_iA_{0j}+\partial_jA_{0i}\right),
\nonumber\\
&&A_{0i}=\sum_{\bf k} ik_ik[-k\tau a_{00}({\bf k})\cos(k\tau)+k\tau b_{00} ({\bf k})\sin(k\tau)
+a_{00}({\bf k})\sin(k\tau)
\nonumber\\
&&\qquad+b_{00} ({\bf k})\cos(k\tau)]e^{i{\bf k}\cdot {\bf x}}, 
\nonumber\\
&&A_{ij}=\sum_{\bf k}k_ik_jk\tau[a_{00}({\bf k})\sin(k\tau)+b_{00} ({\bf k})\cos(k\tau)]e^{i{\bf k}\cdot {\bf x}}
\nonumber\\
&&\qquad+\sum_{\bf k}\bigg[\delta_{ij}k^2-3k_ik_j\bigg]\bigg[-a_{00} ({\bf k})\cos(k\tau)+b_{00}({\bf k})\sin(k\tau)
\nonumber\\
&&\qquad+\frac{1}{k\tau}[a_{00} ({\bf k})\sin(k\tau)+b_{00}({\bf k})\cos(k\tau)
\bigg]e^{i{\bf k}\cdot {\bf x}}.
\label{6.42}
\end{eqnarray}
%
(In order to derive the solutions given in (\ref{6.42}) we needed to include terms that would vanish identically in the left-hand sides of the second-order differential equations so that the first-order $\nabla_{L}A^{L}_{\phantom{L}M}=0$ conditions would then be satisfied.) In this solution we then need to satisfy 
%
\begin{eqnarray}
(\nabla_{\alpha}\nabla^{\alpha}-2H^2)F_{\mu\nu}=A_{\mu\nu},
\label{6.43}
\end{eqnarray}
%
which for the representative $A_{00}$ and $F_{00}$ is of the form
%
\begin{eqnarray}
\ddot{F}_{00}-\frac{2}{\tau}\dot{F}_{00}+k^2F_{00}=-\sum_{\bf k}\frac{k^3}{H^2\tau} [a_{00}({\bf k})\sin(k\tau)+b_{00} ({\bf k})\cos(k\tau)]e^{i{\bf k}\cdot {\bf x}},
\label{6.44}
\end{eqnarray}
% 
with solution
%
\begin{eqnarray}
F_{00}=\sum_{\bf k}\frac{k^2}{2H^2} [-a_{00}({\bf k})\cos(k\tau)+b_{00} ({\bf k})\sin(k\tau)]e^{i{\bf k}\cdot {\bf x}}.
\label{6.45}
\end{eqnarray}
% 


Now in order to get a decomposition theorem in the form given in (\ref{6.28a}) we would need $\chi$ to vanish, i.e. we would need $a_2({\bf k})$ and $b_2({\bf k})$ to vanish. And that would mean that $a_{00}({\bf k})$ and $b_{00}({\bf k})$ would have to vanish as well, and thus not only would $A_{00}$ have to vanish but so would all the other components of $A_{\mu\nu}$ as well. A decomposition theorem would thus require that
%
\begin{eqnarray}
(\nabla_{\alpha}\nabla^{\alpha}-2H^2)F_{\mu\nu}=0,
\label{6.46}
\end{eqnarray}
%
for all components of $F_{\mu\nu}$. To look for a non-trivial solution to (\ref{6.46}) in order to show that the decomposition theorem does in fact have a solution, we note that in a plane wave (\ref{6.46}) reduces to 
%
\begin{eqnarray}
\ddot{F}_{00}-\frac{2}{\tau}\dot{F}_{00}+k^2F_{00}=0,
\label{6.47}
\end{eqnarray}
%
for the representative $F_{00}$ component. The non-trivial solution to (\ref{6.47}) is of the form 
%
\begin{eqnarray}
F_{00}=\sum_{\bf k} k^2\tau^2[c_{00}({\bf k})j_1(k\tau)+d_{00} ({\bf k})y_1(k\tau)]e^{i{\bf k}\cdot {\bf x}}.
\label{6.48}
\end{eqnarray}
% 
The form for $F_{00}$ given in (\ref{6.48}) and its $F_{\mu\nu}$ analogs together with $\chi=0$ thus constitute a non-trivial solution that corresponds to the decomposition theorem, so in this sense the decomposition theorem can be recovered, as it is a specific solution to the full evolution equations. However, there is no compelling reason to restrict the solutions to (\ref{6.35a}) to the trivial $A_{\mu\nu}=0$, with it being (\ref{6.38a}), (\ref{6.42}), (\ref{6.45}) and (\ref{6.48}) that provide the most general solution in the $F_{00}$ sector and its analogs according to 
%
\begin{eqnarray}
F_{00}&=&\sum_{\bf k}\frac{k^2}{2H^2} [-a_{00}({\bf k})\cos(k\tau)+b_{00} ({\bf k})\sin(k\tau)]e^{i{\bf k}\cdot {\bf x}}
\nonumber\\
&&+\sum_{\bf k} k^2\tau^2[c_{00}({\bf k})j_1(k\tau)+d_{00} ({\bf k})y_1(k\tau)]e^{i{\bf k}\cdot {\bf x}},
\label{6.49}
\end{eqnarray}
% 
while at the same time (\ref{6.26a}) is the most general solution in the $\chi$ sector as constrained by (\ref{6.41}). Moreover, in this solution we can choose the coefficients in (\ref{6.41}) so that $F_{\mu\nu}$ and $\chi$ are localized in space. Thus no spatially asymptotic boundary coefficient could affect them. In fact suppose that we could have constrained the solutions by an asymptotic condition. We would need one that would force $A_{\mu\nu}$ to have to vanish in $(\nabla_{\alpha}\nabla^{\alpha}-4H^2)A_{\mu\nu}=0$ while not at the same time forcing $F_{\mu\nu}$ to have to vanish in $(\nabla_{\alpha}\nabla^{\alpha}-2H^2)F_{\mu\nu}=0$, something that would not obviously appear possible to achieve. Thus as we see, in this general solution the decomposition theorem does not hold. And just as we noted in Sec. \ref{ss:decomp_svt4_basis}, in the SVT4 case asymptotic boundary conditions do not force us to the decomposition theorem, to thus provide a completely solvable cosmological model in which the decomposition theorem does not hold. However, we should point out that while we could not make $\chi$ vanish through spatial boundary conditions it would be possible to force $\chi$ to vanish at all times by judiciously choosing initial  conditions at an initial time.  However, there would not appear to be any compelling rationale for doing so, and  to nonetheless do so would appear to be quite contrived. Thus absent any compelling rationale for such a judicious choice or for any other choice at all for that matter (i.e., no compelling rationale that would force $\chi$ to vanish) the decomposition theorem would not hold for SVT4 fluctuations around a de Sitter background.


\subsubsection{Defining the SVT4 Fluctuations With a Conformal Factor}
\label{sss:defining_svt4_without_conformal_factor}

In a 
%
\begin{align}
ds^2=\frac{1}{(\tau H)^2}(d\tau^2-dx^2-dy^2-dz^2)
\label{6.50}
\end{align}
%
de Sitter background with fluctuations of the form 
%
\begin{align}
h_{\mu\nu}=\frac{1}{(\tau H)^2}(-2g_{\mu\nu}\chi+2\tilde{\nabla}_{\mu}\tilde{\nabla}_{\nu}F
+ \tilde{\nabla}_{\mu}F_{\nu}+\tilde{\nabla}_{\nu}F_{\mu}+2F_{\mu\nu}), 
\label{6.51}
\end{align}
%
where $\tilde{\nabla}^{\mu}F_{\mu}=0$, $\tilde{\nabla}^{\nu}F_{\mu\nu}=0$, $g^{\mu\nu}F_{\mu\nu}=0$, and where, as per (\ref{6.2}), the $\tilde{\nabla}_{\mu}$ denote derivatives with respect to the flat Minkowski $\eta_{\mu\nu}dx^{\mu}dx^{\nu}$ metric, we write the fluctuation Einstein tensor as 
%
\begin{eqnarray}
\delta G_{00}&=& -6 \dot{\chi} \tau^{-1} - 2 \tau^{-1} \tilde{\nabla}^2\dot{F} - 2 \tilde{\nabla}^2\chi -2 \tau^{-1} \tilde{\nabla}^2F_{0}- \overset{..}{F}_{00} - 2 \dot{F}_{00} \tau^{-1} + \tilde{\nabla}^2F_{00},
\nonumber\\ 
\delta G_{0i}&=& -2 \tau^{-1} \tilde{\nabla}_{i}\overset{..}{F} + 6 \tau^{-2} \tilde{\nabla}_{i}\dot{F} - 2 \tilde{\nabla}_{i}\dot{\chi} - 2 \tau^{-1} \tilde{\nabla}_{i}\chi +3 \dot{F}_{i} \tau^{-2} 
\nonumber\\
&&- 2 \tau^{-1} \tilde{\nabla}_{i}\dot{F}_{0} + 3 \tau^{-2} \tilde{\nabla}_{i}F_{0}- \overset{..}{F}_{0i}
 + 6 F_{0i} \tau^{-2} +  \tilde{\nabla}^2F_{0i} - 2 \tau^{-1} \tilde{\nabla}_{i}F_{00},
\nonumber\\ 
\delta G_{ij}&=& -2 \overset{..}{\chi}\delta_{ij} + 6 \overset{..}{F}\delta_{ij} \tau^{-2} - 2 \overset{...}{F}\delta_{ij} \tau^{-1} + 2 \dot{\chi}\delta_{ij} \tau^{-1} + 2\delta_{ij} \tau^{-1} \tilde{\nabla}^2\dot{F} 
\nonumber\\
&&+ 2\delta_{ij} \tilde{\nabla}^2\chi - 2 \tau^{-1} \tilde{\nabla}_{j}\tilde{\nabla}_{i}\dot{F}
 + 6 \tau^{-2} \tilde{\nabla}_{j}\tilde{\nabla}_{i}F - 2 \tilde{\nabla}_{j}\tilde{\nabla}_{i}\chi +6 \dot{F}_{0}\delta_{ij} \tau^{-2} 
 \nonumber\\
 &&- 2 \overset{..}{F}_{0}\delta_{ij} \tau^{-1} + 2\delta_{ij} \tau^{-1} \tilde{\nabla}^2F_{0} + 3 \tau^{-2} \tilde{\nabla}_{i}F_{j} 
 + 3 \tau^{-2} \tilde{\nabla}_{j}F_{i} - 2 \tau^{-1} \tilde{\nabla}_{j}\tilde{\nabla}_{i}F_{0}
 \nonumber\\
 &&- \overset{..}{F}_{ij} + 6 F_{ij} \tau^{-2} + 6 F_{00}\delta_{ij} \tau^{-2} +2 \dot{F}_{ij} \tau^{-1} +\tilde{\nabla}^2F_{ij}
- 2 \tau^{-1} \tilde{\nabla}_{i}F_{0j} 
\nonumber\\
&&- 2 \tau^{-1} \tilde{\nabla}_{j}F_{0i},
\nonumber\\
g^{\mu\nu}\delta G_{\mu\nu} &=& 18 H^2 \overset{..}{F} - 6 H^2 \overset{...}{F} \tau + 12 H^2 \dot{\chi} \tau - 6 H^2 \overset{..}{\chi} \tau^2 + 6 H^2 \tau \tilde{\nabla}^2\dot{F} + 6 H^2 \tilde{\nabla}^2F 
\nonumber \\ 
&& + 6 H^2 \tau^2 \tilde{\nabla}^2\chi +24 H^2 \dot{F}_{0} - 6 H^2 \overset{..}{F}_{0} \tau + 6 H^2 \tau \tilde{\nabla}_{k}^2F_{0}+24 H^2 F_{00}.
\label{6.52}
\end{eqnarray}
%
Here the dot denotes the derivative with respect to the conformal time $\tau$ and $\tilde{\nabla}^2=\delta^{ij}\tilde{\nabla}_i\tilde{\nabla}_j$. With a $3H^2h_{\mu\nu}$ perturbation  the fluctuation equations take the form
%
\begin{eqnarray}
\Delta_{00}&=& -6 \overset{..}{F} \tau^{-2} - 6 \dot{\chi} \tau^{-1} - 6 \tau^{-2} \chi - 2 \tau^{-1} \tilde{\nabla}^2\dot{F} - 2 \tilde{\nabla}^2\chi -6 \dot{F}_{0} \tau^{-2} 
\nonumber\\
&&- 2 \tau^{-1} \tilde{\nabla}^2F_{0}
- \overset{..}{F}_{00} 
- 6 F_{00} \tau^{-2} - 2 \dot{F}_{00} \tau^{-1} + \tilde{\nabla}^2F_{00}=0,
\nonumber\\ 
\Delta_{0i}&=& -2 \tau^{-1} \tilde{\nabla}_{i}\overset{..}{F} - 2 \tilde{\nabla}_{i}\dot{\chi} - 2 \tau^{-1} \tilde{\nabla}_{i}\chi -2 \tau^{-1} \tilde{\nabla}_{i}\dot{F}_{0}- \overset{..}{F}_{0i} + \tilde{\nabla}^2F_{0i} 
\nonumber\\
&&- 2 \tau^{-1} \tilde{\nabla}_{i}F_{00}=0,
\nonumber\\ 
\Delta_{ij}&=& -2 \overset{..}{\chi}\delta_{ij} + 6 \overset{..}{F}\delta_{ij} \tau^{-2} - 2 \overset{...}{F}\delta_{ij} \tau^{-1} + 2 \dot{\chi}\delta_{ij} \tau^{-1} + 6\delta_{ij} \tau^{-2} \chi + 2\delta_{ij} \tau^{-1} \tilde{\nabla}^2\dot{F} 
\nonumber\\
&&+ 2\delta_{ij} \tilde{\nabla}^2\chi 
 - 2 \tau^{-1} \tilde{\nabla}_{j}\tilde{\nabla}_{i}\dot{F} - 2 \tilde{\nabla}_{j}\tilde{\nabla}_{i}\chi +6 \dot{F}_{0}\delta_{ij} \tau^{-2} - 2 \overset{..}{F}_{0}\delta_{ij} \tau^{-1} 
 \nonumber\\
 &&+ 2\delta_{ij} \tau^{-1} \tilde{\nabla}^2F_{0}  - 2 \tau^{-1} \tilde{\nabla}_{j}\tilde{\nabla}_{i}F_{0}- \overset{..}{F}_{ij} + 6 F_{00}\delta_{ij} \tau^{-2} + 2 \dot{F}_{ij} \tau^{-1} + \tilde{\nabla}^2F_{ij}
 \nonumber\\
 && -2 \tau^{-1} \tilde{\nabla}_{i}F_{0j} - 2 \tau^{-1} \tilde{\nabla}_{j}F_{0i}=0,
\nonumber\\
g^{\mu\nu}\Delta_{\mu\nu} &=& 24 H^2 \overset{..}{F} - 6 H^2 \overset{...}{F} \tau + 12 H^2 \dot{\chi} \tau - 6 H^2 \overset{..}{\chi} \tau^2 + 24 H^2 \chi + 6 H^2 \tau \tilde{\nabla}^2\dot{F} 
\nonumber \\ 
&& + 6 H^2 \tau^2 \tilde{\nabla}^2\chi +24 H^2 \dot{F}_{0} - 6 H^2 \overset{..}{F}_{0} \tau + 6 H^2 \tau \tilde{\nabla}^2F_{0}+24 H^2 F_{00}=0,
\nonumber\\
\label{6.53}
\end{eqnarray}
%
where $\Delta_{\mu\nu}=\delta G_{\mu\nu}+8\pi G \delta T_{\mu\nu}$.
On introducing $\alpha=\dot{F}+\tau\chi+F_0$ the perturbative equations simplify to 
%
\begin{eqnarray}
\Delta_{00}&=& -6 \dot{\alpha} \tau^{-2} - 2 \tau^{-1} \tilde{\nabla}^2\alpha - \overset{..}{F}_{00}  - 6 F_{00} \tau^{-2} - 2 \dot{F}_{00} \tau^{-1} + \tilde{\nabla}^2F_{00}=0,
\nonumber\\ 
\Delta_{0i}&=& -2 \tau^{-1} \tilde{\nabla}_{i}\dot{\alpha}- \overset{..}{F}_{0i} +  \tilde{\nabla}^2F_{0i} - 2 \tau^{-1} \tilde{\nabla}_{i}F_{00}=0,
\nonumber\\ 
\Delta_{ij}&=&\delta_{ij} \left[- 2 \ddot{\alpha}\tau^{-1}+  6 \dot{\alpha} \tau^{-2}  + 2\tau^{-1} \tilde{\nabla}^2\alpha + 6 F_{00} \tau^{-2}\right]-2\tau^{-1} \tilde{\nabla}_{i}\tilde{\nabla}_{j}\alpha
\nonumber\\
&& - \overset{..}{F}_{ij}  + 2 \dot{F}_{ij} \tau^{-1} + \tilde{\nabla}^2F_{ij} -2 \tau^{-1} \tilde{\nabla}_{i}F_{0j} - 2 \tau^{-1} \tilde{\nabla}_{j}F_{0i}=0,
\nonumber\\
H^{-2}g^{\mu\nu}\Delta_{\mu\nu} &=& 24\dot{\alpha} - 6  \overset{..}{\alpha} \tau + 6  \tau \tilde{\nabla}^2\alpha +24 F_{00}=0.
\label{6.54}
\end{eqnarray}
%
We thus see that $\alpha$ and $F_{\mu\nu}$ are gauge invariant for a total of six (one plus five) gauge-invariant components, just as needed. (In passing we note that the gauge invariant  $\alpha=\dot{F}+\tau\chi+F_0$ actually mixes scalars and vectors.)

While we have written $\Delta_{\mu\nu}$ in the non-manifestly covariant form given (\ref{6.54}) as this will be convenient for actually solving $\Delta_{\mu\nu}=0$ below, since the SVT4 approach is covariant we are able to write the rank two  tensor $\Delta_{\mu\nu}$ in a manifestly covariant form. To do so we introduce a unit  timelike four-vector $U^{\mu}$ whose only non-zero component is $U^{0}$. In terms of this $U^{\mu}$ the gauge-invariant $\alpha$ is now given by the manifestly general coordinate scalar $\alpha=U^{\mu}\partial_{\mu}F+\chi/H\Omega+U^{\mu}F_{\mu}$, while the $F_{00}$ term in $g^{\mu\nu}\Delta_{\mu\nu}$ can be written as $U^{\mu}U^{\nu}F_{\mu\nu}$.


If there is to be a decomposition theorem then (\ref{6.54}) would have to break up into 
%
\begin{align}
-6 \dot{\alpha} \tau^{-2} - 2 \tau^{-1} \tilde{\nabla}^2\alpha=0, \quad - \overset{..}{F}_{00}  - 6 F_{00} \tau^{-2} - 2 \dot{F}_{00} \tau^{-1} + \tilde{\nabla}^2F_{00}&=0,
\nonumber\\ 
-2 \tau^{-1} \tilde{\nabla}_{i}\dot{\alpha}=0,\quad - \overset{..}{F}_{0i} +  \tilde{\nabla}^2F_{0i} - 2 \tau^{-1} \tilde{\nabla}_{i}F_{00}&=0,
\nonumber\\ 
\delta_{ij} \left[- 2 \ddot{\alpha}\tau^{-1}+  6 \dot{\alpha} \tau^{-2}  + 2\tau^{-1} \tilde{\nabla}^2\alpha \right]-2\tau^{-1} \tilde{\nabla}_{i}\tilde{\nabla}_{j}\alpha&=0,
\nonumber\\
6 \delta_{ij}F_{00} \tau^{-2} - \overset{..}{F}_{ij}  + 2 \dot{F}_{ij} \tau^{-1} + \tilde{\nabla}^2F_{ij} - 2 \tau^{-1} \tilde{\nabla}_{i}F_{0j} - 2 \tau^{-1} \tilde{\nabla}_{j}F_{0i}&=0,
\nonumber\\
24\dot{\alpha} - 6  \overset{..}{\alpha} \tau + 6  \tau \tilde{\nabla}^2\alpha=0,\quad 24 F_{00}&=0,
\label{6.55}
\end{align}
%
to then yield
%
\begin{eqnarray}
&& \dot{\alpha}=0,\quad \tilde{\nabla}_{i}\tilde{\nabla}_{j}\alpha=0,\quad \tilde{\nabla}^2\alpha=0,\quad F_{00}=0,\quad  - \ddot{F}_{0i} +  \tilde{\nabla}^2F_{0i} =0,
\nonumber\\ 
&&  - \overset{..}{F}_{ij}  + 2 \dot{F}_{ij} \tau^{-1} + \tilde{\nabla}^2F_{ij} - 2 \tau^{-1} \tilde{\nabla}_{i}F_{0j} - 2 \tau^{-1} \tilde{\nabla}_{j}F_{0i}=0,
\label{6.56}
\end{eqnarray}
%
with the $\epsilon^{ijk}\tilde{\nabla}_j\Delta_{0k}=0$ condition not being needed as it is satisfied identically. The solution to (\ref{6.56}) is the form 
%
\begin{eqnarray}
\nonumber\\ 
&& \alpha=0,\quad F_{00}=0,\quad F_{0i}=\sum _{\bf k}f_{0i}({\bf k})e^{i{\bf k}\cdot {\bf x}-ik\tau},\quad  ik^jf_{0j}({\bf k})=0,
\nonumber\\
&& F_{ij}=\sum _{\bf k}[f_{ij}({\bf k})+\tau\hat{f}_{ij}({\bf k})]e^{i{\bf k}\cdot {\bf x}-ik\tau},
\nonumber\\
&&-ikf_{ij}({\bf k})+\hat{f}_{ij}({\bf k})=ik_jf_{0i}({\bf k})+ik_if_{0j}({\bf k}),
\\
&&\delta^{ij}f_{ij}({\bf k})=0,\quad
\delta^{ij}\hat{f}_{ij}({\bf k})=0,\quad ik^jf_{ij}({\bf k})=-ikf_{0i}({\bf k}),\quad ik^j\hat{f}_{ij}({\bf k})=0,
\nonumber
\label{6.57}
\end{eqnarray}
%
and while the most general solution for  $\alpha$ would be  a constant,  we have imposed an asymptotic spatial boundary condition, which sets the constant to zero.


We now solve the full (\ref{6.54}) exactly  to determine whether and under what conditions (\ref{6.57}) might hold. Eliminating $F_{00}$ between the $\Delta_{00}=0$ and $g^{\mu\nu}\Delta_{\mu\nu}=0$ equations in (\ref{6.54}) yields
%
\begin{eqnarray}
-\frac{\tau}{4}\left(\frac{\partial ^2}{\partial \tau^2}-\tilde{\nabla}^2\right)\left(\frac{\partial ^2}{\partial \tau^2}-\tilde{\nabla}^2\right)\alpha=0,
\label{6.58}
\end{eqnarray}
%
with general solution
%
\begin{eqnarray}
\alpha=\sum_{\bf k}\left(a_{\bf k}+\tau b_{\bf k}\right)e^{i{\bf k}\cdot {\bf x}-ik\tau},
\label{6.59}
\end{eqnarray}
%
where $a_{\bf k}$ and $b_{\bf k}$ are independent of ${\bf x}$ and $\tau$. Given $\alpha$, $F_{00}$ then evaluates to 
%
\begin{eqnarray}
F_{00}=\sum_{\bf k}\left[a_{00}({\bf k})+\tau b_{00}({\bf k})\right]e^{i{\bf k}\cdot {\bf x}-ik\tau},
\label{6.60}
\end{eqnarray}
%
where
%
\begin{eqnarray}
a_{00}({\bf k})=ika_{\bf k}-b_{\bf k},\quad b_{00}({\bf k})=\frac{ik}{2}b_{\bf k}.
\label{6.61}
\end{eqnarray}
%
Inserting these solutions for $\alpha$ and $F_{00}$ into $\Delta_{0i}=0$ then yields
%
\begin{eqnarray}
\ddot{F}_{0i}-\tilde{\nabla}^2F_{0i}=-\sum_{\bf k}kk_ib_{\bf k}e^{i{\bf k}\cdot {\bf x}-ik\tau},
\label{6.62}
\end{eqnarray}
%
with solution 
%
\begin{eqnarray}
F_{0i}=\sum_{\bf k}\left[a_{0i}({\bf k})+\tau b_{0i}({\bf k})\right]e^{i{\bf k}\cdot {\bf x}-ik\tau},
\label{6.63}
\end{eqnarray}
%
where
%
\begin{eqnarray}
b_{0i}({\bf k})=-\frac{ik_i}{2}b_{\bf k}.
\label{6.64}
\end{eqnarray}
%
With $F_{0i}$ obeying the transverse condition $\partial^iF_{0i}-\dot{F}_{00}=0$, we obtain 
%
\begin{eqnarray}
ik^ia_{0i}({\bf k})=k^2a_{\bf k}+\frac{3ik}{2}b_{\bf k}.
\label{6.65}
\end{eqnarray}
%
Finally, from $\Delta_{ij}=0$ we obtain 
%
\begin{eqnarray}
\ddot{F}_{ij}  - \frac{2}{\tau} \dot{F}_{ij} - \tilde{\nabla}^2F_{ij}=
\sum_{\bf k}\left[\delta_{ij}ikb_{\bf k}+2k_ik_ja_{\bf k}-2ik_ia_{0j}-2ik_ja_{0i}\right]\frac{1}{\tau}e^{i{\bf k}\cdot {\bf x}-ik\tau}.
\nonumber\\
\label{6.66}
\end{eqnarray}
%
We can thus set 
%
\begin{eqnarray}
F_{ij}=\sum_{\bf k}\left[a_{ij}({\bf k})+\tau b_{ij}({\bf k})\right]e^{i{\bf k}\cdot {\bf x}-ik\tau},
\label{6.67}
\end{eqnarray}
%
where 
%
\begin{eqnarray}
2ika_{ij}({\bf k})-2b_{ij}({\bf k})=\delta_{ij}ikb_{\bf k}+2k_ik_ja_{\bf k}-2ik_ia_{0j}-2ik_ja_{0i}.
\label{6.68}
\end{eqnarray}
%
With $F_{ij}$ obeying the transverse and traceless conditions $\partial^{j}F_{ij}=\dot{F}_{0i}$, $\delta^{ij}F_{ij}-F_{00}=0$, we obtain 
%
\begin{eqnarray}
&&ik^ja_{ij}({\bf k})=-ika_{0i}({\bf k})-\frac{ik_i}{2}b_{\bf k},\quad ik^jb_{ij}({\bf k})=-\frac{kk_i}{2}b_{\bf k}, 
\nonumber\\
&&
\delta^{ij}a_{ij}({\bf k})=ika_{\bf k}-b_{\bf k},
\quad \delta^{ij}b_{ij}({\bf k})=\frac{ik}{2}b_{\bf k}.
\label{6.69}
\end{eqnarray}
%
Equations (\ref{6.58}) to (\ref{6.69}) provide us with the most general solution to (\ref{6.54}).


Having now obtained the exact solution, we see that  we do not get the decomposition theorem solution given in (\ref{6.57}). If we want to get the exact solution to reduce to the decomposition theorem solution we would need to set $\alpha$ and $F_{00}$ to zero, and this would be a particular solution to the fluctuation equations. However, there is no reason to set them to zero, and certainly no spatial asymptotic condition that could do so. And even if there were to be one, then such an asymptotic condition would have to suppress $\alpha$ and $F_{00}$ while at the same time not suppressing $F_{0i}$ and $F_{ij}$, even though though all of the fluctuation components have precisely the same asymptotic spatial behavior. We could possibly set $\alpha$ and $F_{00}$ to zero at all times via judiciously chosen initial conditions, but there would not appear to be any compelling reason for doing that either. As we had seen in our study of SVT4 without a conformal factor we would only be able to recover the decomposition theorem solution if we were to set $\chi$ to zero, and just as with wanting to set $\alpha$ and $F_{00}$ to zero, for $\chi$ there is also no reason  to do so. Thus in parallel with our analysis of SVT4 with no conformal factor, we find that similarly for SVT4 with a conformal factor no decomposition theorem is obtained in the de Sitter background case.


\subsubsection{Some General Comments}
\label{sss:some_general_comments}

While we have discussed SVT4 fluctuations around a de Sitter background as this is a rich enough system to show that one does not in general get a decomposition theorem, this discussion is not the one that is relevant to the early universe  inflationary model since that model is not described by an explicit cosmological constant but by a scalar field instead. Specifically, if we have a scalar field $S(x)$ with a Lagrangian density $L(S)=K(S)-V(S)$, then at the $S=S_0$ minimum of the $V(S)$ potential with constant $S_0$ the potential acts as a cosmological constant $V(S_0)$ and one has a background de Sitter geometry. If we now perturb the background the potential will change to $V(\delta S)$ even though $V(S_0)$ will not change at all. With there also being a change $K(\delta S)$ in  the scalar field kinetic energy, all of the terms in the background $T_{\mu\nu}=\partial_{\mu}S\partial_{\nu}S-g_{\mu\nu}L(S)$ will be perturbed and $\delta T_{\mu\nu}$ will not be of the form $\delta T_{\mu\nu}=\delta g_{\mu\nu}V(S_0)$ that we studied above. Nonetheless, it would not appear that there would obviously be an SVT4 decomposition theorem in  this more general scalar field case.


To conclude this section we note that in the above study of Einstein gravity SVT4 fluctuations around a de Sitter background we found in the no conformal prefactor case that the tensor fluctuations obeyed (\ref{6.31a}), viz.  
%
\begin{eqnarray}
(\nabla_{\alpha}\nabla^{\alpha}-4H^2)(\nabla_{\alpha}\nabla^{\alpha}-2H^2)F_{\mu\nu}=0.
\label{6.70}
\end{eqnarray}
%
Even though (\ref{6.70}) was obtained in Einstein gravity, this very same structure for $F_{\mu\nu}$ also appears in conformal gravity. In \cite{mannheim_2012}  the perturbative conformal gravity Bach tensor $\delta W_{\mu\nu}$ was calculated for fluctuations around a de Sitter background of the form $h_{\mu\nu}=K_{\mu\nu}+g_{\mu\nu}g^{\alpha\beta}h_{\alpha\beta}/4$ (i.e. a traceless but not necessarily transverse $K_{\mu\nu}$), and was found to take the  form
%
\begin{align}
\delta W_{\mu\nu}&=\frac{1}{2}[\nabla_{\alpha}\nabla^{\alpha}-4H^2][\nabla_{\beta}\nabla^{\beta}-2H^2]K_{\mu\nu}
\nonumber\\
&
-\frac{1}{2}[\nabla_{\beta}\nabla^{\beta}-4H^2][
\nabla_{\mu}\nabla_{\lambda}K^{\lambda}_{\phantom{\lambda}\nu}
+\nabla_{\nu}\nabla_{\lambda}K^{\lambda}_{\phantom{\lambda}\mu}]
\nonumber\\
&+\frac{1}{6}[g_{\mu\nu}\nabla_{\alpha}\nabla^{\alpha}+2\nabla_{\mu}\nabla_{\nu}
-6H^2g_{\mu\nu}]\nabla_{\kappa}\nabla_{\lambda}K^{\kappa\lambda}.
\label{6.71}
\end{align}
%
Evaluating $\delta W_{\mu\nu}$ for the fluctuation $h_{\mu\nu}$ given in (\ref{6.1}) in the same de Sitter background is found to yield 
%
\begin{eqnarray}
\delta W_{\mu\nu}= (\nabla_{\alpha}\nabla^{\alpha}-4H^2)(\nabla_{\alpha}\nabla^{\alpha}-2H^2)F_{\mu\nu},
\label{6.72}
\end{eqnarray}
%
i.e. the same structure that we would have obtained from (\ref{6.71}) had we made $K_{\mu\nu}$ transverse and replaced it by $2F_{\mu\nu}$ (even though the relation of $F_{\mu\nu}$ to  $h_{\mu\nu}$ is not the same as that of $K_{\mu\nu}$ to $h_{\mu\nu}$). We recognize the structure of the conformal gravity (\ref{6.72}) as being none other than that of the standard gravity (\ref{6.70}). The transverse-traceless sector of standard gravity (viz. gravity waves) thus has a conformal structure.


Now in a geometry that is conformal to flat such as de Sitter, the background $W_{\mu\nu}$ vanishes identically. Thus from the conformal gravity equation of motion (\ref{AP3}) for the Bach tensor it follows that the background $T_{\mu\nu}$ also vanishes identically. In the absence of a new source $\delta \bar{T}_{\mu\nu}$, for conformal gravity fluctuations around de Sitter  we can thus set 
%
\begin{eqnarray}
4\alpha_g\delta W_{\mu\nu}-\delta T_{\mu\nu}=0,\quad 4\alpha_g\delta W_{\mu\nu}=0,
\label{6.73}
\end{eqnarray}
%
since $\delta T_{\mu\nu}=0$. And since $\delta T_{\mu\nu}$ is zero,  it follows that  $\delta W_{\mu\nu}$ is gauge invariant all on its own in a background that is conformal to flat. And with it being traceless, the five degree of freedom $\delta W_{\mu\nu}$ can only depend on the five degree of freedom $F_{\mu\nu}$, just as we see in (\ref{6.72}). 

Now from  (\ref{6.19}) we can identify $(\nabla_{\alpha}\nabla^{\alpha}-2H^2)F_{\mu\nu}$ as the transverse-traceless piece of $\delta G_{\mu\nu}-3H^2h_{\mu\nu}$ in a de Sitter background, and thus can set 
%
\begin{eqnarray}
(\nabla_{\alpha}\nabla^{\alpha}-4H^2)(\delta G_{\mu\nu}+\delta T_{\mu\nu})^{T\theta}=(\nabla_{\alpha}\nabla^{\alpha}-4H^2)(\nabla_{\alpha}\nabla^{\alpha}-2H^2)F_{\mu\nu}
\label{6.74}
\end{eqnarray}
%
for the transverse ($T$) traceless ($\theta$) sector of $\delta G_{\mu\nu}+\delta T_{\mu\nu}$. Thus given (\ref{6.72}) we can set
%
\begin{eqnarray}
\delta W_{\mu\nu}=(\nabla_{\alpha}\nabla^{\alpha}-4H^2)(\delta G_{\mu\nu}+\delta T_{\mu\nu})^{T\theta}.
\label{6.75}
\end{eqnarray}
%
We thus generalize the flat space fluctuation relation $\delta W_{\mu\nu}=\nabla_{\alpha}\nabla^{\alpha}\delta G_{\mu\nu}^{T\theta}$ to the de Sitter case. Finally, we note that if the  $\delta\bar{T}_{\mu\nu}$ source is present, then its tracelessness in the conformal case restricts its form in (\ref{6.20}) to $\delta \bar{T}_{\mu\nu}=\bar{F}_{\mu\nu}$, with the conformal gravity fluctuation equation in a de Sitter background then taking the form 
%
\begin{eqnarray}
4\alpha_g(\nabla_{\alpha}\nabla^{\alpha}-4H^2)(\nabla_{\alpha}\nabla^{\alpha}-2H^2)F_{\mu\nu}=\bar{F}_{\mu\nu}.
\label{6.76}
\end{eqnarray}
%
Thus with or without $\delta \bar{T}_{\mu\nu}$, in the conformal gravity SVT4 de Sitter case $\delta W_{\mu\nu}$ depends on $F_{\mu\nu}$ alone, and with there being no dependence on $\chi$ the decomposition theorem is automatic. 
%%%%%%%%%%%%%%%%%%%%%%%%%%%%%%%%%%%%%%%%%%%%
\subsection{General Robertson Walker}
\label{ss:general_rw_SVT4}
%%%%%%%%%%%%%%%%%%%%%%%%%%%%%%%%%%%%%%%%%%%%

\subsubsection{The Background}
\label{sss:the_background_svt4_rw}

Let us take the background metric and the 3-space Ricci tensor to be of the form 
%
\begin{eqnarray}
ds^2 &=&-g_{\mu\nu}dx^{\mu}dx^{\nu}=\Omega^2(\tau)\left(d\tau^2 -\tilde{\gamma}_{ij} dx^i dx^j\right),\quad \tilde{R}_{ij} = -2k \tilde{\gamma}_{ij}.
\label{12.1}
\end{eqnarray}
%
Given the symmetry of the 4-geometry, the 4-space Ricci tensor and the 4-space Einstein tensor can be written as 
%
\begin{eqnarray}
R_{\mu\nu} &=& (A+B)U_\mu U_\nu + g_{\mu\nu}B,
\nonumber\\
G_{\mu\nu}&=& \tfrac{1}{2} A g_{\mu \nu } -  \tfrac{1}{2} B g_{\mu \nu } + A U_{\mu } U_{\nu } + B U_{\mu } U_{\nu },
\label{12.2}
\end{eqnarray}
%
where $A$ and $B$ are functions of $\tau$ alone and $U^{\mu}$ is a unit 4-vector that obeys $g_{\mu\nu}U^{\mu}U^{\nu}=-1$. With a background perfect fluid radiation era or matter era source of the form
%
\begin{eqnarray} 
T_{\mu\nu} &=& (\rho+p)U_\mu U_\nu +  p g_{\mu\nu},
\label{12.3}
\end{eqnarray}
%
where $\rho$ and $p$ are functions of $\tau$, the background Einstein equations are of the form
%
\begin{eqnarray}
\Delta_{\mu\nu}^{(0)}&=&\tfrac{1}{2} A g_{\mu \nu } -  \tfrac{1}{2} B g_{\mu \nu } + g_{\mu \nu } p + A U_{\mu } U_{\nu } + B U_{\mu } U_{\nu } + p U_{\mu } U_{\nu } 
\nonumber\\
&&+ U_{\mu } U_{\nu } \rho=0,
\label{12.4}
\end{eqnarray}
%
with solution 
%
\begin{eqnarray}
A &=& -\tfrac{1}{2} (3p+\rho)= -3 \dot{\Omega}^2 \Omega^{-4} + 3 \overset{..}{\Omega} \Omega^{-3}, 
\nonumber\\
B&=& \tfrac{1}{2}(p-\rho)=- \dot{\Omega}^2 \Omega^{-4} -  \overset{..}{\Omega} \Omega^{-3} - 2 k \Omega^{-2}, 
\nonumber\\
\rho &=& \tfrac{1}{2} (- A - 3 B)= 3 \dot{\Omega}^2 \Omega^{-4} + 3 k \Omega^{-2},
\nonumber\\
 p &=& \tfrac{1}{2} (- A + B)
= \dot{\Omega}^2 \Omega^{-4} - 2 \overset{..}{\Omega} \Omega^{-3} -  k \Omega^{-2}.
\label{12.5}
\end{eqnarray}
%

\subsubsection{The SVT4 Fluctuations}
\label{sss:fluctuations_svt4}

While we have incorporated a prefactor of $\Omega^2(\tau)$ in the background metric, we have found it more convenient to not include such a prefactor in the fluctuations. We thus take the background plus fluctuation metric to be of the form
%
\begin{eqnarray}
ds^2 &=&-[g_{\mu\nu}+h_{\mu\nu}]dx^{\mu}dx^{\nu},
\nonumber\\
 h_{\mu\nu}&=& -2 g_{\mu\nu}\chi + 2\nabla_\mu \nabla_\nu F +\nabla_\mu F_\nu +\nabla_\nu F_\mu+ 2F_{\mu\nu},
\label{12.6}
\end{eqnarray}
%
where the $\nabla_{\mu}$ derivatives are with respect to the full background $g_{\mu\nu}$, with respect to which $\nabla^{\mu}F_{\mu}=0$, $\nabla^{\mu}F_{\mu\nu}=0$. In analog to our discussion of SVT3 Robertson-Walker fluctuations given above, we set
%
\begin{eqnarray}
\delta U_{\mu} &=& (V_\mu + \nabla_\mu V) + U_\mu U^\alpha(V_\alpha + \nabla_\alpha V)-U_\mu\left(\tfrac{1}{2} U^\alpha U^\beta h_{\alpha\beta}\right), \quad Q_\mu = F_\mu + \nabla_\mu F, 
\nonumber\\
\hat{V}&=& V-U^\alpha Q_\alpha,
\nonumber\\
\delta \hat{\rho}{} &=& \delta \rho-(\rho+p)( Q^{\alpha } U_{\alpha } \nabla_{\beta }U^{\beta }-Q^{\alpha } U^{\beta } \nabla_{\alpha }U_{\beta }),
\nonumber\\
\delta \hat{p}{} &=& \delta p - \tfrac{1}{3} Q^{\alpha } \nabla_{\alpha }(3p+\rho) +  \tfrac{1}{3} (\rho+p) Q^{\alpha } U_{\alpha } \nabla_{\beta }U^{\beta }.
\label{12.7}
\end{eqnarray}
%
With these definitions and quite a bit of algebra we find that we can write the fluctuation equation $\Delta_{\mu\nu}=0$ as
%
\begin{eqnarray}
&&\Delta_{\mu\nu}= (g_{\mu \nu } + U_{\mu } U_{\nu }) \delta \hat{p}{} + U_{\mu } U_{\nu } \delta \hat{\rho}{} + \bigl((A -  B) g_{\mu \nu } + 2 (A + B) U_{\mu } U_{\nu }\bigr) \chi \nonumber \\ 
&& - 2 (A + B) U_{\mu } U_{\nu } U^{\alpha } \nabla_{\alpha }\hat{V}{} + 2 g_{\mu \nu } \nabla_{\alpha }\nabla^{\alpha }\chi 
\nonumber\\
&&-  (A + B) U_{\nu } \nabla_{\mu }\hat{V}{} -  (A + B) U_{\mu } \nabla_{\nu }\hat{V}{}  - 2 \nabla_{\nu }\nabla_{\mu }\chi -2 (A + B) U_{\mu } U_{\nu } U^{\alpha } V_{\alpha }
\nonumber\\
&& -  (A + B) U_{\nu } V_{\mu }  -  (A + B) U_{\mu } V_{\nu }+2 (A + B) U_{\mu } U_{\nu } U^{\alpha } U^{\beta } F_{\alpha \beta } 
\nonumber\\
&&+ 2 (A + B) U_{\nu } U^{\alpha } F_{\mu \alpha } + (\tfrac{1}{3} A + B) F_{\mu \nu }  + 2 (A + B) U_{\mu } U^{\alpha } F_{\nu \alpha } 
\nonumber\\
&&+ \nabla_{\alpha }\nabla^{\alpha }F_{\mu \nu }=0,
\nonumber\\ 
\nonumber\\
&&g^{\mu\nu}\Delta_{\mu\nu}= 3 \delta \hat{p}{} -  \delta \hat{\rho}{} + 2 (A - 3 B) \chi + 6 \nabla_{\alpha }\nabla^{\alpha }\chi +2 (A + B) U^{\alpha } U^{\beta } F_{\alpha \beta }=0,
\nonumber\\
\label{12.8}
\end{eqnarray}
%
or as
%
%
\begin{eqnarray}
\Delta_{\mu\nu}&=& (g_{\mu \nu } + U_{\mu } U_{\nu }) \delta \hat{p}{} + U_{\mu } U_{\nu } \delta \hat{\rho}{} + \bigl(-2 p g_{\mu \nu } - 2 (p + \rho) U_{\mu } U_{\nu }\bigr) \chi 
\nonumber\\
&&+ 2 (p + \rho) U_{\mu } U_{\nu } U^{\alpha } \nabla_{\alpha }\hat{V}{}  + 2 g_{\mu \nu } \nabla_{\alpha }\nabla^{\alpha }\chi + (p + \rho) U_{\nu } \nabla_{\mu }\hat{V}{} 
\nonumber\\
&&+ (p + \rho) U_{\mu } \nabla_{\nu }\hat{V}{} - 2 \nabla_{\nu }\nabla_{\mu }\chi +2 (p + \rho) U_{\mu } U_{\nu } U^{\alpha } V_{\alpha } \nonumber \\ 
&& + (p + \rho) U_{\nu } V_{\mu } + (p + \rho) U_{\mu } V_{\nu }-2 (p + \rho) U_{\mu } U_{\nu } U^{\alpha } U^{\beta } F_{\alpha \beta } 
\nonumber\\
&&- 2 (p + \rho) U_{\nu } U^{\alpha } F_{\mu \alpha } -  \tfrac{2}{3} \rho F_{\mu \nu }  - 2 (p + \rho) U_{\mu } U^{\alpha } F_{\nu \alpha }
\nonumber\\
&& + \nabla_{\alpha }\nabla^{\alpha }F_{\mu \nu }=0,
\nonumber\\ 
\nonumber\\
g^{\mu\nu}\Delta_{\mu\nu}&=& 3 \delta \hat{p}{} -  \delta \hat{\rho}{} + (-6 p + 2 \rho) \chi + 6 \nabla_{\alpha }\nabla^{\alpha }\chi 
\nonumber\\
&&-2 (p + \rho) U^{\alpha } U^{\beta } F_{\alpha \beta }=0.
\label{12.9}
\end{eqnarray}
%
As written, $\Delta_{\mu\nu}$ only depends on the metric fluctuations $F_{\mu\nu}$ and $\chi$  and  the source fluctuations $\delta \hat{\rho}$,  $\delta \hat{p}$, $\hat{V}$ and $V_i$. Comparing with the SVT3 (\ref{9.13}) to (\ref{9.17})  where there are $\alpha$, $\gamma$, $B_i-\dot{E}_i$ and $E_{ij}$ metric fluctuations and the same set of source fluctuations, we find, just as in the de Sitter background case, that  in a general Robertson-Walker background the SVT4 formalism is far more compact than the SVT3 formalism. 

As a check on our result we note that in a background de Sitter geometry with $\rho=-p=3H^2$, $k=0$, $\Omega=1/\tau H$, $\delta \hat{\rho}=\delta \rho=0$,  $\delta \hat{p}=\delta p=0$, (\ref{12.9}) reduces to 
%
\begin{eqnarray}
\Delta_{\mu\nu}&=& 6 H^2 g_{\mu \nu } \chi + 2 g_{\mu \nu } \nabla_{\alpha }\nabla^{\alpha }\chi - 2 \nabla_{\nu }\nabla_{\mu }\chi -2 H^2 F_{\mu \nu } + \nabla_{\alpha }\nabla^{\alpha }F_{\mu \nu }=0,
\nonumber\\ 
g^{\mu\nu}\Delta_{\mu\nu}&=& 24 H^2 \chi + 6 \nabla_{\alpha }\nabla^{\alpha }\chi=0. 
\label{12.10}
\end{eqnarray}
%
We recognize (\ref{12.10}) as (\ref{6.24}) and (\ref{6.25a}), just as required.

Finally, since the SVT4 fluctuation equations involve the $\nabla_{\alpha }\nabla^{\alpha }$ operator with its curved space harmonic basis functions, as before there will again be no decomposition theorem unless we choose some judicious initial conditions.


%%%%%%%%%%%%%%%%%%%%%%%%%%%%%%%%%%%%%%%%%%%%
\subsection{$\delta W_{\mu\nu}$ Conformal to Flat}
\label{ss:deltaw_conformal_flat_svt4}
%%%%%%%%%%%%%%%%%%%%%%%%%%%%%%%%%%%%%%%%%%%%
For conformal gravity SVT4 fluctuations associated with the metric $g_{\mu\nu}+h_{\mu\nu}$ where the background metric $g_{\mu\nu}$ is of the conformal to flat form given in (\ref{13.7}), we recall that for completely arbitrary conformal factor $\Omega(x)$ the fluctuation $\delta W_{\mu\nu}$ is given by the remarkably simple expression  \cite{amarasinghe_2019}

%
\begin{eqnarray}
\delta W_{\mu\nu}&=&\frac{1}{2}\Omega^{-2}\bigg{(}\partial_{\sigma}\partial^{\sigma}\partial_{\tau}\partial^{\tau}[\Omega^{-2}K_{\mu\nu}]
-\partial_{\sigma}\partial^{\sigma}\partial_{\mu}\partial^{\alpha}[\Omega^{-2}K_{\alpha\nu}]
-\partial_{\sigma}\partial^{\sigma}\partial_{\nu}\partial^{\alpha}[\Omega^{-2}K_{\alpha\mu}]
\nonumber\\
&+&\frac{2}{3}\partial_{\mu}\partial_{\nu}\partial^{\alpha}\partial^{\beta}[\Omega^{-2}K_{\alpha\beta}]+\frac{1}{3}\eta_{\mu\nu}\partial_{\sigma}\partial^{\sigma}\partial^{\alpha}\partial^{\beta}[\Omega^{-2}K_{\alpha\beta}]\bigg{)},
\label{13.18}
\end{eqnarray} 
%
where all derivatives are four-dimensional derivatives with respect to a flat Minkowski metric, and where $K_{\mu\nu}$ is given by $K_{\mu\nu}=h_{\mu\nu}-(1/4)g_{\mu\nu}g^{\alpha\beta}h_{\alpha\beta}$. If we now make the SVT4 expansion
%
\begin{eqnarray}
h_{\mu\nu}=\Omega^2(x)\left[-2\eta_{\mu\nu}\chi+2\partial_{\mu}\partial_{\nu}F
+ \partial_{\mu}F_{\nu}+\partial_{\nu}F_{\mu}+2F_{\mu\nu}\right],
\label{13.19}
\end{eqnarray}
%
where the derivatives and the transverse and tracelessness  $\partial^{\mu}F_{\mu}=0$, $\partial^{\nu}F_{\mu\nu}=0$, $\eta^{\mu\nu}F_{\mu\nu}=0$ conditions are with respect to a flat Minkowski background, we find that (\ref{13.18}) reduces to
%
\begin{eqnarray}
\delta W_{\mu\nu}&=&\Omega^{-2}\partial_{\sigma}\partial^{\sigma}\partial_{\tau}\partial^{\tau}F_{\mu\nu}.
\label{13.20}
\end{eqnarray} 
%
This expression is remarkable not just in its simplicity but in the fact that all components of $F_{\mu\nu}$ are completely decoupled from each other, with (\ref{13.20}) being diagonal in the $\mu,\nu$ indices. Since (\ref{13.20}) only contains $F_{\mu\nu}$ with none of $\chi$, $F$ or $F_{\mu}$ appearing  in it, unlike in the Einstein gravity SVT4 case where one needs initial conditions to establish the decomposition theorem, in the conformal gravity SVT4 case the decomposition theorem is automatic.