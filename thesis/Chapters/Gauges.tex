
\chapter{Constructing and Imposing Gauge Conditions}
\label{c:constructing_gauge_conditions}
In the context of conformal gravity, we continue work done in obtaining solutions to the cosmological fluctuation equations \cite{mannheim_2012} by constructing a gauge condition that is invariant under conformal transformations. Referred to as the conformal gauge, we find that in backgrounds that are conformal to flat, the conformal gravity cosmological fluctuation equations can be brought to an exceedingly simple form, comprised only of a single term. This calculation requires a series of many steps which are given in detail within the chapter, consisting of first motivating and constructing the conformal gauge, composing the fluctuation equations and using curvature identities and covariant derivatives to reduce its form, and finally imposing the conformal gauge itself, with expansion of covariant derivatives into flat Minkowski partial derivatives to yield \eqref{AP61}, one of the seminal results of this thesis. Using the conformal properties detailed in Sect. \ref{ss:conformal_invariance}, we find the extra degree of symmetry afforded by conformal invariance provides significant simplifications regarding the trace and moreover allows a very straightforward treatment of the entire cosmological fluctuation equations.

Such a streamlined approach may be contrasted with the Einstein fluctuations, which are also studied here in Sect. \ref{s:compact_expressions_ein} within differing choices of gauges. Specifically, we form a generalized gauge constraint and vary the coefficients in order to cast $\delta G_{\mu\nu}$ into as reduced and compact form possible, permitting readily solvable integral solutions to be obtained in some simple choices of background geometry (de Sitter and Minkowski). We find that is the coupling of the trace of the fluctuations that prevents the Einstein fluctuation equations from being able to be completely decoupled, unlike the case of conformal gravity where the trace is efficiently isolated as a consequence of conformal invariance.

To demonstrate a concrete application of the fluctuation equations in conformal gravity, in Sect. \ref{s:rw_radiation_conformal_gravity_sol} we evaluate and solve the fluctuations in a $k=-1$ Robertson Walker radiation era cosmology. Here we find that gravitational perturbations naturally have substantial growth, with a leading order time behavior $\propto t^4$ (with $t$ the comoving time). Such may be contrasted with comparatively damped radiation era $t^{1/2}$ leading order behavior one obtains in standard Einstein gravity \cite{weinberg_2008}.

%%%%%%%%%%%%%%%%%%%%%%%%%%%%%%%%%%%%%%%%%%%%
\section{The Conformal Gauge and General Solutions in Conformal Gravity}
\label{s:conformal_gauge_sols}
%%%%%%%%%%%%%%%%%%%%%%%%%%%%%%%%%%%%%%%%%%%%

%%%%%%%%%%%%%%%%%%%%%%%%%%%%%%%%%%%%%%%%%%%%
\subsection{The Conformal Gauge}
\label{ss:conformal_gauge}
%%%%%%%%%%%%%%%%%%%%%%%%%%%%%%%%%%%%%%%%%%%%
As discussed in Sect. \ref{ss:fluctuations_around_flat_in_the_harmonic_gauge}, we recall that under an infinitesimal transformation of coordinates of the form
%
\begin{eqnarray}
x^{\mu}\rightarrow x^{\mu}+\epsilon^{\mu}(x),
\label{liecoord}
\end{eqnarray}
%
the metric perturbation $h_{\mu\nu}$ will transform as
%
\begin{eqnarray}
h_{\mu\nu}-\nabla_{\nu}\epsilon_{\mu}-\nabla_{\mu}\epsilon_{\nu},
\label{covarh}
\end{eqnarray}
%
with contravariant components transforming similarly as
%
\begin{eqnarray}
h^{\mu\nu}-\nabla^{\nu}\epsilon^{\mu}-\nabla^{\mu}\epsilon^{\nu}.
\label{contrah}
\end{eqnarray}
%
Here all covariant derivatives are taken with respect to the background metric $g_{\mu\nu}^{(0)}$. For every solution $h_{\mu\nu}$ that satisfies the fluctuation equations (i.e. $\delta G_{\mu\nu} = \delta T_{\mu\nu}$ or $\delta W_{\mu\nu} = \delta T_{\mu\nu}$), there exists a transformed $h'_{\mu\nu}$ that will also serve as a solution. Thus with the freedom of fixing the four possible arbitrary space-time functions $\epsilon^\mu(x)$, one may eliminate the four gauge degrees of freedom within $h_{\mu\nu}$ by imposing a gauge condition satisfying four equations. Within Ch. \ref{c:formalism} we have already explored the form of the Einstein fluctuation equations in the harmonic gauge as well as the conformal gravity fluctuation equations in the transverse gauge.

In continuing the discussion within conformal gravity, we also recall that in background that are conformal to flat, the gravitational sector fluctuation $\delta W_{\mu\nu}$ (the Bach tensor, analogous to the Einstein tensor) depends only upon the trace free contribution of $h_{\mu\nu}$, to thus be able to be expressed entirely in terms of the 9 component $K_{\mu\nu}$, obeying $K^{\nu\sigma}g^{(0)}_{\nu\sigma}=0$. Our focus is then to determine the most appropriate gauge condition as applied to $K_{\mu\nu}$. We have seen in Sect. \ref{ss:fluctuations_around_flat_in_the_tranverse_gauge} that imposition of the transverse gauge (i.e. $\nabla^\mu K_{\mu\nu} = 0$) was effective in reducing the fluctuation equations into a readily solvable form within a flat space background. However, the analogous imposition of the transverse gauge in curved conformal to flat backgrounds does not provide the same degree of reduction. Rather, in the more general background, the fluctuations take a form where tensor components are tightly coupled, and no simple solution is immediately available. 

In an attempt to remedy the situation, we seek to find a gauge condition that makes contact with the desirable behavior within the flat space background. That is, we want a gauge condition that reduces to the transverse gauge if the background geometry is flat. As the content of this work concerns cosmology, even more specifically we desire a gauge condition that is suitable for conformal to flat backgrounds. The space of all conformal flat backgrounds is in fact quite large, with cosmologically relevant geometries only comprising a small subspace of all possible conformal flat metrics. The above gauge constraint criteria will in fact be satisfied if we can find a gauge condition that is conformally covariant (i.e. invariant under conformal transformation up to an overall scale factor). 

To this end, we first note that under the first order coordinate transform of \eqref{liecoord}, $K^{\mu\nu}$ transforms as
%
\begin{eqnarray}
K^{\mu\nu}\rightarrow K^{\mu\nu}-\nabla^{\nu}\epsilon^{\mu}-\nabla^{\mu}\epsilon^{\nu}+\frac{1}{2}g_{(0)}^{\mu\nu}\nabla_{\alpha}\epsilon^{\alpha}.
\label{AP17}
\end{eqnarray}
% 
With the transverse gauge serving as a starting point, we find its behavior under coordinate transformation, which takes the from
%
\begin{eqnarray}
\nabla_{\nu}K^{\mu\nu}=
\partial_{\nu}K^{\mu\nu}
+K^{\nu\sigma}g_{(0)}^{\mu\rho}\partial_{\nu}g^{(0)}_{\rho\sigma}
-\frac{1}{2}K^{\nu\sigma}g_{(0)}^{\mu\rho}\partial_{\rho}g^{(0)}_{\nu\sigma}
+\frac{1}{2}K^{\mu\sigma}g_{(0)}^{\nu\rho}\partial_{\sigma}g^{(0)}_{\rho\nu},
\label{AP18}
\end{eqnarray}
% 
Under a conformal transformation $\nabla_{\nu}K^{\mu\nu}$ transforms as
%
\begin{eqnarray}
\nabla_{\nu}K^{\mu\nu}\rightarrow \Omega^{-2}\nabla_{\nu}K^{\mu\nu}+4\Omega^{-3}K^{\mu\sigma}\partial_{\sigma}\Omega.
\label{AP19}
\end{eqnarray}
% 
Hence, as alluded to, the transverse gauge condition $\nabla_{\nu}K^{\mu\nu}=0$ is not conformal invariant. To determine such a coordinate gauge condition that is conformal invariant, we refer to \cite{amarasinghe_2019} and note that under a conformal transformation the quantity $K^{\mu\nu}g_{(0)}^{\alpha\beta}\partial_{\nu}g^{(0)}_{\alpha\beta}$ transforms as
%
\begin{eqnarray}
K^{\mu\nu}g_{(0)}^{\alpha\beta}\partial_{\nu}g^{(0)}_{\alpha\beta}
\rightarrow \Omega^{-2}K^{\mu\nu}g_{(0)}^{\alpha\beta}\partial_{\nu}g^{(0)}_{\alpha\beta}
+8\Omega^{-3}K^{\mu\nu}\partial_{\nu}\Omega.
\label{AP20}
\end{eqnarray}
%
Hence, it then follows that 
%
\begin{eqnarray}
\nabla_{\nu}K^{\mu\nu}-\frac{1}{2} K^{\mu\nu}g_{(0)}^{\alpha\beta}\partial_{\nu}g^{(0)}_{\alpha\beta}&\rightarrow &\Omega^{-2}\left[\nabla_{\nu}K^{\mu\nu}
-\frac{1}{2} K^{\mu\nu}g_{(0)}^{\alpha\beta}\partial_{\nu}g^{(0)}_{\alpha\beta}\right]
\nonumber\\
&=&\overline{\nabla_{\nu}K^{\mu\nu}}-\frac{1}{2} \bar{K}^{\mu\nu}\bar{g}_{(0)}^{\alpha\beta}\partial_{\nu}\bar{g}^{(0)}_{\alpha\beta}.
\label{AP21}
\end{eqnarray}
% 
Here, we clarify that $\overline{\nabla_{\nu}K^{\mu\nu}}$ is to be evaluated in a geometry with metric $\bar{g}^{(0)}_{\mu\nu}$ according to 
%
\begin{eqnarray}
\overline{\nabla_{\nu}K^{\mu\nu}}=
\partial_{\nu}\bar{K}^{\mu\nu}
+\bar{K}^{\nu\sigma}\bar{g}_{(0)}^{\mu\rho}\partial_{\nu}\bar{g}^{(0)}_{\rho\sigma}
-\frac{1}{2}\bar{K}^{\nu\sigma}\bar{g}_{(0)}^{\mu\rho}\partial_{\rho}\bar{g}^{(0)}_{\nu\sigma}
+\frac{1}{2}\bar{K}^{\mu\sigma}\bar{g}_{(0)}^{\nu\rho}\partial_{\sigma}\bar{g}^{(0)}_{\rho\nu}.
\label{AP22}
\end{eqnarray}
% 
We see that we have achieved the desired result, with the quantity $\nabla_{\nu}K^{\mu\nu}- K^{\mu\nu}g_{(0)}^{\alpha\beta}\partial_{\nu}g^{(0)}_{\alpha\beta}/2$ conformal covariance. Expressed in three equivalent forms, the gauge condition
%
\begin{eqnarray}
&&\nabla_{\nu}K^{\mu\nu}=\frac{1}{2}K^{\mu\nu}g_{(0)}^{\alpha\beta}\partial_{\nu}g^{(0)}_{\alpha\beta},
\nonumber\\
&&\partial_{\nu}K^{\mu\nu}+\Gamma^{\mu(0)}_{\nu\sigma}K^{\sigma\nu}
+\Gamma^{\nu(0)}_{\nu\sigma}K^{\mu\sigma}=K^{\mu\nu}\Gamma^{\alpha(0)}_{\alpha\nu},
\nonumber\\
&&\partial_{\nu}K^{\mu\nu}+\Gamma^{\mu(0)}_{\nu\sigma}K^{\sigma\nu}=0,
\label{AP23}
\end{eqnarray}
%
is aptly referred to as the conformal gauge.

As a check, we note that when the background is flat Minkowski ($g^{(0)}_{\alpha\beta}=\eta_{\alpha\beta}$), (\ref{AP23}) indeed reduces to the transverse condition $\partial_{\nu}K^{\mu\nu}=0$. Hence we may construct fluctuations around a conformal to flat background in the conformal gauge by conformally transforming fluctuations around a flat background in the transverse gauge. Such a method is not shared within standard Einstein gravity, but in conformal gravity it will be prove to be very beneficial in simplifying the fluctuation equations. 

%%%%%%%%%%%%%%%%%%%%%%%%%%%%%%%%%%%%%%%%%%%%
\subsection{$\delta W_{\mu\nu}$ in an Arbitrary Background}
\label{ss:fluctuation_eqns_around_arb_background_cgauge}
%%%%%%%%%%%%%%%%%%%%%%%%%%%%%%%%%%%%%%%%%%%%

%%%%%%%%%%%%%%%%%%%%%%%%%%%%%%%%%%%%%%%%%%%%
\subsubsection{Composing the Fluctuation Equations}
\label{sss:setting_up_fluctuation_eqns_cgauge}
%%%%%%%%%%%%%%%%%%%%%%%%%%%%%%%%%%%%%%%%%%%%

With the conformal gauge in hand, we proceed in a sequence of steps in order to implement it with the fluctuation equations. Prior to perturbing $W_{\mu\nu}$ we present a useful identity
%
\begin{eqnarray}
\nabla_{\beta}\nabla_{\nu}T_{\lambda \mu}=\nabla_{\nu}\nabla_{\beta}T_{\lambda \mu}+R_{\lambda\sigma\nu\beta}T^{\sigma}_{\phantom{\sigma}\mu}-R_{\sigma\mu\nu\beta}T_{\lambda}^{\phantom{\lambda}\sigma},
\label{AP41}
\end{eqnarray}
%
which holds for any rank two tensor. We then express $W^{\mu\nu}$ as 
%                                                                               
\begin{eqnarray}
W_{\mu \nu}&=&
-\frac{1}{6}g_{\mu\nu}\nabla_{\beta}\nabla^{\beta}R^{\alpha}_{\phantom{\alpha}\alpha}
+\nabla_{\beta}\nabla^{\beta}R_{\mu\nu}                    
-\frac{1}{3}\nabla_{\mu}\nabla_{\nu}R^{\alpha}_{\phantom{\alpha}\alpha}  
-R^{\beta\sigma} R_{\sigma\mu\beta\nu}   
\nonumber\\
&-&R^{\beta\sigma} R_{\sigma\nu\beta\mu}  
+\frac{1}{2}g_{\mu\nu}R_{\alpha\beta}R^{\alpha\beta}                                            
+\frac{2}{3}R^{\alpha}_{\phantom{\alpha}\alpha}R_{\mu\nu}                              
-\frac{1}{6}g_{\mu\nu}(R^{\alpha}_{\phantom{\alpha}\alpha})^2.
\label{AP42}
\end{eqnarray}                                 
%
We set the metric as the most general $g_{\mu\nu}+h_{\mu\nu}$, where $g_{\mu\nu}$ denotes an arbitrary background metric and $\delta g_{\mu\nu}=h_{\mu\nu}$ an arbitrary and general fluctuation. Upon perturbing $W^{\mu\nu}$ we then obtain
%
\begin{eqnarray}
&&\delta W_{\mu\nu}(h_{\mu\nu})=\tfrac{1}{2} h_{\mu \nu} R_{\alpha \beta} R^{\alpha \beta} -  g_{\mu \nu} h^{\alpha \beta} R_{\alpha}{}^{\gamma} R_{\beta \gamma} -  \tfrac{2}{3} h^{\alpha \beta} R_{\alpha \beta} R_{\mu \nu} + \tfrac{1}{3} g_{\mu \nu} h^{\alpha \beta} R_{\alpha \beta} R
\nonumber\\
&& -  \tfrac{1}{6} h_{\mu \nu} R^2 
+ h^{\alpha \beta} R_{\alpha}{}^{\gamma} R_{\mu \beta \nu \gamma} + h^{\alpha \beta} R_{\alpha}{}^{\gamma} R_{\mu \gamma \nu \beta} -  \tfrac{1}{6} h_{\mu \nu} \nabla_{\alpha}\nabla^{\alpha}R -  h^{\alpha \beta} \nabla_{\beta}\nabla_{\alpha}R_{\mu \nu} 
\nonumber\\
&&+ \tfrac{1}{6} g_{\mu \nu} h^{\alpha \beta} \nabla_{\beta}\nabla_{\alpha}R + \tfrac{1}{6} g_{\mu \nu} h^{\alpha \beta} \nabla_{\gamma}\nabla^{\gamma}R_{\alpha \beta} 
+ \tfrac{1}{3} h^{\alpha \beta} \nabla_{\mu}\nabla_{\nu}R_{\alpha \beta}
+\tfrac{1}{3} R \nabla_{\alpha}\nabla^{\alpha}h_{\mu \nu}
\nonumber\\
&& + R_{\mu \beta \nu \gamma} \nabla_{\alpha}\nabla^{\gamma}h^{\alpha \beta} + R_{\mu \gamma \nu \beta} \nabla_{\alpha}\nabla^{\gamma}h^{\alpha \beta} -  \tfrac{1}{3} R \nabla_{\alpha}\nabla_{\mu}h_{\nu}{}^{\alpha} -  \tfrac{1}{3} R \nabla_{\alpha}\nabla_{\nu}h_{\mu}{}^{\alpha} 
\nonumber\\
&&-  \tfrac{1}{6} \nabla_{\alpha}h_{\mu \nu} \nabla^{\alpha}R 
+ \tfrac{1}{6} g_{\mu \nu} \nabla^{\alpha}R \nabla_{\beta}h_{\alpha}{}^{\beta} -  \nabla_{\alpha}h^{\alpha \beta} \nabla_{\beta}R_{\mu \nu} -  \tfrac{2}{3} R_{\mu \nu} \nabla_{\beta}\nabla_{\alpha}h^{\alpha \beta} 
\nonumber\\
&&+ \tfrac{1}{3} g_{\mu \nu} R \nabla_{\beta}\nabla_{\alpha}h^{\alpha \beta} + \tfrac{1}{2} R_{\nu}{}^{\alpha} \nabla_{\beta}\nabla_{\alpha}h_{\mu}{}^{\beta} 
-  R^{\alpha \beta} \nabla_{\beta}\nabla_{\alpha}h_{\mu \nu} 
+ \tfrac{1}{2} R_{\mu}{}^{\alpha} \nabla_{\beta}\nabla_{\alpha}h_{\nu}{}^{\beta} 
\nonumber\\
&&-  \tfrac{1}{2} R_{\nu}{}^{\alpha} \nabla_{\beta}\nabla^{\beta}h_{\mu \alpha} -  \tfrac{1}{2} R_{\mu}{}^{\alpha} \nabla_{\beta}\nabla^{\beta}h_{\nu \alpha} + \tfrac{1}{2} \nabla_{\beta}\nabla^{\beta}\nabla_{\alpha}\nabla^{\alpha}h_{\mu \nu} 
-  \tfrac{1}{2} \nabla_{\beta}\nabla^{\beta}\nabla_{\alpha}\nabla_{\mu}h_{\nu}{}^{\alpha} 
\nonumber\\
&&
-  \tfrac{1}{2} \nabla_{\beta}\nabla^{\beta}\nabla_{\alpha}\nabla_{\nu}h_{\mu}{}^{\alpha} -  \tfrac{1}{2} R_{\nu}{}^{\alpha} \nabla_{\beta}\nabla_{\mu}h_{\alpha}{}^{\beta} + R^{\alpha \beta} \nabla_{\beta}\nabla_{\mu}h_{\nu \alpha} -  \tfrac{1}{2} R_{\mu}{}^{\alpha} \nabla_{\beta}\nabla_{\nu}h_{\alpha}{}^{\beta} 
\nonumber\\
&&+ R^{\alpha \beta} \nabla_{\beta}\nabla_{\nu}h_{\mu \alpha} 
+ \nabla_{\alpha}R_{\nu \beta} \nabla^{\beta}h_{\mu}{}^{\alpha} 
-  \nabla_{\beta}R_{\nu \alpha} \nabla^{\beta}h_{\mu}{}^{\alpha} + \nabla_{\alpha}R_{\mu \beta} \nabla^{\beta}h_{\nu}{}^{\alpha} 
\nonumber\\
&&-  \nabla_{\beta}R_{\mu \alpha} \nabla^{\beta}h_{\nu}{}^{\alpha} -  g_{\mu \nu} R^{\alpha \beta} \nabla_{\gamma}\nabla_{\beta}h_{\alpha}{}^{\gamma} 
+ \tfrac{2}{3} g_{\mu \nu} R^{\alpha \beta} \nabla_{\gamma}\nabla^{\gamma}h_{\alpha \beta} 
-  R_{\mu \alpha \nu \beta} \nabla_{\gamma}\nabla^{\gamma}h^{\alpha \beta} 
\nonumber\\
&&+ \tfrac{1}{6} g_{\mu \nu} \nabla_{\gamma}\nabla^{\gamma}\nabla_{\beta}\nabla_{\alpha}h^{\alpha \beta} + \tfrac{1}{3} g_{\mu \nu} \nabla_{\gamma}R_{\alpha \beta} \nabla^{\gamma}h^{\alpha \beta} -  \nabla_{\beta}R_{\nu \alpha} \nabla_{\mu}h^{\alpha \beta} 
+ \tfrac{1}{6} \nabla^{\alpha}R \nabla_{\mu}h_{\nu \alpha} 
\nonumber\\
&&
-  \tfrac{1}{6} R^{\alpha \beta} \nabla_{\mu}\nabla_{\nu}h_{\alpha \beta} -  \nabla_{\beta}R_{\mu \alpha} \nabla_{\nu}h^{\alpha \beta} + \tfrac{1}{3} \nabla_{\mu}R_{\alpha \beta} \nabla_{\nu}h^{\alpha \beta} + \tfrac{1}{6} \nabla^{\alpha}R \nabla_{\nu}h_{\mu \alpha} 
\nonumber\\
&&+ \tfrac{1}{3} \nabla_{\mu}h^{\alpha \beta} \nabla_{\nu}R_{\alpha \beta} 
-  \tfrac{1}{2} R^{\alpha \beta} \nabla_{\nu}\nabla_{\mu}h_{\alpha \beta} 
+ \tfrac{1}{3} \nabla_{\nu}\nabla_{\mu}\nabla_{\beta}\nabla_{\alpha}h^{\alpha \beta}
+\tfrac{2}{3} R_{\mu \nu} \nabla_{\alpha}\nabla^{\alpha}h 
\nonumber\\
&&-  \tfrac{1}{3} g_{\mu \nu} R \nabla_{\alpha}\nabla^{\alpha}h + \tfrac{1}{2} \nabla_{\alpha}\nabla^{\alpha}\nabla_{\nu}\nabla_{\mu}h 
-  \tfrac{1}{12} g_{\mu \nu} \nabla_{\alpha}h \nabla^{\alpha}R + \tfrac{1}{2} \nabla_{\alpha}R_{\mu \nu} \nabla^{\alpha}h 
\nonumber\\
&&
+ \tfrac{1}{2} g_{\mu \nu} R^{\alpha \beta} \nabla_{\beta}\nabla_{\alpha}h -  \tfrac{1}{6} g_{\mu \nu} \nabla_{\beta}\nabla^{\beta}\nabla_{\alpha}\nabla^{\alpha}h -  R_{\mu \alpha \nu \beta} \nabla^{\beta}\nabla^{\alpha}h + \tfrac{1}{3} R \nabla_{\nu}\nabla_{\mu}h 
\nonumber\\
&&-  \tfrac{1}{3} \nabla_{\nu}\nabla_{\mu}\nabla_{\alpha}\nabla^{\alpha}h.
\label{AP43}
\end{eqnarray}
%
Here we clarify that all covariant derivatives are evaluated with respect to the background $g_{\mu\nu}$ (and use a more compact notation where the Ricci scalar $R$ denotes $R^{\alpha}_{\phantom{\alpha}\alpha}$.) Inspecting (\ref{AP43}), we observe a total of 62 different terms, with ten of these depending on the trace $h=g^{\mu\nu}h_{\mu\nu}$. 

We now substitute $h_{\mu\nu}=K_{\mu\nu}+(1/4)g_{\mu\nu}h$ in (\ref{AP43}) and, as anticipated and discussed in Sect. \ref{ss:conformal_invariance}, we find that $\delta W_{\mu\nu}(h_{\mu\nu})$ breaks into two pieces; one that depends only on $K_{\mu\nu}$ comprising 52 terms and one that depends only on $h=g_{\mu\nu}h^{\mu\nu}$ with 19 total terms. These two separate components have been obtained and take the form
%
\begin{eqnarray}
&&\delta W_{\mu\nu}(K_{\mu\nu})=\tfrac{1}{2} K_{\mu \nu} R_{\alpha \beta} R^{\alpha \beta} -  g_{\mu \nu} K^{\alpha \beta} R_{\alpha}{}^{\gamma} R_{\beta \gamma} -  \tfrac{2}{3} K^{\alpha \beta} R_{\alpha \beta} R_{\mu \nu} 
\nonumber\\
&&+ \tfrac{1}{3} g_{\mu \nu} K^{\alpha \beta} R_{\alpha \beta} R -  \tfrac{1}{6} K_{\mu \nu} R^2 
+ K^{\alpha \beta} R_{\alpha}{}^{\gamma} R_{\mu \beta \nu \gamma} 
+ K^{\alpha \beta} R_{\alpha}{}^{\gamma} R_{\mu \gamma \nu \beta} -  \tfrac{1}{6} K_{\mu \nu} \nabla_{\alpha}\nabla^{\alpha}R 
\nonumber\\
&&-  K^{\alpha \beta} \nabla_{\beta}\nabla_{\alpha}R_{\mu \nu} + \tfrac{1}{6} g_{\mu \nu} K^{\alpha \beta} \nabla_{\beta}\nabla_{\alpha}R + \tfrac{1}{6} g_{\mu \nu} K^{\alpha \beta} \nabla_{\gamma}\nabla^{\gamma}R_{\alpha \beta} 
+ \tfrac{1}{3} K^{\alpha \beta} \nabla_{\mu}\nabla_{\nu}R_{\alpha \beta}
\nonumber\\
&&
+\tfrac{1}{3} R \nabla_{\alpha}\nabla^{\alpha}K_{\mu \nu} + R_{\mu \beta \nu \gamma} \nabla_{\alpha}\nabla^{\gamma}K^{\alpha \beta} + R_{\mu \gamma \nu \beta} \nabla_{\alpha}\nabla^{\gamma}K^{\alpha \beta} -  \tfrac{1}{3} R \nabla_{\alpha}\nabla_{\mu}K_{\nu}{}^{\alpha} 
\nonumber\\
&&-  \tfrac{1}{3} R \nabla_{\alpha}\nabla_{\nu}K_{\mu}{}^{\alpha} 
-  \tfrac{1}{6} \nabla_{\alpha}K_{\mu \nu} \nabla^{\alpha}R 
+ \tfrac{1}{6} g_{\mu \nu} \nabla^{\alpha}R \nabla_{\beta}K_{\alpha}{}^{\beta} -  \nabla_{\alpha}K^{\alpha \beta} \nabla_{\beta}R_{\mu \nu} 
\nonumber\\
&&-  \tfrac{2}{3} R_{\mu \nu} \nabla_{\beta}\nabla_{\alpha}K^{\alpha \beta} + \tfrac{1}{3} g_{\mu \nu} R \nabla_{\beta}\nabla_{\alpha}K^{\alpha \beta} + \tfrac{1}{2} R_{\nu}{}^{\alpha} \nabla_{\beta}\nabla_{\alpha}K_{\mu}{}^{\beta} 
-  R^{\alpha \beta} \nabla_{\beta}\nabla_{\alpha}K_{\mu \nu} 
\nonumber\\
&&
+ \tfrac{1}{2} R_{\mu}{}^{\alpha} \nabla_{\beta}\nabla_{\alpha}K_{\nu}{}^{\beta} -  \tfrac{1}{2} R_{\nu}{}^{\alpha} \nabla_{\beta}\nabla^{\beta}K_{\mu \alpha} -  \tfrac{1}{2} R_{\mu}{}^{\alpha} \nabla_{\beta}\nabla^{\beta}K_{\nu \alpha} + \tfrac{1}{2} \nabla_{\beta}\nabla^{\beta}\nabla_{\alpha}\nabla^{\alpha}K_{\mu \nu} 
\nonumber\\
&&-  \tfrac{1}{2} \nabla_{\beta}\nabla^{\beta}\nabla_{\alpha}\nabla_{\mu}K_{\nu}{}^{\alpha} 
-  \tfrac{1}{2} \nabla_{\beta}\nabla^{\beta}\nabla_{\alpha}\nabla_{\nu}K_{\mu}{}^{\alpha} -  \tfrac{1}{2} R_{\nu}{}^{\alpha} \nabla_{\beta}\nabla_{\mu}K_{\alpha}{}^{\beta} + R^{\alpha \beta} \nabla_{\beta}\nabla_{\mu}K_{\nu \alpha} 
\nonumber\\
&&-  \tfrac{1}{2} R_{\mu}{}^{\alpha} \nabla_{\beta}\nabla_{\nu}K_{\alpha}{}^{\beta} 
+ R^{\alpha \beta} \nabla_{\beta}\nabla_{\nu}K_{\mu \alpha} 
+ \nabla_{\alpha}R_{\nu \beta} \nabla^{\beta}K_{\mu}{}^{\alpha} 
-  \nabla_{\beta}R_{\nu \alpha} \nabla^{\beta}K_{\mu}{}^{\alpha} 
\nonumber\\
&&+ \nabla_{\alpha}R_{\mu \beta} \nabla^{\beta}K_{\nu}{}^{\alpha} -  \nabla_{\beta}R_{\mu \alpha} \nabla^{\beta}K_{\nu}{}^{\alpha} -  g_{\mu \nu} R^{\alpha \beta} \nabla_{\gamma}\nabla_{\beta}K_{\alpha}{}^{\gamma} 
+ \tfrac{2}{3} g_{\mu \nu} R^{\alpha \beta} \nabla_{\gamma}\nabla^{\gamma}K_{\alpha \beta} 
\nonumber\\
&&
-  R_{\mu \alpha \nu \beta} \nabla_{\gamma}\nabla^{\gamma}K^{\alpha \beta} + \tfrac{1}{6} g_{\mu \nu} \nabla_{\gamma}\nabla^{\gamma}\nabla_{\beta}\nabla_{\alpha}K^{\alpha \beta} + \tfrac{1}{3} g_{\mu \nu} \nabla_{\gamma}R_{\alpha \beta} \nabla^{\gamma}K^{\alpha \beta} 
\nonumber\\
&&-  \nabla_{\beta}R_{\nu \alpha} \nabla_{\mu}K^{\alpha \beta} 
+ \tfrac{1}{6} \nabla^{\alpha}R \nabla_{\mu}K_{\nu \alpha} 
-  \tfrac{1}{6} R^{\alpha \beta} \nabla_{\mu}\nabla_{\nu}K_{\alpha \beta} 
-  \nabla_{\beta}R_{\mu \alpha} \nabla_{\nu}K^{\alpha \beta}  
\nonumber\\
&&+ \tfrac{1}{3} \nabla_{\mu}R_{\alpha \beta} \nabla_{\nu}K^{\alpha \beta} 
+ \tfrac{1}{6} \nabla^{\alpha}R \nabla_{\nu}K_{\mu \alpha} + \tfrac{1}{3} \nabla_{\mu}K^{\alpha \beta} \nabla_{\nu}R_{\alpha \beta} 
-  \tfrac{1}{2} R^{\alpha \beta} \nabla_{\nu}\nabla_{\mu}K_{\alpha \beta}
\nonumber\\
&&+ \tfrac{1}{3} \nabla_{\nu}\nabla_{\mu}\nabla_{\beta}\nabla_{\alpha}K^{\alpha \beta},
\label{AP44}
\end{eqnarray}
%
%
\begin{eqnarray}
&&\delta W_{\mu\nu}(h)=- \tfrac{1}{8} g_{\mu \nu} R_{\alpha \beta} R^{\alpha \beta} h -  \tfrac{1}{6} R_{\mu \nu} R h + \tfrac{1}{24} g_{\mu \nu} R^2 h + \tfrac{1}{2} R^{\alpha \beta} R_{\mu \alpha \nu \beta} h
\nonumber\\
&& -  \tfrac{1}{4} h \nabla_{\alpha}\nabla^{\alpha}R_{\mu \nu} + \tfrac{1}{24} g_{\mu \nu} h \nabla_{\alpha}\nabla^{\alpha}R 
+ \tfrac{1}{12} h \nabla_{\nu}\nabla_{\mu}R
+\tfrac{1}{4} \nabla_{\alpha}\nabla^{\alpha}\nabla_{\nu}\nabla_{\mu}h
\nonumber\\
&& -  \tfrac{1}{4} \nabla_{\alpha}R_{\mu \nu} \nabla^{\alpha}h -  \tfrac{1}{2} R_{\mu \alpha \nu \beta} \nabla^{\beta}\nabla^{\alpha}h + \tfrac{1}{4} \nabla_{\mu}R_{\nu \alpha}\nabla^{\alpha}h  -  \tfrac{1}{4} \nabla_{\alpha}R_{\nu}{}^{\alpha} \nabla_{\mu}h
\nonumber\\
&& + \tfrac{1}{4} R_{\nu}{}^{\alpha} \nabla_{\mu}\nabla_{\alpha}h
 + \tfrac{1}{4} \nabla_{\nu}R_{\mu \alpha} \nabla^{\alpha}h + \tfrac{1}{8} \nabla_{\nu}R\nabla_{\mu}h  -  \tfrac{1}{4} \nabla_{\alpha}R_{\mu}{}^{\alpha} \nabla_{\nu}h + \tfrac{1}{8} \nabla_{\mu}R \nabla_{\nu}h +
 \nonumber\\
 && \tfrac{1}{4} R_{\mu}{}^{\alpha} \nabla_{\nu}\nabla_{\alpha}h  
-  \tfrac{1}{4} \nabla_{\nu}\nabla_{\mu}\nabla_{\alpha}\nabla^{\alpha}h.
\label{AP45}
\end{eqnarray}
%
%%%%%%%%%%%%%%%%%%%%%%%%%%%%%%%%%%%%%%%%%%%%
\subsubsection{Trace Properties}
\label{sss:decoupling_trace_cgauge}
Within Sect. \ref{ss:conformal_invariance}, we illustrated an important property of the trace of the fluctation, here given within \eqref{AP45}, where we have demonstrated its form being able to represent as proportional to the background $W_{\mu\nu}$. To show and verify such a form, we first note the identity 
%
\begin{eqnarray}
\nabla_{\kappa}\nabla_{\nu}V_{\lambda}-\nabla_{\nu}\nabla_{\kappa}V_{\lambda}=V^{\sigma}R_{\lambda\sigma\nu\kappa}
\label{AP46}
\end{eqnarray}
%
which holds for any vector $V_{\lambda}$. Upon setting $V_{\lambda}=\nabla_{\lambda}h$ and $T_{\lambda\mu}=\nabla_{\lambda}\nabla_{\mu}h$ in (\ref{AP41}) we obtain
%
\begin{eqnarray}
\nabla_{\nu}\nabla_{\mu}\nabla_{\alpha}\nabla^{\alpha}h
&=&g^{\alpha\beta}\nabla_{\nu}[\nabla_{\alpha}\nabla_{\mu}\nabla_{\beta}h
+R_{\beta\sigma\alpha\mu}\nabla^{\sigma}h]
\nonumber\\
&=&g^{\alpha\beta}\nabla_{\nu}[\nabla_{\alpha}\nabla_{\beta}\nabla_{\mu}h
+R_{\beta\sigma\alpha\mu}\nabla^{\sigma}h]
\nonumber\\
&=&g^{\alpha\beta}[\nabla_{\alpha}\nabla_{\nu}\nabla_{\beta}\nabla_{\mu}h
+R_{\beta\sigma\alpha\nu}\nabla^{\sigma}\nabla_{\mu}h
-R_{\sigma\mu\alpha\nu}\nabla_{\beta}\nabla^{\sigma}h
\nonumber\\
&&
+R_{\beta\sigma\alpha\mu}\nabla_{\nu}\nabla^{\sigma}h
+\nabla_{\nu}R_{\beta\sigma\alpha\mu}\nabla^{\sigma}h]
\nonumber\\
&=&g^{\alpha\beta}[\nabla_{\alpha}[\nabla_{\beta}\nabla_{\nu}\nabla_{\mu}h
+R_{\mu\sigma\beta\nu}\nabla^{\sigma}h]
+R_{\beta\sigma\alpha\nu}\nabla^{\sigma}\nabla_{\mu}h
\nonumber\\
&&
-R_{\sigma\mu\alpha\nu}\nabla_{\beta}\nabla^{\sigma}h
+R_{\beta\sigma\alpha\mu}\nabla_{\nu}\nabla^{\sigma}h
+\nabla_{\nu}R_{\beta\sigma\alpha\mu}\nabla^{\sigma}h].
\label{AP47}
\end{eqnarray}
%
Recalling the curvature relation 
%
\begin{eqnarray}
\nabla^{\nu}R_{\nu\mu\kappa\eta}=\nabla_{\kappa}R_{\mu\eta}-\nabla_{\eta}R_{\mu\kappa},
\label{AP48}
\end{eqnarray}
%
it then follows that 
%
\begin{eqnarray}
&&\nabla_{\nu}\nabla_{\mu}\nabla_{\alpha}\nabla^{\alpha}h-\nabla_{\alpha}\nabla^{\alpha}\nabla_{\nu}\nabla_{\mu}h
=R_{\mu\sigma\alpha\nu}\nabla^{\alpha}\nabla^{\sigma}h
+\nabla_{\mu}R_{\nu\sigma}\nabla^{\sigma}h
-\nabla_{\sigma}R_{\nu\mu}
\nabla^{\sigma}h\nonumber\\
&&+R_{\sigma\nu}\nabla^{\sigma}\nabla_{\mu}h
-R_{\sigma\mu\alpha\nu}\nabla^{\alpha}\nabla^{\sigma}h
+R_{\sigma\mu}\nabla_{\nu}\nabla^{\sigma}h
+\nabla_{\nu}R_{\sigma\mu}\nabla^{\sigma}h.
\label{AP49}
\end{eqnarray}
%
Finally, using the Bianchi identity $\nabla^{\alpha}R_{\mu\alpha}=(1/2)\nabla _{\mu}R$ we thus observe that all twelve terms within (\ref{AP45}) that contain the gradient of $h$ must all cancel. For those seven terms that remain, we can further observe that they are directly equivalent to the definition of the background $W_{\mu\nu}$. Thus, the trace component $\delta W_{\mu\nu}(h)$ reduces to the extremely compact form of
%
\begin{eqnarray}
\delta W_{\mu\nu}(h)=-\frac{1}{4}W_{\mu\nu}h.
\label{AP50}
\end{eqnarray}
%
To gain further insight into the form of \eqref{AP50}, it is instructive to consider the fluctuation of the Weyl tensor itself. In evaluating the perturbed Weyl tensor around an arbitrary background, iwe obtain $\delta C_{\lambda\mu\nu\kappa}=\delta C_{\lambda\mu\nu\kappa}(K_{\mu\nu})+\delta C_{\lambda\mu\nu\kappa}(h)$, where
%
\begin{eqnarray}
&&\delta C_{\lambda\mu\nu\kappa}(K_{\mu\nu})=- \tfrac{1}{6} g_{\kappa \mu} g_{\lambda \nu} K^{\alpha \beta} R_{\alpha \beta} + \tfrac{1}{6} g_{\kappa \lambda} g_{\mu \nu} K^{\alpha \beta} R_{\alpha \beta} + \tfrac{1}{2} K_{\mu \nu} R_{\kappa \lambda} -  \tfrac{1}{2} K_{\lambda \nu} R_{\kappa \mu}  
\nonumber\\
&&-  \tfrac{1}{2} K_{\kappa \mu} R_{\lambda \nu} 
+ \tfrac{1}{2} K_{\kappa \lambda} R_{\mu \nu} -  \tfrac{1}{6} g_{\mu \nu} K_{\kappa \lambda} R 
+ \tfrac{1}{6} g_{\lambda \nu} K_{\kappa \mu} R + \tfrac{1}{6} g_{\kappa \mu} K_{\lambda \nu} R -  \tfrac{1}{6} g_{\kappa \lambda} K_{\mu \nu} R 
\nonumber\\
&& + K_{\lambda}{}^{\alpha} R_{\kappa \nu \mu \alpha} + \tfrac{1}{4} g_{\mu \nu} \nabla_{\alpha}\nabla^{\alpha}K_{\kappa \lambda} -  \tfrac{1}{4} g_{\lambda \nu} \nabla_{\alpha}\nabla^{\alpha}K_{\kappa \mu} 
-  \tfrac{1}{4} g_{\kappa \mu} \nabla_{\alpha}\nabla^{\alpha}K_{\lambda \nu}  
\nonumber\\
&&
+ \tfrac{1}{4} g_{\kappa \lambda} \nabla_{\alpha}\nabla^{\alpha}K_{\mu \nu} 
-  \tfrac{1}{4} g_{\mu \nu} \nabla_{\alpha}\nabla_{\kappa}K_{\lambda}{}^{\alpha} + \tfrac{1}{4} g_{\lambda \nu} \nabla_{\alpha}\nabla_{\kappa}K_{\mu}{}^{\alpha} -  \tfrac{1}{4} g_{\mu \nu} \nabla_{\alpha}\nabla_{\lambda}K_{\kappa}{}^{\alpha}  
\nonumber\\
&&
 + \tfrac{1}{4} g_{\kappa \mu} \nabla_{\alpha}\nabla_{\lambda}K_{\nu}{}^{\alpha} 
+ \tfrac{1}{4} g_{\lambda \nu} \nabla_{\alpha}\nabla_{\mu}K_{\kappa}{}^{\alpha} -  \tfrac{1}{4} g_{\kappa \lambda} \nabla_{\alpha}\nabla_{\mu}K_{\nu}{}^{\alpha} + \tfrac{1}{4} g_{\kappa \mu} \nabla_{\alpha}\nabla_{\nu}K_{\lambda}{}^{\alpha}  
\nonumber\\
&&-  \tfrac{1}{4} g_{\kappa \lambda} \nabla_{\alpha}\nabla_{\nu}K_{\mu}{}^{\alpha} 
-  \tfrac{1}{6} g_{\kappa \mu} g_{\lambda \nu} \nabla_{\beta}\nabla_{\alpha}K^{\alpha \beta} 
+ \tfrac{1}{6} g_{\kappa \lambda} g_{\mu \nu} \nabla_{\beta}\nabla_{\alpha}K^{\alpha \beta} 
-  \tfrac{1}{2} \nabla_{\kappa}\nabla_{\lambda}K_{\mu \nu}  
\nonumber\\
&&+ \tfrac{1}{2} \nabla_{\kappa}\nabla_{\mu}K_{\lambda \nu} 
+ \tfrac{1}{2} \nabla_{\kappa}\nabla_{\nu}K_{\lambda \mu}  
- \tfrac{1}{2} \nabla_{\nu}\nabla_{\kappa}K_{\lambda \mu} + \tfrac{1}{2} \nabla_{\nu}\nabla_{\lambda}K_{\kappa \mu} 
-  \tfrac{1}{2} \nabla_{\nu}\nabla_{\mu}K_{\kappa \lambda}
\nonumber\\
\label{AP51}
\end{eqnarray}
%
and
%
\begin{eqnarray}
\delta C_{\lambda\mu\nu\kappa}(h)&=&\bigg[\tfrac{1}{8} g_{\mu \nu}  R_{\kappa \lambda} -  \tfrac{1}{8} g_{\lambda \nu} R_{\kappa \mu} -  \tfrac{1}{8} g_{\kappa \mu}  R_{\lambda \nu} +\tfrac{1}{8} g_{\kappa \lambda}  R_{\mu \nu}  
\nonumber\\
&&+ \tfrac{1}{24} g_{\kappa \mu} g_{\lambda \nu}  R -  \tfrac{1}{24} g_{\kappa \lambda} g_{\mu \nu}  R -  \tfrac{1}{4}  R_{\kappa \nu \lambda \mu}\bigg]h
\nonumber\\
&=&\frac{1}{4}hC_{\lambda\mu\nu\kappa}.
\label{AP52}
\end{eqnarray}
%
Inspection of \eqref{AP52} reveals that if the background Weyl tensor $C_{\lambda\mu\nu\kappa}$ is zero then it follows that $\delta C_{\lambda\mu\nu\kappa}$ has no dependence upon $h$. Given a vanishing background Weyl tensor, we may also observe that from (\ref{AP3}) $\delta W^{\mu\nu}$ can therebfore be expressed as
%
\begin{eqnarray}
\delta W^{\mu\nu}=2\nabla_{\kappa}\nabla_{\lambda}\delta C^{\mu\lambda\nu\kappa}-
R_{\kappa\lambda}\delta C^{\mu\lambda\nu\kappa},
\label{AP53}
\end{eqnarray}
%
and thereby also be independent of $h$. 

%%%%%%%%%%%%%%%%%%%%%%%%%%%%%%%%%%%%%%%%%%%%
\subsubsection{Differential Commutations}
\label{sss:imposing_cgauge}
%%%%%%%%%%%%%%%%%%%%%%%%%%%%%%%%%%%%%%%%%%%%

Before we can express (\ref{AP44}) as a form ready for application of the conformal gauge condition, we must first commute the differential operators as per (\ref{AP41}) and (\ref{AP46}).  On performing the commutations for $\delta W_{\mu\nu}^{}(K_{\mu\nu})$ we obtain
%
\begin{eqnarray}
&&\delta W_{\mu\nu}^{}(K_{\mu\nu})=\tfrac{1}{2} K_{\mu \nu} R_{\alpha \beta} R^{\alpha \beta} -  \tfrac{1}{2} K_{\nu}{}^{\alpha} R_{\alpha \beta} R_{\mu}{}^{\beta} -  \tfrac{2}{3} K^{\alpha \beta} R_{\alpha \beta} R_{\mu \nu} + K^{\alpha \beta} R_{\mu \alpha} R_{\nu \beta}  
\nonumber\\
&&-  \tfrac{1}{2} K_{\mu}{}^{\alpha} R_{\alpha \beta} R_{\nu}{}^{\beta} + \tfrac{1}{3} g_{\mu \nu} K^{\alpha \beta} R_{\alpha \beta} R 
+ \tfrac{1}{3} K_{\nu}{}^{\alpha} R_{\mu \alpha} R + \tfrac{1}{3} K_{\mu}{}^{\alpha} R_{\nu \alpha} R -  \tfrac{1}{6} K_{\mu \nu} R^2  
\nonumber\\
&&-  g_{\mu \nu} K^{\alpha \beta} R^{\gamma \kappa} R_{\alpha \gamma \beta \kappa} -  \tfrac{2}{3} K^{\alpha \beta} R R_{\mu \alpha \nu \beta} -  K_{\nu}{}^{\alpha} R^{\beta \gamma} R_{\mu \beta \alpha \gamma} + 2 K^{\alpha \beta} R_{\alpha}{}^{\gamma} R_{\mu \gamma \nu \beta} 
\nonumber\\
&&+ 2 K^{\alpha \beta} R_{\alpha \gamma \beta \kappa} R_{\mu}{}^{\gamma}{}_{\nu}{}^{\kappa} -  K_{\mu}{}^{\alpha} R^{\beta \gamma} R_{\nu \beta \alpha \gamma} + \tfrac{1}{3} R \nabla_{\alpha}\nabla^{\alpha}K_{\mu \nu} -  \tfrac{1}{6} K_{\mu \nu} \nabla_{\alpha}\nabla^{\alpha}R  
\nonumber\\
&&+ \tfrac{1}{2} R_{\nu}{}^{\alpha} \nabla_{\alpha}\nabla_{\beta}K_{\mu}{}^{\beta} + \tfrac{1}{2} R_{\mu}{}^{\alpha} \nabla_{\alpha}\nabla_{\beta}K_{\nu}{}^{\beta} 
-  \tfrac{1}{6} \nabla_{\alpha}K_{\mu \nu} \nabla^{\alpha}R + \tfrac{1}{6} g_{\mu \nu} \nabla^{\alpha}R \nabla_{\beta}K_{\alpha}{}^{\beta} 
\nonumber\\
&& -  \nabla_{\alpha}K^{\alpha \beta} \nabla_{\beta}R_{\mu \nu} -  \tfrac{2}{3} R_{\mu \nu} \nabla_{\beta}\nabla_{\alpha}K^{\alpha \beta} + \tfrac{1}{3} g_{\mu \nu} R \nabla_{\beta}\nabla_{\alpha}K^{\alpha \beta} -  R^{\alpha \beta} \nabla_{\beta}\nabla_{\alpha}K_{\mu \nu} 
\nonumber\\
&&-  K^{\alpha \beta} \nabla_{\beta}\nabla_{\alpha}R_{\mu \nu} + \tfrac{1}{6} g_{\mu \nu} K^{\alpha \beta} \nabla_{\beta}\nabla_{\alpha}R + \tfrac{1}{2} K_{\nu}{}^{\alpha} \nabla_{\beta}\nabla^{\beta}R_{\mu \alpha} + \tfrac{1}{2} K_{\mu}{}^{\alpha} \nabla_{\beta}\nabla^{\beta}R_{\nu \alpha}  
\nonumber\\
&&+ \tfrac{1}{2} \nabla_{\beta}\nabla^{\beta}\nabla_{\alpha}\nabla^{\alpha}K_{\mu \nu} 
-  \tfrac{1}{2} \nabla_{\beta}\nabla^{\beta}\nabla_{\mu}\nabla_{\alpha}K_{\nu}{}^{\alpha} -  \tfrac{1}{2} \nabla_{\beta}\nabla^{\beta}\nabla_{\nu}\nabla_{\alpha}K_{\mu}{}^{\alpha}  
\nonumber\\
&&-  g_{\mu \nu} R^{\alpha \beta} \nabla_{\beta}\nabla_{\gamma}K_{\alpha}{}^{\gamma}  
+ \nabla_{\alpha}R_{\nu \beta} \nabla^{\beta}K_{\mu}{}^{\alpha} + \nabla_{\alpha}R_{\mu \beta} \nabla^{\beta}K_{\nu}{}^{\alpha} 
+ \tfrac{2}{3} g_{\mu \nu} R^{\alpha \beta} \nabla_{\gamma}\nabla^{\gamma}K_{\alpha \beta} 
\nonumber\\
&& - 2 R_{\mu \alpha \nu \beta} \nabla_{\gamma}\nabla^{\gamma}K^{\alpha \beta} + \tfrac{1}{6} g_{\mu \nu} K^{\alpha \beta} \nabla_{\gamma}\nabla^{\gamma}R_{\alpha \beta} -  K^{\alpha \beta} \nabla_{\gamma}\nabla^{\gamma}R_{\mu \alpha \nu \beta}  
\nonumber\\
&&+ \tfrac{1}{6} g_{\mu \nu} \nabla_{\gamma}\nabla^{\gamma}\nabla_{\beta}\nabla_{\alpha}K^{\alpha \beta} 
+ \tfrac{1}{3} g_{\mu \nu} \nabla_{\gamma}R_{\alpha \beta} \nabla^{\gamma}K^{\alpha \beta} - 2 \nabla_{\gamma}R_{\mu \alpha \nu \beta} \nabla^{\gamma}K^{\alpha \beta} 
\nonumber\\
&&+ R_{\mu \beta \nu \gamma} \nabla^{\gamma}\nabla_{\alpha}K^{\alpha \beta} + R_{\mu \gamma \nu \beta} \nabla^{\gamma}\nabla_{\alpha}K^{\alpha \beta} -  \nabla_{\beta}R_{\nu \alpha} \nabla_{\mu}K^{\alpha \beta} 
+ \tfrac{1}{6} \nabla^{\alpha}R \nabla_{\mu}K_{\nu \alpha} 
\nonumber\\
&&-  \tfrac{1}{3} R \nabla_{\mu}\nabla_{\alpha}K_{\nu}{}^{\alpha} -  \tfrac{1}{2} R_{\nu}{}^{\alpha} \nabla_{\mu}\nabla_{\beta}K_{\alpha}{}^{\beta} + R^{\alpha \beta} \nabla_{\mu}\nabla_{\beta}K_{\nu \alpha} -  \nabla_{\beta}R_{\mu \alpha} \nabla_{\nu}K^{\alpha \beta}  
\nonumber\\
&&+ \tfrac{1}{3} \nabla_{\mu}R_{\alpha \beta} \nabla_{\nu}K^{\alpha \beta} 
+ \tfrac{1}{6} \nabla^{\alpha}R \nabla_{\nu}K_{\mu \alpha} + \tfrac{1}{3} \nabla_{\mu}K^{\alpha \beta} \nabla_{\nu}R_{\alpha \beta} -  \tfrac{1}{3} R \nabla_{\nu}\nabla_{\alpha}K_{\mu}{}^{\alpha}  
\nonumber\\
&&-  \tfrac{1}{2} R_{\mu}{}^{\alpha} \nabla_{\nu}\nabla_{\beta}K_{\alpha}{}^{\beta} + R^{\alpha \beta} \nabla_{\nu}\nabla_{\beta}K_{\mu \alpha} -  \tfrac{2}{3} R^{\alpha \beta} \nabla_{\nu}\nabla_{\mu}K_{\alpha \beta} 
+ \tfrac{1}{3} K^{\alpha \beta} \nabla_{\nu}\nabla_{\mu}R_{\alpha \beta} 
\nonumber\\
&& + \tfrac{1}{3} \nabla_{\nu}\nabla_{\mu}\nabla_{\beta}\nabla_{\alpha}K^{\alpha \beta}.
\label{AP54}
\end{eqnarray}
%

Comprising 59 terms, we note that when evaluated within a flat background, (\ref{AP54}) reduces to 
\begin{eqnarray}
\delta W_{\mu\nu}^{}(K_{\mu\nu})&=&\tfrac{1}{2} \nabla_{\beta}\nabla^{\beta}\nabla_{\alpha}\nabla^{\alpha}K_{\mu \nu} -  \tfrac{1}{2} \nabla_{\beta}\nabla^{\beta}\nabla_{\mu}\nabla_{\alpha}K_{\nu}{}^{\alpha} -  \tfrac{1}{2} \nabla_{\beta}\nabla^{\beta}\nabla_{\nu}\nabla_{\alpha}K_{\mu}{}^{\alpha}  
\nonumber\\
&&+ \tfrac{1}{6} g_{\mu \nu} \nabla_{\gamma}\nabla^{\gamma}\nabla_{\beta}\nabla_{\alpha}K^{\alpha \beta}+ \tfrac{1}{3} \nabla_{\nu}\nabla_{\mu}\nabla_{\beta}\nabla_{\alpha}K^{\alpha \beta}.
\end{eqnarray}
Hence, \eqref{AP54} reduces to the result of \eqref{AP24}, namely the simplified flat fluctuation equations without any imposition of a gauge condition - such a result provides an affirmative check on our calculation thus far.


%%%%%%%%%%%%%%%%%%%%%%%%%%%%%%%%%%%%%%%%%%%%
\subsection{$\delta W_{\mu\nu}$ in a Conformal to Flat Minkowski Background}
\label{ss:fluctuations_around_conformal_flat_cgauge}
%%%%%%%%%%%%%%%%%%%%%%%%%%%%%%%%%%%%%%%%%%%%

%%%%%%%%%%%%%%%%%%%%%%%%%%%%%%%%%%%%%%%%%%%%
\subsubsection{Implementing the Conformal Gauge Condition}
\label{sss:implementing_cgauge}
%%%%%%%%%%%%%%%%%%%%%%%%%%%%%%%%%%%%%%%%%%%%

With \eqref{AP54} now being ready for insertion of the conformal gauge, we evaluate (\ref{AP54}) in the conformal to flat background given in (\ref{AP6}) recalling that we take $\Omega(x)$ to be a completely general and arbitrary spacetime function. In the  $g_{\mu\nu}=\Omega^2(x)\eta_{\mu\nu}$ background  the gauge condition $\nabla_{\nu}K^{\mu\nu}=\frac{1}{2}K^{\mu\nu}g^{\alpha\beta}\partial_{\nu}g_{\alpha\beta}$ takes the form
%
\begin{eqnarray}
\nabla_{\nu}K^{\mu\nu}-\frac{1}{2}K^{\mu\nu}\Omega^{-2}\eta^{\alpha\beta}\eta_{\alpha\beta}\partial_{\nu}\Omega^2&=&\nabla_{\nu}K^{\mu\nu}-4\Omega^{-1}K^{\mu\nu}\partial_{\nu}\Omega 
\nonumber\\
&=&\partial_{\nu}K^{\mu\nu}+6\Omega^{-1}K^{\mu\nu}\partial_{\nu}\Omega-4\Omega^{-1}K^{\mu\nu}\partial_{\nu}\Omega
\nonumber\\
&=&\partial_{\nu}K^{\mu\nu}+2\Omega^{-1}K^{\mu\nu}\partial_{\nu}\Omega=\Omega^{-2}\partial_{\nu}(\Omega^{2}K^{\mu\nu}) 
\nonumber\\
&=&0. 
\label{AP55}
\end{eqnarray}
%
We can factor out a contribution of $\Omega^2(x)$ from the fluctuation by setting $K^{\mu\nu}=\Omega^{-2}(x)k^{\mu\nu}$ and $K_{\mu\nu}=\Omega^{2}(x)k_{\mu\nu}$. For clarification, here indices on $k^{\mu\nu}$ and $k_{\mu\nu}=\eta_{\mu\alpha}\eta_{\nu\beta}k^{\alpha\beta}$ are raised and lowered with $\eta_{\mu\nu}$. As a result, (\ref{AP55}) can thus be expressed as the familiar transverse form $\partial_{\nu}k^{\mu\nu}=0$, noting that the conformal gauge condition is such that there is no longer a dependence upon the conformal factor. 
%%%%%%%%%%%%%%%%%%%%%%%%%%%%%%%
	\footnote{Within (\ref{AP55}) we note that we have taken $\Omega(x)$ to be a general function of the coordinates so that we additionally encompass the special case of Robertson-Walker geometries with arbitrary spatial curvature $k$. As given in Appendix \ref{ab:cosmologies}, $\Omega(x)$ will only depend on the time coordinate $t$ for $k=0$, while for $k\ne 0$, $\Omega(x)$ will depend on both $t$ and the radial coordinate $r$.}
%%%%%%%%%%%%%%%%%%%%%%%%%%%%%%%
Before we alas evaluate the (\ref{AP54}) within a conformal Minkowski background in the conformal gauge given in (\ref{AP55}), we observe that the gauge condition 
%
\begin{eqnarray}
\nabla_{\nu}K^{\mu\nu}=4\Omega^{-1}K^{\mu\nu}\partial_{\nu}\Omega
\label{AP56}
\end{eqnarray}
% 
in fact possesses the same form of a covariant gauge condition for a background metric $\Omega^2(x) g_{\mu\nu}$ with any $g_{\mu\nu}$. Hence our treatment is fully covariant. It follows that when (\ref{AP56}) is imposed in a conformal to flat but not necessarily Minkowski background (e.g. polar coordinates of the form $ds^2=dt^2-dr^2-r^2d\theta^2-r^2\sin^2\theta d\phi^2$) then (\ref{AP54}) will take the form
%
\begin{eqnarray}
&&\delta W_{\mu\nu}=\frac{1}{2}\Omega^{-4}\tilde{\nabla}_{\beta}\tilde{\nabla}^{\beta}\tilde{\nabla}_{\alpha}\tilde{\nabla}^{\alpha}K_{\mu \nu}-  4\Omega^{-5} \tilde{\nabla}_{\beta}\tilde{\nabla}_{\alpha}K_{\mu \nu} \tilde{\nabla}^{\beta}\tilde{\nabla}^{\alpha}\Omega 
\nonumber\\
&&- 2\Omega^{-5}  \tilde{\nabla}_{\alpha}\tilde{\nabla}^{\alpha}\Omega \tilde{\nabla}_{\beta}\tilde{\nabla}^{\beta}K_{\mu \nu}-  4 \Omega^{-5}\tilde{\nabla}^{\alpha}\Omega \tilde{\nabla}_{\beta}\tilde{\nabla}^{\beta}\tilde{\nabla}_{\alpha}K_{\mu \nu}  
\nonumber\\
&&-  \Omega^{-5}K_{\mu \nu} \tilde{\nabla}_{\beta}\tilde{\nabla}^{\beta}\tilde{\nabla}_{\alpha}\tilde{\nabla}^{\alpha}\Omega -  4\Omega^{-5} \tilde{\nabla}_{\alpha}K_{\mu \nu} \tilde{\nabla}_{\beta}\tilde{\nabla}^{\beta}\tilde{\nabla}^{\alpha}\Omega + 6\Omega^{-6} \tilde{\nabla}_{\alpha}\Omega \tilde{\nabla}^{\alpha}\Omega \tilde{\nabla}_{\beta}\tilde{\nabla}^{\beta}K_{\mu \nu}  
\nonumber\\
&&+ 12\Omega^{-6} \tilde{\nabla}^{\alpha}\Omega \tilde{\nabla}_{\beta}\tilde{\nabla}_{\alpha}K_{\mu \nu} \tilde{\nabla}^{\beta}\Omega
+ 3\Omega^{-6} K_{\mu \nu} \tilde{\nabla}_{\alpha}\tilde{\nabla}^{\alpha}\Omega \tilde{\nabla}_{\beta}\tilde{\nabla}^{\beta}\Omega  
\nonumber\\
&&+ 12 \Omega^{-6}\tilde{\nabla}_{\alpha}K_{\mu \nu} \tilde{\nabla}^{\alpha}\Omega \tilde{\nabla}_{\beta}\tilde{\nabla}^{\beta}\Omega 
+ 24\Omega^{-6}  \tilde{\nabla}^{\alpha}\Omega \tilde{\nabla}_{\beta}K_{\mu \nu} \tilde{\nabla}^{\beta}\tilde{\nabla}_{\alpha}\Omega 
\nonumber\\
&&
+ 6\Omega^{-6} K_{\mu \nu} \tilde{\nabla}_{\beta}\tilde{\nabla}_{\alpha}\Omega \tilde{\nabla}^{\beta}\tilde{\nabla}^{\alpha}\Omega 
+ 12\Omega^{-6} K_{\mu \nu} \tilde{\nabla}^{\alpha}\Omega \tilde{\nabla}_{\beta}\tilde{\nabla}^{\beta}\tilde{\nabla}_{\alpha}\Omega  
\nonumber\\
&&-  24 \Omega^{-7}K_{\mu \nu} \tilde{\nabla}_{\alpha}\Omega \tilde{\nabla}^{\alpha}\Omega \tilde{\nabla}_{\beta}\tilde{\nabla}^{\beta}\Omega 
-  48\Omega^{-7}  \tilde{\nabla}_{\alpha}\Omega \tilde{\nabla}^{\alpha}\Omega \tilde{\nabla}_{\beta}K_{\mu \nu} \tilde{\nabla}^{\beta}\Omega 
\nonumber\\
&&
-  48\Omega^{-7}  K_{\mu \nu} \tilde{\nabla}^{\alpha}\Omega \tilde{\nabla}_{\beta}\tilde{\nabla}_{\alpha}\Omega \tilde{\nabla}^{\beta}\Omega 
+ 60\Omega^{-8} K_{\mu \nu} \tilde{\nabla}_{\alpha}\Omega \tilde{\nabla}^{\alpha}\Omega \tilde{\nabla}_{\beta}\Omega \tilde{\nabla}^{\beta}\Omega.
\label{AP57}
\end{eqnarray}
%
Here we have introduced $\tilde{\nabla}_{\alpha}$ to denote the covariant derivative with respect to the flat (but not necessarily Minkowski) background $g_{\mu\nu}$ such that $\tilde{\nabla}^{\alpha}$ is equal to $g^{\alpha\beta}\tilde{\nabla}_{\alpha}$. Observing \eqref{AP57}, we remarkably find that the 17 terms can be factored into one single compact expression
%
\begin{eqnarray}
\delta W_{\mu\nu}(K_{\mu\nu})=\frac{1}{2}\Omega^{-2}\tilde{\nabla}_{\alpha}\tilde{\nabla}^{\alpha}\tilde{\nabla}_{\beta}\tilde{\nabla}^{\beta}(\Omega^{-2}K_{\mu\nu})
=\frac{1}{2}\Omega^{-2}\tilde{\nabla}_{\alpha}\tilde{\nabla}^{\alpha}\tilde{\nabla}_{\beta}\tilde{\nabla}^{\beta}k_{\mu\nu},
\label{AP58}
\end{eqnarray}
%
where we we again note that $k_{\mu\nu}=\Omega^{-2}(x)K_{\mu\nu}$. Thus (\ref{AP58}) embodies the representation of fluctuations around a geometry that is conformal to an arbitrary flat background metric, within the  $\nabla_{\nu}K^{\mu\nu}=4\Omega^{-1}K^{\mu\nu}\partial_{\nu}\Omega$ gauge. Upon setting $\Omega(x)=1$,  (\ref{AP58}) reconciles our result within Sect. \ref{ss:fluctuations_around_flat_in_the_tranverse_gauge} as a representation of the fluctuation equations around a flat background geometry in the transverse gauge $\nabla_{\nu}K^{\mu\nu}=0$ as 
%
\begin{eqnarray}
\delta W_{\mu\nu}(K_{\mu\nu})=\frac{1}{2}\tilde{\nabla}_{\alpha}\tilde{\nabla}^{\alpha}\tilde{\nabla}_{\beta}\tilde{\nabla}^{\beta}K_{\mu\nu}.
\label{AP59}
\end{eqnarray}
%

%%%%%%%%%%%%%%%%%%%%%%%%%%%%%%%%%%%%%%%%%%%%
\subsubsection{Further Fluctuation Reduction}
\label{sss:obtaining_fluctuation_eqns_in_cgauge}
%%%%%%%%%%%%%%%%%%%%%%%%%%%%%%%%%%%%%%%%%%%%

Although the forms of (\ref{AP58}) and (\ref{AP59}) are exceedingly simple, from a practical perspective, these are not of straightforward use since they involve covariant derivatives that mix the various components of $K_{\mu\nu}$. To fix this issue, we note that (\ref{AP58}) and (\ref{AP59}) also apply in the gauge given in (\ref{AP55}). Hence within the conformal gauge, fluctuations around a conformal to Minkowski geometry take the partial derivative form
%
\begin{eqnarray}
&&\delta W_{\mu\nu}(K_{\mu\nu})=\frac{1}{2}\Omega^{-4}\partial_{\beta}\partial^{\beta}\partial_{\alpha}\partial^{\alpha}K_{\mu \nu}-  4\Omega^{-5} \partial_{\beta}\partial_{\alpha}K_{\mu \nu} \partial^{\beta}\partial^{\alpha}\Omega 
\nonumber\\
&&- 2\Omega^{-5}  \partial_{\alpha}\partial^{\alpha}\Omega \partial_{\beta}\partial^{\beta}K_{\mu \nu}-  4 \Omega^{-5}\partial^{\alpha}\Omega \partial_{\beta}\partial^{\beta}\partial_{\alpha}K_{\mu \nu}  
-  \Omega^{-5}K_{\mu \nu} \partial_{\beta}\partial^{\beta}\partial_{\alpha}\partial^{\alpha}\Omega 
\nonumber\\
&& -  4\Omega^{-5} \partial_{\alpha}K_{\mu \nu} \partial_{\beta}\partial^{\beta}\partial^{\alpha}\Omega 
+ 6\Omega^{-6} \partial_{\alpha}\Omega \partial^{\alpha}\Omega \partial_{\beta}\partial^{\beta}K_{\mu \nu} + 12\Omega^{-6} \partial^{\alpha}\Omega \partial_{\beta}\partial_{\alpha}K_{\mu \nu} \partial^{\beta}\Omega 
\nonumber\\
&&+ 3\Omega^{-6} K_{\mu \nu} \partial_{\alpha}\partial^{\alpha}\Omega \partial_{\beta}\partial^{\beta}\Omega + 12 \Omega^{-6}\partial_{\alpha}K_{\mu \nu} \partial^{\alpha}\Omega \partial_{\beta}\partial^{\beta}\Omega+ 24\Omega^{-6}  \partial^{\alpha}\Omega \partial_{\beta}K_{\mu \nu} \partial^{\beta}\partial_{\alpha}\Omega  
\nonumber\\
&&+ 6\Omega^{-6} K_{\mu \nu} \partial_{\beta}\partial_{\alpha}\Omega \partial^{\beta}\partial^{\alpha}\Omega 
+ 12\Omega^{-6} K_{\mu \nu} \partial^{\alpha}\Omega \partial_{\beta}\partial^{\beta}\partial_{\alpha}\Omega -  24 \Omega^{-7}K_{\mu \nu} \partial_{\alpha}\Omega \partial^{\alpha}\Omega \partial_{\beta}\partial^{\beta}\Omega  
\nonumber\\
&&-  48\Omega^{-7}  \partial_{\alpha}\Omega \partial^{\alpha}\Omega \partial_{\beta}K_{\mu \nu} \partial^{\beta}\Omega-  48\Omega^{-7}  K_{\mu \nu} \partial^{\alpha}\Omega \partial_{\beta}\partial_{\alpha}\Omega \partial^{\beta}\Omega 
\nonumber\\
&&
+ 60\Omega^{-8} K_{\mu \nu} \partial_{\alpha}\Omega \partial^{\alpha}\Omega \partial_{\beta}\Omega \partial^{\beta}\Omega.
\label{AP60}
\end{eqnarray}
%
Inspection of \eqref{AP60} reveals that it remarkably simplifies to
%
\begin{eqnarray}
\delta W_{\mu\nu}(K_{\mu\nu})=\frac{1}{2}\Omega^{-2}\eta^{\sigma\rho}\eta^{\alpha\beta}\partial_{\sigma}\partial_{\rho} \partial_{\alpha}\partial_{\beta}(\Omega^{-2}K_{\mu\nu})
=\frac{1}{2}\Omega^{-2}\eta^{\sigma\rho}\eta^{\alpha\beta}\partial_{\sigma}\partial_{\rho} \partial_{\alpha}\partial_{\beta}k_{\mu\nu}.
\label{AP61}
\end{eqnarray}
%
As written, (\ref{AP61}) may be recognized as being of the exact form given within Sect. \ref{ss:conformal_invariance} based on grounds on conformal invariance and transformation properties of the Bach tensor. Regardless of the $g_{\mu\nu}=\Omega^2(x)\eta_{\mu\nu}$ background not being flat, in (\ref{AP60}) and (\ref{AP61}) all derivatives are flat Minkowski (i.e. associated with the metric $ds^2=-\eta_{\alpha\beta}dx^{\alpha}dx^{\beta}=dt^2-dx^2-dy^2-dz^2$). In terms of these partial derivatives, we observe a significant feature in that (\ref{AP60}) and (\ref{AP61}) are diagonal in the $(\mu,\nu)$ indices (i.e. there is no mixing of the components of $k_{\mu\nu}$ from the different operator).  Consequently, the reductive significance of conformal symmetry is exemplified in our starting point with a the 62 term $\delta W_{\mu\nu}(h_{\mu\nu})$ given in (\ref{AP43}) and arriving at the single term (\ref{AP61}).

\subsection{Calculation Summary}
\label{ss:summary_cgauge}

As we have covered numerous steps in the derivation of the fluctuations within conformal gravity, we provide a summary overview of the procedure given. We began with a general $W_{\mu\nu}$ in the form of in (\ref{AP42}) and then perturbed $W_{\mu\nu}$ to first order around a general background  with metric $g_{\mu\nu}$ with a perturbed metric of the form  $g_{\mu\nu}+\delta g_{\mu\nu}=g_{\mu\nu}+h_{\mu\nu}$. With the identity 
%
\begin{eqnarray}
\delta R^{\lambda}_{\phantom{\lambda}\mu\nu\kappa}&=&
\partial \delta\Gamma^{\lambda}_{\mu\nu}/\partial x^{\kappa}
+\Gamma^{\lambda}_{\kappa\sigma}\delta\Gamma^{\sigma}_{\mu\nu}
-\Gamma^{\sigma}_{\mu\kappa}\delta\Gamma^{\lambda}_{\nu\sigma}
-\partial \delta\Gamma^{\lambda}_{\mu\kappa}/\partial x^{\nu}
-\Gamma^{\lambda}_{\nu\sigma}\delta\Gamma^{\sigma}_{\mu\kappa}
+\Gamma^{\sigma}_{\mu\nu}\delta\Gamma^{\lambda}_{\kappa\sigma}
\nonumber\\
&=&
\nabla_{\kappa}\delta\Gamma^{\lambda}_{\mu\nu}
-\nabla_{\nu}\delta\Gamma^{\lambda}_{\mu\kappa},
\end{eqnarray}
%
where $\delta\Gamma^{\lambda}_{\mu\nu}=(1/2)g^{\lambda \rho}[\nabla_{\nu}\delta g_{\rho\mu}+\nabla_{\mu}\delta g_{\rho\nu}-\nabla_{\rho}\delta g_{\mu\nu}]$, we then evaluate $\delta W_{\mu\nu}$ to lowest order in $\delta g_{\mu\nu}$ to obtain (\ref{AP43}). 

To address trace simplifications, in (\ref{AP43}) we set  $h_{\mu\nu}=K_{\mu\nu}+(1/4)g_{\mu\nu}h$ where $h=g^{\mu\nu}h_{\mu\nu}$ and $g^{\mu\nu}K_{\mu\nu}=0$, from which it follows that $\delta W_{\mu\nu}$ may be expressed as two contributions viz. $\delta W_{\mu\nu}(K_{\mu\nu})$ in (\ref{AP44}) and $\delta W_{\mu\nu}(h)$ in (\ref{AP45}). Commuting covariant derivatives we observed and confirmed that $\delta W_{\mu\nu}(h)=-(1/4)W_{\mu\nu}h$ as shown in (\ref{AP50}). Consequently, this has established that $\delta W_{\mu\nu}(h)$ will vanish if the background $W_{\mu\nu}$ vanishes (a result that applied to the backgrounds of Robertson-Walker and de Sitter cosmologies).

For geometries in which the background $W_{\mu\nu}$ vanishes, $\delta W_{\mu\nu}$ reduces to $\delta W_{\mu\nu}(K_{\mu\nu})$, with $\delta W_{\mu\nu}$ to then only be dependent on the traceless fluctuation $K_{\mu\nu}$ as given in (\ref{AP44}). With further commutation of covariant derivatives we then expressed (\ref{AP44}) in the form given in (\ref{AP54}). Now in a form ready for conformal gauge implementation, we apply condition (\ref{AP23}) within a conformal flat background to yield $\delta W_{\mu\nu}(K_{\mu\nu})$ given as (\ref{AP61}), the main result of this section.


%%%%%%%%%%%%%%%%%%%%%%%%%%%%%%%%%%%%%%%%%%%%
\section{Imposing Specific Gauges within $\delta G_{\mu\nu}$}
\label{s:compact_expressions_ein}
%%%%%%%%%%%%%%%%%%%%%%%%%%%%%%%%%%%%%%%%%%%%

With  background plus fluctuation metric of the form $ds^2=\Omega^2(x)[dt^2-\delta_{ij}dx^idx^j]-\Omega^2(x)f_{\mu\nu}dx^{\mu}dx^{\nu}$, the fluctuation in the Einstein tensor is given by 
%
\begin{eqnarray}
&&\delta G_{\mu\nu}=- \tfrac{1}{2}\tilde{\nabla}_{\alpha}\tilde{\nabla}_{\mu}f_{\nu}{}^{\alpha} -  \tfrac{1}{2} \tilde{\nabla}_{\alpha}\tilde{\nabla}_{\nu}f_{\mu}{}^{\alpha} -  \eta^{\alpha \beta} \eta_{\mu \nu} \Omega^{-1}\tilde{\nabla}_{\alpha}f \tilde{\nabla}_{\beta}\Omega + \eta^{\alpha \beta} \Omega^{-1} \tilde{\nabla}_{\alpha}f_{\mu \nu} \tilde{\nabla}_{\beta}\Omega
\nonumber\\
&& + \eta^{\beta \alpha} f_{\mu \nu} \Omega^{-2}\tilde{\nabla}_{\alpha}\Omega \tilde{\nabla}_{\beta}\Omega 
-  \tfrac{1}{2} \eta^{\alpha \beta} \eta_{\mu \nu} \tilde{\nabla}_{\beta}\tilde{\nabla}_{\alpha}f + \tfrac{1}{2} \eta_{\mu \nu}\tilde{\nabla}_{\beta}\tilde{\nabla}_{\alpha}f^{\alpha \beta} + \tfrac{1}{2} \eta^{\alpha \beta} \tilde{\nabla}_{\beta}\tilde{\nabla}_{\alpha}f_{\mu \nu}
\nonumber\\
&& + 2 \eta_{\mu \nu} f^{\alpha \beta} \Omega^{-1} \tilde{\nabla}_{\beta}\tilde{\nabla}_{\alpha}\Omega - 2 \eta^{\alpha \beta} f_{\mu \nu} \Omega^{-1} \tilde{\nabla}_{\beta}\tilde{\nabla}_{\alpha}\Omega 
+ 2 \eta^{\alpha \gamma} \eta_{\mu \nu} \Omega^{-1} \tilde{\nabla}_{\beta}f_{\alpha}{}^{\beta} \tilde{\nabla}_{\gamma}\Omega 
\nonumber\\
&&-  \eta^{\alpha \gamma} \eta^{\beta \kappa} \eta_{\mu \nu} f_{\alpha \beta} \Omega^{-2} \tilde{\nabla}_{\gamma}\Omega\tilde{\nabla}_{\kappa}\Omega -  \eta^{\alpha \beta} \Omega^{-1}\tilde{\nabla}_{\beta}\Omega \tilde{\nabla}_{\mu}f_{\nu \alpha} -  \eta^{\alpha \beta} \Omega^{-1}\tilde{\nabla}_{\beta}\Omega \tilde{\nabla}_{\nu}f_{\mu \alpha} 
\nonumber\\
&&+ \tfrac{1}{2} \tilde{\nabla}_{\nu}\tilde{\nabla}_{\mu}f,
\label{AP76}
\end{eqnarray}
%
where $\Omega(x)$ is an arbitrary function of $x_{\mu}$. In our exploration of various gauges below, we make use of the trace free contribution of the conformally factored metric perturbation $k_{\mu\nu}=f_{\mu\nu}-(1/4)\eta^{\alpha\beta}f_{\alpha\beta}$ where the tracelessness is defined with respect to the background $\eta^{\mu\nu}k_{\mu\nu}=0$. With the substitution of $k_{\mu\nu}$, we may express (\ref{AP76}) as 
%
\begin{eqnarray}
\delta G_{\mu\nu}&=&- \tfrac{1}{4} \eta^{\alpha \beta} \eta_{\mu \nu} \Omega^{-1} \partial_{\alpha}\Omega \partial_{\beta}f + \eta^{\alpha \beta} \Omega^{-1} \partial_{\alpha}k_{\mu \nu} \partial_{\beta}\Omega + \eta^{\beta \alpha} k_{\mu \nu} \Omega^{-2} \partial_{\alpha}\Omega \partial_{\beta}\Omega  
\nonumber\\
&&+ \tfrac{1}{2} \eta^{\alpha \beta} \partial_{\beta}\partial_{\alpha}k_{\mu \nu} -  \tfrac{1}{4} \eta^{\alpha \beta} \eta_{\mu \nu} \partial_{\beta}\partial_{\alpha}f 
- 2 \eta^{\alpha \beta} k_{\mu \nu} \Omega^{-1} \partial_{\beta}\partial_{\alpha}\Omega -  \tfrac{1}{2} \eta^{\alpha \beta} \partial_{\beta}\partial_{\mu}k_{\nu \alpha}  
\nonumber\\
&&-  \tfrac{1}{2} \eta^{\alpha \beta} \partial_{\beta}\partial_{\nu}k_{\mu \alpha} + 2 \eta^{\alpha \beta} \eta^{\gamma \kappa} \eta_{\mu \nu} \Omega^{-1} \partial_{\beta}\Omega \partial_{\kappa}k_{\alpha \gamma} 
-  \eta^{\alpha \gamma} \eta^{\beta \kappa} \eta_{\mu \nu} k_{\alpha \beta} \Omega^{-2} \partial_{\gamma}\Omega \partial_{\kappa}\Omega 
\nonumber\\
&& + \tfrac{1}{2} \eta^{\alpha \beta} \eta^{\gamma \kappa} \eta_{\mu \nu} \partial_{\kappa}\partial_{\beta}k_{\alpha \gamma} + 2 \eta^{\alpha \beta} \eta^{\gamma \kappa} \eta_{\mu \nu} k_{\alpha \gamma} \Omega^{-1} \partial_{\kappa}\partial_{\beta}\Omega -  \eta^{\alpha \beta} \Omega^{-1} \partial_{\beta}\Omega \partial_{\mu}k_{\nu \alpha} 
\nonumber\\
&&-  \eta^{\alpha \beta} \Omega^{-1} \partial_{\beta}\Omega \partial_{\nu}k_{\mu \alpha} -  \tfrac{1}{4} \Omega^{-1} \partial_{\mu}\Omega \partial_{\nu}f -  \tfrac{1}{4} \Omega^{-1} \partial_{\mu}f \partial_{\nu}\Omega + \tfrac{1}{4} \partial_{\nu}\partial_{\mu}f,
\label{D1}
\end{eqnarray}
%
with $f$ denoting $\eta^{\alpha\beta}f_{\alpha\beta}$. As our goal is to reduce the \eqref{D1} into as compact form as possible, we explore a most general set of possible gauges with variable coefficients in the form of the gauge condition
%
\begin{eqnarray}
\eta^{\alpha\beta}\partial_{\alpha}k_{\beta\nu} = \Omega^{-1} J \eta^{\alpha\beta}k_{\nu\alpha}\partial_\beta \Omega + P \partial_\nu f+ R \Omega^{-1} f\partial_\nu \Omega.
\label{D2}
\end{eqnarray}
%
Here $J$, $P$, and $R$ represent the constant coefficients of which we will vary. On taking  $J = -2$, $P = 1/2$, $R = 0$ the gauge condition becomes
%
\begin{eqnarray}
\eta^{\alpha\beta}\partial_{\alpha}k_{\beta\nu} = -2 \Omega^{-1}  \eta^{\alpha\beta}k_{\nu\alpha}\partial_\beta \Omega + \tfrac{1}{2} \partial_\nu f.
\label{D3}
\end{eqnarray}
%
If we elect to take an $\Omega$  that is strictly dependent upon conformal time $\tau$, then evaluation of $\delta G_{\mu\nu}$ leads to
%
\begin{eqnarray}
\delta G_{00}&=&
(3 \Omega^{-2} \dot{\Omega}^2 -  \Omega^{-1} \ddot{\Omega} + \tfrac{1}{2} \eta^{\mu \nu} \partial_{\mu} \partial_{\nu} -  \Omega^{-1} \dot{\Omega} \partial_{0}) k_{00} - \tfrac{1}{4} ( \Omega^{-1} \dot{\Omega} \partial_{0} + \partial_{0} \partial_{0}) f,
\nonumber\\
\delta G_{0i}&=&
(\Omega^{-1} \ddot{\Omega} + \tfrac{1}{2} \eta^{\mu \nu} \partial_{\mu} \partial_{\nu} -  \Omega^{-1} \dot{\Omega} \partial_{0}) k_{0i} - \tfrac{1}{4}  (\Omega^{-1} \dot{\Omega} \partial_{i} +\partial_{i} \partial_{0}) f,
\nonumber\\
\delta G_{ij}&=&
\delta_{ij}(-2 \Omega^{-2} \dot{\Omega}^2 + \Omega^{-1} \ddot{\Omega}) k_{00} + (- \Omega^{-2} \dot{\Omega}^2 + 2 \Omega^{-1} \ddot{\Omega} + \tfrac{1}{2} \eta^{\mu \nu} \partial_{\mu} \partial_{\nu}  
\nonumber\\
&&-  \Omega^{-1} \dot{\Omega} \partial_{0}) k_{ij}, 
- \tfrac{1}{4} (\delta_{ij} \Omega^{-1} \dot{\Omega} \partial_{0} + \partial_{i} \partial_{j}) f,
\nonumber\\
\eta^{\mu\nu}\delta G_{\mu\nu}&=&(-10\Omega^{-2} \dot{\Omega}^2 +6  \Omega^{-1} \ddot{\Omega})k_{00}- \tfrac{1}{4}  (2\Omega^{-1} \dot{\Omega} \partial_{0} +\eta^{\mu\nu}\partial_{\mu}\partial_{\nu}) f,
\label{D4}
\end{eqnarray}
%
where as usual a dot denotes a derivative with respect to $\tau$, and where $\partial_0$ denotes $\partial_{\tau}$. We see that $\delta G_{\mu\nu}$ is now expressed in a form that nearly decouples each tensor component, with it being the $k_{00}$ and $f$ dependence  $\delta G_{\mu\nu}$ that prevents it from doing so. However, one can solve the equations exactly once $\delta T_{\mu\nu}$ is specified, as one can use the $\delta G_{00}$ and $\eta^{\mu\nu}\delta G_{\mu\nu}$ equations to can determine $k_{00}$ and $f$, and then from $\delta G_{\mu\nu}$ one can determine all remaining components of $k_{\mu\nu}$. 


To illustrate an example, if the background is the de Sitter geometry, i.e. where we set $\Omega(\tau)=1/H\tau$ with $H$ constant, then (\ref{D4}) takes the form
%
\begin{eqnarray}
\delta G_{00}&=&
(\tau^{-2} + \tfrac{1}{2} \eta^{\mu \nu} \partial_{\mu} \partial_{\nu} + \tau^{-1} \partial_{0}) k_{00} + \tfrac{1}{4} (\tau^{-1} \partial_{0} -   \partial_{0} \partial_{0}) f,
\nonumber\\
\delta G_{0i}&=&
(2 \tau^{-2} + \tfrac{1}{2} \eta^{\mu \nu} \partial_{\mu} \partial_{\nu} + \tau^{-1} \partial_{0}) k_{0i} + \tfrac{1}{4} ( \tau^{-1} \partial_{i} -  \partial_{i} \partial_{0}) f,
\nonumber\\
\delta G_{ij}&=&
(3 \tau^{-2} + \tfrac{1}{2} \eta^{\mu \nu} \partial_{\mu} \partial_{\nu} + \tau^{-1} \partial_{0}) k_{ij} + \tfrac{1}{4} (\delta_{ij} \tau^{-1} \partial_{0} -   \partial_{i} \partial_{j}) f,
\nonumber\\
\eta^{\mu\nu}\delta G_{\mu\nu}&=&2\tau^{-2}k_{00}+ \tfrac{1}{4}  (2\tau^{-1}\partial_{0} -\eta^{\mu\nu}\partial_{\mu}\partial_{\nu}) f.
\label{D5}
\end{eqnarray}
%
These equations form a compact set of reduced terms, a form which was possible by the explicit incorporation of the conformal factor $\Omega(\tau)$ into both the background and the fluctuation (which also holds for (\ref{D4}). Hence, in this gauge the $\delta G_{\mu\nu}=-8\pi G \delta T_{\mu\nu}$ fluctuation equations facilitate a readily integrable solution.

Another convenient decomposition of $\delta G_{\mu\nu}$ corresponds to the gauge choice $J=-4$, $R=2P-3/2$, with $P$ arbitrary. In the de Sitter background, the fluctuations take the form
%
\begin{eqnarray}
\delta G_{00}&=&(-2 \tau^{-2}
+ \tfrac{1}{2} \eta^{\mu \nu} \partial_{\mu} \partial_{\nu}
+ 3 \tau^{-1} \partial_{0}) k_{00}
+ \big[(\tfrac{3}{4} -  P) \tau^{-2} 
\nonumber\\
&&
+ \tfrac{1}{4}(1-2P) \eta^{\mu \nu} \partial_{\mu} \partial_{\nu}
+ P \tau^{-1} \partial_{0}
+( \tfrac{1}{4} -P)\partial^2_{0}\big]f,
\nonumber\\
\delta G_{0i}&=&\tau^{-1} \partial_{i} k_{00}
+ (\tau^{-2}
+ \tfrac{1}{2} \eta^{\mu \nu} \partial_{\mu} \partial_{\nu}
+ 2 \tau^{-1} \partial_{0}) k_{0i}
+ \big[(P- \tfrac{1}{2}) \tau^{-1} \partial_{i} 
\nonumber\\
&&
+ (\tfrac{1}{4}-P) \partial_{i} \partial_{0} \big]f,
\nonumber\\
\delta G_{ij}&=&\delta_{ij}\tau^{-2} k_{00}
+\tau^{-1} \partial_{j} k_{0i}
+ \tau^{-1} \partial_{i} k_{0j}
+ (3 \tau^{-2}
+ \tfrac{1}{2} \eta^{\mu \nu} \partial_{\mu} \partial_{\nu}
+ \tau^{-1} \partial_{0}) k_{ij}
\nonumber\\
&+& \delta_{ij}\left[(\tfrac{3}{4}-P) \tau^{-2}
+ \tfrac{1}{4}(2P-1) \eta^{\mu \nu} \partial_{\mu} \partial_{\nu}
+(P-1)\tau^{-1} \partial_{0}\right]f 
\nonumber\\
&&
+ (\tfrac{1}{4}-P) \partial_{i} \partial_{j}f.
\\
\eta^{\alpha\beta}\delta G_{\alpha\beta}&=&(P-\tfrac{3}{4})( \eta^{\alpha\beta}\partial_{\alpha}\partial_{\beta}f +4\tau^{-1}\partial_0 f-6\tau^{-2}f)=(P-\tfrac{3}{4})\tau^2 \eta^{\alpha\beta}\partial_{\alpha}\partial_{\beta}(\tau^{-2}f).
\nonumber
\label{D6}
\end{eqnarray}
%
We see conveniently that $\eta^{\mu\nu}\delta G_{\mu\nu}=\Omega^2g^{\mu\nu}\delta G_{\mu\nu}$ depends only upon the trace $f$ of the fluctuation. Hence we can foremost solve for $f$ and then proceed to the other components of the fluctuation in turn. We may further observe the component $\eta^{\alpha\beta}\delta G_{\alpha\beta}$ takes the form of the flat space free massless particle wave operator acting on $\tau^{-2} f$. Hence, we may solve $g^{\mu\nu}\delta G_{\mu\nu}=-8\pi G g^{\mu\nu}\delta T_{\mu\nu}$ via integration by the $D^{(4)}(x-y)$ Green's function obeying $\eta^{\alpha\beta}\partial_{\alpha}\partial_{\beta}D^{(4)}(x-y)=\delta^4(x-y)$. We thus can form the exact integral solution as
%
\begin{eqnarray}
f=\eta^{\mu\nu}f_{\mu\nu}=-\frac{8\pi G}{(P-\tfrac{3}{4})}\tau^2(x) \int d^4yD^{(4)}(x-y)\tau^{-2}(y)\eta^{\mu\nu}\delta T_{\mu\nu}(y),
\label{D7}
\end{eqnarray}
%
where it is implied that $\tau(x)=x^0$, $\tau(y)=y^0$.

%%%%%%%%%%%%%%%%%%%%%%%%%%%%%%%%%%%%%%%%%%%%
\section{Robertson Walker Radiation Era Conformal Gravity Solution}
\label{s:rw_radiation_conformal_gravity_sol}
%%%%%%%%%%%%%%%%%%%%%%%%%%%%%%%%%%%%%%%%%%%%

Motivated by the excellent agreement between the fits to the galactic rotation curves of 138 spiral galaxies presented in \cite{mannheim_2011,mannheim_2012, obrien_mannheim_2012} and to fits of the accelerating universe Hubble plot data presented in \cite{mannheim_2006,mannheim_2017}, we thus consider conformal gravity fluctuations in Robertson-Walker cosmologies with negative $k$.

In the conformal gravity fitting of the current era Hubble plot, we note from \cite{mannheim_1990} that the cosmological constant term is found to dominate over the perfect fluid contribution. However, in the early universe, the radiation era perfect fluid is the dominant contribution since $a(t)$ is small and $A/a^4(t)$ is large. Moreover, if the $A/a^4(t)$ radiation contribution is dominant, then since $k$ is given by $k=-\dot{a}^2-2A/a^2S_0^2$ when the $\alpha$ contribution in (\ref{A18}) is negligible, we are phenomenologically steered towards negative $k$. Thus for studying fluctuation growth in the early universe the only relevant solution for $a(t)$ is that of $a(t,\alpha=0,k<0,A>0)$. Using this solution, upon setting $k=-1/L^2$, $d^2=2AL^4/S_0^2$, and $L^2a^2(t)=(d^2+t^2)$, we obtain
%
\begin{eqnarray}
\tau=L\int_0^t \frac{dt}{(d^2+t^2)^{1/2}}=L~{\rm arc sinh}\left(\frac{t}{d}\right),\qquad t=d\sinh p,
\label{B1}
\end{eqnarray}
%
where $p=\tau/L$. We have seen that from (\ref{AP29}), fluctuations around a flat background grow linearly in the relevant time variable, which according to the $k<0$ (\ref{A14}) is $p^{\prime}$. Upon setting $\Omega(p,\chi)=La(p)(\cosh p+\cosh \chi)$, we may express (\ref{A14}) in the form
%
\begin{eqnarray}
ds^2=\Omega^2(p,\chi)\left[dp^{\prime 2}-dx^{\prime 2} -dy^{\prime 2} -dz^{\prime 2}\right],
\label{B2}
\end{eqnarray}
%
i.e. the conformal to flat form. From Sect. \ref{s:conformal_gauge_sols}, we recall that the conformal gravity fluctuation (\ref{AP61}) takes the form
%
\begin{eqnarray}
\delta W_{\mu\nu}=(1/2)\Omega^{-2}\eta^{\sigma\rho}\eta^{\alpha\beta}\partial_{\sigma}\partial_{\rho} \partial_{\alpha}\partial_{\beta}k_{\mu\nu}
\end{eqnarray}
%
where $k_{\mu\nu}=\Omega^{-2}(x)K_{\mu\nu}$. Solving this fourth-order wave equation in terms of momentum eiginstates as done in Sect. \ref{ss:fluctuations_around_flat_in_the_tranverse_gauge} (cf. (\ref{AP29})), we express the solution to $\delta W_{\mu\nu}=0$ in the primed-variable form
%
\begin{eqnarray}
k_{\mu\nu}=A^{\prime}_{\mu\nu}e^{ik^{\prime}\cdot x^{\prime}}+(n^{\prime}\cdot x^{\prime})B^{\prime}_{\mu\nu}e^{ik^{\prime}\cdot x^{\prime}}+A^{\prime *}_{\mu\nu}e^{-ik^{\prime}\cdot x^{\prime}}+(n^{\prime}\cdot x^{\prime})B^{\prime *}_{\mu\nu}e^{-ik^{\prime}\cdot x^{\prime}},
\label{B3}
\end{eqnarray}
%
where $\eta^{\mu\nu}k^{\prime}_{\mu}k^{\prime}_{\nu}=0$. Within this coordinate basis, the conformal gauge condition takes the form $\partial^{\prime}_{\nu}k^{\mu\nu}=0$, with $k^{\mu\nu}=\eta^{\mu\alpha}\eta^{\nu\beta}k_{\alpha\beta}$. Substituting \eqref{B3} into the conformal gauge, it follows that
%
\begin{eqnarray}
&&ik^{\prime \nu}\left[A^{\prime}_{\mu\nu}e^{ik^{\prime}\cdot x^{\prime}}+(n^{\prime}\cdot x^{\prime})B^{\prime}_{\mu\nu}e^{ik^{\prime}\cdot x^{\prime}}-A^{\prime *}_{\mu\nu}e^{-ik^{\prime}\cdot x^{\prime}}-(n^{\prime}\cdot x^{\prime})B^{\prime *}_{\mu\nu}e^{-ik^{\prime}\cdot x^{\prime}}\right]
\nonumber\\
&&+n^{\prime\nu}\left[B^{\prime}_{\mu\nu}e^{ik^{\prime}\cdot x^{\prime}}+B^{\prime *}_{\mu\nu}e^{-ik^{\prime}\cdot x^{\prime}}\right]=0.
\label{B4}
\end{eqnarray}
%
For these expression to hold for all $x'$, it then follows that we must have
%
\begin{eqnarray}
&&ik^{\prime \nu}A^{\prime}_{\mu\nu}+n^{ \prime\nu}B^{\prime}_{\mu\nu}=0,\quad ik^{\prime \nu}B^{\prime}_{\mu\nu}=0, 
\nonumber\\
&&
-ik^{\prime \nu}A^{\prime *}_{\mu\nu}+n^{ \prime\nu}B^{\prime *}_{\mu\nu}=0,\quad -ik^{\prime \nu}B^{\prime *}_{\mu\nu}=0.
\label{B5}
\end{eqnarray}
%
Inspection of \eqref{B3} show that the $B^{\prime}_{\mu\nu}$ term has leading order large time behavior via its $n^{\prime}\cdot x^{\prime}$ prefactor. Consequently, continuing to work to leading order, we ignore the non-leading $A^{\prime}_{\mu\nu}$ modes and may take the $B^{\prime}_{\mu\nu}$ modes to obey the transverse momentum space conditions $ik^{\prime \nu}B^{\prime}_{\mu\nu}=0$ and $-ik^{\prime \nu}B^{\prime *}_{\mu\nu}=0$. It follows then that components of the $B^{\prime}_{\mu\nu}$ modes have the same $(n^{\prime}\cdot x^{\prime})e^{ik^{\prime}\cdot x^{\prime}}$ leading behavior in coordinate space. More specifically, since $n^{\prime \mu}=(1,0,0,0)$ the fluctuations grow linearly in the time variable $p^{\prime}$ associated with (\ref{B2}).

At present, we have expressed the solutions in terms of the conformal to flat coordinate system; however, to make contact with their leading order behavior in terms of the canonical comoving coordinates $t$ and $r$, we need to both reexpress $\Omega(p,\chi)$ in terms of the comoving coordinates $(t,r)$ and transform the components of the fluctuation $K_{\mu\nu}$ to the comoving coordinates. Starting first with $\Omega(p,\chi)$, we note that from $K_{\mu\nu} = \Omega^2 k_{\mu\nu}$ and \eqref{AP61} that the $K_{\mu\nu}$ fluctuations grow as $\Omega^2(p,\chi)p^{\prime}$. Making use of (\ref{A14}) and (\ref{A11}) from Appendix \ref{ab:cosmologies}, and coordinate transformation (\ref{B1}) one calculates $\Omega^2(p,\chi)p^{\prime}$ to obtain
%
\begin{eqnarray}
\Omega^2(p,\chi)p^{\prime}&=&L^2a^2(p)(\cosh p+\cosh \chi)^2p^{\prime}=L^2a^2(p)\sinh p (\cosh p+\cosh \chi)
\nonumber\\
&=&(d^2+t^2)\frac{t}{d}\left[\left(1+\frac{t^2}{d^2}\right)^{1/2}+\left(1+\frac{r^2}{L^2}\right)^{1/2}\right].
\label{B6}
\end{eqnarray}
%
Analysis of \eqref{B6} reveals that $\Omega^2(p,\chi)p^{\prime}$ for $t\ll d$ grows linearly in $t$, for later times and grows as $t^4$ when $t\gg d$. Consequently, in the given conformal to flat Minkowski coordinate system of (\ref{A14}), we see that the leading time behavior ($t \gg d$ specifically) of all the components of $K_{\mu\nu}$ is proportional to $t^4$.

The second task that remains is to transform the fluctuation $K_{\mu\nu}$ itself. To this end, one can note that in transforming from (\ref{A14}) to (\ref{A13}), there is no alteration of the behavior of $p^{\prime}$, with all dependence relegated to the conformal factor. However, one must additional be careful to note that the final transformation between spatial Cartesian coordinates (\ref{A14})  and polar coordinates (\ref{A13}) does in fact introduce a dependence on $r^{\prime}$. One can observe that, any angular component induces a dependence on the comoving $t$. For instance, in terms of the $K_{x^{\prime}x^{\prime}}$ type fluctuations in the (\ref{A14}) coordinate system, we have the following transformation relations in the (\ref{A13}) coordinate system
%
\begin{align}
K_{\theta\theta}&=(r^{\prime}\cos\theta\cos\phi)^2K_{x^{\prime}x^{\prime}}+(r^{\prime}\cos\theta\sin\phi)^2K_{y^{\prime}y^{\prime}}+(r^{\prime}\sin\theta)^2K_{z^{\prime}z^{\prime}}
\nonumber\\
&=\frac{\sinh^2\chi}{(\cosh p+\cosh \chi)^2}[\cos^2\theta\cos^2\phi K_{x^{\prime}x^{\prime}}+\cos^2\theta\sin^2\phi K_{y^{\prime}y^{\prime}}+\sin^2\theta K_{z^{\prime}z^{\prime}}]
\nonumber\\
&=\frac{r^2d^2}{[L(d^2+t^2)^{1/2}+d(L^2+r^2)^{1/2}]^2}\times
\nonumber\\
&\qquad[\cos^2\theta\cos^2\phi K_{x^{\prime}x^{\prime}}+\cos^2\theta\sin^2\phi K_{y^{\prime}y^{\prime}}+\sin^2\theta K_{z^{\prime}z^{\prime}}],
\nonumber\\
K_{r^{\prime}\theta}&=r^{\prime}\cos\theta\sin\theta\cos^2\phi K_{x^{\prime}x^{\prime}}+r^{\prime}\cos\theta\sin\theta\sin^2\phi K_{y^{\prime}y^{\prime}}-r^{\prime}\sin\theta\cos\theta K_{z^{\prime}z^{\prime}}
\nonumber\\
&=\frac{\sinh\chi}{(\cosh p+\cosh \chi)}\times 
\nonumber\\
&\qquad[\cos\theta\sin\theta\cos^2\phi K_{x^{\prime}x^{\prime}}+\cos\theta\sin\theta\sin^2\phi K_{y^{\prime}y^{\prime}}-\sin\theta\cos\theta K_{z^{\prime}z^{\prime}}],
\nonumber\\
K_{p^{\prime}\theta}&=r^{\prime}\cos\theta\cos\phi K_{p^{\prime}x^{\prime}}+r^{\prime}\cos\theta\sin\phi K_{p^{\prime}y^{\prime}}-r^{\prime}\sin\theta K_{p^{\prime}z^{\prime}}
\nonumber\\
&=\frac{\sinh\chi}{(\cosh p+\cosh \chi)}[\cos\theta\cos\phi K_{p^{\prime}x^{\prime}}+\cos\theta\sin\phi K_{p^{\prime}y^{\prime}}-\sin\theta K_{p^{\prime}z^{\prime}}]. 
\label{B7}
\end{align}
%
We may continue to compute the analogous expressions for $K_{\theta\phi}$, $K_{\phi\phi}$, $K_{r^{\prime}\phi}$ and $K_{p^{\prime}\phi}$. Within \eqref{B7}, we see that the $r^{\prime}=\sinh \chi/(\cosh p +\cosh \chi)$ prefactor has leading comoving time behavior as $t^0$ if $p=\chi$, (i.e. $t=r$ with both $t$ and $r$ large) or as $t^{-1}$ if $p\gg \chi$ (i.e. $t \gg r$). These correspond to the lightlike and timelike modes respectively. 

In transforming from from (\ref{A13}) to (\ref{A4}), we note that the angular sector is unaffected by the transformation. Thus the angular sector fluctuations $K_{\theta\theta}$, $K_{\theta  \phi}$, $K_{\phi\phi}$ associated with the comoving Robertson-Walker geometry given in (\ref{A4}) will have leading order growth as $t^4$ when including the contribution of the prefactor in (\ref{B7}) for lightlike coordinates and as $t^2$ for the timelike case. Moreover, since the lightlike $ds^2=0$ is both general coordinate invariant and conformal invariant, lightlike modes associated with the (\ref{A14}) metric will transform into lightlike modes associated with the metric (\ref{A4}). Finally, we note that a $t^4$ growth for the lightlike modes is a rather significant growth rate, one that cannot be obtained from standard Einstein gravity when using the same radiation matter source.

To further address the non-angular modes, we recall that the full coordinate transformation of the fluctuation is given by
%
\begin{eqnarray}
K_{\mu\nu}=\frac{\partial x^{\prime \alpha}}{\partial x^{\mu}}\frac{\partial x^{\prime\beta}}{\partial x^{\nu}}K^{\prime}_{\alpha\beta},
\label{B8}
\end{eqnarray}
%
and transformations between the $(p^{\prime}, r^{\prime})$, $(p,\chi)$ and $(t,r)$ take the form
%
\begin{eqnarray}
&&\frac{\partial p^{\prime }}{\partial p}=\frac{\partial r^{\prime }}{\partial \chi}=\frac{1+\cosh p\cosh\chi}{[\cosh p+\cosh\chi]^2},\qquad
\frac{\partial p^{\prime }}{\partial \chi}=\frac{\partial r^{\prime }}{\partial p}=-\frac{\sinh p\sinh\chi}{[\cosh p+\cosh\chi]^2}, 
\nonumber\\
&& \frac{\partial p}{\partial t}=\frac{1}{La(t)},\qquad \frac{\partial \chi}{\partial r}=\frac{1}{L\cosh\chi}.
\label{B9}
\end{eqnarray}
%

To interpret (\ref{B9}), we must determine the relation between $p$ and $\chi$, namely whether $p=\chi$ or $p\gg \chi$. Inspecting (\ref{B1}), we may observe that $t=d\sinh p$ when $La(t)=(d^2+t^2)^{1/2}$. Hence for large $t$ with $p=\chi$ also large, we obtain
%
\begin{eqnarray}
\frac{\partial p^{\prime }}{\partial p}=\frac{\partial r^{\prime }}{\partial \chi}=1,\qquad
\frac{\partial p^{\prime }}{\partial \chi}=\frac{\partial r^{\prime }}{\partial p}=1,\qquad \frac{\partial p}{\partial t}=\frac{1}{t},\qquad \frac{\partial \chi}{\partial r}=\frac{d}{Lt}.
\label{B10}
\end{eqnarray}
%
Consequently, as we go from the  $(p^{\prime},r^{\prime})$ coordinates to $(p,\chi)$ coordinates, the leading time behavior is unchanged. In further transforming (\ref{A10}) to the comoving (\ref{A4}) ($(p,\chi)$ to $(t,r)$) we obtain $1/t^2$ suppression in the $K_{tt}$, $K_{tr}$ and $K_{rr}$ sectors, a $1/t$ suppression for $K_{t\theta}$, $K_{t\phi}$, $K_{r\theta}$ and $K_{r\phi}$, and no suppression for $K_{\theta\theta}$, $K_{\theta\phi}$ and $K_{\phi\phi}$.

Lastly, we must include the contribution of the $\Omega^2(p,\chi)p^{\prime}\propto t^4$ prefactor and obtain $K_{tt}$, $K_{tr}$ and $K_{rr}$  growing as $t^2$, $K_{t\theta}$, $K_{t\phi}$, $K_{r\theta}$ and $K_{r\phi}$ growing as $t^3$, and $K_{\theta\theta}$, $K_{\theta\phi}$ and $K_{\phi\phi}$ growing as $t^4$. For this large $t$ and large $p=\chi$ scenario, the $K_{tt}$, $K_{tr}$, $K_{rr}$, $K_{t\theta}$, $K_{t\phi}$, $K_{r\theta}$ and $K_{r\phi}$ are suppressed with respect to $K_{\theta\theta}$, $K_{\theta\phi}$ and $K_{\phi\phi}$, so the leading growth will be the $t^4$ growth associated with the angular $K_{\theta\theta}$, $K_{\theta\phi}$ and $K_{\phi\phi}$. 


Now investigating the leading order behavior for large $p\gg \chi$, the transformations take the form
%
\begin{eqnarray}
&&\frac{\partial p^{\prime }}{\partial p}=\frac{\partial r^{\prime }}{\partial \chi}=\frac{d\cosh\chi}{t},\qquad
\frac{\partial p^{\prime }}{\partial \chi}=\frac{\partial r^{\prime }}{\partial p}=-\frac{d\sinh\chi}{t}, 
\nonumber\\
&& \frac{\partial p}{\partial t}=\frac{1}{t},\qquad \frac{\partial \chi}{\partial r}=\frac{1}{L\cosh\chi}.
\label{B11}
\end{eqnarray}
%
As we go from the  $(p^{\prime},r^{\prime})$ coordinates to $(p,\chi)$ coordinates, incorporating the $t^4$ dependence of $\Omega^2(p,\chi)p^{\prime}$ and including the prefactor in (\ref{B7}), we proceed analogous to before and determine growth of $K_{tt}\sim t^0$, $K_{tr}\sim t^1$, $K_{t\theta}\sim t^1$, $K_{t\phi}\sim t^1$, while $K_{rr}\sim t^2$, $K_{r\theta}\sim t^2$, $K_{r\phi}\sim t^2$,  $K_{\theta\theta}\sim t^2$, $K_{\theta\phi}\sim t^2$ and $K_{\phi\phi}\sim t^2$.
%%%%%%%%%%%%%%%%%
\footnote{Such suppression in this case can be viewed as the following: if $\chi$ is negligible then so is $r$, and thus the spatial part of the metric in (\ref{A4}) effectively becomes flat.}
%%%%%%%%%%%%%%%.
Hence, for large $p \gg \chi$,  $K_{tr}$, $K_{t\theta}$ and $K_{t\phi}$ all have $t^1$ leading order time behavior, and $K_{rr}$, $K_{r\theta}$, $K_{r\phi}$,  $K_{\theta\theta}$, $K_{\theta\phi}$ and $K_{\phi\phi}$  all have $t^2$ leading order behavior, with $t^2$ being the overall leading growth.



