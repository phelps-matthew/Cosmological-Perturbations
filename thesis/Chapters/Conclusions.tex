\chapter{Conclusions}
\label{C7}

In the analysis of cosmological perturbations, we have delved into the two methods of constructing and solving the equations of motion: the SVT decomposition of the fluctuations and the imposition of specific gauge constraints. As regards the former, we have seen that the SVT basis provides a very convenient formalism in expressing the 10 degrees of freedom within the metric perturbation $h_{\mu\nu}$. With the SVT3 quantities being defined as spatial integrals of $h_{\mu\nu}$, the SVT construction itself is thus intrinsically non-local, with the existence of the integrals themselves requiring proper asymptotic convergence. Upon implementing the SVT basis within the cosmological fluctuation equations as applied to de Sitter and Roberston Walker backgrounds, one can readily form a set of six gauge invariant combinations comprised of two scalars, a two-component transverse vector, and a traceless rank two tensor. 

Expressing the fluctuation equations solely in terms of these gauge invariants, one can solve them exactly and achieve an decoupling of the gauge invariants by application of higher derivatives which serve to project out the longitudinal and transverse components. By solving these higher derivative equations of motion exactly, we are able to assess the necessary conditions required for the decomposition theorem - a theorem asserting that scalars, vectors, and tensors decouple in the equations of motion themselves - to hold. We find that solely by imposing the same asymptotic boundary conditions at $r=\infty$ that are needed to set up the scalar, vector, tensor basis in the first place, one then indeed does get the decomposition theorem for fluctuations around a flat background, around a de Sitter background or around a spatially flat Robertson-Walker background with $k=0$. However, for fluctuations around a Robertson-Walker background with $k\neq 0$ one additionally has to require that fluctuations be well-behaved at $r=0$ in order to get the decomposition theorem to hold. The distinction between these two classes of solution is that in the first three (flat, de Sitter, $k=0$ Robertson-Walker) the background metric can be written as a conformal to flat metric with a conformal factor is solely time dependent, i.e. $\Omega(\tau)$. In this class, the SVT gauge invariant combinations are $\phi + \psi + \dot B - \ddot E$,  $B_i-\dot{E}_i$, $E_{ij}$, and $-\psi+\Omega^{-1}\dot{\Omega}(B-\dot{E})$ (up to a minor variation given the selected linear combination of scalars). In the second class ($k\neq 0$ Robertson-Walker) the background metric can be written as a conformal to flat metric with the conformal factor depending on both $r$ and $\tau$, i.e. $\Omega(r,\tau)$. Here the gauge invariant combinations are $\phi + \psi + \dot B - \ddot E$,  $B_i-\dot{E}_i$, $E_{ij}$, and $\psi-\Omega^{-1}\dot{\Omega}(B-\dot{E})-\Omega^{-1}\delta^{ij}\partial_j\Omega(E_i+\partial_iE)$ (up to a minor variation given the selected linear combination of scalars), where the evident intertwining of SVT scalars and vectors thus prevents establishing a decomposition theorem. Hence, in conformal to flat backgrounds asymptotic boundary conditions alone will in general only secure the decomposition theorem if the conformal factor depends solely on the conformal time. However using the underlying gauge freedom in the theory one can find gauges in which the scalars and vectors are not intertwined to thus recover the decomposition theorem.

Given that the SVT3 basis is not manifestly covariant with, for instance, scalars transforming into vectors under the full four dimensional transformations, we have introduced an alternative SVT4 formalism (and more generally developed the $D$ dimensional SVTD basis) that appropriately matches the underlying transformation group. When contrasted with SVT3, implementation of the SVT4 formalism provides significant simplification, with the fluctuation equations being able to be expressed in more compact and readily solvable forms. Similar to SVT3, one can apply higher derivatives in order to separate the scalar, vector, and tensor sectors at the level of the equations of motion. However, to obtain a decomposition theorem, one need not only impose good asymptotic spatial behavior but also must require initial conditions as well. 

As regards the second method of solution where one directly imposes gauge conditions (without enacting a decomposition), we have constructed a gauge condition appropriate to conformal gravity that remains invariant under conformal transformations. Referred to as the conformal gauge, upon implementing it within a generical conformal to flat background, one can express the entire perturbative Bach tensor succinctly as
%
\begin{eqnarray}
\delta W_{\mu\nu}(K_{\mu\nu})&=&\frac{1}{2}\Omega^{-2}\eta^{\sigma\rho}\eta^{\alpha\beta}\partial_{\sigma}\partial_{\rho} \partial_{\alpha}\partial_{\beta}(\Omega^{-2}K_{\mu\nu})
\nonumber\\
&=&\frac{1}{2}\Omega^{-2}\eta^{\sigma\rho}\eta^{\alpha\beta}\partial_{\sigma}\partial_{\rho} \partial_{\alpha}\partial_{\beta}k_{\mu\nu}.
\nonumber
\end{eqnarray}
%
with $K_{\mu\nu}$ being the traceless component of the general metric perturbation $h_{\mu\nu}$. With the $k=-1$ Roberston Walker geometry being the relevant cosmology within conformal gravity, we solve the radiation era fluctuation equations exactly. In terms of momentum eigenstates of the fourth order equation of motion, we find that the solutions grow as $t^4$, with $t$ the comoving time. Such can be contrasted with the $t^{1/2}$ growth one obtains in standard Einstein radiation era cosmology. Finally, we evaluate a large set of possible gauges within Einstein gravity with a de Sitter background, finding compact forms that facilitate exact integral solutions, with coupling only occurring between the trace and the remaining metric components. 