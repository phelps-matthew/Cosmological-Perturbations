\chapter{Conclusions}
\label{C7}




arise in regards to spatially asymptotic


at hand that one may use to solve the fluctuation equations, within this work we instill both. 

the perturbative equations of motion have historically been acknowledged as forming a rather complex set of coupled non-linear differential tensor equations. For instance, within Weinberg's cosmology book \cite{weinberg_2008}, he notes that even after performing additional simplifications that these equations are still ``fearsomely complicated''. 






extensive methods of simplifying the equations of motion and eliminating non-physical gauge modes are required in order to construct the perturbative solutions
%%%%%%%%%%%%%%%%%%%%%%%%%%%%%%%%%%%%%%%%%%%%%%%%%%%%%%%%%%%%%%%%%%%%%%%%%%%%%%%%%%%%%%%%%%%%%%%%%%%%%


*touch on decomposition theorem. maybe a little mock example
*conformal gravity - while we do not dig into details on qm here, has been proposed as a ... citations. dope as properties
*overview section
*imposing gauges or for intrinsic gauge invariants. difficult problem to solve, complicated fluctuations. as we move to curved spacetimes, solving not straightforward even with gauge constraints
** SVT3, scalar modes, CMB power spectrum. then motivate SVT4. projectors


%%%%%%%%%%%%%%%%%%%%%%%%%%%%%%%%%%%%%%%%%%%%%%%%%%%%%%%%%%%%%%%%%%%%%%%%%%%%%%%%%%%%%%%%%%%%%%%%%%%%%
solve equations, Determine conditions required for decomposition theorem. Does not hold unless further input. by going to higher derivatives. See if we can impose asymptotic boundary conditions. 

start introducing perfect fluid source. RW k=0 radiation. Determine matter + gravitaional gauge invariants. $\tau^2 e_ij$ sector goes as $t^{1/2}$, decomp follows w/ spatially asympt. bc's.

generalize to all RW, perturb perfect fluid, requires equation of state. identify gauge invariants, use many vuarvature relations, commutations 4th order. seek help from svt in terms of $h_{\mu\nu}$ to determine gauge invariants here. In order to solve, need to determine $\Omega(\tau)$ and reduce 11 dof's to 10. We then determine the form of $\Omega(\tau)$ in all curvatures in radiation and matter dominated. Reduce from 11 to 10 by specifying equation of state. We interpolate btween radiation and matter: transition between two eras is complicationed, but propto $p=w\rho$ in high temp (radiation) and low temp (matter). Transition era = recombination. $p=w\rho$ not always valid. Solve by suming over complete basis of modes associated with propagation of spinless massive particle in chosen $g_{\mu\nu}$ background. Complicated, but not done here generally. 

k=-1 RW general As the implications of boundary conditions are very sensitive to the sign of the coefficient of $k$, and we will need to monitor both positive and negative coefficient cases below. In implementing evolution equations that involve products of derivative operators such as the generic $(\tilde{\nabla}^2+\alpha)(\tilde{\nabla}^2+\beta)F=0$ . scalar sector checks out given good behavior (bounded) at infinity and origin. Seem to find vector that is bounded, well behaved at both, but does not obey decomp theorem. see end of vector section. Same for tensor sector.

n Sec. \ref{ss:rw_k=-1_svt3} we have seen that there are realizations of the evolution equations in the scalar, vector, and tensor sectors that would not lead to a decomposition theorem in those sectors. However, equally there are other realizations that given the boundary conditions would lead to a decomposition theorem. Thus we need to determine which realizations are the relevant ones. To this end we look not at the individual higher-derivative equations obeyed by the separate scalar, vector, and tensor sectors, but at how these various sectors interface with each other in the original second-order $\Delta_{\mu\nu}=0$ equations themselves. Any successful such interface would require that all the terms in $\Delta_{\mu\nu}=0$ would have to have the same $\chi$ behavior. Noting that the scalar modes appear with two $\tilde{\nabla}$ derivatives in $\Delta_{ij}=0$, the vector sector appears with one $\tilde{\nabla}$ derivative and the tensor appears with none, we need to compare derivatives of scalars with vectors and derivatives of vectors with tensors. 

If we force bc that vector and tensor modes vanish at $\chi=\infty$ instead of limiting to a constant value, then decomp holds. 

compute svt3 conformal gravity, instead working in conformal flat. Imposing boundary conditions leads to simple evolution equations. We can invert svt3 quantities in terms of $\delta W_{\mu\nu}$, to serve as alternative integral relations in the RW background. 

SVT4 minkowski, delta G is purely gauge invariant in zero background. Evaluate in ds4 w/o conformal factor. Make use of SVTD in constant 3 space. For scalar $\chi$ to obey decomp theorem, require very particular solution. General solution not at all forced to $\chi =0$; specific solution to the full evolution equations. No compelling reason to choose so. $F_{\mu\nu}$ and $\chi$ can still be localized in space, thus no spatially asymptotic bc could affect them. Completely solvable though. Could enforce decomp theorem with judiciously choosing IC's at an intial time. No compelling rationale for doing so. 

ds4 with a conformal factor. GI mixes scalars and vectors. Introduce $U^\mu$ to express covariantly. Find exact solutions. Again, same story. 

Do SVT4 desitter in conformal gravity. Simple structure, in fact TT sector has same form as standard Einstein gravity. Find relation between the two. Below, and also for flat space. Decomp is automatic, only $F_{\mu\nu}$. 
\begin{eqnarray}
\delta W_{\mu\nu}=(\nabla_{\alpha}\nabla^{\alpha}-4H^2)(\delta G_{\mu\nu}+\delta T_{\mu\nu})^{T\theta}.
\end{eqnarray}

By introducing timelike $U^\mu$ can express generalized SVT4 RW flucations in compact covariant form. Same story with decomp. 

dW conformal to flat. Beautiful super simple. 
\begin{eqnarray}
\delta W_{\mu\nu}&=&\Omega^{-2}\partial_{\sigma}\partial^{\sigma}\partial_{\tau}\partial^{\tau}F_{\mu\nu}.
\end{eqnarray} 

by looking at ds4 svt4 we saw mixing, and thus one shold only look for decomp theorem for gauge invariants and not for seaprate scalar vector, tensors sects and gauage invariance can in general intertwine them. We present example occuring in SVT3 to show not just artifact of SVT4. 

ads svt3 mixing. 
\begin{eqnarray}
\delta = \phi -\psi + \dot B - \ddot E + \frac{2}{z}(\tilde\nabla_3 E + E_3),
\end{eqnarray}

General conformal to flat. $eta=\psi -\Omega^{-1}\dot{\Omega}(B-\dot E)+\Omega^{-1}\tilde\nabla^i\Omega(E_i+\tilde\nabla_i E)$ Spatially dependent $\Omega(x)$ leads to inseperable gauge invariant not foudn in non-conformal geoms. Procedure is one must first determine GI's, then separate, not other way around. For some geoms no choice of coords can undo intertwiing (if conformal factor pulled out). To express the $\eta$ in terms of a curvature invariant, one cannot make recourse to $\delta W_{\mu\nu}$, however, in traceless radiation we can take $\delta(g_{\mu\nu}G^{\mu\nu}$ to determine $\eta$ in terms of the $h_{\mu\nu}$. 

Using gauge freedom, imposing gauge, we can decouple intertwining in gauge invariants.