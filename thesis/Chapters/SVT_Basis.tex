
\chapter{Scalar, Vector, Tensor (SVT) Decomposition}
\label{c:scalar_vector_tensor_basis}

In the field of perturbative cosmology, it is standard to first introduce a 3+1 decomposition of the metric perturbation followed by a decomposition into  scalars, vectors, and tensors according to the underlying background 3-space. A la the SVT decomposition \cite{ellis_maartens_maccallum_2009, bertschinger_2000, bardeen_1980, mukhanov_1992}, referred to as SVT3 with this work. 

In this chapter, we first develop the SVT3 formalism by separately forming relations between SVT3 components and integrals over $h_{\mu\nu}$, as well as their inversions (i.e. higher derivative relations between $h_{\mu\nu}$ and SVT3 quantities.) In utilizing the SVT3 decomposition in conformal gravity, we also form the analogous relations between the traceless $K_{\mu\nu}$ and the requisite SVT3 components.

We then investigate the behavior of the SVT3 quantities under gauge transformations, and along with reference to the flat space gauge invariant Einstein tensor $\delta G_{\mu\nu}$, we construct the set of gauge invariant quantities. (In fact, such a gauge invariant construction lies behind the core utility of implementing the SVT3 decomposition in the first place). In forming the gauge invariants, we highlight their dependence upon underlying assumptions of being asymptotically well-behaved and we investigate the consequence of gauge invariance absent of any defined asymptotic behavior. 

To provide a manifestly covariant decomposition, we introduce the general SVTD decomposition for scalars, vectors, and tensors within arbitrary dimension $D$ \cite{phelps_2019}. Analgous to the SVT3 discussion, we provide integral and derivative relations between SVT4 components and $h_{\mu\nu}$ and find that the cosmological fluctuation equations take a considerably more compact and simple form in the new SVT4 formalism. Being simpler, these equations are thus more straightforward to solve. The gauge invariants are then constructed and the role of asymptotic behavior within the establishment of the SVT4 decomposition itself is analyzed. 

With help from the structure of the conformal gravity fluctuation $\delta W_{\mu\nu}$, we are able to determine the relationship between the SVT3 and SVT4 quantities themselves.

Finally, we introduce and analyze a core element of this work called the cosmological decomposition theorem. This theorem asserts that SVT3 scalars, vectors, and tensors decouple within the fluctuation equations \emph{themselves}, to thus evolve independently. We carefully inspect the dependence of the decomposition upon underlying constraints related to boundary conditions, determining that the theorem requires fluctuations to be well behaved asymptotically for such a equation decomposition to occur. The analysis is repeated with respect to the SVT4 decomposition, where the validity of the decomposition does not hold generally but rather must be determined on a case by case basis according the specific background geometry.
%%%%%%%%%%%%%%%%%%%%%%%%%%%%%%%%%%%%%
\section{SVT3}
\label{s:svt3}
%%%%%%%%%%%%%%%%%%%%%%%%%%%%%%%%%%%%%
%
The discussion of the three dimensional SVT expansion begins by taking a flat background geometry of the form $ds^2=dt^2-\delta_{ij}dx^idx^j$ where $\delta_{ij}$ represents a generic flat 3-space metric (equating to the Kronecker delta for a Minkowski background). Upon introducing a metric fluctuation $h_{\mu\nu}$ and performing a 3+1 decomposition, the geometry may be written as
%
%%%%%%%%
	\footnote{In application to cosmological backgrounds, we will find it convenient to decompose the fluctuation around a conformal to flat background by incorporating an explicit factor of $\Omega^2(x)$, with the perturbed geometry taking the form
	\begin{eqnarray}
	ds^2 &=& \Omega^2(x) \bigg[ (1+2\phi) dt^2 -2(\tilde{\nabla}_i B +B_i)dt dx^i - [(1-2\psi)\delta_{ij} +2\tilde{\nabla}_i\tilde{\nabla}_j E
	\nonumber\\
	&& + \tilde{\nabla}_i E_j + \tilde{\nabla}_j E_i + 2E_{ij}]dx^i dx^j\bigg].
	\label{AP62}
	\end{eqnarray}
	%
	Here $\Omega(x)$ is an arbitrary function of the coordinates, where $\tilde{\nabla}_i=\partial/\partial x^i$ (with Latin index) and  $\tilde{\nabla}^i=\delta^{ij}\tilde{\nabla}_j$ (not $\Omega^{-2}\delta^{ij}\tilde{\nabla}_j$) are defined with respect to the background 3-space metric $\delta_{ij}$. SVT3 elements obey the same relations as in \eqref{APsvt3_rel}, i.e. transverse and traceless with respect to the background 3-space metric.}
%%%%%%%%
%
\begin{eqnarray}
ds^2 &=&(-\eta_{\mu\nu}-h_{\mu\nu})dx^{\mu}dx^{\nu}
\nonumber\\
&=&(1+2\phi) dt^2 -2(\tilde{\nabla}_i B +B_i)dt dx^i - [(1-2\psi)\delta_{ij} +2\tilde{\nabla}_i\tilde{\nabla}_j E 
\nonumber\\
&&+ \tilde{\nabla}_i E_j + \tilde{\nabla}_j E_i + 2E_{ij}]dx^i dx^j,
\label{2.1}
\end{eqnarray}
%
where $\tilde{\nabla}_i=\partial/\partial x^i$ and  $\tilde{\nabla}^i=\delta^{ij}\tilde{\nabla}_j$  (with Latin indices) are defined with respect to the background three-space metric $\delta_{ij}$. In addition, the SVT3 components within (\ref{2.1}) are required to obey
%
\begin{eqnarray}
\delta^{ij}\tilde{\nabla}_j B_i = 0,\quad \delta^{ij}\tilde{\nabla}_j E_i = 0, \quad E_{ij}=E_{ji},\quad \delta^{jk}\tilde{\nabla}_kE_{ij} = 0, \quad \delta^{ij}E_{ij} = 0.
\label{2.2}
\label{APsvt3_rel}
\end{eqnarray}
%
As written, (\ref{2.1}) contains ten elements, whose transformations are defined with respect to the background spatial sector as four 3-dimensional scalars ($\phi$, $B$, $\psi$, $E$), two transverse 3-dimensional vectors ($B_i$, $E_i$) each with two independent degrees of freedom, and one symmetric 3-dimensional transverse-traceless tensor ($E_{ij}$) with two degrees of freedom. A la, the scalar, vector, tensor (SVT) decomposition. 

To implement the decomposition of $h_{\mu\nu}$ to the SVT3 form in \eqref{2.1}, we utilize transverse and transverse-traceless projection operators as applied to tensor and vector components to yield a decomposition into scalars, vectors, and tensors. Both the 3+1 decomposition and projection operators have been derived in developed in detail within Appendix \ref{aa:svt_projection}.
%
%%%%%%%%%%%%%%%%%%%%%%%%%%%%%%%%%%%%%
\subsection{SVT3 in Terms of $h_{\mu\nu}$ in a Conformal Flat Background}
%%%%%%%%%%%%%%%%%%%%%%%%%%%%%%%%%%%%%

Following \cite{amarasinghe_2019, phelps_2019} and making use of the projection operators in Appendix \ref{aa:svt_projection}, we express the ten degrees of freedom of the SVT3 components in a conformal to flat background in terms of the original fluctuations $h_{\mu\nu}$. First we introduce the 3-dimensional Green's function obeying
%
\begin{eqnarray}
\delta^{ij}\tilde{\nabla}_i\tilde{\nabla}_jD^{(3)}(\mathbf{x}-\mathbf{y})=\delta^3(\mathbf{x}-\mathbf{y}).
\label{AP64}
\end{eqnarray}
%
Upon setting $h_{\mu\nu}=\Omega^2(x)f_{\mu\nu}$, the line element of (\ref{AP62}) takes the form 
%
\begin{eqnarray}
ds^2&=&-[\Omega^2(x)\eta_{\alpha\beta}+h_{\alpha\beta}]dx^{\alpha}dx^{\beta}
\nonumber\\
&=&-\Omega^2(x)[\eta_{\alpha\beta}+f_{\alpha\beta}]dx^{\alpha}dx^{\beta}
\nonumber\\
&=&\Omega^2(x)\left[dt^2-\delta_{ij}dx^idx^j-f_{00}dt^2-2f_{0i}dtdx^i-f_{ij}dx^idx^j\right],
\nonumber\\
\delta^{ij}f_{ij}&=&-6\psi+2\tilde{\nabla}_i\tilde{\nabla}^iE,
\tilde{\nabla}^jf_{ij}=-2\tilde{\nabla}_i\psi+2\tilde{\nabla}_i\tilde{\nabla}_k\tilde{\nabla}^kE+\tilde{\nabla}_k\tilde{\nabla}^kE_{i},
\nonumber\\
\tilde{\nabla}^i \tilde{\nabla}^jf_{ij}&=&-2\tilde{\nabla}_k\tilde{\nabla}^k\psi+2\tilde{\nabla}_k\tilde{\nabla}^k\tilde{\nabla}_{\ell}\tilde{\nabla}^{\ell}E
\nonumber\\
&=&\frac{4}{3}\tilde{\nabla}_k\tilde{\nabla}^k\tilde{\nabla}_{\ell}\tilde{\nabla}^{\ell}E+\frac{1}{3}\tilde{\nabla}_k\tilde{\nabla}^k\delta^{ij}f_{ij}
\nonumber\\
&=&4\tilde{\nabla}_k\tilde{\nabla}^k\psi+\tilde{\nabla}_k\tilde{\nabla}^k(\delta^{ij}f_{ij}),
\nonumber\\
2\phi&=&-f_{00},\qquad
B=\int d^3yD^{(3)}(\mathbf{x}-\mathbf{y})\tilde{\nabla}_y^if_{0i},\qquad B_i=f_{0i}-\tilde{\nabla}_iB,
\nonumber\\
\psi&=&\frac{1}{4}\int d^3yD^{(3)}(\mathbf{x}-\mathbf{y})\tilde{\nabla}_y^k\tilde{\nabla}_y^{\ell}f_{k\ell}-\frac{1}{4}\delta^{k\ell}f_{k\ell},
\nonumber\\
\qquad
E&=&\int d^3yD^{(3)}(\mathbf{x}-\mathbf{y})\left[\frac{3}{4}\int d^3zD^{(3)}(\mathbf{y}-\mathbf{z})\tilde{\nabla}_z^k\tilde{\nabla}_z^{\ell}f_{k\ell}-\frac{1}{4}\delta^{k\ell}f_{k\ell}\right],
\nonumber\\
E_i&=&\int d^3yD^{(3)}(\mathbf{x}-\mathbf{y})\bigg{[}\tilde{\nabla}_y^jf_{ij}
-\tilde{\nabla}^y_i\int d^3zD^{(3)}(\mathbf{y}-\mathbf{z})\tilde{\nabla}_z^k\tilde{\nabla}_z^{\ell}f_{k\ell}\bigg{]},
\nonumber\\
2E_{ij}&=&f_{ij}+2\psi\delta_{ij} -2\tilde{\nabla}_i\tilde{\nabla}_j E - \tilde{\nabla}_i E_j - \tilde{\nabla}_j E_i, 
\label{AP65}
\end{eqnarray}
%
One may readily check that $B_i$, $E_i$, and $E_{ij}$ are indeed transverse by applying appropriate derivatives, thus confirming their obeying (\ref{APsvt3_rel}).
%%%%%
 \footnote{In (\ref{AP65}) a symbol such as $\tilde{\nabla}_y^i$, $y$ indicates that the derivative is taken with respect to the $y$ coordinate and likewise for other latin coordinates.}
 %%%%%
The integral form of the inversions of the SVT3 components is unique up to integration by parts, which plays a role in the analysis of asymptotic behavior, discussed in detail within Sect. \ref{ss:gauge_struct_svt3}.

We)where here and throughout we use the notation given in \cite{weinberg_1972}

%%%%%%%%%%%%%%%%%%%%%%%%%%%%%%%%%%%%%
\subsection{SVT3 in Terms of the Traceless $k_{\mu\nu}$ in a Conformal Flat Background}
\label{ss:svt3_in_terms_of_k_mu_nu}
%%%%%%%%%%%%%%%%%%%%%%%%%%%%%%%%%%%%%
We have shown in Sect. \ref{s:conformal_gravity} that in conformal to flat backgrounds, the perturbed Bach tensor $\delta W_{\mu\nu}$ may be expressed entirely in terms of the traceless $K_{\mu\nu}$. As such, it will prove useful to be able to express the SVT components in terms of the traceless part of $f_{\mu\nu}$. Defining $K_{\mu\nu}=\Omega^2k_{\mu\nu}$, we have
\begin{eqnarray}
K_{\mu\nu}=h_{\mu\nu}-(1/4)\Omega^2\eta_{\mu\nu}\Omega^{-2}\eta^{\alpha\beta}h_{\alpha\beta}=h_{\mu\nu}-(1/4)\eta_{\mu\nu}\eta^{\alpha\beta}h_{\alpha\beta},
\end{eqnarray}
whereby we factor out the conformal factor to form the traceless $k_{\mu\nu}$ as
\begin{eqnarray}
k_{\mu\nu}=f_{\mu\nu}-(1/4)\eta_{\mu\nu}[-f_{00}+\delta^{ij}f_{ij}].
\end{eqnarray}
%
Substituting $f_{\mu\nu}$ in terms of this $k_{\mu\nu}$, we obtain from \eqref{AP65} the following integral relations for the SVT components:
\begin{eqnarray}
k_{00}&=&\frac{3}{4}f_{00}+\frac{1}{4}\delta^{k\ell}f_{k\ell},\qquad k_{0i}=f_{0i},\qquad k_{ij}=f_{ij}+\frac{1}{4}\delta_{ij}f_{00}-\frac{1}{4}\delta_{ij}\delta^{k\ell}f_{k\ell},
\nonumber\\
\phi&=&-\frac{1}{2}f_{00},\qquad
B=\int d^3yD^{(3)}(\mathbf{x}-\mathbf{y})\tilde{\nabla}_y^ik_{0i},\qquad B_i=k_{0i}-\tilde{\nabla}_iB,
\nonumber\\
\psi&=&\frac{1}{4}\int d^3yD^{(3)}(\mathbf{x}-\mathbf{y})\tilde{\nabla}_y^k\tilde{\nabla}_y^{\ell}k_{k\ell}-\frac{3}{4}k_{00}+\frac{1}{2}f_{00},
\nonumber\\
\qquad
E&=&\int d^3yD^{(3)}(\mathbf{x}-\mathbf{y})\left[\frac{3}{4}\int d^3zD^{(3)}(\mathbf{y}-\mathbf{z})\tilde{\nabla}_z^k\tilde{\nabla}_z^{\ell}k_{k\ell}-\frac{1}{4}k_{00}\right],
\nonumber\\
E_i&=&\int d^3yD^{(3)}(\mathbf{x}-\mathbf{y})\bigg{[}\tilde{\nabla}_y^jk_{ij}
-\tilde{\nabla}^y_i\int d^3zD^{(3)}(\mathbf{y}-\mathbf{z})\tilde{\nabla}_z^k\tilde{\nabla}_z^{\ell}k_{k\ell}\bigg{]},
\nonumber\\
2E_{ij}&+&2\tilde{\nabla}_i\tilde{\nabla}_j E +\tilde{\nabla}_i E_j +\tilde{\nabla}_j E_i
=k_{ij}-\frac{1}{2}\delta_{ij}k_{00}
\nonumber\\
&&\qquad\qquad\qquad\qquad\qquad\qquad+\frac{1}{2}\delta_{ij}\int d^3yD^{(3)}(\mathbf{x}-\mathbf{y})\tilde{\nabla}_y^k\tilde{\nabla}_y^{\ell}k_{k\ell}.
\label{AP66}
\end{eqnarray}
%
Here can see that all SVT3 components can be expressed in terms of $k_{\mu\nu}$ along with a single component of $f_{\mu\nu}=\Omega^{-2}(x)h_{\mu\nu}$, namely $f_{00}$.  Recalling that $\delta W_{\mu\nu}$ can only depend on $k_{\mu\nu}$, we note that the combination $\phi+\psi$ is independent of $f_{00}$ and only depends on $k_{\mu\nu}$. Hence, we expect this coupled combination to be associated with the scalar SVT component of conformal gravity. Indeed, we confirm such a relation later in Sect. \ref{ss:deltaW_conformal_flat_SVT3}.

%%%%%%%%%%%%%%%%%%%%%%%%%%%%%%%%%%%%%
\subsection{Gauge Structure and Asymptotic Behavior}
\label{ss:gauge_struct_svt3}
%%%%%%%%%%%%%%%%%%%%%%%%%%%%%%%%%%%%%
As given in \eqref{2.1} and its integral form in \eqref{AP65}, we have shown the form of the SVT3 decomposition of $h_{\mu\nu}$ comprising 10 independent components of scalars, vectors, and tensors. Due to the coordinate freedom, it must hold that linear combinations of the SVT quantities form precisely six gauge invariant quantities (a reduction from ten initial degrees of freedom minus four coordinate transformations). Consequently, we seek to determine the coefficient combinations of the SVT quantities that form the gauge invariants. In general, this may be accomplished by manipulating the relations between the SVT components and the components of $h_{\mu\nu}$ in a general background. This procedure is carried out in \eqref{2.6} in a flat background and in \eqref{9.46a} within a general Roberston Walker background. Before discussing these results, it is informative to first analyze the structure of the gauge invariants within Einstein gravity in a source-free flat background. With the background $T_{\mu\nu}=0$ vanishing, the perturbed Einstein tensor $\delta G_{\mu\nu}$ itself is a completely gauge invariant tensor. As a function only of the metric, inspection of the components of the Einstein tensor will thus reveal the appropriate flat space gauge invariant combinations. The Einstein fluctuation takes the form,
%
\begin{eqnarray}
\delta G_{00}&=&- 2 \delta^{ab} \tilde{\nabla}_{b}\tilde{\nabla}_{a}\psi,
\nonumber\\
\delta G_{0i}&=&- 2 \tilde{\nabla}_{i}\dot{\psi}+ \tfrac{1}{2} \delta^{ab} \tilde{\nabla}_{b}\tilde{\nabla}_{a}(B_{i} -  \dot{E}_{i}),
\nonumber\\
\delta G_{ij}&=&- 2 \delta_{ij} \ddot{\psi} -  \delta^{ab} \delta_{ij} \tilde{\nabla}_{b}\tilde{\nabla}_{a}(\phi+\dot{B}  -\ddot{E})+ \delta^{ab} \delta_{ij} \tilde{\nabla}_{b}\tilde{\nabla}_{a}\psi 
	\nonumber\\
&&
+ \tilde{\nabla}_{j}\tilde{\nabla}_{i}(\phi+\dot{B} -  \ddot{E})
-  \tilde{\nabla}_{j}\tilde{\nabla}_{i}\psi
+ \tfrac{1}{2} \tilde{\nabla}_{i}(\dot{B}_{j} - \ddot{E}_{j}) + \tfrac{1}{2} \tilde{\nabla}_{j}(\dot{B}_{i}  
- \ddot{E}_{i})
\nonumber\\
&&- \ddot{E}_{ij} + \delta^{ab} \tilde{\nabla}_{b}\tilde{\nabla}_{a}E_{ij},
\nonumber\\
g^{\mu\nu}\delta G_{\mu\nu}&=&-\delta G_{00}+\delta^{ij}\delta G_{ij}=4 \delta^{ab} \tilde{\nabla}_{b}\tilde{\nabla}_{a}\psi -6\ddot{\psi}-2 \delta^{ab} \tilde{\nabla}_{b}\tilde{\nabla}_{a}(\phi+\dot{B}  -\ddot{E}),
\nonumber\\
\label{2.3}
\end{eqnarray}
%
where the dot denotes the time derivative $\partial/\partial x^0$. As mentioned, while the generic metric fluctuation $h_{\mu\nu}$ has ten components, because of the freedom to make four gauge transformations on the coordinates (i.e $h_{\mu\nu}\rightarrow h_{\mu\nu}-\partial_{\mu}\epsilon_{\nu}-\partial_{\nu}\epsilon_{\mu}$), $\delta G_{\mu\nu}$ can only depend on a total of six of them. Looking at the individual components of $\delta G_{\mu\nu}$, we see that these are proportional to the combinations $\psi$, $\phi+\dot{B}  -\ddot{E}$, $B_{i} -  \dot{E}_{i}$, and $E_{ij}$. 

However, with these identifications, there still remains a degree of ambiguity as to whether the combinations listed form the gauge invariants, or whether it is in fact derivative combinations that are truly gauge invariant. Here one must proceed carefully, as the gauge invariance of $\delta G_{\mu\nu}$ entails that only when taken with the various derivatives that appear in (\ref{2.3}) will these combinations be gauge invariant. For example, we may only state definitively that $\delta G_{00}$ is gauge invariant (hence $\delta^{ab} \tilde{\nabla}_{b}\tilde{\nabla}_{a}\psi$). The gauge invariance of $\psi$ itself cannot be assumed through the analysis on $\delta G_{\mu\nu}$ alone.


To further investigate gauge invariance issues, we express each of the various SVT3 components in terms  of combinations of the original components of $h_{\mu\nu}$. Such a procedure has been derived in \cite{amarasinghe_2019} and is to be contrasted with the integral formulations in \eqref{AP65}. Specifically, in \eqref{AP65} we mentioned uniqueness up to integration by parts, whereas here such issues are avoided as we merely apply sequences of derivatives to $h_{\mu\nu}$ to form the requisite gauge invariant structure, and afterward analyze which SVT combinations are formed as a result. The gauge invariants take the following form: using the definition
%
\begin{eqnarray}
2\phi&=&-h_{00},\quad B_i+\tilde{\nabla}_iB=h_{0i},
\nonumber\\
h_{ij}&=&-2\psi\delta_{ij} +2\tilde{\nabla}_i\tilde{\nabla}_j E + \tilde{\nabla}_i E_j + \tilde{\nabla}_j E_i + 2E_{ij},
\label{2.4}
\end{eqnarray}
%
we apply derivatives to obtain the relations
%
\begin{eqnarray}
\delta^{ij}h_{ij}&=&-6\psi+2\tilde{\nabla}_i\tilde{\nabla}^iE,\quad
\tilde{\nabla}^jh_{ij}=-2\tilde{\nabla}_i\psi+2\tilde{\nabla}_i\tilde{\nabla}_k\tilde{\nabla}^kE+\tilde{\nabla}_k\tilde{\nabla}^kE_{i},
\nonumber\\
\tilde{\nabla}^i \tilde{\nabla}^jh_{ij}&=&-2\tilde{\nabla}_k\tilde{\nabla}^k\psi+2\tilde{\nabla}_k\tilde{\nabla}^k\tilde{\nabla}_{\ell}\tilde{\nabla}^{\ell}E,
\label{2.5}
\end{eqnarray}
%
which then allow us to form the gauge invariants, taking the form
%
\begin{eqnarray}
\tilde{\nabla}_k\tilde{\nabla}^k\psi&=&\frac{1}{4} \left[\tilde{\nabla}^i \tilde{\nabla}^jh_{ij}-\tilde{\nabla}_k\tilde{\nabla}^k(\delta^{ij}h_{ij})\right],
\nonumber\\
\tilde{\nabla}_k\tilde{\nabla}^k\tilde{\nabla}_{\ell}\tilde{\nabla}^{\ell}E&=&\frac{3}{4} \tilde{\nabla}^i \tilde{\nabla}^jh_{ij}-\frac{1}{4}\tilde{\nabla}_k\tilde{\nabla}^k(\delta^{ij}h_{ij}),
\nonumber\\
\tilde{\nabla}_k\tilde{\nabla}^kB&=&\tilde{\nabla}^kh_{0k},
\nonumber\\
\tilde{\nabla}_k\tilde{\nabla}^kB_i&=&\tilde{\nabla}_k\tilde{\nabla}^kh_{0i}-\tilde{\nabla}_i\tilde{\nabla}^kh_{0k},
\nonumber\\
\tilde{\nabla}_k\tilde{\nabla}^k\tilde{\nabla}_{\ell}\tilde{\nabla}^{\ell}E_i&=&\tilde{\nabla}_k\tilde{\nabla}^k\nabla^jh_{ij}-\nabla_i\tilde{\nabla}^k\tilde{\nabla}^{\ell}h_{k\ell},
\nonumber\\
\tilde{\nabla}_k\tilde{\nabla}^kE_{ij}&=&\frac{1}{2}\big[\tilde{\nabla}_k\tilde{\nabla}^kh_{ij}-\tilde{\nabla}_i\tilde{\nabla}^kh_{kj}-\tilde{\nabla}_j\tilde{\nabla}^kh_{ki}
\nonumber\\
&&+\tilde{\nabla}_i\tilde{\nabla}_j(\delta^{k\ell}h_{k\ell})\big]+\delta_{ij}\tilde{\nabla}_k\tilde{\nabla}^k\psi
\nonumber\\
&&+\tilde{\nabla}_i\tilde{\nabla}_j\psi,
\nonumber\\
\tilde{\nabla}_{\ell}\tilde{\nabla}^{\ell}\tilde{\nabla}_k\tilde{\nabla}^kE_{ij}&=&
\frac{1}{2} \tilde{\nabla}_{\ell}\tilde{\nabla}^{\ell}\big[\tilde{\nabla}_k\tilde{\nabla}^kh_{ij}-\tilde{\nabla}_i\tilde{\nabla}^kh_{kj}-\tilde{\nabla}_j\tilde{\nabla}^kh_{ki}
\nonumber\\
&&+\tilde{\nabla}_i\tilde{\nabla}_j(\delta^{k\ell}h_{k\ell})\big]+\frac{1}{4}\left[\delta_{ij}\tilde{\nabla}_{\ell}\tilde{\nabla}^{\ell}+\tilde{\nabla}_i\tilde{\nabla}_j \right]\times
\nonumber\\
&&\left[\tilde{\nabla}^m \tilde{\nabla}^nh_{mn}-\tilde{\nabla}_k\tilde{\nabla}^k(\delta^{mn}h_{mn}) \right],
\nonumber\\
\tilde{\nabla}_{\ell}\tilde{\nabla}^{\ell} \tilde{\nabla}_k\tilde{\nabla}^k(B_i-\dot{E}_i)&=&
\tilde{\nabla}_{\ell}\tilde{\nabla}^{\ell}\tilde{\nabla}_k\tilde{\nabla}^kh_{0i}
-\tilde{\nabla}_{\ell}\tilde{\nabla}^{\ell}\tilde{\nabla}_i\tilde{\nabla}^kh_{0k}
-\partial_0\tilde{\nabla}_{\ell}\tilde{\nabla}^{\ell}\tilde{\nabla}^jh_{ij}
\nonumber\\
&&+\partial_0\tilde{\nabla}_{i}\tilde{\nabla}^{k}\tilde{\nabla}^{\ell}h_{k\ell},
\nonumber\\
\tilde{\nabla}_k\tilde{\nabla}^k\tilde{\nabla}_{\ell}\tilde{\nabla}^{\ell}(\phi+\dot{B}-\ddot{E})&=&
-\tfrac{1}{2}\tilde{\nabla}_k\tilde{\nabla}^k\tilde{\nabla}_{\ell}\tilde{\nabla}^{\ell}h_{00}
+\tilde{\nabla}_{\ell}\tilde{\nabla}^{\ell}\partial_0\tilde{\nabla}^kh_{0k}
-\tfrac{3}{4}\partial_0^2\tilde{\nabla}^i\tilde{\nabla}^jh_{ij}
\nonumber\\
&&+\tfrac{1}{4}\partial_0^2\tilde{\nabla}_{k}\tilde{\nabla}^{k}(\delta^{ij}h_{ij}).
\label{2.6}
\end{eqnarray}
%
Given (\ref{2.6}) one can readily check that under a gauge transformation $h_{\mu\nu}\rightarrow h_{\mu\nu}-\partial_{\mu}\epsilon_{\nu}-\partial_{\nu}\epsilon_{\mu}$ the combinations  $\tilde{\nabla}_k\tilde{\nabla}^k\psi $, $\tilde{\nabla}_{\ell}\tilde{\nabla}^{\ell}\tilde{\nabla}_k\tilde{\nabla}^kE_{ij}$, $\tilde{\nabla}_{\ell}\tilde{\nabla}^{\ell}\tilde{\nabla}_k\tilde{\nabla}^k(B_i-\dot{E}_i)$ and $ \tilde{\nabla}_k\tilde{\nabla}^k\tilde{\nabla}_{\ell}\tilde{\nabla}^{\ell}(\phi+\dot{B}-\ddot{E})$ are gauge invariant. We see here that it was in fact necessary to apply higher order derivatives than found in $\delta G_{\mu\nu}$ in order to express each of the SVT3 components entirely in terms of combinations of components of the $h_{\mu\nu}$. Hence, we repeat that it is not the quantities $\psi$, $E_{ij}$, $B_i-\dot{E}_i$ and $\phi+\dot{B}-\ddot{E}$ themselves that are necessarily gauge invariant; rather, it is their derivatives that are  gauge invariant. In comparing \eqref{2.6} to (\ref{2.3}) we see that it is the quantity $\tilde{\nabla}_k\tilde{\nabla}^k\psi$ that appears in $\delta G_{00}$ and that it is the combination $ \tilde{\nabla}_k\tilde{\nabla}^kE_{ij}-\delta_{ij}\tilde{\nabla}_k\tilde{\nabla}^k\psi-\tilde{\nabla}_i\tilde{\nabla}_j\psi$ that appears in  $\delta G_{ij}$. Thus these  combinations are automatically gauge invariant.

To touch basis with \eqref{AP65}, we could proceed to integrate the relevant equations in (\ref{2.6}) in order to check gauge invariance for $\psi$, $\phi+\dot{B}-\ddot{E}$, $B_{i}-\dot{E_i}$ and $E_{ij}$ themselves, since we can set

%
\begin{eqnarray}
\psi&=&\frac{1}{4}\int d^3yD^{(3)}(\mathbf{x}-\mathbf{y})\left[\tilde{\nabla}_y^k \tilde{\nabla}_y^{\ell}h_{k\ell}-\tilde{\nabla}^y_m\tilde{\nabla}_y^m(\delta^{k\ell}h_{k\ell})\right],
\nonumber\\
\phi+\dot{B}-\ddot{E}&=&-\frac{1}{2} h_{00}
+\partial_0\left[\int d^3y D^{(3)}(\mathbf x - \mathbf y) \tilde\nabla^k_y h_{0k}\right]
\nonumber\\
&-&\partial_0^2\bigg[\int d^3y D^{(3)}(\mathbf x - \mathbf y) \int d^3z D^{(3)}(\mathbf y - \mathbf z)\times
\nonumber\\
&&\left[ \frac{3}{4} \tilde{\nabla}^i \tilde{\nabla}^jh_{ij}-\frac{1}{4}\tilde{\nabla}_k\tilde{\nabla}^k(\delta^{ij}h_{ij})
\right]\bigg]
\nonumber\\
&=&-\tfrac{1}{2}\tilde{\nabla}_{\ell}\tilde{\nabla}^{\ell} \tilde{\nabla}_k\tilde{\nabla}^k\int d^3y D^{(3)}(\mathbf x - \mathbf y) \int d^3z D^{(3)}(\mathbf y - \mathbf z)h_{00}
\nonumber\\
&+&\partial_0\tilde{\nabla}_{\ell}\tilde{\nabla}^{\ell}\int d^3y D^{(3)}(\mathbf x - \mathbf y) \int d^3z D^{(3)}(\mathbf y - \mathbf z)\nabla^k_z h_{0k}
\nonumber\\
&-&\partial_0^2\bigg[\int d^3y D^{(3)}(\mathbf x - \mathbf y) \int d^3z D^{(3)}(\mathbf y - \mathbf z)\times
\nonumber\\
&&\left[ \frac{3}{4} \tilde{\nabla}^i \tilde{\nabla}^jh_{ij}-\frac{1}{4}\tilde{\nabla}_k\tilde{\nabla}^k(\delta^{ij}h_{ij})
\right]\bigg],
\label{2.7}
\end{eqnarray}
%
and 
%
\begin{eqnarray}
B_i &-&\dot{E}_i= \int d^3y D^{(3)}(\mathbf x - \mathbf y)\left[ \tilde\nabla^k_y \tilde\nabla_k^y h_{0i}
- \tilde\nabla_i^y \tilde\nabla^k_y h_{0k} \right]
\nonumber\\
&-&\partial_0\left[\int d^3y D^{(3)}(\mathbf x - \mathbf y) \int d^3z D^{(3)}(\mathbf y - \mathbf z)
\left[ \tilde\nabla^k_z \tilde\nabla_k^z \tilde\nabla^j_z h_{ij}-\tilde\nabla_i^z \tilde\nabla^k_z \tilde\nabla^{\ell}_z h_{k\ell}\right]\right]
\nonumber\\
&=&\tilde{\nabla}_{\ell}\tilde{\nabla}^{\ell} \int d^3y D^{(3)}(\mathbf x - \mathbf y) \int d^3z D^{(3)}(\mathbf y - \mathbf z)\left[ \tilde\nabla^k_z \tilde\nabla_k^z h_{0i}
- \tilde\nabla_i^z \tilde\nabla^k_z h_{0k} \right]
\nonumber\\
&-&\partial_0\left[\int d^3y D^{(3)}(\mathbf x - \mathbf y) \int d^3z D^{(3)}(\mathbf y - \mathbf z)
\left[ \tilde\nabla^k_z \tilde\nabla_k^z \tilde\nabla^j_z h_{ij}-\tilde\nabla_i^z \tilde\nabla^k_z \tilde\nabla^{\ell}_z h_{k\ell}\right]\right],
\nonumber\\
E_{ij}&=&\frac{1}{2}\int d^3yD^{(3)}(\mathbf{x}-\mathbf{y})\left[\tilde{\nabla}^y_k\tilde{\nabla}_y^kh_{ij}-\tilde{\nabla}^y_i\tilde{\nabla}_y^kh_{kj}-\tilde{\nabla}^y_j\tilde{\nabla}_y^kh_{ki}+\tilde{\nabla}^y_i\tilde{\nabla}^y_j(\delta^{k\ell}h_{k\ell})\right]
\nonumber\\
&+&\frac{1}{4}\int d^3yD^{(3)}(\mathbf{x}-\mathbf{y})\left[\delta_{ij}\tilde{\nabla}^y_{\ell}\tilde{\nabla}_y^{\ell}+\tilde{\nabla}^y_i\tilde{\nabla}^y_j\right]\int d^3zD^{(3)}(\mathbf{y}-\mathbf{z})\times
\nonumber\\
&&\left[\tilde{\nabla}_z^m \tilde{\nabla}_z^{n}h_{mn}-\tilde{\nabla}^z_k\tilde{\nabla}_z^k(\delta^{mn}h_{mn})\right],
\label{2.8}
\end{eqnarray}
%
where we make use of the flat space Green's function $D^{(3)}(\mathbf{x}-\mathbf{y})$ obeying 
%
\begin{eqnarray}
\delta^{ij}\tilde{\nabla}_i\tilde{\nabla}_jD^{(3)}(\mathbf{x}-\mathbf{y})&=&\delta^3(\mathbf{x}-\mathbf{y}),\quad
D^{(3)}(\mathbf{x}-\mathbf{y})=-\frac{1}{4\pi |\mathbf{x}-\mathbf{y}|},
\nonumber\\
\int d^3\mathbf{y}e^{i\mathbf{q}\cdot\mathbf{y}}D^{(3)}(\mathbf{x}-\mathbf{y})&=&-\frac{e^{i\mathbf{q}\cdot\mathbf{x}}}{q^2}.
\label{2.9}
\end{eqnarray}
%
(Here $q^2=\delta^{ij}q_{i}q_{j}$, and in $\tilde{\nabla}_y^i$ the $y$ indicates that the derivative is taken with respect to the $y$ coordinate, and likewise for other coordinate choices.)

As eluded to below \eqref{AP65}, there remains however an issue within (\ref{2.7}) and (\ref{2.8}). Specifically, since $\psi$ and $E_{ij}$ are manifestly gauge invariant as is (they are expressed in terms of the gauge-invariant flat 3-space $\delta R_{ij}$ and $\delta^{ij}\delta R_{ij}$), in order to show the gauge invariance of $\phi+\dot{B}-\ddot{E}$ and $B_i -\dot{E}_i$ we would need to be able to integrate by parts (i.e., for $\phi+\dot{B}-\ddot{E}$  we would need to bring $\tilde{\nabla}_{\ell}\tilde{\nabla}^{\ell} \tilde{\nabla}_k\tilde{\nabla}^k$ and $\tilde{\nabla}_{\ell}\tilde{\nabla}^{\ell}$ inside the double integral, while for $B_i-\dot{E}_i$ we would need to bring $\tilde{\nabla}_{\ell}\tilde{\nabla}^{\ell}$ inside, and similarly to show tranverness for $B_i -\dot{E}_i$ and $E_{ij}$ we need to be able to integrate by parts.) Consequently, we are then forced to one of three scenarios. Either a) we must put constraints on how $h_{\mu\nu}$ is to behave asymptotically, or b) restrict to requiring in the $E_{ij}$ sector that only $\tilde{\nabla}_{\ell}\tilde{\nabla}^{\ell}\tilde{\nabla}_k\tilde{\nabla}^kE_{ij}$ be gauge invariant and that only $\tilde{\nabla}_{\ell}\tilde{\nabla}^{\ell}\tilde{\nabla}_k\tilde{\nabla}^kE_{ij}$ be transverse or c) in the $E_{ij}$ plus $\psi$ sector restrict to requiring that only $\tilde{\nabla}_k\tilde{\nabla}^kE_{ij}-\delta_{ij}\tilde{\nabla}_k\tilde{\nabla}^k\psi-\tilde{\nabla}_i\tilde{\nabla}_j\psi$ be gauge invariant and that only $\tilde{\nabla}_k\tilde{\nabla}^kE_{ij}$ be transverse. 


%%%%%%%%%%%%%%%%%%%%%%%%%%%%%%%%%%%%%
	\footnote{
	In a similar manner, we may also integrate the remaining SVT3 components, obtaining
	%
	\begin{eqnarray}
	2\phi&=&-h_{00},\quad
	B=\int d^3yD^{(3)}(\mathbf{x}-\mathbf{y})\tilde{\nabla}_y^ih_{0i},
	\nonumber\\
	B_i&=&h_{0i}-\tilde{\nabla}_i\int d^3yD^{(3)}(\mathbf{x}-\mathbf{y})\tilde{\nabla}_y^ih_{0i},
	\nonumber\\
	E&=&\frac{1}{4}\int d^3yD^{(3)}(\mathbf{x}-\mathbf{y})\int d^3zD^{(3)}(\mathbf{y}-\mathbf{z})\left[3\tilde{\nabla}_z^k\tilde{\nabla}_z^{\ell}h_{k\ell}-\tilde{\nabla}^z_k\tilde{\nabla}_z^k(\delta^{k\ell}h_{k\ell})\right],
	\nonumber\\
	E_i&=&\int d^3yD^{(3)}(\mathbf{x}-\mathbf{y})\int d^3zD^{(3)}(\mathbf{y}-\mathbf{z})\left[\tilde{\nabla}^z_k\tilde{\nabla}_z^k\nabla_z^jh_{ij}-\nabla^z_i\tilde{\nabla}_z^k\tilde{\nabla}_z^{\ell}h_{k\ell}\right]
	\label{2.12}
	\end{eqnarray}
	%
	As constructed, we see that $\tilde{\nabla}^iB_i=0$. However to show $\nabla^iE_i=0$, we need to be able to integrate by parts. Using (\ref{2.4}) and (\ref{2.6}) directly, we can then show that both $\tilde{\nabla}_k\tilde{\nabla}^k\tilde{\nabla}_{\ell}\tilde{\nabla}^{\ell}(\phi+\dot{B}-\ddot{E})$ and $\tilde{\nabla}_k\tilde{\nabla}^k\tilde{\nabla}_{\ell}\tilde{\nabla}^{\ell}(B_i-\dot{E}_i)$ are gauge invariant, with the gauge invariance of $\phi+\dot{B}-\ddot{E}$ and $B_i-\dot{E}_i$ themselves then following when defining $B$, $B_i$, $E$ and $E_i$ according to (\ref{2.12}). Hence, granted the freedom to integrate by parts, we can show that for fluctuations around flat spacetime all of the six $\psi$, $E_{ij}$,  $\phi+\dot{B}-\ddot{E}$ and $B_i-\dot{E}_i$ quantities that appear in $\delta G_{\mu\nu}$ as given in (\ref{2.3}) are gauge invariant.
	}
%%%%%%%%%%%%%%%%%%%%%%%%%%%%%%%%%%%%%

To see how method a), constraining the asymptotic behavior of $h_{\mu\nu}$, may resolve the issues of integration by parts, we shall take $h_{\mu\nu}$  to be localized in space and oscillating in time. Specifically, for each mode we will set $h_{ij}=\epsilon_{ij}(q)e^{i\mathbf{q}\cdot\mathbf{x}-i\omega(q) t}$ with $\omega(q)$ as yet undefined (and thus not necessarily equal to $q$), and where $\epsilon_{ij}(q)$ serves as the polarization tensor. As a localized packet, we constrain the form of the polarization tensor by excluding any functional dependence of the form $\delta(q)$ or $\delta(q)/q$. Thus, referring to \eqref{2.7} and \eqref{2.8}, for spatially localized fluctuations comprising a single mode, the quantities $\psi$ and $E_{ij}$ given in (\ref{2.7}) and (\ref{2.8}) evaluate to
%
\begin{eqnarray}
\psi&=&e^{i\mathbf{q}\cdot\mathbf{x}-i\omega(q) t}\frac{[q^kq^{\ell}\epsilon_{k\ell}(q)-q^2\delta^{k\ell}\epsilon_{k\ell}(q)]}{4q^2},
\nonumber\\
E_{ij}&=&e^{i\mathbf{q}\cdot\mathbf{x}-i\omega(q) t}\bigg{[}\frac{[q^2\epsilon_{ij}(q)-q_iq^k\epsilon_{kj}(q)-q_jq^k\epsilon_{ki}(q)+q_iq_j\delta^{k\ell}\epsilon_{k\ell}(q)]}{2q^2}
\nonumber\\
&+&\frac{(\delta_{ij}q^2+q_iq_j)[q^kq^{\ell}\epsilon_{k\ell}(q)-q^2\delta^{k\ell}\epsilon_{k\ell}(q)]}{4q^4}\bigg{]}.
\label{2.10}
\end{eqnarray}
%
With application of $\tilde\nabla^j$, one may confirm the transverse relation $\tilde{\nabla}^jE_{ij}=0$. To construct a wave packet, we sum over all modes viz. $h_{ij}=\sum_qa_q\epsilon_{ij}(q)e^{i\mathbf{q}\cdot\mathbf{x}-i\omega(q) t}$, to then obtain
%
\begin{eqnarray}
\psi&=&\sum_qa_qe^{i\mathbf{q}\cdot\mathbf{x}-i\omega(q) t}\frac{[q^kq^{\ell}\epsilon_{k\ell}(q)-q^2\delta^{k\ell}\epsilon_{k\ell}(q)]}{4q^2},
\nonumber\\
E_{ij}&=&\sum_qa_qe^{i\mathbf{q}\cdot\mathbf{x}-i\omega(q) t}\bigg{[}\frac{[q^2\epsilon_{ij}(q)-q_iq^k\epsilon_{kj}(q)-q_jq^k\epsilon_{ki}(q)+q_iq_j\delta^{k\ell}\epsilon_{k\ell}(q)]}{2q^2}
\nonumber\\
&+&\frac{(\delta_{ij}q^2+q_iq_j)[q^kq^{\ell}\epsilon_{k\ell}(q)-q^2\delta^{k\ell}\epsilon_{k\ell}(q)]}{4q^4}\bigg{]},
\label{2.11}
\end{eqnarray}
%
where again $\tilde{\nabla}^jE_{ij}=0$. Since the set of all $e^{i\mathbf{q}\cdot\mathbf{x}-i\omega (q)t}$ plane waves is complete for fluctuations around flat, any mode can be expanded as a general sum $h_{ij}=\sum_qa_q\epsilon_{ij}(q)e^{i\mathbf{q}\cdot\mathbf{x}-i\omega(q) t}$, with it following that (\ref{2.11}) then holds for the complete plane wave basis. Hence, by constructing the $\psi$ and $E_{ij}$ in a localized plane-wave basis, we confirm the transverse relation $\tilde\nabla^j E_{ij} = 0$ without encountering issues related to integration by parts.

While we have demonstrated the role asymptotic behavior plays within tradeoff of transverse behavior vs. gauge invariance, it is also of importance to consider under conditions the SVT3 decomposition of $h_{\mu\nu}$ may be afforded in the first place. We revisit the SVT3 derivation constructed in \cite{amarasinghe_2019} with an eye towards boundary conditions and asymptotic behavior. 

Let us suppose that we are given a general vector $A_i$ and we desire to extract out its transverse and longitudinal components, to thereby construct a relation $A_i=V_i+\partial_iL$ where $\partial_iV^i=0$. Applying $\partial^i$, it follows that
%
\begin{eqnarray}
\partial_i\partial^iL=\partial_iA^i.
\label{2.13}
\end{eqnarray}
%
Recalling the Green's identity
%
\begin{eqnarray}
A \partial_i\partial^iB-B \partial_i\partial^iA=\partial_i(A\partial^iB-B\partial^iA),
\label{1.9}
\end{eqnarray}
and introducing the Green's function
%
\begin{eqnarray}
\partial_i\partial^i D(\mathbf x-\mathbf y) = \delta^3(\mathbf x- \mathbf y),
\end{eqnarray}
%
the general  solution to (\ref{2.13}) is of the form 
%
\begin{eqnarray}
L({\bf x})&=&\int d^3yD^{(3)}(\mathbf{x}-\mathbf{y})\partial^y_jA^j({\bf y})\\
\nonumber\\
&&+\int dS_y^i\left[L({\bf y})\partial^y_iD^{(3)}(\mathbf{x}-\mathbf{y})-D^{(3)}(\mathbf{x}-\mathbf{y})\partial^y_iL({\bf y})\right].
\label{2.14}
\end{eqnarray}
%
Now utilizing $A_i = V_i + \partial_i L$, it follows that 
%
\begin{eqnarray}
A_i({\bf x})&=&V_i({\bf x})+\partial^x_iL=V_i({\bf x})+\partial^x_i\int d^3yD^{(3)}(\mathbf{x}-\mathbf{y})\partial^y_jA^j({\bf y})
\nonumber\\
&+&\partial^x_i\int dS_y^i\left[L({\bf y})\partial^y_iD^{(3)}(\mathbf{x}-\mathbf{y})-D^{(3)}(\mathbf{x}-\mathbf{y})\partial^y_iL({\bf y})\right].
\label{2.15}
\end{eqnarray}
%
Upon applying $\partial_x^i$ to (\ref{2.15}), we obtain
%
\begin{eqnarray}
&&\partial_x^i\partial^x_i\int dS_y^i\left[L({\bf y})\partial^y_iD^{(3)}(\mathbf{x}-\mathbf{y})-D^{(3)}(\mathbf{x}-\mathbf{y})\partial^y_iL({\bf y})\right]=0,
\label{2.16}
\end{eqnarray}
%
to thus establish that $\partial_i\int dS^i(L\partial_iD^{(3)}-D^{(3)}\partial_iL)$ is transverse. At this point, we appear to have constructed two transverse components - both $V_i$ and 
%
\begin{eqnarray}
\partial_i\int dS^i(L\partial_iD^{(3)}-D^{(3)}\partial_iL)
\end{eqnarray}
%
itself. To allow $V_i(x)$ to serve as the unique transverse component of $A_i(x)$, we must then require that $\partial_i\int dS^i(L\partial_iD^{(3)}-D^{(3)}\partial_iL)$ vanish. Consequently, we see that we must inadvertently impose a constraint on $A_i$, namely that $A_i$ be spatially asymptotically well-behaved. Thus, the unique decomposition of $A_i$ into longitudinal and traverse components necessarily entails an assumption regarding the asymptotic behavior of $A_i$. 

%%%%%%%%%%%%%%%%%%%%%%%%%%%%%%%%%%%%%
\section{SVTD}
\label{s:svtd}
%%%%%%%%%%%%%%%%%%%%%%%%%%%%%%%%%%%%%

While the SVT3 formalism presented thus far has demonstrated that the gauge invariant SVT3 quantities are covariant, one may note that these quantities have been defined with respect to the three-dimensional subspace of the full four-dimensional spacetime. With the general $h_{\mu\nu}$ behaving under coordinate transformation as
%
\begin{eqnarray}
h_{\mu\nu} \to h_{\mu\nu}' = \frac{\partial x'^\alpha}{\partial x^\mu}\frac{\partial x'^\beta}{\partial x^\nu} h_{\alpha\beta},
\end{eqnarray}
%
we see that a quantity such as $h_{00} = -2\phi$ may transform into a scalar involving vector components. To see this, let us form the four vector $h_{0\mu}$, which in terms of SVT3 components takes the form
\begin{eqnarray}
h_{0\mu} = \begin{pmatrix}-2\phi\\B_1 + \tilde\nabla_1 B\\B_2 + \tilde\nabla_2 B\\B_3 + \tilde\nabla_3. B\end{pmatrix}
\end{eqnarray}
Now, with the full four-dimensional coordinate transformation mixing each of the four components of $h_{0\mu}$, we see that the transformation of $\phi$ may induce a contribution due to a vector $B_i$. 

Thus to decompose the $h_{\mu\nu}$ into a set of scalars, vectors, and tensors that remain closed under the Poincare group, we must develop a formalism that matches the underlying space-time dimensionality; namely an SVT4 formalism. We proceed to do so here in a flat spacetime, following the series of steps given within \cite{phelps_2019}. It is no additional overhead to generalize this to $D$ dimensions here, with even further generalization to arbitrary curved spacetimes found in detail within Appendix \ref{aa:svt_projection}.

We defined Greek indices to range over the full D-dimensional space and begin with the construction of a symmetric rank two tensor $F_{\mu\nu}$, taken to be transverse and traceless in the full D-dimensional space.
%%%%%%%%%%%%%%%%%%%%%%%%
\footnote{The previously introduced $E_{ij}$ was only transverse and traceless in a 3-dimensional subspace.}
%%%%%%%%%%%%%%%%%%%%%%%%
Accounting for the dimensionality and the transverse traceless constraints, $F_{\mu\nu}$ tensor will have $D(D+1)/2-D-1=(D+1)(D-2)/2$ components. To facilitate the decomposition, we introduce a D-dimensional  vector $W_{\mu}$. In terms of this $W_{\mu}$ and $h$, and motivated by \cite{mannheim_2005}, we define the general $h_{\mu\nu}$ fluctuation around a flat D-dimensional space to be of the form
%
\begin{eqnarray}
h_{\mu\nu}=2F_{\mu\nu}+\nabla_{\nu}W_{\mu}+\nabla_{\mu}W_{\nu}+\frac{2-D}{D-1}\nabla_{\mu}\nabla_{\nu}\int d^DyD^{(D)}(x-y)\nabla^{\alpha}W_{\alpha}
\nonumber\\
-\frac{g_{\mu\nu}}{D-1}(\nabla^{\alpha}W_{\alpha}-h)-\frac{\nabla_{\mu}\nabla_{\nu}}{D-1}\int d^DyD^{(D)}(x-y)h,
\label{3.1}
\end{eqnarray}
%
where the flat spacetime $D^{(D)}(x-y)$ obeys 
%
\begin{eqnarray}
g^{\mu\nu}\nabla_{\mu}\nabla_{\nu}D^{(D)}(x-y)=\delta^{(D)}(x-y).
\label{3.2}
\end{eqnarray}
%
Parallel to the SVT3 asymptotic behavior discussed in Sec. \ref{ss:gauge_struct_svt3}, it is implied in the form of $h_{\mu\nu}$ that the D-dimensional integrals exist, specifically with $\nabla^{\alpha}W_{\alpha}$ being sufficiently well-behaved at infinity.
To make the $F_{\mu\nu}$ that is defined by (\ref{3.1}) be transverse and traceless requires D+1 conditions, D to be supplied by $W_{\mu}$ and one  to be supplied by $h$. Given the defined \eqref{3.1}, one may take the trace and confirm that $F_{\mu\nu}$ is traceless as written. To assess whether $F_{\mu\nu}$ is transverse, we apply $\nabla^{\nu}$  to (\ref{3.1}) yielding
%
\begin{eqnarray}
\nabla^{\nu}h_{\nu\mu}=\nabla_{\alpha}\nabla^{\alpha}W_{\mu}.
\label{3.3}
\end{eqnarray}
% 
Thus \eqref{3.3} serves to define the as yet determined $D$ components of $W_\mu$ in terms of the $h_{\mu\nu}$. Questions on the asymptotic behavior of $\nabla^{\alpha}W_{\alpha}$ are thus directly linked the asymptotic behavior of $h_{\mu\nu}$. Hence, for a $W_{\mu}$ that obeys (\ref{3.3}) and is sufficiently bounded, the D-dimensional rank two tensor $F_{\mu\nu}$ is transverse and traceless.

Uponj applying $\nabla_{\alpha}\nabla^{\alpha}$ to (\ref{3.1}) we obtain
%
\begin{eqnarray}
\nabla_{\alpha}\nabla^{\alpha}h_{\mu\nu}&=&2\nabla_{\alpha}\nabla^{\alpha}F_{\mu\nu}+\nabla_{\nu}\nabla^{\alpha}h_{\alpha\mu}+\nabla_{\mu}\nabla^{\alpha}h_{\alpha\nu}+\frac{2-D}{D-1}\nabla_{\mu}\nabla_{\nu}\nabla^{\alpha}W_{\alpha}
\nonumber\\
&-&\frac{g_{\mu\nu}}{D-1}(\nabla^{\alpha}\nabla^{\beta}h_{\alpha\beta}-\nabla_{\alpha}\nabla^{\alpha}h)-\frac{\nabla_{\mu}\nabla_{\nu}}{D-1}h,
\label{3.4}
\end{eqnarray}
%
and on rearranging we obtain
%
\begin{eqnarray}
&&\nabla_{\alpha}\nabla^{\alpha}h_{\mu\nu}-\nabla_{\nu}\nabla^{\alpha}h_{\alpha\mu}-\nabla_{\mu}\nabla^{\alpha}h_{\alpha\nu}+\nabla_{\mu}\nabla_{\nu}h
\nonumber\\
&=&2\nabla_{\alpha}\nabla^{\alpha}F_{\mu\nu}+\frac{2-D}{D-1}\nabla_{\mu}\nabla_{\nu}[\nabla^{\alpha}W_{\alpha}-h]
\nonumber\\
&&-\frac{g_{\mu\nu}}{D-1}(\nabla^{\alpha}\nabla^{\beta}h_{\alpha\beta}-\nabla_{\alpha}\nabla^{\alpha}h).
\label{3.5}
\end{eqnarray}
%
Now in flat spacetime, we note that the perturbed curvature tensors $\delta R$ and $\delta R_{\mu\nu}$ are independelty gauge invariant. Thus, armed with the SVTD formalism, we use these to determine the gauge invariants
%
\begin{eqnarray}
2\delta R_{\mu\nu} = \nabla_{\alpha}\nabla^{\alpha}h_{\mu\nu}-\nabla_{\nu}\nabla^{\alpha}h_{\alpha\mu}-\nabla_{\mu}\nabla^{\alpha}h_{\alpha\nu}+\nabla_{\mu}\nabla_{\nu}h
\label{dricci}
\end{eqnarray}
%
%
\begin{eqnarray}
-\delta R = \nabla^{\alpha}\nabla^{\beta}h_{\alpha\beta}-\nabla_{\alpha}\nabla^{\alpha}h.
\label{driccis}
\end{eqnarray}
Making use of
%
\begin{eqnarray}
\nabla_{\beta}\nabla^{\beta}[\nabla^{\alpha}W_{\alpha}-h]=\nabla^{\alpha}\nabla^{\beta}h_{\alpha\beta}-\nabla_{\alpha}\nabla^{\alpha}h,
\label{3.6}
\end{eqnarray}
%
we define
%
\begin{eqnarray}
\nabla^{\alpha}W_{\alpha}-h=\int d^DyD^{(D)}(x-y)[\nabla^{\alpha}\nabla^{\beta}h_{\alpha\beta}-\nabla_{\alpha}\nabla^{\alpha}h],
\label{3.7}
\end{eqnarray}
%
and with this solution and reference to \eqref{dricci} and \eqref{driccis} we see that  $\nabla_{\alpha}\nabla^{\alpha}F_{\mu\nu}$ is thus gauge invariant. However, we note again that parallel to the discussion the SVT3 tensor $E_{ij}$, to show that $\nabla_{\alpha}\nabla^{\alpha}F_{\mu\nu}$ is transverse requires that we one can justify an integrate by parts.

For the remaining SVTD components, we make the following definitions
%
\begin{eqnarray}
2\chi&=&\frac{1}{D-1}[\nabla^{\alpha}W_{\alpha}-h],\quad 
\quad 2F=\frac{1}{D-1}\int d^DyD^{(D)}(x-y)[D\nabla^{\alpha}W_{\alpha}-h],
\nonumber\\
F_{\mu}&=&W_{\mu}-\nabla_{\mu}\int d^DyD^{(D)}(x-y)\nabla^{\alpha}W_{\alpha}.
\label{3.8}
\end{eqnarray}
%
As constructed, within (\ref{3.8}) $F_\mu$ is defined such that $\nabla^{\mu}F_{\mu}=0$. As for $\chi$, we note with (\ref{3.7}) we find that  $\chi$ is the integral of a gauge-invariant function so that $\nabla_{\alpha}\nabla^{\alpha}\chi$ is also automatically gauge invariant. Finally, with (\ref{3.8}) we can express (\ref{3.1}) in terms of the SVTD quantities as
%
\begin{eqnarray}
h_{\mu\nu}=-2g_{\mu\nu}\chi+2\nabla_{\mu}\nabla_{\nu}F
+ \nabla_{\mu}F_{\nu}+\nabla_{\nu}F_{\mu}+2F_{\mu\nu},
\label{3.9}
\end{eqnarray}
%
to thus complete the SVTD basis decomposition of $h_{\mu\nu}$. In a general D-dimensional basis $F_{\mu\nu}$ has $(D+1)(D-2)/2$ components, the transverse $F_{\mu}$ has $D-1$ components, the two scalars $\chi$ and $F$ each have one component, and together they comprise the $D(D+1)/2$ components of a general $h_{\mu\nu}$. If we  elect to take $D=3$, (i.e. a decomposition of $h_{ij}$) we can then recognize (\ref{3.9}) as the spatial piece of SVT3 given in (\ref{2.1}), providing a check on our results.



%%%%%%%%%%%%%%%%%%%%%%%%%%%%%%%%%%%%%
\subsection{Gauge Structure ($D=4$)}
\label{S1e}
%%%%%%%%%%%%%%%%%%%%%%%%%%%%%%%%%%%%%
%
Counting the degrees of freedom of the fluctuation $\delta G_{\mu\nu}$ around flat D-dimensional we arrive at a total of $D(D+1)/2-D=D(D-1)/2$ independent gauge-invariant combinations. For the SVTD components, $F_{\mu\nu}$ has $(D+1)(D-2)/2$ components and $\chi$ has one. That is, the combination of $\chi$ and $F_{\mu\nu}$ precisely forms a total of $D(D-1)/2$. Moreoever with the derivatives of both them being gauge invariant, we deduce that $\delta G_{\mu\nu}$ can only depend on $F_{\mu\nu}$ and $\chi$. Applying appropriate derivatives to the quantities given (\ref{3.8}) and (\ref{3.9}), we indeed form the following gauge invariant relations
%
\begin{eqnarray}
2\nabla_{\alpha}\nabla^{\alpha}\chi&=&\frac{1}{D-1}\left[\nabla^{\alpha}\nabla^{\beta}h_{\alpha\beta}-\nabla_{\alpha}\nabla^{\alpha}h\right],
\nonumber\\
2\nabla_{\alpha}\nabla^{\alpha}\nabla_{\beta}\nabla^{\beta}F_{\mu\nu}&=&\nabla_{\beta}\nabla^{\beta}\left[\nabla_{\alpha}\nabla^{\alpha}h_{\mu\nu}-\nabla_{\nu}\nabla^{\alpha}h_{\alpha\mu}-\nabla_{\mu}\nabla^{\alpha}h_{\alpha\nu}+\nabla_{\mu}\nabla_{\nu}h\right]
\nonumber\\
&+&\frac{1}{D-1}\left[(D-2)\nabla_{\mu}\nabla_{\nu}+g_{\mu\nu}\nabla_{\gamma}\nabla^{\gamma}\right][\nabla^{\alpha}\nabla^{\beta}h_{\alpha\beta}-\nabla_{\alpha}\nabla^{\alpha}h],
\nonumber\\
\delta R_{\mu\nu}&=&\frac{1}{2}[2\nabla_{\alpha}\nabla^{\alpha}F_{\mu\nu}+2(2-D)\nabla_{\mu}\nabla_{\nu}\chi-2g_{\mu\nu}\nabla_{\alpha}\nabla^{\alpha}\chi],
\nonumber\\
\delta R&=&2(1-D)\nabla_{\alpha}\nabla^{\alpha}\chi,
\nonumber\\
\delta G_{\mu\nu}&=&\delta R_{\mu\nu}-\frac{1}{2}g_{\mu\nu}g^{\alpha\beta}\delta R_{\alpha\beta}=\nabla_{\alpha}\nabla^{\alpha}F_{\mu\nu}
\nonumber\\
&&+(D-2)(g_{\mu\nu}\nabla_{\alpha}\nabla^{\alpha}-\nabla_{\mu}\nabla_{\nu})\chi,
\nonumber\\
g^{\mu\nu}\delta G_{\mu\nu}&=&(D-2)(D-1)\nabla_{\alpha}\nabla^{\alpha}\chi.
\label{3.10}
\end{eqnarray}
%
As written, we confirm that $\delta G_{\mu\nu}$ indeed depends upon only $F_{\mu\nu}$ and $\chi$, with one readily being able to check that $\delta G_{\mu\nu}$ obeys conservation $\nabla^{\nu}\delta G_{\mu\nu}=0$. Inspection of the components of $\delta G_{\mu\nu}$ reveals that only infer that from $g^{\mu\nu}\delta G_{\mu\nu}$ that $\nabla_{\alpha}\nabla^{\alpha}\chi$ is alone gauge invariant. However, by applying $\nabla_{\alpha}\nabla^{\alpha}$ to the $\delta G_{\mu\nu}$ equation we can then additionally deduce that $\nabla_{\alpha}\nabla^{\alpha}\nabla_{\beta}\nabla^{\beta}F_{\mu\nu}$ is gauge invariant. Importantly, we note that $F_{\mu\nu}$ nor $\nabla_\alpha\nabla^\alpha F_{\mu\nu}$ may be determined to gauge invariant without consideration issues related to its asymptotic behavior.  We will continue a discussion of the role of gauge invariance and boundedness in the following section. In addition, further detail of the SVTD construction, formalism, and gauge invariance is discussed within Appendex \ref{aa:svt_projection}. 

Demonstrated in \eqref{3.10}, we also observe that when written in the SVT4 basis the fluctuation equations take a considerably simpler form than when written according to the SVT3 basis (a result that will continue to carry over when we study the SVT4 fluctuation equations in curved background within Ch. \ref{c:construction_and_solution_of_svt}). Hence, by implementing the covariant formalism we have effectively replaced the six gauge-invariant combinations $\psi$, $E_{ij}$,  $\phi+\dot{B}-\ddot{E}$ and $B_i-\dot{E}_i$ of SVT3 by the six gauge-invariant combinations $F_{\mu\nu}$ and $\chi$ of SVT4. Consequently, the SVTD approach has provided a more compact set of gauge-invariant combinations and equations within $\delta G_{\mu\nu}$ and will prove to be simpler to solve, hence justfying the beneficial utility of matching the fundamental transformation group associated with GR.

For geometries more general than flat spacetime, since the gauge-invariant equations must contain a total of six gauge-invariant degrees of freedom, they must be comprised of the five-component $F_{\mu\nu}$ and some combination of the five other components that appear in (\ref{3.9}).


%%%%%%%%%%%%%%%%%%%%%%%%%%%%%%%%%%%%%
\section{Relating SVT3 to SVT4}
\label{s:relating_svt3_to_svt4}
%%%%%%%%%%%%%%%%%%%%%%%%%%%%%%%%%%%%%

Thus far we have presented both and SVT3 and SVT4 decomposition of the metric fluctuation $h_{\mu\nu}$. In the SVT3 development, we have determined the individual SVT3 components in terms of $h_{\mu\nu}$ in both integral form \eqref{AP66} and in terms of higher order derivative relations \eqref{2.6}. In addition, we have determined the SVT3 gauge invariants, using the gauge invariant $\delta G_{\mu\nu}$ as reference. Likewise, we have repeated the analysis for SVT4. Given that both formalisms present valid decompositions with their structure only differing the dimensionality of the underlying space-time slicing, it remains to show the relationship between SVT3 and SVT4, which we develop here.

For fluctuations around a flat background, we have seen that the SVT4 transverse traceless tensor $F_{\mu\nu}$ is comprised of five independent components. Counting the degrees of freedom, to compose the $F_{\mu\nu}$ in terms of SVT3 quantities, we must include the two-component transverse-traceless three-space rank two tensor, a two-component transverse three-space vector and a one-component three-space scalar. In addition, we have determined that if $F_{\mu\nu}$ is asymptotically well behaved to thus permit integration by parts, $F_{\mu\nu}$ itself is then gauge invariant and thus its associated combinations of SVT3 quantities must be gauge invariant. Such a requirement demands that the SVT3 vector component of $F_{\mu\nu}$ must be proportional to $B_i - \dot E_i$ (with $E_{ij}$ being the only SVT3 tensor, we have already account for such). To lastly pin down the last remaining gauge invariant associated with the SVT3 scalar presents a difficulty, as we would appear to not have any means to signify whether it is $\psi$ or $\phi+\dot B - \ddot E$ that should comprise $F_{\mu\nu}$. One may try to use the relation for $\chi$ with $\delta G_{\mu\nu}$ to fix the remaining degree of freedom, but comparing the SVT3 and SVT4 expansions of $\delta G_{\mu\nu}$ as given in (\ref{2.3}) and (\ref{3.10})  does in fact not enable us to uniquely specify the needed scalar combination. 

Consequently, we thus need to construct the fluctuation equations associated with a pure metric gravitational tensor other than the Einstein $\delta G_{\mu\nu}$, with the requirement that such a tensor also be independently gauge invariant in the flat background. It would also be beneficial if such a gravitational tensor were able to form a separation between $F_{\mu\nu}$ and $\chi$. Earlier we have determined that the perturbed curvature tensors themselves are gauge invariant, and thus we make appeal to any remaining curvature tensors not yet considered. Specifically, we have found such a tensor that meets the requisite properties, namely the Bach tensor associated with conformal gravity. As a pure metric tensor obtained by variation of the conformal gravity action of \eqref{AP1}, we restate its form here \cite{mannheim_2006}
%
\begin{eqnarray}
\delta W_{\mu\nu}&=&\frac{1}{2}(\eta^{\rho}_{\phantom{\rho} \mu} \partial^{\alpha}\partial_{\alpha}-\partial^{\rho}\partial_{\mu})
(\eta^{\sigma}_{\phantom{\sigma} \nu} \partial^{\beta}\partial_{\beta}-
\partial^{\sigma}\partial_{\nu})K_{\rho \sigma}
\nonumber\\
&&- \frac{1}{6}(\eta_{\mu \nu} \partial^{\gamma}\partial_{\gamma}-
\partial_{\mu}\partial_{\nu})(\eta^{\rho \sigma} \partial^{\delta}\partial_{\delta}-
\partial^{\rho}\partial^{\sigma})K_{\rho\sigma},
\label{4.5}
\end{eqnarray}
%
With $W_{\mu\nu}^{(0)}$ vanishing in conformal to flat background, and recalling that $\delta W_{\mu\nu}$ is traceless, the perturbed Bach tensor comprises five independent gauge invariant components. In the SVT4 basis, it must then solely be a function of the five component gauge invariant $F_{\mu\nu}$. Naturally, we proceed to evaluate \eqref{4.5} in the SVT3 basis to determine its form, given as
%
\begin{eqnarray}
\delta W_{00}  &=& -\frac{2}{3} \delta^{mn}\delta^{\ell k}\tilde{\nabla}_m\tilde{\nabla}_n\tilde{\nabla}_{\ell}\tilde{\nabla}_k (\phi + \psi +\dot{B}-\ddot{E}),
\nonumber\\	
\delta W_{0i} &=&  -\frac{2}{3} \delta^{mn}\tilde{\nabla}_i\tilde{\nabla}_m\tilde{\nabla}_n\partial_0(\phi +\psi +\dot{B}-\ddot{E})
+\frac{1}{2}\bigg[\delta^{mn}\delta^{\ell k}\tilde{\nabla}_m\tilde{\nabla}_n\tilde{\nabla}_{\ell}\tilde{\nabla}_k(B_i - \dot{E}_i)
\nonumber\\
&& -  \delta^{\ell k}\tilde{\nabla}_{\ell}\tilde{\nabla}_k \partial_0^2(B_i - \dot{E}_i)\bigg],
\nonumber\\	
\delta W_{ij}  &=& \frac{1}{3}\bigg{[} \delta_{ij}\delta^{\ell k}\tilde{\nabla}_{\ell}\tilde{\nabla}_k  \partial_0^2(\phi+ \psi+\dot{B}-\ddot{E}) + \delta^{\ell k}\tilde{\nabla}_{\ell}\tilde{\nabla}_k \tilde{\nabla}_i\tilde{\nabla}_j (\phi + \psi +\dot{B}-\ddot{E}) 
\nonumber\\
&&- \delta_{ij} \delta^{mn}\delta^{\ell k}\tilde{\nabla}_m\tilde{\nabla}_n\tilde{\nabla}_{\ell}\tilde{\nabla}_k(\phi + \psi +\dot{B}-\ddot{E}) -3\tilde{\nabla}_i\tilde{\nabla}_j \partial_0^2(\phi + \psi +\dot{B}-\ddot{E})\bigg{] }
\nonumber\\
&&+\frac{1}{2}\bigg[ \delta^{\ell k}\tilde{\nabla}_{\ell}\tilde{\nabla}_k \tilde{\nabla}_i   \partial_0(B_j - \dot{E}_j)+ \delta^{\ell k}\tilde{\nabla}_{\ell}\tilde{\nabla}_k \tilde{\nabla}_j \partial_0(B_i - \dot{E}_i) - \tilde{\nabla}_i\partial_0^3(B_j - \dot{E}_j)
\nonumber\\
&&-\tilde{\nabla}_j\partial_0^3(B_i - \dot{E}_i)\bigg] +\left[\delta^{mn}\tilde{\nabla}_m\tilde{\nabla}_n-\partial_0^2\right]^2E_{ij},
\label{4.6}
\end{eqnarray}
%
where $K_{\mu\nu}=h_{\mu\nu}-(1/4)g_{\mu\nu}h$. Likewise,  evaluating (\ref{4.5}) in the SVT4 basis given in (\ref{3.9}) yields 
%
\begin{eqnarray}
\delta W_{\mu\nu}=\nabla_{\alpha}\nabla^{\alpha}\nabla_{\beta}\nabla^{\beta}F_{\mu\nu}.
\label{4.7}
\end{eqnarray}
%
As an aside, we note the extremely compact and simple structure of the SVT4 fluctuation equation in conformal gravity, with its exact solution being readily obtainable (particularly compared to $\delta G_{\mu\nu}$ within \eqref{3.10}).

Inspection of the SVT3 structure reveals that $\phi + \psi +\dot{B}-\ddot{E}$ is to unambiguously serve as the three-dimensional scalar piece of $F_{\mu\nu}$. In addition, from (\ref{3.10}) we can identify $\chi$ according to $3\nabla_{\alpha}\nabla^{\alpha}\chi=-\delta^{ij}\tilde{\nabla}_i\tilde{\nabla}_j(\phi  +\psi +\dot{B}-\ddot{E})+3\delta^{ij}\tilde{\nabla}_{i}\tilde{\nabla}_{j}\psi-3\ddot{\psi}$. Thus we have determined the underlying relationship between SVT3 and SVT4, with fluctuations around flat spacetime quantity $F_{\mu\nu}$ necessarily containing $E_{ij}$, $B_i-\dot{E}_i$ and $\phi + \psi +\dot{B}-\ddot{E}$. 

%%%%%%%%%%%%%%%%%%%%%%%%%%%%%%%%%%%%%
\section{Decomposition Theorem and Boundary Conditions}
\label{s:decomposition_theorem}
%%%%%%%%%%%%%%%%%%%%%%%%%%%%%%%%%%%%%

In attempts to solve the cosmological fluctuation equations that have been presented in the literature, appeal is commonly made to the decomposition theorem. The theorem entails the assertation that within the fluctuation equations \emph{themselves} the scalar, vector, and tensor sectors decouple and evolve independent. As mentioned in the introduction, the investigation of the decomposition theorem forms a core component of this work. We proceed by inspecting the interplay of asymptotic behavior and boundary conditions in establishing the validity of theorem below, as applied to both SVT3 and SVT4, taking a flat background and reduced fluctuation equation forms as illustrative examples. Within Ch. \ref{c:construction_and_solution_of_svt}, we touch basis again with the decomposition theorem as applied to specific cosmological geometries in order to determine the underlying assumptions that need to hold for such a theory to be valid.

%%%%%%%%%%%%%%%%%%%%%%%%%%%%%%%%%%%%%
\subsection{SVT3}
%%%%%%%%%%%%%%%%%%%%%%%%%%%%%%%%%%%%%
%
We begin with a schematic example of an SVT3 representation of two vector fields which have been decomposed into their traverse vector and longitudinal scalar components
\begin{eqnarray}
B_i+\partial_iB=C_i+\partial_iC,
\label{1.3}
\end{eqnarray}
%
where the $B$ and $B_i$ are given by (\ref{2.1}), obeying $\partial^i B_i = 0$ and where the $C$ and $C_i$ are to represent functions given by the evolution equations with $C_i$ also obeying  $\partial_iC^i=0$. Now, according to the decomposition theorem, the vectors $B_i$ and $C_i$ are taken to decouple and evolve independent of the scalars, with an analogous statement holding for the scalars. Thus, we are to take
%
\begin{eqnarray}
B_i= C_i,\quad \partial_iB=\partial_iC.
\label{1.4}
\end{eqnarray}
%
It is clear, however, that in such an action, (\ref{1.4}) does not follow from (\ref{1.3}). For if we apply $\partial^i$ and $\epsilon^{ijk}\partial_j$  to (\ref{1.3}) we obtain 
%
\begin{eqnarray}
\partial^i\partial_i(B-C)=0,\quad \epsilon^{ijk}\partial_j(B_k-C_k)=0,
\label{1.5}
\end{eqnarray}
%
and from this we can only conclude that $B$ and $C$ are defined up to an arbitrary scalar $D$ obeying $\partial^i\partial_iD=0$. For the vector sector analogously, $B_k$ and $C_k$ can only differ by any function $D_k$ that obeys $\epsilon^{ijk}\partial_jD_k=0$, i.e. an irrotational field $D_k$ expressed as the gradient of a scalar. In composing (\ref{1.5}) we have appropriately separated the scalar and vector components within(\ref{1.3}), obtaining a decomposition for the components that does not follow by proceeding from (\ref{1.5}) to (\ref{1.4}). Specifically, without providing some additional information or constraints, one cannot directly proceed from (\ref{1.5}) to (\ref{1.4}). However, given the imposition of such constraints, such a decomposition may be possible, with the constraints in fact needing to be in form of spatially asymptotic boundary conditions. Alternatively, if one were to elect to form a decomposition without the imposition of boundary condition or constraints, we see from \eqref{1.5} that we necessarily need to go to higher derivatives in order to establish the decomposition. Thus we will continue to investigate both branches of applying said decomposition theorem.

Continuing with the example of (\ref{1.5}), we first proceed from (\ref{1.5}) to (\ref{1.4}) by imposing asymptotic boundary conditions. As we have done within \eqref{2.10}, we take a complete basis of three-dimensional plane waves (serving as a complete basis for $\partial_i\partial^i$) and we express the general solution for $B-C=D$ in the form
%
\begin{eqnarray}
D=\sum _{\bf k}a_{\bf k}e^{i\textbf{k}\cdot \textbf{x}},
\label{1.6}
\end{eqnarray}
%
where the $a_{\bf k}$ are constrained to obey 
%
\begin{eqnarray}
{\bf k}^2a_{\bf k}=0.
\label{1.7}
\end{eqnarray}
%
Inspecting (\ref{1.7}) itself carefully, we observe that (\ref{1.7}) does not lead to $a_{\bf k}=0$ directly. In fact since $k^2\delta(k)=0$, $k^2\delta(k)/k=0$ we can take other forms for the $a_{\mathbf k}$, with the general form of
%
\begin{eqnarray}
a_{\bf k}&=&\alpha_k\delta(k_x)\delta(k_y)\delta(k_z)
\nonumber\\
&&+\beta_k\left[\frac{\delta (k_x)\delta (k_y)\delta (k_z)}{k_x}+\frac{\delta (k_x)\delta (k_y)\delta (k_z)}{k_y}+\frac{\delta (k_x)\delta (k_y)\delta (k_z)}{k_z}\right],
\label{1.8}
\end{eqnarray}
%
with constants $\alpha_k$ and $\beta_k$. Substituting $\alpha_k$ as given into (\ref{1.6}) would serve to eliminate the ${\bf x}$ dependence from $D$ and consequently provide a constant $D$ that does not vanish at spatial infinity. Alternatively, if we substitute the $\beta_k$ term  into (\ref{1.6}) we would construct a scalar $D$ that grows linearly in ${\bf x}$, to therefore also not vanish at spatial infinity. Hence by imposing asymptotic conditions, we can exclude solutions containing a non-zero $D$ obeying $\partial_i\partial^iD=0$, to thus yield the requisite decomposition theorem. Consequently, we have connected validity of the decomposition theorem as being hinged upon the very existence of the SVT3 basis in the first place as both require asymptotic boundary conditions.

We can further our understanding the role of boundary conditions by inspecting the behavior of $\partial_i\partial^iD=0$ in coordinate space. Recalling the Green's identity
%
\begin{eqnarray}
A \partial_i\partial^iB-B \partial_i\partial^iA=\partial_i(A\partial^iB-B\partial^iA),
\label{1.9}
\end{eqnarray}
%
we take a general scalar $A$ to obey $\partial_i\partial^iA=0$ and take $B$ to be the Green's function $D^{(3)}(\mathbf{x}-\mathbf{y})$ which obeys
%
\begin{eqnarray}
\partial_i\partial^iD^{(3)}(\mathbf{x}-\mathbf{y})=\delta^3(\mathbf{x}-\mathbf{y}).
\label{1.10}
\end{eqnarray}
%
As a result, we can now express $A$ as an asymptotic surface term of the form 
%
\begin{eqnarray}
A({\bf x}) =\int dS_y^i\left[A({\bf y})\partial^y_iD^{(3)}(\mathbf{x}-\mathbf{y})-D^{(3)}(\mathbf{x}-\mathbf{y})\partial^y_iA({\bf y})\right],
\label{1.11}
\end{eqnarray}
%
with $dS$ representing the integration over a closed surface $S$. If the asymptotic surface term vanishes, then $A$ will also vanish identically. Hence we arrive at two non-trivial solutions to $\partial_i\partial^iA=0$, a) with $A$ constant or b) of the form ${\bf n}\cdot {\bf x}$ with ${\bf n}$ a normal vector. However, both of these must be explicitly excluded if they are to be well behaved asymptotically. Hence, requiring that the asymptotic surface term in (\ref{1.11}) vanish consequently forces the remaining solution to $\partial_i\partial^iA=0$ to be $A=0$.

Using a polar coordinate basis, the formulation is adapted by using a polar (flat) metric $\gamma_{ij}$ and replacing (\ref{1.11}) by
%
\begin{eqnarray}
A(\textbf{x})=\int dS\left[A(\mathbf{y})\frac{\partial D^{(3)}(\mathbf{x},\mathbf{y})}{\partial  n} -D^{(3)}(\mathbf{x},\mathbf{y})\frac{\partial A(\mathbf{y})}{\partial n}\right],
\label{1.12a}
\end{eqnarray}
%
with $\partial/\partial n$ is the normal derivative to the surface S, and with the Green's function obeying
%
\begin{eqnarray}
\nabla_i\nabla^iD^{(3)}(\mathbf{x},\mathbf{y})=\gamma^{-1/2}\delta^3(\mathbf{x}-\mathbf{y}).
\label{1.13a}
\end{eqnarray}
%
Taking $D^{(3)}(\mathbf{x},\mathbf{y})=-1/4\pi|\mathbf{x}-\mathbf{y}|$, the surface integral then becomes
%
\begin{eqnarray}
A(\textbf{x})=\frac{1}{4\pi} \int dS\left[\frac{1}{|\mathbf{x}-\mathbf{y}|}\frac{\partial A(\mathbf{y})}{\partial n}-
A(\mathbf{y})\frac{\partial}{\partial  n}\frac{1}{|\mathbf{x}-\mathbf{y}|}\right].
\label{1.14a}
\end{eqnarray}
%
As written, the asymptotic surface term will vanish conditioned on $A(\mathbf{y})$ behaving as $1/r^{n}$ for $n>0$. 

With asympyotic conditions having been shown to yield the decomposition theorem, we now focus the discussion on the higher derivative relations \footnote{As with (\ref{2.6}) we note that we need to go to fourth-order derivatives to establish decomposition.} necessary for said decomposition. Specifically, we apply sequences of derivatives to the flat space fluctuation of the Einstein tensor (which we recall is gauge invariant entirely on its own in this background), to obtain
%
\begin{eqnarray}
0&=&\delta^{ab} \tilde{\nabla}_{b}\tilde{\nabla}_{a}\psi,
\nonumber\\
0&=&\delta^{ab} \tilde{\nabla}_{b}\tilde{\nabla}_{a} \delta^{cd} \tilde{\nabla}_{c}\tilde{\nabla}_{d}(\phi+\dot{B}  -\ddot{E}),
\nonumber\\
0&=&\delta^{ab} \tilde{\nabla}_{b}\tilde{\nabla}_{a} \delta^{cd} \tilde{\nabla}_{c}\tilde{\nabla}_{d}(B_i-\dot{E}_i),
\nonumber\\
0&=&\delta^{ab} \tilde{\nabla}_{b}\tilde{\nabla}_{a} \delta^{cd} \tilde{\nabla}_{c}\tilde{\nabla}_{d}(-\ddot{E}_{ij}+\delta^{ef} \tilde{\nabla}_{e}\tilde{\nabla}_{f}E_{ij}),
\label{2.19}
\end{eqnarray}
%
Inspection of \eqref{2.3} shows that a decomposition theorem requires
%
\begin{eqnarray}
0&=&- 2 \delta^{ab} \tilde{\nabla}_{b}\tilde{\nabla}_{a}\psi,
\nonumber\\
0&=&- 2 \tilde{\nabla}_{i}\dot{\psi},
\nonumber\\
0&=&\tfrac{1}{2} \delta^{ab} \tilde{\nabla}_{b}\tilde{\nabla}_{a}(B_{i} -  \dot{E}_{i}),
\nonumber\\
0&=&-2 \delta_{ij} \ddot{\psi} -  \delta^{ab} \delta_{ij} \tilde{\nabla}_{b}\tilde{\nabla}_{a}(\phi+\dot{B}  -\ddot{E})+ \delta^{ab} \delta_{ij} \tilde{\nabla}_{b}\tilde{\nabla}_{a}\psi 
\nonumber\\
&& +\tilde{\nabla}_{j}\tilde{\nabla}_{i}(\phi+\dot{B} -  \ddot{E}) - \tilde{\nabla}_{j}\tilde{\nabla}_{i}\psi,
\nonumber\\
0&=&\tfrac{1}{2} \tilde{\nabla}_{i}(\dot{B}_{j} - \ddot{E}_{j}) + \tfrac{1}{2} \tilde{\nabla}_{j}(\dot{B}_{i} 
- \ddot{E}_{i}),
\nonumber\\
0&=&- \ddot{E}_{ij} + \delta^{ab} \tilde{\nabla}_{b}\tilde{\nabla}_{a}E_{ij}.
\label{2.20}
\end{eqnarray}
%
Within \eqref{2.20}, for any SVT quantity $D$ that obeys  
\begin{eqnarray}
\delta^{ab} \tilde{\nabla}_{a}\tilde{\nabla}_{b}D=0\qquad \rm{or} \qquad \delta^{ab} \tilde{\nabla}_{a}\tilde{\nabla}_{b}\delta^{cd} \tilde{\nabla}_{c}\tilde{\nabla}_{d}D=0
\label{hdconts}
\end{eqnarray} if impose spatial boundary conditions such that $D$ or $\delta^{ab} \tilde{\nabla}_{a}\tilde{\nabla}_{b}D$ vanishes,  the decomposition theorem will then follow for the fluctuation $\delta G_{\mu\nu}$. Thus, while a decoupling of the SVT representations can be achieved by going to higher derivatives, it is still necessary to constrain the behavior of such quantities according to \eqref{hdconts} in order to recover the decomposition theorem. 

%%%%%%%%%%%%%%%%%%%%%%%%%%%%%%%%%%%%%
\subsection{SVT4}
%%%%%%%%%%%%%%%%%%%%%%%%%%%%%%%%%%%%%

Similar to our treatement of the SVT3 decomposition theorem, we now investigation the status of the theorem in the SVT4 basis. We being with the four-dimensional analog of (\ref{1.3}): 
%
\begin{eqnarray}
F_{\mu}+\partial_{\mu}F=C_{\mu}+\partial_{\mu}C,
\label{1.34a}
\end{eqnarray}
%
where the $F$ and $F_{\mu}$ are given by (\ref{1.32a}), and where the $C$ and $C_{\mu}$ are representative of the fluctuation equatoins, with $C_{\mu}$ obeying  $\partial_{\mu}C^{\mu}=0$. For the decomposition theorem to hold one must take
%
\begin{eqnarray}
F_{\mu}= C_{\mu},\quad \partial_{\mu}F=\partial_{\mu}C.
\label{1.35a}
\end{eqnarray}
%
Just as with the SVT3 case, (\ref{1.35a}) does not follow from (\ref{1.34a}), since on applying $\partial_{\mu}$  and $\epsilon_{\mu\nu\sigma\tau}n^{\nu}\partial^{\sigma}$ we obtain
%
\begin{eqnarray}
\partial_{\mu}\partial^{\mu}(F-C)=0,\quad \epsilon_{\mu\nu\sigma\tau}n^{\nu}\partial^{\sigma}(F^{\tau}-C^{\tau})=0.
\label{1.36a}
\end{eqnarray}
%
By applying derivatives, we have indeed successfully decomposed the components. However, this is determined only up to a function  $D=F-C$, which need only to obey $\partial_\mu\partial^\mu D = 0$. We proceed to use a complete basis of plane waves to represent $D$ (now four-dimensional plane waves, covering the $\partial_{\mu}\partial^{\mu}$ operator), to obtain
%
\begin{eqnarray}
D=\sum _{\bf k}a_{\bf k}e^{i\textbf{k}\cdot \textbf{x}-ikt},
\label{1.37a}
\end{eqnarray}
%
where $k=|{\bf k}|$. Importantly, in contrast to the $a_{\bf k}$ in (\ref{1.6}) which obey $k_ik^ia_{\bf k}=0$, here we have no such constraint on $a_{\bf k}$, as the $a_{\bf k}$ obey $k_{\mu}k^{\mu}a_{\bf k}=0$ where $k_{\mu}k^{\mu}={\bf k}^2-k^2$ is identically zero. In addition we can continue retaining the form of $\partial_{\mu}\partial^{\mu}D=0$ by taking $a_{\bf k}=\exp(-a^2{\bf k}\cdot{\bf k})$. Upon taking the real part of $D$, we obtain
%
\begin{eqnarray}
{\rm Re}[D]&=&\rm Re\left[\int \frac{d^3k}{(2\pi)^3}e^{-a^2k^2+i\textbf{k}\cdot \textbf{x}-ikt}\right]
\nonumber\\
&=&
\frac{1}{16\pi^{3/2}a^3}\left[\frac{r+t}{r}e^{-(r+t)^2/4a^2}+\frac{r-t}{r}e^{-(r-t)^2/4a^2}\right].
\label{1.38a}
\end{eqnarray}
%
We see that \eqref{1.38a} is in fact not constrained by a spatially asymptotic condition since 
as $r\rightarrow \infty$, ${\rm Re}[D]$ falls off as $\exp(-r^2)$, both for fixed $t$ and for points on the light cone where $r=\pm t$. In fact, such a $D$ is even being well-behaved at $r=0$. Thus because of the metric signature (i.e. opposing sign of space and time coordinates), the quantity $k_\mu k^\mu=0$ leads to a form of $D$ that cannot generally satisfy the decomposition theorem, even in presence of the constraint $\partial_\mu \partial^\mu D = 0$. The investigation of whether a decomposition theorem can be satisfied at all must be performed on a case by case basis according to the differing background geometries, of which we explore within Ch. \ref{c:construction_and_solution_of_svt}.
