
\chapter{Decomposition Theorem and Boundary Conditions}
\label{c:decomposition_theorem}


One of the key features of this present study is in exploring the role that these very same boundary conditions play in establishing the so-called decomposition theorem. Specifically, in attempts to solve cosmological  evolution fluctuation equations that have been presented in the literature appeal is commonly made to the decomposition theorem in which it is assumed that the fluctuation equations are solved independently by the separate scalar, vector and tensor sectors, so that these sectors then evolve independently. Thus for the schematic example in which the fluctuation equations  take the generic flat space form
%
\begin{eqnarray}
B_i+\partial_iB=C_i+\partial_iC,
\label{1.3}
\end{eqnarray}
%
where the $B$ and $B_i$ are given by (\ref{1.1}), and where the $C$ and $C_i$ are functions given by the evolution equations with $C_i$ obeying  $\partial_iC^i=0$, the decomposition theorem requires that one set
%
\begin{eqnarray}
B_i= C_i,\quad \partial_iB=\partial_iC.
\label{1.4}
\end{eqnarray}
%
However, (\ref{1.4}) does not follow from (\ref{1.3}), since on applying $\partial^i$ and $\epsilon^{ijk}\partial_j$  to (\ref{1.3}) we obtain 
%
\begin{eqnarray}
\partial^i\partial_i(B-C)=0,\quad \epsilon^{ijk}\partial_j(B_k-C_k)=0,
\label{1.5}
\end{eqnarray}
%
and from this we can only conclude that $B$ and $C$ can differ by any function $B-C=D$ that obeys $\partial^i\partial_iD=0$, while $B_k$ and $C_k$ can differ by any function $B_k-C_k=D_k$ that obeys $\epsilon^{ijk}\partial_jD_k=0$, i.e. by any $D_k$ that can be written as the gradient of a scalar. Thus in (\ref{1.5}) we have separated the various scalar and vector components that are present in (\ref{1.3}), to thus obtain a decomposition for the components. However, we cannot proceed from (\ref{1.5}) to (\ref{1.4}) without providing some further information, and as we show below, in order to do so in this particular case we will only need to impose spatially asymptotic boundary conditions of the type that we referred to above.


We would like to state again that in obtaining (\ref{1.5}) we have not obtained (\ref{1.4}) itself, viz. the conditions that would be required by the decomposition theorem, but have instead obtained a derivative version of it. As we will see below, in the various cosmological models that we shall study it will be characteristic that while the scalar, vector and tensor combinations can indeed be separated, they can only be separated at a higher-derivative level. And not only that, in some cases they only separate at a quite high derivative level. The objective of this paper is to first seek the relevant higher-derivative level at which the general scalar, vector and tensor combinations do indeed separate in some relevant cosmological models, and to then seek conditions such as asymptotic boundary conditions under which the scalar, vector and tensor combinations might then separate at the level of the equations of motion themselves. In not all of the cases that we study will this prove to be the case. The analyses of all of the various cosmological models of interest involve many steps, with the most straightforward for the reader being the SVT3 analysis of fluctuations around a de Sitter background that we provide in Sec. \ref{S7}. 


However, before getting into the complexity of actual cosmological models, and so as to give the reader a sense of what is needed in order to derive the decomposition theorem, in this introduction we shall provide a few generic examples that do not involve the need to go to a particularly high derivative level. Thus for (\ref{1.5}) itself for instance we now show that in order to be able to proceed from (\ref{1.5}) to (\ref{1.4}) we need to impose asymptotic boundary conditions. Specifically, since three-dimensional plane waves form a complete basis for the operator $\partial_i\partial^i$, we can write a general solution for $B-C=D$ in the form
%
\begin{eqnarray}
D=\sum _{\bf k}a_{\bf k}e^{i\textbf{k}\cdot \textbf{x}},
\label{1.6}
\end{eqnarray}
%
where the $a_{\bf k}$ are constrained to obey 
%
\begin{eqnarray}
{\bf k}^2a_{\bf k}=0.
\label{1.7}
\end{eqnarray}
%
However, in and of itself (\ref{1.7}) does not lead to $a_{\bf k}=0$ (and thus to $D=0$) as this is not the only allowed solution to (\ref{1.7}). Rather, since $k^2\delta(k)=0$, $k^2\delta(k)/k=0$ we can set
%
\begin{eqnarray}
a_{\bf k}&=&\alpha_k\delta(k_x)\delta(k_y)\delta(k_z)
\nonumber\\
&&+\beta_k\left[\frac{\delta (k_x)\delta (k_y)\delta (k_z)}{k_x}+\frac{\delta (k_x)\delta (k_y)\delta (k_z)}{k_y}+\frac{\delta (k_x)\delta (k_y)\delta (k_z)}{k_z}\right],
\label{1.8}
\end{eqnarray}
%
where $\alpha_k$ and $\beta_k$ are constants. Inserting the $\alpha_k$ term  into (\ref{1.6}) would remove the ${\bf x}$ dependence from $D$ and provide a constant $D$ that would then not vanish at spatial infinity. Inserting the $\beta_k$ term  into (\ref{1.6}) would provide a $D$ that grows linearly in ${\bf x}$, to thus also not vanish at spatial infinity. Thus a spatially convergent  form for $D$ that would, for instance  be provided by taking  $a_{\bf k}$ to be a convergent $\exp(-{\bf k}^2a^2)$ Gaussian in momentum space would be excluded, with  $\partial_i\partial^iD=0$ having no localized solutions at all. If we can exclude non-zero $D$, we can set $B=C$, and thus via (\ref{1.3})  can set $B_i=C_i$.  A spatially asymptotic boundary condition is thus needed in order to recover a decomposition theorem. Consequently, we can correlate the establishing of the decomposition theorem with the very existence of the SVT3 basis in the first place as both require asymptotic boundary conditions.


Further insight into the role of boundary conditions can be provided by studying the behavior of $\partial_i\partial^iD=0$ in coordinate space. To this end  we recall the identity
%
\begin{eqnarray}
A \partial_i\partial^iB-B \partial_i\partial^iA=\partial_i(A\partial^iB-B\partial^iA).
\label{1.9}
\end{eqnarray}
%
Taking the generic $A$, like $D$, to be a function that obeys $\partial_i\partial^iA=0$ and taking $B$ to be the propagator $D^{(3)}(\mathbf{x}-\mathbf{y})$ that obeys 
%
\begin{eqnarray}
\partial_i\partial^iD^{(3)}(\mathbf{x}-\mathbf{y})=\delta^3(\mathbf{x}-\mathbf{y}),
\label{1.10}
\end{eqnarray}
%
enables us to write $A$ as an asymptotic surface term of the form 
%
\begin{eqnarray}
A({\bf x}) =\int dS_y^i\left[A({\bf y})\partial^y_iD^{(3)}(\mathbf{x}-\mathbf{y})-D^{(3)}(\mathbf{x}-\mathbf{y})\partial^y_iA({\bf y})\right],
\label{1.11}
\end{eqnarray}
%
as integrated over a closed surface $S$. The vanishing of the asymptotic surface term then makes $A$ vanish identically. Thus the two non-trivial solutions to $\partial_i\partial^iA=0$, viz. $A$  is a constant or of the form ${\bf n}\cdot {\bf x}$ where ${\bf n}$ is a reference vector (the coordinate analogs of  (\ref{1.8})), are then excluded by an asymptotic boundary condition. Requiring that the asymptotic surface term in (\ref{1.11}) vanish then forces the only solution to $\partial_i\partial^iA=0$ to be $A=0$.

In cosmology where it is convenient to use  polar coordinates, one has to adapt (\ref{1.11}). When written  in polar coordinates with still flat metric $\gamma_{ij}$ and metric determinant $\gamma$, (\ref{1.11}) is replaced by 

%
\begin{eqnarray}
A(\textbf{x})=\int dS\left[A(\mathbf{y})\frac{\partial D^{(3)}(\mathbf{x},\mathbf{y})}{\partial  n} -D^{(3)}(\mathbf{x},\mathbf{y})\frac{\partial A(\mathbf{y})}{\partial n}\right],
\label{1.12a}
\end{eqnarray}
%
where $\partial/\partial n$ is the out-directed normal derivative on the surface S, and the propagator obeys
%
\begin{eqnarray}
\nabla_i\nabla^iD^{(3)}(\mathbf{x},\mathbf{y})=\gamma^{-1/2}\delta^3(\mathbf{x}-\mathbf{y}).
\label{1.13a}
\end{eqnarray}
%
For $D^{(3)}(\mathbf{x},\mathbf{y})=-1/4\pi|\mathbf{x}-\mathbf{y}|$, (\ref{1.12a}) takes the form
%
\begin{eqnarray}
A(\textbf{x})=\frac{1}{4\pi} \int dS\left[\frac{1}{|\mathbf{x}-\mathbf{y}|}\frac{\partial A(\mathbf{y})}{\partial n}-
A(\mathbf{y})\frac{\partial}{\partial  n}\frac{1}{|\mathbf{x}-\mathbf{y}|}\right].
\label{1.14a}
\end{eqnarray}
%
The asymptotic surface term will thus vanish if $A(\mathbf{y})$ behaves as $1/r^{n}$ where $n$ is positive.


