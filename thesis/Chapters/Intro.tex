
\chapter{Introduction}
\label{c:introduction}

In the theory of cosmological fluctuations, the perturbative equations of motion have historically been acknowledged as forming a quite difficult and complex set of coupled non-linear differential tensor equations. 
%%%%%%%%%%%%%%%%%%%%%%%%%%%%%%%%%%%%%%%%%%%%
\footnote{For instance, within Weinberg's cosmology book \cite{weinberg_2008}, he notes that even after performing additional simplifications, the cosmological perturbative equations are still ``fearsomely complicated''.}
%%%%%%%%%%%%%%%%%%%%%%%%%%%%%%%%%%%%%%%%%%%
As a result, extensive methods of simplifying the equations of motion and eliminating non-physical gauge modes have been continually developed in order to construct the perturbative solutions, with such methods broadly falling into one of two categories. 

The first entails the construction and imposition of a suitable gauge condition that aims to reduce the equations of motion into a simplified form such that a solution is readily obtainable. Here, no further decomposition upon the fluctuations is performed, with solutions being expressed directly in terms of the metric tensor fluctuation and the energy momentum tensor components. While use of such gauges are quite effective at facilitating solutions within Minkowski background geometries, their efficacy begins to diminish when faced with backgrounds associated with a non-vanishing curvature tensor and hence backgrounds that are most relevant to the study of cosmology. However, we note that there are some interesting exceptions regarding a gauge's reductive power that can occur if one works within gravitational theories that possess additional symmetries in comparison to standard Einstein gravity; we will in fact explore such an alternative theory, namely conformal gravity, extensively throughout this work. 

The second method of simplification of the equations of motion consists of decomposing the metric and matter fluctuations into a basis of scalars, vectors, and tensors. First developed by Lifshitz in 1946 \cite{lifshitz_2017}, the scalar, vector, tensor (SVT) basis is characterized according to how the fluctuations transform under three-dimensional rotations, and has become the de facto approach to modern treatments of cosmology \cite{bardeen_1980, bertschinger_2000, ellis_maartens_maccallum_2009, mukhanov_1992, york_1973, weinberg_2008}. In fact, such an SVT decomposition is deeply intertwined within the calculation of large scale observables in the universe such as dominant scalar field contribution within the cosmic microwave background power spectrum \cite{hu_dodelson_2002,kodama_sasaki_1984}. Within literature \cite{kodama_sasaki_1984, bardeen_1980, ellis_maartens_maccallum_2009}, the utilization of the SVT basis actually goes well beyond just a means of decomposing the perturbations, with its main efficacy rather coming from the assertion that within the equations of motion, the scalar, vector, and tensor sectors are to completely decouple. Such assertion is referred to as the \emph{cosmological decomposition theorem}.

As the analysis of the cosmological decomposition theorem forms one of the primary focal points of this work, we motivation the investigation into its application by considering a representative example of a scalar-vector decomposition of two vectors fields into their transverse and longitudinal components via
%
\begin{eqnarray}
B_i+\partial_iB=C_i+\partial_iC.
\label{I.3}
\end{eqnarray}
%
Here a transverse $B_i$ obeys $\partial^i B_i = 0$ and $C$ and $C_i$ are to represent functions given by the evolution equations with $C_i$ also transverse and thus obeying $\partial_iC^i=0$. According to the decomposition theorem, the vectors $B_i$ and $C_i$ are taken to decouple and evolve independent of the scalars, with an analogous statement holding for the scalars. Hence, the theorem implies
%
\begin{eqnarray}
B_i= C_i,\quad \partial_iB=\partial_iC.
\label{I.4}
\end{eqnarray}
%
It is clear, however, that (\ref{I.4}) does not directly follow from (\ref{I.3}). Rather, if we apply $\partial^i$ and $\epsilon^{ijk}\partial_j$  to (\ref{I.3}) we obtain 
%
\begin{eqnarray}
\partial^i\partial_i(B-C)=0,\quad \epsilon^{ijk}\partial_j(B_k-C_k)=0,
\label{I.5}
\end{eqnarray}
%
and from this one can only conclude that $B$ and $C$ are defined up to an arbitrary scalar $D$ obeying $\partial^i\partial_iD=0$. For the vector sector analogously, $B_k$ and $C_k$ can only differ by any function $D_k$ that obeys $\epsilon^{ijk}\partial_jD_k=0$, i.e. an irrotational field $D_k$ expressed as the gradient of a scalar. In composing (\ref{I.5}) we have appropriately separated the scalar and vector components within(\ref{I.3}), obtaining a decomposition for the components that does not follow by proceeding from (\ref{I.5}) to (\ref{I.4}). Specifically, without providing some additional information or constraints, one cannot directly proceed from (\ref{I.5}) to (\ref{I.4}). However, given the imposition of such constraints, such a decomposition may be possible, with the constraints in fact needing to be in form of spatially asymptotic boundary conditions. Alternatively, if one were to elect to form a decomposition without the imposition of boundary condition or constraints, we see from \eqref{I.5} that we necessarily need to go to higher derivatives in order to establish the decomposition. Thus we will continue to investigate both branches of applying said decomposition theorem.

	he application of the SVT decomposition 


the perturbative equations of motion have historically been acknowledged as forming a rather complex set of coupled non-linear differential tensor equations. For instance, within Weinberg's cosmology book \cite{weinberg_2008}, he notes that even after performing additional simplifications that these equations are still ``fearsomely complicated''. 






extensive methods of simplifying the equations of motion and eliminating non-physical gauge modes are required in order to construct the perturbative solutions
%%%%%%%%%%%%%%%%%%%%%%%%%%%%%%%%%%%%%%%%%%%%%%%%%%%%%%%%%%%%%%%%%%%%%%%%%%%%%%%%%%%%%%%%%%%%%%%%%%%%%


*touch on decomposition theorem. maybe a little mock example
*conformal gravity - while we do not dig into details on qm here, has been proposed as a ... citations. dope as properties
*overview section
*imposing gauges or for intrinsic gauge invariants. difficult problem to solve, complicated fluctuations. as we move to curved spacetimes, solving not straightforward even with gauge constraints
** SVT3, scalar modes, CMB power spectrum. then motivate SVT4. projectors
* weinberg formidablly complicated


%%%%%%%%%%%%%%%%%%%%%%%%%%%%%%%%%%%%%%%%%%%%%%%%%%%%%%%%%%%%%%%%%%%%%%%%%%%%%%%%%%%%%%%%%%%%%%%%%%%%%
solve equations, Determine conditions required for decomposition theorem. Does not hold unless further input. by going to higher derivatives. See if we can impose asymptotic boundary conditions. 

start introducing perfect fluid source. RW k=0 radiation. Determine matter + gravitaional gauge invariants. $\tau^2 e_ij$ sector goes as $t^{1/2}$, decomp follows w/ spatially asympt. bc's.

generalize to all RW, perturb perfect fluid, requires equation of state. identify gauge invariants, use many vuarvature relations, commutations 4th order. seek help from svt in terms of $h_{\mu\nu}$ to determine gauge invariants here. In order to solve, need to determine $\Omega(\tau)$ and reduce 11 dof's to 10. We then determine the form of $\Omega(\tau)$ in all curvatures in radiation and matter dominated. Reduce from 11 to 10 by specifying equation of state. We interpolate btween radiation and matter: transition between two eras is complicationed, but propto $p=w\rho$ in high temp (radiation) and low temp (matter). Transition era = recombination. $p=w\rho$ not always valid. Solve by suming over complete basis of modes associated with propagation of spinless massive particle in chosen $g_{\mu\nu}$ background. Complicated, but not done here generally. 

k=-1 RW general As the implications of boundary conditions are very sensitive to the sign of the coefficient of $k$, and we will need to monitor both positive and negative coefficient cases below. In implementing evolution equations that involve products of derivative operators such as the generic $(\tilde{\nabla}^2+\alpha)(\tilde{\nabla}^2+\beta)F=0$ . scalar sector checks out given good behavior (bounded) at infinity and origin. Seem to find vector that is bounded, well behaved at both, but does not obey decomp theorem. see end of vector section. Same for tensor sector.

n Sec. \ref{ss:rw_k=-1_svt3} we have seen that there are realizations of the evolution equations in the scalar, vector, and tensor sectors that would not lead to a decomposition theorem in those sectors. However, equally there are other realizations that given the boundary conditions would lead to a decomposition theorem. Thus we need to determine which realizations are the relevant ones. To this end we look not at the individual higher-derivative equations obeyed by the separate scalar, vector, and tensor sectors, but at how these various sectors interface with each other in the original second-order $\Delta_{\mu\nu}=0$ equations themselves. Any successful such interface would require that all the terms in $\Delta_{\mu\nu}=0$ would have to have the same $\chi$ behavior. Noting that the scalar modes appear with two $\tilde{\nabla}$ derivatives in $\Delta_{ij}=0$, the vector sector appears with one $\tilde{\nabla}$ derivative and the tensor appears with none, we need to compare derivatives of scalars with vectors and derivatives of vectors with tensors. 

If we force bc that vector and tensor modes vanish at $\chi=\infty$ instead of limiting to a constant value, then decomp holds. 

compute svt3 conformal gravity, instead working in conformal flat. Imposing boundary conditions leads to simple evolution equations. We can invert svt3 quantities in terms of $\delta W_{\mu\nu}$, to serve as alternative integral relations in the RW background. 

SVT4 minkowski, delta G is purely gauge invariant in zero background. Evaluate in ds4 w/o conformal factor. Make use of SVTD in constant 3 space. For scalar $\chi$ to obey decomp theorem, require very particular solution. General solution not at all forced to $\chi =0$; specific solution to the full evolution equations. No compelling reason to choose so. $F_{\mu\nu}$ and $\chi$ can still be localized in space, thus no spatially asymptotic bc could affect them. Completely solvable though. Could enforce decomp theorem with judiciously choosing IC's at an intial time. No compelling rationale for doing so. 

ds4 with a conformal factor. GI mixes scalars and vectors. Introduce $U^\mu$ to express covariantly. Find exact solutions. Again, same story. 

Do SVT4 desitter in conformal gravity. Simple structure, in fact TT sector has same form as standard Einstein gravity. Find relation between the two. Below, and also for flat space. Decomp is automatic, only $F_{\mu\nu}$. 
\begin{eqnarray}
\delta W_{\mu\nu}=(\nabla_{\alpha}\nabla^{\alpha}-4H^2)(\delta G_{\mu\nu}+\delta T_{\mu\nu})^{T\theta}.
\end{eqnarray}

By introducing timelike $U^\mu$ can express generalized SVT4 RW flucations in compact covariant form. Same story with decomp. 

dW conformal to flat. Beautiful super simple. 
\begin{eqnarray}
\delta W_{\mu\nu}&=&\Omega^{-2}\partial_{\sigma}\partial^{\sigma}\partial_{\tau}\partial^{\tau}F_{\mu\nu}.
\end{eqnarray} 

by looking at ds4 svt4 we saw mixing, and thus one shold only look for decomp theorem for gauge invariants and not for seaprate scalar vector, tensors sects and gauage invariance can in general intertwine them. We present example occuring in SVT3 to show not just artifact of SVT4. 

ads svt3 mixing. 
\begin{eqnarray}
\delta = \phi -\psi + \dot B - \ddot E + \frac{2}{z}(\tilde\nabla_3 E + E_3),
\end{eqnarray}

General conformal to flat. $eta=\psi -\Omega^{-1}\dot{\Omega}(B-\dot E)+\Omega^{-1}\tilde\nabla^i\Omega(E_i+\tilde\nabla_i E)$ Spatially dependent $\Omega(x)$ leads to inseperable gauge invariant not foudn in non-conformal geoms. Procedure is one must first determine GI's, then separate, not other way around. For some geoms no choice of coords can undo intertwiing (if conformal factor pulled out). To express the $\eta$ in terms of a curvature invariant, one cannot make recourse to $\delta W_{\mu\nu}$, however, in traceless radiation we can take $\delta(g_{\mu\nu}G^{\mu\nu}$ to determine $\eta$ in terms of the $h_{\mu\nu}$. 

Using gauge freedom, imposing gauge, we can decouple intertwining in gauge invariants. 