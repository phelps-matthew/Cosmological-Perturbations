\documentclass[12pt, letterpaper, center, noupper]{uconnthesis}

\ThesisLineSpacing{2}
\usepackage[square,sort,comma,numbers]{natbib}
%\bibliographystyle{plainurl} %permits urls within bibliography; causes some errors
\renewcommand{\bibsection}{} %removes double header for bibliography section
\usepackage[fleqn]{mathtools} 
\usepackage{amssymb,hyperref,xcolor}
%\usepackage[varg]{txfonts}
%\usepackage[nointegrals]{wasysym}
\usepackage{graphicx}
%\usepackage[colorlinks=true]{hyperref}
%\hypersetup{colorlinks, linkcolor={blue!50!black}, citecolor={red!70!black}, urlcolor={blue!80!black}}
\hypersetup{colorlinks, linkcolor=[HTML]{83A598}, citecolor=[HTML]{CC241D}, urlcolor=[HTML]{458588}}
\graphicspath{{Figs/}}
\setcounter{tocdepth}{2}
\allowdisplaybreaks
%
% gruvbox it out
\pagecolor[HTML]{282828}
\makeatletter
\newcommand{\globalcolor}[1]{%
	\color[HTML]{#1}\global\let\default@color\current@color
}
\makeatother
% stop gruvboxing it out
%
\begin{document}
\globalcolor{EBDBB2}
\abstract{
In the theory of cosmological perturbations \cite{phelps_2019}, \cite{amarasinghe_2019} extensive methods of simplifying the equations of motion and eliminating non-physical gauge modes are required in order to construct the perturbative solutions. One approach is to fix the gauge freedom by imposing coordinate constraints. In the context of conformal gravity, we aim to continue work done in obtaining solutions to the cosmological fluctuation equations \cite{mannheim_2012} by constructing a gauge condition that is conformally invariant. Determination of the appropriate gauge would permit the full set of exact solutions to the fluctuations equations to be obtained. Another method used  extensively in cosmology is the scalar, vector, tensor (SVT) decomposition\cite{ellis_maartens_maccallum_2009}. Here the perturbed metric is decomposed in terms of SO(3) representations whereby it is asserted that the scalars, vectors, and tensors decouple within the equations of motion. We aim to investigate issues of gauge invariance and asymptotic behavior in an integral formalism of the SVT decomposition. Moreover, we propose to characterize the role of boundary conditions in the SVT separation of the equations of motion.
}

\title{Cosmological Fluctuations in Standard and Conformal Gravity}
\author{Matthew Phelps}
\submityear{2020}
\authorspreviousdegreelong{
B.S., University of Colorado Colorado Springs, 2014
}
\authorspreviousdegreeshort{B.S.}

\MajorAdvisor{Philip Mannheim}
\AssociateAdvisorA{Alex Kovner}
\AssociateAdvisorB{Vasili Kharschenko}

%\dedication{
%Dedication goes here.
%}

\acknowledgements{
Insert acknowledgments here.
}

%\maketitle
%\frontmatter
\tableofcontents
%\listoffigures
%\listoftables
\mainmatter

% Chapters of the thesis

\chapter{Introduction}
\label{c:introduction}

solve equations, Determine conditions required for decomposition theorem. Does not hold unless further input. by going to higher derivatives. See if we can impose asymptotic boundary conditions. 

start introducing perfect fluid source. RW k=0 radiation. Determine matter + gravitaional gauge invariants. $\tau^2 e_ij$ sector goes as $t^{1/2}$, decomp follows w/ spatially asympt. bc's.

generalize to all RW, perturb perfect fluid, requires equation of state. identify gauge invariants, use many vuarvature relations, commutations 4th order. seek help from svt in terms of $h_{\mu\nu}$ to determine gauge invariants here. In order to solve, need to determine $\Omega(\tau)$ and reduce 11 dof's to 10. We then determine the form of $\Omega(\tau)$ in all curvatures in radiation and matter dominated. Reduce from 11 to 10 by specifying equation of state. We interpolate btween radiation and matter: transition between two eras is complicationed, but propto $p=w\rho$ in high temp (radiation) and low temp (matter). Transition era = recombination. $p=w\rho$ not always valid. Solve by suming over complete basis of modes associated with propagation of spinless massive particle in chosen $g_{\mu\nu}$ background. Complicated, but not done here generally. 

k=-1 RW general As the implications of boundary conditions are very sensitive to the sign of the coefficient of $k$, and we will need to monitor both positive and negative coefficient cases below. In implementing evolution equations that involve products of derivative operators such as the generic $(\tilde{\nabla}^2+\alpha)(\tilde{\nabla}^2+\beta)F=0$ . scalar sector checks out given good behavior (bounded) at infinity and origin. Seem to find vector that is bounded, well behaved at both, but does not obey decomp theorem. see end of vector section. Same for tensor sector.

n Sec. \ref{ss:rw_k=-1_svt3} we have seen that there are realizations of the evolution equations in the scalar, vector, and tensor sectors that would not lead to a decomposition theorem in those sectors. However, equally there are other realizations that given the boundary conditions would lead to a decomposition theorem. Thus we need to determine which realizations are the relevant ones. To this end we look not at the individual higher-derivative equations obeyed by the separate scalar, vector, and tensor sectors, but at how these various sectors interface with each other in the original second-order $\Delta_{\mu\nu}=0$ equations themselves. Any successful such interface would require that all the terms in $\Delta_{\mu\nu}=0$ would have to have the same $\chi$ behavior. Noting that the scalar modes appear with two $\tilde{\nabla}$ derivatives in $\Delta_{ij}=0$, the vector sector appears with one $\tilde{\nabla}$ derivative and the tensor appears with none, we need to compare derivatives of scalars with vectors and derivatives of vectors with tensors. 

If we force bc that vector and tensor modes vanish at $\chi=\infty$ instead of limiting to a constant value, then decomp holds. 

compute svt3 conformal gravity, instead working in conformal flat. Imposing boundary conditions leads to simple evolution equations. We can invert svt3 quantities in terms of $\delta W_{\mu\nu}$, to serve as alternative integral relations in the RW background. 

SVT4 minkowski, delta G is purely gauge invariant in zero background. Evaluate in ds4 w/o conformal factor. Make use of SVTD in constant 3 space. For scalar $\chi$ to obey decomp theorem, require very particular solution. General solution not at all forced to $\chi =0$; specific solution to the full evolution equations. No compelling reason to choose so. $F_{\mu\nu}$ and $\chi$ can still be localized in space, thus no spatially asymptotic bc could affect them. Completely solvable though. Could enforce decomp theorem with judiciously choosing IC's at an intial time. No compelling rationale for doing so. 

ds4 with a conformal factor. GI mixes scalars and vectors. Introduce $U^\mu$ to express covariantly. Find exact solutions. Again, same story. 

Do SVT4 desitter in conformal gravity. Simple structure, in fact TT sector has same form as standard Einstein gravity. Find relation between the two. Below, and also for flat space. Decomp is automatic, only $F_{\mu\nu}$. 
\begin{eqnarray}
\delta W_{\mu\nu}=(\nabla_{\alpha}\nabla^{\alpha}-4H^2)(\delta G_{\mu\nu}+\delta T_{\mu\nu})^{T\theta}.
\end{eqnarray}

By introducing timelike $U^\mu$ can express generalized SVT4 RW flucations in compact covariant form. Same story with decomp. 

dW conformal to flat. Beautiful super simple. 
\begin{eqnarray}
\delta W_{\mu\nu}&=&\Omega^{-2}\partial_{\sigma}\partial^{\sigma}\partial_{\tau}\partial^{\tau}F_{\mu\nu}.
\end{eqnarray} 

by looking at ds4 svt4 we saw mixing, and thus one shold only look for decomp theorem for gauge invariants and not for seaprate scalar vector, tensors sects and gauage invariance can in general intertwine them. We present example occuring in SVT3 to show not just artifact of SVT4. 

ads svt3 mixing. 
\begin{eqnarray}
\delta = \phi -\psi + \dot B - \ddot E + \frac{2}{z}(\tilde\nabla_3 E + E_3),
\end{eqnarray}

General conformal to flat. $eta=\psi -\Omega^{-1}\dot{\Omega}(B-\dot E)+\Omega^{-1}\tilde\nabla^i\Omega(E_i+\tilde\nabla_i E)$ Spatially dependent $\Omega(x)$ leads to inseperable gauge invariant not foudn in non-conformal geoms. Procedure is one must first determine GI's, then separate, not other way around. For some geoms no choice of coords can undo intertwiing (if conformal factor pulled out). To express the $\eta$ in terms of a curvature invariant, one cannot make recourse to $\delta W_{\mu\nu}$, however, in traceless radiation we can take $\delta(g_{\mu\nu}G^{\mu\nu}$ to determine $\eta$ in terms of the $h_{\mu\nu}$. 

Using gauge freedom, imposing gauge, we can decouple intertwining in gauge invariants. 

\chapter{Formalism}
\label{c:formalism}

Before we can enter the discussion of the technical methods used to decompose and simplify the cosmological fluctuation equations, we must first introduce the necessary formalism describing the interaction of gravitation and matter. The general procedure, repeated for both standard and conformal gravity, consists of varying a classical gravitational action (a general coordinate scalar) with respect to the metric, with stationary solutions yielding the equations of motion. The metric is then decomposed into zeroth and first order contributions where we obtain the background and perturbed fluctuation equations, respectively. Serving as a prototypical example of what is to come, we illustrate the form of the fluctuation equations in their simplest configuration, namely within a source-less Minkowski background geometry. Following convention \cite{weinberg_1972}, we impose a standard gauge condition (e.g, the harmonic or transverse gauge), allowing us to solve the equations of motion exactly.

In the case of conformal gravity, there are particular properties not shared within Einstein gravity \cite{mannheim_2012} that deserve special attention which are also explored here. Namely, the additional symmetry contained within conformal gravity permits extremely useful transformation properties and directly leads to very compact equations of motion (with one less degree of freedom) if the background metric itself can be shown the exhibit the same underlying symmetry properties. In addition, for matter actions relevant to conformal gravity (actions necessarily possessing conformal invariance) we explore two non-trivial geometries \cite{mannheim_kazanas_1988, mannheim_kazanas_1989, mannheim_1990} in which the background energy momentum tensor may be shown to vanish. We contrast the separation of gauge invariance within each sector (i.e. the gravitational and matter sector) with Einstein gravity, as applied to the equations of motion within the presence of non-trivial vacuum geometries.

Finally, we provide an overview of the spacetime geometries studied in cosmology and their underlying motivations. Via coordinate transformations, each of the cosmological geometries of interest can be cast into a conformal to flat form, a detail whose importance cannot be understated and serves a crucial role in the development and solution of the fluctuation equations throughout this work.

%%%%%%%%%%%%%%%%%%%%%%%%%%%%%%%%%%%%%%%%%%%%
\section{Einstein Gravity}
\label{s:einstein_gravity}
%%%%%%%%%%%%%%%%%%%%%%%%%%%%%%%%%%%%%%%%%%%%
The formulation of the Einstein field equations first begins by introducing the Einstein-Hilbert action \cite{weinberg_1972}
\begin{eqnarray}
I_{\text{EH}} = -\frac{1}{16\pi G} \int d^4x (-g)^{1/2}  g^{\mu\nu}R_{\mu\nu}.
\end{eqnarray}
Variation of this action with respect to $g_{\mu\nu}$ yields the Einstein tensor
\begin{eqnarray}
\frac{16\pi G}{(-g)^{1/2}} \frac{\delta I_{\text{EH}}}{\delta g_{\mu\nu}}= G^{\mu\nu} = R^{\mu\nu} - \frac{1}{2}g^{\mu\nu}R^\alpha{}_\alpha.
\label{Eintensor}
\end{eqnarray}
Upon specification of a matter action, $I_\text{M}$, an energy momentum tensor may likewise be constructed by variation with respect to the metric,
\begin{eqnarray}
\frac{2}{(-g)^{1/2}} \frac{ \delta I_\text{M}}{\delta g_{\mu\nu}} = T_{\mu\nu}. 
\end{eqnarray}
Requiring the total gravitational + matter action $I_{\text{EH}}+I_\text{M}$ to be stationary under variation of $g_{\mu\nu}$ then yields the Einstein equations of motion
\begin{eqnarray}
R^{\mu\nu} - \frac{1}{2}g^{\mu\nu}R^\alpha{}_\alpha = -8\pi G T^{\mu\nu}.
\label{EinEOM}
\end{eqnarray}
The Einstein tensor itself is covariantly conserved via the Bianchi identities,
\begin{eqnarray}
\nabla_\mu R^{\mu\nu} = \frac{1}{2}\nabla^\nu R^\mu{}_\mu \implies \nabla_\mu G^{\mu\nu} = 0.
\end{eqnarray}

As a first step towards describing fluctuations in the universe, we may decompose the metric $g_{\mu\nu}(x)$ into a background metric and a first order perturbation according to
\begin{eqnarray}
g_{\mu\nu}(x) = g_{\mu\nu}^{(0)}(x) + h_{\mu\nu}(x),\qquad g^{\mu\nu}h_{\mu\nu} \equiv h,
\end{eqnarray}
whereby $G_{\mu\nu}$ can be decomposed as
\begin{eqnarray}
G_{\mu\nu} = G_{\mu\nu}(g_{\mu\nu}^{(0)}) + \delta G_{\mu\nu}(h_{\mu\nu}).
\end{eqnarray}
By virtue of the Bianchi identities, the 10 components of the symmetric $G_{\mu\nu}$ can be reduced to 6 independent components in total. In terms of perturbations of the curvature tensors, the decomposition of $G_{\mu\nu}$ takes the form
\begin{eqnarray}
G_{\mu\nu}^{(0)} &=& R_{\mu\nu}^{(0)} -\frac{1}{2} g_{\mu\nu}^{(0)} R_\alpha^{(0)\alpha}
\label{Einzero}
\\
\delta G_{\mu\nu} &=& \delta R_{\mu\nu} - \frac{1}{2} h_{\mu\nu} R_\alpha^{(0)\alpha} -\frac{1}{2}g_{\mu\nu}\delta R^\alpha{}_\alpha.
\label{Einone}
\end{eqnarray}
Under a coordinate transformation of the form $x^\mu \to x^\mu - \epsilon^\mu(x)$, with $\epsilon^\mu$ of $\mathcal O(h)$, the perturbed metric transforms as
\begin{eqnarray}
h_{\mu\nu} \to h_{\mu\nu} + \nabla_\mu \epsilon_\mu + \nabla_\mu \epsilon_\nu. 
\label{gaugeh}
\end{eqnarray}
For every solution $h_{\mu\nu}$ to $\delta G_{\mu\nu}+8\pi G \delta T_{\mu\nu}$, a transformed $h'_{\mu\nu}=h_{\mu\nu} + \nabla_\mu \epsilon_\mu + \nabla_\mu \epsilon_\nu$ will also serve as a solution - hence the set of four functions $\epsilon^\mu$ serve to define the gauge freedom under coordinate transformation. If the gauge is fixed, the initial 10 components of the symmetric $h_{\mu\nu}$ are reduced to 6 individual components. As will be discussed later, one can also construct gauge invariant quantities as functions of the $h_{\mu\nu}$ and express the field equations entirely in terms of 6 gauge invariant functions (see Ch. \ref{c:scalar_vector_tensor_basis}).

As regards the perturbed gravitational and energy momentum tensors, under $x^\mu \to x^\mu - \epsilon^\mu(x)$ they transform as
\begin{eqnarray}
\delta G_{\mu\nu} \to \delta G_{\mu\nu} + {}^{(0)}G^\lambda{}_\mu \nabla_\nu \epsilon_\lambda +  {}^{(0)}G^{\lambda}{}_{\nu}\nabla_\mu \epsilon_\mu + \nabla_\lambda  G^{(0)}_{\mu\nu} \epsilon^\lambda
\nonumber\\
\delta T_{\mu\nu} \to \delta T_{\mu\nu} + {}^{(0)}T^\lambda{}_\mu \nabla_\nu \epsilon_\lambda +  {}^{(0)}T^{\lambda}{}_{\nu}\nabla_\mu \epsilon_\mu + \nabla_\lambda  T^{(0)}_{\mu\nu} \epsilon^\lambda.
\label{gaugetranstensor}
\end{eqnarray}
If $G_{\mu\nu}^{(0)}=0$, then $\delta G_{\mu\nu}$ will be separately gauge invariant. On the other hand, if $ G_{\mu\nu}^{(0)} \ne 0$, then it is only $\delta G_{\mu\nu} + 8\pi G \delta T_{\mu\nu}$ that is gauge invariant by virtue of the background equations of motion. (The transformation behavior of tensors under the gauge transformation as given in \eqref{gaugetranstensor}, otherwise known as the Lie derivative, is in fact generic to all tensors defined on the manifold. One can check that it readily holds for \eqref{gaugeh}). 

With aim towards a description of fluctuations in the universe, let us perturb the metric around an arbitrary background according to
\begin{eqnarray}
ds^2=g_{\mu\nu}dx^\mu dx^\nu = (g^{(0)}_{\mu\nu} + h_{\mu\nu} + \mathcal O(h^2)+ ...)dx^\mu dx^\nu.
\end{eqnarray}
The zeroth order $G_{\mu\nu}^{(0)}$ is given as \eqref{Einzero} and the first order fluctuation evaluates to 
\begin{eqnarray}
\delta G_{\mu\nu} &=& 
- \tfrac{1}{2} h_{\mu \nu } R + \tfrac{1}{2} g_{\mu \nu } h^{\alpha \beta } R_{\alpha \beta } + \tfrac{1}{2} \nabla_{\alpha }\nabla^{\alpha }h_{\mu \nu } -  \tfrac{1}{2} g_{\mu \nu } \nabla_{\alpha }\nabla^{\alpha }h \nonumber\\
&& -  \tfrac{1}{2} \nabla_{\alpha }\nabla_{\mu }h_{\nu }{}^{\alpha } -  \tfrac{1}{2} \nabla_{\alpha }\nabla_{\nu }h_{\mu }{}^{\alpha } + \tfrac{1}{2} g_{\mu \nu } \nabla_{\beta }\nabla_{\alpha }h^{\alpha \beta } + \tfrac{1}{2} \nabla_{\mu }\nabla_{\nu }h.
\label{dEin}
\end{eqnarray}
(Here the covariant derivatives $\nabla$ are defined with respect to the background $g_{\mu\nu}^{(0)}$ and all curvature tensors are taken as zeroth order). For the purpose of illustrating gauge fixing and, later, the SVT decomposition, we evaluate \eqref{dEin} within a Minkowski background viz.
\begin{eqnarray}
ds^2 &=& (\eta_{\mu\nu} + h_{\mu\nu})dx^\mu dx^\nu, \qquad \eta_{\mu\nu} = \text{diag}(-1,1,1,1),\qquad G^{(0)}_{\mu\nu} = 0
\nonumber\\
\delta G_{\mu\nu} &=& \tfrac{1}{2} \nabla_{\alpha }\nabla^{\alpha }h_{\mu \nu } -  \tfrac{1}{2} g_{\mu \nu } \nabla_{\alpha }\nabla^{\alpha }h -  \tfrac{1}{2} \nabla_{\alpha }\nabla_{\mu }h_{\nu }{}^{\alpha } -  \tfrac{1}{2} \nabla_{\alpha }\nabla_{\nu }h_{\mu }{}^{\alpha } 
\nonumber\\
&&+ \tfrac{1}{2} g_{\mu \nu } \nabla_{\beta }\nabla_{\alpha }h^{\alpha \beta } + \tfrac{1}{2} \nabla_{\mu }\nabla_{\nu }h.
\label{dEinflat}
\end{eqnarray}
In then taking $\delta T_{\mu\nu}=0$, the equations of motion describing the gravitational fluctuations in an empty universe (vacuum) are given by 
\begin{eqnarray}
0=\tfrac{1}{2} \nabla_{\alpha }\nabla^{\alpha }h_{\mu \nu } -  \tfrac{1}{2} g_{\mu \nu } \nabla_{\alpha }\nabla^{\alpha }h + \tfrac{1}{2} g_{\mu \nu } \nabla_{\beta }\nabla_{\alpha }h^{\alpha \beta } -  \tfrac{1}{2} \nabla_{\mu }\nabla_{\alpha }h_{\nu }{}^{\alpha }
\nonumber\\
 -  \tfrac{1}{2} \nabla_{\nu }\nabla_{\alpha }h_{\mu }{}^{\alpha } + \tfrac{1}{2} \nabla_{\nu }\nabla_{\mu }h.
\label{dEinflatEOM}
\end{eqnarray}


%%%%%%%%%%%%%%%%%%%%%%%%%%%%%%%%%%%%%%%%%%%%
\subsection{Fluctuations Around Flat in the Harmonic Gauge}
\label{ss:fluctuations_around_flat_in_the_harmonic_gauge}
%%%%%%%%%%%%%%%%%%%%%%%%%%%%%%%%%%%%%%%%%%%%

In order to solve \eqref{dEinflatEOM}, we have two general approaches: a). Use the coordinate freedom in $h_{\mu\nu}$ to impose a specific gauge, typically one in which the equations of motion are most simplified
b). Determine gauge invariant quantities as functions of $h_{\mu\nu}$ and express the equation of motion entirely in terms of the gauge invariants. 

As an example of using method a) to solve \eqref{dEinflatEOM}, we may select coordinates that satisfy the harmonic gauge condition \cite{weinberg_1972}
\begin{eqnarray}
\nabla^\mu h_{\mu\nu} - \tfrac{1}{2}\nabla_\nu h = 0,
\end{eqnarray} 
whereby \eqref{dEinflatEOM} reduces to 
\begin{eqnarray}
0= \tfrac{1}{2} \nabla_{\alpha }\nabla^{\alpha } \left(h_{\mu\nu} - \tfrac{1}{2} g_{\mu\nu} h\right).
\end{eqnarray}
The trace defines a solution for $h$ whereafter $h_{\mu\nu}$ can be solved as $\nabla_\alpha\nabla^\alpha h_{\mu\nu} = 0$. 

We will see that method b) is facilitated by the use of the scalar, vector, tensor decomposition as discussed in detail within Ch. \ref{c:scalar_vector_tensor_basis}.

%%%%%%%%%%%%%%%%%%%%%%%%%%%%%%%%%%%%%%%%%%%%
\section{Conformal Gravity}
\label{s:conformal_gravity}
%%%%%%%%%%%%%%%%%%%%%%%%%%%%%%%%%%%%%%%%%%%%

Conformal gravity is a candidate theory of gravitation based on a pure metric action that is not only invariant under local Lorentz transformations (to thus possess the properties of general coordinate invariance and adherenace to the equivalence principle as found in Einstein gravity) but also invariant under local conformal transformations (i.e. transformations of the form $g_{\mu\nu}(x) \to e^{2\alpha(x)}g_{\mu\nu}(x)$ with $\alpha(x)$ arbitrary). Such a metric action that obeys these symmetries is uniquely prescribed and is given by a polynomial function of the Riemann tensor \cite{mannheim_2006}
%
\begin{eqnarray}
I_{\rm W}&=&-\alpha_g\int d^4x\, (-g)^{1/2}C_{\lambda\mu\nu\kappa}
C^{\lambda\mu\nu\kappa}
\nonumber\\
&&\equiv -2\alpha_g\int d^4x\, (-g)^{1/2}\left[R_{\mu\kappa}R^{\mu\kappa}-\frac{1}{3} (R^{\alpha}_{\phantom{\alpha}\alpha})^2\right],
\label{AP1}
\end{eqnarray}
% 
where $\alpha_g$ is a dimensionless  gravitational coupling constant, and
%
\begin{eqnarray}
C_{\lambda\mu\nu\kappa}&=& R_{\lambda\mu\nu\kappa}
-\frac{1}{2}\left(g_{\lambda\nu}R_{\mu\kappa}-
g_{\lambda\kappa}R_{\mu\nu}-
g_{\mu\nu}R_{\lambda\kappa}+
g_{\mu\kappa}R_{\lambda\nu}\right)
\nonumber\\
&&+\frac{1}{6}R^{\alpha}_{\phantom{\alpha}\alpha}\left(
g_{\lambda\nu}g_{\mu\kappa}-
g_{\lambda\kappa}g_{\mu\nu}\right)
\label{AP2}
\end{eqnarray}
% 
is the conformal Weyl tensor \cite{bach_1921}. Under conformal transformation $g_{\mu\nu}(x) \to e^{2\alpha(x)}g_{\mu\nu}(x)$, the Weyl tensor transforms as $C^\lambda{}_{\mu\nu\kappa} \to C^\lambda{}_{\mu\nu\kappa}$. As the tracless component of the Riemann tensor, $C_{\lambda\mu\nu\kappa}$ vanishes in geometries that are conformal to flat. 

As described in \cite{mannheim_2012}, conformal invariance requires that there be no intrinsic mass scales at the level of the Lagrangian; rather, mass scales must come from the vacuum via spontaneous symmetry breaking. In such a process, particles may localize and bind into inhomogeneities comprising astrophysical objects of interest (e.g. stars and galaxies). With inhomogeneities violating the conformal symmetry in the spacetime geometry, the transition from a cosmological background geometry to the cosmological fluctuations associated with inhomogeneities corresponds to a shift from conformal to flat geometries to non-conformal flat geometries. However, when decomposed into a background and first order contribution, the underlying conformal symmetry contained within the background allows one to tame the complexity of the fluctuations due to the transformation properties of the Weyl tensor.

Variation of the Weyl action (\ref{AP1}) with respect to the metric $g_{\mu\nu}(x)$ yields the fourth-order derivative gravitational equations of motion \cite{mannheim_2006} \cite{mannheim_1998}
%
\begin{eqnarray}
-\frac{2}{(-g)^{1/2}}\frac{\delta I_{\rm W}}{\delta g_{\mu\nu}}&=&4\alpha_g W^{\mu\nu}=4\alpha_g\left[2\nabla_{\kappa}\nabla_{\lambda}C^{\mu\lambda\nu\kappa}-
R_{\kappa\lambda}C^{\mu\lambda\nu\kappa}\right]
\nonumber\\
&=&4\alpha_g\left[W^{\mu
	\nu}_{(2)}-\frac{1}{3}W^{\mu\nu}_{(1)}\right]=T^{\mu\nu},
\label{AP3}
\end{eqnarray}
% 
where tensors $W^{\mu \nu}_{(1)}$ and $W^{\mu \nu}_{(2)}$ are given by
%                                                                               
\begin{eqnarray}
W^{\mu \nu}_{(1)}&=&
2g^{\mu\nu}\nabla_{\beta}\nabla^{\beta}R^{\alpha}_{\phantom{\alpha}\alpha}                                             
-2\nabla^{\nu}\nabla^{\mu}R^{\alpha}_{\phantom{\alpha}\alpha}                          
-2 R^{\alpha}_{\phantom{\alpha}\alpha}R^{\mu\nu}                              
+\frac{1}{2}g^{\mu\nu}(R^{\alpha}_{\phantom{\alpha}\alpha})^2,
\nonumber\\
W^{\mu \nu}_{(2)}&=&
\frac{1}{2}g^{\mu\nu}\nabla_{\beta}\nabla^{\beta}R^{\alpha}_{\phantom{\alpha}\alpha}
+\nabla_{\beta}\nabla^{\beta}R^{\mu\nu}                    
-\nabla_{\beta}\nabla^{\nu}R^{\mu\beta}                       
-\nabla_{\beta}\nabla^{\mu}R^{\nu \beta}  
\nonumber\\            
&-& 2R^{\mu\beta}R^{\nu}_{\phantom{\nu}\beta}                                    
+\frac{1}{2}g^{\mu\nu}R_{\alpha\beta}R^{\alpha\beta}.
\label{AP4}
\end{eqnarray}     
Here $T^{\mu\nu}$ is the conformal invariant matter source energy-momentum tensor. With $I_{\rm W}$ being both general coordinate invariant and conformal invariant, $W^{\mu\nu}$ is automatically covariantly conserved and traceless and obeys $\nabla_{\nu}W^{\mu\nu}=0$, $g_{\mu\nu}W^{\mu\nu}=0$ (i.e. without needing to impose any equation of motion or stationarity, thus holding on every variational path).                            

Upon first glance, the fourth order \eqref{AP4} takes a considerably more complex form in comparison to the relatively terse second order Einstein equations
%
\begin{eqnarray}
-\frac{1}{8\pi G}\left(R^{\mu\nu} -\frac{1}{2}g^{\mu\nu}R^{\alpha}_{\phantom{\alpha}\alpha}\right)=T^{\mu\nu}.
\label{AP5}
\end{eqnarray}
%
However, in solving the vacuum equations associated with \eqref{AP4}, two types of solutions arise: those associated with a vanishing Weyl tensor and those associated with a vanishing Ricci tensor. As a consequence of the former, since all cosmological relevant geometries of interest can be expressed in the conformal to flat form, 
%
\begin{eqnarray}
ds^2=-\Omega^2(t,x,y,z)\eta_{\mu\nu}x^{\mu}x^{\nu}=\Omega^2(t,x,y,z)[dt^2-dx^2-dy^2-dz^2],
\label{AP6}
\end{eqnarray}
%
for appropriate choices of the conformal factor $\Omega(t,x,y,z)$ they serve as vacuum solutions. Regarding the latter, solutions with vanishing Ricci tensor necessarily encompass all vacuum solutions to Einstein gravity. In this sense, the set of solutions to the vacuum equations in conformal gravity forms a superset of all such vacuum equations in Einstein gravity. As a particular example, both gravitational theories admit the Schwarzschild solution exterior to a static, spherically symmetric source \cite{mannheim_kazanas_1988}, with the Schwarzschild  solution geometry not expressible in conformal to flat form. 


%%%%%%%%%%%%%%%%%%%%%%%%%%%%%%%%%%%%%%%%%%%%
\subsection{Conformal Invariance}
\label{ss:conformal_invariance}
%%%%%%%%%%%%%%%%%%%%%%%%%%%%%%%%%%%%%%%%%%%%


With the Weyl action \eqref{AP1} being locally conformal invariant, $W^{\mu\nu}(x)$ possesses the transformation property that upon a conformal transformation of the form
%
\begin{eqnarray}
g_{\mu\nu}(x)\rightarrow \Omega^2(x) g_{\mu\nu}(x)=\bar{g}_{\mu\nu}(x),\qquad
g^{\mu\nu}(x)\rightarrow \Omega^{-2}(x) g^{\mu\nu}(x)=\bar{g}^{\mu\nu}(x),
\label{AP7}
\end{eqnarray}
% 
$W^{\mu\nu}(x)$ and $W_{\mu\nu}(x)$ transform as 
%
\begin{eqnarray}
W^{\mu\nu}(x)\rightarrow \Omega^{-6}(x) W^{\mu\nu}(x)=\bar{W}^{\mu\nu}(x),
\nonumber\\
W_{\mu\nu}(x)\rightarrow \Omega^{-2}(x) W_{\mu\nu}(x)=\bar{W}_{\mu\nu}(x),
\label{AP8}
\end{eqnarray}
%
where the functional dependence of $\bar{W}_{\mu\nu}(x)$ on $\bar{g}_{\mu\nu}(x)$ is equivalent to that of
$W_{\mu\nu}(x)$ on $g_{\mu\nu}(x)$. To be noted is that (\ref{AP8}) holds regardless of whether or not the metric $g_{\mu\nu}(x)$ is conformal to flat. If we further decompose $g_{\mu\nu}(x)$ and $\bar{g}_{\mu\nu}(x)$ into a background metric and a first order perturbation according to
%
\begin{eqnarray}
ds^2&=&-[g^{(0)}_{\mu\nu}+h_{\mu\nu}]dx^{\mu}dx^{\nu},\qquad g_{\mu\nu}(x)=g^{(0)}_{\mu\nu}(x)+h_{\mu\nu}(x),
\nonumber\\ 
g^{\mu\nu}(x)&=&g_{(0)}^{\mu\nu}(x)-h^{\mu\nu}(x),\qquad
\bar{g}_{\mu\nu}(x)=\bar{g}^{(0)}_{\mu\nu}(x)+\bar{h}_{\mu\nu}(x),
\nonumber\\
\bar{g}^{\mu\nu}(x)&=&\bar{g}_{(0)}^{\mu\nu}(x)-\bar{h}^{\mu\nu}(x),
\label{AP9}
\end{eqnarray}
% 
then $W_{\mu\nu}(x)$ and $\bar{W}_{\mu\nu}(x)$ will decompose as 
%
\begin{eqnarray}
W_{\mu\nu}(g_{\mu\nu})&=& W^{(0)}_{\mu\nu}(g^{(0)}_{\mu\nu})+\delta W_{\mu\nu}(h_{\mu\nu}),
\nonumber\\
\bar{W}_{\mu\nu}(\bar{g}_{\mu\nu})&=&\bar{W}^{(0)}_{\mu\nu}(\bar{g}^{(0)}_{\mu\nu})+\delta \bar{W}_{\mu\nu}(\bar{h}_{\mu\nu}).
\label{AP10}
\end{eqnarray}
%
To clarify, within \eqref{AP10} $W_{\mu\nu}(h_{\mu\nu})$ is evaluated in a background geometry with metric $g^{(0)}_{\mu\nu}(x)$, whereas $\bar{W}_{\mu\nu}(\bar{h}_{\mu\nu})$ is evaluated in a background geometry with metric $\bar{g}^{(0)}_{\mu\nu}(x)$. 

Since the gravitational sector $W_{\mu\nu}$ is conformal invariant, the matter sector $T_{\mu\nu}$ must necessarily also transform as $\Omega^{-2}(x) T_{\mu\nu}(x)=\bar{T}_{\mu\nu}(x)$. Repeating a decomposition into background and first order components, we obtain for the energy momentum tensor
%
\begin{eqnarray}
T_{\mu\nu}(g_{\mu\nu})= T^{(0)}_{\mu\nu}(g^{(0)}_{\mu\nu})+\delta T_{\mu\nu}(h_{\mu\nu}),\qquad
\bar{T}_{\mu\nu}(\bar{g}_{\mu\nu})=\bar{T}^{(0)}_{\mu\nu}(\bar{g}^{(0)}_{\mu\nu})+\delta \bar{T}_{\mu\nu}(\bar{h}_{\mu\nu}).
\label{AP11}
\end{eqnarray}
%
The utility of the conformal transformation properties described allow us to find solutions around conformally transformed geometries using only knowledge of the form of the solution around the original geometry. Specifically, if we know how to solve for fluctuations $h_{\mu\nu}(x)$ around a background $g^{(0)}_{\mu\nu}(x)$, (that is, if $g^{(0)}_{\mu\nu}(x)$ is such a geometry that solutions to $\delta W_{\mu\nu}(h_{\mu\nu})=\delta T_{\mu\nu}(h_{\mu\nu})/4\alpha_g$ may be found apriori) then we can find obtain solutions to  $\delta \bar{W}_{\mu\nu}(\bar{h}_{\mu\nu})=\delta \bar{T}_{\mu\nu}(\bar{h}_{\mu\nu})/4\alpha_g$  for fluctuations $\bar{h}_{\mu\nu}(x)$ around a background metric $\bar{g}^{(0)}_{\mu\nu}(x)$ by defining 
%
\begin{eqnarray}
\bar{h}_{\mu\nu}(x)=\Omega^2(x)h_{\mu\nu}(x),\qquad \delta \bar{W}_{\mu\nu}(\bar{h}_{\mu\nu})=\Omega^{-2}(x)\delta W_{\mu\nu}(h_{\mu\nu}).
\label{AP12}
\end{eqnarray}
%
Implementing conformal gravity solutions found in past work (e.g. \cite{mannheim_2006}), one can use the determined structure of the fluctuations around a flat background to construct the fluctuations around any background that is conformal to flat by virtue of \eqref{AP12}. As mentioned, since all cosmologically relevant background geometries can be cast into the conformal to flat form, the conformal transformation properties give rise to an extremely convenient and powerful methodology to solving that fluctuation equations, despite the fourth-order nature and expansive form of the gravitational equations of motion.

We can continue to use conformal invariance (i.e. under a conformal transformation of general metric $g_{\mu\nu}\rightarrow \Omega^2(x)g_{\mu\nu}$ the Bach tensor  $W_{\mu\nu}$ transforms as $W_{\mu\nu}\rightarrow \Omega^{-2}(x)W_{\mu\nu}$) to determine the trace depedendent properties of $W_{\mu\nu}$. Taking $h$ as a first order perturbation in the metric and using the conformal transformation properties, we find up to first order 
\begin{align}
W_{\mu\nu}\left(g^{(0)}_{\mu\nu} + \frac h4 g^{(0)}_{\mu\nu}\right) &=W_{\mu\nu}\left[\left(1+\frac h4\right)g^{(0)}_{\mu\nu} \right]= W_{\mu\nu}^{(0)}(g^{(0)}_{\mu\nu}) +\delta W_{\mu\nu}\left(\frac h4g^{(0)}_{\mu\nu}\right)\nonumber \\
&=\left(1-\frac h4\right)W_{\mu\nu}(g^{(0)}_{\mu\nu}) \nonumber,
\end{align}
and hence
\begin{eqnarray}
-\frac h4 W_{\mu\nu}(g_{\mu\nu}^{(0)}) = \delta W_{\mu\nu}\left(\frac h4 g^{(0)}_{\mu\nu}\right)\label{wtrace1},
\end{eqnarray}
or, in its full form
\begin{align}
\delta W_{\mu\nu}(\tfrac{h}{4}g^{(0)}_{\mu\nu}) &= - \tfrac{1}{4} h (- \tfrac{1}{6} g_{\mu \nu} R^2 + \tfrac{1}{2} g_{\mu \nu} R_{\alpha \beta} R^{\alpha \beta} + \tfrac{2}{3} R R_{\mu \nu} - 2 R^{\alpha \beta} R_{\mu \alpha \nu \beta}  \nonumber\\
&\qquad- \tfrac{1}{6} g_{\mu \nu} \nabla_{\alpha}\nabla^{\alpha}R + \nabla_{\alpha}\nabla^{\alpha}R_{\mu \nu} -  \nabla_{\mu}\nabla^{\alpha}R_{\nu \alpha} 
\nonumber\\
&\qquad-  \nabla_{\nu}\nabla^{\alpha}R_{\mu \alpha} + \tfrac{2}{3} \nabla_{\nu}\nabla_{\mu}R)  
\nonumber\\
&= -\tfrac{h}{4}W_{\mu\nu}(g^{(0)}_{\mu\nu}). 
\label{dwtrace1}
\end{align}
To take make full use of the dependence of $\delta W_{\mu\nu}$ on $h$ we introduce the quantity $K_{\mu\nu}(x)$ defined as 
%
\begin{eqnarray}
K_{\mu\nu}(x)=h_{\mu\nu}(x)-\frac{1}{4}g^{(0)}_{\mu\nu}(x)g_{(0)}^{\alpha\beta}h_{\alpha\beta},
\label{AP13}
\end{eqnarray}
%
with $K_{\mu\nu}$ being traceless with respect to the background metric $g_{(0)}^{\mu\nu}$.
Substituting \eqref{AP13} into $\delta W_{\mu\nu}(h_{\mu\nu})$ we obtain
\begin{eqnarray}
\delta W_{\mu\nu}(h_{\mu\nu}) = \delta W_{\mu\nu}\left(K_{\mu\nu}+\frac h4g^{(0)}_{\mu\nu}\right)= \delta W_{\mu\nu}(K_{\mu\nu}) +\delta W_{\mu\nu}\left(\frac h4g^{(0)}_{\mu\nu}\right)\label{wtrace2}.
\end{eqnarray}
If we work in a background that is conformal to flat, then \eqref{wtrace1} will vanish which implies from \eqref{wtrace2} that
\begin{eqnarray}
\delta W_{\mu\nu}(h_{\mu\nu}) = \delta W_{\mu\nu}(K_{\mu\nu}).
\end{eqnarray}
Hence in a conformal to flat geometry, the trace of $h_{\mu\nu}$ not only decouples but also vanishes, with the fluctuation equations being able to be entirely expressed in terms of the nine component $K_{\mu\nu}$.

We may also find a relationship between $h_{\mu\nu}$ and the trace of the fluctuation $\delta W_{\mu\nu}$ itself. First, we note that the tracelessness of $W_{\mu\nu}$ implies
\begin{eqnarray}
g^{\mu\nu}W_{\mu\nu}(g_{\mu\nu}) = \left({ g^{(0)\mu\nu}-h^{\mu\nu}}\right)\left({ W^{(0)}_{\mu\nu}+ \delta W_{\mu\nu}}\right)=0.
\end{eqnarray}
To first order this equates to,
\begin{eqnarray}
-h^{\mu\nu}W^{(0)}_{\mu\nu} + g^{(0)\mu\nu}\delta W_{\mu\nu} = 0
\end{eqnarray}
and thus we obtain
\begin{eqnarray}
g^{(0)\mu\nu}\delta W_{\mu\nu}(h_{\mu\nu}) = h^{\mu\nu}W_{\mu\nu}(g^{(0)}_{\mu\nu}).
\end{eqnarray}
For reference, we state the full form of the above as
\begin{align}
g^{(0)}{}^{\mu\nu}\delta W_{\mu\nu} &= h^{\mu \nu} (- \tfrac{1}{6} g_{\mu \nu} R^2 + \tfrac{1}{2} g_{\mu \nu} R_{\alpha \beta} R^{\alpha \beta} + \tfrac{2}{3} R R_{\mu \nu} - 2 R^{\alpha \beta} R_{\mu \alpha \nu \beta}
\nonumber\\
&\qquad -  \tfrac{1}{6} g_{\mu \nu} \nabla_{\alpha}\nabla^{\alpha}R  +\nabla_{\alpha}\nabla^{\alpha}R_{\mu \nu} -  \nabla_{\mu}\nabla^{\alpha}R_{\nu \alpha} -  \nabla_{\nu}\nabla^{\alpha}R_{\mu \alpha} 
\nonumber\\
&\qquad + \tfrac{2}{3} \nabla_{\nu}\nabla_{\mu}R) \nonumber\\
&=h^{\mu\nu}W_{\mu\nu}(g^{(0)}_{\mu\nu})
\end{align}
Consequently, in a conformal to flat background, the trace of the fluctuation tensor itself will vanish. 

Owing to this decoupling of the trace, for conformal to flat backgrounds one may substitute the usage of (\ref{AP10}) instead by the usage of
%
\begin{eqnarray}
W_{\mu\nu}(g_{\mu\nu})&=& W^{(0)}_{\mu\nu}(g^{(0)}_{\mu\nu})+\delta W_{\mu\nu}(K_{\mu\nu}),
\nonumber\\
\bar{W}_{\mu\nu}(\bar{g}_{\mu\nu})&=&\bar{W}^{(0)}_{\mu\nu}(\bar{g}^{(0)}_{\mu\nu})+\delta\bar{W}_{\mu\nu}(\bar{K}_{\mu\nu}),
\label{AP14}
\end{eqnarray}
%
where
%
\begin{eqnarray}
\bar{g}^{(0)}_{\mu\nu}(x)=\Omega^2(x)g^{(0)}_{\mu\nu}(x),
\label{AP15}
\end{eqnarray}
% 
%
\begin{eqnarray}
\bar{K}_{\mu\nu}(x)=\Omega^2(x)K_{\mu\nu}(x).
\label{AP16}
\end{eqnarray}
% 
Consequently, in the context of conformal gravity, when constructing fluctuations in a $\bar{g}^{(0)}_{\mu\nu}$ background from the fluctuations in a $g^{(0)}_{\mu\nu}$ background that is conformal to flat, we here on utilize (\ref{AP16}) rather than (\ref{AP12}). 

Summarizing the conformal properties of conformal gravity, we have shown that for fluctuations around a conformal to flat background, we can reduce the equations to a dependence on the traceless $K_{\mu\nu}$ without needing to make any reference to the actual detailed form of the fluctuation equations at all. Given that one also possesses the freedom to make four general coordinate transformations, one can further reduce the nine-component $K_{\mu\nu}$ to five independent components, all without needing to make any reference to the fluctuation equations. Any further reduction in the number of independent components of $K_{\mu\nu}$ can only be achieved through use of residual gauge invariances or the structure of the fluctuation equations themselves. Within Ch. \ref{c:constructing_gauge_conditions} we make use of the coordinate freedom and implement a particular gauge condition (motivated by its conformal transformation properties) that yields fluctuation equations in which there is no mixing of any of the components of $K_{\mu\nu}$ with each other.


%%%%%%%%%%%%%%%%%%%%%%%%%%%%%%%%%%%%%%%%%%%%
\subsection{Fluctuations Around Flat in the Transverse Gauge}
\label{ss:fluctuations_around_flat_in_the_tranverse_gauge}
%%%%%%%%%%%%%%%%%%%%%%%%%%%%%%%%%%%%%%%%%%%%
To illustrate the use of gauge conditions within conformal gravity, we investiage fluctuations around a Minkowski background. In such a background it was found in \cite{mannheim_2006}, that $\delta W_{\mu\nu}$ takes the form, prior to imposing any gauge conditions
%
\begin{eqnarray}
\delta W_{\mu\nu}&=&\frac{1}{2}(\eta^{\rho}_{\phantom{\rho} \mu} \partial^{\alpha}\partial_{\alpha}-\partial^{\rho}\partial_{\mu})
(\eta^{\sigma}_{\phantom{\sigma} \nu} \partial^{\beta}\partial_{\beta}-
\partial^{\sigma}\partial_{\nu})K_{\rho \sigma}
\nonumber\\
&-& 
\frac{1}{6}(\eta_{\mu \nu} \partial^{\gamma}\partial_{\gamma}-
\partial_{\mu}\partial_{\nu})(\eta^{\rho \sigma} \partial^{\delta}\partial_{\delta}-
\partial^{\rho}\partial^{\sigma})K_{\rho\sigma}.
\label{AP24}
\end{eqnarray}
%
Within a flat background, the Lie derivative of $K^{\mu\nu}$ leads to $\partial_{\nu}K^{\mu\nu}\rightarrow \partial_{\nu}K^{\mu\nu}-\partial_{\nu}\partial^{\nu}\epsilon^{\mu}-\partial^{\mu}\partial_{\nu}\epsilon^{\nu}/2$ and $\partial_{\mu}\partial_{\nu}K^{\mu\nu}\rightarrow \partial_{\mu}\partial_{\nu}K^{\mu\nu}-3\partial_{\mu}\partial^{\mu}\partial_{\nu}\epsilon^{\nu}/2$. Hence, in order to construct a gauge condition $\partial_{\nu}K^{\mu\nu} = f^\mu$ for arbitrary $f^\mu$, we can solve for $\partial_{\nu}\epsilon^{\nu}$ and then for $\epsilon^{\mu}$ in order to set $\partial_{\nu}K^{\mu\nu}=f^\mu$. If we elect to select an $f^\mu$ such that $\partial_{\mu}K^{\mu\nu}=0$ (i.e. the transverse gauge condition), then (\ref{AP24}) reduces to the remarkably compact and simple form
%
\begin{eqnarray}
\delta W_{\mu\nu}=\frac{1}{2}\eta^{\sigma\rho}\eta^{\alpha\beta}\partial_{\sigma}\partial_{\rho} \partial_{\alpha}\partial_{\beta}K_{\mu \nu}.
\label{AP25}
\end{eqnarray}
%
Note that the tensor components of $K_{\mu\nu}$ that were coupled in (\ref{AP24}) are now completely decoupled in (\ref{AP25}). (This may be constrasted with conformal to flat fluctuations in Einstein gravity where, as we as far as we aware, there is no gauge in which such a complete decoupling occurs. In Sec. \ref{s:compact_expressions_ein} we present a selection of gauges that yield maximal simplification and decoupling, with it being evident that a complete decoupling is prevented only by the presence of the trace $h$ of the metric fluctuation).

To solve $4\alpha_g \delta W_{\mu\nu} = \delta T_{\mu\nu}$ with its associated \eqref{AP25}, we define the fourth-order derivative Green's function which obeys
%
\begin{eqnarray}
\partial_{\alpha}\partial^{\alpha} \partial_{\beta}\partial^{\beta}D(x-x^{\prime})=\delta^4(x-x^{\prime}),
\label{AP26}
\end{eqnarray}
%
in which the solution (in the transverse gauge) is given by 
%
\begin{eqnarray}
K_{\mu\nu}(x)=\frac{1}{2\alpha_g}\int d^4x^{\prime}D(x-x^{\prime})\delta T_{\mu\nu}(x^{\prime}).
\label{AP27}
\end{eqnarray}
%

The retarded Green's function  solution to (\ref{AP26}) \cite{mannheim_2007} is given by
%
\begin{eqnarray}
D^{(FO)}(x-x^{\prime})=\frac{1}{8\pi}\theta (t-t^{\prime}-|{\bf x}-{\bf x}^{\prime}|),
\label{AP28}
\end{eqnarray}
%
with $\theta (t-t^{\prime}-|{\bf x}-{\bf x}^{\prime}|)$ vanishing outside the light cone. 

The solutions to the fourth order wave equation $\partial_{\alpha}\partial^{\alpha} \partial_{\beta}\partial^{\beta}K_{\mu \nu}=0$ may be solved in terms of momentum eiginstates, given by \cite{riegert_1984a,mannheim_2012}
%
\begin{eqnarray}
K_{\mu\nu}=A_{\mu\nu}e^{ik\cdot x}+(n\cdot x)B_{\mu\nu}e^{ik\cdot x}+A^*_{\mu\nu}e^{-ik\cdot x}+(n\cdot x)B^*_{\mu\nu}e^{-ik\cdot x},
\label{AP29}
\end{eqnarray}
%
where $k_0^2={\bf k}^2$, where $A_{\mu\nu}$ and $B_{\mu\nu}$ are polarization tensors, and where $n^{\mu}=(1,0,0,0)$ is a unit timelike vector.  With $n\cdot x=t$, we see that fluctuations around a flat background grow linearly in time. In total, given $\delta T_{\mu\nu}$, (\ref{AP27}) can be solved completely, and for a localized $\delta T_{\mu\nu}$ the asymptotic solution for $K_{\mu\nu}$ is given by (\ref{AP29}). (In Sec. \ref{s:rw_radiation_conformal_gravity_sol}, we analogously construct the eigenstate solutions to the wave equation within a curved Robertson Walker radiation era ($k=-1$) background to find solutions with leading time behavior of $t^4$.) 
%
%%%%%%%%%%%%%%%%%%%%%%%%%%%%%%%%%%%%%%%%%%%%
\subsection{On the Energy Momentum Tensor}
\label{ss:on_the_energy_momentum_tensor}
%%%%%%%%%%%%%%%%%%%%%%%%%%%%%%%%%%%%%%%%%%%%

The matter field $T^{\mu\nu}$ in conformal gravity is behaves in a different nature in comparison to standard Einstein gravity. The root of the difference of the energy momentum tensor between the two theories stems from the statement that $4\alpha_gW^{\mu\nu}=T^{\mu\nu}$ must be conformally invariant, from which it follows that $T^{\mu\nu}$ must transform in the same manner under conformal transformations as $W^{\mu\nu}$. Consequently, in conformal to flat cosmological backgrounds where $W^{\mu\nu}$ vanishes, $T^{\mu\nu}$ must also vanish. However, in the literature two ways in which it could vanish non-trivially have been identified, one involving a conformally coupled scalar field \cite{mannheim_1990}, and the other involving a conformal perfect fluid \cite{mannheim_2000}.

In the case of a conformally coupled scalar field $S(x)$ we define the matter action as
%                                                                               
\begin{eqnarray}
I_S&=&-\int d^4x(-g)^{1/2}\left[\frac{1}{2}\nabla_{\mu}S
\nabla^{\mu}S-\frac{1}{12}S^2R^\mu_{\phantom         
	{\mu}\mu}+\lambda_S S^4\right]
\label{AP32}
\end{eqnarray}                                 
% 
where  $\lambda_S$ is a dimensionless coupling constant. As written, the $I_{\rm S}$ action is the most general curved space matter action for the $S(x)$ field that is invariant under both general coordinate transformations and local conformal transformations of the form
$S(x)\rightarrow e^{-\alpha(x)}S(x)$,  $g_{\mu\nu}(x)\rightarrow e^{2\alpha(x)}g_{\mu\nu}(x)$ \cite{mannheim_1990}. Variation of $I_S$ with respect to  $S(x)$ yields the scalar field equation of motion
%                                                                               
\begin{eqnarray}
\nabla_{\mu}\nabla^{\mu}S+\frac{1}{6}SR^\mu_{\phantom{\mu}\mu}
-4\lambda_S S^3=0,
\label{AP33}
\end{eqnarray}                                 
%                                                               
while variation with respect to the metric yields the matter field energy-momentum tensor 
%                                                                               
\begin{eqnarray}
T_{\rm S}^{\mu \nu}&=&\frac{2}{3}\nabla^{\mu} \nabla^{\nu} S
-\frac{1}{6}g^{\mu\nu}\nabla_{\alpha}S\nabla^{\alpha}S
-\frac{1}{3}S\nabla^{\mu}\nabla^{\nu}S
\nonumber \\             
&+&\frac{1}{3}g^{\mu\nu}S\nabla_{\alpha}\nabla^{\alpha}S                           
-\frac{1}{6}S^2\left(R^{\mu\nu}
-\frac{1}{2}g^{\mu\nu}R^\alpha_{\phantom{\alpha}\alpha}\right)-g^{\mu\nu}\lambda_S S^4. 
\label{AP34}
\end{eqnarray}                                 
% 
By using \eqref{AP33} within \eqref{AP34}, the energy-momentum tensor obeys the expected traceless condition
\begin{eqnarray}
g_{\mu\nu}T_{\rm S}^{\mu\nu}=0.
\end{eqnarray}
%
If we take the scalar field as the spontaneously broken non-zero constant expectation value $S_0$, the scalar field wave equation and the energy-momentum tensor reduce to the form
%                                                                               
\begin{eqnarray}
R^\alpha_{\phantom{\alpha}\alpha}&=&24\lambda_S S_0^2,
\nonumber\\
T_{\rm S}^{\mu \nu}&=& 
-\frac{1}{6} S_0^2\left(R^{\mu\nu}-\frac{1}{2}g^{\mu\nu}
R^\alpha_{\phantom{\alpha}\alpha}\right)-g^{\mu\nu}\lambda_S S_0^4
\nonumber\\
&=&-\frac{1}{6} S_0^2\left(R^{\mu\nu}-\frac{1}{4}g^{\mu\nu}
R^\alpha_{\phantom{\alpha}\alpha}\right).
\label{AP35}
\end{eqnarray}                                 
%  
Now, in a de Sitter ($dS_4$) geometry defined by
\begin{eqnarray}
R^{\lambda\mu\sigma\nu}&=&K[g^{\mu \sigma}g^{\lambda \nu}-g^{\mu \nu}g^{\lambda \sigma}],\qquad R^{\mu\nu}=-3Kg^{\mu\nu}
\nonumber\\
R^\alpha_{\phantom{\alpha}\alpha}&=&-12K,\qquad R^{\mu\nu}=(1/4)g^{\mu\nu}
R^\alpha_{\phantom{\alpha}\alpha},
\end{eqnarray}
since $W^{\mu \nu}$ vanishes identically, $T_{\rm S}^{\mu \nu}$ will also  vanish identically in the same geometry. And with curvature constant $K$ being taken as $K=-2\lambda_S S_0^2$ we find that though $W^{\mu\nu}$ and $T^{\mu\nu}$ both vanish identically, as noted in \cite{mannheim_1990}, the conformal cosmology governed by $4\alpha_gW^{\mu\nu}=T^{\mu\nu}$ admits of a non-trivial de Sitter geometry solution, with a non-vanishing four-curvature $K=-2\lambda_S S_0^2$.

To discuss another avenue in which $T^{\mu\nu}$ can vanish non-trivially \cite{mannheim_2000}, we set $\lambda_S=0$ within $I_S$ (an operation which still preserves the conformal invariance of $I_S$), and we evaluate the scalar field wave equation in a generic Robertson-Walker geometry defined as
%
\begin{eqnarray}
ds^2=dt^2-a^2(t)\left[\frac{dr^2}{1-kr^2}+r^2d\theta^2+r^2\sin^2\theta d\phi^2\right]
=dt^2-a^2(t)\gamma_{ij}dx^idx^j.
\label{AP36}
\end{eqnarray}
%
As discussed in \cite{mannheim_2000}, solutions to the scalar field wave equation (\ref{AP33}) within the Roberston Walker geometry obey
%
\begin{eqnarray}
\frac{1}{f(p)}\left[\frac{d^2f}{dp^2}+kf(p)\right]=\frac{1}{g(r,\theta,\phi)}\gamma^{-1/2}\partial_i[\gamma^{1/2}\gamma^{ij}\partial_jg(r,\theta,\phi)]=-\lambda^2,
\label{AP37}
\end{eqnarray}
%
where $p=\int dt/a(t)$, $S=f(p)g(r,\theta,\phi)/a(t)$, $\gamma^{ij}$ is the metric of the spatial part of the Robertson-Walker metric, and $\lambda^2$ is a separation constant. Inspection of (\ref{AP37}) reveals that $f(p)$ obeys a harmonic oscillator equation with characteristic frequencies $\omega^2=\lambda^2+k$. In addition, we may look for separable solutions in $g(r,\theta,\phi)$ viz.
\begin{eqnarray}
g(r,\theta,\phi)=g^{\ell}_{\lambda}(r)Y^m_{\ell}(\theta,\phi),
\end{eqnarray}
with $g^{\ell}_{\lambda}(r)$ necessarily obeying
%
\begin{eqnarray}
\left[(1-kr^2)\frac{\partial^2}{\partial r^2}+\frac{(2-3kr^2)}{r}\frac{\partial}{\partial r}-\frac{\ell(\ell+1)}{r^2}+\lambda^2\right]g^{\ell}_{\lambda}(r)=0.
\label{AP38}
\end{eqnarray}
%
From here, we proceed with an interesting step and perform an incoherent averaging over all allowed spatial modes associated with a given $\omega$. Upon calculating the sum over all modes, for each $\omega$ we obtain  \cite{mannheim_2000}  
%
\begin{eqnarray}
T_S^{\mu\nu}=\frac{\omega^2(g^{\mu\nu}+4U^{\mu}U^{\nu})}{6\pi^2a^4(t)}=
\frac{(\lambda^2+k^2)(g^{\mu\nu}+4U^{\mu}U^{\nu})}{6\pi^2a^4(t)},
\label{AP39}
\end{eqnarray}
%
where $U^{\mu}$ is a unit timelike vector. With \eqref{AP39} being traceless, the incoherent averaging over the spatial modes has yielded an energy momentum tensor of the perfect fluid form, namely
\begin{eqnarray}
T_S^{\mu\nu} = (\rho+p)U^\mu U^\nu + p g^{\mu\nu},\qquad \rho = 3p,
\end{eqnarray}
for appropriate values of $\rho$ and $p$. Inspecting \eqref{AP39}, we see that if $\omega^2=0$, $T^{\mu\nu}_S=0$ and if $\omega^2=\lambda^2+k$, we can satisfy $T^{\mu\nu}_S=0$ non-trivially if and only if $k$ is negative. Thus, we proceed with $k$ negative. In performing an incoherent averaging for $T^{00}_S$ (recalling that we are taking $\omega =0$ here), we obtain \cite{mannheim_2000}
%
\begin{eqnarray}
T_S^{00}=\frac{1}{6}\sum_{\ell,m}\left[\sum _{i=1}^3\gamma^{ii}|\partial_i(g^{\ell}_{(-k)^{1/2}}Y^{m}_{\ell}(\theta,\phi))|^2+k|g^{\ell}_{(-k)^{1/2}}Y^{m}_{\ell}(\theta,\phi)|^2\right].
\label{AP40}
\end{eqnarray}
%
It has been shown in \cite{mannheim_2000} that the sum in (\ref{AP40}) in fact vanishes identically. With scalar field modes providing a positive contribution to $T^{\mu\nu}_S$, the negative contributions of the gravitational field from its negative spatial curvature serve to cancel the scalar modes identically, resulting in a vanishing $T^{00}_S$.
As regards the solutions to (\ref{AP38}), with negative $k$ these are determined to be associated Legendre functions. Despite $T^{\mu\nu}_S$ vanishing non-trivially, (\ref{AP38}) still contains an infinite number of solutions, each labelled with a different $\ell$ and $m$. Hence, we shown that $T^{\mu\nu}_S$ admits of a non-trival vacuum solution that can be obtained by taking an incoherent average over the spatial modes associated with the solutions of the scalar field.

While the choice of negative $k$ may warrant concern in the standard treatment of gravitation and cosmology, where the universe geometry is phenomenologically taken as $k=0$, in conformal gravity it poses no such restriction as evidenced in past work \cite{mannheim_obrien_2012,mannheim_obrien_2011,obrien_mannheim_2012,mannheim_kazanas_1988,mannheim_kazanas_1989,mannheim_kazanas_1994}. In applications of conformal gravity to astrophysical and cosmological data it has been found that phenomenologically $k$ should be negative. Specifically, in previous works within conformal cosmology negative $k$ fits to the accelerating universe Hubble plot data have been presented in \cite{mannheim_2006,mannheim_2017}, along with negative $k$ conformal gravity fits to galactic rotation curves  presented in \cite{mannheim_2006,mannheim_2017}.

A last aspect worth mentioning in regards to the difference between the matter source in conformal and Einstein gravity concerns the interplay of gauge invariance. While a background $T^{\mu\nu}$ may vanish, this does not necessarily imply that its perturbation will also vanish (with its vanishing dependent upon the vanishing of $\delta W^{\mu\nu}$, which in all cosmological applications is necessarily non-zero). Now, in standard Einstein gravity with a non-zero background $T^{\mu\nu}$, neither the fluctuation in the background Einstein tensor or the fluctuation in the background $T^{\mu\nu}$ will separately be gauge invariant. It is only the perturbation of the entire $R^{\mu\nu} -g^{\mu\nu}R^{\alpha}_{\phantom{\alpha}\alpha}/2+8\pi GT^{\mu\nu}$ that is gauge invariant. Namely, one must impose the background equations of motion to the fluctuation equations to ensure gauge invariance. Moreover, there are no nontrivial background solutions to $G^{\mu\nu}_{(0)}=0$ - all solutions demand a vanishing curvature tensor. However, within conformal gravity, any background that is conformal to flat will cause the background fluctuations to vanish and we have identified two scenarios in which the $T^{\mu\nu}_S$ itself vanishes non-trivially. Consequently, the background equations of motion serve no role in enforcing gauge invariance within $4\alpha_g\delta W^{\mu\nu} = \delta T^{\mu\nu}$, and thus $\delta W^{\mu\nu}$ and $\delta T^{\mu\nu}$ are separately gauge invariant.  Specifically, we shall find that in any background that is conformal to flat, $\delta W^{\mu\nu}$ can be expressed entirely in terms of the gauge invariant components of the metric. Through the following chapters, we will illustrate the role of gauge invariance explicitly in both standard and conformal gravity using a Scalar, Vector, Tensor formulation as well as through the imposition of gauge conditions.

%%%%%%%%%%%%%%%%%%%%%%%%%%%%%%%%%%%%%%%%%%%%
\section{Cosmological Geometries}
\label{s:cosmological_geometries}
%%%%%%%%%%%%%%%%%%%%%%%%%%%%%%%%%%%%%%%%%%%%
The cosmological principle asserts that on a large enough scale, the structure of spacetime is homogeneous and isotropic. Allowing for expansion or contraction of the universe over time, the generic metric that satisfies these criteria is the Friedmann–Lemaître–Robertson–Walker (FLRW, commonly cited as RW) \cite{kodama_sasaki_1984} metric
\begin{eqnarray}
ds^2 = -dt^2 + a(t)^2\left[ \frac{dr^2}{1-kr^2} + r^2d\theta^2 + r^2\sin^2\theta d\theta^2\right]. 
\label{FRLW}
\end{eqnarray}
Here the scale factor $a(t)$ describes the expansion of space in the universe and $k \in \{-1,0,1\}$ describes the global geometry of the universe, being spatially hyperbolic, flat, or spherical respectively. 

In a universe dominated by a cosmological constant (as relevant to inflation), one may solve the Einstein equations $G_{\mu\nu} = -8\pi G \Lambda g_{\mu\nu}$ to determine the requisite metric. For $\Lambda > 0$, the solution is the deSitter geometry (and $\Lambda < 0$ the anti deSitter geometry), which describes a maximally symmetric space with curvature tensors of the form
\begin{eqnarray}
R_{\lambda\mu\nu\kappa} = \Lambda (g_{\lambda\nu}g_{\nu\kappa}-g_{\lambda\kappa}g_{\mu\nu}),
\qquad R_{\mu\kappa} = -3\Lambda g_{\mu\kappa},\qquad R=-12\Lambda.
\end{eqnarray} 
With deSitter space possessing higher symmetry than the most general FLRW space (i.e. more Killing vectors), it is in fact a special case of Roberston Walker as can be seen in the choice of coordinates 
\begin{eqnarray}
ds^2 = -dt^2 + e^{2\Lambda t} (dr^2 + r^2d\theta^2 + r^2\sin^2\theta d\phi^2),
\end{eqnarray}
which corresponds to $k=0$, $a(t) = e^{2\Lambda t}$ within \eqref{FRLW} and further discussed within Appendix \ref{abs:ds4} and Appendix \ref{abs:ds4_ads4_radiation}.

A remarkable aspect about the Roberston Walker geometry (and $dS_4$ or $AdS_4$ by extension) is that in each global geometry (hyperbolic, flat, spherical) the space can be written in conformal to flat form. As a simple example, if we take the $k=0$ (flat 3-space) metric of \eqref{FRLW} according to
\begin{eqnarray}
ds^2 = -dt^2 + a(t)^2\left[ dr^2 + r^2d\theta^2 + r^2\sin^2\theta d\theta^2\right],
\end{eqnarray}
then in transforming the time coordinate via $\tau = \int \frac{dt}{a(t)}$, the geometry may be written in the form
\begin{eqnarray}
ds^2 = a(\tau^2)( -d\tau^2 + dr^2 + r^2d\theta^2 + r^2\sin^2\theta d\phi^2). 
\end{eqnarray}
If we are interested in the $k=-1/L^2$ hyperbolic space, a proper choice of coordinates allows the Roberston Walker to be expressed as
\begin{eqnarray}
ds^2=\frac{4L^2 a^2(p',r')}{[1-(p'+r')^2][1-(p'-r')^2]} \left[ -dp'^2 + dr'^2+r'^2 d\theta^2 + r'^2 \sin^2\theta d\phi^2\right]
\end{eqnarray}
whereas for the $k=1/L^2$ spherical 3-space \eqref{FRLW} takes the form
\begin{eqnarray}
ds^2=\frac{4L^2 a^2(p',r')}{[1+(p'+r')^2][1+(p'-r')^2]} \left[ -dp'^2 + dr'^2+r'^2 d\theta^2 + r'^2 \sin^2\theta d\phi^2\right].
\end{eqnarray}
The coordinate transformations necessary to bring the co-moving Roberston Walker forms into the conformal to flat basis are given in detail within Appendix \ref{ab:cosmologies}, including the conformal factors relevant to the radation era.

As mentioned at the end of Sec. \ref{ss:on_the_energy_momentum_tensor}, since all the cosmological geometries of interest possess a coordinate expression where the space is conformal to flat, within such geometries the background Bach tensor vanishes $W^{(0)}_{\mu\nu} = 0$ to thus render $\delta W_{\mu\nu}$ to independently be a gauge invariant tensor, i.e. without reference to the equation of motion 
%

\chapter{Scalar, Vector, Tensor (SVT) Decomposition}
\label{c:scalar_vector_tensor_basis}

In the field of perturbative cosmology, it is standard to first introduce a 3+1 decomposition of the metric perturbation followed by a decomposition into  scalars, vectors, and tensors according to the underlying background 3-space. A la the SVT decomposition \cite{ellis_maartens_maccallum_2009, bertschinger_2000, bardeen_1980, mukhanov_1992}, referred to as SVT3 with this work. 

In this chapter, we first develop the SVT3 formalism by separately forming relations between SVT3 components and integrals over $h_{\mu\nu}$, as well as their inversions (i.e. higher derivative relations between $h_{\mu\nu}$ and SVT3 quantities.) In utilizing the SVT3 decomposition in conformal gravity, we also form the analogous relations between the traceless $K_{\mu\nu}$ and the requisite SVT3 components.

We then investigate the behavior of the SVT3 quantities under gauge transformations, and along with reference to the flat space gauge invariant Einstein tensor $\delta G_{\mu\nu}$, we construct the set of gauge invariant quantities. (In fact, such a gauge invariant construction lies behind the core utility of implementing the SVT3 decomposition in the first place). In forming the gauge invariants, we highlight their dependence upon underlying assumptions of being asymptotically well-behaved and we investigate the consequence of gauge invariance absent of any defined asymptotic behavior. 

To provide a manifestly covariant decomposition, we introduce the general SVTD decomposition for scalars, vectors, and tensors within arbitrary dimension $D$ \cite{phelps_2019}. Analgous to the SVT3 discussion, we provide integral and derivative relations between SVT4 components and $h_{\mu\nu}$ and find that the cosmological fluctuation equations take a considerably more compact and simple form in the new SVT4 formalism. Being simpler, these equations are thus more straightforward to solve. The gauge invariants are then constructed and the role of asymptotic behavior within the establishment of the SVT4 decomposition itself is analyzed. 

With help from the structure of the conformal gravity fluctuation $\delta W_{\mu\nu}$, we are able to determine the relationship between the SVT3 and SVT4 components, with particular emphasis on the gauge invariant quantities.

Finally, we introduce and analyze a core theorem within this work called the cosmological decomposition theorem. This theorem asserts that SVT3 scalars, vectors, and tensors decouple within the fluctuation equations \emph{themselves}, to thus evolve independently. We carefully inspect the dependence of the decomposition upon underlying constraints related to boundary conditions, determining that the theorem requires fluctuations to be well behaved asymptotically for such an equation decoupling to occur. The analysis is repeated with respect to the SVT4 decomposition, where the validity of the decomposition does not hold generally but rather must be determined on a case by case basis according the specific background geometry.
%%%%%%%%%%%%%%%%%%%%%%%%%%%%%%%%%%%%%
\section{SVT3}
\label{s:svt3}
%%%%%%%%%%%%%%%%%%%%%%%%%%%%%%%%%%%%%
%
The discussion of the three dimensional SVT expansion begins by taking a flat background geometry of the form $ds^2=dt^2-\delta_{ij}dx^idx^j$ where $\delta_{ij}$ represents a generic flat 3-space metric (equating to the Kronecker delta for a Minkowski background). Upon introducing a metric fluctuation $h_{\mu\nu}$ and performing a 3+1 decomposition, the geometry may be written as
%
%%%%%%%%
	\footnote{In application to cosmological backgrounds, we will find it convenient to decompose the fluctuation around a conformal to flat background by incorporating an explicit factor of $\Omega^2(x)$, with the perturbed geometry taking the form
	\begin{eqnarray}
	ds^2 &=& \Omega^2(x) \bigg[ (1+2\phi) dt^2 -2(\tilde{\nabla}_i B +B_i)dt dx^i - [(1-2\psi)\delta_{ij} +2\tilde{\nabla}_i\tilde{\nabla}_j E
	\nonumber\\
	&& + \tilde{\nabla}_i E_j + \tilde{\nabla}_j E_i + 2E_{ij}]dx^i dx^j\bigg].
	\label{AP62}
	\end{eqnarray}
	%
	Here $\Omega(x)$ is an arbitrary function of the coordinates, where $\tilde{\nabla}_i=\partial/\partial x^i$ (with Latin index) and  $\tilde{\nabla}^i=\delta^{ij}\tilde{\nabla}_j$ (not $\Omega^{-2}\delta^{ij}\tilde{\nabla}_j$) are defined with respect to the background 3-space metric $\delta_{ij}$. SVT3 elements obey the same relations as in \eqref{APsvt3_rel}, i.e. transverse and traceless with respect to the background 3-space metric.}
%%%%%%%%
%
\begin{eqnarray}
ds^2 &=&(-\eta_{\mu\nu}-h_{\mu\nu})dx^{\mu}dx^{\nu}
\nonumber\\
&=&(1+2\phi) dt^2 -2(\tilde{\nabla}_i B +B_i)dt dx^i - [(1-2\psi)\delta_{ij} +2\tilde{\nabla}_i\tilde{\nabla}_j E 
\nonumber\\
&&+ \tilde{\nabla}_i E_j + \tilde{\nabla}_j E_i + 2E_{ij}]dx^i dx^j,
\label{2.1}
\end{eqnarray}
%
where $\tilde{\nabla}_i=\partial/\partial x^i$ and  $\tilde{\nabla}^i=\delta^{ij}\tilde{\nabla}_j$  (with Latin indices) are defined with respect to the background three-space metric $\delta_{ij}$. In addition, the SVT3 components within (\ref{2.1}) are required to obey
%
\begin{eqnarray}
\delta^{ij}\tilde{\nabla}_j B_i = 0,\quad \delta^{ij}\tilde{\nabla}_j E_i = 0, \quad E_{ij}=E_{ji},\quad \delta^{jk}\tilde{\nabla}_kE_{ij} = 0, \quad \delta^{ij}E_{ij} = 0.
\label{2.2}
\label{APsvt3_rel}
\end{eqnarray}
%
As written, (\ref{2.1}) contains ten elements, whose transformations are defined with respect to the background spatial sector as four 3-dimensional scalars ($\phi$, $B$, $\psi$, $E$), two transverse 3-dimensional vectors ($B_i$, $E_i$) each with two independent degrees of freedom, and one symmetric 3-dimensional transverse-traceless tensor ($E_{ij}$) with two degrees of freedom. A la, the scalar, vector, tensor (SVT) decomposition. 

To implement the decomposition of $h_{\mu\nu}$ to the SVT3 form in \eqref{2.1}, we utilize transverse and transverse-traceless projection operators as applied to tensor and vector components to yield a decomposition into scalars, vectors, and tensors. Both the 3+1 decomposition and projection operators have been derived in developed in detail within Appendix \ref{aa:svt_projection}.
%
%%%%%%%%%%%%%%%%%%%%%%%%%%%%%%%%%%%%%
\subsection{SVT3 in Terms of $h_{\mu\nu}$ in a Conformal Flat Background}
%%%%%%%%%%%%%%%%%%%%%%%%%%%%%%%%%%%%%

Following \cite{amarasinghe_2019, phelps_2019} and making use of the projection operators in Appendix \ref{aa:svt_projection}, we express the ten degrees of freedom of the SVT3 components in a conformal to flat background in terms of the original fluctuations $h_{\mu\nu}$. First we introduce the 3-dimensional Green's function obeying
%
\begin{eqnarray}
\delta^{ij}\tilde{\nabla}_i\tilde{\nabla}_jD^{(3)}(\mathbf{x}-\mathbf{y})=\delta^3(\mathbf{x}-\mathbf{y}).
\label{AP64}
\end{eqnarray}
%
Upon setting $h_{\mu\nu}=\Omega^2(x)f_{\mu\nu}$, the line element of (\ref{AP62}) takes the form 
%
\begin{eqnarray}
ds^2&=&-[\Omega^2(x)\eta_{\alpha\beta}+h_{\alpha\beta}]dx^{\alpha}dx^{\beta}
\nonumber\\
&=&-\Omega^2(x)[\eta_{\alpha\beta}+f_{\alpha\beta}]dx^{\alpha}dx^{\beta}
\nonumber\\
&=&\Omega^2(x)\left[dt^2-\delta_{ij}dx^idx^j-f_{00}dt^2-2f_{0i}dtdx^i-f_{ij}dx^idx^j\right],
\nonumber\\
\delta^{ij}f_{ij}&=&-6\psi+2\tilde{\nabla}_i\tilde{\nabla}^iE,
\tilde{\nabla}^jf_{ij}=-2\tilde{\nabla}_i\psi+2\tilde{\nabla}_i\tilde{\nabla}_k\tilde{\nabla}^kE+\tilde{\nabla}_k\tilde{\nabla}^kE_{i},
\nonumber\\
\tilde{\nabla}^i \tilde{\nabla}^jf_{ij}&=&-2\tilde{\nabla}_k\tilde{\nabla}^k\psi+2\tilde{\nabla}_k\tilde{\nabla}^k\tilde{\nabla}_{\ell}\tilde{\nabla}^{\ell}E
\nonumber\\
&=&\frac{4}{3}\tilde{\nabla}_k\tilde{\nabla}^k\tilde{\nabla}_{\ell}\tilde{\nabla}^{\ell}E+\frac{1}{3}\tilde{\nabla}_k\tilde{\nabla}^k\delta^{ij}f_{ij}
\nonumber\\
&=&4\tilde{\nabla}_k\tilde{\nabla}^k\psi+\tilde{\nabla}_k\tilde{\nabla}^k(\delta^{ij}f_{ij}),
\nonumber\\
2\phi&=&-f_{00},\qquad
B=\int d^3yD^{(3)}(\mathbf{x}-\mathbf{y})\tilde{\nabla}_y^if_{0i},\qquad B_i=f_{0i}-\tilde{\nabla}_iB,
\nonumber\\
\psi&=&\frac{1}{4}\int d^3yD^{(3)}(\mathbf{x}-\mathbf{y})\tilde{\nabla}_y^k\tilde{\nabla}_y^{\ell}f_{k\ell}-\frac{1}{4}\delta^{k\ell}f_{k\ell},
\nonumber\\
\qquad
E&=&\int d^3yD^{(3)}(\mathbf{x}-\mathbf{y})\left[\frac{3}{4}\int d^3zD^{(3)}(\mathbf{y}-\mathbf{z})\tilde{\nabla}_z^k\tilde{\nabla}_z^{\ell}f_{k\ell}-\frac{1}{4}\delta^{k\ell}f_{k\ell}\right],
\nonumber\\
E_i&=&\int d^3yD^{(3)}(\mathbf{x}-\mathbf{y})\bigg{[}\tilde{\nabla}_y^jf_{ij}
-\tilde{\nabla}^y_i\int d^3zD^{(3)}(\mathbf{y}-\mathbf{z})\tilde{\nabla}_z^k\tilde{\nabla}_z^{\ell}f_{k\ell}\bigg{]},
\nonumber\\
2E_{ij}&=&f_{ij}+2\psi\delta_{ij} -2\tilde{\nabla}_i\tilde{\nabla}_j E - \tilde{\nabla}_i E_j - \tilde{\nabla}_j E_i, 
\label{AP65}
\end{eqnarray}
%
One may readily check that $B_i$, $E_i$, and $E_{ij}$ are indeed transverse by applying appropriate derivatives, thus confirming their obeying (\ref{APsvt3_rel}).
%%%%%
 \footnote{In (\ref{AP65}) a symbol such as $\tilde{\nabla}_y^i$, $y$ indicates that the derivative is taken with respect to the $y$ coordinate and likewise for other latin coordinates.}
 %%%%%
The integral form of the inversions of the SVT3 components is unique up to integration by parts, which plays a role in the analysis of asymptotic behavior, discussed in detail within Sect. \ref{ss:gauge_struct_svt3}.

We)where here and throughout we use the notation given in \cite{weinberg_1972}

%%%%%%%%%%%%%%%%%%%%%%%%%%%%%%%%%%%%%
\subsection{SVT3 in Terms of the Traceless $k_{\mu\nu}$ in a Conformal Flat Background}
\label{ss:svt3_in_terms_of_k_mu_nu}
%%%%%%%%%%%%%%%%%%%%%%%%%%%%%%%%%%%%%
We have shown in Sect. \ref{s:conformal_gravity} that in conformal to flat backgrounds, the perturbed Bach tensor $\delta W_{\mu\nu}$ may be expressed entirely in terms of the traceless $K_{\mu\nu}$. As such, it will prove useful to be able to express the SVT components in terms of the traceless part of $f_{\mu\nu}$. Defining $K_{\mu\nu}=\Omega^2k_{\mu\nu}$, we have
\begin{eqnarray}
K_{\mu\nu}=h_{\mu\nu}-(1/4)\Omega^2\eta_{\mu\nu}\Omega^{-2}\eta^{\alpha\beta}h_{\alpha\beta}=h_{\mu\nu}-(1/4)\eta_{\mu\nu}\eta^{\alpha\beta}h_{\alpha\beta},
\end{eqnarray}
whereby we factor out the conformal factor to form the traceless $k_{\mu\nu}$ as
\begin{eqnarray}
k_{\mu\nu}=f_{\mu\nu}-(1/4)\eta_{\mu\nu}[-f_{00}+\delta^{ij}f_{ij}].
\end{eqnarray}
%
Substituting $f_{\mu\nu}$ in terms of this $k_{\mu\nu}$, we obtain from \eqref{AP65} the following integral relations for the SVT components:
\begin{eqnarray}
k_{00}&=&\frac{3}{4}f_{00}+\frac{1}{4}\delta^{k\ell}f_{k\ell},\qquad k_{0i}=f_{0i},\qquad k_{ij}=f_{ij}+\frac{1}{4}\delta_{ij}f_{00}-\frac{1}{4}\delta_{ij}\delta^{k\ell}f_{k\ell},
\nonumber\\
\phi&=&-\frac{1}{2}f_{00},\qquad
B=\int d^3yD^{(3)}(\mathbf{x}-\mathbf{y})\tilde{\nabla}_y^ik_{0i},\qquad B_i=k_{0i}-\tilde{\nabla}_iB,
\nonumber\\
\psi&=&\frac{1}{4}\int d^3yD^{(3)}(\mathbf{x}-\mathbf{y})\tilde{\nabla}_y^k\tilde{\nabla}_y^{\ell}k_{k\ell}-\frac{3}{4}k_{00}+\frac{1}{2}f_{00},
\nonumber\\
\qquad
E&=&\int d^3yD^{(3)}(\mathbf{x}-\mathbf{y})\left[\frac{3}{4}\int d^3zD^{(3)}(\mathbf{y}-\mathbf{z})\tilde{\nabla}_z^k\tilde{\nabla}_z^{\ell}k_{k\ell}-\frac{1}{4}k_{00}\right],
\nonumber\\
E_i&=&\int d^3yD^{(3)}(\mathbf{x}-\mathbf{y})\bigg{[}\tilde{\nabla}_y^jk_{ij}
-\tilde{\nabla}^y_i\int d^3zD^{(3)}(\mathbf{y}-\mathbf{z})\tilde{\nabla}_z^k\tilde{\nabla}_z^{\ell}k_{k\ell}\bigg{]},
\nonumber\\
2E_{ij}&+&2\tilde{\nabla}_i\tilde{\nabla}_j E +\tilde{\nabla}_i E_j +\tilde{\nabla}_j E_i
=k_{ij}-\frac{1}{2}\delta_{ij}k_{00}
\nonumber\\
&&\qquad\qquad\qquad\qquad\qquad\qquad+\frac{1}{2}\delta_{ij}\int d^3yD^{(3)}(\mathbf{x}-\mathbf{y})\tilde{\nabla}_y^k\tilde{\nabla}_y^{\ell}k_{k\ell}.
\label{AP66}
\end{eqnarray}
%
Here can see that all SVT3 components can be expressed in terms of $k_{\mu\nu}$ along with a single component of $f_{\mu\nu}=\Omega^{-2}(x)h_{\mu\nu}$, namely $f_{00}$.  Recalling that $\delta W_{\mu\nu}$ can only depend on $k_{\mu\nu}$, we note that the combination $\phi+\psi$ is independent of $f_{00}$ and only depends on $k_{\mu\nu}$. Hence, we expect this coupled combination to be associated with the scalar SVT component of conformal gravity. Indeed, we confirm such a relation later in Sect. \ref{ss:deltaW_conformal_flat_SVT3}.

%%%%%%%%%%%%%%%%%%%%%%%%%%%%%%%%%%%%%
\subsection{Gauge Structure and Asymptotic Behavior}
\label{ss:gauge_struct_svt3}
%%%%%%%%%%%%%%%%%%%%%%%%%%%%%%%%%%%%%
As given in \eqref{2.1} and its integral form in \eqref{AP65}, we have shown the form of the SVT3 decomposition of $h_{\mu\nu}$ comprising 10 independent components of scalars, vectors, and tensors. Due to the coordinate freedom, it must hold that linear combinations of the SVT quantities form precisely six gauge invariant quantities (a reduction from ten initial degrees of freedom minus four coordinate transformations). Consequently, we seek to determine the coefficient combinations of the SVT quantities that form the gauge invariants. In general, this may be accomplished by manipulating the relations between the SVT components and the components of $h_{\mu\nu}$ in a general background. This procedure is carried out in \eqref{2.6} in a flat background and in \eqref{9.46a} within a general Roberston Walker background. Before discussing these results, it is informative to first analyze the structure of the gauge invariants within Einstein gravity in a source-free flat background. With the background $T_{\mu\nu}=0$ vanishing, the perturbed Einstein tensor $\delta G_{\mu\nu}$ itself is a completely gauge invariant tensor. As a function only of the metric, inspection of the components of the Einstein tensor will thus reveal the appropriate flat space gauge invariant combinations. The Einstein fluctuation takes the form,
%
\begin{eqnarray}
\delta G_{00}&=&- 2 \delta^{ab} \tilde{\nabla}_{b}\tilde{\nabla}_{a}\psi,
\nonumber\\
\delta G_{0i}&=&- 2 \tilde{\nabla}_{i}\dot{\psi}+ \tfrac{1}{2} \delta^{ab} \tilde{\nabla}_{b}\tilde{\nabla}_{a}(B_{i} -  \dot{E}_{i}),
\nonumber\\
\delta G_{ij}&=&- 2 \delta_{ij} \ddot{\psi} -  \delta^{ab} \delta_{ij} \tilde{\nabla}_{b}\tilde{\nabla}_{a}(\phi+\dot{B}  -\ddot{E})+ \delta^{ab} \delta_{ij} \tilde{\nabla}_{b}\tilde{\nabla}_{a}\psi 
	\nonumber\\
&&
+ \tilde{\nabla}_{j}\tilde{\nabla}_{i}(\phi+\dot{B} -  \ddot{E})
-  \tilde{\nabla}_{j}\tilde{\nabla}_{i}\psi
+ \tfrac{1}{2} \tilde{\nabla}_{i}(\dot{B}_{j} - \ddot{E}_{j}) + \tfrac{1}{2} \tilde{\nabla}_{j}(\dot{B}_{i}  
- \ddot{E}_{i})
\nonumber\\
&&- \ddot{E}_{ij} + \delta^{ab} \tilde{\nabla}_{b}\tilde{\nabla}_{a}E_{ij},
\nonumber\\
g^{\mu\nu}\delta G_{\mu\nu}&=&-\delta G_{00}+\delta^{ij}\delta G_{ij}=4 \delta^{ab} \tilde{\nabla}_{b}\tilde{\nabla}_{a}\psi -6\ddot{\psi}-2 \delta^{ab} \tilde{\nabla}_{b}\tilde{\nabla}_{a}(\phi+\dot{B}  -\ddot{E}),
\nonumber\\
\label{2.3}
\end{eqnarray}
%
where the dot denotes the time derivative $\partial/\partial x^0$. As mentioned, while the generic metric fluctuation $h_{\mu\nu}$ has ten components, because of the freedom to make four gauge transformations on the coordinates (i.e $h_{\mu\nu}\rightarrow h_{\mu\nu}-\partial_{\mu}\epsilon_{\nu}-\partial_{\nu}\epsilon_{\mu}$), $\delta G_{\mu\nu}$ can only depend on a total of six of them. Looking at the individual components of $\delta G_{\mu\nu}$, we see that these are proportional to the combinations $\psi$, $\phi+\dot{B}  -\ddot{E}$, $B_{i} -  \dot{E}_{i}$, and $E_{ij}$. 

However, with these identifications, there still remains a degree of ambiguity as to whether the combinations listed form the gauge invariants, or whether it is in fact derivative combinations that are truly gauge invariant. Here one must proceed carefully, as the gauge invariance of $\delta G_{\mu\nu}$ entails that only when taken with the various derivatives that appear in (\ref{2.3}) will these combinations be gauge invariant. For example, we may only state definitively that $\delta G_{00}$ is gauge invariant (hence $\delta^{ab} \tilde{\nabla}_{b}\tilde{\nabla}_{a}\psi$). The gauge invariance of $\psi$ itself cannot be assumed through the analysis on $\delta G_{\mu\nu}$ alone.


To further investigate gauge invariance issues, we express each of the various SVT3 components in terms  of combinations of the original components of $h_{\mu\nu}$. Such a procedure has been derived in \cite{amarasinghe_2019} and is to be contrasted with the integral formulations in \eqref{AP65}. Specifically, in \eqref{AP65} we mentioned uniqueness up to integration by parts, whereas here such issues are avoided as we merely apply sequences of derivatives to $h_{\mu\nu}$ to form the requisite gauge invariant structure, and afterward analyze which SVT combinations are formed as a result. The gauge invariants take the following form: using the definition
%
\begin{eqnarray}
2\phi&=&-h_{00},\quad B_i+\tilde{\nabla}_iB=h_{0i},
\nonumber\\
h_{ij}&=&-2\psi\delta_{ij} +2\tilde{\nabla}_i\tilde{\nabla}_j E + \tilde{\nabla}_i E_j + \tilde{\nabla}_j E_i + 2E_{ij},
\label{2.4}
\end{eqnarray}
%
we apply derivatives to obtain the relations
%
\begin{eqnarray}
\delta^{ij}h_{ij}&=&-6\psi+2\tilde{\nabla}_i\tilde{\nabla}^iE,\quad
\tilde{\nabla}^jh_{ij}=-2\tilde{\nabla}_i\psi+2\tilde{\nabla}_i\tilde{\nabla}_k\tilde{\nabla}^kE+\tilde{\nabla}_k\tilde{\nabla}^kE_{i},
\nonumber\\
\tilde{\nabla}^i \tilde{\nabla}^jh_{ij}&=&-2\tilde{\nabla}_k\tilde{\nabla}^k\psi+2\tilde{\nabla}_k\tilde{\nabla}^k\tilde{\nabla}_{\ell}\tilde{\nabla}^{\ell}E,
\label{2.5}
\end{eqnarray}
%
which then allow us to form the gauge invariants, taking the form
%
\begin{eqnarray}
\tilde{\nabla}_k\tilde{\nabla}^k\psi&=&\frac{1}{4} \left[\tilde{\nabla}^i \tilde{\nabla}^jh_{ij}-\tilde{\nabla}_k\tilde{\nabla}^k(\delta^{ij}h_{ij})\right],
\nonumber\\
\tilde{\nabla}_k\tilde{\nabla}^k\tilde{\nabla}_{\ell}\tilde{\nabla}^{\ell}E&=&\frac{3}{4} \tilde{\nabla}^i \tilde{\nabla}^jh_{ij}-\frac{1}{4}\tilde{\nabla}_k\tilde{\nabla}^k(\delta^{ij}h_{ij}),
\nonumber\\
\tilde{\nabla}_k\tilde{\nabla}^kB&=&\tilde{\nabla}^kh_{0k},
\nonumber\\
\tilde{\nabla}_k\tilde{\nabla}^kB_i&=&\tilde{\nabla}_k\tilde{\nabla}^kh_{0i}-\tilde{\nabla}_i\tilde{\nabla}^kh_{0k},
\nonumber\\
\tilde{\nabla}_k\tilde{\nabla}^k\tilde{\nabla}_{\ell}\tilde{\nabla}^{\ell}E_i&=&\tilde{\nabla}_k\tilde{\nabla}^k\nabla^jh_{ij}-\nabla_i\tilde{\nabla}^k\tilde{\nabla}^{\ell}h_{k\ell},
\nonumber\\
\tilde{\nabla}_k\tilde{\nabla}^kE_{ij}&=&\frac{1}{2}\big[\tilde{\nabla}_k\tilde{\nabla}^kh_{ij}-\tilde{\nabla}_i\tilde{\nabla}^kh_{kj}-\tilde{\nabla}_j\tilde{\nabla}^kh_{ki}
\nonumber\\
&&+\tilde{\nabla}_i\tilde{\nabla}_j(\delta^{k\ell}h_{k\ell})\big]+\delta_{ij}\tilde{\nabla}_k\tilde{\nabla}^k\psi
\nonumber\\
&&+\tilde{\nabla}_i\tilde{\nabla}_j\psi,
\nonumber\\
\tilde{\nabla}_{\ell}\tilde{\nabla}^{\ell}\tilde{\nabla}_k\tilde{\nabla}^kE_{ij}&=&
\frac{1}{2} \tilde{\nabla}_{\ell}\tilde{\nabla}^{\ell}\big[\tilde{\nabla}_k\tilde{\nabla}^kh_{ij}-\tilde{\nabla}_i\tilde{\nabla}^kh_{kj}-\tilde{\nabla}_j\tilde{\nabla}^kh_{ki}
\nonumber\\
&&+\tilde{\nabla}_i\tilde{\nabla}_j(\delta^{k\ell}h_{k\ell})\big]+\frac{1}{4}\left[\delta_{ij}\tilde{\nabla}_{\ell}\tilde{\nabla}^{\ell}+\tilde{\nabla}_i\tilde{\nabla}_j \right]\times
\nonumber\\
&&\left[\tilde{\nabla}^m \tilde{\nabla}^nh_{mn}-\tilde{\nabla}_k\tilde{\nabla}^k(\delta^{mn}h_{mn}) \right],
\nonumber\\
\tilde{\nabla}_{\ell}\tilde{\nabla}^{\ell} \tilde{\nabla}_k\tilde{\nabla}^k(B_i-\dot{E}_i)&=&
\tilde{\nabla}_{\ell}\tilde{\nabla}^{\ell}\tilde{\nabla}_k\tilde{\nabla}^kh_{0i}
-\tilde{\nabla}_{\ell}\tilde{\nabla}^{\ell}\tilde{\nabla}_i\tilde{\nabla}^kh_{0k}
-\partial_0\tilde{\nabla}_{\ell}\tilde{\nabla}^{\ell}\tilde{\nabla}^jh_{ij}
\nonumber\\
&&+\partial_0\tilde{\nabla}_{i}\tilde{\nabla}^{k}\tilde{\nabla}^{\ell}h_{k\ell},
\nonumber\\
\tilde{\nabla}_k\tilde{\nabla}^k\tilde{\nabla}_{\ell}\tilde{\nabla}^{\ell}(\phi+\dot{B}-\ddot{E})&=&
-\tfrac{1}{2}\tilde{\nabla}_k\tilde{\nabla}^k\tilde{\nabla}_{\ell}\tilde{\nabla}^{\ell}h_{00}
+\tilde{\nabla}_{\ell}\tilde{\nabla}^{\ell}\partial_0\tilde{\nabla}^kh_{0k}
-\tfrac{3}{4}\partial_0^2\tilde{\nabla}^i\tilde{\nabla}^jh_{ij}
\nonumber\\
&&+\tfrac{1}{4}\partial_0^2\tilde{\nabla}_{k}\tilde{\nabla}^{k}(\delta^{ij}h_{ij}).
\label{2.6}
\end{eqnarray}
%
Given (\ref{2.6}) one can readily check that under a gauge transformation $h_{\mu\nu}\rightarrow h_{\mu\nu}-\partial_{\mu}\epsilon_{\nu}-\partial_{\nu}\epsilon_{\mu}$ the combinations  $\tilde{\nabla}_k\tilde{\nabla}^k\psi $, $\tilde{\nabla}_{\ell}\tilde{\nabla}^{\ell}\tilde{\nabla}_k\tilde{\nabla}^kE_{ij}$, $\tilde{\nabla}_{\ell}\tilde{\nabla}^{\ell}\tilde{\nabla}_k\tilde{\nabla}^k(B_i-\dot{E}_i)$ and $ \tilde{\nabla}_k\tilde{\nabla}^k\tilde{\nabla}_{\ell}\tilde{\nabla}^{\ell}(\phi+\dot{B}-\ddot{E})$ are gauge invariant. We see here that it was in fact necessary to apply higher order derivatives than found in $\delta G_{\mu\nu}$ in order to express each of the SVT3 components entirely in terms of combinations of components of the $h_{\mu\nu}$. Hence, we repeat that it is not the quantities $\psi$, $E_{ij}$, $B_i-\dot{E}_i$ and $\phi+\dot{B}-\ddot{E}$ themselves that are necessarily gauge invariant; rather, it is their derivatives that are  gauge invariant. In comparing \eqref{2.6} to (\ref{2.3}) we see that it is the quantity $\tilde{\nabla}_k\tilde{\nabla}^k\psi$ that appears in $\delta G_{00}$ and that it is the combination $ \tilde{\nabla}_k\tilde{\nabla}^kE_{ij}-\delta_{ij}\tilde{\nabla}_k\tilde{\nabla}^k\psi-\tilde{\nabla}_i\tilde{\nabla}_j\psi$ that appears in  $\delta G_{ij}$. Thus these  combinations are automatically gauge invariant.

To touch basis with \eqref{AP65}, we could proceed to integrate the relevant equations in (\ref{2.6}) in order to check gauge invariance for $\psi$, $\phi+\dot{B}-\ddot{E}$, $B_{i}-\dot{E_i}$ and $E_{ij}$ themselves, since we can set

%
\begin{eqnarray}
\psi&=&\frac{1}{4}\int d^3yD^{(3)}(\mathbf{x}-\mathbf{y})\left[\tilde{\nabla}_y^k \tilde{\nabla}_y^{\ell}h_{k\ell}-\tilde{\nabla}^y_m\tilde{\nabla}_y^m(\delta^{k\ell}h_{k\ell})\right],
\nonumber\\
\phi+\dot{B}-\ddot{E}&=&-\frac{1}{2} h_{00}
+\partial_0\left[\int d^3y D^{(3)}(\mathbf x - \mathbf y) \tilde\nabla^k_y h_{0k}\right]
\nonumber\\
&-&\partial_0^2\bigg[\int d^3y D^{(3)}(\mathbf x - \mathbf y) \int d^3z D^{(3)}(\mathbf y - \mathbf z)\times
\nonumber\\
&&\left[ \frac{3}{4} \tilde{\nabla}^i \tilde{\nabla}^jh_{ij}-\frac{1}{4}\tilde{\nabla}_k\tilde{\nabla}^k(\delta^{ij}h_{ij})
\right]\bigg]
\nonumber\\
&=&-\tfrac{1}{2}\tilde{\nabla}_{\ell}\tilde{\nabla}^{\ell} \tilde{\nabla}_k\tilde{\nabla}^k\int d^3y D^{(3)}(\mathbf x - \mathbf y) \int d^3z D^{(3)}(\mathbf y - \mathbf z)h_{00}
\nonumber\\
&+&\partial_0\tilde{\nabla}_{\ell}\tilde{\nabla}^{\ell}\int d^3y D^{(3)}(\mathbf x - \mathbf y) \int d^3z D^{(3)}(\mathbf y - \mathbf z)\nabla^k_z h_{0k}
\nonumber\\
&-&\partial_0^2\bigg[\int d^3y D^{(3)}(\mathbf x - \mathbf y) \int d^3z D^{(3)}(\mathbf y - \mathbf z)\times
\nonumber\\
&&\left[ \frac{3}{4} \tilde{\nabla}^i \tilde{\nabla}^jh_{ij}-\frac{1}{4}\tilde{\nabla}_k\tilde{\nabla}^k(\delta^{ij}h_{ij})
\right]\bigg],
\label{2.7}
\end{eqnarray}
%
and 
%
\begin{eqnarray}
B_i &-&\dot{E}_i= \int d^3y D^{(3)}(\mathbf x - \mathbf y)\left[ \tilde\nabla^k_y \tilde\nabla_k^y h_{0i}
- \tilde\nabla_i^y \tilde\nabla^k_y h_{0k} \right]
\nonumber\\
&-&\partial_0\left[\int d^3y D^{(3)}(\mathbf x - \mathbf y) \int d^3z D^{(3)}(\mathbf y - \mathbf z)
\left[ \tilde\nabla^k_z \tilde\nabla_k^z \tilde\nabla^j_z h_{ij}-\tilde\nabla_i^z \tilde\nabla^k_z \tilde\nabla^{\ell}_z h_{k\ell}\right]\right]
\nonumber\\
&=&\tilde{\nabla}_{\ell}\tilde{\nabla}^{\ell} \int d^3y D^{(3)}(\mathbf x - \mathbf y) \int d^3z D^{(3)}(\mathbf y - \mathbf z)\left[ \tilde\nabla^k_z \tilde\nabla_k^z h_{0i}
- \tilde\nabla_i^z \tilde\nabla^k_z h_{0k} \right]
\nonumber\\
&-&\partial_0\left[\int d^3y D^{(3)}(\mathbf x - \mathbf y) \int d^3z D^{(3)}(\mathbf y - \mathbf z)
\left[ \tilde\nabla^k_z \tilde\nabla_k^z \tilde\nabla^j_z h_{ij}-\tilde\nabla_i^z \tilde\nabla^k_z \tilde\nabla^{\ell}_z h_{k\ell}\right]\right],
\nonumber\\
E_{ij}&=&\frac{1}{2}\int d^3yD^{(3)}(\mathbf{x}-\mathbf{y})\left[\tilde{\nabla}^y_k\tilde{\nabla}_y^kh_{ij}-\tilde{\nabla}^y_i\tilde{\nabla}_y^kh_{kj}-\tilde{\nabla}^y_j\tilde{\nabla}_y^kh_{ki}+\tilde{\nabla}^y_i\tilde{\nabla}^y_j(\delta^{k\ell}h_{k\ell})\right]
\nonumber\\
&+&\frac{1}{4}\int d^3yD^{(3)}(\mathbf{x}-\mathbf{y})\left[\delta_{ij}\tilde{\nabla}^y_{\ell}\tilde{\nabla}_y^{\ell}+\tilde{\nabla}^y_i\tilde{\nabla}^y_j\right]\int d^3zD^{(3)}(\mathbf{y}-\mathbf{z})\times
\nonumber\\
&&\left[\tilde{\nabla}_z^m \tilde{\nabla}_z^{n}h_{mn}-\tilde{\nabla}^z_k\tilde{\nabla}_z^k(\delta^{mn}h_{mn})\right],
\label{2.8}
\end{eqnarray}
%
where we make use of the flat space Green's function $D^{(3)}(\mathbf{x}-\mathbf{y})$ obeying 
%
\begin{eqnarray}
\delta^{ij}\tilde{\nabla}_i\tilde{\nabla}_jD^{(3)}(\mathbf{x}-\mathbf{y})&=&\delta^3(\mathbf{x}-\mathbf{y}),\quad
D^{(3)}(\mathbf{x}-\mathbf{y})=-\frac{1}{4\pi |\mathbf{x}-\mathbf{y}|},
\nonumber\\
\int d^3\mathbf{y}e^{i\mathbf{q}\cdot\mathbf{y}}D^{(3)}(\mathbf{x}-\mathbf{y})&=&-\frac{e^{i\mathbf{q}\cdot\mathbf{x}}}{q^2}.
\label{2.9}
\end{eqnarray}
%
(Here $q^2=\delta^{ij}q_{i}q_{j}$, and in $\tilde{\nabla}_y^i$ the $y$ indicates that the derivative is taken with respect to the $y$ coordinate, and likewise for other coordinate choices.)

As eluded to below \eqref{AP65}, there remains however an issue within (\ref{2.7}) and (\ref{2.8}). Specifically, since $\psi$ and $E_{ij}$ are manifestly gauge invariant as is (they are expressed in terms of the gauge-invariant flat 3-space $\delta R_{ij}$ and $\delta^{ij}\delta R_{ij}$), in order to show the gauge invariance of $\phi+\dot{B}-\ddot{E}$ and $B_i -\dot{E}_i$ we would need to be able to integrate by parts (i.e., for $\phi+\dot{B}-\ddot{E}$  we would need to bring $\tilde{\nabla}_{\ell}\tilde{\nabla}^{\ell} \tilde{\nabla}_k\tilde{\nabla}^k$ and $\tilde{\nabla}_{\ell}\tilde{\nabla}^{\ell}$ inside the double integral, while for $B_i-\dot{E}_i$ we would need to bring $\tilde{\nabla}_{\ell}\tilde{\nabla}^{\ell}$ inside, and similarly to show tranverness for $B_i -\dot{E}_i$ and $E_{ij}$ we need to be able to integrate by parts.) Consequently, we are then forced to one of three scenarios. Either a) we must put constraints on how $h_{\mu\nu}$ is to behave asymptotically, or b) restrict to requiring in the $E_{ij}$ sector that only $\tilde{\nabla}_{\ell}\tilde{\nabla}^{\ell}\tilde{\nabla}_k\tilde{\nabla}^kE_{ij}$ be gauge invariant and that only $\tilde{\nabla}_{\ell}\tilde{\nabla}^{\ell}\tilde{\nabla}_k\tilde{\nabla}^kE_{ij}$ be transverse or c) in the $E_{ij}$ plus $\psi$ sector restrict to requiring that only $\tilde{\nabla}_k\tilde{\nabla}^kE_{ij}-\delta_{ij}\tilde{\nabla}_k\tilde{\nabla}^k\psi-\tilde{\nabla}_i\tilde{\nabla}_j\psi$ be gauge invariant and that only $\tilde{\nabla}_k\tilde{\nabla}^kE_{ij}$ be transverse. 


%%%%%%%%%%%%%%%%%%%%%%%%%%%%%%%%%%%%%
	\footnote{
	In a similar manner, we may also integrate the remaining SVT3 components, obtaining
	%
	\begin{eqnarray}
	2\phi&=&-h_{00},\quad
	B=\int d^3yD^{(3)}(\mathbf{x}-\mathbf{y})\tilde{\nabla}_y^ih_{0i},
	\nonumber\\
	B_i&=&h_{0i}-\tilde{\nabla}_i\int d^3yD^{(3)}(\mathbf{x}-\mathbf{y})\tilde{\nabla}_y^ih_{0i},
	\nonumber\\
	E&=&\frac{1}{4}\int d^3yD^{(3)}(\mathbf{x}-\mathbf{y})\int d^3zD^{(3)}(\mathbf{y}-\mathbf{z})\left[3\tilde{\nabla}_z^k\tilde{\nabla}_z^{\ell}h_{k\ell}-\tilde{\nabla}^z_k\tilde{\nabla}_z^k(\delta^{k\ell}h_{k\ell})\right],
	\nonumber\\
	E_i&=&\int d^3yD^{(3)}(\mathbf{x}-\mathbf{y})\int d^3zD^{(3)}(\mathbf{y}-\mathbf{z})\left[\tilde{\nabla}^z_k\tilde{\nabla}_z^k\nabla_z^jh_{ij}-\nabla^z_i\tilde{\nabla}_z^k\tilde{\nabla}_z^{\ell}h_{k\ell}\right]
	\label{2.12}
	\end{eqnarray}
	%
	As constructed, we see that $\tilde{\nabla}^iB_i=0$. However to show $\nabla^iE_i=0$, we need to be able to integrate by parts. Using (\ref{2.4}) and (\ref{2.6}) directly, we can then show that both $\tilde{\nabla}_k\tilde{\nabla}^k\tilde{\nabla}_{\ell}\tilde{\nabla}^{\ell}(\phi+\dot{B}-\ddot{E})$ and $\tilde{\nabla}_k\tilde{\nabla}^k\tilde{\nabla}_{\ell}\tilde{\nabla}^{\ell}(B_i-\dot{E}_i)$ are gauge invariant, with the gauge invariance of $\phi+\dot{B}-\ddot{E}$ and $B_i-\dot{E}_i$ themselves then following when defining $B$, $B_i$, $E$ and $E_i$ according to (\ref{2.12}). Hence, granted the freedom to integrate by parts, we can show that for fluctuations around flat spacetime all of the six $\psi$, $E_{ij}$,  $\phi+\dot{B}-\ddot{E}$ and $B_i-\dot{E}_i$ quantities that appear in $\delta G_{\mu\nu}$ as given in (\ref{2.3}) are gauge invariant.
	}
%%%%%%%%%%%%%%%%%%%%%%%%%%%%%%%%%%%%%

To see how method a), constraining the asymptotic behavior of $h_{\mu\nu}$, may resolve the issues of integration by parts, we shall take $h_{\mu\nu}$  to be localized in space and oscillating in time. Specifically, for each mode we will set $h_{ij}=\epsilon_{ij}(q)e^{i\mathbf{q}\cdot\mathbf{x}-i\omega(q) t}$ with $\omega(q)$ as yet undefined (and thus not necessarily equal to $q$), and where $\epsilon_{ij}(q)$ serves as the polarization tensor. As a localized packet, we constrain the form of the polarization tensor by excluding any functional dependence of the form $\delta(q)$ or $\delta(q)/q$. Thus, referring to \eqref{2.7} and \eqref{2.8}, for spatially localized fluctuations comprising a single mode, the quantities $\psi$ and $E_{ij}$ given in (\ref{2.7}) and (\ref{2.8}) evaluate to
%
\begin{eqnarray}
\psi&=&e^{i\mathbf{q}\cdot\mathbf{x}-i\omega(q) t}\frac{[q^kq^{\ell}\epsilon_{k\ell}(q)-q^2\delta^{k\ell}\epsilon_{k\ell}(q)]}{4q^2},
\nonumber\\
E_{ij}&=&e^{i\mathbf{q}\cdot\mathbf{x}-i\omega(q) t}\bigg{[}\frac{[q^2\epsilon_{ij}(q)-q_iq^k\epsilon_{kj}(q)-q_jq^k\epsilon_{ki}(q)+q_iq_j\delta^{k\ell}\epsilon_{k\ell}(q)]}{2q^2}
\nonumber\\
&+&\frac{(\delta_{ij}q^2+q_iq_j)[q^kq^{\ell}\epsilon_{k\ell}(q)-q^2\delta^{k\ell}\epsilon_{k\ell}(q)]}{4q^4}\bigg{]}.
\label{2.10}
\end{eqnarray}
%
With application of $\tilde\nabla^j$, one may confirm the transverse relation $\tilde{\nabla}^jE_{ij}=0$. To construct a wave packet, we sum over all modes viz. $h_{ij}=\sum_qa_q\epsilon_{ij}(q)e^{i\mathbf{q}\cdot\mathbf{x}-i\omega(q) t}$, to then obtain
%
\begin{eqnarray}
\psi&=&\sum_qa_qe^{i\mathbf{q}\cdot\mathbf{x}-i\omega(q) t}\frac{[q^kq^{\ell}\epsilon_{k\ell}(q)-q^2\delta^{k\ell}\epsilon_{k\ell}(q)]}{4q^2},
\nonumber\\
E_{ij}&=&\sum_qa_qe^{i\mathbf{q}\cdot\mathbf{x}-i\omega(q) t}\bigg{[}\frac{[q^2\epsilon_{ij}(q)-q_iq^k\epsilon_{kj}(q)-q_jq^k\epsilon_{ki}(q)+q_iq_j\delta^{k\ell}\epsilon_{k\ell}(q)]}{2q^2}
\nonumber\\
&+&\frac{(\delta_{ij}q^2+q_iq_j)[q^kq^{\ell}\epsilon_{k\ell}(q)-q^2\delta^{k\ell}\epsilon_{k\ell}(q)]}{4q^4}\bigg{]},
\label{2.11}
\end{eqnarray}
%
where again $\tilde{\nabla}^jE_{ij}=0$. Since the set of all $e^{i\mathbf{q}\cdot\mathbf{x}-i\omega (q)t}$ plane waves is complete for fluctuations around flat, any mode can be expanded as a general sum $h_{ij}=\sum_qa_q\epsilon_{ij}(q)e^{i\mathbf{q}\cdot\mathbf{x}-i\omega(q) t}$, with it following that (\ref{2.11}) then holds for the complete plane wave basis. Hence, by constructing the $\psi$ and $E_{ij}$ in a localized plane-wave basis, we confirm the transverse relation $\tilde\nabla^j E_{ij} = 0$ without encountering issues related to integration by parts.

While we have demonstrated the role asymptotic behavior plays within tradeoff of transverse behavior vs. gauge invariance, it is also of importance to consider under conditions the SVT3 decomposition of $h_{\mu\nu}$ may be afforded in the first place. We revisit the SVT3 derivation constructed in \cite{amarasinghe_2019} with an eye towards boundary conditions and asymptotic behavior. 

Let us suppose that we are given a general vector $A_i$ and we desire to extract out its transverse and longitudinal components, to thereby construct a relation $A_i=V_i+\partial_iL$ where $\partial_iV^i=0$. Applying $\partial^i$, it follows that
%
\begin{eqnarray}
\partial_i\partial^iL=\partial_iA^i.
\label{2.13}
\end{eqnarray}
%
Recalling the Green's identity
%
\begin{eqnarray}
A \partial_i\partial^iB-B \partial_i\partial^iA=\partial_i(A\partial^iB-B\partial^iA),
\label{1.9}
\end{eqnarray}
and introducing the Green's function
%
\begin{eqnarray}
\partial_i\partial^i D(\mathbf x-\mathbf y) = \delta^3(\mathbf x- \mathbf y),
\end{eqnarray}
%
the general  solution to (\ref{2.13}) is of the form 
%
\begin{eqnarray}
L({\bf x})&=&\int d^3yD^{(3)}(\mathbf{x}-\mathbf{y})\partial^y_jA^j({\bf y})\\
\nonumber\\
&&+\int dS_y^i\left[L({\bf y})\partial^y_iD^{(3)}(\mathbf{x}-\mathbf{y})-D^{(3)}(\mathbf{x}-\mathbf{y})\partial^y_iL({\bf y})\right].
\label{2.14}
\end{eqnarray}
%
Now utilizing $A_i = V_i + \partial_i L$, it follows that 
%
\begin{eqnarray}
A_i({\bf x})&=&V_i({\bf x})+\partial^x_iL=V_i({\bf x})+\partial^x_i\int d^3yD^{(3)}(\mathbf{x}-\mathbf{y})\partial^y_jA^j({\bf y})
\nonumber\\
&+&\partial^x_i\int dS_y^i\left[L({\bf y})\partial^y_iD^{(3)}(\mathbf{x}-\mathbf{y})-D^{(3)}(\mathbf{x}-\mathbf{y})\partial^y_iL({\bf y})\right].
\label{2.15}
\end{eqnarray}
%
Upon applying $\partial_x^i$ to (\ref{2.15}), we obtain
%
\begin{eqnarray}
&&\partial_x^i\partial^x_i\int dS_y^i\left[L({\bf y})\partial^y_iD^{(3)}(\mathbf{x}-\mathbf{y})-D^{(3)}(\mathbf{x}-\mathbf{y})\partial^y_iL({\bf y})\right]=0,
\label{2.16}
\end{eqnarray}
%
to thus establish that $\partial_i\int dS^i(L\partial_iD^{(3)}-D^{(3)}\partial_iL)$ is transverse. At this point, we appear to have constructed two transverse components - both $V_i$ and 
%
\begin{eqnarray}
\partial_i\int dS^i(L\partial_iD^{(3)}-D^{(3)}\partial_iL)
\end{eqnarray}
%
itself. To allow $V_i(x)$ to serve as the unique transverse component of $A_i(x)$, we must then require that $\partial_i\int dS^i(L\partial_iD^{(3)}-D^{(3)}\partial_iL)$ vanish. Consequently, we see that we must inadvertently impose a constraint on $A_i$, namely that $A_i$ be spatially asymptotically well-behaved. Thus, the unique decomposition of $A_i$ into longitudinal and traverse components necessarily entails an assumption regarding the asymptotic behavior of $A_i$. 

%%%%%%%%%%%%%%%%%%%%%%%%%%%%%%%%%%%%%
\section{SVTD}
\label{s:svtd}
%%%%%%%%%%%%%%%%%%%%%%%%%%%%%%%%%%%%%

While the SVT3 formalism presented thus far has demonstrated that the gauge invariant SVT3 quantities are covariant, one may note that these quantities have been defined with respect to the three-dimensional subspace of the full four-dimensional spacetime. With the general $h_{\mu\nu}$ behaving under coordinate transformation as
%
\begin{eqnarray}
h_{\mu\nu} \to h_{\mu\nu}' = \frac{\partial x'^\alpha}{\partial x^\mu}\frac{\partial x'^\beta}{\partial x^\nu} h_{\alpha\beta},
\end{eqnarray}
%
we see that a quantity such as $h_{00} = -2\phi$ may transform into a scalar involving vector components. To see this, let us form the four vector $h_{0\mu}$, which in terms of SVT3 components takes the form
\begin{eqnarray}
h_{0\mu} = \begin{pmatrix}-2\phi\\B_1 + \tilde\nabla_1 B\\B_2 + \tilde\nabla_2 B\\B_3 + \tilde\nabla_3 B\end{pmatrix}.
\end{eqnarray}
Now, with the full four-dimensional coordinate transformation mixing each of the four components of $h_{0\mu}$, we see that the transformation of $\phi$ may induce a contribution due to a vector $B_i$. 

Thus to decompose the $h_{\mu\nu}$ into a set of scalars, vectors, and tensors that remain closed under the Poincare group, we must develop a formalism that matches the underlying space-time dimensionality; namely an SVT4 formalism. We proceed to do so here in a flat spacetime, following the series of steps given within \cite{phelps_2019}. It is no additional overhead to generalize this to $D$ dimensions here, with even further generalization to arbitrary curved spacetimes found in detail within Appendix \ref{aa:svt_projection}.

We defined Greek indices to range over the full D-dimensional space and begin with the construction of a symmetric rank two tensor $F_{\mu\nu}$, taken to be transverse and traceless in the full D-dimensional space.
%%%%%%%%%%%%%%%%%%%%%%%%
\footnote{The previously introduced $E_{ij}$ was only transverse and traceless in a 3-dimensional subspace.}
%%%%%%%%%%%%%%%%%%%%%%%%
Accounting for the dimensionality and the transverse traceless constraints, $F_{\mu\nu}$ tensor will have $D(D+1)/2-D-1=(D+1)(D-2)/2$ components. To facilitate the decomposition, we introduce a D-dimensional  vector $W_{\mu}$. In terms of this $W_{\mu}$ and $h$, and motivated by \cite{mannheim_2005}, we define the general $h_{\mu\nu}$ fluctuation around a flat D-dimensional space to be of the form
%
\begin{eqnarray}
h_{\mu\nu}=2F_{\mu\nu}+\nabla_{\nu}W_{\mu}+\nabla_{\mu}W_{\nu}+\frac{2-D}{D-1}\nabla_{\mu}\nabla_{\nu}\int d^DyD^{(D)}(x-y)\nabla^{\alpha}W_{\alpha}
\nonumber\\
-\frac{g_{\mu\nu}}{D-1}(\nabla^{\alpha}W_{\alpha}-h)-\frac{\nabla_{\mu}\nabla_{\nu}}{D-1}\int d^DyD^{(D)}(x-y)h,
\label{3.1}
\end{eqnarray}
%
where the flat spacetime $D^{(D)}(x-y)$ obeys 
%
\begin{eqnarray}
g^{\mu\nu}\nabla_{\mu}\nabla_{\nu}D^{(D)}(x-y)=\delta^{(D)}(x-y).
\label{3.2}
\end{eqnarray}
%
Parallel to the SVT3 asymptotic behavior discussed in Sec. \ref{ss:gauge_struct_svt3}, it is implied in the form of $h_{\mu\nu}$ that the D-dimensional integrals exist, specifically with $\nabla^{\alpha}W_{\alpha}$ being sufficiently well-behaved at infinity.
To make the $F_{\mu\nu}$ that is defined by (\ref{3.1}) be transverse and traceless requires D+1 conditions, D to be supplied by $W_{\mu}$ and one  to be supplied by $h$. Given the defined \eqref{3.1}, one may take the trace and confirm that $F_{\mu\nu}$ is traceless as written. To assess whether $F_{\mu\nu}$ is transverse, we apply $\nabla^{\nu}$  to (\ref{3.1}) yielding
%
\begin{eqnarray}
\nabla^{\nu}h_{\nu\mu}=\nabla_{\alpha}\nabla^{\alpha}W_{\mu}.
\label{3.3}
\end{eqnarray}
% 
Thus \eqref{3.3} serves to define the as yet determined $D$ components of $W_\mu$ in terms of the $h_{\mu\nu}$. Questions on the asymptotic behavior of $\nabla^{\alpha}W_{\alpha}$ are thus directly linked the asymptotic behavior of $h_{\mu\nu}$. Hence, for a $W_{\mu}$ that obeys (\ref{3.3}) and is sufficiently bounded, the D-dimensional rank two tensor $F_{\mu\nu}$ is transverse and traceless.

Uponj applying $\nabla_{\alpha}\nabla^{\alpha}$ to (\ref{3.1}) we obtain
%
\begin{eqnarray}
\nabla_{\alpha}\nabla^{\alpha}h_{\mu\nu}&=&2\nabla_{\alpha}\nabla^{\alpha}F_{\mu\nu}+\nabla_{\nu}\nabla^{\alpha}h_{\alpha\mu}+\nabla_{\mu}\nabla^{\alpha}h_{\alpha\nu}+\frac{2-D}{D-1}\nabla_{\mu}\nabla_{\nu}\nabla^{\alpha}W_{\alpha}
\nonumber\\
&-&\frac{g_{\mu\nu}}{D-1}(\nabla^{\alpha}\nabla^{\beta}h_{\alpha\beta}-\nabla_{\alpha}\nabla^{\alpha}h)-\frac{\nabla_{\mu}\nabla_{\nu}}{D-1}h,
\label{3.4}
\end{eqnarray}
%
and on rearranging we obtain
%
\begin{eqnarray}
&&\nabla_{\alpha}\nabla^{\alpha}h_{\mu\nu}-\nabla_{\nu}\nabla^{\alpha}h_{\alpha\mu}-\nabla_{\mu}\nabla^{\alpha}h_{\alpha\nu}+\nabla_{\mu}\nabla_{\nu}h
\nonumber\\
&=&2\nabla_{\alpha}\nabla^{\alpha}F_{\mu\nu}+\frac{2-D}{D-1}\nabla_{\mu}\nabla_{\nu}[\nabla^{\alpha}W_{\alpha}-h]
\nonumber\\
&&-\frac{g_{\mu\nu}}{D-1}(\nabla^{\alpha}\nabla^{\beta}h_{\alpha\beta}-\nabla_{\alpha}\nabla^{\alpha}h).
\label{3.5}
\end{eqnarray}
%
Now in flat spacetime, we note that the perturbed curvature tensors $\delta R$ and $\delta R_{\mu\nu}$ are independelty gauge invariant. Thus, armed with the SVTD formalism, we use these to determine the gauge invariants
%
\begin{eqnarray}
2\delta R_{\mu\nu} = \nabla_{\alpha}\nabla^{\alpha}h_{\mu\nu}-\nabla_{\nu}\nabla^{\alpha}h_{\alpha\mu}-\nabla_{\mu}\nabla^{\alpha}h_{\alpha\nu}+\nabla_{\mu}\nabla_{\nu}h
\label{dricci}
\end{eqnarray}
%
%
\begin{eqnarray}
-\delta R = \nabla^{\alpha}\nabla^{\beta}h_{\alpha\beta}-\nabla_{\alpha}\nabla^{\alpha}h.
\label{driccis}
\end{eqnarray}
Making use of
%
\begin{eqnarray}
\nabla_{\beta}\nabla^{\beta}[\nabla^{\alpha}W_{\alpha}-h]=\nabla^{\alpha}\nabla^{\beta}h_{\alpha\beta}-\nabla_{\alpha}\nabla^{\alpha}h,
\label{3.6}
\end{eqnarray}
%
we define
%
\begin{eqnarray}
\nabla^{\alpha}W_{\alpha}-h=\int d^DyD^{(D)}(x-y)[\nabla^{\alpha}\nabla^{\beta}h_{\alpha\beta}-\nabla_{\alpha}\nabla^{\alpha}h],
\label{3.7}
\end{eqnarray}
%
and with this solution and reference to \eqref{dricci} and \eqref{driccis} we see that  $\nabla_{\alpha}\nabla^{\alpha}F_{\mu\nu}$ is thus gauge invariant. However, we note again that parallel to the discussion the SVT3 tensor $E_{ij}$, to show that $\nabla_{\alpha}\nabla^{\alpha}F_{\mu\nu}$ is transverse requires that we one can justify an integrate by parts.

For the remaining SVTD components, we make the following definitions
%
\begin{eqnarray}
2\chi&=&\frac{1}{D-1}[\nabla^{\alpha}W_{\alpha}-h],\quad 
\quad 2F=\frac{1}{D-1}\int d^DyD^{(D)}(x-y)[D\nabla^{\alpha}W_{\alpha}-h],
\nonumber\\
F_{\mu}&=&W_{\mu}-\nabla_{\mu}\int d^DyD^{(D)}(x-y)\nabla^{\alpha}W_{\alpha}.
\label{3.8}
\end{eqnarray}
%
As constructed, within (\ref{3.8}) $F_\mu$ is defined such that $\nabla^{\mu}F_{\mu}=0$. As for $\chi$, we note with (\ref{3.7}) we find that  $\chi$ is the integral of a gauge-invariant function so that $\nabla_{\alpha}\nabla^{\alpha}\chi$ is also automatically gauge invariant. Finally, with (\ref{3.8}) we can express (\ref{3.1}) in terms of the SVTD quantities as
%
\begin{eqnarray}
h_{\mu\nu}=-2g_{\mu\nu}\chi+2\nabla_{\mu}\nabla_{\nu}F
+ \nabla_{\mu}F_{\nu}+\nabla_{\nu}F_{\mu}+2F_{\mu\nu},
\label{3.9}
\end{eqnarray}
%
to thus complete the SVTD basis decomposition of $h_{\mu\nu}$. In a general D-dimensional basis $F_{\mu\nu}$ has $(D+1)(D-2)/2$ components, the transverse $F_{\mu}$ has $D-1$ components, the two scalars $\chi$ and $F$ each have one component, and together they comprise the $D(D+1)/2$ components of a general $h_{\mu\nu}$. If we  elect to take $D=3$, (i.e. a decomposition of $h_{ij}$) we can then recognize (\ref{3.9}) as the spatial piece of SVT3 given in (\ref{2.1}), providing a check on our results.



%%%%%%%%%%%%%%%%%%%%%%%%%%%%%%%%%%%%%
\subsection{Gauge Structure ($D=4$)}
\label{S1e}
%%%%%%%%%%%%%%%%%%%%%%%%%%%%%%%%%%%%%
%
Counting the degrees of freedom of the fluctuation $\delta G_{\mu\nu}$ around flat D-dimensional we arrive at a total of $D(D+1)/2-D=D(D-1)/2$ independent gauge-invariant combinations. For the SVTD components, $F_{\mu\nu}$ has $(D+1)(D-2)/2$ components and $\chi$ has one. That is, the combination of $\chi$ and $F_{\mu\nu}$ precisely forms a total of $D(D-1)/2$. Moreoever with the derivatives of both being gauge invariant, we deduce that $\delta G_{\mu\nu}$ can only depend on $F_{\mu\nu}$ and $\chi$. Applying appropriate derivatives to the quantities given (\ref{3.8}) and (\ref{3.9}), we indeed form the following gauge invariant relations
%
\begin{eqnarray}
2\nabla_{\alpha}\nabla^{\alpha}\chi&=&\frac{1}{D-1}\left[\nabla^{\alpha}\nabla^{\beta}h_{\alpha\beta}-\nabla_{\alpha}\nabla^{\alpha}h\right],
\nonumber\\
2\nabla_{\alpha}\nabla^{\alpha}\nabla_{\beta}\nabla^{\beta}F_{\mu\nu}&=&\nabla_{\beta}\nabla^{\beta}\left[\nabla_{\alpha}\nabla^{\alpha}h_{\mu\nu}-\nabla_{\nu}\nabla^{\alpha}h_{\alpha\mu}-\nabla_{\mu}\nabla^{\alpha}h_{\alpha\nu}+\nabla_{\mu}\nabla_{\nu}h\right]
\nonumber\\
&+&\frac{1}{D-1}\left[(D-2)\nabla_{\mu}\nabla_{\nu}+g_{\mu\nu}\nabla_{\gamma}\nabla^{\gamma}\right][\nabla^{\alpha}\nabla^{\beta}h_{\alpha\beta}-\nabla_{\alpha}\nabla^{\alpha}h],
\nonumber\\
\delta R_{\mu\nu}&=&\frac{1}{2}[2\nabla_{\alpha}\nabla^{\alpha}F_{\mu\nu}+2(2-D)\nabla_{\mu}\nabla_{\nu}\chi-2g_{\mu\nu}\nabla_{\alpha}\nabla^{\alpha}\chi],
\nonumber\\
\delta R&=&2(1-D)\nabla_{\alpha}\nabla^{\alpha}\chi,
\nonumber\\
\delta G_{\mu\nu}&=&\delta R_{\mu\nu}-\frac{1}{2}g_{\mu\nu}g^{\alpha\beta}\delta R_{\alpha\beta}=\nabla_{\alpha}\nabla^{\alpha}F_{\mu\nu}
\nonumber\\
&&+(D-2)(g_{\mu\nu}\nabla_{\alpha}\nabla^{\alpha}-\nabla_{\mu}\nabla_{\nu})\chi,
\nonumber\\
g^{\mu\nu}\delta G_{\mu\nu}&=&(D-2)(D-1)\nabla_{\alpha}\nabla^{\alpha}\chi.
\label{3.10}
\end{eqnarray}
%
As written, we confirm that $\delta G_{\mu\nu}$ indeed depends upon only $F_{\mu\nu}$ and $\chi$, with one readily being able to check that $\delta G_{\mu\nu}$ obeys conservation $\nabla^{\nu}\delta G_{\mu\nu}=0$. Inspection of the components of $\delta G_{\mu\nu}$ reveals that only infer that from $g^{\mu\nu}\delta G_{\mu\nu}$ that $\nabla_{\alpha}\nabla^{\alpha}\chi$ is alone gauge invariant. However, by applying $\nabla_{\alpha}\nabla^{\alpha}$ to the $\delta G_{\mu\nu}$ equation we can then additionally deduce that $\nabla_{\alpha}\nabla^{\alpha}\nabla_{\beta}\nabla^{\beta}F_{\mu\nu}$ is gauge invariant. Importantly, we note that $F_{\mu\nu}$ nor $\nabla_\alpha\nabla^\alpha F_{\mu\nu}$ may be determined to gauge invariant without consideration issues related to its asymptotic behavior.  We will continue a discussion of the role of gauge invariance and boundedness in the following section. In addition, further detail of the SVTD construction, formalism, and gauge invariance is discussed within Appendex \ref{aa:svt_projection}. 

Demonstrated in \eqref{3.10}, we also observe that when written in the SVT4 basis the fluctuation equations take a considerably simpler form than when written according to the SVT3 basis (a result that will continue to carry over when we study the SVT4 fluctuation equations in curved background within Ch. \ref{c:construction_and_solution_of_svt}). Hence, by implementing the covariant formalism we have effectively replaced the six gauge-invariant combinations $\psi$, $E_{ij}$,  $\phi+\dot{B}-\ddot{E}$ and $B_i-\dot{E}_i$ of SVT3 by the six gauge-invariant combinations $F_{\mu\nu}$ and $\chi$ of SVT4. Consequently, the SVTD approach has provided a more compact set of gauge-invariant combinations and equations within $\delta G_{\mu\nu}$ and will prove to be simpler to solve, hence justfying the beneficial utility of matching the fundamental transformation group associated with GR.

For geometries more general than flat spacetime, since the gauge-invariant equations must contain a total of six gauge-invariant degrees of freedom, they must be comprised of the five-component $F_{\mu\nu}$ and some combination of the five other components that appear in (\ref{3.9}).


%%%%%%%%%%%%%%%%%%%%%%%%%%%%%%%%%%%%%
\section{Relating SVT3 to SVT4}
\label{s:relating_svt3_to_svt4}
%%%%%%%%%%%%%%%%%%%%%%%%%%%%%%%%%%%%%

Thus far we have presented both and SVT3 and SVT4 decomposition of the metric fluctuation $h_{\mu\nu}$. In the SVT3 development, we have determined the individual SVT3 components in terms of $h_{\mu\nu}$ in both integral form \eqref{AP66} and in terms of higher order derivative relations \eqref{2.6}. In addition, we have determined the SVT3 gauge invariants, using the gauge invariant $\delta G_{\mu\nu}$ as reference. Likewise, we have repeated the analysis for SVT4. Given that both formalisms present valid decompositions with their structure only differing the dimensionality of the underlying space-time slicing, it remains to show the relationship between SVT3 and SVT4, which we develop here.

For fluctuations around a flat background, we have seen that the SVT4 transverse traceless tensor $F_{\mu\nu}$ is comprised of five independent components. Counting the degrees of freedom, to compose the $F_{\mu\nu}$ in terms of SVT3 quantities, we must include the two-component transverse-traceless three-space rank two tensor, a two-component transverse three-space vector and a one-component three-space scalar. In addition, we have determined that if $F_{\mu\nu}$ is asymptotically well behaved to thus permit integration by parts, $F_{\mu\nu}$ itself is then gauge invariant and thus its associated combinations of SVT3 quantities must be gauge invariant. Such a requirement demands that the SVT3 vector component of $F_{\mu\nu}$ must be proportional to $B_i - \dot E_i$ (with $E_{ij}$ being the only SVT3 tensor, we have already account for such). To lastly pin down the last remaining gauge invariant associated with the SVT3 scalar presents a difficulty, as we would appear to not have any means to signify whether it is $\psi$ or $\phi+\dot B - \ddot E$ that should comprise $F_{\mu\nu}$. One may try to use the relation for $\chi$ with $\delta G_{\mu\nu}$ to fix the remaining degree of freedom, but comparing the SVT3 and SVT4 expansions of $\delta G_{\mu\nu}$ as given in (\ref{2.3}) and (\ref{3.10})  does in fact not enable us to uniquely specify the needed scalar combination. 

Consequently, we thus need to construct the fluctuation equations associated with a pure metric gravitational tensor other than the Einstein $\delta G_{\mu\nu}$, with the requirement that such a tensor also be independently gauge invariant in the flat background. It would also be beneficial if such a gravitational tensor were able to form a separation between $F_{\mu\nu}$ and $\chi$. Earlier we have determined that the perturbed curvature tensors themselves are gauge invariant, and thus we make appeal to any remaining curvature tensors not yet considered. Specifically, we have found such a tensor that meets the requisite properties, namely the Bach tensor associated with conformal gravity. As a pure metric tensor obtained by variation of the conformal gravity action of \eqref{AP1}, we restate its form here \cite{mannheim_2006}
%
\begin{eqnarray}
\delta W_{\mu\nu}&=&\frac{1}{2}(\eta^{\rho}_{\phantom{\rho} \mu} \partial^{\alpha}\partial_{\alpha}-\partial^{\rho}\partial_{\mu})
(\eta^{\sigma}_{\phantom{\sigma} \nu} \partial^{\beta}\partial_{\beta}-
\partial^{\sigma}\partial_{\nu})K_{\rho \sigma}
\nonumber\\
&&- \frac{1}{6}(\eta_{\mu \nu} \partial^{\gamma}\partial_{\gamma}-
\partial_{\mu}\partial_{\nu})(\eta^{\rho \sigma} \partial^{\delta}\partial_{\delta}-
\partial^{\rho}\partial^{\sigma})K_{\rho\sigma},
\label{4.5}
\end{eqnarray}
%
With $W_{\mu\nu}^{(0)}$ vanishing in conformal to flat background, and recalling that $\delta W_{\mu\nu}$ is traceless, the perturbed Bach tensor comprises five independent gauge invariant components. In the SVT4 basis, it must then solely be a function of the five component gauge invariant $F_{\mu\nu}$. Naturally, we proceed to evaluate \eqref{4.5} in the SVT3 basis to determine its form, given as
%
\begin{eqnarray}
\delta W_{00}  &=& -\frac{2}{3} \delta^{mn}\delta^{\ell k}\tilde{\nabla}_m\tilde{\nabla}_n\tilde{\nabla}_{\ell}\tilde{\nabla}_k (\phi + \psi +\dot{B}-\ddot{E}),
\nonumber\\	
\delta W_{0i} &=&  -\frac{2}{3} \delta^{mn}\tilde{\nabla}_i\tilde{\nabla}_m\tilde{\nabla}_n\partial_0(\phi +\psi +\dot{B}-\ddot{E})
+\frac{1}{2}\bigg[\delta^{mn}\delta^{\ell k}\tilde{\nabla}_m\tilde{\nabla}_n\tilde{\nabla}_{\ell}\tilde{\nabla}_k(B_i - \dot{E}_i)
\nonumber\\
&& -  \delta^{\ell k}\tilde{\nabla}_{\ell}\tilde{\nabla}_k \partial_0^2(B_i - \dot{E}_i)\bigg],
\nonumber\\	
\delta W_{ij}  &=& \frac{1}{3}\bigg{[} \delta_{ij}\delta^{\ell k}\tilde{\nabla}_{\ell}\tilde{\nabla}_k  \partial_0^2(\phi+ \psi+\dot{B}-\ddot{E}) + \delta^{\ell k}\tilde{\nabla}_{\ell}\tilde{\nabla}_k \tilde{\nabla}_i\tilde{\nabla}_j (\phi + \psi +\dot{B}-\ddot{E}) 
\nonumber\\
&&- \delta_{ij} \delta^{mn}\delta^{\ell k}\tilde{\nabla}_m\tilde{\nabla}_n\tilde{\nabla}_{\ell}\tilde{\nabla}_k(\phi + \psi +\dot{B}-\ddot{E}) -3\tilde{\nabla}_i\tilde{\nabla}_j \partial_0^2(\phi + \psi +\dot{B}-\ddot{E})\bigg{] }
\nonumber\\
&&+\frac{1}{2}\bigg[ \delta^{\ell k}\tilde{\nabla}_{\ell}\tilde{\nabla}_k \tilde{\nabla}_i   \partial_0(B_j - \dot{E}_j)+ \delta^{\ell k}\tilde{\nabla}_{\ell}\tilde{\nabla}_k \tilde{\nabla}_j \partial_0(B_i - \dot{E}_i) - \tilde{\nabla}_i\partial_0^3(B_j - \dot{E}_j)
\nonumber\\
&&-\tilde{\nabla}_j\partial_0^3(B_i - \dot{E}_i)\bigg] +\left[\delta^{mn}\tilde{\nabla}_m\tilde{\nabla}_n-\partial_0^2\right]^2E_{ij},
\label{4.6}
\end{eqnarray}
%
where $K_{\mu\nu}=h_{\mu\nu}-(1/4)g_{\mu\nu}h$. Likewise,  evaluating (\ref{4.5}) in the SVT4 basis given in (\ref{3.9}) yields 
%
\begin{eqnarray}
\delta W_{\mu\nu}=\nabla_{\alpha}\nabla^{\alpha}\nabla_{\beta}\nabla^{\beta}F_{\mu\nu}.
\label{4.7}
\end{eqnarray}
%
As an aside, we note the extremely compact and simple structure of the SVT4 fluctuation equation in conformal gravity, with its exact solution being readily obtainable (particularly compared to $\delta G_{\mu\nu}$ within \eqref{3.10}).

Inspection of the SVT3 structure reveals that $\phi + \psi +\dot{B}-\ddot{E}$ is to unambiguously serve as the three-dimensional scalar piece of $F_{\mu\nu}$. In addition, from (\ref{3.10}) we can identify $\chi$ according to $3\nabla_{\alpha}\nabla^{\alpha}\chi=-\delta^{ij}\tilde{\nabla}_i\tilde{\nabla}_j(\phi  +\psi +\dot{B}-\ddot{E})+3\delta^{ij}\tilde{\nabla}_{i}\tilde{\nabla}_{j}\psi-3\ddot{\psi}$. Thus we have determined the underlying relationship between SVT3 and SVT4, with fluctuations around flat spacetime quantity $F_{\mu\nu}$ necessarily containing $E_{ij}$, $B_i-\dot{E}_i$ and $\phi + \psi +\dot{B}-\ddot{E}$. 

%%%%%%%%%%%%%%%%%%%%%%%%%%%%%%%%%%%%%
\section{Decomposition Theorem and Boundary Conditions}
\label{s:decomposition_theorem}
%%%%%%%%%%%%%%%%%%%%%%%%%%%%%%%%%%%%%

In attempts to solve the cosmological fluctuation equations that have been presented in the literature, appeal is commonly made to the decomposition theorem. The theorem entails the assertation that within the fluctuation equations \emph{themselves} the scalar, vector, and tensor sectors decouple and evolve independent. As mentioned in the introduction, the investigation of the decomposition theorem forms a core component of this work. We proceed by inspecting the interplay of asymptotic behavior and boundary conditions in establishing the validity of theorem below, as applied to both SVT3 and SVT4, taking a flat background and reduced fluctuation equation forms as illustrative examples. Within Ch. \ref{c:construction_and_solution_of_svt}, we touch basis again with the decomposition theorem as applied to specific cosmological geometries in order to determine the underlying assumptions that need to hold for such a theory to be valid.

%%%%%%%%%%%%%%%%%%%%%%%%%%%%%%%%%%%%%
\subsection{SVT3}
%%%%%%%%%%%%%%%%%%%%%%%%%%%%%%%%%%%%%
%
We begin with a schematic example of an SVT3 representation of two vector fields which have been decomposed into their traverse vector and longitudinal scalar components
\begin{eqnarray}
B_i+\partial_iB=C_i+\partial_iC,
\label{1.3}
\end{eqnarray}
%
where the $B$ and $B_i$ are given by (\ref{2.1}), obeying $\partial^i B_i = 0$ and where the $C$ and $C_i$ are to represent functions given by the evolution equations with $C_i$ also obeying  $\partial_iC^i=0$. Now, according to the decomposition theorem, the vectors $B_i$ and $C_i$ are taken to decouple and evolve independent of the scalars, with an analogous statement holding for the scalars. Thus, we are to take
%
\begin{eqnarray}
B_i= C_i,\quad \partial_iB=\partial_iC.
\label{1.4}
\end{eqnarray}
%
It is clear, however, that in such an action, (\ref{1.4}) does not follow from (\ref{1.3}). For if we apply $\partial^i$ and $\epsilon^{ijk}\partial_j$  to (\ref{1.3}) we obtain 
%
\begin{eqnarray}
\partial^i\partial_i(B-C)=0,\quad \epsilon^{ijk}\partial_j(B_k-C_k)=0,
\label{1.5}
\end{eqnarray}
%
and from this we can only conclude that $B$ and $C$ are defined up to an arbitrary scalar $D$ obeying $\partial^i\partial_iD=0$. For the vector sector analogously, $B_k$ and $C_k$ can only differ by any function $D_k$ that obeys $\epsilon^{ijk}\partial_jD_k=0$, i.e. an irrotational field $D_k$ expressed as the gradient of a scalar. In composing (\ref{1.5}) we have appropriately separated the scalar and vector components within(\ref{1.3}), obtaining a decomposition for the components that does not follow by proceeding from (\ref{1.5}) to (\ref{1.4}). Specifically, without providing some additional information or constraints, one cannot directly proceed from (\ref{1.5}) to (\ref{1.4}). However, given the imposition of such constraints, such a decomposition may be possible, with the constraints in fact needing to be in form of spatially asymptotic boundary conditions. Alternatively, if one were to elect to form a decomposition without the imposition of boundary condition or constraints, we see from \eqref{1.5} that we necessarily need to go to higher derivatives in order to establish the decomposition. Thus we will continue to investigate both branches of applying said decomposition theorem.

Continuing with the example of (\ref{1.5}), we first proceed from (\ref{1.5}) to (\ref{1.4}) by imposing asymptotic boundary conditions. As we have done within \eqref{2.10}, we take a complete basis of three-dimensional plane waves (serving as a complete basis for $\partial_i\partial^i$) and we express the general solution for $B-C=D$ in the form
%
\begin{eqnarray}
D=\sum _{\bf k}a_{\bf k}e^{i\textbf{k}\cdot \textbf{x}},
\label{1.6}
\end{eqnarray}
%
where the $a_{\bf k}$ are constrained to obey 
%
\begin{eqnarray}
{\bf k}^2a_{\bf k}=0.
\label{1.7}
\end{eqnarray}
%
Inspecting (\ref{1.7}) itself carefully, we observe that (\ref{1.7}) does not lead to $a_{\bf k}=0$ directly. In fact since $k^2\delta(k)=0$, $k^2\delta(k)/k=0$ we can take other forms for the $a_{\mathbf k}$, with the general form of
%
\begin{eqnarray}
a_{\bf k}&=&\alpha_k\delta(k_x)\delta(k_y)\delta(k_z)
\nonumber\\
&&+\beta_k\left[\frac{\delta (k_x)\delta (k_y)\delta (k_z)}{k_x}+\frac{\delta (k_x)\delta (k_y)\delta (k_z)}{k_y}+\frac{\delta (k_x)\delta (k_y)\delta (k_z)}{k_z}\right],
\label{1.8}
\end{eqnarray}
%
with constants $\alpha_k$ and $\beta_k$. Substituting $\alpha_k$ as given into (\ref{1.6}) would serve to eliminate the ${\bf x}$ dependence from $D$ and consequently provide a constant $D$ that does not vanish at spatial infinity. Alternatively, if we substitute the $\beta_k$ term  into (\ref{1.6}) we would construct a scalar $D$ that grows linearly in ${\bf x}$, to therefore also not vanish at spatial infinity. Hence by imposing asymptotic conditions, we can exclude solutions containing a non-zero $D$ obeying $\partial_i\partial^iD=0$, to thus yield the requisite decomposition theorem. Consequently, we have connected validity of the decomposition theorem as being hinged upon the very existence of the SVT3 basis in the first place as both require asymptotic boundary conditions.

We can further our understanding the role of boundary conditions by inspecting the behavior of $\partial_i\partial^iD=0$ in coordinate space. Recalling the Green's identity
%
\begin{eqnarray}
A \partial_i\partial^iB-B \partial_i\partial^iA=\partial_i(A\partial^iB-B\partial^iA),
\label{1.9}
\end{eqnarray}
%
we take a general scalar $A$ to obey $\partial_i\partial^iA=0$ and take $B$ to be the Green's function $D^{(3)}(\mathbf{x}-\mathbf{y})$ which obeys
%
\begin{eqnarray}
\partial_i\partial^iD^{(3)}(\mathbf{x}-\mathbf{y})=\delta^3(\mathbf{x}-\mathbf{y}).
\label{1.10}
\end{eqnarray}
%
As a result, we can now express $A$ as an asymptotic surface term of the form 
%
\begin{eqnarray}
A({\bf x}) =\int dS_y^i\left[A({\bf y})\partial^y_iD^{(3)}(\mathbf{x}-\mathbf{y})-D^{(3)}(\mathbf{x}-\mathbf{y})\partial^y_iA({\bf y})\right],
\label{1.11}
\end{eqnarray}
%
with $dS$ representing the integration over a closed surface $S$. If the asymptotic surface term vanishes, then $A$ will also vanish identically. Hence we arrive at two non-trivial solutions to $\partial_i\partial^iA=0$, a) with $A$ constant or b) of the form ${\bf n}\cdot {\bf x}$ with ${\bf n}$ a normal vector. However, both of these must be explicitly excluded if they are to be well behaved asymptotically. Hence, requiring that the asymptotic surface term in (\ref{1.11}) vanish consequently forces the remaining solution to $\partial_i\partial^iA=0$ to be $A=0$.

Using a polar coordinate basis, the formulation is adapted by using a polar (flat) metric $\gamma_{ij}$ and replacing (\ref{1.11}) by
%
\begin{eqnarray}
A(\textbf{x})=\int dS\left[A(\mathbf{y})\frac{\partial D^{(3)}(\mathbf{x},\mathbf{y})}{\partial  n} -D^{(3)}(\mathbf{x},\mathbf{y})\frac{\partial A(\mathbf{y})}{\partial n}\right],
\label{1.12a}
\end{eqnarray}
%
with $\partial/\partial n$ is the normal derivative to the surface S, and with the Green's function obeying
%
\begin{eqnarray}
\nabla_i\nabla^iD^{(3)}(\mathbf{x},\mathbf{y})=\gamma^{-1/2}\delta^3(\mathbf{x}-\mathbf{y}).
\label{1.13a}
\end{eqnarray}
%
Taking $D^{(3)}(\mathbf{x},\mathbf{y})=-1/4\pi|\mathbf{x}-\mathbf{y}|$, the surface integral then becomes
%
\begin{eqnarray}
A(\textbf{x})=\frac{1}{4\pi} \int dS\left[\frac{1}{|\mathbf{x}-\mathbf{y}|}\frac{\partial A(\mathbf{y})}{\partial n}-
A(\mathbf{y})\frac{\partial}{\partial  n}\frac{1}{|\mathbf{x}-\mathbf{y}|}\right].
\label{1.14a}
\end{eqnarray}
%
As written, the asymptotic surface term will vanish conditioned on $A(\mathbf{y})$ behaving as $1/r^{n}$ for $n>0$. 

With asympyotic conditions having been shown to yield the decomposition theorem, we now focus the discussion on the higher derivative relations \footnote{As with (\ref{2.6}) we note that we need to go to fourth-order derivatives to establish decomposition.} necessary for said decomposition. Specifically, we apply sequences of derivatives to the flat space fluctuation of the Einstein tensor (which we recall is gauge invariant entirely on its own in this background), to obtain
%
\begin{eqnarray}
0&=&\delta^{ab} \tilde{\nabla}_{b}\tilde{\nabla}_{a}\psi,
\nonumber\\
0&=&\delta^{ab} \tilde{\nabla}_{b}\tilde{\nabla}_{a} \delta^{cd} \tilde{\nabla}_{c}\tilde{\nabla}_{d}(\phi+\dot{B}  -\ddot{E}),
\nonumber\\
0&=&\delta^{ab} \tilde{\nabla}_{b}\tilde{\nabla}_{a} \delta^{cd} \tilde{\nabla}_{c}\tilde{\nabla}_{d}(B_i-\dot{E}_i),
\nonumber\\
0&=&\delta^{ab} \tilde{\nabla}_{b}\tilde{\nabla}_{a} \delta^{cd} \tilde{\nabla}_{c}\tilde{\nabla}_{d}(-\ddot{E}_{ij}+\delta^{ef} \tilde{\nabla}_{e}\tilde{\nabla}_{f}E_{ij}),
\label{2.19}
\end{eqnarray}
%
Inspection of \eqref{2.3} shows that a decomposition theorem requires
%
\begin{eqnarray}
0&=&- 2 \delta^{ab} \tilde{\nabla}_{b}\tilde{\nabla}_{a}\psi,
\nonumber\\
0&=&- 2 \tilde{\nabla}_{i}\dot{\psi},
\nonumber\\
0&=&\tfrac{1}{2} \delta^{ab} \tilde{\nabla}_{b}\tilde{\nabla}_{a}(B_{i} -  \dot{E}_{i}),
\nonumber\\
0&=&-2 \delta_{ij} \ddot{\psi} -  \delta^{ab} \delta_{ij} \tilde{\nabla}_{b}\tilde{\nabla}_{a}(\phi+\dot{B}  -\ddot{E})+ \delta^{ab} \delta_{ij} \tilde{\nabla}_{b}\tilde{\nabla}_{a}\psi 
\nonumber\\
&& +\tilde{\nabla}_{j}\tilde{\nabla}_{i}(\phi+\dot{B} -  \ddot{E}) - \tilde{\nabla}_{j}\tilde{\nabla}_{i}\psi,
\nonumber\\
0&=&\tfrac{1}{2} \tilde{\nabla}_{i}(\dot{B}_{j} - \ddot{E}_{j}) + \tfrac{1}{2} \tilde{\nabla}_{j}(\dot{B}_{i} 
- \ddot{E}_{i}),
\nonumber\\
0&=&- \ddot{E}_{ij} + \delta^{ab} \tilde{\nabla}_{b}\tilde{\nabla}_{a}E_{ij}.
\label{2.20}
\end{eqnarray}
%
Within \eqref{2.20}, for any SVT quantity $D$ that obeys  
\begin{eqnarray}
\delta^{ab} \tilde{\nabla}_{a}\tilde{\nabla}_{b}D=0\qquad \rm{or} \qquad \delta^{ab} \tilde{\nabla}_{a}\tilde{\nabla}_{b}\delta^{cd} \tilde{\nabla}_{c}\tilde{\nabla}_{d}D=0
\label{hdconts}
\end{eqnarray} if impose spatial boundary conditions such that $D$ or $\delta^{ab} \tilde{\nabla}_{a}\tilde{\nabla}_{b}D$ vanishes,  the decomposition theorem will then follow for the fluctuation $\delta G_{\mu\nu}$. Thus, while a decoupling of the SVT representations can be achieved by going to higher derivatives, it is still necessary to constrain the behavior of such quantities according to \eqref{hdconts} in order to recover the decomposition theorem. 

%%%%%%%%%%%%%%%%%%%%%%%%%%%%%%%%%%%%%
\subsection{SVT4}
%%%%%%%%%%%%%%%%%%%%%%%%%%%%%%%%%%%%%

Similar to our treatment of the SVT3 decomposition theorem, we now investigation the status of the theorem in the SVT4 basis. We being with the four-dimensional analog of (\ref{1.3}): 
%
\begin{eqnarray}
F_{\mu}+\partial_{\mu}F=C_{\mu}+\partial_{\mu}C,
\label{1.34a}
\end{eqnarray}
%
where the $F$ and $F_{\mu}$ are given by (\ref{3.8}), and where the $C$ and $C_{\mu}$ are representative of the fluctuation equations, with $C_{\mu}$ obeying  $\partial_{\mu}C^{\mu}=0$. For the decomposition theorem to hold one must take
%
\begin{eqnarray}
F_{\mu}= C_{\mu},\quad \partial_{\mu}F=\partial_{\mu}C.
\label{1.35a}
\end{eqnarray}
%
Just as with the SVT3 case, (\ref{1.35a}) does not follow from (\ref{1.34a}), since on applying $\partial_{\mu}$  and $\epsilon_{\mu\nu\sigma\tau}n^{\nu}\partial^{\sigma}$ we obtain
%
\begin{eqnarray}
\partial_{\mu}\partial^{\mu}(F-C)=0,\quad \epsilon_{\mu\nu\sigma\tau}n^{\nu}\partial^{\sigma}(F^{\tau}-C^{\tau})=0.
\label{1.36a}
\end{eqnarray}
%
By applying derivatives, we have indeed successfully decomposed the components. However, such is determined only up to a function  $D=F-C$, which need only to obey $\partial_\mu\partial^\mu D = 0$. We proceed to use a complete basis of plane waves to represent $D$ (now four-dimensional plane waves, covering the $\partial_{\mu}\partial^{\mu}$ operator), to obtain
%
\begin{eqnarray}
D=\sum _{\bf k}a_{\bf k}e^{i\textbf{k}\cdot \textbf{x}-ikt},
\label{1.37a}
\end{eqnarray}
%
where $k=|{\bf k}|$. Importantly, in contrast to the $a_{\bf k}$ in (\ref{1.6}) which obey $k_ik^ia_{\bf k}=0$, here we have no such constraint on $a_{\bf k}$, as the $a_{\bf k}$ obey $k_{\mu}k^{\mu}a_{\bf k}=0$ where $k_{\mu}k^{\mu}={\bf k}^2-k^2$ is identically zero. In addition we can continue retaining the form of $\partial_{\mu}\partial^{\mu}D=0$ by taking $a_{\bf k}=\exp(-a^2{\bf k}\cdot{\bf k})$. Upon taking the real part of $D$, we obtain
%
\begin{eqnarray}
{\rm Re}[D]&=&\rm Re\left[\int \frac{d^3k}{(2\pi)^3}e^{-a^2k^2+i\textbf{k}\cdot \textbf{x}-ikt}\right]
\nonumber\\
&=&
\frac{1}{16\pi^{3/2}a^3}\left[\frac{r+t}{r}e^{-(r+t)^2/4a^2}+\frac{r-t}{r}e^{-(r-t)^2/4a^2}\right].
\label{1.38a}
\end{eqnarray}
%
We see that \eqref{1.38a} is in fact not constrained by a spatially asymptotic condition since 
as $r\rightarrow \infty$, ${\rm Re}[D]$ falls off as $\exp(-r^2)$, both for fixed $t$ and for points on the light cone where $r=\pm t$. In fact, such a $D$ is even being well-behaved at $r=0$. Thus because of the metric signature (i.e. opposing sign of space and time coordinates), the quantity $k_\mu k^\mu=0$ leads to a form of $D$ that cannot generally satisfy the decomposition theorem, even in presence of the constraint $\partial_\mu \partial^\mu D = 0$. The investigation of whether a decomposition theorem can be satisfied at all must be performed on a case by case basis according to the differing background geometries, of which we explore within Ch. \ref{c:construction_and_solution_of_svt}.


\chapter{Construction and Solution of SVT Fluctuation Equations}
\label{c:construction_and_solution_of_svt}

%%%%%%%%%%%%%%%%%%%%%%%%%%%%%%%%%%%%%%%%%%%%
\section{SVT3}
\label{s:svt3_construction}
%%%%%%%%%%%%%%%%%%%%%%%%%%%%%%%%%%%%%%%%%%%%

%%%%%%%%%%%%%%%%%%%%%%%%%%%%%%%%%%%%%%%%%%%%
\subsection{$dS_4$}
\label{ss:ds4_svt3}
%%%%%%%%%%%%%%%%%%%%%%%%%%%%%%%%%%%%%%%%%%%%
In the SVT3 formulation within a de Sitter background, the background and fluctuation metric can be expressed in the conformal to flat form
%
\begin{eqnarray}
ds^2 &=&\frac{1}{H^2\tau^2}\bigg{[}(1+2\phi) d\tau^2 -2(\tilde{\nabla}_i B +B_i)d\tau dx^i - [(1-2\psi)\delta_{ij} +2\tilde{\nabla}_i\tilde{\nabla}_j E 
\nonumber\\
&&+ \tilde{\nabla}_i E_j + \tilde{\nabla}_j E_i + 2E_{ij}]dx^i dx^j\bigg{]},
\label{7.1}
\end{eqnarray}
%
where the $\tilde{\nabla}_{i}$ denote derivatives with respect to the flat 3-space $\delta_{ij}dx^idx^j$ metric.
In terms of the SVT3 form for the fluctuations the components of the perturbed $\delta G_{\mu\nu}$ are given by \cite{amarasinghe_2019}
%
\begin{eqnarray}
\delta G_{00}&=&-\frac{6}{\tau}\dot{\psi}-\frac{2}{\tau}\tilde{\nabla}^2(\tau \psi +B-\dot{E}),
\nonumber\\
\delta G_{0i}&=&\frac{1}{2}\tilde{\nabla}^2(B_i-\dot{E}_i)+\frac{1}{\tau^2}\tilde{\nabla}_i(3B-2\tau^2\dot{\psi}+2\tau \phi)+\frac{3}{\tau^2}B_i,
\nonumber\\
\delta G_{ij}&=&\frac{\delta_{ij}}{\tau^2}\bigg[-2\tau^2\ddot{\psi}+2\tau\dot{\phi}+4\tau\dot{\psi}-6\phi-6\psi
\nonumber\\
&&+\tilde{\nabla}^2\left(2\tau B-\tau^2\dot{B}+\tau^2\ddot{E}-2\tau\dot{E}-\tau^2\phi+\tau^2\psi\right)\bigg]
\nonumber\\
&&+\frac{1}{\tau^2}\tilde{\nabla}_i\tilde{\nabla}_j\left[-2\tau B +\tau^2\dot{B}-\tau^2\ddot{E}+2\tau\dot{E}+6E+\tau^2\phi-\tau^2\psi\right]
\nonumber\\
&&+\frac{1}{2\tau^2}\tilde{\nabla}_i\left[-2\tau B_j+2\tau\dot{E}_j+\tau^2\dot{B}_j-\tau^2\ddot{E}_j+6E_j\right]
\nonumber\\
&&+\frac{1}{2\tau^2}\tilde{\nabla}_j\left[-2\tau B_i+2\tau\dot{E}_i+\tau^2\dot{B}_i-\tau^2\ddot{E}_i+6E_i\right]
\nonumber\\
&&-\ddot{E}_{ij}+\frac{6}{\tau^2}E_{ij}+\frac{2}{\tau}\dot{E}_{ij}+\tilde{\nabla}^2E_{ij},
\label{7.2}
\end{eqnarray}
%
where the dot denotes the derivative with respect to the conformal time $\tau$ and $\tilde{\nabla}^2=\delta^{ij}\tilde{\nabla}_i\tilde{\nabla}_j$. 
For the de Sitter SVT3 metric the gauge-invariant metric combinations are (see e.g. \cite{amarasinghe_2019})
%
\begin{eqnarray}
\alpha=\phi+\psi+\dot{B}-\ddot{E} ,\quad \beta=\tau\psi+B-\dot{E}, \quad B_i-\dot{E}_i,\quad E_{ij}.
\label{7.3}
\end{eqnarray}
%
(For a generic SVT3 metric with a general conformal factor $\Omega(\tau)$ the quantity $-(\Omega/\dot{\Omega})\psi+B-\dot{E}$ is gauge invariant, to thus become $\beta$ when $\Omega(\tau)=1/H\tau$, with the other gauge invariants being independent of $\Omega(\tau)$.)
In terms of the gauge-invariant combinations the fluctuation equations $\Delta_{\mu\nu}=\delta G_{\mu\nu}+\delta T_{\mu\nu}=0$ take the form
%
\begin{eqnarray}
\Delta_{00}&=&-\frac{6}{\tau^2}(\dot{\beta}-\alpha)-\frac{2}{\tau}\tilde{\nabla}^2\beta=0,
\label{7.4}
\end{eqnarray}
%
\begin{eqnarray}
\Delta_{0i}&=&\frac{1}{2}\tilde{\nabla}^2(B_i-\dot{E}_i)-\frac{2}{\tau}\tilde{\nabla}_i(\dot{\beta}-\alpha)=0,
\label{7.5}
\end{eqnarray}
%
\begin{eqnarray}
\Delta_{ij}&=&\frac{\delta_{ij}}{\tau^2}\left[-2\tau(\ddot{\beta}-\dot{\alpha})+6(\dot{\beta}-\alpha)+\tau \tilde{\nabla}^2(2\beta-\tau \alpha)\right]
+\frac{1}{\tau}\tilde{\nabla}_i\tilde{\nabla}_j(-2 \beta +\tau\alpha)
\nonumber\\
&+&\frac{1}{2\tau}\tilde{\nabla}_i[-2(B_j-\dot{E}_j)+\tau(\dot{B}_j-\ddot{E}_j)]
+\frac{1}{2\tau}\tilde{\nabla}_j[-2(B_i-\dot{E}_i)+\tau(\dot{B}_i-\ddot{E}_i)]
\nonumber\\
&-&\ddot{E}_{ij}+\frac{2}{\tau}\dot{E}_{ij}+\tilde{\nabla}^2E_{ij}=0,
\label{7.6}
\end{eqnarray}
%
\begin{eqnarray}
g^{\mu\nu}\Delta_{\mu\nu}&=&H^2[-6\tau(\ddot{\beta}-\dot{\alpha})+24(\dot{\beta}-\alpha)
+6\tau \tilde{\nabla}^2\beta-2\tau^2\tilde{\nabla}^2\alpha]=0,
\label{7.7}
\end{eqnarray}
%
to thus be manifestly gauge invariant.

If there is to be a decomposition theorem the S, V and T components of $\Delta_{\mu\nu}$ will satisfy $\Delta_{\mu\nu}=0$ independently, to thus be required to obey
%
\begin{eqnarray}
&&-\frac{6}{\tau^2}(\dot{\beta}-\alpha)-\frac{2}{\tau}\tilde{\nabla}^2\beta=0,\quad \frac{1}{2}\tilde{\nabla}^2(B_i-\dot{E}_i)=0, \quad \frac{2}{\tau}\tilde{\nabla}_i(\dot{\beta}-\alpha)=0,
\nonumber\\
&&\frac{\delta_{ij}}{\tau^2}\left[-2\tau(\ddot{\beta}-\dot{\alpha})+6(\dot{\beta}-\alpha)+\tau \tilde{\nabla}^2(2\beta-\tau\alpha)\right]+ \frac{1}{\tau^2}\tilde{\nabla}_i\tilde{\nabla}_j(-2\tau \beta +\tau^2\alpha)=0,
\nonumber\\
&&\frac{1}{2\tau^2}\tilde{\nabla}_i[-2\tau (B_j-\dot{E}_j)+\tau^2(\dot{B}_j-\ddot{E}_j)]
\nonumber\\
&&\qquad\qquad\qquad\qquad+\frac{1}{2\tau^2}\tilde{\nabla}_j[-2\tau (B_i-\dot{E}_i)+\tau^2(\dot{B}_i-\ddot{E}_i)]=0,
\nonumber\\
&&-\ddot{E}_{ij}+\frac{2}{\tau}\dot{E}_{ij}+\tilde{\nabla}^2E_{ij}=0.
\label{7.8}
\end{eqnarray}
%

To determine whether or not these conditions might hold we need to solve the fluctuation equations $\Delta_{\mu\nu}=0$ directly, to see what the structure of the solutions might look like.  To this end we first apply $\tau\partial_{\tau}-1$ to $-\tau^2\Delta_{00}/2$,  to obtain
%
\begin{eqnarray}
\tau^2\tilde{\nabla}^2\dot{\beta}+3\tau(\ddot{\beta}-\dot{\alpha})-3(\dot{\beta}-\alpha)=0,
\label{7.9}
\end{eqnarray}
%
and then add $3\tau^2\Delta_{00}$ to  $g^{\mu\nu}\Delta_{\mu\nu}/H^2$ to obtain
%
\begin{eqnarray}
\tau^2\tilde{\nabla}^2\alpha+3\tau(\ddot{\beta}-\dot{\alpha})-3(\dot{\beta}-\alpha)=0.
\label{7.10}
\end{eqnarray}
%
Combining these equations and using $\Delta_{00}=0$ we thus obtain
%
\begin{eqnarray}
\tilde{\nabla}^2(\alpha-\dot{\beta})=0,\quad \tilde{\nabla}^2\beta=0,
\label{7.11}
\end{eqnarray}
%
and 
%
\begin{eqnarray}
\tau^2\tilde{\nabla}^2(\alpha+\dot{\beta})+6\tau(\ddot{\beta}-\dot{\alpha})-6(\dot{\beta}-\alpha)=0.
\label{7.12}
\end{eqnarray}
%
Applying $\tilde{\nabla}^2$ then gives
%
\begin{eqnarray}
\tilde{\nabla}^4(\alpha+\dot{\beta})=0,\quad \tilde{\nabla}^4(\alpha-\dot{\beta})=0.
\label{7.13}
\end{eqnarray}
%
Applying $\tilde{\nabla}^2$ to $\Delta_{0i}$ in turn then gives
%
\begin{eqnarray}
\tilde{\nabla}^4(B_i-\dot{E}_i)=0,
\label{7.14}
\end{eqnarray}
%
while applying $\epsilon^{ijk}\tilde{\nabla}_j$ to $\Delta_{0k}$ gives
%
\begin{eqnarray}
\frac{1}{2}\epsilon^{ijk}\tilde{\nabla}_j\tilde{\nabla}^2(B_k-\dot{E}_k)=0.
\label{7.15}
\end{eqnarray}
%
Finally, to obtain an equation that only involves $E_{ij}$ we apply $\tilde{\nabla}^4$ to $\Delta_{ij}$, to obtain
%
\begin{eqnarray}
\tilde{\nabla}^4\left(-\ddot{E}_{ij}+\frac{2}{\tau}\dot{E}_{ij}+\tilde{\nabla}^2E_{ij}\right)=0.
\label{7.16}
\end{eqnarray}
%
As we see, we can isolate all the individual S, V and T gauge-invariant combinations, to thus give decomposition for the individual SVT3 components. However, the relations we obtain look nothing like the relations that a decomposition theorem would require, and thus without some further input we do not obtain a decomposition theorem.

To provide some further input we impose some asymptotic boundary conditions. To this end  we recall from Sec. \ref{s:decomposition_theorem} that for any spatially asymptotically bounded function $A$ that obeys $\tilde{\nabla}^2A=0$, the only solution is $A=0$. If $A$ obeys $\tilde{\nabla}^4A=0$, we must first set $\tilde{\nabla}^2A=C$, so that $\tilde{\nabla}^2C=0$. Imposing boundary conditions for $C$ enables us to set $C$=0. In such a case we can then set $\tilde{\nabla}^2A=0$, and with sufficient asymptotic convergence can then set $A=0$. Now a function could obey $\tilde{\nabla}^2A=0$ trivially by being independent of the spatial coordinates altogether, and only depend on $\tau$. However, it then would not vanish at spatial infinity, and we can thus exclude this possibility. With such spatial convergence for all of the S, V and T components we can then set 
%
\begin{eqnarray}
\alpha=0,\quad \dot{\beta}=0,\quad \beta =0,\quad B_i-\dot{E}_i=0,\quad -\tau\ddot{E}_{ij}+2\dot{E}_{ij}+\tau\tilde{\nabla}^2E_{ij}=0.
\label{7.17}
\end{eqnarray}
%
Since this solution coincides with the solution that would be obtained to (\ref{7.8}) under the same boundary conditions, we see that under these asymptotic boundary conditions we have a decomposition theorem.


In this solution all components of the SVT3 decomposition vanish identically except the rank two tensor $E_{ij}$. Taking $E_{ij}$ to behave as $\epsilon_{ij}\tau^2f(\tau)g({\bf x})$ where $\epsilon_{ij}$ is a polarization tensor, we find that the solution obeys
%
\begin{eqnarray}
\frac{\tau^2 \ddot{f}+2\tau \dot{f}-2f}{\tau^2f}=\frac{\tilde{\nabla}^2g}{g}=-k^2,
\label{7.18}
\end{eqnarray}
%
where $k^2$ is a separation constant. Consequently $E_{ij}$ is given as
%
\begin{eqnarray}
E_{ij}=\epsilon_{ij}({\bf k})\tau^2[a_1({\bf k})j_1(k\tau)+b_1({\bf k})y_1(k\tau)]e^{i{\bf k}\cdot{\bf x}},
\label{7.19}
\end{eqnarray}
%
where ${\bf k}\cdot{\bf k}=k^2$, $j_1$ and $y_1$ are spherical Bessel functions, and $a_1({\bf k})$ and $b_1({\bf k})$ are spacetime independent constants. For $E_{ij}$ to obey the transverse and traceless conditions $\delta^{ij}E_{ij}=0$, $\tilde{\nabla}^jE_{ij}=0$ the polarization tensor $\epsilon_{ij}({\bf k})$ must obey $\delta^{ij}\epsilon_{ij}=0$, ${\bf k}^{j}\epsilon_{ij}({\bf k})=0$.
Then, by taking a family of separation constants we can form a transverse-traceless wave packet
%
\begin{eqnarray}
E_{ij}&=&\sum_{\bf k}\epsilon_{ij}({\bf k})\tau^2[a_1({\bf k})j_1(k\tau)+b_1({\bf k})y_1(k\tau)]e^{i{\bf k}\cdot{\bf x}}\nonumber\\
&=&
\sum_{\bf k}\epsilon_{ij}({\bf k})\bigg[a_1({\bf k})\left(\frac{\sin(k\tau)}{k^2}-\frac{\tau\cos(k\tau)}{k}\right)
\nonumber\\
&&+b_1({\bf k})\left(\frac{\cos(k\tau)}{k^2}+\frac{\tau\sin(k\tau)}{k}\right)\bigg],
\label{7.20}
\end{eqnarray}
%
and can choose the $a_1({\bf k})$ and $b_1({\bf k})$ coefficients to make the packet be as well-behaved at spatial infinity as desired. Finally, since according to (\ref{7.1}) the full fluctuation is given not by $E_{ij}$ but by $2E_{ij}/H^2\tau^2$, then with $\tau=e^{-Ht}/H$, through the $\cos(k\tau)/k^2$ term we find that at large comoving time  $E_{ij}/\tau^2$ behaves as $e^{2Ht}$, viz. the standard de Sitter fluctuation exponential growth.

%%%%%%%%%%%%%%%%%%%%%%%%%%%%%%%%%%%%%%%%%%%%
\subsection{Robertson Walker $k=0$ Radiation Era}
\label{ss:rw_k=0_radiation_svt3}
%%%%%%%%%%%%%%%%%%%%%%%%%%%%%%%%%%%%%%%%%%%%
\label{S8}
In comoving coordinates a spatially flat Robertson-Walker background metric takes the form $ds^2=dt^2-a^2(t)\delta_{ij}dx^idx^j$. In the radiation era where a perfect fluid pressure $p$ and energy density $\rho$ are related by $\rho=3p$, the background energy-momentum tensor is given by the traceless
%
\begin{eqnarray}
T_{\mu\nu}=p(4U_{\mu}U_{\nu}+g_{\mu\nu}),
\label{8.1}
\end{eqnarray}
%
where $g^{\mu\nu}U_{\mu}U_{\nu}=-1$, $U^{0}=1$, $U_0=-1$, $U^{i}=0$, $U_i=0$. With this source the background Einstein equations $G_{\mu\nu}=-T_{\mu\nu}$ with $8\pi G=1$ fix $a(t)$ to be $a(t)=t^{1/2}$. In conformal to flat coordinates we set $\tau=\int dt/t^{1/2}=2t^{1/2}$, with the conformal factor being given by $\Omega(\tau)=\tau/2$. In conformal to flat coordinates the background pressure is of the form $p=4/\tau^4$ while $U^{0}=2/\tau$, $U_0=-\tau/2$. In this coordinate system the SVT3 fluctuation metric as written with an explicit conformal factor is of the form 
%
\begin{eqnarray}
ds^2 &=&\frac{\tau^2}{4}\bigg{[}(1+2\phi) d\tau^2 -2(\tilde{\nabla}_i B +B_i)d\tau dx^i - [(1-2\psi)\delta_{ij} +2\tilde{\nabla}_i\tilde{\nabla}_j E
\nonumber\\
&& + \tilde{\nabla}_i E_j + \tilde{\nabla}_j E_i + 2E_{ij}]dx^i dx^j\bigg{]},
\label{8.2}
\end{eqnarray}
%
and the fluctuation energy-momentum tensor is of the form
%
\begin{eqnarray}
\delta T_{\mu\nu}=\delta p(4U_{\mu}U_{\nu}+g_{\mu\nu})+p(4\delta U_{\mu}U_{\nu}+4U_{\mu}\delta U_{\nu}+h_{\mu\nu}).
\label{8.3}
\end{eqnarray}
%
As written, we might initially expect there to be five fluctuation variables in the fluctuation energy-momentum tensor: $p$ and the four components of $\delta U_{\mu}$. However, varying  $g^{\mu\nu}U_{\mu}U_{\nu}=-1$ gives 
%
\begin{eqnarray}
\delta g^{00}U_{0}U_{0}+2g^{00}U_{0}\delta U_{0}=0,
\label{8.4}
\end{eqnarray}
%
i.e. 
%
\begin{eqnarray}
\delta U_{0}=-\frac{1}{2}(g^{00})^{-1}(-g^{00}g^{00}\delta g_{00})U_{0}=-\frac{\tau\phi}{2}.
\label{8.5}
\end{eqnarray}
%
Thus $\delta U_{0}$ is not an independent of the metric fluctuations, and we need the fluctuation equations to only fix six (viz. ten minus four) independent gauge-invariant metric fluctuations and the  four $\delta p$ and $\delta U_i$ (counting all components). With ten $\Delta_{\mu\nu}=0$ equations, we can nicely determine all of them. 


To this end we evaluate $\delta G_{\mu\nu}$, to obtain 
%
\begin{eqnarray}
\delta G_{00}&=&\frac{6}{\tau}\dot{\psi}+\frac{2}{\tau}\tilde{\nabla}^2(-\tau \psi +B-\dot{E}),
\nonumber\\
\delta G_{0i}&=&\frac{1}{2}\tilde{\nabla}^2(B_i-\dot{E}_i)+\frac{1}{\tau^2}\tilde{\nabla}_i(-B-2\tau^2\dot{\psi}-2\tau \phi)-\frac{1}{\tau^2}B_i,
\nonumber\\
\delta G_{ij}&=&\frac{\delta_{ij}}{\tau^2}\bigg[-2\tau^2\ddot{\psi}-2\tau\dot{\phi}-4\tau\dot{\psi}+2\phi+2\psi
\nonumber\\
&&+\tilde{\nabla}^2\left(-2\tau B-\tau^2\dot{B}+\tau^2\ddot{E}+2\tau\dot{E}-\tau^2\phi+\tau^2\psi\right)\bigg]
\nonumber\\
&&+\frac{1}{\tau^2}\tilde{\nabla}_i\tilde{\nabla}_j\left[2\tau B +\tau^2\dot{B}-\tau^2\ddot{E}-2\tau\dot{E}-2E+\tau^2\phi-\tau^2\psi\right]
\nonumber\\
&&+\frac{1}{2\tau^2}\tilde{\nabla}_i\left[2\tau B_j-2\tau\dot{E}_j+\tau^2\dot{B}_j-\tau^2\ddot{E}_j-2E_j\right]
\nonumber\\
&&+\frac{1}{2\tau^2}\tilde{\nabla}_j\left[2\tau B_i-2\tau\dot{E}_i+\tau^2\dot{B}_i-\tau^2\ddot{E}_i-2E_i\right]
\nonumber\\
&&-\ddot{E}_{ij}-\frac{2}{\tau^2}E_{ij}-\frac{2}{\tau}\dot{E}_{ij}+\tilde{\nabla}^2E_{ij},
\label{8.6}
\end{eqnarray}
%
where the dot denotes $\partial_\tau$.
For a spatially flat Robertson-Walker  metric the gauge-invariant metric combinations are (see e.g. \cite{amarasinghe_2019})
%
\begin{eqnarray}
\alpha=\phi+\psi+\dot{B}-\ddot{E} ,\quad \gamma=-\tau\psi+B-\dot{E}, \quad B_i-\dot{E}_i,\quad E_{ij}.
\label{8.7}
\end{eqnarray}
%
(For a generic (\ref{8.2}) with conformal factor $\Omega(\tau)$, the quantity $-(\Omega/\dot{\Omega})\psi+B-\dot{E}$ becomes $\gamma$ when $\Omega(\tau)=\tau/2$, with the other gauge invariants being independent of $\Omega(\tau)$.)
In terms of the gauge-invariant combinations the fluctuation equations $\Delta_{\mu\nu}=\delta G_{\mu\nu}+\delta T_{\mu\nu}=0$ take the form
%
\begin{eqnarray}
\Delta_{00}=-\frac{16}{\tau^3}\delta U_{0}-\frac{8}{\tau^2}\phi+\frac{3\tau^2}{4}\left(\delta p -\frac{16}{\tau^4}\psi\right)
+\frac{6}{\tau^2}(\alpha-\dot{\gamma})+\frac{2}{\tau}\tilde{\nabla}^2\gamma=0,
\label{8.8}
\end{eqnarray}
%
\begin{eqnarray}
\Delta_{0i}&=&-\frac{8}{\tau^3}\delta U_{i}+\frac{4}{\tau}\tilde{\nabla}_i\psi
+\frac{1}{2}\tilde{\nabla}^2(B_i-\dot{E}_i)-\frac{2}{\tau}\tilde{\nabla}_i(\alpha-\dot{\gamma})=0,
\label{8.9}
\end{eqnarray}
%
\begin{eqnarray}
\Delta_{ij}&=&\frac{\delta_{ij}}{4\tau^2}\left[\tau^4\delta p-16\psi-8\tau(\dot{\alpha}-\ddot{\gamma})+8(\alpha-\dot{\gamma})-4\tau \tilde{\nabla}^2(\tau\alpha+2\gamma)\right]
\nonumber\\
&&+\frac{1}{\tau}\tilde{\nabla}_i\tilde{\nabla}_j(\tau\alpha+2\gamma) +\frac{1}{2\tau}\tilde{\nabla}_i[2(B_j-\dot{E}_j)+\tau(\dot{B}_j-\ddot{E}_j)]
\nonumber\\
&&+\frac{1}{2\tau}\tilde{\nabla}_j[2(B_i-\dot{E}_i)+\tau(\dot{B}_i-\ddot{E}_i)]
-\ddot{E}_{ij}-\frac{2}{\tau}\dot{E}_{ij}+\tilde{\nabla}^2E_{ij}=0,
\label{8.10}
\end{eqnarray}
%
\begin{eqnarray}
g^{\mu\nu}\Delta_{\mu\nu}&=&\frac{64}{\tau^5}\delta U_0+\frac{32}{\tau^4}\phi -\frac{24}{\tau^3}(\dot{\alpha}-\ddot{\gamma})-\frac{8}{\tau^2}\tilde{\nabla}^2\alpha-\frac{24}{\tau^3}\tilde{\nabla}^2\gamma=0.
\label{8.11}
\end{eqnarray}
%
Since $\Delta_{\mu\nu}$ is gauge invariant, we see that it is not $\delta U_0$, $\delta U_i$ and $\delta p$ themselves that are gauge invariant. Rather it is the combinations $\delta U_0+\tau\phi/2$, $\delta p-16\psi/\tau^4$, and $\delta U_i-\tau^2\tilde{\nabla}_i\psi/2$ that are gauge invariant instead. Since we have shown above that $\delta U_0+\tau\phi/2$ is actually zero, confirming that it is equal to a gauge-invariant quantity provides a nice check on our calculation. However, since $\delta U_0+\tau\phi/2$ is zero, we can replace the $\Delta_{00}$ and $g^{\mu\nu}\Delta_{\mu\nu}$ equations by 
%
\begin{eqnarray}
\Delta_{00}&=&\frac{3\tau^2}{4}\left(\delta p -\frac{16}{\tau^4}\psi\right)
+\frac{6}{\tau^2}(\alpha-\dot{\gamma})+\frac{2}{\tau}\tilde{\nabla}^2\gamma=0,
\label{8.12}
\end{eqnarray}
%
%
\begin{eqnarray}
g^{\mu\nu}\Delta_{\mu\nu}&=& -\frac{24}{\tau^3}(\dot{\alpha}-\ddot{\gamma})-\frac{8}{\tau^2}\tilde{\nabla}^2\alpha-\frac{24}{\tau^3}\tilde{\nabla}^2\gamma=0.
\label{8.13}
\end{eqnarray}
%

To put $\delta U_i$ in a more convenient form we decompose it into transverse and longitudinal components  as $\delta U_i=V_i+\tilde{\nabla}_iV$ where 
%
\begin{eqnarray}
\tilde{\nabla}^iV_i=0,\quad \tilde{\nabla}^2V=\tilde{\nabla}^i\delta U_i,\quad V({\bf x},\tau)=\int d^3{\bf y}D^{(3)}({\bf x}-{\bf y})\tilde{\nabla}^i_y\delta U_i({\bf y}, \tau).
\label{8.14}
\end{eqnarray}
%
In terms of these components the $\Delta_{0i}=0$ equation takes the form
%
\begin{eqnarray}
\Delta_{0i}=-\frac{8}{\tau^3}V_{i}+\frac{1}{2}\tilde{\nabla}^2(B_i-\dot{E}_i)-\frac{8}{\tau^3}\tilde{\nabla}_iV+\frac{4}{\tau}\tilde{\nabla}_i\psi
-\frac{2}{\tau}\tilde{\nabla}_i(\alpha-\dot{\gamma})=0.
\label{8.15}
\end{eqnarray}
%
Applying $\tilde{\nabla}^i$ to $\Delta_{0i}$ and then $\tilde{\nabla}^2$ in turn yields
%
\begin{eqnarray}
\tilde{\nabla}^i\Delta_{0i}&=&\tilde{\nabla}^2\left[-\frac{8}{\tau^3}V+\frac{4}{\tau}\psi
-\frac{2}{\tau}(\alpha-\dot{\gamma})\right]=0,
\nonumber\\
&& \tilde{\nabla}^2\left[-\frac{8}{\tau^3}V_{i}+\frac{1}{2}\tilde{\nabla}^2(B_i-\dot{E}_i)\right]=0,\label{8.16}
\end{eqnarray}
%
while applying $\epsilon^{ijk}\tilde{\nabla}_j$ to $\Delta_{0k}$ yields
%
\begin{eqnarray}
\epsilon^{ijk}\tilde{\nabla}_j\Delta_{0k}&=&\epsilon^{ijk}\tilde{\nabla}_j\left[-\frac{8}{\tau^3}V_{k}+\frac{1}{2}\tilde{\nabla}^2(B_k-\dot{E}_k)\right]=0.
\label{8.17}
\end{eqnarray}
%
We thus identify $V-\tau^2\psi/2$ and $V_i$ as being gauge invariant.

To determine more of the structure of the solutions to the fluctuation equations we apply $\tilde{\nabla}^i\tilde{\nabla}^j$ to the $\Delta_{ij}=0$ equation and then use the $\Delta_{00}=0$ equation to eliminate $\delta p -16\psi/\tau^4$ from $\Delta_{ij}$, to obtain
%
\begin{eqnarray}
\tilde{\nabla}^2\bigg{[}\frac{\tau^2}{4}\delta p-\frac{4}{\tau^2}\psi-\frac{2}{\tau}(\dot{\alpha}-\ddot{\gamma})+\frac{2}{\tau^2}(\alpha-\dot{\gamma})\bigg{]}
\nonumber\\
=-\frac{2}{3\tau}\left[
\tilde{\nabla}^4\gamma+3\tilde{\nabla}^2(\dot{\alpha}-\ddot{\gamma})\right]=0.
\label{8.18}
\end{eqnarray}
% 
With the application of $\tilde{\nabla}^2$ to the $g^{\mu\nu}\Delta_{\mu\nu}=0$ equation yielding
%
\begin{eqnarray}
3\tilde{\nabla}^2(\dot{\alpha}-\ddot{\gamma})+\tau \tilde{\nabla}^4\alpha+3\tilde{\nabla}^4\gamma=0,
\label{8.19}
\end{eqnarray}
% 
we obtain
%
\begin{eqnarray}
\tilde{\nabla}^4(\tau \alpha+2\gamma)=0.
\label{8.20}
\end{eqnarray}
% 
On applying $\partial_{\tau}$ to (\ref{8.20}) and using (\ref{8.20}) and (\ref{8.18}) we obtain
%
\begin{eqnarray}
3\tau\tilde{\nabla}^4\dot{\alpha}=-3\tilde{\nabla}^4\alpha-6\tilde{\nabla}^4\dot{\gamma}
=\frac{6}{\tau}\tilde{\nabla}^4\gamma-6\tilde{\nabla}^4\dot{\gamma}
=3\tau\tilde{\nabla}^4\ddot{\gamma}-\tau\tilde{\nabla}^6\gamma.
\label{8.21}
\end{eqnarray}
%
The parameter $\gamma$ thus obeys
%
\begin{eqnarray}
\tilde{\nabla}^4\left(\tilde{\nabla}^2\gamma-3\ddot{\gamma}-\frac{6}{\tau}\dot{\gamma}+\frac{6}{\tau^2}\gamma\right)=0.
\label{8.22}
\end{eqnarray}
%
We can treat (\ref{8.22})  as a second-order differential equation for $\tilde{\nabla}^4\gamma$.
On setting $\tilde{\nabla}^4\gamma=f_k(\tau)g_k({\bf x})$ for a single mode, on separating  with a separation constant $-k^2$ according to
%
\begin{eqnarray}
\frac{\tilde{\nabla}^2g_k}{g_k}=\frac{3\ddot{f}_k+6\dot{f}_k/\tau-6f_k/\tau^2}{f_k}=-k^2,
\label{8.23}
\end{eqnarray}
%
we find that the $\tau$ dependence of $f_k(\tau)$ is given by $j_1(k\tau/\surd{3})$ and $y_1(k\tau/\surd{3})$, while the spatial dependence is given by plane waves. Since the set of plane waves is complete, the general solution to (\ref{8.22}) can be written as
%
\begin{eqnarray}
\gamma&=&\sum_{\bf k}[a_1({\bf k})j_1(k\tau/\surd{3})+b_1({\bf k})y_1(k\tau/\surd{3})]e^{i{\bf k}\cdot{\bf x}} +{\rm delta~function~terms},
\label{8.24}
\end{eqnarray}
%
where the delta function terms are solutions to $\tilde{\nabla}^4\gamma=0$, solutions that, in analog to  (\ref{1.8}),  are generically of the form $\delta(k)$, $\delta(k)/k$, $\delta(k)/k^2$, $\delta(k)/k^3$.


Proceeding the same way for $\alpha$  we obtain 
%
\begin{eqnarray}
3\tilde{\nabla}^4\dot{\alpha}=3\tilde{\nabla}^4\ddot{\gamma}-\tilde{\nabla}^6\gamma
=-\frac{3}{2}\tilde{\nabla}^4(\tau\ddot{\alpha}+2\dot{\alpha})+\frac{\tau}{2}\tilde{\nabla}^6\alpha.
\label{8.25}
\end{eqnarray}
%
The parameter $\alpha$ thus obeys
%
\begin{eqnarray}
\tilde{\nabla}^4\left(\tilde{\nabla}^2\alpha-3\ddot{\alpha}-\frac{12}{\tau}\dot{\alpha}\right)=0.
\label{8.26}
\end{eqnarray}
%
On setting $\tilde{\nabla}^4\alpha=d_k(\tau)e_k({\bf x})$ for a single mode, we can separate with a separation constant $-k^2$ according to
%
\begin{eqnarray}
\frac{\tilde{\nabla}^2e_k}{e}=\frac{3\ddot{d}_k+12\dot{d}_k/\tau}{d_k}=-k^2.
\label{8.27}
\end{eqnarray}
%
The $\tau$ dependence of $d_k(\tau)$ is thus given by $j_1(k\tau/\surd{3})/\tau$ and  $y_1(k\tau/\surd{3})/\tau$, while the spatial dependence is given by plane waves. The general solution to (\ref{8.26}) is thus given by 
%
\begin{eqnarray}
\alpha&=&\frac{1}{\tau}\sum_{\bf k}[m_1({\bf k})j_1(k\tau/\surd{3})+n_1({\bf k})y_1(k\tau/\surd{3})]e^{i{\bf k}\cdot{\bf x}}
\nonumber\\
&& +{\rm delta~function~terms},
\label{8.28}
\end{eqnarray}
%
where the delta function terms are solutions to $\tilde{\nabla}^4\alpha=0$. Finally, we recall that $\alpha$ and $\gamma$ are related through $\tilde{\nabla}^4(\tau\alpha+2\gamma)=0$,  with the coefficients  thus obeying
%
\begin{eqnarray}
m_1({\bf k})+2a_1({\bf k})=0,\quad n_1({\bf k})+2b_1({\bf k})=0.
\label{8.29}
\end{eqnarray}
%


Having determined $\alpha$ and $\gamma$, we can now determine $\delta p-16\psi/\tau^4$ from the $\Delta_{00}=0$ equation, and obtain
%
\begin{eqnarray}
\nonumber\\
&&\delta p -\frac{16}{\tau^4}\psi=-\frac{8}{\tau^4}(\alpha-\dot{\gamma})-\frac{8}{3\tau^3}\tilde{\nabla}^2\gamma
\nonumber\\
&=&\sum_{\bf k}\left[\frac{8}{\tau^4}\left(\frac{2}{\tau}+\frac{\partial}{\partial \tau}\right)
+\frac{8k^2}{3\tau^3}\right]\left[a_1({\bf k})j_1(k\tau/\surd{3})+b_1({\bf k})y_1(k\tau/\surd{3})\right]e^{i{\bf k}\cdot{\bf x}}
\nonumber\\
&& +{\rm delta~function~terms}
\nonumber\\
&=&\sum_{\bf k}a_1({\bf k})\left[\frac{8k}{\tau^4\surd{3}}j_0(k\tau/\surd{3})+\frac{8k^2}{3\tau^3}j_1(k\tau/\surd{3})\right]
\nonumber\\
&&+\sum_{\bf k}b_1({\bf k})\left[\frac{8k}{\tau^4\surd{3}}y_0(k\tau/\surd{3})+\frac{8k^2}{3\tau^3}y_1(k\tau/\surd{3})\right]
\nonumber\\
&&+{\rm delta~function~terms}.
\label{8.30}
\end{eqnarray}
%
To determine $B_i-\dot{E}_i$ we apply $\tilde{\nabla}^j$ to $\Delta_{ij}=0$, to obtain
%
\begin{eqnarray}
\tilde{\nabla}^2\left[\frac{1}{\tau}(B_i-\dot{E}_i)+\frac{1}{2}(\dot{B}_i-\ddot{E}_i)\right]=\tilde{\nabla}_i\left[\frac{2}{\tau}(\dot{\alpha}-\ddot{\gamma})+\frac{2}{3\tau}\tilde{\nabla}^2\gamma\right],
\label{8.31}
\end{eqnarray}
%
from which it follows that
%
\begin{eqnarray}
\tilde{\nabla}^i\tilde{\nabla}^2\left[\frac{1}{\tau}(B_i-\dot{E}_i)+\frac{1}{2}(\dot{B}_i-\ddot{E}_i)\right]=\tilde{\nabla}^2\left[\frac{2}{\tau}(\dot{\alpha}-\ddot{\gamma})+\frac{2}{3\tau}\tilde{\nabla}^2\gamma\right]=0.
\label{8.32}
\end{eqnarray}
%
On now applying $\tilde{\nabla}^2$ to (\ref{8.31}) we  obtain
%
\begin{eqnarray}
\tilde{\nabla}^4\left[\frac{1}{\tau}(B_i-\dot{E}_i)+\frac{1}{2}(\dot{B}_i-\ddot{E}_i)\right]=0,
\label{8.33}
\end{eqnarray}
%
just as required for consistency with (\ref{8.18}).
Equation (\ref{8.33}) can be satisfied through a $1/\tau^2$ conformal time dependence, and while it could also be satisfied via a spatial dependence that satisfies $\tilde{\nabla}^4(B_i-\dot{E}_i)=0$, viz. the above delta function terms. With plane waves being complete we can thus set
%
\begin{eqnarray}
B_i-\dot{E}_i=\frac{1}{\tau^2}\sum_{\bf k}a_i({\bf k})e^{i{\bf k}\cdot{\bf x}}
+F(\tau)\times {\rm delta~function~terms},
\label{8.34}
\end{eqnarray}
%
where the $a_i({\bf k})$ are transverse vectors that obeys $k^ia_i({\bf k})=0$, and where $F(\tau)$ is an arbitrary function of $\tau$.

After solving (\ref{8.31}) and (\ref{8.32}), from the second equation in (\ref{8.16}) we can then determine $V_i$ as it obeys 
%
\begin{eqnarray}
\frac{8}{\tau^3}\tilde{\nabla}^2V_{i}=\frac{1}{2}\tilde{\nabla}^4(B_i-\dot{E}_i)=\frac{1}{2\tau^2}\sum_{\bf k}k^4a_i({\bf k})e^{i{\bf k}\cdot{\bf x}}.
\label{8.35}
\end{eqnarray}
%
From (\ref{8.17}) we can infer that 
%
\begin{eqnarray}
-\frac{8}{\tau^3}V_{i}+\frac{1}{2}\tilde{\nabla}^2(B_i-\dot{E}_i)=\tilde{\nabla}_iA,
\label{8.36}
\end{eqnarray}
%
where $A$ is a scalar function that obeys $\tilde{\nabla}^2A=0$, with $\tilde{\nabla}_iA$ being curl free. We recognize $A$ as an integration constant for the integration of (\ref{8.35}). From the first equation in (\ref{8.16}) we additionally obtain
%
\begin{eqnarray}
&&\tilde{\nabla}^2(-\frac{8}{\tau^3}V+\frac{4}{\tau}\psi)
=\frac{2}{\tau}\tilde{\nabla}^2(\alpha-\dot{\gamma})
\nonumber\\
&=&\frac{2}{\tau\surd{3}}\sum_{\bf k}k^3[a_1({\bf k})j_0(k\tau/\surd{3})+b_1({\bf k})y_0(k\tau/\surd{3})]e^{i{\bf k}\cdot{\bf x}}
\nonumber\\
&&+ {\rm delta~function~terms.}
\label{8.37}
\end{eqnarray}
%

To determine an equation for $E_{ij}$ we note that the $\delta_{ij}$ term in $\Delta _{ij}$ can be written as 
$(\delta_{ij}/4\tau^2)[-8\tau(\dot{\alpha}-\ddot{\gamma})-4\tau^2\tilde{\nabla}^2\alpha-(32/3)\tau\tilde{\nabla}^2\gamma]$. Through use of (\ref{8.18}), (\ref{8.19}) and (\ref{8.20}), we can show that $\tilde{\nabla}^2$ applied to this term gives zero. Then given (\ref{8.33}) and (\ref{8.20}), from (\ref{8.10}) it follows that $E_{ij}$ obeys 
%
\begin{eqnarray}
\tilde{\nabla}^4\left(-\ddot{E}_{ij}-\frac{2}{\tau}\dot{E}_{ij}+\tilde{\nabla}^2E_{ij}\right)=0.
\label{8.38}
\end{eqnarray}
%
Setting $\tilde{\nabla}^4E_{ij}=\epsilon_{ij}({\bf k})f_k(\tau)g_k({\bf x})$ for a momentum mode, the $\tau$ dependence  is given as $j_0(k\tau)$ and $y_0(k\tau)$, with the general solution being of the form
%
\begin{eqnarray}
E_{ij}&=&\sum_{\bf k}[a^0_{ij}({\bf k})j_0(k\tau)+b^0_{ij}({\bf k})y_0(k\tau)]e^{i{\bf k}\cdot{\bf x}}
+ {\rm delta~function~terms}.
\label{8.39}
\end{eqnarray}
%
Since according to (\ref{8.2}) the full tensor fluctuation is given not by $E_{ij}$ but by $\tau^2E_{ij}/2$, then with $\tau=2t^{1/2}$, we find that at large comoving time  $\tau^2 E_{ij}$ behaves as $t^{1/2}$. Thus to summarize,  we have constructed the exact and general solution to the SVT3 $k=0$ radiation era Robertson-Walker fluctuation equations for all of the dynamical degrees of freedom $\alpha$, $\beta$, $B_i-\dot{E}_i$, $E_{ij}$, $\delta p-16\psi/\tau^4$, $V-\tau^2\psi/2$ and $V_i$. 

For a decomposition theorem to hold the condition $\Delta_{\mu\nu}=0$ would need to decompose into 

%
\begin{eqnarray}
&&\frac{3\tau^2}{4}\left(\delta p -\frac{16}{\tau^4}\psi\right)
+\frac{6}{\tau^2}(\alpha-\dot{\gamma})+\frac{2}{\tau}\tilde{\nabla}^2\gamma=0,
\label{8.40}
\end{eqnarray}
%
\begin{eqnarray}
&&-\frac{8}{\tau^3}V_{i}+\frac{1}{2}\tilde{\nabla}^2(B_i-\dot{E}_i)=0,
\label{8.41}
\end{eqnarray}
%
%
\begin{eqnarray}
&&-\frac{8}{\tau^3}\tilde{\nabla}_iV+\frac{4}{\tau}\tilde{\nabla}_i\psi
-\frac{2}{\tau}\tilde{\nabla}_i(\alpha-\dot{\gamma})=0,
\label{8.42}
\end{eqnarray}
%
\begin{eqnarray}
&&\frac{\delta_{ij}}{4\tau^2}\left[\tau^4\delta p-16\psi-8\tau(\dot{\alpha}-\ddot{\gamma})+8(\alpha-\dot{\gamma})-4\tau \tilde{\nabla}^2(\tau\alpha+2\gamma)\right]
\nonumber\\
&&+\frac{1}{\tau}\tilde{\nabla}_i\tilde{\nabla}_j(\tau\alpha+2\gamma)=0,
\label{8.43}
\end{eqnarray}
%
\begin{eqnarray}
\frac{1}{2\tau}\tilde{\nabla}_i[2(B_j-\dot{E}_j)+\tau(\dot{B}_j-\ddot{E}_j)]
+\frac{1}{2\tau}\tilde{\nabla}_j[2(B_i-\dot{E}_i)+\tau(\dot{B}_i-\ddot{E}_i)]=0,
\label{8.44}
\end{eqnarray}
%
\begin{eqnarray}
&&-\ddot{E}_{ij}-\frac{2}{\tau}\dot{E}_{ij}+\tilde{\nabla}^2E_{ij}=0,
\label{8.45}
\end{eqnarray}
%
\begin{eqnarray}
&& -\frac{24}{\tau^3}(\dot{\alpha}-\ddot{\gamma})-\frac{8}{\tau^2}\tilde{\nabla}^2\alpha-\frac{24}{\tau^3}\tilde{\nabla}^2\gamma=0.
\label{8.46}
\end{eqnarray}
%

To determine whether these conditions might hold, we note that in the $\alpha$, $\gamma$ sector the (\ref{8.40}) and (\ref{8.46})  equations are the same as in the general $\Delta_{\mu\nu}=0$ case (viz. (\ref{8.12}) and (\ref{8.13})), but (\ref{8.43}) is different. If we use (\ref{8.12}) to substitute for $\delta p$ in (\ref{8.43}) we obtain 
%
\begin{eqnarray}
&&\frac{\delta_{ij}}{4\tau^2}\left[-8\tau(\dot{\alpha}-\ddot{\gamma})-
4\tau^2\tilde{\nabla}^2\alpha-\frac{32\tau}{3} \tilde{\nabla}^2\gamma\right]+\frac{1}{\tau}\tilde{\nabla}_i\tilde{\nabla}_j(\tau\alpha+2\gamma)=0.
\label{8.47}
\end{eqnarray}
%
Given the differing behaviors of $\delta_{ij}$ and $\tilde{\nabla}_i\tilde{\nabla}_j$ it follows that the terms that they multiply  in (\ref{8.47}) must each vanish, and thus we can set
%
\begin{eqnarray}
&&-8\tau(\dot{\alpha}-\ddot{\gamma})-
4\tau^2\tilde{\nabla}^2\alpha-\frac{32\tau}{3} \tilde{\nabla}^2\gamma=0,
\label{8.48}
\end{eqnarray}
%
\begin{eqnarray}
\tau\alpha+2\gamma=0.
\label{8.49}
\end{eqnarray}
%
Combining these equations then gives
%
\begin{eqnarray}
&&3(\dot{\alpha}-\ddot{\gamma})+ \tilde{\nabla}^2\gamma=0.
\label{8.50}
\end{eqnarray}
%
We recognize (\ref{8.18}) as the $\tilde{\nabla}^2$ derivative of (\ref{8.50}) and recognize (\ref{8.20}) as the $\tilde{\nabla}^4$ derivative of (\ref{8.49}).

Similarly, in the $V$,$V_i$ sector we recognize the two equations that appear in (\ref{8.16}) as the $\nabla^2$ derivative of (\ref{8.41}) and the $\tilde{\nabla}^i$ derivative of (\ref{8.42}), with (\ref{8.17}) being the curl of (\ref{8.41}). In the $B_i-\dot{E}_i$ sector we recognize (\ref{8.33}) as the $\tilde{\nabla}^j\tilde{\nabla}^2$ derivative of (\ref{8.44}), and in the $E_{ij}$ sector we recognize (\ref{8.38}) as the $\tilde{\nabla}^4$ derivative of  (\ref{8.45}). Consequently we see that if spatially asymptotic boundary conditions are such that the only solutions to $\Delta_{\mu\nu}=0$ are also solutions to (\ref{8.40}) to (\ref{8.46}) (i.e. vanishing of all delta function terms and integration constants that would lead to non-vanishing asymptotics), then the decomposition theorem follows. Otherwise it does not. Finally, we should note that, as constructed, in the matter sector we have found solutions for the gauge-invariant quantities $\delta p-16\psi/\tau^4$, and $V-\tau^2\psi/2$. However since $\psi$ is not gauge invariant, by choosing a gauge in which $\psi=0$, we would then have solutions for $\delta p$ and $V$ alone.
%%%%%%%%%%%%%%%%%%%%%%%%%%%%%%%%%%%%%%%%%%%%
\subsection{General Roberston Walker}
\label{ss:general_rw_svt3}
%%%%%%%%%%%%%%%%%%%%%%%%%%%%%%%%%%%%%%%%%%%%
\subsubsection{Constructing the Fluctuations}
\label{sss:setting_up_the_equations}


Having seen how things work in a particular background Robertson-Walker case (radiation era with $k=0$, Sec. \ref{ss:rw_k=0_radiation_svt3}), we now present a general analysis that can be applied to any background Robertson-Walker geometry with any background perfect fluid equation of state. To characterize a general Robertson-Walker background there are two straightforward options. One is to write the background metric in a conformal to flat form $ds^2=\Omega^2(\tau,x^i)(d\tau^2-\delta_{ij}dx^idx^j)$ with $\Omega(\tau,x^i)$ depending on both the conformal time $\tau=\int dt/a(t)$ and the spatial coordinates. The other is to write the background geometry as conformal to a static Robertson-Walker geometry: 
%
\begin{eqnarray}
ds^2&=&\Omega^2(\tau)[d\tau^2-\tilde{\gamma}_{ij}dx^idx^j]
\nonumber\\
&=&\Omega^2(\tau)\left[ d\tau^2-\frac{dr^2}{1-kr^2}-r^2d\theta^2-r^2\sin\theta^2d\phi^2\right],
\label{9.1}
\end{eqnarray}
%
with $\Omega(\tau)$ depending only on $\tau$, and with $\tilde{\gamma}_{ij}dx^idx^j$ denoting the spatial sector of the metric. These two formulations of the background metric are coordinate equivalent as one can transform one into the other by a general coordinate transformation without any need to make a conformal transformation on the background metric (see Appendix \ref{ab:cosmologies} and Sec. \ref{ss:deltaW_conformal_flat_SVT3}). For our purposes in this section we shall take (\ref{9.1}) to be the background metric, and shall take the fluctuation metric to be of the form
%
\begin{eqnarray}
ds^2&=&\Omega^2(\tau)\bigg[2\phi d\tau^2 -2(\tilde{\nabla}_i B +B_i)d\tau dx^i - [-2\psi\tilde{\gamma}_{ij} +2\tilde{\nabla}_i\tilde{\nabla}_j E
\nonumber\\
&& + \tilde{\nabla}_i E_j + \tilde{\nabla}_j E_i + 2E_{ij}]dx^i dx^j\bigg].
\label{9.2}
\end{eqnarray}
%
In (\ref{9.2})  $\tilde{\nabla}_i=\partial/\partial x^i$ and  $\tilde{\nabla}^i=\tilde{\gamma}^{ij}\tilde{\nabla}_j$  (with Latin indices) are defined with respect to the background three-space metric $\tilde{\gamma}_{ij}$. And with
%
\begin{eqnarray}
\tilde{\gamma}^{ij}\tilde{\nabla}_j V_i=\tilde{\gamma}^{ij}[\partial_j V_i-\tilde{\Gamma}^{k}_{ij}V_k]
\label{9.3}
\end{eqnarray}
%
for any 3-vector $V_i$ in a 3-space with 3-space connection $\tilde{\Gamma}^{k}_{ij}$, the elements of (\ref{9.2}) are required to obey
%
\begin{eqnarray}
\tilde{\gamma}^{ij}\tilde{\nabla}_j B_i = 0,\quad \tilde{\gamma}^{ij}\tilde{\nabla}_j E_i = 0, \quad E_{ij}=E_{ji},\quad \tilde{\gamma}^{jk}\tilde{\nabla}_kE_{ij} = 0, \quad\tilde{\gamma}^{ij}E_{ij} = 0.
\label{9.4}
\end{eqnarray}
%
With the  3-space sector of the background geometry being maximally 3-symmetric, it is described by a Riemann tensor of the form
%
\begin{eqnarray}
\tilde{R}_{ijk\ell}=k[\tilde{\gamma}_{jk}\tilde{\gamma}_{i\ell}-\tilde{\gamma}_{ik}\tilde{\gamma}_{j\ell}].
\label{9.5}
\end{eqnarray}
%
In \cite{amarasinghe_2019} (and as discussed in (\ref{9.43a}) to (\ref{9.47a}) below), it was shown that for the fluctuation metric given in (\ref{9.2}) with $\Omega(\tau)$ being arbitrary function of $\tau$, the gauge-invariant metric combinations are $\phi + \psi + \dot B - \ddot E$, $ - \dot\Omega^{-1}\Omega \psi + B - \dot E$, $B_i-\dot{E}_i$, and $E_{ij}$. As we shall see, the fluctuation equations will explicitly depend on these specific combinations.



We take the background $T_{\mu\nu}$ to be of the perfect fluid form
%
\begin{eqnarray}
T_{\mu\nu}=(\rho+p)U_{\mu}U_{\nu}+pg_{\mu\nu},
\label{9.6}
\end{eqnarray}
%
with fluctuation
%
\begin{eqnarray}
\delta T_{\mu\nu}=(\delta\rho+\delta p)U_{\mu}U_{\nu}+\delta pg_{\mu\nu}+(\rho+p)(\delta U_{\mu}U_{\nu}+U_{\mu}\delta U_{\nu})+ph_{\mu\nu}.
\label{9.7}
\end{eqnarray}
%
Here  $g^{\mu\nu}U_{\mu}U_{\nu}=-1$, $U^{0}=\Omega^{-1}(\tau)$, $U_0=-\Omega(\tau)$, $U^{i}=0$, $U_i=0$ for the background, while for the fluctuation we have 
%
\begin{eqnarray}
\delta g^{00}U_{0}U_{0}+2g^{00}U_{0}\delta U_{0}=0,
\label{9.8}
\end{eqnarray}
%
i.e. 
%
\begin{eqnarray}
\delta U_{0}=-\frac{1}{2}(g^{00})^{-1}(-g^{00}g^{00}\delta g_{00})U_{0}=-\Omega(\tau)\phi.
\label{9.9}
\end{eqnarray}
%
Thus just as in Sec. \ref{S8} we see that $\delta U_0$ is not an independent degree of freedom. As in Sec. \ref{S8} we shall set $\delta U_i=V_i+\tilde{\nabla}_iV$, where now $\tilde{\gamma}^{ij}\tilde{\nabla}_j V_i=\tilde{\gamma}^{ij}[\partial_j V_i-\tilde{\Gamma}^{k}_{ij}V_k]=0$. As constructed, in general we have 11 fluctuation variables, the six from the metric together with the three $\delta U_i$, and $\delta\rho$ and $\delta p$. But we only have ten fluctuation equations. Thus to solve the theory when there is both a $\delta \rho$ and a $\delta p$ we will need some constraint between $\delta p$ and $\delta \rho$, a point we return to below.

For the background Einstein equations we have
%
\begin{eqnarray}
G_{00}&=& -3k - 3 \dot{\Omega}^2\Omega^{-2},\quad G_{0i} =0,
\quad G_{ij} = \tilde{\gamma}_{ij}\left[k - \dot\Omega^2\Omega^{-2}+ 2\ddot\Omega \Omega^{-1}\right],
\nonumber\\
G_{00}+8\pi G T_{00} &=& -3k - 3 \dot\Omega^2\Omega^{-2} + \Omega^2 \rho=0,
\nonumber\\
G_{ij}+8\pi G T_{ij}&=& \tilde{\gamma}_{ij}\left[k - \dot\Omega^2\Omega^{-2} + 2\ddot\Omega \Omega^{-1}  + \Omega^2 p\right]=0,
\nonumber\\
\rho &=& 3k\Omega^{-2}+3\dot\Omega^2 \Omega^{-4},\quad p = -k\Omega^{-2} + \dot\Omega^2\Omega^{-4} -2\ddot\Omega \Omega^{-3},
\nonumber\\
 p &=& -\rho -\frac{1}{3} \frac{\Omega}{\dot\Omega}\dot\rho,
\label{9.10}
\end{eqnarray}
%
(after setting $8\pi G=1$), with the last relation following from $\nabla_{\nu}T^{\mu\nu}=0$, viz. conservation of the energy-momentum tensor in the full 4-space. To solve these equations we would need an equation of state that would relate $\rho$ and $p$. However we do not need to impose one just yet, as we shall generate the fluctuation equations as subject to (\ref{9.10}) but without needing to specify the form of $\Omega(\tau)$ or a relation  between $\rho$ and $p$.

For $\delta G_{\mu\nu}$ we have
%
\begin{eqnarray}
\delta G_{00}&=& -6 k \phi - 6 k \psi + 6 \dot{\psi} \dot{\Omega} \Omega^{-1} + 2 \dot{\Omega} \Omega^{-1} \tilde{\nabla}_{a}\tilde{\nabla}^{a}B - 2 \dot{\Omega} \Omega^{-1} \tilde{\nabla}_{a}\tilde{\nabla}^{a}\dot{E}
\nonumber\\
&& - 2 \tilde{\nabla}_{a}\tilde{\nabla}^{a}\psi, 
\nonumber\\ 
\delta G_{0i}&=& 3 k \tilde{\nabla}_{i}B -  \dot{\Omega}^2 \Omega^{-2} \tilde{\nabla}_{i}B + 2 \overset{..}{\Omega} \Omega^{-1} \tilde{\nabla}_{i}B - 2 k \tilde{\nabla}_{i}\dot{E} - 2 \tilde{\nabla}_{i}\dot{\psi} - 2 \dot{\Omega} \Omega^{-1} \tilde{\nabla}_{i}\phi\nonumber\\
&& +2 k B_{i} -  k \dot{E}_{i} -  B_{i} \dot{\Omega}^2 \Omega^{-2} + 2 B_{i} \overset{..}{\Omega} \Omega^{-1} + \tfrac{1}{2} \tilde{\nabla}_{a}\tilde{\nabla}^{a}B_{i} -  \tfrac{1}{2} \tilde{\nabla}_{a}\tilde{\nabla}^{a}\dot{E}_{i},
\nonumber\\ 
\delta G_{ij}&=& -2 \overset{..}{\psi}\tilde{\gamma}_{ij} + 2 \dot{\Omega}^2\tilde{\gamma}_{ij} \phi \Omega^{-2} + 2 \dot{\Omega}^2\tilde{\gamma}_{ij} \psi \Omega^{-2} - 2 \dot{\phi} \dot{\Omega}\tilde{\gamma}_{ij} \Omega^{-1} - 4 \dot{\psi} \dot{\Omega}\tilde{\gamma}_{ij} \Omega^{-1}\nonumber\\
&& - 4 \overset{..}{\Omega}\tilde{\gamma}_{ij} \phi \Omega^{-1} - 4 \overset{..}{\Omega}\tilde{\gamma}_{ij} \psi \Omega^{-1} - 2 \dot{\Omega}\tilde{\gamma}_{ij} \Omega^{-1} \tilde{\nabla}_{a}\tilde{\nabla}^{a}B - \tilde{\gamma}_{ij} \tilde{\nabla}_{a}\tilde{\nabla}^{a}\dot{B} 
\nonumber\\
&&+\tilde{\gamma}_{ij} \tilde{\nabla}_{a}\tilde{\nabla}^{a}\overset{..}{E} + 2 \dot{\Omega}\tilde{\gamma}_{ij} \Omega^{-1} \tilde{\nabla}_{a}\tilde{\nabla}^{a}\dot{E} 
 - \tilde{\gamma}_{ij} \tilde{\nabla}_{a}\tilde{\nabla}^{a}\phi +\tilde{\gamma}_{ij} \tilde{\nabla}_{a}\tilde{\nabla}^{a}\psi 
 \nonumber\\
 &&+ 2 \dot{\Omega} \Omega^{-1} \tilde{\nabla}_{j}\tilde{\nabla}_{i}B + \tilde{\nabla}_{j}\tilde{\nabla}_{i}\dot{B} -  \tilde{\nabla}_{j}\tilde{\nabla}_{i}\overset{..}{E} - 2 \dot{\Omega} \Omega^{-1} \tilde{\nabla}_{j}\tilde{\nabla}_{i}\dot{E}  + 2 k \tilde{\nabla}_{j}\tilde{\nabla}_{i}E
 \nonumber\\
 && - 2 \dot{\Omega}^2 \Omega^{-2} \tilde{\nabla}_{j}\tilde{\nabla}_{i}E + 4 \overset{..}{\Omega} \Omega^{-1} \tilde{\nabla}_{j}\tilde{\nabla}_{i}E + \tilde{\nabla}_{j}\tilde{\nabla}_{i}\phi -  \tilde{\nabla}_{j}\tilde{\nabla}_{i}\psi +\dot{\Omega} \Omega^{-1} \tilde{\nabla}_{i}B_{j}
 \nonumber\\
 && + \tfrac{1}{2} \tilde{\nabla}_{i}\dot{B}_{j}
 -  \tfrac{1}{2} \tilde{\nabla}_{i}\overset{..}{E}_{j} -  \dot{\Omega} \Omega^{-1} \tilde{\nabla}_{i}\dot{E}_{j} + k \tilde{\nabla}_{i}E_{j} -  \dot{\Omega}^2 \Omega^{-2} \tilde{\nabla}_{i}E_{j}
 \nonumber\\
 && + 2 \overset{..}{\Omega} \Omega^{-1} \tilde{\nabla}_{i}E_{j} + \dot{\Omega} \Omega^{-1} \tilde{\nabla}_{j}B_{i} + \tfrac{1}{2} \tilde{\nabla}_{j}\dot{B}_{i}  -  \tfrac{1}{2} \tilde{\nabla}_{j}\overset{..}{E}_{i} -  \dot{\Omega} \Omega^{-1} \tilde{\nabla}_{j}\dot{E}_{i} 
 \nonumber\\
 &&+ k \tilde{\nabla}_{j}E_{i} -  \dot{\Omega}^2 \Omega^{-2} \tilde{\nabla}_{j}E_{i} + 2 \overset{..}{\Omega} \Omega^{-1} \tilde{\nabla}_{j}E_{i}- \overset{..}{E}_{ij} - 2 \dot{\Omega}^2 E_{ij} \Omega^{-2} \nonumber \\ 
&& - 2 \dot{E}_{ij} \dot{\Omega} \Omega^{-1} + 4 \overset{..}{\Omega} E_{ij} \Omega^{-1} + \tilde{\nabla}_{a}\tilde{\nabla}^{a}E_{ij},
\nonumber\\
g^{\mu\nu}\delta G_{\mu\nu} &=& 6 \dot{\Omega}^2 \phi \Omega^{-4} + 6 \dot{\Omega}^2 \psi \Omega^{-4} - 6 \dot{\phi} \dot{\Omega} \Omega^{-3} - 18 \dot{\psi} \dot{\Omega} \Omega^{-3} - 12 \overset{..}{\Omega} \phi \Omega^{-3}
\nonumber\\
&& - 12 \overset{..}{\Omega} \psi \Omega^{-3} - 6 \overset{..}{\psi} \Omega^{-2} + 6 k \phi \Omega^{-2}  + 6 k \psi \Omega^{-2} - 6 \dot{\Omega} \Omega^{-3} \tilde{\nabla}_{a}\tilde{\nabla}^{a}B
\nonumber\\
&& - 2 \Omega^{-2} \tilde{\nabla}_{a}\tilde{\nabla}^{a}\dot{B} + 2 \Omega^{-2} \tilde{\nabla}_{a}\tilde{\nabla}^{a}\overset{..}{E} + 6 \dot{\Omega} \Omega^{-3} \tilde{\nabla}_{a}\tilde{\nabla}^{a}\dot{E} \nonumber \\ 
&& - 2 \dot{\Omega}^2 \Omega^{-4} \tilde{\nabla}_{a}\tilde{\nabla}^{a}E + 4 \overset{..}{\Omega} \Omega^{-3} \tilde{\nabla}_{a}\tilde{\nabla}^{a}E + 2 k \Omega^{-2} \tilde{\nabla}_{a}\tilde{\nabla}^{a}E
\nonumber\\
&& - 2 \Omega^{-2} \tilde{\nabla}_{a}\tilde{\nabla}^{a}\phi + 4 \Omega^{-2} \tilde{\nabla}_{a}\tilde{\nabla}^{a}\psi. 
\label{9.11}
\end{eqnarray}
%
We introduce
%
\begin{eqnarray}
\alpha  &=& \phi + \psi + \dot B - \ddot E,\quad \gamma = - \dot\Omega^{-1}\Omega \psi + B - \dot E,\quad \hat{V} = V-\Omega^2 \dot \Omega^{-1}\psi,
\nonumber\\
\delta \hat{\rho}&=&\delta \rho - 12 \dot{\Omega}^2 \psi \Omega^{-4} + 6 \overset{..}{\Omega} \psi \Omega^{-3} - 6 k \psi \Omega^{-2}=\delta\rho +\frac{\Omega}{\dot{\Omega}}\dot{\rho}\psi=\delta \rho-3(\rho+p)\psi,
\nonumber\\
\delta \hat{p}&=&\delta p - 4 \dot{\Omega}^2 \psi \Omega^{-4} + 8 \overset{..}{\Omega} \psi \Omega^{-3} + 2 k \psi \Omega^{-2} - 2 \overset{...}{\Omega} \dot{\Omega}^{-1} \psi \Omega^{-2}=\delta p +\frac{\Omega}{\dot{\Omega}}\dot{p}\psi,
\label{9.12}
\end{eqnarray}
%
(in (\ref{9.12}) we used (\ref{9.10})), where, as we show below, the functions $\delta \hat{\rho}$, $\delta \hat{p}$ and $\hat{V} $ are gauge invariant. (The $\gamma$ introduced in (\ref{8.7}) is a special case of the $\gamma$ introduced in (\ref{9.12}).) Given (\ref{9.12}) we can express the components of $\Delta_{\mu\nu}=\delta G_{\mu\nu}+8\pi G\delta T_{\mu\nu}$ quite compactly. Specifically,  on using (\ref{9.10}) for the background but without imposing any relation between the background $\rho$ and $p$, we obtain evolution equations of the form 
%
\begin{eqnarray}
\Delta_{00}&=& 6 \dot{\Omega}^2 \Omega^{-2}(\alpha-\dot\gamma) + \delta \hat{\rho} \Omega^2 + 2 \dot{\Omega} \Omega^{-1} \tilde{\nabla}_{a}\tilde{\nabla}^{a}\gamma=0, 
\label{9.13}
\end{eqnarray}
%
%
\begin{eqnarray}
\Delta_{0i}&=& -2 \dot{\Omega} \Omega^{-1} \tilde{\nabla}_{i}(\alpha - \dot\gamma) + 2 k \tilde{\nabla}_{i}\gamma 
+(-4 \dot{\Omega}^2 \Omega^{-3}  + 2 \overset{..}{\Omega} \Omega^{-2}  - 2 k \Omega^{-1}) \tilde{\nabla}_{i}\hat{V}
\nonumber\\
&& +k(B_i-\dot E_i)+ \tfrac{1}{2} \tilde{\nabla}_{a}\tilde{\nabla}^{a}(B_{i} - \dot{E}_{i})
+ (-4 \dot{\Omega}^2 \Omega^{-3} 
\nonumber\\
&&+ 2 \overset{..}{\Omega} \Omega^{-2} - 2 k \Omega^{-1})V_{i}=0,
\label{9.14}
\end{eqnarray}
%
%
\begin{eqnarray}
\Delta_{ij}&=& \tilde{\gamma}_{ij}\big[ 2 \dot{\Omega}^2 \Omega^{-2}(\alpha-\dot\gamma)
-2  \dot{\Omega} \Omega^{-1}(\dot\alpha -\ddot\gamma)-4\ddot\Omega\Omega^{-1}(\alpha-\dot\gamma)
+ \Omega^2 \delta \hat{p}
\nonumber\\
&&-\tilde\nabla_a\tilde\nabla^a( \alpha + 2\dot\Omega \Omega^{-1}\gamma) \big] 
+\tilde\nabla_i\tilde\nabla_j( \alpha + 2\dot\Omega \Omega^{-1}\gamma)
+\dot{\Omega} \Omega^{-1} \tilde{\nabla}_{i}(B_{j}-\dot E_j)
\nonumber\\
&&+\tfrac{1}{2} \tilde{\nabla}_{i}(\dot{B}_{j}-\ddot{E}_j)
+\dot{\Omega} \Omega^{-1} \tilde{\nabla}_{j}(B_{i}-\dot E_i)+\tfrac{1}{2} \tilde{\nabla}_{j}(\dot{B}_{i}-\ddot{E}_i)
\nonumber\\
&&- \overset{..}{E}_{ij} - 2 k E_{ij} - 2 \dot{E}_{ij} \dot{\Omega} \Omega^{-1} + \tilde{\nabla}_{a}\tilde{\nabla}^{a}E_{ij}=0,
\label{9.15}
\end{eqnarray}
%
%
\begin{eqnarray}
\tilde{\gamma}^{ij}\Delta_{ij} &=&  6 \dot{\Omega}^2 \Omega^{-2}(\alpha-\dot\gamma)
-6  \dot{\Omega} \Omega^{-1}(\dot\alpha -\ddot\gamma)-12\ddot\Omega\Omega^{-1}(\alpha-\dot\gamma)+ 3\Omega^2 \delta \hat{p}
\nonumber\\
&&-2\tilde\nabla_a\tilde\nabla^a( \alpha + 2\dot\Omega \Omega^{-1}\gamma)=0,
\label{9.16}
\end{eqnarray}
%
%
\begin{eqnarray}
g^{\mu\nu}\Delta_{\mu\nu}&=& 3 \delta \hat{p} -  \delta \hat{\rho}
-12 \overset{..}{\Omega}  \Omega^{-3}(\alpha - \dot\gamma) -6 \dot{\Omega} \Omega^{-3}(\dot{\alpha} -\ddot\gamma)
\nonumber\\
&&
-2 \Omega^{-2} \tilde{\nabla}_{a}\tilde{\nabla}^{a}(\alpha +3\dot\Omega\Omega^{-1}\gamma)=0.
\label{9.17}
\end{eqnarray}
%

Starting from the general identities
%
\begin{eqnarray}
\nabla_{k}\nabla_{n}T_{\ell m}-\nabla_{n}\nabla_{k}T_{\ell m}&=&T^{s}_{\phantom{s}m}R_{\ell s n k}+T_{\ell}^{\phantom{\ell}s}R_{ms n k},
\nonumber\\
\nabla_{k}\nabla_{n}A_{m}-\nabla_{n}\nabla_{k}A_{m}&=&A^{s}R_{ms n k}
\label{9.18}
\end{eqnarray}
%
that hold for any rank two tensor or vector in any geometry, for the 3-space Robertson-Walker geometry where $\tilde{R}_{msnk}=k(\tilde{\gamma}_{sn}\tilde{\gamma}_{mk}-\tilde{\gamma}_{mn}\tilde{\gamma}_{sk})$ 
we obtain
%
\begin{eqnarray}
&&\tilde\nabla_i\tilde\nabla_a\tilde\nabla^aA_j-\tilde\nabla_a\tilde\nabla^a\tilde\nabla_iA_j 
= 2k\tilde{\gamma}_{ij}\tilde{\nabla}_aA^a-2k(\tilde\nabla_i A_j + \tilde\nabla_j A_i),
\nonumber\\
&&\tilde\nabla^j\tilde\nabla_a\tilde\nabla^aA_j=
(\tilde\nabla_a\tilde\nabla^a+2k)\tilde\nabla^j A_j,\quad \tilde{\nabla}^j\tilde{\nabla}_iA_j=\tilde{\nabla}_i\tilde{\nabla}^jA_j+2kA_i
\label{9.19}
\end{eqnarray}
%
for any 3-vector $A_i$ in a maximally symmetric 3-geometry with 3-curvature $k$. Similarly, noting that for any scalar $S$ in any geometry we have
%
\begin{eqnarray}
&&\nabla_a\nabla_b\nabla_iS=\nabla_a\nabla_i\nabla_bS=\nabla_i\nabla_a\nabla_bS+\nabla^sSR_{bsia},
\nonumber\\
&&\nabla_{\ell}\nabla_k\nabla_n\nabla_{m}S=\nabla_n\nabla_{m}\nabla_{\ell}\nabla_kS
+\nabla_{n}[\nabla^sSR_{ksm\ell}]
+\nabla^s\nabla_kSR_{msn\ell}
\nonumber\\
&&\qquad\qquad+\nabla_m\nabla^sSR_{ksn\ell}
+\nabla_{\ell}[\nabla^sSR_{msnk}],
\label{9.20}
\end{eqnarray}
%
in a Robertson-Walker 3-geometry background we obtain 
%
\begin{eqnarray}
\tilde{\nabla}_a\tilde{\nabla}^a\tilde{\nabla}_iS&=&\tilde{\nabla}_i\tilde{\nabla}_a\tilde{\nabla}^aS+2k\tilde{\nabla}_iS,
\nonumber\\ \tilde{\nabla}_a\tilde{\nabla}^a\tilde{\nabla}_i\tilde{\nabla}_jS&=&\tilde{\nabla}_i\tilde{\nabla}_j\tilde{\nabla}_a\tilde{\nabla}^aS+6k(\tilde{\nabla}_i\tilde{\nabla}_j-\tfrac{1}{3}\tilde{\gamma}_{ij}\tilde{\nabla}_a\tilde{\nabla}^a)S,
\nonumber\\
\tilde{\nabla}_a\tilde{\nabla}^a\tilde{\nabla}_i\tilde{\nabla}_{j}S&=&\tilde{\nabla}_i\tilde{\nabla}_{j}\tilde{\nabla}_a\tilde{\nabla}^aS
+6k\tilde{\nabla}_i\tilde{\nabla}_{j}S-2k\tilde{\gamma}_{ij}\tilde{\nabla}_a\tilde{\nabla}^aS.
\label{9.21}
\end{eqnarray}
%
Thus we find that 
%
\begin{eqnarray}
\tilde\nabla^i \Delta_{0i} &=& 
\tilde\nabla_a\tilde\nabla^a\big[ -2 \dot{\Omega} \Omega^{-1} (\alpha - \dot\gamma) + 2 k \gamma 
+(-4 \dot{\Omega}^2 \Omega^{-3}  + 2 \overset{..}{\Omega} \Omega^{-2}  - 2 k \Omega^{-1}) \hat{V}\big]
\nonumber\\
&=&0,
\label{9.22}
\end{eqnarray}
%
and thus
%
\begin{eqnarray}
(\tilde{\nabla}_k\tilde\nabla^k -2k)\Delta_{0i} &=& (\tilde{\nabla}_k\tilde\nabla^k-2k)\bigg[k(B_i-\dot E_i)+ \tfrac{1}{2} \tilde{\nabla}_{a}\tilde{\nabla}^{a}(B_{i} - \dot{E}_{i})
\nonumber\\
&&
+ (-4 \dot{\Omega}^2 \Omega^{-3} + 2 \overset{..}{\Omega} \Omega^{-2} - 2 k \Omega^{-1})V_{i}\bigg]
=0.~~~~~~
\label{9.23}
\end{eqnarray}
%
Also we obtain
%
\begin{eqnarray}
\epsilon^{ij\ell}\tilde{\nabla}_j\Delta_{0i}&=&\epsilon^{ij\ell}\tilde{\nabla}_j\bigg[k(B_i-\dot E_i)+ \tfrac{1}{2} \tilde{\nabla}_{a}\tilde{\nabla}^{a}(B_{i} - \dot{E}_{i})
\nonumber\\
&&
+ (-4 \dot{\Omega}^2 \Omega^{-3} + 2 \overset{..}{\Omega} \Omega^{-2} - 2 k \Omega^{-1})V_{i}\bigg]=0.
\label{9.24}
\end{eqnarray}
%
Conferring Sec. \ref{ss:ds4_svt4}, we have noted that in a de Sitter space if a tensor $A^{P}_{\phantom{P}M}$ is transverse and traceless then so is $\nabla_L\nabla^LA^{P}_{\phantom{P}M}$ Since this holds in any maximally symmetric space the quantity  $\tilde{\nabla}_a\tilde{\nabla}^aE_{ij}$ is transverse and traceless too. Thus given (\ref{9.19}) we  obtain
%
\begin{eqnarray}
\tilde\nabla^j\Delta_{ij}&=& \tilde{\nabla}_{i}[ 2 \dot{\Omega}^2 \Omega^{-2}(\alpha-\dot\gamma)
-2  \dot{\Omega} \Omega^{-1}(\dot\alpha -\ddot\gamma)-4\ddot\Omega\Omega^{-1}(\alpha-\dot\gamma)+ \Omega^2 \delta \hat{p}
\nonumber\\
&&+ 2 k(\alpha + 2 \dot{\Omega}  \Omega^{-1} \gamma)]
+[ \tilde{\nabla}_{a}\tilde{\nabla}^{a}+2k][\tfrac{1}{2}(\dot{B}_i-\ddot{E}_i)+\dot{\Omega}\Omega^{-1}(B_i-\dot{E}_i)]=0,
\nonumber\\
\label{9.25}
\end{eqnarray}
%
%
\begin{eqnarray}
\tilde\nabla^i\tilde\nabla^j\Delta_{ij}&=& \tilde{\nabla}_{a}\tilde{\nabla}^{a}[2 \dot{\Omega}^2 \Omega^{-2}(\alpha-\dot\gamma)
-2  \dot{\Omega} \Omega^{-1}(\dot\alpha -\ddot\gamma)-4\ddot\Omega\Omega^{-1}(\alpha-\dot\gamma)+ \Omega^2 \delta \hat{p}
\nonumber \\ 
&& + 2 k(\alpha + 2 \dot{\Omega}  \Omega^{-1} \gamma)]=0.
\label{9.26}
\end{eqnarray}
%
Thus we obtain
%
\begin{eqnarray}
3\tilde\nabla^i\tilde\nabla^j\Delta_{ij}-\tilde\nabla_a\tilde\nabla^a(\tilde{\gamma}^{ij}\Delta_{ij})
=2\tilde{\nabla}^2[\tilde{\nabla}^2+3k](\alpha+2\dot{\Omega}\Omega^{-1}\gamma)=0,
\label{9.27}
\end{eqnarray}
%
%
\begin{eqnarray}
\tilde\nabla^i\tilde\nabla^j\Delta_{ij}+k\tilde{\gamma}^{ij}\Delta_{ij}
&=&[\tilde{\nabla}^2+3k][2 \dot{\Omega}^2 \Omega^{-2}(\alpha-\dot\gamma)
-2  \dot{\Omega} \Omega^{-1}(\dot\alpha -\ddot\gamma)
\nonumber\\
&&-4\ddot\Omega\Omega^{-1}(\alpha-\dot\gamma)+ \Omega^2 \delta \hat{p}]=0.
\label{9.28}
\end{eqnarray}
%
We now define $A=2 \dot{\Omega}^2 \Omega^{-2}(\alpha-\dot\gamma)-2  \dot{\Omega} \Omega^{-1}(\dot\alpha -\ddot\gamma)-4\ddot\Omega\Omega^{-1}(\alpha-\dot\gamma)+ \Omega^2 \delta \hat{p}$ and $C=\alpha+2\dot{\Omega}\Omega^{-1}\gamma$. And using (\ref{9.21}) obtain
%
\begin{eqnarray}
(\tilde{\nabla}_a\tilde{\nabla}^a+k)\tilde{\nabla}_i(A+2kC)=\tilde{\nabla}_i(\tilde{\nabla}_a\tilde{\nabla}^a+3k)(A+2kC),
\label{9.29}
\end{eqnarray}
%
and thus with  (\ref{9.27}) and (\ref{9.28}) obtain
%
\begin{eqnarray}
(\tilde{\nabla}_a\tilde{\nabla}^a-2k)(\tilde{\nabla}_a\tilde{\nabla}^a+k)\tilde{\nabla}_i(A+2kC)&=&
\tilde{\nabla}_i\tilde{\nabla}_a\tilde{\nabla}^a(\tilde{\nabla}_b\tilde{\nabla}^b+3k)(A+2kC)
\nonumber\\
&=&0.
\label{9.30}
\end{eqnarray}
%
Consequently, on comparing with (\ref{9.25})  we obtain
%
\begin{eqnarray}
(\tilde{\nabla}_a\tilde{\nabla}^a-2k)(\tilde{\nabla}_b\tilde{\nabla}^b+k)\tilde\nabla^j\Delta_{ij}&=&
(\tilde{\nabla}_a\tilde{\nabla}^a-2k)(\tilde{\nabla}_b\tilde{\nabla}^b+k)[\tilde{\nabla}_{c}\tilde{\nabla}^{c}+2k]\times
\nonumber\\
&&[\tfrac{1}{2}(\dot{B}_i-\ddot{E}_i)+\dot{\Omega}\Omega^{-1}(B_i-\dot{E}_i)]=0,
\label{9.31}
\end{eqnarray}
%
to give a relation that only involves $B_i-\dot{E}_{i}$.

To obtain a relation that involves $E_{ij}$ we proceed as follows. We note that sector of $\Delta_{ij}$ that contains the above $A$ and $C$ can be written as 
%
\begin{eqnarray}
D_{ij}=\tilde{\gamma}_{ij}(A-\tilde{\nabla}_a\tilde{\nabla}^aC)+\tilde{\nabla}_i\tilde{\nabla}_jC.
\label{9.32}
\end{eqnarray}
%
We thus introduce 
%
\begin{eqnarray}
A_{ij}&=&D_{ij}-\frac{1}{3}\tilde{\gamma}_{ij}\tilde{\gamma}^{ab}D_{ab}=(\tilde{\nabla}_i\tilde{\nabla}_j-\tfrac{1}{3} \tilde{\gamma}_{ij}\tilde{\nabla}_a\tilde{\nabla}^a)C,
\nonumber\\
B_{ij}&=&\Delta_{ij}-\frac{1}{3}\tilde{\gamma}_{ij}\tilde{\gamma}^{ab}\Delta_{ab}=(\tilde{\nabla}_i\tilde{\nabla}_j-\tfrac{1}{3} \tilde{\gamma}_{ij}\tilde{\nabla}_a\tilde{\nabla}^a)C
\nonumber\\
&+&\dot{\Omega} \Omega^{-1} \tilde{\nabla}_{i}(B_{j}-\dot E_j)+\tfrac{1}{2} \tilde{\nabla}_{i}(\dot{B}_{j}-\ddot{E}_j)
+\dot{\Omega} \Omega^{-1} \tilde{\nabla}_{j}(B_{i}-\dot E_i)+\tfrac{1}{2} \tilde{\nabla}_{j}(\dot{B}_{i}-\ddot{E}_i)
\nonumber\\
&-& \overset{..}{E}_{ij} - 2 k E_{ij} - 2 \dot{E}_{ij} \dot{\Omega} \Omega^{-1} + \tilde{\nabla}_{a}\tilde{\nabla}^{a}E_{ij}=0,
\label{9.33}
\end{eqnarray}
%
with (\ref{9.33}) defining $A_{ij}$ and $B_{ij}$, and with $A$ dropping out. Using (\ref{9.19}) and the third relation in (\ref{9.21}) we obtain
%
\begin{eqnarray}
(\tilde{\nabla}_b\tilde{\nabla}^b-3k)A_{ij}=
(\tilde{\nabla}_i\tilde{\nabla}_j-\tfrac{1}{3} \tilde{\gamma}_{ij}\tilde{\nabla}_a\tilde{\nabla}^a)(\tilde{\nabla}_b\tilde{\nabla}^b+3k)C,
\label{9.34}
\end{eqnarray}
%
and via (\ref{9.21}) and (\ref{9.22}) thus obtain 
%
\begin{eqnarray}
(\tilde{\nabla}_a\tilde{\nabla}^a-6k)(\tilde{\nabla}_b\tilde{\nabla}^b-3k)A_{ij}&=&
(\tilde{\nabla}_i\tilde{\nabla}_j-\tfrac{1}{3} \tilde{\gamma}_{ij}\tilde{\nabla}_a\tilde{\nabla}^a)\tilde{\nabla}_b\tilde{\nabla}^b(\tilde{\nabla}_c\tilde{\nabla}^c+3k)C
\nonumber\\
&=&0.
\label{9.35}
\end{eqnarray}
%
Comparing with the structure of $\Delta_{ij}$ and $\tilde{\gamma}^{ij}\Delta_{ij}$, we thus obtain
%
\begin{eqnarray}
&&(\tilde{\nabla}_a\tilde{\nabla}^a-6k)(\tilde{\nabla}_b\tilde{\nabla}^b-3k)[B_{ij}-A_{ij}]
=(\tilde{\nabla}_a\tilde{\nabla}^a-6k)(\tilde{\nabla}_b\tilde{\nabla}^b-3k)
\nonumber\\
&&\times
\big{[}\dot{\Omega} \Omega^{-1} \tilde{\nabla}_{i}(B_{j}-\dot E_j)+\tfrac{1}{2} \tilde{\nabla}_{i}(\dot{B}_{j}-\ddot{E}_j)
+\dot{\Omega} \Omega^{-1} \tilde{\nabla}_{j}(B_{i}-\dot E_i)+\tfrac{1}{2} \tilde{\nabla}_{j}(\dot{B}_{i}-\ddot{E}_i)
\nonumber\\
&& -\overset{..}{E}_{ij} - 2 k E_{ij} - 2  \dot{\Omega} \Omega^{-1}\dot{E}_{ij} + \tilde{\nabla}_{a}\tilde{\nabla}^{a}E_{ij}\big{]}=0.
\label{9.36}
\end{eqnarray}
%
We now note that for any vector $A_i$ that obeys $\tilde{\nabla}^iA_i=0$, through repeated use of the first relation in (\ref{9.19}) we obtain 
%
\begin{eqnarray}
&&(\tilde{\nabla}_b\tilde{\nabla}^b-3k)(\tilde{\nabla}_iA_j+\tilde{\nabla}_jA_i)=
\tilde{\nabla}_i(\tilde{\nabla}_b\tilde{\nabla}^b+k)A_j+
\tilde{\nabla}_j(\tilde{\nabla}_b\tilde{\nabla}^b+k)A_i,
\nonumber\\
&&(\tilde{\nabla}_a\tilde{\nabla}^a-6k)(\tilde{\nabla}_b\tilde{\nabla}^b-3k)(\tilde{\nabla}_iA_j+\tilde{\nabla}_jA_i)
=\tilde{\nabla}_i(\tilde{\nabla}_a\tilde{\nabla}^a-2k)(\tilde{\nabla}_b\tilde{\nabla}^b+k)A_j
\nonumber\\
&&\qquad\qquad+
\tilde{\nabla}_j(\tilde{\nabla}_a\tilde{\nabla}^a-2k)(\tilde{\nabla}_b\tilde{\nabla}^b+k)A_i.
\label{9.37}
\end{eqnarray}
%
On using the first relation in (\ref{9.19}) again,  it follows that 
%
\begin{eqnarray}
&&(\tilde{\nabla}_c\tilde{\nabla}^c-2k)(\tilde{\nabla}_a\tilde{\nabla}^a-6k)(\tilde{\nabla}_b\tilde{\nabla}^b-3k)(\tilde{\nabla}_iA_j+\tilde{\nabla}_jA_i)
\nonumber\\
&=&\tilde{\nabla}_i(\tilde{\nabla}_c\tilde{\nabla}^c+2k)(\tilde{\nabla}_a\tilde{\nabla}^a-2k)(\tilde{\nabla}_b\tilde{\nabla}^b+k)A_j
\nonumber\\
&&+
\tilde{\nabla}_j(\tilde{\nabla}_c\tilde{\nabla}^c+2k)(\tilde{\nabla}_a\tilde{\nabla}^a-2k)(\tilde{\nabla}_b\tilde{\nabla}^b+k)A_i.
\label{9.38}
\end{eqnarray}
%
On setting $A_i=\tfrac{1}{2}(\dot{B}_i-\ddot{E}_i)+\dot{\Omega}\Omega^{-1}(B_i-\dot{E}_i)$ (so that $A_i$ is such that $\tilde{\nabla}^iA_i=0$), and recalling (\ref{9.31}) we obtain
%
\begin{eqnarray}
&&(\tilde{\nabla}_c\tilde{\nabla}^c-2k)(\tilde{\nabla}_a\tilde{\nabla}^a-6k)(\tilde{\nabla}_b\tilde{\nabla}^b-3k)
\nonumber\\
&&\times\bigg{[}
\tilde{\nabla}_i[\tfrac{1}{2}(\dot{B}_j-\ddot{E}_j)+\dot{\Omega}\Omega^{-1}(B_j-\dot{E}_j)]+\tilde{\nabla}_j[\tfrac{1}{2}(\dot{B}_i-\ddot{E}_i)+\dot{\Omega}\Omega^{-1}(B_i-\dot{E}_i)]\bigg{]}\nonumber\\
&&=0.
\label{9.39}
\end{eqnarray}
%
Thus finally from (\ref{9.36}) we obtain
%
\begin{eqnarray}
&&(\tilde{\nabla}_c\tilde{\nabla}^c-2k)(\tilde{\nabla}_a\tilde{\nabla}^a-6k)(\tilde{\nabla}_b\tilde{\nabla}^b-3k)\times
\nonumber\\
&&
\big{[}
- \overset{..}{E}_{ij} - 2 k E_{ij} - 2  \dot{\Omega} \Omega^{-1}\dot{E}_{ij} + \tilde{\nabla}_{a}\tilde{\nabla}^{a}E_{ij}\big{]}=0.
\label{9.40}
\end{eqnarray}
%
Thus with (\ref{9.27}), (\ref{9.28}), (\ref{9.31}), (\ref{9.40}) together with (\ref{9.13}), (\ref{9.22}) and (\ref{9.23}) we have succeeded in decomposing the fluctuation equations for the components, with the various components obeying derivative equations that are higher than second order.

With (\ref{9.27}) only involving $\alpha +2\dot{\Omega}\Omega^{-1}\gamma$, with (\ref{9.31}) only involving 
$B_i-\dot{E}_i$, and with (\ref{9.40}) only involving $E_{ij}$, and with all components of $\Delta_{\mu\nu}$ being gauge invariant, we recognize $C=\alpha +2\dot{\Omega}\Omega^{-1}\gamma$, $B_i-\dot{E}_i$ and $E_{ij}$ as being gauge invariant. With $B_i-\dot{E}_i$ being gauge invariant, from (\ref{9.23}) we recognize $V_i$ as being gauge invariant too. While we have identified some gauge-invariant quantities we note that by manipulating $\Delta_{\mu\nu}$ so as to obtain derivative expressions in which each of these quantities appears on its own, we cannot establish the gauge invariance of all 11 of the fluctuation variables this way since $\Delta_{\mu\nu}$ only has 10 components. However, just as with fluctuations around flat spacetime, in analog to (\ref{2.6}) below we shall obtain derivative relations between the SVT3 fluctuations and the $h_{\mu\nu}$ fluctuations by manipulating  (\ref{9.2}). As we show below, this will enable us  to establish the gauge invariance of the remaining fluctuation quantities.

\subsubsection{Decomposition Theorem Requirements}
\label{sss:what_is_needed_decomposition_theorem_svt3}

To get a decomposition theorem for $\Delta_{\mu\nu}=0$ we would require 
%
\begin{eqnarray}
0&=&6 \dot{\Omega}^2 \Omega^{-2}(\alpha-\dot\gamma) + \delta \hat{\rho}{} \Omega^2 + 2 \dot{\Omega} \Omega^{-1} \tilde{\nabla}_{a}\tilde{\nabla}^{a}\gamma, 
\nonumber\\
0&=&-2 \dot{\Omega} \Omega^{-1} \tilde{\nabla}_{i}(\alpha - \dot\gamma) + 2 k \tilde{\nabla}_{i}\gamma 
+(-4 \dot{\Omega}^2 \Omega^{-3}  + 2 \overset{..}{\Omega} \Omega^{-2}  - 2 k \Omega^{-1}) \tilde{\nabla}_{i}\hat{V},
\nonumber\\
0&=&k(B_i-\dot E_i)+ \tfrac{1}{2} \tilde{\nabla}_{a}\tilde{\nabla}^{a}(B_{i} - \dot{E}_{i})
+ (-4 \dot{\Omega}^2 \Omega^{-3} + 2 \overset{..}{\Omega} \Omega^{-2} - 2 k \Omega^{-1})V_{i},
\nonumber\\
0&=&\tilde{\gamma}_{ij}\big[ 2 \dot{\Omega}^2 \Omega^{-2}(\alpha-\dot\gamma)
-2  \dot{\Omega} \Omega^{-1}(\dot\alpha -\ddot\gamma)-4\ddot\Omega\Omega^{-1}(\alpha-\dot\gamma)+ \Omega^2 \delta \hat{p}
\nonumber\\
&&-\tilde\nabla_a\tilde\nabla^a( \alpha + 2\dot\Omega \Omega^{-1}\gamma) \big] 
+\tilde\nabla_i\tilde\nabla_j( \alpha + 2\dot\Omega \Omega^{-1}\gamma),
\nonumber\\
0&=&\dot{\Omega} \Omega^{-1} \tilde{\nabla}_{i}(B_{j}-\dot E_j)+\tfrac{1}{2} \tilde{\nabla}_{i}(\dot{B}_{j}-\ddot{E}_j)
+\dot{\Omega} \Omega^{-1} \tilde{\nabla}_{j}(B_{i}-\dot E_i)+\tfrac{1}{2} \tilde{\nabla}_{j}(\dot{B}_{i}-\ddot{E}_i),
\nonumber\\
0&=&- \overset{..}{E}_{ij} - 2 k E_{ij} - 2 \dot{E}_{ij} \dot{\Omega} \Omega^{-1} + \tilde{\nabla}_{a}\tilde{\nabla}^{a}E_{ij},
\nonumber\\
0&=&6 \dot{\Omega}^2 \Omega^{-2}(\alpha-\dot\gamma)
-6  \dot{\Omega} \Omega^{-1}(\dot\alpha -\ddot\gamma)-12\ddot\Omega\Omega^{-1}(\alpha-\dot\gamma)+ 3\Omega^2 \delta \hat{p}
\nonumber\\
&&-2\tilde\nabla_a\tilde\nabla^a( \alpha + 2\dot\Omega \Omega^{-1}\gamma),
\nonumber\\
0&=&3 \delta \hat{p}{} -  \delta \hat{\rho}
-12 \overset{..}{\Omega}  \Omega^{-3}(\alpha - \dot\gamma) -6 \dot{\Omega} \Omega^{-3}(\dot{\alpha} -\ddot\gamma)
-2 \Omega^{-2} \tilde{\nabla}_{a}\tilde{\nabla}^{a}(\alpha +3\dot\Omega\Omega^{-1}\gamma).
\nonumber\\
\label{9.41}
\end{eqnarray}
%
With $\tilde{\gamma}_{ij}$ and $\tilde{\nabla}_{i}\tilde{\nabla}_j$ not being equal to each other, we would immediately obtain 
%
\begin{eqnarray}
2 \dot{\Omega}^2 \Omega^{-2}(\alpha-\dot\gamma)
-2  \dot{\Omega} \Omega^{-1}(\dot\alpha -\ddot\gamma)-4\ddot\Omega\Omega^{-1}(\alpha-\dot\gamma)+ \Omega^2 \delta \hat{p}  &=&0,
\nonumber\\
\alpha + 2\dot\Omega \Omega^{-1}\gamma&=&0.
\label{9.42}
\end{eqnarray}
%
We recognize the equations for the components of the fluctuations as being derivatives of the relations that are required of the decomposition theorem. We thus need to see if we can find boundary conditions that would force the solutions to the higher-derivative fluctuation equations to obey (\ref{9.41}) and (\ref{9.42}).

\subsubsection{Gauge Invariance}
\label{sss:establishing_gauge_invariance}

Starting with (\ref{9.2}), setting $h_{\mu\nu}=\Omega^2(\tau)f_{\mu\nu}$,  $f=\tilde{\gamma}^{ij}f_{ij}=-6\psi+2\tilde{\nabla}_i\tilde{\nabla}^iE$ and taking appropriate derivatives, then following quite a bit of algebra we obtain
%
\begin{eqnarray}
(3k+\tilde{\nabla}^b\tilde{\nabla}_b)\tilde{\nabla}^a\tilde{\nabla}_a\alpha&=&-\frac{1}{2}(3k+\tilde{\nabla}^b\tilde{\nabla}_b)\tilde{\nabla}^i\tilde{\nabla}_if_{00}
\nonumber\\
&&
+\frac{1}{4}\tilde{\nabla}^a\tilde{\nabla}_a\left(-2kf-\tilde{\nabla}^b\tilde{\nabla}_bf+\tilde{\nabla}^m\tilde{\nabla}^nf_{mn}\right)
\nonumber\\
&&
+\partial_0(3k+\tilde{\nabla}^b\tilde{\nabla}_b)\tilde{\nabla}^if_{0i}
-\frac{1}{4}\partial^2_0\left(3\tilde{\nabla}^m\tilde{\nabla}^nf_{mn}-\tilde{\nabla}^a\tilde{\nabla}_af\right),
\nonumber\\
\label{9.43a}
\end{eqnarray}
%
%
\begin{eqnarray}
(3k+\tilde{\nabla}^b\tilde{\nabla}_b)\tilde{\nabla}^a\tilde{\nabla}_a\gamma
&=&-\frac{1}{4}\Omega\dot{\Omega}^{-1}\tilde{\nabla}^a\tilde{\nabla}_a\left(-2kf-\tilde{\nabla}^b\tilde{\nabla}_bf+\tilde{\nabla}^m\tilde{\nabla}^nf_{mn}\right)
\nonumber\\
&&
+\bigg[(3k+\tilde{\nabla}^b\tilde{\nabla}_b)\tilde{\nabla}^if_{0i}-\frac{1}{4}\partial_0\big(3\tilde{\nabla}^m\tilde{\nabla}^nf_{mn}
\nonumber\\
&&-\tilde{\nabla}^a\tilde{\nabla}_af\big)\bigg],
\label{9.44a}
\end{eqnarray}
%
%
\begin{eqnarray}
(\tilde{\nabla}^a\tilde{\nabla}_a-2k)(\tilde{\nabla}^i\tilde{\nabla}_i +2k)(B_j-\dot{E_j})&=&(\tilde{\nabla}^i\tilde{\nabla}_i +2k)(\tilde{\nabla}^a\tilde{\nabla}_af_{0j}-2kf_{0j}
\nonumber\\
&&
-\tilde{\nabla}_j\tilde{\nabla}^af_{0a})
-\partial_0\tilde{\nabla}^a\tilde{\nabla}_a\tilde{\nabla}^if_{ij}
\nonumber\\
&&
+\partial_0\tilde{\nabla}_j\tilde{\nabla}^a\tilde{\nabla}^bf_{ab}
+2k\partial_0\tilde{\nabla}^if_{ij},
\label{9.45a}
\end{eqnarray}
%


%
\begin{eqnarray}
&&2(3k+\tilde{\nabla}^c\tilde{\nabla}_c)(-3k+\tilde{\nabla}^b\tilde{\nabla}_b)(\tilde{\nabla}^a\tilde{\nabla}_a-6k)(\tilde{\nabla}^d\tilde{\nabla}_d-2k)E_{ij}
\nonumber
\\
&&=(3k+\tilde{\nabla}^c\tilde{\nabla}_c)(-3k+\tilde{\nabla}^b\tilde{\nabla}_b)(\tilde{\nabla}^a\tilde{\nabla}_a-6k)(\tilde{\nabla}^d\tilde{\nabla}_d-2k)f_{ij}
\nonumber
\\
&&+\frac{1}{2}(-3k+\tilde{\nabla}^c\tilde{\nabla}_c)(\tilde{\nabla}^b\tilde{\nabla}_b-6k)(\tilde{\nabla}^a\tilde{\nabla}_a-2k)(-2kf-\tilde{\nabla}^d\tilde{\nabla}_df+\tilde{\nabla}^m\tilde{\nabla}^nf_{mn}) \tilde{\gamma}_{ij}
\nonumber
\\
&&-2(\tilde{\nabla}^c\tilde{\nabla}_c-2k)\Bigg[\frac{1}{4}(3k+\tilde{\nabla}^b\tilde{\nabla}_b)\tilde{\nabla}_i\tilde{\nabla}_j\left(3\tilde{\nabla}^m\tilde{\nabla}^nf_{mn}-\tilde{\nabla}^a\tilde{\nabla}_af\right)
\nonumber
\\
&&-k\tilde{\gamma}_{ij}\tilde{\nabla}_d\tilde{\nabla}^d((3k+\tilde{\nabla}^e\tilde{\nabla}_e)f+\frac{3}{2}(-2kf-\tilde{\nabla}^g\tilde{\nabla}_gf+\tilde{\nabla}^m\tilde{\nabla}^nf_{mn}))
\nonumber
\\
&&-k\tilde{\gamma}_{ij}(-3k+\tilde{\nabla}^h\tilde{\nabla}_h)((3k+\tilde{\nabla}^m\tilde{\nabla}_m)f+\frac{3}{2}(-2kf-\tilde{\nabla}^n\tilde{\nabla}_nf+\tilde{\nabla}^m\tilde{\nabla}^nf_{mn}))\Bigg]
\nonumber
\\
&&-\frac{1}{3}(-3k+\tilde{\nabla}^c\tilde{\nabla}_c)(3k+\tilde{\nabla}^b\tilde{\nabla}_b)\Bigg(\tilde{\nabla}_i\tilde{\nabla}^a\tilde{\nabla}_a(3\tilde{\nabla}^df_{dj}-\tilde{\nabla}_jf)
\nonumber
\\
&&+\tilde{\nabla}_j\tilde{\nabla}^d\tilde{\nabla}_d(3\tilde{\nabla}^bf_{bi}-\tilde{\nabla}_if)-2\tilde{\nabla}_i\tilde{\nabla}_j(3\tilde{\nabla}^b\tilde{\nabla}^cf_{bc}-\tilde{\nabla}^e\tilde{\nabla}_ef)
\nonumber
\\
&&-2k\tilde{\nabla}_i(3\tilde{\nabla}^af_{aj}-\tilde{\nabla}_jf)-2k\tilde{\nabla}_j(3\tilde{\nabla}^af_{ai}-\tilde{\nabla}_if)\Bigg).
\label{9.46a}
\end{eqnarray}
%
Despite its somewhat forbidding appearance (\ref{9.46a}) is actually a derivative of 
%
\begin{align}
&2(\tilde{\nabla}^a\tilde{\nabla}_a-2k)(\tilde{\nabla}^b\tilde{\nabla}_b-3k)E_{ij}
=(\tilde{\nabla}^a\tilde{\nabla}_a-2k)(\tilde{\nabla}^b\tilde{\nabla}_b-3k)f_{ij}
\nonumber\\
&+\tfrac{1}{2}\tilde{\nabla}_i\tilde{\nabla}_j\left[\tilde{\nabla}^a\tilde{\nabla}^bf_{ab}+(\tilde{\nabla}^a\tilde{\nabla}_a+4k)f\right]-(\tilde{\nabla}^a\tilde{\nabla}_a-3k)(\tilde{\nabla}_i\tilde{\nabla}^bf_{jb}+\tilde{\nabla}_j\tilde{\nabla}^bf_{ib})
\nonumber\\
&+\tfrac{1}{2}\tilde{\gamma}_{ij}\left[(\tilde{\nabla}^a\tilde{\nabla}_a-4k)\tilde{\nabla}^b\tilde{\nabla}^cf_{bc}
-(\tilde{\nabla}_a\tilde{\nabla}^a\tilde{\nabla}_b\tilde{\nabla}^b-2k\tilde{\nabla}^a\tilde{\nabla}^a+4k^2)f\right],
\label{9.47a}
\end{align}
%
a relation that itself can be  derived from the $D=3$ version of (\ref{6.4}) with $H^2=k$ by application  of $(\tilde{\nabla}^a\tilde{\nabla}_a-2k)(\tilde{\nabla}^a\tilde{\nabla}_a-3k)$ to (\ref{6.4}). One can check the validity of these relations by inserting 
%
\begin{eqnarray}
h_{0i}&=&\Omega^2(\tau)f_{0i}=\Omega^2(\tau)[\tilde{\nabla}_iB+B_i],
\\
h_{ij}&=&\Omega^2(\tau)f_{ij}=\Omega^2(\tau)[-2\psi\tilde{\gamma}_{ij} +2\tilde{\nabla}_i\tilde{\nabla}_j E + \tilde{\nabla}_i E_j + \tilde{\nabla}_j E_i + 2E_{ij}]
\nonumber
\label{9.48a}
\end{eqnarray}
%
into them. And one can check their gauge invariance by inserting $h_{\mu\nu}\rightarrow h_{\mu\nu}-\nabla_{\mu}\epsilon_{\nu}-\nabla_{\mu}\epsilon_{\mu}$ into them. We thus establish that the metric fluctuations $\alpha$, $\gamma$, $B_i-\dot{E}_i$ and $E_{ij}$ are gauge invariant. And from  (\ref{9.13}), (\ref{9.22}), (\ref{9.23}) and (\ref{9.28}) can thus establish that the matter fluctuations $\delta \hat{\rho}$, $\hat{V}$, $V_i$ and $\delta \hat{p}$ are gauge invariant too. Interestingly, we see that in going from fluctuations around flat to fluctuations around Robertson-Walker with arbitrary $k$ and arbitrary dependence of $\Omega(\tau)$ on $\tau$ the gauge-invariant metric fluctuation combinations $\alpha$, $\gamma$, $B_i-\dot{E}_i$ and $E_{ij}$  remain the same, though $\gamma$ does depend generically on $\Omega(\tau)$. 


\subsubsection{Background Solution}
\label{sss:solving_the_background_svt3}

In order to actually solve the fluctuation equations we will need to determine the appropriate background $\Omega(\tau)$, and we will also need to deal with the fact that, as noted above,  the fluctuation equations contain more degrees of freedom (11) than there are evolution equations (10). For the background first we note that no matter what the value of $k$, from (\ref{9.10}) we see that if $\rho=3p$ then $\rho=3/\Omega^4$,  as written in a convenient normalization (one which differs from the one used in Sec. \ref{S8}), while if $p=0$ we have $\rho=3/\Omega^3$. Once we specify a background equation of state that relates $\rho$ and $p$ we can solve for $\Omega (\tau)$ and $t=\int \Omega(\tau)d\tau$. We thus obtain 
%
\begin{eqnarray}
p=\rho/3,~k=0:&&\quad \Omega=\tau,\quad p=1/\tau^4,\quad \rho=3/\tau^4,\quad t=\tau^2/2,
\nonumber\\
&&\qquad a(t)=\Omega(\tau)=(2t)^{1/2},
\nonumber\\
p=\rho/3,~k=-1:&&\quad \Omega=\sinh\tau,\quad p=1/\sinh^4\tau,\quad \rho=3/\sinh^4\tau,
\nonumber\\
&& \qquad t=\cosh\tau,\quad a(t)=\Omega(\tau)=(t^2-1)^{1/2},
\nonumber\\
p=\rho/3,~k=+1:&&\quad \Omega=\sin\tau,\quad p=1/\sin^4\tau,\quad \rho=3/\sin^4\tau,\quad t=-\cos\tau,
\nonumber\\
&& \qquad a(t)=\Omega(\tau)=(1-t^2)^{1/2},
\nonumber\\
p=0,~k=0:&&\quad \Omega=\tau^2/4,\quad p=0,\quad \rho=192/\tau^6,\quad t=\tau^{3}/12,
\nonumber\\
&&\qquad a(t)=\Omega(\tau)=(3t/2)^{2/3},
\nonumber\\
p=0,~k=-1:&&\quad \Omega=\sinh^2(\tau/2),\quad p=0,\quad \rho=3/\sinh^6(\tau/2),
\nonumber\\
&& \qquad t=\tfrac{1}{2}[\sinh\tau-\tau],\quad a(t)=\Omega(\tau),
\nonumber\\
p=0,~k=1:&&\quad \Omega=\sin^2(\tau/2),\quad p=0,\quad \rho=3/\sin^6(\tau/2),
\nonumber\\
&& \qquad t=\tfrac{1}{2}[\tau-\sin\tau],\quad a(t)=\Omega(\tau).
\label{9.49}
\end{eqnarray}
%
For $p=0$ and $k=\pm 1$ we cannot obtain $a(t)$ in a closed form. Consequently one ordinarily only determines $a(t)$ in parametric form. As we see, the conformal time $\tau$ can serve as the appropriate parameter.

\subsubsection{Relation between $\delta\rho$ and $\delta p$}
\label{sss:relating_dp_drho_svt3}

To reduce the number of fluctuation variables from 11 to 10 we follow kinetic theory, and first consider a relativistic flat spacetime ideal $N$ particle classical gas of spinless particles each of mass $m$ in a volume $V$ at a temperature $T$. As discussed for instance in \cite{mannheim_2006}, for this
system one can use a basis of momentum eigenmodes, with the Helmholtz free energy $A(V,T)$ being given as 
%                                                                               
\begin{equation}
e^{-A(V,T)/NkT}=V\int
d^3pe^{-(p^2+m^2)^{1/2}/kT}, 
\label{9.50}
\end{equation}                                 
%  
so that the pressure takes the simple form 
%                                                                               
\begin{equation}
p=-\left(\frac{\partial A}{ \partial
	V}\right)_T=\frac{NkT}{V},
\label{9.51}
\end{equation}                                 
% 
while the internal energy $U=\rho V$ evaluates in terms of Bessel
functions as  
%                                                                               
\begin{equation}
U=A-T\left(\frac{\partial A}{ \partial
	T}\right)_V=3NkT+Nm\frac{K_1(m/kT)}{K_2(m/kT)}.
\label{9.52}
\end{equation}                                 
% 
In the high and low temperature limits (the radiation and matter eras)
we then find that the expression for $U$ simplifies to
%                                                                               
\begin{eqnarray}
\rho=\frac{U}{V}\rightarrow
\frac{3NkT}{V}=3p,&&\quad \frac{m}{kT}\rightarrow 0,
\nonumber \\
\rho=\frac{U}{V} \rightarrow
\frac{Nm}{V}+\frac{3NkT}{2V}=\frac{Nm}{V}+\frac{3p}{2} \approx
\frac{Nm}{V},&&\quad \frac{m}{kT}
\rightarrow \infty.
\label{9.53}
\end{eqnarray}                                 
% 
Consequently, while $p$ and $\rho$ are nicely proportional to each other ($p=w\rho$)
in the high temperature radiation and the low temperature matter eras
(where $w(T\rightarrow\infty)=1/3$ and $w(T\rightarrow 0)=0$), we also
see that in transition region between the two eras their relationship is
altogether more complicated. Since such a transition era occurs fairly close to recombination, it is this complicated relation that should be used there. Use of a $p=w\rho$ equation of state would at best only be valid at temperatures which are very different from
those of order $m/K$, though for massless particles it would be of
course be valid to use $p=\rho/3$ at all temperatures. 

As derived, these expressions only hold in a flat  Minkowski spacetime. However, $A(V,T)$ only involves  an integration over the spatial 3-momentum. Thus for the spatially flat $k=0$ Robertson-Walker metric $ds^2=dt^2-a^2(t)(dr^2+r^2d\theta^2+r^2\sin^2\theta d\phi^2)$, all of these kinetic theory relations will continue to hold with $T$ taken to depend on the comoving time $t$. (Typically $T\sim 1/a(t)$.) Suppose we now perturb the system and obtain a perturbed $\delta T$ that now depends on both $t$ and $r,\theta,\phi$. In the radiation era where $p=\rho/3$ we would obtain $\delta p =\delta \rho/3$ (and thus $3\delta\hat{p}=\delta\hat{\rho}$). In the matter era where $p=0$, from (\ref{9.53}) we would obtain $\delta p=2\delta \rho/3$. In the intermediate region the relation would be much more complicated. Nonetheless in all cases we would have reduced the number of independent fluctuation variables, though we note that not just in the radiation and matter eras but even in the intermediate region, it is standard in cosmological perturbation theory to use $\delta p/\delta \rho=v^2$ where $v^2$ is taken to be a spacetime-independent constant.

While we can use the above $A(V,T)$ for spatially flat cosmologies with $k=0$, for spatially curved cosmologies with non-zero $k$ we cannot use a mode basis made out of 3-momentum eigenstates at all. One has to adapt the basis to a curved 3-space by replacing
$(p^2+m^2)^{1/2}/kT$ by $(dx^{\mu}/d\tau)U^{\nu}g_{\mu\nu}/kT$ (see e.g. \cite{mannheim_2006}), while replacing
$\int d^3p$ by a sum over a complete set of basis modes associated with
the propagation of a spinless massive particle in the chosen
$g_{\mu\nu}$ background, and then follow the steps above to see what
generalization of (\ref{9.53}) might then ensue. While tractable in principle it is not straightforward to do this in practice, and we will not do it here. While one would need to do this in order to obtain a $k\neq 0$ generalization of the $k=0$  $\delta p/\delta \rho=v^2$ relation, and while such a generalization would be needed in order to solve the $k\neq 0$ fluctuation equations completely, since our purpose here is only to test for the validity of the decomposition theorem, we will not actually need to find a relation between $\delta p$ and $\delta \rho$, since as we now see, we will be able to test for the validity of the decomposition theorem without actually needing to know the specific form of such a relation at all, or even needing to specify any particular form for the background $\Omega(\tau)$ either for that matter.
%%%%%%%%%%%%%%%%%%%%%%%%%%%%%%%%%%%%%%%%%%%%
\subsection{Robertson Walker $k=-1$}
\label{ss:rw_k=-1_svt3}
%%%%%%%%%%%%%%%%%%%%%%%%%%%%%%%%%%%%%%%%%%%%

\subsubsection{The Scalar Sector}
\label{sss:scalar_sector}

We have seen that the scalar sector evolution equations (\ref{9.22}), (\ref{9.26}), (\ref{9.27}) and (\ref{9.28}) involve derivatives of the form $\tilde{\nabla}^2$, $\tilde{\nabla}^2+3k$ where the coefficient of $k$ is either zero or positive, while the vector and tensor sectors equations (\ref{9.23}), (\ref{9.31}) and (\ref{9.40}) also involve derivatives such as $\tilde{\nabla}^2-2k$, $\tilde{\nabla}^2-3k$, and $\tilde{\nabla}^2-6k$ in which the coefficient of $k$ is negative. As the implications of boundary conditions are very sensitive to the sign of the coefficient of $k$, and we will need to monitor both positive and negative coefficient cases below. In implementing evolution equations that involve products of derivative operators such as the generic $(\tilde{\nabla}^2+\alpha)(\tilde{\nabla}^2+\beta)F=0$ ($F$ denotes scalar, vector or tensor), we can satisfy these relations by $(\tilde{\nabla}^2+\alpha)F=0$,  by $(\tilde{\nabla}^2+\beta)F=0$, or by $F=0$. The decomposition theorem will only follow if boundary conditions prevent us from satisfying $(\tilde{\nabla}^2+\alpha)F=0$ or   $(\tilde{\nabla}^2+\beta)F=0$ with non-zero $F$, leaving $F=0$ as the only remaining possibility. It is the purpose of this section to explore whether or not boundary conditions do force us to $F=0$ in any of the scalar, vector or tensor sectors. While a decomposition theorem would immediately hold if they do, as we will show in Sec. \ref{ss:recovering_decomposition_theorem} the interplay of the vector and tensor sectors in the $\Delta_{ij}=0$ relation given in (\ref{9.15}) will still force us to a decomposition theorem even if they do not.


To illustrate what is involved it is sufficient to restrict $k$ to $k=-1$, and to take the metric to be of the form 
%
\begin{eqnarray}
ds^2=\Omega^2(\tau)\left[ d\tau^2-d\chi^2-\sinh^2\chi d\theta^2-\sinh^2\chi\sin^2\theta d\phi^2\right],
\label{10.1b}
\end{eqnarray}
%
where $r=\sinh \chi$. Since the analysis leading to the structure for $\Delta_{\mu\nu}$ given in (\ref{9.13}) to (\ref{9.17}) is completely covariant these equations equally hold if we represent the spatial sector of the metric as given in (\ref{10.1b}). With $k=-1$ the scalar sector evolution equations involving the $\tilde{\nabla}^2$ and $\tilde{\nabla}^2-3$ operators take the form
%
\begin{eqnarray}
(\tilde{\nabla}_a\tilde{\nabla}^a+A_S)S=0.
\label{10.2b}
\end{eqnarray}
%
(Here $S$ is to denote the full combinations of scalar sector  components that appear in  (\ref{9.22}), (\ref{9.26}), (\ref{9.27}) and (\ref{9.28}).)  In (\ref{10.2b}) we have introduced a generic scalar sector constant $A_S$, whose values in  (\ref{9.22}), (\ref{9.26}), (\ref{9.27}) and (\ref{9.28})  are $(0,-3)$. On setting $S(\chi,\theta,\phi)=S_{\ell}(\chi)Y^m_{\ell}(\theta,\phi)$ (\ref{10.2b}) reduces to 
%  
\begin{eqnarray}
\left[\frac{d^2}{d\chi^2}+2\frac{\cosh\chi }{\sinh\chi}\frac{d }{ d\chi}
-\frac{\ell(\ell+1)}{ \sinh^2\chi}+A_S\right]S_{\ell}=0.
\label{10.3b}
\end{eqnarray}
%
In the  $\chi\rightarrow \infty$ and $\chi\rightarrow 0$ limits  we take the solution to behave as $e^{\lambda \chi}$ (times an irrelevant polynomial in $\chi$), and as $\chi^n$, to thus obtain
%
\begin{eqnarray}
&&\lambda^2+2\lambda+A_S=0,\quad \lambda=-1\pm(1-A_S)^{1/2},
\nonumber\\
&&\lambda(A_S=0)=(-2,~0),\quad \lambda(A_S=-3)=(-3,~1),
\nonumber\\
&&n(n-1)+2n-\ell(\ell+1)=0,\quad n=\ell,-\ell-1.
\label{10.4b}
\end{eqnarray}
%
For each of $A_S=0$ and  $A_S=-3$ one solution converges at $\chi=\infty$ and the other diverges at $\chi=\infty$. Thus we need to see how they match up with the solutions at $\chi=0$, where one solution is well-behaved and the other is not. 

So to this end we look for exact solutions to (\ref{10.3b}). Thus,  as discussed for instance in \cite{bander_itzykson_1966,mannheim_kazanas_1988}, and as appropriately generalized here,  (\ref{10.3b}) 
admits of solutions (known as associated Legendre functions) of the form 
%
\begin{eqnarray}
S_{\ell}=\sinh^{\ell}\chi\left(\frac{1}{ \sinh\chi} \frac{d }{ d\chi}\right)^{\ell+1}f(\chi),
\label{10.5b}
\end{eqnarray}
%
where $f(\chi)$ obeys
%
\begin{eqnarray}
\left[\frac{d^3}{d\chi^3}+\nu^2\frac{d}{d\chi}\right]f(\chi)=0,\quad \nu^2=A_S-1,
\label{10.6b}
\end{eqnarray}
%
with $f(\chi)$ thus obeying 
%
\begin{eqnarray}
f(\nu^2>0)&=&\cos\nu\chi,~\sin\nu\chi,\quad f(\nu^2=-\mu^2<0)=\cosh\mu\chi,~\sinh\mu\chi,
\nonumber\\
f(\nu^2=0)&=&\chi,~\chi^2.
\label{10.7b}
\end{eqnarray}
%
For each $f(\chi)$ this would lead to solutions of the form
%
\begin{eqnarray}
\hat{S}_0&=&\frac{1}{\sinh\chi}\frac{df}{d\chi},\quad \hat{S}_1=\frac{d\hat{S}_0}{d\chi},
\nonumber\\ \hat{S}_2&=&\sinh\chi\frac{d}{d\chi}\left[\frac{\hat{S}_1}{\sinh\chi}\right],\quad \hat{S}_3=\sinh^2\chi\frac{d}{d\chi}\left[\frac{\hat{S}_2}{\sinh^2\chi}\right],....
\label{10.8b}
\end{eqnarray}
%
However, on evaluating these expressions it can happen that some of these solutions vanish. Thus for $A_S=0$ for instance where $f(\chi)=(\sinh\chi,\cosh\chi)$ the two solutions with $\ell=0$ are $\cosh\chi/\sinh\chi$ and $1$. However this would lead to the two solutions with $\ell=1$ being $1/\sinh^2\chi$ and $0$. To address this point we note that suppose we have obtained some non-zero solution $\hat{S}_{\ell}$. Then, a second solution of the form $\hat{f}_{\ell}(\chi)\hat{S}_{\ell}(\chi)$ may be found by inserting $\hat{f}_{\ell}(\chi)\hat{S}_{\ell}(\chi)$ into (\ref{10.3b}), to yield
%
\begin{eqnarray}
\hat{S}_{\ell}\frac{d^2 \hat{f}_{\ell}}{ d\chi^2}+2\hat{S}_{\ell}\frac{\cosh\chi }{ \sinh\chi}\frac{d \hat{f}_{\ell}}{ d\chi}+2\frac{d \hat{S}_{\ell}}{ d\chi}\frac{d \hat{f}_{\ell}}{ d\chi}=0,
\label{10.9b}
\end{eqnarray}
%
which integrates to
%
\begin{eqnarray}
\frac{d \hat{f}_{\ell}}{ d\chi}=\frac{1}{\sinh^2\chi\hat{S}_{\ell}^2},~~~~~\hat{f}_{\ell}\hat{S}_{\ell}=\hat{S}_{\ell}\int \frac{d\chi }{\sinh^2\chi\hat{S}_{\ell}^2}.
\label{10.10b}
\end{eqnarray}
%
Thus for $\ell=1$, from the non-trivial $A_S=0$ solution $\hat{S}_{1}=1/\sinh^2\chi$ we obtain a second solution of the form $\hat{f}_{\ell}\hat{S}_{\ell}=\cosh\chi/\sinh\chi-\chi/\sinh^2\chi$. However, once we have this second solution we can then return to (\ref{10.8b}) and use it to obtain the subsequent solutions associated with higher $\ell$ values, since use of the chain in (\ref{10.8b}) only requires that at any point the elements in it are solutions regardless of how they may or may not have been found.

Having the form given in (\ref{10.10b}) is  useful for another purpose, as it allows us to relate the behaviors of the solutions in the $\chi\rightarrow \infty$ and $\chi\rightarrow 0$ limits. Thus suppose that $\hat{S}_{\ell}$ behaves as $e^{\lambda\chi}$ and as $\chi^{\ell}$ in these two limits. Then $\hat{f}_{\ell}\hat{S}_{\ell}$ must behave as $e^{-(\lambda+2)\chi}$ and $\chi^{-\ell-1}$ in the two limits. Alternatively, if $\hat{S}_{\ell}$ behaves as $e^{\lambda\chi}$ and as $\chi^{-\ell-1}$ in these two limits, then $\hat{f}_{\ell}\hat{S}_{\ell}$ must behave as $e^{-(\lambda+2)\chi}$ and $\chi^{\ell}$ in the two limits. Comparing with (\ref{10.4b}), we note that if we set $\lambda=-1\pm(1-A_S)^{1/2}$ then consistently we find that $-(\lambda+2)=-1\mp(1-A_S)^{1/2}$. However, this analysis shows that we cannot directly identify which  $\chi\rightarrow \infty$ behavior is associated with which $\chi\rightarrow 0$ behavior (the insertion of either $\chi^{\ell}$ or $\chi^{-\ell -1}$ into (\ref{10.10b}) generates the other, with both behaviors thus being required in any $\hat{S}_{\ell}$, $\hat{f}_{\ell}\hat{S}_{\ell}$ pair), and to determine which is which we thus need to construct the asymptotic solutions directly.

For $A_S=0$, $\nu=i$,  the relevant $f(\nu^2)$ given in (\ref{10.7b}) are $\cosh \chi$ and $\sinh \chi$. 
Consequently, we find the first few $S^{(i)}_{\ell}$, $i=1,2$ solutions to $\tilde{\nabla}_a\tilde{\nabla}^aS=0$ to be of the form 
%
\begin{align}
&\hat{S}^{(1)}_{0}(A_S=0)=\frac{\cosh\chi}{\sinh\chi},\quad \hat{S}^{(2)}_{0}(A_S=0)=1,
\nonumber\\
&\hat{S}^{(1)}_{1}(A_S=0)=\frac{1}{\sinh^2\chi},\quad \hat{S}^{(2)}_{1}(A_S=0)=\frac{\cosh\chi}{\sinh\chi}-\frac{\chi}{\sinh^2\chi},
\nonumber\\
&\hat{S}^{(1)}_{2}(A_S=0)=\frac{\cosh\chi}{\sinh^3\chi},\quad \hat{S}^{(2)}_{2}(A_S=0)=1+\frac{3}{\sinh^2\chi}-\frac{3\chi\cosh\chi}{\sinh^3\chi},
\nonumber\\
&\hat{S}^{(1)}_{3}(A_S=0)=\frac{4}{\sinh^2\chi}+\frac{5}{\sinh^4\chi},
\nonumber\\
& \hat{S}^{(2)}_{3}(A_S=0)=
\frac{2\cosh\chi}{\sinh\chi}+\frac{15\cosh\chi}{\sinh^3\chi}-\frac{12\chi}{\sinh^2\chi}-\frac{15\chi}{\sinh^4\chi}.
\label{10.11b}
\end{align}
%
From this pattern we see that the solutions that are bounded at $\chi=\infty$ are badly-behaved at $\chi=0$, while the solutions that are  well-behaved at $\chi=0$ are unbounded at $\chi=\infty$. Thus all of these $A_S=0$ solutions are excluded by a requirement that solutions be  bounded at $\chi=\infty$ and be well-behaved at $\chi=0$.

For $A_S=-3$, $\nu=2i$, the relevant $f(\nu^2)$ given in (\ref{10.7b}) are $\cosh 2\chi$ and $\sinh 2\chi$. Consequently, the first few solutions  to $(\tilde{\nabla}_a\tilde{\nabla}^a-3)S=0$ are of the form
%
\begin{eqnarray}
&&\hat{S}^{(1)}_0(A_S=-3)=\cosh\chi,\quad \hat{S}^{(2)}_0(A_S=-3)=2\sinh\chi+\frac{1}{\sinh\chi},
\nonumber\\
&&\hat{S}^{(1)}_1(A_S=-3)=\sinh\chi,\quad \hat{S}^{(2)}_1(A_S=-3)=2\cosh\chi-\frac{\cosh\chi}{\sinh^2\chi},
\nonumber\\
&&\hat{S}^{(1)}_2(A_S=-3)=2\cosh\chi-\frac{3\cosh\chi}{\sinh^2\chi}+\frac{3\chi}{\sinh^3\chi},\quad \hat{S}^{(2)}_2(A_S=-3)=\frac{1}{\sinh^3\chi},
\nonumber\\
&&\hat{S}^{(1)}_3(A_S=-3)=2\sinh\chi-\frac{5}{\sinh\chi}-\frac{15}{\sinh^3\chi}+\frac{15\chi\cosh\chi}{\sinh^4\chi},
\nonumber\\
&& \hat{S}^{(2)}_3(A_S=-3)=\frac{\cosh\chi}{\sinh^4\chi}.
\label{10.12b}
\end{eqnarray}
%
From this pattern we again see that the solutions that are bounded at $\chi=\infty$ are badly-behaved at $\chi=0$, while the solutions that are  well-behaved at $\chi=0$ are unbounded at $\chi=\infty$. Thus all of these $A_S=-3$ solutions are also excluded by a requirement that solutions be  bounded at $\chi=\infty$ and be well-behaved at $\chi=0$.

With all of these $A_S=0$, $A_S=-3$ solutions being excluded we must realize (\ref{9.22}), (\ref{9.26}), (\ref{9.27}) and (\ref{9.28}) by 
%
\begin{eqnarray}
-2 \dot{\Omega} \Omega^{-1} (\alpha - \dot\gamma) + 2 k \gamma 
+(-4 \dot{\Omega}^2 \Omega^{-3}  + 2 \overset{..}{\Omega} \Omega^{-2}  - 2 k \Omega^{-1}) \hat{V}=0,
\label{10.13b}
\end{eqnarray}
%
%
\begin{eqnarray}
&&2 \dot{\Omega}^2 \Omega^{-2}(\alpha-\dot\gamma)
-2  \dot{\Omega} \Omega^{-1}(\dot\alpha -\ddot\gamma)-4\ddot\Omega\Omega^{-1}(\alpha-\dot\gamma)+ \Omega^2 \delta \hat{p}
+ 2 k(\alpha + 2 \dot{\Omega}  \Omega^{-1} \gamma)]
\nonumber\\
&&=0,
\label{10.14b}
\end{eqnarray}
%
%
\begin{eqnarray}
\alpha+2\dot{\Omega}\Omega^{-1}\gamma=0,
\label{10.15b}
\end{eqnarray}
%
%
\begin{eqnarray}
2 \dot{\Omega}^2 \Omega^{-2}(\alpha-\dot\gamma)
-2  \dot{\Omega} \Omega^{-1}(\dot\alpha -\ddot\gamma)-4\ddot\Omega\Omega^{-1}(\alpha-\dot\gamma)+ \Omega^2 \delta \hat{p}=0.
\label{10.16b}
\end{eqnarray}
%
%
These equations are augmented by (\ref{9.13}), (\ref{9.16}) and (\ref{9.17})
%
\begin{eqnarray}
6 \dot{\Omega}^2 \Omega^{-2}(\alpha-\dot\gamma) + \delta \hat{\rho} \Omega^2 + 2 \dot{\Omega} \Omega^{-1} \tilde{\nabla}_{a}\tilde{\nabla}^{a}\gamma=0, 
\label{10.17b}
\end{eqnarray}
%
%
\begin{eqnarray}
&&6 \dot{\Omega}^2 \Omega^{-2}(\alpha-\dot\gamma)
-6  \dot{\Omega} \Omega^{-1}(\dot\alpha -\ddot\gamma)-12\ddot\Omega\Omega^{-1}(\alpha-\dot\gamma)+ 3\Omega^2 \delta \hat{p}
\nonumber\\
&&-2\tilde\nabla_a\tilde\nabla^a(\alpha + 2\dot\Omega \Omega^{-1}\gamma)=0,
\label{10.18b}
\end{eqnarray}
%
%
\begin{eqnarray}
&&3 \delta \hat{p}-  \delta \hat{\rho}
-12 \overset{..}{\Omega}  \Omega^{-3}(\alpha - \dot\gamma) -6 \dot{\Omega} \Omega^{-3}(\dot{\alpha} -\ddot\gamma)
\nonumber\\
&&-2 \Omega^{-2} \tilde{\nabla}_{a}\tilde{\nabla}^{a}(\alpha +3\dot\Omega\Omega^{-1}\gamma)=0.
\label{10.19b}
\end{eqnarray}
%
On taking the $\tilde{\nabla}_i$ derivative of (\ref{10.13b}), we recognize (\ref{10.13b}) to (\ref{10.19b})  as precisely being the scalar sector ones given in (\ref{9.41}) and (\ref{9.42}). We thus establish the decomposition theorem in the scalar sector. 

\subsubsection{The Vector Sector}
\label{sss:vector_sector}

To determine the structure of $k=-1$ solutions to the vector sector  (\ref{9.23}) and (\ref{9.31}), we first need to evaluate the quantity $\tilde{\nabla}_a\tilde{\nabla}^aV^i$, where $V^i$ obeys the transverse condition 
%
\begin{eqnarray}
\tilde\nabla_a V^a&=& \frac{V_{2} \cos\theta}{\sin\theta \sinh^2\chi} + \frac{2 V_{1} \cosh\chi}{\sinh\chi} + \partial_{1}V_{1} + \frac{\partial_{2}V_{2}}{\sinh^2\chi} + \frac{\partial_{3}V_{3}}{\sin^2\theta \sinh^2\chi}
\nonumber\\
&=&0.
\label{10.20b}
\end{eqnarray}
%
On implementing this condition, the $(\chi,\theta,\phi) \equiv (1,2,3)$ components of $\tilde{\nabla}_a\tilde{\nabla}^aV^i$ take the form
% 
\begin{eqnarray}
\tilde{\nabla}_a\tilde{\nabla}^aV^1&=&V_{1} \left(2 + \frac{2}{\sinh^2\chi}\right) + \frac{4 \cosh\chi \partial_{1}V_{1}}{\sinh\chi} + \partial_{1}\partial_{1}V_{1} + \frac{\cos\theta \partial_{2}V_{1}}{\sin\theta \sinh^2\chi} + \frac{\partial_{2}\partial_{2}V_{1}}{\sinh^2\chi}
\nonumber\\
&& + \frac{\partial_{3}\partial_{3}V_{1}}{\sin^2\theta \sinh^2\chi},
\nonumber\\ 
\tilde{\nabla}_a\tilde{\nabla}^aV^2&=& V_{2} \left(- \frac{2}{\sinh^4\chi} + \frac{1}{\sin^2\theta \sinh^4\chi} -  \frac{2}{\sinh^2\chi}\right) + \frac{4 V_{1} \cos\theta \cosh\chi}{\sin\theta \sinh^3\chi}
\nonumber\\
&& + \frac{2 \cos\theta \partial_{1}V_{1}}{\sin\theta \sinh^2\chi} + \frac{\partial_{1}\partial_{1}V_{2}}{\sinh^2\chi}  + \frac{2 \cosh\chi \partial_{2}V_{1}}{\sinh^3\chi} + \frac{3 \cos\theta \partial_{2}V_{2}}{\sin\theta \sinh^4\chi}
\nonumber\\
&& + \frac{\partial_{2}\partial_{2}V_{2}}{\sinh^4\chi} + \frac{\partial_{3}\partial_{3}V_{2}}{\sin^2\theta \sinh^4\chi},
\nonumber\\ 
\tilde{\nabla}_a\tilde{\nabla}^aV^3&=& - \frac{2 V_{3}}{\sin^2\theta \sinh^2\chi} + \frac{\partial_{1}\partial_{1}V_{3}}{\sin^2\theta \sinh^2\chi} -  \frac{\cos\theta \partial_{2}V_{3}}{\sin^3\theta \sinh^4\chi} + \frac{\partial_{2}\partial_{2}V_{3}}{\sin^2\theta \sinh^4\chi}
\nonumber\\
&& + \frac{2 \cosh\chi \partial_{3}V_{1}}{\sin^2\theta \sinh^3\chi} + \frac{2 \cos\theta \partial_{3}V_{2}}{\sin^3\theta \sinh^4\chi} + \frac{\partial_{3}\partial_{3}V_{3}}{\sin^4\theta \sinh^4\chi}.
\label{10.21b}
\end{eqnarray}
% 
To explore the structure of the $k=-1$ vector sector we seek solutions to
%
\begin{eqnarray}
(\tilde{\nabla}_a\tilde{\nabla}^a+A_V)V_i=0.
\label{10.22b}
\end{eqnarray}
%
(Here $V_i$ is to denote the full combinations of vector components that appear in (\ref{9.23}) and (\ref{9.31}).) In (\ref{10.22b}) we have introduced a generic vector sector constant $A_V$, whose values in (\ref{9.23}) and (\ref{9.31})  are $(2,-1,-2)$.

Conveniently, we find that the equation for $V_1$ involves no mixing with $V_2$ or $V_3$, and can thus be solved directly. On setting $V_1(\chi,\theta,\phi)=g_{1,\ell}(\chi)Y_{\ell}^m(\theta,\phi)$, the equation for $V_1$ reduces to 
%
\begin{eqnarray}
\left[\frac{d^2}{d\chi^2}+4\frac{\cosh\chi}{ \sinh\chi}\frac{d }{d\chi}
+2+A_V+\frac{2 }{ \sinh^2\chi}-\frac{\ell(\ell+1)}{ \sinh^2\chi}\right]g_{1,\ell}=0.
\label{10.23b}
\end{eqnarray}
%
To check the $\chi \rightarrow \infty$ and $\chi \rightarrow 0$ limits, we take the solutions to behave as $e^{\lambda\chi}$ (times an irrelevant polynomial in $\chi$) and $\chi^n$ in these two limits. For (\ref{10.23b}) the limits give
%
\begin{eqnarray}
&&\lambda^2+4\lambda+2+A_V=0,\quad\lambda=-2\pm (2-A_V)^{!/2},\quad
\lambda(A_V=2)=(-2,~-2),
\nonumber\\
&& \lambda(A_V=-1)=-2\pm \surd{3},\quad \lambda(A_V=-2)=(0,-4),
\nonumber\\
&&
n(n-1)+4n+2-\ell(\ell+1)=0,\quad n=\ell-1, -\ell-2.
\label{10.24b}
\end{eqnarray}
%
Thus for $A_V=2$ and $A_V=-1$ both solutions are bounded at infinity, while for $A_V=-2$ one solution is bounded at infinity. Moreover, for each value of $A_V$ one of the solutions will  be well-behaved as $\chi\rightarrow 0$ for any $\ell\geq 1$ while the other solution will not be.  Thus for $A_V=2$ there will always be one $\ell\geq 1$ solution that is bounded at $\chi=\infty$ and well-behaved at $\chi=0$. To determine whether we can obtain a solution that is bounded in both limits for $A_V=-1$, $A_V=-2$ we need to explicitly find the solutions in closed form. 


To this end we need to put (\ref{10.23b})  into the form of a differential equation whose solutions are known. We thus set $g_{1,\ell}=\alpha_{\ell}/\sinh\chi$, to find that (\ref{10.23b}) takes the form
%
\begin{eqnarray}
\left[\frac{d^2 }{d\chi^2}+2\frac{\cosh\chi}{ \sinh\chi}\frac{d }{d\chi}
-\frac{\ell(\ell+1) }{\sinh^2\chi}+A_V-1\right]\alpha_{\ell}=0.
\label{10.25b}
\end{eqnarray}
%
We recognize (\ref{10.25b}) as being in the form given in  (\ref{10.3b}), which we discussed above, with $\nu^2=A_V-2$.


Thus for $A_V=2$, viz. $\nu=0$  in (\ref{10.7b}) and $f(\nu^2=0)=\chi,~\chi^2$,  we find $V^{(i)}_{\ell}$, $i=1,2$ solutions to $(\tilde{\nabla}_a\tilde{\nabla}^a+2)V_1=0$ of the form 
%
\begin{eqnarray}
\hat{V}^{(1)}_0(A_V=2)&=&\frac{1}{ \sinh^2\chi},\quad \hat{V}^{(2)}_0(A_V=2)=\frac{\chi }{ \sinh^2\chi},
\nonumber\\
\hat{V}^{(1)}_1(A_V=2)&=&\frac{\cosh \chi }{ \sinh^3\chi},\quad \hat{V}^{(2)}_1(A_V=2)=\frac{1}{ \sinh^2\chi}-\frac{\chi\cosh\chi}{ \sinh^3\chi},
\nonumber\\
\hat{V}^{(1)}_2(A_V=2)&=&\frac{2}{ \sinh^2\chi}+\frac{3}{\sinh^4\chi},
\nonumber\\
\hat{V}^{(2)}_2(A_V=2)&=&\frac{3\cosh\chi}{\sinh^3\chi}-\frac{2\chi}{\sinh^2\chi}-\frac{3\chi }{\sinh^4\chi},
\nonumber\\
\hat{V}^{(1)}_3(A_V=2)&=&\frac{2\cosh\chi}{\sinh^3\chi}+\frac{5\cosh\chi}{\sinh^5\chi},
\nonumber\\
 \hat{V}^{(2)}_3(A_V=2)&=&\frac{11}{\sinh^2\chi}+\frac{15}{\sinh^4\chi}-\frac{6\chi\cosh\chi}{\sinh^3\chi}-\frac{15\chi\cosh\chi }{\sinh^5\chi}.~~~
\label{10.26b}
\end{eqnarray}
%
The just as required by (\ref{10.24b}), the $\hat{V}^{(2)}_{\ell}(A_V=2)$ solutions with $\ell \geq1$ are bounded at  $\chi=\infty$ and well-behaved at $\chi=0$. Since they can thus not be excluded by boundary conditions at $\chi=\infty$ and $\chi=0$ (though boundary conditions do exclude modes with $\ell=0$), solutions to (\ref{9.23}) and (\ref{9.31}) do not become the vector sector solutions associated with (\ref{9.41}). Thus if we implement (\ref{9.31}) by $(\tilde{\nabla}_a\tilde{\nabla}^a+2)V_i=0$, the  decomposition theorem will fail in the vector sector for modes with $\ell \geq 1$. Thus an equation such as (\ref{9.31}) 
will be solved by 
%
\begin{eqnarray}
(\tilde{\nabla}_a\tilde{\nabla}^a-1)(\tilde{\nabla}_b\tilde{\nabla}^b-2)\left[\tfrac{1}{2}(\dot{B}_i-\ddot{E}_i)+\dot{\Omega}\Omega^{-1}(B_i-\dot{E}_i)\right]=V_i,
\label{10.27b}
\end{eqnarray}
%
and not by
%
\begin{eqnarray}
\tfrac{1}{2}(\dot{B}_i-\ddot{E}_i)+\dot{\Omega}\Omega^{-1}(B_i-\dot{E}_i)=0.
\label{10.28b}
\end{eqnarray}
%
Thus (\ref{9.31}) is solved by the $\chi$ dependence of $B_i-\dot{E}_i$ and not by its $\tau$ dependence, i.e., not by the $B_i-\dot{E_i}=1/\Omega^2$ dependence on $\tau$ that one would have obtained from the decomposition-theorem-required (\ref{10.28b}). This then raises the question of what does fix the $\tau$ dependence in the vector sector. We will address this issue below.

For $A_V=-2$ we see that $\nu^2=-4$ and that $f(\nu^2)=\cosh 2\chi,\sinh 2\chi$. However in the scalar case discussed above where $\nu^2=A_S-1$, $\nu^2$ would also obey $\nu^2=-4$ if $A_S=-3$. Thus for $A_V=-2$ we can obtain the solutions to $(\tilde{\nabla}_a\tilde{\nabla}^a-2)V_1=0$ directly from (\ref{10.12b}), and after implementing $g_{1,\ell}=\alpha_{\ell}/\sinh\chi$ we  obtain 
%
\begin{eqnarray}
&&\hat{V}^{(1)}_0(A_V=-2)=\frac{\cosh\chi}{\sinh\chi},\quad \hat{V}^{(2)}_0(A_V=-2)=2+\frac{1}{\sinh^2\chi},
\nonumber\\
&&\hat{V}^{(1)}_1(A_V=-2)=1,\quad \hat{V}^{(2)}_1(A_V=-2)=2\frac{\cosh\chi}{\sinh\chi}-\frac{\cosh\chi}{\sinh^3\chi},
\nonumber\\
&&\hat{V}^{(1)}_2(A_V=-2)=2\frac{\cosh\chi}{\sinh\chi}-\frac{3\cosh\chi}{\sinh^3\chi}+\frac{3\chi}{\sinh^4\chi},\quad \hat{V}^{(2)}_2(A_V=-2)=\frac{1}{\sinh^4\chi},
\nonumber\\
&&\hat{V}^{(1)}_3(A_V=-2)=2-\frac{5}{\sinh^2\chi}-\frac{15}{\sinh^4\chi}+\frac{15\chi\cosh\chi}{\sinh^5\chi},
\nonumber\\
&& \hat{V}^{(2)}_3(A_V=-2)=\frac{\cosh\chi}{\sinh^5\chi}.
\label{10.29b}
\end{eqnarray}
%
As required by (\ref{10.24b}), the $\hat{V}^{(2)}_2(A_V=-2)$ and $\hat{V}^{(2)}_3(A_V=-2)$ solutions  are bounded at  $\chi=\infty$. However, they are not well-behaved at $\chi=0$. Since they thus can  be excluded by boundary conditions at $\chi=\infty$ and $\chi=0$, if we implement (\ref{9.31}) by $(\tilde{\nabla}_a\tilde{\nabla}^a-2)V_i=0$,  the only allowed solution will be $V_i=0$, and the decomposition theorem will then follow.

Finally, for $A_V=-1$, viz. $\nu=i\surd{3}$, $f(\nu^2)=e^{\chi\surd{3}},e^{-\chi\surd{3}}$, the solutions to $(\tilde{\nabla}_a\tilde{\nabla}^a-1)V_1=0$ are of the form
%
\begin{eqnarray}
&&\hat{V}^{(1)}_0(A_V=-1)=\frac{e^{\chi\surd{3}}}{\sinh^2\chi},\quad \hat{V}^{(2)}_0(A_V=-1)=\frac{e^{-\chi\surd{3}}}{\sinh^2\chi},
\nonumber\\
&&\hat{V}^{(1)}_1(A_V=-1)=\frac{e^{\chi\surd{3}}}{\sinh^3\chi}\left[\surd{3}\sinh\chi-\cosh\chi\right],
\nonumber\\
&& \hat{V}^{(2)}_1(A_V=-1)=\frac{e^{-\chi\surd{3}}}{\sinh^3\chi}\left[-\surd{3}\sinh\chi-\cosh\chi\right],
\nonumber\\
&&\hat{V}^{(1)}_2(A_V=-1)=\frac{e^{\chi\surd{3}}}{\sinh^4\chi}\left[3-3\surd{3}\cosh\chi\sinh\chi+5\sinh^2\chi
\right],
\nonumber\\
&&\hat{V}^{(2)}_2(A_V=-1)=\frac{e^{-\chi\surd{3}}}{\sinh^4\chi}\left[3+3\surd{3}\cosh\chi\sinh\chi+5\sinh^2\chi\right],
\nonumber\\
&&\hat{V}^{(1)}_3(A_V=-1)=\frac{e^{\chi\surd{3}}}{\sinh^5\chi}\bigg[15\surd{3}\sinh\chi+14\surd{3}\sinh^3\chi
-15\cosh\chi
\nonumber\\
&&\qquad-24\cosh\chi\sinh^2\chi\bigg],
\nonumber\\
&&\hat{V}^{(2)}_3(A_V=-1)=\frac{e^{-\chi\surd{3}}}{\sinh^5\chi}\bigg[-15\surd{3}\sinh\chi-14\surd{3}\sinh^3\chi
-15\cosh\chi
\nonumber\\
&&\qquad-24\cosh\chi\sinh^2\chi\bigg].
\label{10.30b}
\end{eqnarray}
%
All of these solutions are bounded at $\chi=\infty$ and all $\hat{V}^{(1)}_{\ell}(A_V=-1)-\hat{V}^{(2)}_{\ell}(A_V=-1)$ with $\ell\geq 1$ are well-behaved at $\chi=0$.  Thus if implement (\ref{9.31}) by $(\tilde{\nabla}_a\tilde{\nabla}^a-1)V_i=0$,  we are not forced to $V_i=0$, with the decomposition theorem not then following in this sector.



\subsubsection{The Tensor Sector}
\label{sss:tensor_sector}

For $k=-1$ the transverse-traceless tensor sector modes need to satisfy 
%
\begin{eqnarray}
\tilde{\gamma}^{ab}T_{ab}&=& T_{11} + \frac{T_{22}}{\sinh^2\chi} + \frac{T_{33}}{\sin^2\theta \sinh^2\chi} =0,
\nonumber\\
\tilde\nabla_a T^{a 1}&=& - \frac{\cosh\chi T_{22}}{\sinh^3\chi} -  \frac{\cosh\chi T_{33}}{\sin^2\theta \sinh^3\chi} + \frac{\cos\theta T_{12}}{\sin\theta \sinh^2\chi} + \frac{2 \cosh\chi T_{11}}{\sinh\chi} + \partial_{1}T_{11} 
\nonumber\\
&&+ \frac{\partial_{2}T_{12}}{\sinh^2\chi}  + \frac{\partial_{3}T_{13}}{\sin^2\theta \sinh^2\chi}=0, \nonumber\\
\tilde\nabla_a T^{a 2}&=& - \frac{\cos\theta T_{33}}{\sin^3\theta \sinh^4\chi} + \frac{\cos\theta T_{22}}{\sin\theta \sinh^4\chi} + \frac{2 \cosh\chi T_{12}}{\sinh^3\chi} + \frac{\partial_{1}T_{12}}{\sinh^2\chi} + \frac{\partial_{2}T_{22}}{\sinh^4\chi} 
\nonumber\\
&&+ \frac{\partial_{3}T_{23}}{\sin^2\theta \sinh^4\chi}=0,
\nonumber\\
\tilde\nabla_a T^{a 3}&=& \frac{\cos\theta T_{23}}{\sin^3\theta \sinh^4\chi} + \frac{2 \cosh\chi T_{13}}{\sin^2\theta \sinh^3\chi} + \frac{\partial_{1}T_{13}}{\sin^2\theta \sinh^2\chi} + \frac{\partial_{2}T_{23}}{\sin^2\theta \sinh^4\chi}
\nonumber\\
&& + \frac{\partial_{3}T_{33}}{\sin^4\theta \sinh^4\chi}=0.
\label{10.31b}
\end{eqnarray}
%
Under these conditions the components of $\tilde{\nabla}_a\tilde{\nabla}^aT^{ij}$ evaluate to
%
\begin{eqnarray}
\tilde{\nabla}_a\tilde{\nabla}^aT^{11}&=& T_{11} \left(6 + \frac{6}{\sinh^2\chi}\right) + \frac{6 \cosh\chi \partial_{1}T_{11}}{\sinh\chi} + \partial_{1}\partial_{1}T_{11} + \frac{\cos\theta \partial_{2}T_{11}}{\sin\theta \sinh^2\chi} 
\nonumber\\
&&+ \frac{\partial_{2}\partial_{2}T_{11}}{\sinh^2\chi} + \frac{\partial_{3}\partial_{3}T_{11}}{\sin^2\theta \sinh^2\chi},
\nonumber\\ 
\tilde{\nabla}_a\tilde{\nabla}^aT^{22}&=& \frac{4 T_{22}}{\sinh^6\chi} -  \frac{4 T_{22}}{\sin^2\theta \sinh^6\chi} + \frac{4 T_{11}}{\sinh^4\chi} -  \frac{2 T_{22}}{\sinh^4\chi} -  \frac{2 T_{11}}{\sin^2\theta \sinh^4\chi} 
\nonumber\\
&&+ \frac{2 T_{11}}{\sinh^2\chi} -  \frac{2 \cosh\chi \partial_{1}T_{22}}{\sinh^5\chi} + \frac{\partial_{1}\partial_{1}T_{22}}{\sinh^4\chi} + \frac{4 \cosh\chi \partial_{2}T_{12}}{\sinh^5\chi} 
\nonumber\\
&&+ \frac{\cos\theta \partial_{2}T_{22}}{\sin\theta \sinh^6\chi} 
+ \frac{\partial_{2}\partial_{2}T_{22}}{\sinh^6\chi} -  \frac{4 \cos\theta \partial_{3}T_{23}}{\sin^3\theta \sinh^6\chi} 
+ \frac{\partial_{3}\partial_{3}T_{22}}{\sin^2\theta \sinh^6\chi},
\nonumber\\ 
\tilde{\nabla}_a\tilde{\nabla}^aT^{33}&=& \frac{2T_{33}} {\sin^4\theta\sinh^6\chi}\left(1-{\sinh^2\chi}\right) + T_{11} \left(\frac{2}{\sin^4\theta \sinh^4\chi} + \frac{2}{\sin^2\theta \sinh^2\chi}\right) 
\nonumber\\
&&-  \frac{4 \cos\theta \cosh\chi T_{12}}{\sin^3\theta \sinh^5\chi} 
 -  \frac{4 \cos\theta \partial_{1}T_{12}}{\sin^3\theta \sinh^4\chi} -  \frac{2 \cosh\chi \partial_{1}T_{33}}{\sin^4\theta \sinh^5\chi} + \frac{\partial_{1}\partial_{1}T_{33}}{\sin^4\theta \sinh^4\chi} 
 \nonumber\\
 &&+ \frac{4 \cos\theta \partial_{2}T_{11}}{\sin^3\theta \sinh^4\chi} + \frac{\cos\theta \partial_{2}T_{33}}{\sin^5\theta \sinh^6\chi}  + \frac{\partial_{2}\partial_{2}T_{33}}{\sin^4\theta \sinh^6\chi} + \frac{4 \cosh\chi \partial_{3}T_{13}}{\sin^4\theta \sinh^5\chi} 
 \nonumber\\
 &&+ \frac{\partial_{3}\partial_{3}T_{33}}{\sin^6\theta \sinh^6\chi},
\nonumber\\ 
\tilde{\nabla}_a\tilde{\nabla}^aT^{12}&=& T_{12} \left(- \frac{1}{\sin^2\theta \sinh^4\chi} -  \frac{2}{\sinh^2\chi}\right)
 + \frac{2 \cosh\chi \partial_{1}T_{12}}{\sinh^3\chi} + \frac{\partial_{1}\partial_{1}T_{12}}{\sinh^2\chi} 
 \nonumber\\
 &&+ \frac{2 \cosh\chi \partial_{2}T_{11}}{\sinh^3\chi}  + \frac{\cos\theta \partial_{2}T_{12}}{\sin\theta \sinh^4\chi} + \frac{\partial_{2}\partial_{2}T_{12}}{\sinh^4\chi}
 -  \frac{2 \cos\theta \partial_{3}T_{13}}{\sin^3\theta \sinh^4\chi}
 \nonumber\\
 && + \frac{\partial_{3}\partial_{3}T_{12}}{\sin^2\theta \sinh^4\chi},
\nonumber\\ 
\tilde{\nabla}_a\tilde{\nabla}^aT^{13}&=& - \frac{2 T_{13}}{\sin^2\theta \sinh^2\chi} + \frac{2 \cosh\chi \partial_{1}T_{13}}{\sin^2\theta \sinh^3\chi} + \frac{\partial_{1}\partial_{1}T_{13}}{\sin^2\theta \sinh^2\chi} -  \frac{\cos\theta \partial_{2}T_{13}}{\sin^3\theta \sinh^4\chi}
\nonumber\\
&& + \frac{\partial_{2}\partial_{2}T_{13}}{\sin^2\theta \sinh^4\chi}  + \frac{2 \cosh\chi \partial_{3}T_{11}}{\sin^2\theta \sinh^3\chi} + \frac{2 \cos\theta \partial_{3}T_{12}}{\sin^3\theta \sinh^4\chi} + \frac{\partial_{3}\partial_{3}T_{13}}{\sin^4\theta \sinh^4\chi},
\nonumber\\ 
\tilde{\nabla}_a\tilde{\nabla}^aT^{23}&=& T_{23} \left(\frac{2(1-\sinh^2\chi)}{\sin^2\theta\sinh^6\chi} -  \frac{1}{\sin^4\theta \sinh^6\chi}\right) + \frac{2 \cos\theta \partial_{1}T_{13}}{\sin^3\theta \sinh^4\chi} 
\nonumber\\
&&-  \frac{2 \cosh\chi \partial_{1}T_{23}}{\sin^2\theta \sinh^5\chi} + \frac{\partial_{1}\partial_{1}T_{23}}{\sin^2\theta \sinh^4\chi} + \frac{2 \cosh\chi \partial_{2}T_{13}}{\sin^2\theta \sinh^5\chi} + \frac{\cos\theta \partial_{2}T_{23}}{\sin^3\theta \sinh^6\chi} 
\nonumber\\
&&+ \frac{\partial_{2}\partial_{2}T_{23}}{\sin^2\theta \sinh^6\chi} + \frac{2 \cosh\chi \partial_{3}T_{12}}{\sin^2\theta \sinh^5\chi} + \frac{2 \cos\theta \partial_{3}T_{22}}{\sin^3\theta \sinh^6\chi} 
 + \frac{\partial_{3}\partial_{3}T_{23}}{\sin^4\theta \sinh^6\chi}.
 \nonumber\\
\label{10.32b}
\end{eqnarray}
%

Following our analysis of the vector sector, in the $k=-1$ tensor sector we seek solutions to
%
\begin{eqnarray}
(\tilde{\nabla}_a\tilde{\nabla}^a+A_T)T_{ij}=0.
\label{10.33b}
\end{eqnarray}
%
(Here $T_{ij}$ is to denote the full combination of  tensor components that appears in (\ref{9.40}).)  In (\ref{10.33b}) we have introduced a generic tensor sector constant $A_T$, whose values in (\ref{9.40})  are $(2,3,6)$.
Conveniently, we find that the equation for $T_{11}$ involves no mixing with any other components of $T_{ij}$, and can thus be solved directly. On setting $T_{11}(\chi,\theta,\phi)=h_{11,\ell}(\chi)Y_{\ell}^m(\theta,\phi)$, the equation for $T_{11}$ reduces to 
%
\begin{eqnarray}
\left[\frac{d^2}{d\chi^2}+6\frac{\cosh\chi}{ \sinh\chi}\frac{d }{d\chi}
+6+\frac{6 }{ \sinh^2\chi}-\frac{\ell(\ell+1)}{ \sinh^2\chi}+A_T\right]h_{11,\ell}=0.
\label{10.34b}
\end{eqnarray}
%

To determine the $\chi \rightarrow \infty$ and $\chi \rightarrow 0$ limits, we take the solutions to behave as $e^{\lambda\chi}$ (times an irrelevant polynomial in $\chi$) and $\chi^n$ in these two limits. For (\ref{10.34b}) the limits give
%
\begin{eqnarray}
&&\lambda^2+6\lambda+6+A_T,\quad \lambda=-3\pm(3-A_T)^{1/2},
\nonumber\\
&&\lambda(A_T=2)=(-4,~-2),
\nonumber\\
&& \lambda(A_T=3)=(-3,~-3),\quad \lambda(A_T=6)=-3\pm i\surd{3},
\nonumber\\
&&n(n-1)+6n+6-\ell(\ell+1)=0,\quad n=\ell-2, -\ell-3.
\label{10.35b}
\end{eqnarray}
%
Thus for any allowed $A_T$, every solution to (\ref{10.34b}) is bounded at $\chi=\infty$, while for each $A_T$ one of the solutions will be well-behaved as $\chi\rightarrow 0$ for any $\ell\geq 2$.  Thus for $\ell=2,3,4,..$  there will always be one solution for any allowed $A_T$ that is bounded at $\chi=\infty$ and well-behaved at $\chi=0$, with all solutions with $\ell=0$ or $\ell=1$ being excluded.

To solve (\ref{10.34b}) we set $h_{11,\ell}=\gamma_{\ell}/\sinh^2\chi$ to obtain:
%
\begin{eqnarray}
\left[\frac{d^2}{d\chi^2}+2\frac{\cosh\chi}{\sinh\chi}\frac{d}{d\chi}
-\frac{\ell(\ell+1) }{ \sinh^2\chi}-2+A_T\right]\gamma_{\ell}=0.
\label{10.36b}
\end{eqnarray}
%
We recognize (\ref{10.36b}) as being (\ref{10.3b}), and can set $\nu^2=A_T-3$ in (\ref{10.7b}), viz. $\nu^2=(-1,0,3)$ for $A_T=2,3,6$. For $A_T=2$ we see that $\nu^2=-1$. However in the scalar case discussed above where $\nu^2=A_S-1$, $\nu^2$ would also obey $\nu^2=-1$ if $A_S=0$. Thus for $A_T=2$ we can obtain the solutions to $(\tilde{\nabla}_a\tilde{\nabla}^a+2)T_{11}=0$ directly from (\ref{10.11b}), and after implementing $h_{11,\ell}=\gamma_{\ell}/\sinh^2\chi$ we obtain $T^{(1)}_{\ell}$, $T^{(2)}_{\ell}$ solutions to (\ref{10.34b}) of the form 
%
\begin{align}
&\hat{T}^{(1)}_{0}(A_T=2)=\frac{\cosh\chi}{\sinh^3\chi},\quad \hat{T}^{(2)}_{0}(A_T=2)=\frac{1}{\sinh^2\chi},
\nonumber\\
&\hat{T}^{(1)}_{1}(A_T=2)=\frac{1}{\sinh^4\chi},\quad \hat{T}^{(2)}_{1}(A_T=2)=\frac{\cosh\chi}{\sinh^3\chi}-\frac{\chi}{\sinh^4\chi},
\nonumber\\
&\hat{T}^{(1)}_{2}(A_T=2)=\frac{\cosh\chi}{\sinh^5\chi},
\nonumber\\
& \hat{T}^{(2)}_{2}(A_T=2)=\frac{1}{\sinh^2\chi}+\frac{3}{\sinh^4\chi}-\frac{3\chi\cosh\chi}{\sinh^5\chi},
\nonumber\\
&\hat{T}^{(1)}_{3}(A_T=2)=\frac{4}{\sinh^4\chi}+\frac{5}{\sinh^6\chi},
\nonumber\\
&\hat{T}^{(2)}_{3}(A_T=2)=
\frac{2\cosh\chi}{\sinh^3\chi}+\frac{15\cosh\chi}{\sinh^5\chi}-\frac{12\chi}{\sinh^4\chi}-\frac{15\chi}{\sinh^6\chi}.
\label{10.37b}
\end{align}
%
All of these solutions are bounded at $\chi=\infty$ and all $\hat{T}^{(2)}_{\ell}(A_T=2)$ with $\ell\geq 2$ are well-behaved at $\chi=0$.  Thus if implement (\ref{9.40}) by $(\tilde{\nabla}_a\tilde{\nabla}^a+2)T_{ij}=0$,  we are not forced to $T_{ij}=0$, with the decomposition theorem not then following in the tensor sector.

For $A_T=3$ we see that $\nu^2=0$. However in the vector case discussed above where $\nu^2=A_V-2$, $\nu^2$ would also obey $\nu^2=0$ if $A_V=2$. Thus for $A_T=3$ we can obtain the solutions to $(\tilde{\nabla}_a\tilde{\nabla}^a+3)T_{11}=0$ directly from (\ref{10.26b}), and after implementing $h^{11}_{\ell}=\alpha_{\ell}/\sinh\chi$ we obtain 
%
\begin{eqnarray}
\hat{T}^{(1)}_0(A_T=3)&=&\frac{1}{ \sinh^3\chi},\quad \hat{T}^{(2)}_0(A_T=3)=\frac{\chi }{\sinh^3\chi},
\nonumber\\
\hat{T}^{(1)}_1(A_T=3)&=&\frac{\cosh \chi }{ \sinh^4\chi},\quad \hat{T}^{(2)}_1(A_T=3)=\frac{1}{ \sinh^3\chi}-\frac{\chi\cosh\chi}{\sinh^4\chi},
\nonumber\\
\hat{T}^{(1)}_2(A_T=3)&=&\frac{2}{ \sinh^3\chi}+\frac{3}{\sinh^5\chi},
\nonumber\\
 \hat{T}^{(2)}_2(A_T=3)&=&\frac{3\cosh\chi}{\sinh^4\chi}-\frac{2\chi}{\sinh^3\chi}-\frac{3\chi }{\sinh^5\chi},
\nonumber\\
\hat{T}^{(1)}_3(A_T=3)&=&\frac{2\cosh\chi}{\sinh^4\chi}+\frac{5\cosh\chi}{\sinh^6\chi},
\nonumber\\
\hat{T}^{(2)}_3(A_T=3)&=&\frac{11}{\sinh^3\chi}+\frac{15}{\sinh^5\chi}-\frac{6\chi\cosh\chi}{\sinh^4\chi}-\frac{15\chi\cosh\chi }{\sinh^6\chi}.~~~
\label{10.38b}
\end{eqnarray}
%
All of these solutions are bounded at $\chi=\infty$ and all $\hat{T}^{(2)}_{\ell}(A_T=3)$ with $\ell\geq 2$ are well-behaved at $\chi=0$.  Thus if implement (\ref{9.40}) by $(\tilde{\nabla}_a\tilde{\nabla}^a+3)T_{ij}=0$,  we are not forced to $T_{ij}=0$, with the decomposition theorem not then following.

A similar outcome occurs for $A_T=6$, and even though we do not evaluate the $A_T=6$ solutions explicitly, according to (\ref{10.35b}) all solutions to $(\tilde{\nabla}_a\tilde{\nabla}^a+6)T_{11}=0$ with $A_T=6$ are bounded at $\chi=\infty$ (behaving as $e^{-3\chi}\cos(\surd{3}\chi)$ and $e^{-3\chi}\sin(\surd{3}\chi)$), with one set of these solutions being well-behaved at $\chi=0$ for all $\ell \geq 2$.  Thus if implement (\ref{9.40}) by $(\tilde{\nabla}_a\tilde{\nabla}^a+6)T_{ij}=0$,  we are not forced to $T_{ij}=0$, with the decomposition theorem again not following in the tensor sector.
%%%%%%%%%%%%%%%%%%%%%%%%%%%%%%%%%%%%%%%%%%%%
\subsubsection{Decomposition Theorem Analysis}
\label{ss:recovering_decomposition_theorem}
%%%%%%%%%%%%%%%%%%%%%%%%%%%%%%%%%%%%%%%%%%%%
In Sec. \ref{ss:rw_k=-1_svt3} we have seen that there are realizations of the evolution equations in the scalar, vector, and tensor sectors that would not lead to a decomposition theorem in those sectors. However, equally there are other realizations that given the boundary conditions would lead to a decomposition theorem. Thus we need to determine which realizations are the relevant ones. To this end we look not at the individual higher-derivative equations obeyed by the separate scalar, vector, and tensor sectors, but at how these various sectors interface with each other in the original second-order $\Delta_{\mu\nu}=0$ equations themselves. Any successful such interface would require that all the terms in $\Delta_{\mu\nu}=0$ would have to have the same $\chi$ behavior. Noting that the scalar modes appear with two $\tilde{\nabla}$ derivatives in $\Delta_{ij}=0$, the vector sector appears with one $\tilde{\nabla}$ derivative and the tensor appears with none, we need to compare derivatives of scalars with vectors and derivatives of vectors with tensors. 

To see how to obtain such a needed common $\chi$ behavior we differentiate the scalar field (\ref{10.3b}) with respect to $\chi$, to obtain
%  
\begin{eqnarray}
&&\left[\frac{d^2}{d\chi^2}+4\frac{\cosh\chi}{\sinh\chi}\frac{d }{ d\chi}
+\frac{2}{\sinh^2\chi}-\frac{\ell(\ell+1)}{\sinh^2\chi}+4+A_S\right]\frac{d S_{\ell}}{d \chi}
+2A_S\frac{\cosh\chi}{\sinh\chi}S_{\ell}
\nonumber\\
&&=0.
\label{11.1}
\end{eqnarray}
%
Comparing with the vector (\ref{10.23b}) we see that up to an overall normalization we can identify $d S_{\ell}/d\chi$ with the vector $g_{1,\ell}$ for modes that obey $A_S=0$ and $A_V=2$, so that these particular scalar and vector modes can interface. As a check, with the vector sector needing $\ell \geq 1$ we differentiate $\hat{S}^{(2)}_1(A_S=0)$ to obtain
%  
\begin{eqnarray}
\frac{d}{d \chi}\hat{S}^{(2)}_1(A_S=0) &=&\frac{d}{d \chi}\left[\frac{\cosh\chi}{\sinh\chi}-\frac{\chi}{\sinh^2\chi}\right]
\nonumber\\
&=&-\frac{2}{ \sinh^2\chi}+\frac{2\chi\cosh\chi}{ \sinh^3\chi}=-2\hat{V}^{(2)}_1(A_V=2).
\label{11.2}
\end{eqnarray}
%

Similarly, if we differentiate the vector field (\ref{10.23b}) with respect to $\chi$ we  obtain
%
\begin{eqnarray}
&&\left[\frac{d^2}{d\chi^2}+6\frac{\cosh\chi}{ \sinh\chi}\frac{d }{d\chi}
+10+A_V+\frac{6 }{ \sinh^2\chi}-\frac{\ell(\ell+1)}{ \sinh^2\chi}\right]\frac{d g_{1,\ell}}{d \chi}
\nonumber\\
&&+2(2+A_V)\frac{\cosh\chi}{\sinh\chi}g_{1,\ell}=0.
\label{11.3}
\end{eqnarray}
%
Comparing with the tensor (\ref{10.34b}) we see that up to an overall normalization we can identify $d g_{1,\ell}/d\chi$ with the tensor $h_{11,\ell}$ for modes that obey $A_V=-2$ and $A_T=2$, so that these particular vector and tensor modes can interface. As a check, with the tensor sector needing $\ell \geq 2$ we differentiate $\hat{V}^{(1)}_2(A_V=-2)$ to obtain
%  
\begin{eqnarray}
\frac{d}{d \chi}\hat{V}^{(1)}_2(A_V=-2)&=& \frac{d}{d \chi}\left[\frac{2\cosh\chi}{\sinh\chi}-\frac{3\cosh\chi }{\sinh^3\chi}+\frac{3\chi}{\sinh^4\chi}\right]
\nonumber\\
&=&\frac{4}{\sinh^2\chi}+\frac{12}{\sinh^4\chi}-\frac{12\chi\cosh\chi}{\sinh^5\chi}
\nonumber\\
&=&4\hat{T}^{(2)}_2(A_T=2).~~
\label{11.4}
\end{eqnarray}
%

Thus while we can interface $A_S=0$ and $A_V=2$, we cannot interface $A_V=2$ with any of the tensor modes. Rather, we must interface the $A_V=-2$ vector modes with the  $A_T=2$ tensor modes. With none of the scalar sector modes meeting the boundary conditions at both $\chi=\infty$ and $\chi=0$ anyway, the scalar sector must satisfy  $\Delta_{\mu\nu}=0$ by itself, with the scalar term contribution to $\Delta_{\mu\nu}=0$ then having to vanish, just as required of the decomposition theorem. However, in the vector and tensor sectors we can achieve a common $\chi$ behavior if we set $B_1-\dot{E}_1=p_1(\tau)\hat{V}^{(1)}_2(A_V=-2)$, $E_{11}=q_{11}(\tau)\hat{T}^{(2)}_2(A_T=2)$, since then the $\Delta_{11}=0$ equation reduces to
%
\begin{eqnarray}
\Delta_{11}&=&\bigg{[}\frac{1}{\sinh^2\chi}+\frac{3}{\sinh^4\chi}-\frac{3\chi\cosh\chi}{\sinh^5\chi}\bigg{]}\times
\nonumber\\
&&\bigg{[} 8\dot{\Omega} \Omega^{-1}p_1(\tau)+4\dot{p}_1(\tau)- \ddot{q}_{11}(\tau) +2 q_{11}(\tau)  -2\dot{\Omega} \Omega^{-1}\dot{q}_{11}(\tau) -2q_{11}(\tau)\bigg{]}=0.
\nonumber\\
\label{11.5}
\end{eqnarray}
%
This relation has a non-trivial solution of the form
%
\begin{eqnarray}
4p_1(\tau)-\dot{q}_{11}(\tau)=\frac{1}{\Omega^2(\tau)},
\label{11.6}
\end{eqnarray}
%
to thereby relate the $\tau$ dependencies of the vector and tensor sectors. With the other components of $V_i$ and $T_{ij}$ being constructed in a similar manner, as such we have provided an exact interface solution in the vector and tensor sectors. However, it only falls short in one regard. Both of $\hat{V}^{(1)}_2(A_V=-2)$ and $\hat{T}^{(2)}_2(A_T=2)$ are well-behaved at $\chi=0$ and $\hat{T}^{(2)}_2(A_T=2)$ vanishes at $\chi=\infty$. However, $\hat{V}^{(1)}_2(A_V=-2)$ does not vanish at $\chi=\infty$, as it limits to a constant value. Imposing a boundary condition that the vector and tensor modes have to vanish at $\chi=\infty$ then excludes this solution, with the decomposition theorem then being recovered according to 
%
\begin{eqnarray}
\tfrac{1}{2}(\dot{B}_i-\ddot{E}_i)+\dot{\Omega}\Omega^{-1}(B_i-\dot{E}_i)&=&0,
\nonumber\\
- \overset{..}{E}_{ij} +2 E_{ij} - 2 \dot{E}_{ij} \dot{\Omega} \Omega^{-1} + \tilde{\nabla}_{a}\tilde{\nabla}^{a}E_{ij}&=&0,
\label{11.7}
\end{eqnarray}
%
with these being the equations that then serve to fix the $\tau$ dependencies in the vector and tensor sectors.  Consequently, we establish that the decomposition theorem does in fact hold for Robertson-Walker cosmologies with non-vanishing spatial 3-curvature after all. 

%%%%%%%%%%%%%%%%%%%%%%%%%%%%%%%%%%%%%%%%%%%%
\subsection{$\delta W_{\mu\nu}$ Conformal to Flat}
\label{ss:deltaW_conformal_flat_SVT3}
%%%%%%%%%%%%%%%%%%%%%%%%%%%%%%%%%%%%%%%%%%%%

Since the SVT3 and SVT4 formulations are not contingent on the choice of evolution equations, we continue our study of cosmological fluctuations by discussing how things work in an alternative to standard Einstein gravity, namely conformal gravity.  For SVT3 fluctuations around a Robertson-Walker background in the conformal gravity case  we have found it more convenient not to use the metric given in (\ref{9.1}), viz.
% 
\begin{eqnarray}
ds^2&=&a^2(\tau)\left[d\tau^2-\frac{dr^2}{1-kr^2}-r^2d\theta^2-r^2\sin^2\theta d\phi^2\right],
\label{13.1}
\end{eqnarray}
% 
but to instead take advantage of the fact that via a general coordinate transformation a non-zero $k$ Robertson-Walker metric can be brought into a form in which it is conformal to flat. (With $k=0$ the metric already is conformal to flat.) The needed transformations for $k<0$ and $k>0$ may for instance be found in \cite{amarasinghe_2019}.  
For the illustrative $k<0$ case for instance, it is convenient to set $k=-1/L^2$, and introduce ${\rm sinh} \chi=r/L$ and $p=\tau/L$, with the  metric given in (\ref{13.1}) then taking the form
%
\begin{eqnarray}
ds^2=L^2a^2(p)\left[dp^2-d\chi^2 -{\rm sinh}^2\chi d\theta^2-{\rm sinh}^2\chi \sin^2\theta d\phi^2\right].
\label{13.2}
\end{eqnarray}
%
Next we introduce
%
\begin{eqnarray}
&&p^{\prime}+r^{\prime}=\tanh[(p+\chi)/2],\quad p^{\prime}-r^{\prime}=\tanh[(p-\chi)/2],
\nonumber\\
&& p^{\prime}=\frac{\sinh p}{\cosh p+\cosh \chi},\quad r^{\prime}=\frac{\sinh \chi}{\cosh p+\cosh \chi},
\label{13.3}
\end{eqnarray}
%
so that
%
\begin{eqnarray}
dp^{\prime 2}-dr^{\prime 2}&=&\frac{1}{4}[dp^2-d\chi^2]{\rm sech}^2[(p+\chi)/2]{\rm sech}^2[(p-\chi)/2],
\nonumber\\
\frac{1}{4}(\cosh p+\cosh \chi)^2&=&{\rm \cosh}^2[(p+\chi)/2]{\rm \cosh}^2[(p-\chi)/2]
\nonumber\\
&=&\frac{1}{[1-(p^{\prime}+r^{\prime})^2][1-(p^{\prime}-r^{\prime})^2]}.
\label{13.4}
\end{eqnarray}
%
With these transformations the line element takes the conformal to flat form
%
\begin{eqnarray}
ds^2=\frac{4L^2a^2(p)}{[1-(p^{\prime}+r^{\prime})^2][1-(p^{\prime}-r^{\prime})^2]}\left[dp^{\prime 2}-dr^{\prime 2} -r^{\prime 2}d\theta^2-r^{\prime 2} \sin^2\theta d\phi^2\right].
\nonumber\\
\label{13.5}
\end{eqnarray}
%
The secpatial sector can then be written in Cartesian form
%
\begin{eqnarray}
ds^2=L^2a^2(p)(\cosh p+\cosh \chi)^2\left[dp^{\prime 2}-dx^{\prime 2} -dy^{\prime 2} -dz^{\prime 2}\right],
\label{13.6}
\end{eqnarray}
%
where $r^{\prime}=(x^{\prime 2}+ y^{\prime 2}+z^{\prime 2})^{1/2}$.  

We note that while our interest in this section is in discussing fluctuations in conformal gravity, a theory that actually has an underlying conformal symmetry, in transforming from (\ref{13.1})  to (\ref{13.6}) we have only made coordinate transformations and have not made any conformal transformation. However, since the $W_{\mu\nu}$ gravitational Bach tensor introduced in (\ref{AP3}) and (\ref{AP4}) above is associated with a conformal theory, under a conformal transformation of the form $g_{\mu\nu}\rightarrow \Omega^{2}(x)g_{\mu\nu}$, $W_{\mu\nu}$ and $\delta W_{\mu\nu}$ respectively transform into $\Omega^{-2}(x)W_{\mu\nu}$ and $\Omega^{-2}(x)\delta W_{\mu\nu}$. Moreover, since this is the case for any background metric that is conformal to flat we need not even restrict to Robertson-Walker or de Sitter, and can consider fluctuations around any background metric of the form 
%
\begin{eqnarray}
ds^2=\Omega^2(x)[dt^2-\delta_{ij}dx^idx^j],
\label{13.7}
\end{eqnarray}
%
where $\delta_{ij}$ is the Kronecker delta function and $\Omega(x)$ is a completely arbitrary function of the four $x^{\mu}$ coordinates.

In this background we take the SVT3  background plus fluctuation line element to be of the form
%
\begin{eqnarray}
ds^2 &=& \Omega^2(x) \bigg[ (1+2\phi) dt^2 -2(\tilde{\nabla}_i B +B_i)dt dx^i - [(1-2\psi)\delta_{ij} +2\tilde{\nabla}_i\tilde{\nabla}_j E
\nonumber\\
&& + \tilde{\nabla}_i E_j + \tilde{\nabla}_j E_i + 2E_{ij}]dx^i dx^j\bigg],
\label{13.8}
\end{eqnarray}
%
where $\Omega(x)$ is an arbitrary function of the coordinates, where $\tilde{\nabla}_i=\partial/\partial x^i$ (with Latin index) and  $\tilde{\nabla}^i=\delta^{ij}\tilde{\nabla}_j$ (i.e. not $\Omega^{-2}\delta^{ij}\tilde{\nabla}_j$) are defined with respect to the background 3-space metric $\delta_{ij}$, and where the elements of (\ref{13.8}) obey
%
\begin{eqnarray}
\delta^{ij}\tilde{\nabla}_j B_i = 0,\quad \delta^{ij}\tilde{\nabla}_j E_i = 0, \quad E_{ij}=E_{ji},\quad \delta^{jk}\tilde{\nabla}_kE_{ij} = 0, \quad \delta^{ij}E_{ij} = 0.
\label{13.9}
\end{eqnarray}
%
For these fluctuations $\delta W_{\mu\nu}$ is readily calculated, and it is found to have the form \cite{amarasinghe_2019} 
%
\begin{eqnarray}
\delta W_{00}  &=& -\frac{2}{3\Omega^2} \delta^{mn}\delta^{\ell k}\tilde{\nabla}_m\tilde{\nabla}_n\tilde{\nabla}_{\ell}\tilde{\nabla}_k \alpha,
\nonumber\\	
\delta W_{0i} &=&  -\frac{2}{3\Omega^2} \delta^{mn}\tilde{\nabla}_i\tilde{\nabla}_m\tilde{\nabla}_n\partial_0\alpha
+\frac{1}{2\Omega^2}\left[\delta^{\ell k}\tilde{\nabla}_{\ell}\tilde{\nabla}_k(\delta^{mn}\tilde{\nabla}_m\tilde{\nabla}_n-\partial_0^2)(B_i - \dot{E}_i)\right],
\nonumber\\	
\delta W_{ij}  &=& \frac{1}{3\Omega^2}\bigg{[} \delta_{ij}\delta^{\ell k}\tilde{\nabla}_{\ell}\tilde{\nabla}_k (\partial_0^2 - \delta^{mn}\tilde{\nabla}_m\tilde{\nabla}_n) 
+(\delta^{\ell k}\tilde{\nabla}_{\ell}\tilde{\nabla}_k -3\partial_0^2)\tilde{\nabla}_i\tilde{\nabla}_j  
\bigg{] }\alpha
\nonumber\\
&&+\frac{1}{2\Omega^2}\left[ \left[\delta^{\ell k}\tilde{\nabla}_{\ell}\tilde{\nabla}_k -\partial_0^2\right]\left[\tilde{\nabla}_i   \partial_0(B_j - \dot{E}_j)+ \tilde{\nabla}_j \partial_0(B_i - \dot{E}_i)\right] \right]
\nonumber\\
&&+\frac{1}{\Omega^2}\left[\delta^{mn}\tilde{\nabla}_m\tilde{\nabla}_n-\partial_0^2\right]^2E_{ij}.
\label{13.10}
\end{eqnarray}
%
where as before $\alpha=\phi + \psi +\dot{B}-\ddot{E}$. We note that the derivatives that appear in (\ref{13.10}) are conveniently with respect to the flat Minkowski metric and not with respect to the full background $ds^2=\Omega^2(x)[dt^2-\delta_{ij}dx^idx^j]$ metric, with the $\Omega(x)$ dependence only appearing as an overall factor. This must be the case since the $\delta W_{\mu\nu}$ given in (\ref{13.10}) is related to the $\delta W_{\mu\nu}$ given in (\ref{4.6}) by an $\Omega^{-2}(x)$  conformal transformation, and the $\delta W_{\mu\nu}$ given in (\ref{4.6}) is associated with fluctuations around flat spacetime.

Unlike the standard gravity case, in a background geometry that is conformal to flat, namely in a background geometry in which the Weyl tensor vanishes,  then according to the conformal transformation properties of Sec. \ref{ss:conformal_invariance} the background $W_{\mu\nu}$ will vanish as well. The background $T_{\mu\nu}$ thus vanishes also. Fluctuations are thus described by $\delta T_{\mu\nu}=0$, and thus by $\delta W_{\mu\nu}=0$, with $\delta W_{\mu\nu}$ as given in (\ref{13.10}), and thus $\alpha$, $B_i-\dot{E}_i$ and $E_{ij}$,  thus being gauge invariant. To check whether a decomposition theorem might hold we thus need to solve the equation $\delta W_{\mu\nu}=0$. To this end we note that since there are derivatives with respect to the purely spatial $\delta^{\ell k}\tilde{\nabla}_{\ell}\tilde{\nabla}_k$, on imposing spatial boundary conditions the relation $\delta W_{00}=0$ immediately sets $\alpha=0$.  With $\alpha=0$, applying the spatial boundary conditions to the relation $W_{0i}=0$ immediately sets $(\delta^{mn}\tilde{\nabla}_m\tilde{\nabla}_n-\partial_0^2)(B_i - \dot{E}_i)=0$, with $\delta W_{ij}=0$ then realizing $\left[\delta^{mn}\tilde{\nabla}_m\tilde{\nabla}_n-\partial_0^2\right]^2E_{ij}=0$. Thus with asymptotic boundary conditions the solution to $\delta W_{\mu\nu}=0$ is
%
\begin{eqnarray}
\alpha=0,\quad (\delta^{mn}\tilde{\nabla}_m\tilde{\nabla}_n-\partial_0^2)(B_i - \dot{E}_i)=0,\quad \left[\delta^{mn}\tilde{\nabla}_m\tilde{\nabla}_n-\partial_0^2\right]^2E_{ij}=0.
\label{13.11}
\end{eqnarray}
%
Since decomposition would require
%
\begin{align}
&\delta^{mn}\delta^{\ell k}\tilde{\nabla}_m\tilde{\nabla}_n\tilde{\nabla}_{\ell}\tilde{\nabla}_k \alpha=0, 
\nonumber\\
&\delta^{mn}\tilde{\nabla}_i\tilde{\nabla}_m\tilde{\nabla}_n\partial_0\alpha=0,
\nonumber\\
&\delta^{\ell k}\tilde{\nabla}_{\ell}\tilde{\nabla}_k(\delta^{mn}\tilde{\nabla}_m\tilde{\nabla}_n-\partial_0^2)(B_i - \dot{E}_i)=0,
\nonumber\\
&\bigg{[} \delta_{ij}\delta^{\ell k}\tilde{\nabla}_{\ell}\tilde{\nabla}_k (\partial_0^2 - \delta^{mn}\tilde{\nabla}_m\tilde{\nabla}_n) 
+(\delta^{\ell k}\tilde{\nabla}_{\ell}\tilde{\nabla}_k -3\partial_0^2)\tilde{\nabla}_i\tilde{\nabla}_j  
\bigg{] }\alpha=0,
\nonumber\\
&\left[\delta^{\ell k}\tilde{\nabla}_{\ell}\tilde{\nabla}_k -\partial_0^2\right]\left[\tilde{\nabla}_i   \partial_0(B_j - \dot{E}_j)+ \tilde{\nabla}_j \partial_0(B_i - \dot{E}_i)\right]=0,
\nonumber\\
& \left[\delta^{mn}\tilde{\nabla}_m\tilde{\nabla}_n-\partial_0^2\right]^2E_{ij}=0,
\label{13.12}
\end{align}
%
we see that the decomposition theorem is recovered.

To underscore this result we note that (\ref{13.10}) can be inverted so as to write each gauge-invariant combination as a separate combination of components of $\delta W_{\mu\nu}$, viz.
%
\begin{eqnarray}
\Omega^{2}\delta W_{00} &=&- \tfrac{2}{3} \tilde{\nabla}_{b}\tilde{\nabla}^{b}\tilde{\nabla}_{a}\tilde{\nabla}^{a}\alpha,
\label{13.13}
\end{eqnarray}
%
%
\begin{eqnarray}
\tilde\nabla_a\tilde\nabla^a(\Omega^2 \delta W_{0i}) - \tilde\nabla_i \tilde\nabla^a(\Omega^2  \delta W_{0a})&=&- \tfrac{1}{2} \tilde{\nabla}_{b}\tilde{\nabla}^{b}\tilde{\nabla}_{a}\tilde{\nabla}^{a}(\overset{..}{B}_{i}-\overset{...}{E}_{i})
\\
&& + \tfrac{1}{2} \tilde{\nabla}_{c}\tilde{\nabla}^{c}\tilde{\nabla}_{b}\tilde{\nabla}^{b}\tilde{\nabla}_{a}\tilde{\nabla}^{a}(B_{i} -\dot{E}_{i}),
\nonumber
\label{13.14}
\end{eqnarray}
%
%
\begin{eqnarray}
&&\tilde\nabla_a \tilde\nabla^a \big[ \tilde\nabla_b \tilde\nabla^b (\Omega^2 \delta W_{ij})- \tilde\nabla_i \tilde\nabla^l(\Omega^2  \delta W_{jl}) -  \tilde\nabla_j \tilde\nabla^l(\Omega^2  \delta W_{il})\big]
\nonumber\\
&&+\tfrac{1}{2}\delta_{ij}\tilde\nabla_a \tilde\nabla^a\big[  \tilde\nabla^k \tilde\nabla^l(\Omega^2  \delta W_{kl})-\tilde\nabla_b \tilde\nabla^b(\Omega^2\delta^{ab}\delta W_{ab})\big]
+\tfrac{1}{2} \tilde\nabla_i\tilde\nabla_j \big[ \tilde\nabla^k \tilde\nabla^l(\Omega^2  \delta W_{kl})
\nonumber\\
&& + \tilde\nabla_a \tilde\nabla^a(\Omega^2 \delta^{ab}\delta W_{ab})\big] 
=\tilde\nabla_a \tilde\nabla^a \tilde\nabla_b \tilde\nabla^b\left[\tilde\nabla_c \tilde\nabla^c - \partial_0^2\right]^2E_{ij}.
\label{13.15}
\end{eqnarray}
%
On setting $\delta W_{\mu\nu}=0$, each of these relations can now be integrated separately, with the decomposition theorem then following. 

For completeness we also note that for  SVT3 fluctuations around the (\ref{9.2}) background $k\neq 0$ Robertson-Walker metric given in (\ref{9.1}) with $\tilde{\gamma}_{ij}dx^idx^j=dr^2/(1-kr^2)+r^2d\theta^2+r^2\sin^2\theta d\phi^2$ and with $\Omega$ actually now more generally being an arbitrary function of $\tau$ and $x^i$, the conformal gravity $\delta W_{\mu\nu}$ is given by 
%
\begin{eqnarray}
\delta W_{00}&=& - \tfrac{2}{3} \Omega^{-2} (\tilde\nabla_a\tilde\nabla^a + 3k)\tilde\nabla_b\tilde\nabla^b \alpha,
\nonumber\\ 
\delta W_{0i}&=& -\tfrac{2}{3} \Omega^{-2}  \tilde\nabla_i (\tilde\nabla_a\tilde\nabla^a + 3k)\dot\alpha
+\tfrac12 \Omega^{-2}\bigg[ \tilde\nabla_a\tilde\nabla^a (\tilde\nabla_b \tilde\nabla^b-\partial_0^2)(B_i-\dot{E}_i)
\nonumber\\
&& -2k(2k+\partial_0^2)(B_i-\dot{E}_i)\bigg],
\nonumber\\ 
\delta W_{ij}&=& -\tfrac{1}{3} \Omega^{-2} \left[ \tilde{\gamma}_{ij} \tilde\nabla_a\tilde\nabla^a (\tilde\nabla_b \tilde\nabla^b +2k-\partial_0^2)\alpha - \tilde\nabla_i\tilde\nabla_j(\tilde\nabla_a\tilde\nabla^a - 3\partial_0^2)\alpha \right]
\nonumber\\
&& +\tfrac{1}{2} \Omega^{-2} \bigg[ \tilde\nabla_i ( \tilde\nabla_a\tilde\nabla^a -2k-\partial_0^2) \partial_0 (B_j-\dot{E}_j)
\nonumber\\
&& 
+  \tilde\nabla_j ( \tilde\nabla_a\tilde\nabla^a -2k-\partial_0^2) \partial_0 (B_i-\dot{E}_i)\bigg]
\nonumber\\
&&+ \Omega^{-2}\left[ (\tilde\nabla_a\tilde\nabla^a-\partial_0)^2 E_{ij} - 4k (\tilde\nabla_a\tilde\nabla^a - k-2\partial_0^2)E_{ij} \right],
\label{13.16}
\end{eqnarray}
%
where again $\alpha = \phi + \psi + \dot B - \ddot E$. These equations can be inverted, and yield
%
\begin{align}
&(\Omega^2\delta W_{00})= - \tfrac{2}{3}  (\tilde\nabla_a\tilde\nabla^a + 3k)\tilde\nabla_b\tilde\nabla^b \alpha,
\nonumber\\
\nonumber\\
&(\tilde\nabla_a\tilde\nabla^a-2k)(\Omega^2\delta W_{0i}) - \tilde\nabla_i \tilde\nabla^a (\Omega^2\delta W_{0a}) =
\tfrac{1}{2} (\tilde\nabla_a\tilde\nabla^a - 2k - \partial_0^2)\times
\nonumber\\
&\qquad(\tilde\nabla_b\tilde\nabla^b + 2k)
(\tilde\nabla_c\tilde\nabla^c -2k)(B_i-\dot{E}_i),
\nonumber\\
\nonumber\\
&(\tilde\nabla_a\tilde\nabla^a-2k)(\tilde\nabla_b\tilde\nabla^b-3k)(\Omega^2\delta W_{ij})
+ \tfrac{1}{2} \tilde\nabla_i\tilde\nabla_j\big[ \tilde\nabla^a\tilde\nabla^b (\Omega^2\delta W_{ab})
\nonumber\\
& + (\tilde\nabla_a\tilde\nabla^a +4k)(\tilde{\gamma}^{bc}(\Omega^2\delta W_{bc}))\big]
+\tfrac{1}{2} \tilde{\gamma}_{ij} \big[ (\tilde\nabla_a\tilde\nabla^a-4k)\tilde\nabla^b\tilde\nabla^c (\Omega^2\delta W_{bc})
\nonumber\\
&-(\tilde\nabla_a\tilde\nabla^a\tilde\nabla_b\tilde\nabla^b -2k \tilde\nabla_a\tilde\nabla^a +4k^2)(\tilde{\gamma}^{bc}(\Omega^2\delta W_{bc}))\big]
\nonumber\\
&-(\tilde\nabla_a\tilde\nabla^a -3k)(\tilde\nabla_i\tilde\nabla^b (\Omega^2\delta W_{jb}) + \tilde\nabla_j \tilde\nabla^b (\Omega^2\delta W_{ib}))
\nonumber\\
&=(\tilde\nabla_a\tilde\nabla^a-2k)(\tilde\nabla_b\tilde\nabla^b-3k)\left[ (\tilde\nabla_a\tilde\nabla^a-\partial_0)^2 E_{ij} - 4k (\tilde\nabla_a\tilde\nabla^a - k-2\partial_0^2)E_{ij} \right].
\label{13.17}
\end{align}
%
With this separation of the gauge-invariant combinations we again have the decomposition theorem. 

%%%%%%%%%%%%%%%%%%%%%%%%%%%%%%%%%%%%%%%%%%%%
\subsection{Gauge Invariants and the Decomposition Theorem}
\label{ss:gauge_invariants_decomp_theorem}
%%%%%%%%%%%%%%%%%%%%%%%%%%%%%%%%%%%%%%%%%%%%
In the SVT4 study of fluctuations around a de Sitter background that we presented in Sec. \ref{ss:ds4_svt4} we had found that one of the gauge-invariant combinations was given by $\alpha=\dot{F}+\tau \chi +F_0$ (see (\ref{6.54})). In this combination $F$ and $\chi$ are scalars while $F_0$ is the fourth component of the vector $F_{\mu}$. In solving the fluctuation equations in this case we actually solved for the gauge-invariant combinations and not for the individual scalar, vector and tensor sectors. In the solution we found that $\alpha=0$.  Thus would entail only that  $\dot{F}+\tau \chi =-F_0$. However, decomposition with respect to scalars, vectors and tensors would in addition entail that $\dot{F}+\tau \chi=0$, and $F_0=0$, something that would not be warranted as it is not required by the fluctuation equations, while moreover not being a gauge-invariant decomposition of the 
gauge-invariant $\alpha$. Thus given this example we in general see that one should only look for a decomposition theorem for gauge-invariant combinations and not look for one for the separate scalar, vector and tensor sectors as gauge invariance can in general intertwine them. Since it might perhaps be thought that this is an artifact of using SVT4 we now present two examples in which it occurs in SVT3. One is fluctuations around an anti de Sitter background, and the other is fluctuations around a completely general conformal to flat background.

\subsubsection{Fluctuations Around an Anti de Sitter Background}
\label{sss:fluctuations_around_ads4}
For an anti de Sitter background in four dimensions we have
%
\begin{eqnarray}
ds^2 &=& \Omega^2(z)\left[ dt^2 - dx^2-dy^2-dz^2\right]= -g_{\mu\nu}dx^{\mu} dx^{\nu},\quad
\Omega(z) = \frac{1}{Hz},
\nonumber\\
R_{\lambda\mu\nu\kappa} &=& -H^2(g_{\mu\nu}g_{\lambda\kappa} -g_{\lambda\nu}g_{\mu\kappa}),
\quad R_{\mu\kappa} =3H^2 g_{\mu\kappa},\quad R = 12H^2,
\nonumber\\
G_{\mu\nu} &=& -3H^2 g_{\mu\nu},\quad T_{\mu\nu} = 3H^2 g_{\mu\nu}.
\label{14.1}
\end{eqnarray}
%
We take the fluctuations to have the standard SVT3 form of
%
\begin{eqnarray}
ds^2 &=&- \Omega^2(z)\left( \eta_{\mu\nu}+ f_{\mu\nu}\right) dx^\mu dx^\nu,
\nonumber\\
f_{00} &=& -2 \phi,\quad f_{0i} = \tilde\nabla_i B + B_i,\quad \tilde{\nabla}^iB_i=0,
\nonumber\\
f_{ij} &=& -2 \psi \delta_{ij} + 2\tilde\nabla_i\tilde\nabla_j E + \tilde\nabla_i E_j
+ \tilde\nabla_i E_j + 2E_{ij},
\nonumber\\
\tilde{\nabla}^iE_i&=&0,\quad \tilde{\nabla}^iE_{ij}=0,\quad \delta^{ij}E_{ij}=0, 
\label{14.2}
\end{eqnarray}
%
where $\tilde{\nabla}_i$ and $\tilde{\nabla}^i=\delta^{ij}\tilde{\nabla}_j$ are defined with respect to a flat three-dimensional background $\eta_{ij}dx^idx^j=\delta_{ij}dx^idx^j$.

On defining
%
\begin{eqnarray}
\alpha &=& \phi +\psi+\dot B - \ddot E, \quad \delta = \phi -\psi + \dot B - \ddot E + \frac{2}{z}(\tilde\nabla_3 E + E_3),
\label{14.3}
\end{eqnarray}
%
following some algebra we find that the components of 
\begin{eqnarray}
\Delta_{\mu\nu}=\delta G_{\mu\nu}+\delta T_{\mu\nu}=\delta G_{\mu\nu}+ 3\Omega^2 H^2 f_{\mu\nu}
\end{eqnarray}
are given by 
\begin{eqnarray}
g^{\mu\nu}\Delta_{\mu\nu}&=& -12 H^2 \alpha - 3 H^2 z^2 \overset{..}{\alpha} + 3 H^2 z^2 \overset{..}{\delta} + 12 H^2 \delta + H^2 z^2 \tilde\nabla^{2}{}\alpha - 3 H^2 z^2 \tilde\nabla^{2}{}\delta 
\nonumber \\ 
&& + 6 H^2 z \tilde{\nabla}_{3}\delta +6 H^2 z (\dot{B}_3-\ddot{E}_3)+24 H^2 E_{33},
\nonumber\\ 
\delta^{ij}\Delta_{ij}&=& -9 z^{-2} \alpha - 3 \overset{..}{\alpha} + 3 \overset{..}{\delta} + 9 z^{-2} \delta - 2 \tilde\nabla^{2}{}\delta + z^{-1} \tilde{\nabla}_{3}\alpha + 5 z^{-1} \tilde{\nabla}_{3}\delta 
\nonumber\\
&&+6 z^{-1} (\dot{B}_3-\ddot{E}_3)+18 z^{-2} E_{33},
\nonumber\\ 
\Delta_{00}&=& 3 z^{-2} \alpha - 3 z^{-2} \delta -  \tilde\nabla^{2}{}\alpha + \tilde\nabla^{2}{}\delta + z^{-1} \tilde{\nabla}_{3}\alpha -  z^{-1} \tilde{\nabla}_{3}\delta -6 z^{-2} E_{33},
\nonumber\\ 
\Delta_{11}&=& -3 z^{-2} \alpha -  \overset{..}{\alpha} + \overset{..}{\delta} + 3 z^{-2} \delta -  \tilde\nabla^{2}{}\delta + \tilde{\nabla}_{1}\tilde{\nabla}_{1}\delta + z^{-1} \tilde{\nabla}_{3}\alpha + z^{-1} \tilde{\nabla}_{3}\delta 
\nonumber\\
&&+2 z^{-1} (\dot{B}_3-\ddot{E}_3) + \tilde{\nabla}_{1}(\dot{B}_1-\ddot{E}_1)
- \overset{..}{E}_{11} + 6 z^{-2} E_{33} + \tilde\nabla^{2}{}E_{11} 
\nonumber\\
&&+ 4 z^{-1} \tilde{\nabla}_{1}E_{13} - 2 z^{-1} \tilde{\nabla}_{3}E_{11},
\nonumber\\ 
\Delta_{22}&=& -3 z^{-2} \alpha -  \overset{..}{\alpha} + \overset{..}{\delta} + 3 z^{-2} \delta -  \tilde\nabla^{2}{}\delta + \tilde{\nabla}_{2}\tilde{\nabla}_{2}\delta + z^{-1} \tilde{\nabla}_{3}\alpha + z^{-1} \tilde{\nabla}_{3}\delta 
\nonumber\\
&&+2 z^{-1} (\dot{B}_3-\ddot{E}_3) + \tilde{\nabla}_{2}(\dot{B}_2-\ddot{E}_2)  - \overset{..}{E}_{22}+ 6 z^{-2} E_{33} + \tilde\nabla^{2}{}E_{22} 
\nonumber\\
&&+ 4 z^{-1} \tilde{\nabla}_{2}E_{23} - 2 z^{-1} \tilde{\nabla}_{3}E_{22},
\nonumber\\ 
\Delta_{33}&=& -3 z^{-2} \alpha -  \overset{..}{\alpha} + \overset{..}{\delta} + 3 z^{-2} \delta -  \tilde\nabla^{2}{}\delta -  z^{-1} \tilde{\nabla}_{3}\alpha + 3 z^{-1} \tilde{\nabla}_{3}\delta + \tilde{\nabla}_{3}\tilde{\nabla}_{3}\delta 
\nonumber\\
&&+2 z^{-1} (\dot{B}_3-\ddot{E}_3) + \tilde{\nabla}_{3}(\dot{B}_3-\ddot{E}_3)
\nonumber \\ 
&&  -\overset{..}{E}_{33}+ 6 z^{-2} E_{33} + \tilde\nabla^{2}{}E_{33} + 2 z^{-1} \tilde{\nabla}_{3}E_{33},
\nonumber\\ 
\Delta_{01}&=& - \tilde{\nabla}_{1}\dot{\alpha} + \tilde{\nabla}_{1}\dot{\delta}+\tfrac{1}{2} \tilde\nabla^{2}{}(B_1-\dot{E}_1) + z^{-1} \tilde{\nabla}_{1}(B_3-\dot{E}_3) -  z^{-1} \tilde{\nabla}_{3}(B_1-\dot{E}_1)
\nonumber\\
&&+2 z^{-1} \dot{E}_{13},
\nonumber\\ 
\Delta_{02}&=& - \tilde{\nabla}_{2}\dot{\alpha} + \tilde{\nabla}_{2}\dot{\delta}+\tfrac{1}{2} \tilde\nabla^{2}{}(B_2-\dot{E}_2) + z^{-1} \tilde{\nabla}_{2}(B_3-\dot{E}_3) -  z^{-1} \tilde{\nabla}_{3}(B_2-\dot{E}_2)
\nonumber\\
&&+2 z^{-1} \dot{E}_{23},
\nonumber\\ 
\Delta_{03}&=& - z^{-1} \dot{\alpha} + z^{-1} \dot{\delta} -  \tilde{\nabla}_{3}\dot{\alpha} + \tilde{\nabla}_{3}\dot{\delta}+\tfrac{1}{2} \tilde\nabla^{2}{}(B_3-\dot{E}_3)+2 z^{-1} \dot{E}_{33},
\nonumber\\ 
\Delta_{12}&=& \tilde{\nabla}_{2}\tilde{\nabla}_{1}\delta +\tfrac{1}{2} \tilde{\nabla}_{1}(\dot{B}_2-\ddot{E}_2) + \tfrac{1}{2} \tilde{\nabla}_{2}(\dot{B}_1-\ddot{E}_1)- \overset{..}{E}_{12} + \tilde\nabla^{2}{}E_{12}
\nonumber\\
&& + 2 z^{-1} \tilde{\nabla}_{1}E_{23} 
+ 2 z^{-1} \tilde{\nabla}_{2}E_{13} - 2 z^{-1} \tilde{\nabla}_{3}E_{12},
\nonumber\\ 
\Delta_{13}&=& - z^{-1} \tilde{\nabla}_{1}\alpha + z^{-1} \tilde{\nabla}_{1}\delta + \tilde{\nabla}_{3}\tilde{\nabla}_{1}\delta +\tfrac{1}{2} \tilde{\nabla}_{1}(\dot{B}_3-\ddot{E}_3) + \tfrac{1}{2} \tilde{\nabla}_{3}(\dot{B}_1-\ddot{E}_1)
\nonumber\\
&&- \overset{..}{E}_{13} 
+ \tilde\nabla^{2}{}E_{13} + 2 z^{-1} \tilde{\nabla}_{1}E_{33},
\nonumber\\ 
\Delta_{23}&=& - z^{-1} \tilde{\nabla}_{2}\alpha + z^{-1} \tilde{\nabla}_{2}\delta + \tilde{\nabla}_{3}\tilde{\nabla}_{2}\delta +\tfrac{1}{2} \tilde{\nabla}_{2}(\dot{B}_3-\ddot{E}_3) + \tfrac{1}{2} \tilde{\nabla}_{3}(\dot{B}_2-\ddot{E}_2)
\nonumber\\
&&- \overset{..}{E}_{23} 
+ \tilde\nabla^{2}{}E_{23} + 2 z^{-1} \tilde{\nabla}_{2}E_{33},
\label{14.4} 
\end{eqnarray}
%
where $\tilde\nabla^{2} = \delta^{ab} \tilde\nabla_a\tilde\nabla_b$. With $\Delta_{\mu\nu}$ being gauge invariant we recognize $\alpha$, $\delta$, $B_i-\dot{E}_i$ and $E_{ij}$ as being gauge invariant. We thus see that one of the gauge-invariant combinations, viz. $\delta$, depends on both scalars and vectors. Since our only purpose here is in establishing that one of the gauge-invariant SVT3 combinations does depend on both scalars and vectors, we shall not seek to solve $\Delta_{\mu\nu}=0$ in this particular case. Though if we were to we would only find expressions for $\alpha$, $\delta$, $B_i-\dot{E}_i$ and $E_{ij}$, and not for the separate scalar and vector components of $\delta$.

\subsubsection{Fluctuations Around a General Conformal to Flat Background}
\label{sss:fluctuations_around_general_conformal_flat}

In \cite{amarasinghe_2019} it was shown that for the arbitrary conformal to flat SVT3 fluctuations of the form
%
\begin{eqnarray}
ds^2 &=&\Omega^2({\bf x},t)\bigg[(1+2\phi) dt^2 -2(\partial_i B +B_i)dt dx^i - [(1-2\psi)\delta_{ij} +2\partial_i\partial_j E 
\nonumber\\
&&+ \partial_i E_j + \partial_j E_i + 2E_{ij}]dx^i dx^j\bigg]
\label{14.5}
\end{eqnarray}
%
with general $\Omega({\bf x},t)$, the metric sector gauge-invariant combinations are
%
\begin{eqnarray}
\alpha &=& \phi +\psi+\dot B - \ddot E, \quad \eta=\psi -\Omega^{-1}\dot{\Omega}(B-\dot E)+\Omega^{-1}\tilde\nabla^i\Omega(E_i+\tilde\nabla_i E),
\nonumber\\
&& B_i-\dot E_i,\quad E_{ij}.
\label{14.6}
\end{eqnarray}
%
Of these invariant combinations three are independent of $\Omega$ altogether and have been encountered frequently throughout this study, while only one, viz. $\eta$, actually depends on $\Omega$ at all. (For specific choices of $\Omega$ the quantity $-\Omega\dot{\Omega}^{-1}\eta$ reduces to the previously introduced $\beta$ in the de Sitter (\ref{7.3}) and to $\gamma$ in the Robertson-Walker (\ref{8.7}) and (\ref{9.2}), while $\alpha-2\eta$ reduces to the anti de Sitter $\delta$ given in (\ref{14.3}).) The invariant $\alpha$ involves scalars alone, the invariant $B_i-E_i$ involves vectors alone, the invariant $E_{ij}$  involves tensors alone, and only the invariant $\eta$ actually involves more than just one of the scalar, vector and tensor sets of modes, with it specifically involving both scalars and vectors. While $\eta$ must always involve scalars, if $\Omega$ has a spatial dependence $\eta$ will also involve the vector $E_i$. A spatial dependence for $\Omega$ is  encountered in our study of anti de Sitter fluctuations as shown in (\ref{14.3}), and is also encountered in SVT3 fluctuations around a Robertson-Walker background with $k\neq 0$,  where the background Robertson-Walker metric as shown in (\ref{13.6}) for $k<0$ (and in \cite{amarasinghe_2019} for $k>0$) is written in a conformal to flat form, with the conformal factor expressly being a function of both time and space coordinates. Thus in such cases we must expect $\Delta_{\mu\nu}$ to depend on $\eta$ itself and not be separable in separate scalar and vector sectors. While this issue is met for $k\neq 0$ Robertson-Walker fluctuations when the background metric is written in the conformal to flat form given in (\ref{13.6}), we note that it is not in fact met for fluctuations around a background Robertson-Walker geometry with metric $ds^2=\Omega^2(\tau)[d\tau^2-dr^2/(1-kr^2)-r^2d\theta^2-r^2\sin^2\theta d\phi^2]$ as given in (\ref{9.1}), since with $\Omega$ only depending on $\tau$ in that case, the gauge-invariant $\gamma = - \dot\Omega^{-1}\Omega \psi + B - \dot E$ as given in (\ref{9.12}) does not involve the vector sector modes. While one can of course transform the background metric given in (\ref{13.6}) into the background metric given in (\ref{9.1}) by a coordinate transformation with fluctuations around the two metrics thus describing the same physics, the very structure of (\ref{14.6}) shows that one cannot simply separate in scalar, vector tensor components at will. Rather one must first separate in gauge-invariant combinations, and only if these combinations turn out not to intertwine any of the scalar, vector and tensor sectors could one then separate in each of the scalar, vector and tensor sectors. Moreover, while one can find a form for the background metric in which there is no such intertwining in the $k\neq 0$ Robertson-Walker case (for $k=0$ $\Omega$ only depends on $\tau$ and so there is no intertwining), this only occurs because of the specific purely $\tau$-dependent form that  the $k\neq 0$ $\Omega$ just happens to take.  For more complicated $\Omega$ there would be no coordinate transformation that would eliminate the intertwining, and so it is of interest to study the spatially dependent $\Omega$ situation in and of itself.

While we of course do not need to explicitly solve for fluctuations around a $k\neq 0$ Robertson-Walker metric when written in a conformal to flat form since in Secs. \ref{ss:general_rw_svt3}, \ref{ss:rw_k=-1_svt3}, and \ref{ss:recovering_decomposition_theorem} we already have solved for fluctuations around the same geometry when written in the general coordinate equivalent form given in (\ref{9.1}), it is nonetheless of interest to explore the structure of fluctuations around the conformal to flat form for a $k\neq 0$ Robertson-Walker geometry. In particular it is of interest to show that the $\eta$ invariant given in (\ref{14.6}) actually behaves quite differently from all the other invariants.   In (\ref{9.43a}) to (\ref{9.47a}) we had obtained some kinematic relations (i.e., relations that do not involve the evolution equations) that express the gauge-invariant combinations in terms of the  $f_{\mu\nu}$ components of the fluctuation metric. Inspection of these relations and of (\ref{9.12}) shows there is  only one, viz. that for the relevant $\eta$ in that case, that depends on $\Omega$. Now the relations given  in (\ref{9.43a}) to (\ref{9.47a}) were derived for fluctuations around the (\ref{9.1}) metric. If we now set $k=0$ in these relations so that the $\tilde{\nabla}$ derivative now refers to flat spacetime, we would anticipate that for fluctuations around (\ref{13.6}) the relations for the gauge-invariant combinations that do not involve $\Omega$ might be replaced by 

%
\begin{align}
\tilde{\nabla}^b\tilde{\nabla}_b\tilde{\nabla}^a\tilde{\nabla}_a\alpha&=-\frac{1}{2}\tilde{\nabla}^b\tilde{\nabla}_b\tilde{\nabla}^i\tilde{\nabla}_if_{00}
+\frac{1}{4}\tilde{\nabla}^a\tilde{\nabla}_a\left(-\tilde{\nabla}^b\tilde{\nabla}_bf+\tilde{\nabla}^m\tilde{\nabla}^nf_{mn}\right)
\nonumber\\
&
+\partial_0\tilde{\nabla}^b\tilde{\nabla}_b\tilde{\nabla}^if_{0i}
-\frac{1}{4}\partial^2_0\left(3\tilde{\nabla}^m\tilde{\nabla}^nf_{mn}-\tilde{\nabla}^a\tilde{\nabla}_af\right),
\label{14.7}
\end{align}
%
%
\begin{align}
\tilde{\nabla}^a\tilde{\nabla}_a\tilde{\nabla}^i\tilde{\nabla}_i(B_j-\dot{E_j})&=\tilde{\nabla}^i\tilde{\nabla}_i (\tilde{\nabla}^a\tilde{\nabla}_af_{0j}-\tilde{\nabla}_j\tilde{\nabla}^af_{0a})
-\partial_0\tilde{\nabla}^a\tilde{\nabla}_a\tilde{\nabla}^if_{ij}
\nonumber\\
&
+\partial_0\tilde{\nabla}_j\tilde{\nabla}^a\tilde{\nabla}^bf_{ab},
\label{14.8}
\end{align}
%
%
\begin{align}
2\tilde{\nabla}^a\tilde{\nabla}_a\tilde{\nabla}^b\tilde{\nabla}_bE_{ij}
&=\tilde{\nabla}^a\tilde{\nabla}_a\tilde{\nabla}^b\tilde{\nabla}_bf_{ij}
+\tfrac{1}{2}\tilde{\nabla}_i\tilde{\nabla}_j\left[\tilde{\nabla}^a\tilde{\nabla}^bf_{ab}+\tilde{\nabla}^a\tilde{\nabla}_af\right]
\nonumber\\
&-\tilde{\nabla}^a\tilde{\nabla}_a(\tilde{\nabla}_i\tilde{\nabla}^bf_{jb}+\tilde{\nabla}_j\tilde{\nabla}^bf_{ib})
\nonumber\\
&
+\tfrac{1}{2}\tilde{\gamma}_{ij}\left[\tilde{\nabla}^a\tilde{\nabla}_a\tilde{\nabla}^b\tilde{\nabla}^cf_{bc}
-\tilde{\nabla}_a\tilde{\nabla}^a\tilde{\nabla}_b\tilde{\nabla}^bf\right],
\label{14.9}
\end{align}
%
(where $f=\delta^{ab}f_{ab}$), with $\alpha$ still being given by (\ref{14.6}). Explicit calculation shows that this anticipation is actually borne out, with (\ref{14.7}), (\ref{14.8}) and (\ref{14.9}) being found to hold for the fluctuations given in (\ref{14.5}), no matter how arbitrary $\Omega$ might be. 

Now in our discussion of the fluctuations associated with the conformal gravity $\delta W_{\mu\nu}$ we had obtained the relations given in (\ref{13.10}) and their inversion as given in (\ref{13.13}), (\ref{13.14}) and (\ref{13.15}). We now note that these relations involve the same gauge-invariant combinations as the ones that appear in (\ref{14.7}), (\ref{14.8}) and (\ref{14.9}), viz. $\alpha$, $B_i-\dot{E_i}$ and $E_{ij}$, with $\eta$ not appearing. That $\eta$ could not appear in $\delta W_{\mu\nu}$ is because $W_{\mu\nu}$ is traceless so that on allowing for four coordinate transformations $\delta W_{\mu\nu}$ can only involve five quantities (a one-component $\alpha$, a two-component $B_i-\dot{E}_i$, and a two-component $E_{ij}$). In this sense then we should think of $\alpha$, $B_i-\dot{E_i}$ and $E_{ij}$ as a unit, with $\eta$ needing to be treated independently.

Since $W_{\mu\nu}$ is zero in a conformal to flat background, it is associated with a background $T_{\mu\nu}$ that is zero, with $\delta W_{\mu\nu}$ then being gauge invariant on its own as $\delta T^{\mu\nu}$ is then zero. Thus to determine a gauge-invariant relation that does involve $\eta$ we should look for a purely geometric gauge-invariant fluctuation relation that does not involve $\delta T_{\mu\nu}$. However, none is immediately available as we have already used up $\delta W_{\mu\nu}$, and in general a fluctuation such as $\delta G_{\mu\nu}$ would not be gauge invariant on its own. However, there is one situation in which not $\delta G_{\mu\nu}$ but  $\delta(g^{\mu\nu}G_{\mu\nu})$ is gauge invariant on its own, namely in the radiation era where $T_{\mu\nu}$, and thus $G_{\mu\nu}$, are traceless, with $\delta (g^{\mu\nu}T_{\mu\nu})$ then being zero, and with the quantity $\delta(g^{\mu\nu}G_{\mu\nu})$ then being gauge invariant on its own.

Thus in a radiation era conformal to flat $k\neq 0$ Robertson-Walker background case as given by (\ref{14.5}) we evaluate 
%
\begin{eqnarray}
g^{\mu\nu}G_{\mu\nu}=6\ddot{\Omega}\Omega^{-3}-6\Omega^{-3}\tilde{\nabla}_a\tilde{\nabla}^a\Omega=0,
\label{14.10}
\end{eqnarray}
%
and on setting $\ddot{\Omega}=\tilde{\nabla}_a\tilde{\nabla}^a\Omega$ obtain 
%
\begin{eqnarray}
\delta(g^{\mu\nu} G_{\mu\nu})&=& -6 \dot{\alpha} \dot{\Omega} \Omega^{-3} - 12 \dot{\eta} \dot{\Omega} \Omega^{-3} - 12 \overset{..}{\Omega} \alpha \Omega^{-3} - 6 \overset{..}{\eta} \Omega^{-2} - 2 \Omega^{-2} \tilde{\nabla}_{a}\tilde{\nabla}^{a}\alpha 
\nonumber\\
&&+ 6 \Omega^{-2} \tilde{\nabla}_{a}\tilde{\nabla}^{a}\eta  - 6 \Omega^{-3} \tilde{\nabla}_{a}\Omega \tilde{\nabla}^{a}\alpha
 + 12 \Omega^{-3} \tilde{\nabla}_{a}\Omega \tilde{\nabla}^{a}\eta
\\
 && -12 (B^{a}-\dot{E}^a) \Omega^{-3} \tilde{\nabla}_{a}\dot{\Omega} - 6 (\dot{B}^{a}-\ddot{E}^a) \Omega^{-3} \tilde{\nabla}_{a}\Omega +12 E^{ab} \Omega^{-3} \tilde{\nabla}_{b}\tilde{\nabla}_{a}\Omega,
 \nonumber
\label{14.11}
\end{eqnarray}
%
where $\alpha$, $B_i-\dot{E}_i$ and $E_{ij}$ are as before, with $\eta$ now being given by the form given in (\ref{14.6}). We thus establish that in the conformal to flat case the appropriate $\eta$ is indeed given by the spatially-dependent $\eta=\psi -\Omega^{-1}\dot{\Omega}(B-\dot E)+\Omega^{-1}\tilde\nabla^i\Omega(E_i+\tilde\nabla_i E)$, just as required.

\subsubsection{Taking Advantage of the Gauge Freedom}
\label{sss:taking_advantage_gauge_freedom}

In \cite{amarasinghe_2019} infinitesimal gauge transformations of the form 
%
\begin{eqnarray}
\bar{h}_{\mu\nu}=h_{\mu\nu}-\nabla _{\mu}\epsilon_{\nu}-\nabla _{\nu}\epsilon_{\mu}
\label{14.12}
\end{eqnarray}
% 
acting on the conformal to flat (\ref{14.5}) with arbitrary $\Omega$ were considered. On introducing gauge parameters
%
\begin{eqnarray}
\epsilon_{\mu}=\Omega^2(x)f_{\mu},\quad f_{0}=-T,\quad f_i=L_i+\tilde{\nabla}_iL,\quad \delta^{ij}\tilde{\nabla}_jL_i=\tilde{\nabla}^iL_i=0,
\label{14.13}
\end{eqnarray}
%
the following transformation relations were found
%
\begin{eqnarray}
\bar{\phi}&=&\phi-\dot{T}-\Omega^{-1}[T\partial_0+(L_i+\tilde{\nabla}_iL)\delta^{ij}\partial_j]\Omega,\quad \bar{B}=B+T-\dot{L},
\nonumber\\
 \bar{\psi}&=&\psi+\Omega^{-1}[T\partial_0+(L_i+\tilde{\nabla}_iL)\delta^{ij}\partial_j]\Omega,
\nonumber\\
\bar{E}&=&E-L,\quad \bar{B}_i=B_i-\dot{L}_i,\quad \bar{E}_i=E_i-L_i, \quad \bar{E}_{ij}=E_{ij},
\label{14.14}
\end{eqnarray}
%
with the elimination of the gauge parameters leading directly to the gauge-invariant combinations shown in (\ref{14.6}). We now specialize to a particular gauge, and pick the gauge parameters so that 
%
\begin{eqnarray}
L_i=E_i,\quad L=E, \quad B+T-\dot{L}=0.
\label{14.15}
\end{eqnarray}
%
With this choice (\ref{14.6}) simplifies to
%
\begin{eqnarray}
\alpha &=& \phi +\psi, \quad \eta=\psi ,\quad B_i,\quad E_{ij},
\label{14.16}
\end{eqnarray}
%
and now $\eta$ only depends on scalars. The combinations given in (\ref{14.16}) constitute the longitudinal or conformal-Newtonian gauge, a gauge that is often considered in cosmological perturbation theory (see e.g. \cite{mukhanov_1992, bertschinger_2000}). We thus see that using the gauge freedom one can find gauges in which there is no intertwining of scalars and vectors, so that for them a decomposition theorem in the separate scalar, vector and tensor sectors is achievable.

%%%%%%%%%%%%%%%%%%%%%%%%%%%%%%%%%%%%%%%%%%%%
\section{SVT4}
\label{s:svt4}
%%%%%%%%%%%%%%%%%%%%%%%%%%%%%%%%%%%%%%%%%%%%
As well as discuss SVT3 fluctuations around general Robertson-Walker backgrounds in Einstein gravity,  it is of interest to discuss SVT4 fluctuations  as well.
%%%%%%%%%%%%%%%%%%%%%%%%%%%%%%%%%%%%%%%%%%%%
\subsection{Minkowski}
\label{ss:minkowski_svt4}
%%%%%%%%%%%%%%%%%%%%%%%%%%%%%%%%%%%%%%%%%%%%

In treating the fluctuation equations there are two types of perturbation that one needs to consider. If we start with the Einstein equations in the presence of some general non-zero background $T_{\mu\nu}$, viz. $G_{\mu\nu}+8\pi G T_{\mu\nu}=0$, the first type is to consider perturbations $\delta G_{\mu\nu}$ and $\delta T_{\mu\nu}$ to the background and look for solutions to  
%
\begin{eqnarray}
\delta G_{\mu\nu}+8\pi G \delta T_{\mu\nu}=0
\label{5.1}
\end{eqnarray}
%
in a background that obeys $G_{\mu\nu}+8\pi G T_{\mu\nu}=0$. If the background is not flat, the fluctuation $\delta G_{\mu\nu}$ will not be gauge invariant  but it will instead be the combination $\delta G_{\mu\nu}+8\pi G \delta T_{\mu\nu}$  that will be expressible in the gauge-invariant SVT3 or SVT4 bases as appropriately generalized to a non-flat background.

The second kind of perturbation is one in which we introduce some new perturbation $\delta \bar{T}_{\mu\nu}$ to a background that obeys $G_{\mu\nu}+8\pi G T_{\mu\nu}=0$. This $\delta \bar{T}_{\mu\nu}$ will modify both the background $G_{\mu\nu}$ and the background $T_{\mu\nu}$ and will lead to a fluctuation equation of the form 
%
\begin{eqnarray}
\delta G_{\mu\nu}+8\pi G \delta T_{\mu\nu}=-8 \pi G \delta \bar{T}_{\mu\nu}. 
\label{5.2}
\end{eqnarray}
%
In (\ref{5.2}) the combination $\delta G_{\mu\nu}+8\pi G \delta T_{\mu\nu}$ will be gauge invariant since structurally it will be of the same form as it would be in the absence of $\delta \bar{T}_{\mu\nu}$, and would thus be gauge invariant since it already was in the absence of $\delta \bar{T}_{\mu\nu}$. In consequence of this any $\delta \bar{T}_{\mu\nu}$ that we could introduce would have to be gauge invariant all on its own.

If there is no background $T_{\mu\nu}$ so that the background metric is flat, the only perturbation that one could consider is $\delta \bar{T}_{\mu\nu}$, with the fluctuation equation then being of the form 
%
\begin{eqnarray}
\delta G_{\mu\nu}=-8 \pi G \delta \bar{T}_{\mu\nu}. 
\label{5.3}
\end{eqnarray}
%
With  $\delta \bar{T}_{\mu\nu}$ obeying $\nabla^{\nu}\delta \bar{T}_{\mu\nu}=0$, in analog to (\ref{3.10}) in general in the SVT4 case $\delta \bar{T}_{\mu\nu}$ must be of the form $\nabla_{\alpha}\nabla^{\alpha}\bar{F}_{\mu\nu}+2(g_{\mu\nu}\nabla_{\alpha}\nabla^{\alpha}-\nabla_{\mu}\nabla_{\nu})\bar{\chi}$, with (\ref{5.3}) taking the form 
%
\begin{eqnarray}
\nabla_{\alpha}\nabla^{\alpha}F_{\mu\nu}+2(g_{\mu\nu}\nabla_{\alpha}\nabla^{\alpha}-\nabla_{\mu}\nabla_{\nu})\chi&=&-8 \pi G[\nabla_{\alpha}\nabla^{\alpha}\bar{F}_{\mu\nu}
\\
&&+2(g_{\mu\nu}\nabla_{\alpha}\nabla^{\alpha}-\nabla_{\mu}\nabla_{\nu})\bar{\chi}]. 
\nonumber
\label{5.4}
\end{eqnarray}
%
The idea behind the decomposition theorem is that the tensor and scalar sectors of (\ref{5.4}) satisfy (\ref{5.4}) independently, so that one can set 
%
\begin{eqnarray}
\nabla_{\alpha}\nabla^{\alpha}F_{\mu\nu}&=&-8 \pi G\nabla_{\alpha}\nabla^{\alpha}\bar{F}_{\mu\nu},
\nonumber\\
(g_{\mu\nu}\nabla_{\alpha}\nabla^{\alpha}-\nabla_{\mu}\nabla_{\nu})\chi&=&-8 \pi G(g_{\mu\nu}\nabla_{\alpha}\nabla^{\alpha}-\nabla_{\mu}\nabla_{\nu})\bar{\chi}. 
\label{5.5}
\end{eqnarray}
%
To see whether this is the case we take the trace of (\ref{5.4}), to obtain 
%
\begin{eqnarray}
\nabla_{\alpha}\nabla^{\alpha}\chi=-8 \pi G\nabla_{\alpha}\nabla^{\alpha}\bar{\chi}. 
\label{5.6}
\end{eqnarray}
%
If we now apply $\nabla_{\alpha}\nabla^{\alpha}$ to (\ref{5.4}), then given (\ref{5.6})  we obtain
%
\begin{eqnarray}
\nabla_{\alpha}\nabla^{\alpha}\nabla_{\beta}\nabla^{\beta}F_{\mu\nu}=-8\pi G \nabla_{\alpha}\nabla^{\alpha}\nabla_{\beta}\nabla^{\beta}\bar{F}_{\mu\nu}.
\label{5.7}
\end{eqnarray}
% 
Now while this does give us an equation that involves $F_{\mu\nu}$ alone, this equation is not the second-order derivative equation $\nabla_{\alpha}\nabla^{\alpha}F_{\mu\nu}=-8 \pi G\nabla_{\alpha}\nabla^{\alpha}\bar{F}_{\mu\nu}$ that one is looking for. Moreover, getting to (\ref{5.6}) and (\ref{5.7}) is initially as far as we can go, since according to  (\ref{3.10}) only $\nabla_{\alpha}\nabla^{\alpha}\nabla_{\beta}\nabla^{\beta}F_{\mu\nu}$ and $\nabla_{\alpha}\nabla^{\alpha}\chi$ are automatically gauge invariant.


Now initially (\ref{5.6})  does not imply that $\chi$ is necessarily equal to $-8\pi G\bar{\chi}$, since they could differ by any function $f$ that obeys $\nabla_{\alpha}\nabla^{\alpha}f=0$, i.e., by any harmonic function of the form $f(\mathbf{q}\cdot\mathbf{x}-q t)$. However, it is the very introduction of $\bar{\chi}$ that is causing $\chi$ to be non-zero in the first place, and thus $\chi$ must be proportional to $\bar{\chi}$. Hence harmonic functions can be ignored. Then with  $\chi=-8\pi G\bar{\chi}$, it follows from (\ref{5.4}) that $\nabla_{\alpha}\nabla^{\alpha}F_{\mu\nu}=-8 \pi G\nabla_{\alpha}\nabla^{\alpha}\bar{F}_{\mu\nu}$. And again, since it is the very introduction of $\bar{F}_{\mu\nu}$ that is causing $\delta G_{\mu\nu}$ to be non-zero in the first place, it must be the case that $F_{\mu\nu}=-8 \pi G\bar{F}_{\mu\nu}$. As we see, (\ref{5.5}) does hold, and thus for an external $\delta \bar{T}_{\mu\nu}$ perturbation to a flat background we  obtain the decomposition theorem.

However, in the absence of any explicit external $\delta \bar{T}_{\mu\nu}$ the discussion is different, and is only of relevance in those cases where there is a background $T_{\mu\nu}$, as otherwise $\delta T_{\mu\nu}$ would be zero. When the background $T_{\mu\nu}$ is non-zero and accordingly the background is not flat, the fluctuation quantity $\delta G_{\mu\nu}+8\pi G \delta T_{\mu\nu}$ can still only depend on six gauge-invariant SVT4 combinations, viz. the curved space generalizations of the above $F_{\mu\nu}$ and one combination of $\chi$, $F$ and $F_{\mu}$. Thus in the following we will explore the SVT4 formulation in some non-flat backgrounds that are of cosmological interest.
%%%%%%%%%%%%%%%%%%%%%%%%%%%%%%%%%%%%%%%%%%%%
\subsection{$dS_4$}
\label{ss:ds4_svt4}
%%%%%%%%%%%%%%%%%%%%%%%%%%%%%%%%%%%%%%%%%%%%

\subsubsection{SVT4 $dS_4$ Basis Without a Conformal Factor}
\label{sss:svt4_without_conformal_factor}

Since a background de Sitter metric can be written as a comoving coordinate system metric with no conformal prefactor [viz. $ds^2=dt^2-e^{2Ht}(dx^2+dy^2+dz^2)$],  or written with a conformal prefactor as  a conformal to flat Minkowski metric [$ds^2=(1/\tau H)^2(d\tau^2-dx^2-dy^2-dz^2)$ where $\tau=e^{-Ht}/H$],  in setting up the SVT4 description of fluctuations around a de Sitter background there are then two options. One is to define the fluctuations in terms of $\chi$, $F$, $F_{\mu}$ and $F_{\mu\nu}$ with no multiplying conformal prefactor so that
%
\begin{eqnarray}
h_{\mu\nu}=-2g_{\mu\nu}\chi+2\nabla_{\mu}\nabla_{\nu}F
+ \nabla_{\mu}F_{\nu}+\nabla_{\nu}F_{\mu}+2F_{\mu\nu},
\label{6.1}
\end{eqnarray}
%
with the $\nabla_{\mu}$ derivatives being fully covariant with respect to the de Sitter background so that $\nabla^{\mu}F_{\mu}=0$, $\nabla^{\nu}F_{\mu\nu}=0$. The second is to define the fluctuations with a conformal prefactor so that the fluctuation metric is written as conformal to a flat Minkowski metric according to
%
\begin{eqnarray}
h_{\mu\nu}=\frac{1}{(\tau H)^2}[-2\eta_{\mu\nu}\chi+2\tilde{\nabla}_{\mu}\tilde{\nabla}_{\nu}F
+ \tilde{\nabla}_{\mu}F_{\nu}+\tilde{\nabla}_{\nu}F_{\mu}+2F_{\mu\nu}],
\label{6.2}
\end{eqnarray}
%
with the $\tilde{\nabla}_{\mu}$ derivatives being with respect to flat Minkowski so that $\tilde{\nabla}^{\mu}F_{\mu}=0$, $\tilde{\nabla}^{\nu}F_{\mu\nu}=0$, i.e. $-\dot{F}_0+\tilde{\nabla}^jF_j=0$, $-\dot{F}_{00}+\tilde{\nabla}^jF_{0j}=0$, $-\dot{F}_{0i}+\tilde{\nabla}^jF_{ij}=0$. We shall discuss both options below starting with (\ref{6.1}). 

However, before doing so and in order to be as general as possible we shall initially work in $D$ dimensions where the de Sitter space Riemann tensor takes the form
%
\begin{eqnarray}
R_{\lambda\mu\nu\kappa}=H^2(g_{\mu\nu}g_{\lambda\kappa}-g_{\lambda\nu}g_{\mu\kappa}),
\quad R_{\mu\kappa}=H^2(1-D)g_{\mu\kappa},\quad R^{\alpha}_{\phantom{\alpha}\alpha}=H^2D(1-D).
\nonumber\\
\label{6.3}
\end{eqnarray}
% 
To construct fluctuations we have found it convenient to generalize (\ref{3.1}) to 
%
\begin{eqnarray}
h_{\mu\nu}&=&2F_{\mu\nu}+\nabla_{\nu}W_{\mu}+\nabla_{\mu}W_{\nu}+\frac{2-D}{D-1}\left[\nabla_{\mu}\nabla_{\nu}
+g_{\mu\nu}H^2\right]\times
\nonumber\\
&&\int d^Dy(-g)^{1/2}D^{(E)}(x,y)\nabla^{\alpha}W_{\alpha}
- \frac{g_{\mu\nu}}{D-1}(\nabla^{\alpha}W_{\alpha}-h)
\nonumber\\
&&-\frac{1}{D-1}\left[\nabla_{\mu}\nabla_{\nu}+g_{\mu\nu}H^2\right]\int d^Dy(-g)^{1/2}D^{(E)}(x,y)h,
\label{6.4}
\end{eqnarray}
%
where in the curved background the Green's function obeys
%
\begin{eqnarray}
\left(\nabla_{\nu}\nabla^{\nu}+H^2D\right)D^{(E)}(x,y)=(-g)^{-1/2}\delta^{(D)}(x-y).
\label{6.5}
\end{eqnarray}
%
With this definition $F_{\mu\nu}$ is automatically traceless. On applying $\nabla^{\nu}$ to (\ref{6.4}) and recalling that for any vector or scalar in a de Sitter space  we have
%
\begin{eqnarray}
(\nabla^{\nu}\nabla_{\mu}-\nabla_{\mu}\nabla^{\nu})W_{\nu}&=&H^2(D-1)W_{\mu},
\nonumber\\
(\nabla^{\nu}\nabla_{\mu}\nabla_{\nu}-\nabla_{\mu}\nabla^{\nu}\nabla_{\nu})V&=&H^2(D-1)\nabla_{\mu}V,
\label{6.6}
\end{eqnarray}
%
we obtain
%
\begin{eqnarray}
\nabla^{\nu}h_{\mu\nu}=\nabla_{\nu}\nabla^{\nu}W_{\mu}+H^2(D-1)W_{\mu},
\label{6.7}
\end{eqnarray}
%
with (\ref{6.7}) serving to define $W_{\mu}$. To decompose $W_{\mu}$ into transverse and longitudinal components we set $W_{\mu}=F_{\mu}+\nabla_{\mu}A$ where $\nabla^{\mu}F_{\mu}=0$, $\nabla^{\mu}W_{\mu}=\nabla^{\mu}\nabla_{\mu}A$, and thus set
%
\begin{eqnarray}
W_{\mu}=F_{\mu}+\nabla_{\mu}\int d^Dy(-g)^{1/2}D^{(D)}(x,y)\nabla^{\alpha}W_{\alpha},
\label{6.8}
\end{eqnarray}
%
where
%
\begin{eqnarray}
\nabla_{\nu}\nabla^{\nu}D^{(D)}(x,y)=(-g)^{-1/2}\delta^{(D)}(x-y).
\label{6.9}
\end{eqnarray}
%
Finally, with
%
\begin{eqnarray}
-2\chi&=&\frac{(2-D)H^2}{D-1}\int d^Dy(-g)^{1/2}D^{(E)}(x,y)\nabla^{\alpha}W_{\alpha}-\frac{\nabla^{\alpha}W_{\alpha}-h}{D-1}
\nonumber\\
&&-\frac{H^2}{D-1}\int d^Dy(-g)^{1/2}D^{(E)}(x,y)h,
\nonumber\\
2F&=&\frac{2-D}{D-1}\int d^Dy(-g)^{1/2}D^{(E)}(x,y)\nabla^{\alpha}W_{\alpha}
\nonumber\\
&&-\frac{1}{D-1}\int d^Dy(-g)^{1/2}D^{(E)}(x,y)h
+2\int d^Dy(-g)^{1/2}D^{(D)}(x,y)\nabla^{\alpha}W_{\alpha},
\nonumber\\
F_{\mu}&=&W_{\mu}-\nabla_{\mu}\int d^Dy(-g)^{1/2}D^{(D)}(x,y)\nabla^{\alpha}W_{\alpha},
\label{6.10}
\end{eqnarray}
%
we can now write $h_{\mu\nu}$ as given (\ref{6.1}), with $F_{\mu\nu}$ being given by the transverse-traceless
%
\begin{eqnarray}
2F_{\mu\nu}=h_{\mu\nu}+2g_{\mu\nu}\chi-2\nabla_{\mu}\nabla_{\nu}F
- \nabla_{\mu}F_{\nu}-\nabla_{\nu}F_{\mu},
\label{6.11}
\end{eqnarray}
%
where $W_{\mu}$ is determined from (\ref{6.7}). In this way then we can decompose $h_{\mu\nu}$ into a covariant SVTD in the de Sitter background case.

As well as the above formulation, which involves the Green's function $D^{(E)}(x,y)$, we should note that there is also an alternate formulation that does not involve it at all, one that implements the tracelessness of $F_{\mu\nu}$ using $D^{(D)}(x,y)$ alone, though it does so at the expense of leading to a more complicated expression for $W_{\mu}$. To this end we replace (\ref{6.4}) by 
%
\begin{eqnarray}
h_{\mu\nu}&=&2F_{\mu\nu}+\nabla_{\nu}W_{\mu}+\nabla_{\mu}W_{\nu}+\frac{2-D}{D-1}\nabla_{\mu}\nabla_{\nu}\int d^Dy(-g)^{1/2}D^{(D)}(x,y)\nabla^{\alpha}W_{\alpha}
\nonumber\\
&-&\frac{g_{\mu\nu}}{D-1}(\nabla^{\alpha}W_{\alpha}-h)-\frac{1}{D-1}\nabla_{\mu}\nabla_{\nu}\int d^Dy(-g)^{1/2}D^{(D)}(x,y)h,
\label{6.12}
\end{eqnarray}
%
with $F_{\mu\nu}$ automatically being traceless. To fix $W_{\mu}$ we evaluate 
%
\begin{eqnarray}
\nabla^{\nu}h_{\mu\nu}&=&\nabla_{\nu}\nabla^{\nu}W_{\mu}+H^2(D-1)W_{\mu}
+H^2(2-D)\times
\nonumber\\
&&\nabla_{\mu}\int d^Dy(-g)^{1/2}D^{(D)}(x,y)\nabla^{\alpha}W_{\alpha}
\nonumber\\
&&
-H^2\nabla_{\mu}\int d^Dy(-g)^{1/2}D^{(D)}(x,y)h,
\nonumber\\
\nabla^{\mu}\nabla^{\nu}h_{\mu\nu}&=&\nabla^{\mu}\nabla_{\nu}\nabla^{\nu}W_{\mu}+H^2(\nabla^{\nu}W_{\nu}-h).
\label{6.13}
\end{eqnarray}
%
In terms of (\ref{6.12}) and (\ref{6.8}) we can set 
%
\begin{eqnarray}
2\chi&=&\frac{1}{D-1}[\nabla^{\alpha}W_{\alpha}-h]
\nonumber\\
 2F&=&\frac{1}{D-1}\int d^Dy(-g)^{1/2}D^{(D)}(x,y)[D\nabla^{\alpha}W_{\alpha}-h],
\nonumber\\
F_{\mu}&=&W_{\mu}-\nabla_{\mu}\int d^Dy(-g)^{1/2}D^{(D)}(x,y)\nabla^{\alpha}W_{\alpha},
\nonumber\\
2F_{\mu\nu}&=&h_{\mu\nu}+2g_{\mu\nu}\chi-2\nabla_{\mu}\nabla_{\nu}F
- \nabla_{\mu}F_{\nu}-\nabla_{\nu}F_{\mu},
\label{6.14}
\end{eqnarray}
%
with $\nabla^{\mu}F_{\mu}=0$ as before, and with (\ref{6.1}) following. Thus either way we are led to (\ref{6.1}) and we now apply it to fluctuations around a background de Sitter geometry.

\subsubsection{Application of SVT4 to de Sitter Fluctuation Equations}
\label{sss:application_to_ds4_svt4}

We now restrict to  four dimensions where in a de Sitter geometry  the background Einstein equations are given by 
%
\begin{eqnarray}
G_{\mu\nu}=-8\pi G T_{\mu\nu}=3H^2g_{\mu\nu}.
\label{6.15}
\end{eqnarray}
% 
The fluctuating Einstein tensor is given by 
%
\begin{eqnarray}
\delta G_{\mu\nu}&=&\frac{1}{2}\left[\nabla_{\alpha}\nabla^{\alpha}h_{\mu\nu}-\nabla_{\nu}\nabla^{\alpha}h_{\alpha\mu}-\nabla_{\mu}\nabla^{\alpha}h_{\alpha\nu}+\nabla_{\mu}\nabla_{\nu}h\right]
\nonumber\\
&&
+\frac{g_{\mu\nu}}{2}\left[\nabla^{\alpha}\nabla^{\beta}h_{\alpha\beta}-\nabla_{\alpha}\nabla^{\alpha}h\right]
+\frac{H^2}{2}\left[4h_{\mu\nu}-g_{\mu\nu}h\right],
\label{6.16}
\end{eqnarray}
% 
while the perturbation in the background $T_{\mu\nu}$ is given by $\delta T_{\mu\nu}=-3H^2h_{\mu\nu}$ (we conveniently set $8\pi G=1$). If we now reexpress these fluctuations in  the SVT4 basis given in (\ref{6.1}) we obtain
%
\begin{eqnarray}
\delta G_{\mu\nu}&=&2g_{\mu\nu}\nabla_{\alpha}\nabla^{\alpha}\chi-2\nabla_{\mu}\nabla_{\nu}\chi
+6H^2\nabla_{\mu}\nabla_{\nu}F
+3H^2 \nabla_{\mu}F_{\nu}+3H^2\nabla_{\nu}F_{\mu}+
\nonumber\\
&&(\nabla_{\alpha}\nabla^{\alpha}+4H^2)F_{\mu\nu},
\label{6.17}
\end{eqnarray}
% 
%
\begin{eqnarray}
3H^2h_{\mu\nu}=3H^2\left[-2g_{\mu\nu}\chi+2\nabla_{\mu}\nabla_{\nu}F
+ \nabla_{\mu}F_{\nu}+\nabla_{\nu}F_{\mu}+2F_{\mu\nu}\right],
\label{6.18}
\end{eqnarray}
% 
and thus 
%
\begin{eqnarray}
\delta G_{\mu\nu}-3H^2h_{\mu\nu}&=&(\nabla_{\alpha}\nabla^{\alpha}-2H^2)F_{\mu\nu}
\nonumber\\
&&+2(g_{\mu\nu}\nabla_{\alpha}\nabla^{\alpha}-\nabla_{\mu}\nabla_{\nu}+3H^2g_{\mu\nu})\chi.
\label{6.19}
\end{eqnarray}
% 
As we see, $\delta G_{\mu\nu}+8\pi G\delta T_{\mu\nu}=\delta G_{\mu\nu}-3H^2h_{\mu\nu}$ only depends on $F_{\mu\nu}$ and $\chi$,  with it thus being these quantities that are  gauge invariant, with the thus non-gauge-invariant $F_{\mu}$ and $F$ dropping out.

In the event that there is an additional source term $\delta\bar{T}_{\mu\nu}$, it must be gauge invariant on its own, and must obey $\nabla^{\nu}\delta \bar{T}_{\mu\nu}=0$ in the de Sitter background, to thus be of the form 
%
\begin{eqnarray}
\delta \bar{T}_{\mu\nu}=\bar{F}_{\mu\nu}+2(g_{\mu\nu}\nabla_{\alpha}\nabla^{\alpha}-\nabla_{\mu}\nabla_{\nu}+3H^2g_{\mu\nu})\bar{\chi}.
\label{6.20}
\end{eqnarray}
% 
With this source the fluctuation equations take the form
%
\begin{eqnarray}
&&(\nabla_{\alpha}\nabla^{\alpha}-2H^2)F_{\mu\nu}+2(g_{\mu\nu}\nabla_{\alpha}\nabla^{\alpha}-\nabla_{\mu}\nabla_{\nu}+3H^2g_{\mu\nu})\chi=\bar{F}_{\mu\nu}
\nonumber\\
&&\qquad\qquad+2(g_{\mu\nu}\nabla_{\alpha}\nabla^{\alpha}-\nabla_{\mu}\nabla_{\nu}+3H^2g_{\mu\nu})\bar{\chi},
\label{6.21}
\end{eqnarray}
%
with trace
%
\begin{eqnarray}
6(\nabla_{\alpha}\nabla^{\alpha}+4H^2)\chi&=&6(\nabla_{\alpha}\nabla^{\alpha}+4H^2)\bar{\chi}.
\label{6.22}
\end{eqnarray}
% 

While the trace condition would only set $\chi=\bar{\chi}+f$ where $f$ obeys  $(\nabla_{\alpha}\nabla^{\alpha}+4H^2)f=0$, when there is a $\bar{\chi}$ source present then it is the cause of fluctuations in the background in the first place, and thus we can only have $\chi=\bar{\chi}$ with any possible $f$ being zero. Then, from  (\ref{6.21}) we obtain 
%
\begin{eqnarray}
(\nabla_{\alpha}\nabla^{\alpha}-2H^2)F_{\mu\nu}=\bar{F}_{\mu\nu},
\label{6.23}
\end{eqnarray}
%
and the decomposition theorem is achieved.

However, if there is no $\delta \bar{T}_{\mu\nu}$ source the fluctuation equations take the form 
%
\begin{eqnarray}
(\nabla_{\alpha}\nabla^{\alpha}-2H^2)F_{\mu\nu}+2(g_{\mu\nu}\nabla_{\alpha}\nabla^{\alpha}-\nabla_{\mu}\nabla_{\nu}+3H^2g_{\mu\nu})\chi=0,
\label{6.24}
\end{eqnarray}
%
with the trace condition  being given by 
%
\begin{eqnarray}
6(\nabla_{\alpha}\nabla^{\alpha}+4H^2)\chi=0,
\label{6.25a}
\end{eqnarray}
%
with spherical Bessel solution
%
\begin{eqnarray}
\chi=\sum_{\bf k} k^2\tau^2[a_2({\bf k})j_2(k\tau)+b_2 ({\bf k})y_2(k\tau)]e^{i{\bf k}\cdot {\bf x}}.
\label{6.26a}
\end{eqnarray}
%
(To obtain this solution for $\chi$ it is more straightforward to use $ds^2=(1/\tau^2 H^2)(d\tau^2-dx^2-dy^2-dz^2)$ as the background de Sitter metric, something we can do regardless of whether or not we include a conformal factor in the fluctuations.) Given the trace condition we can rewrite the evolution equation given in (\ref{6.24}) as 
%
\begin{eqnarray}
(\nabla_{\alpha}\nabla^{\alpha}-2H^2)F_{\mu\nu}-2(g_{\mu\nu}H^2+\nabla_{\mu}\nabla_{\nu})\chi=0.
\label{6.27a}
\end{eqnarray}
%


Since it is not automatic that $\chi$ would obey $(H^2g_{\mu\nu}+\nabla_{\mu}\nabla_{\nu})\chi=0$ even though it does obey  $g^{\mu\nu}(H^2g_{\mu\nu}+\nabla_{\mu}\nabla_{\nu})\chi=0$, it is thus not automatic that (\ref{6.24}) and (\ref{6.27a}) could be replaced by
%
\begin{eqnarray}
(\nabla_{\alpha}\nabla^{\alpha}-2H^2)F_{\mu\nu}=0,\quad (g_{\mu\nu}H^2+\nabla_{\mu}\nabla_{\nu})\chi=0,
\label{6.28a}
\end{eqnarray}
%
as would be required of a decomposition theorem. In fact, since $g_{\mu\nu}\chi$ and $\nabla_{\mu}\nabla_{\nu}\chi$ behave totally differently ($\nabla_{\mu}\nabla_{\nu}\chi$ is non-zero if $\mu\neq \nu$ while $g_{\mu\nu}\chi$ is not), the only way to get a decomposition theorem would be for $\chi$, and thus $a_2({\bf k})$ and $b_2({\bf k})$,  to be zero. As we now show, this can in fact be made to be the case, though it is only a particular solution to the full fluctuation equations.

To explore this possibility we need to obtain an expression that  only depends on $F_{\mu\nu}$, and we note that for any scalar in  $D=4$ de Sitter we  have \cite{mannheim_2012}
%
\begin{eqnarray}
\nabla_{\alpha}\nabla^{\alpha}\nabla_{\mu}\nabla_{\nu}\chi=\nabla_{\mu}\nabla_{\nu}\nabla_{\alpha}\nabla^{\alpha}\chi
-2H^2g_{\mu\nu}\nabla_{\alpha}\nabla^{\alpha}\chi
+8H^2\nabla_{\mu}\nabla_{\nu}\chi.
\label{6.29a}
\end{eqnarray}
% 
Given the trace condition shown in (\ref{6.25a})  we then find that
%
\begin{eqnarray}
(\nabla_{\alpha}\nabla^{\alpha}-4H^2)(g_{\mu\nu}H^2+\nabla_{\mu}\nabla_{\nu})\chi
=(\nabla_{\mu}\nabla_{\nu}-H^2g_{\mu\nu})(\nabla^{\alpha}\nabla^{\alpha}+4H^2)\chi=0,
\label{6.30a}
\end{eqnarray}
%
and from (\ref{6.27a}) we thus obtain the fourth-order derivative equation
%
\begin{eqnarray}
(\nabla_{\alpha}\nabla^{\alpha}-4H^2)(\nabla_{\alpha}\nabla^{\alpha}-2H^2)F_{\mu\nu}=0
\label{6.31a}
\end{eqnarray}
%
for $F_{\mu\nu}$, with a decomposition for the components of the fluctuations thus being found, only in the higher-derivative form given in (\ref{6.30a}) and (\ref{6.31a}) rather than in the second-derivative form given in (\ref{6.28a}). Now $(\nabla_{\alpha}\nabla^{\alpha}-2H^2)F_{\mu\nu}=0$ is a particular solution to (\ref{6.31a}), and for this particular solution it would follow that the only solution to (\ref{6.24}) would then be $\chi=0$, with both the $F_{\mu\nu}$ and $\chi$ sector equations given in  (\ref{6.28a}) then holding.

To determine the conditions under which $(\nabla_{\alpha}\nabla^{\alpha}-2H^2)F_{\mu\nu}=0$ might actually hold we need to look for the general solution to (\ref{6.31a}), and since (\ref{6.31a}) is a covariant equation we can evaluate it in any coordinate system, with conformal to flat Minkowski being the most convenient for the de Sitter background. To this end we recall that in any metric that is conformal to flat Minkowski ($ds^2=-g_{MN}dx^Mdx^N=-\Omega^2(x)\eta_{\mu\nu}dx^{\mu}dx^{\nu}$) one has the relation \cite{mannheim_2012}
%
\begin{eqnarray}
g^{LR}\nabla_{L}\nabla_{R}A_{MN}&=&
\eta^{LR}\Omega^{-2}\partial_{L}\partial_{R}A_{MN}
-2\Omega^{-4}\partial_{M}\Omega\partial_{N}\Omega \eta^{TQ}A_{TQ}
\nonumber\\
&&
-2\eta^{LR}\Omega^{-3}\partial_{L}\partial_{R}\Omega A_{MN}
-2\eta^{LR}\Omega^{-3}\partial_{R}\Omega \partial_{L}A_{MN}
\nonumber\\
&&
+2\Omega^{-4}\eta_{MN}\eta^{TX}\partial_{X}\Omega\eta^{QY} \partial_{Y}\Omega A_{TQ}
+2\eta^{KQ}\Omega^{-3}\partial_{Q}\Omega \partial_{N}A_{KM}
\nonumber\\
&&
+2\eta^{KQ}\Omega^{-3}\partial_{Q}\Omega \partial_{M}A_{KN}
-2\Omega^{-1}\partial_{N}\Omega \nabla_{L}A^{L}_{\phantom{L}M}
\nonumber\\
&&
-2\Omega^{-1}\partial_{M}\Omega \nabla_{L}A^{L}_{\phantom{L}N},
\label{6.32a}
\end{eqnarray}
%
for any rank two tensor $A_{MN}$, with the $\nabla_{L}$ referring to covariant derivatives in the $g_{MN}$ geometry. For an $A_{MN}$ that is transverse and traceless, and for $\Omega=1/\tau H$ (\ref{6.32a}) reduces to  (the dot denotes $\partial/\partial\tau$)
%
\begin{eqnarray}
g^{LR}\nabla_{L}\nabla_{R}A_{MN}&=&
\eta^{LR}\tau^2 H^2\partial_{L}\partial_{R}A_{MN}
+4H^2A_{MN}
-2\tau H^2\dot{A}_{MN}
+2H^2\eta_{MN}A_{00}
\nonumber \\
&+&2\tau H^2\partial_{N}A_{0M}
+2\tau H^2 \partial_{M}A_{0N}.
\label{6.33a}
\end{eqnarray}
%
While the general components of $A_{MN}$ are coupled in (\ref{6.33a}), this is not the case for  $A_{00}$, and so we look at $A_{00}$ and obtain 
%
\begin{eqnarray}
\nabla_{L}\nabla^{L}A_{00}&=&
\eta^{LR}\tau^2 H^2\partial_{L}\partial_{R}A_{00}
+2H^2A_{00}
+2\tau H^2\dot{A}_{00}.
\label{6.34a}
\end{eqnarray}
%
Now in a de Sitter background the identity 
%
\begin{eqnarray}
\nabla_{P}\nabla_{K}\nabla^{K}A^{P}_{\phantom{P}M}
=[\nabla_{K}\nabla^{K}+5H^2]\nabla_{P}A^{P}_{\phantom{P}M}
-2H^2\nabla_{M}A^{P}_{\phantom{P}P}
\end{eqnarray}
%
holds  \cite{mannheim_2012}. Thus if any $A_{MN}$ is transverse and traceless then so is $\nabla_{L}\nabla^{L}A_{MN}$. So let us define $A_{MN}=[\nabla_{L}\nabla^{L}-2H^2]F_{MN}$, with this $A_{MN}$ being traverse and traceless since $F_{MN}$ is. For this $A_{MN}$ (\ref{6.31a}) takes the form
%
\begin{eqnarray}
(\nabla_{\alpha}\nabla^{\alpha}-4H^2)A_{\mu\nu}=0.
\label{6.35a}
\end{eqnarray}
%
Thus for $A_{00}$ we have
%
\begin{eqnarray}
\eta^{LR}\tau^2 H^2\partial_{L}\partial_{R}A_{00}+2\tau H^2\dot{A}_{00}
-2H^2A_{00}=0.
\label{6.36a}
\end{eqnarray}
% 
In a plane wave mode $e^{i{\bf k}\cdot{\bf x}}$ the quantity $A_{00}$ thus obeys
%
\begin{eqnarray}
\ddot{A}_{00}-\frac{2}{\tau}\dot{A}_{00}+k^2A_{00}+\frac{2}{\tau^2}A_{00}=0.
\label{6.37a}
\end{eqnarray}
% 
The general solution to (\ref{6.36a}) is thus
%
\begin{eqnarray}
A_{00}&=&[\nabla_{L}\nabla^{L}-2H^2]F_{00}=\sum_{\bf k} k^4\tau^2[a_{00}({\bf k})j_0(k\tau)+b_{00} ({\bf k})y_0(k\tau)]e^{i{\bf k}\cdot {\bf x}}
\nonumber\\
&=&\sum_{\bf k} k^3\tau[a_{00}({\bf k})\sin(k\tau)+b_{00} ({\bf k})\cos(k\tau)]e^{i{\bf k}\cdot {\bf x}},
\label{6.38a}
\end{eqnarray}
% 
where $a_{00}({\bf k})$ and $b_{00}({\bf k})$ are polarization tensors. (Here and throughout we leave out the complex conjugate solution.)

To see if we can support this solution, or whether we are forced to (\ref{6.28a}), we need to see whether (\ref{6.38a}) is compatible  with  (\ref{6.27a}), and thus require that
%
\begin{eqnarray}
A_{00}&=&(\nabla_{\alpha}\nabla^{\alpha}-2H^2)F_{00}=2(g_{00}H^2+\nabla_{0}\nabla_{0})\chi
\nonumber\\
&=&2\left[-\frac{1}{\tau^2}+\frac{\partial^2}{\partial \tau^2}-\Gamma^{\alpha}_{00}\partial_{\alpha}\right]\chi,
\label{6.39}
\end{eqnarray}
%
when evaluated in the solution  for $\chi$ as given in (\ref{6.26a}). On noting that $\Gamma^{\alpha}_{00}=-\delta^{\alpha}_0/\tau$ in a background de Sitter geometry, we evaluate
%
\begin{eqnarray}
2\left[-\frac{1}{\tau^2}+\frac{\partial^2}{\partial \tau^2}-\Gamma^{\alpha}_{00}\partial_{\alpha}\right][k^2\tau^2j_2(k\tau)]&=&2\left[\frac{\partial^2}{\partial \tau^2}+\frac{1}{\tau}\frac{\partial}{\partial \tau}-\frac{1}{\tau^2}\right][k^2\tau^2j_2(k\tau)]
\nonumber\\
&=&2k^3\tau\sin(k\tau),
\label{6.40}
\end{eqnarray}
%
where we have utilized properties of Bessel functions in the last step. With an analogous expression holding for the $y_2(k\tau)$ term, we thus precisely do confirm (\ref{6.38a}), and on comparing (\ref{6.26a}) with (\ref{6.38a}) obtain
%
\begin{eqnarray}
a_{00}({\bf k})=2a_2({\bf k}), \quad b_{00}({\bf k})=2b_2({\bf k}).
\label{6.41}
\end{eqnarray}
%
As we see, in the general solution we are not at all forced to $\chi=0$ as would be required by the decomposition theorem.

For completeness we note that once we have determined $A_{00}$ we can use (\ref{6.33a}) and the $\nabla_{L}A^{L}_{\phantom{L}M}=0$ and $g^{MN}A_{MN}=0$ conditions to determine the other components of $A_{\mu\nu}$, and note only that they satisfy and behave as  
%
\begin{eqnarray}
&&
\eta^{\mu\nu}\partial_{\mu}A_{0\nu}+\frac{2}{\tau}A_{00}=0,\quad \eta^{\mu\nu}\partial_{\mu}A_{i\nu}+\frac{2}{\tau}A_{0i}=0,
\nonumber\\
&&\left[\frac{\partial^2}{\partial\tau^2}+k^2\right]A_{0i}=\frac{2}{\tau}\partial_iA_{00},
\nonumber\\
&&\left[\frac{\partial^2}{\partial\tau^2}+\frac{2}{\tau}\frac{\partial}{\partial \tau}+k^2\right]A_{ij} =\frac{2}{\tau^2}\delta_{ij}A_{00}+\frac{2}{\tau}\left(\partial_iA_{0j}+\partial_jA_{0i}\right),
\nonumber\\
&&A_{0i}=\sum_{\bf k} ik_ik[-k\tau a_{00}({\bf k})\cos(k\tau)+k\tau b_{00} ({\bf k})\sin(k\tau)
+a_{00}({\bf k})\sin(k\tau)
\nonumber\\
&&\qquad+b_{00} ({\bf k})\cos(k\tau)]e^{i{\bf k}\cdot {\bf x}}, 
\nonumber\\
&&A_{ij}=\sum_{\bf k}k_ik_jk\tau[a_{00}({\bf k})\sin(k\tau)+b_{00} ({\bf k})\cos(k\tau)]e^{i{\bf k}\cdot {\bf x}}
\nonumber\\
&&\qquad+\sum_{\bf k}\bigg[\delta_{ij}k^2-3k_ik_j\bigg]\bigg[-a_{00} ({\bf k})\cos(k\tau)+b_{00}({\bf k})\sin(k\tau)
\nonumber\\
&&\qquad+\frac{1}{k\tau}[a_{00} ({\bf k})\sin(k\tau)+b_{00}({\bf k})\cos(k\tau)
\bigg]e^{i{\bf k}\cdot {\bf x}}.
\label{6.42}
\end{eqnarray}
%
(In order to derive the solutions given in (\ref{6.42}) we needed to include terms that would vanish identically in the left-hand sides of the second-order differential equations so that the first-order $\nabla_{L}A^{L}_{\phantom{L}M}=0$ conditions would then be satisfied.) In this solution we then need to satisfy 
%
\begin{eqnarray}
(\nabla_{\alpha}\nabla^{\alpha}-2H^2)F_{\mu\nu}=A_{\mu\nu},
\label{6.43}
\end{eqnarray}
%
which for the representative $A_{00}$ and $F_{00}$ is of the form
%
\begin{eqnarray}
\ddot{F}_{00}-\frac{2}{\tau}\dot{F}_{00}+k^2F_{00}=-\sum_{\bf k}\frac{k^3}{H^2\tau} [a_{00}({\bf k})\sin(k\tau)+b_{00} ({\bf k})\cos(k\tau)]e^{i{\bf k}\cdot {\bf x}},
\label{6.44}
\end{eqnarray}
% 
with solution
%
\begin{eqnarray}
F_{00}=\sum_{\bf k}\frac{k^2}{2H^2} [-a_{00}({\bf k})\cos(k\tau)+b_{00} ({\bf k})\sin(k\tau)]e^{i{\bf k}\cdot {\bf x}}.
\label{6.45}
\end{eqnarray}
% 


Now in order to get a decomposition theorem in the form given in (\ref{6.28a}) we would need $\chi$ to vanish, i.e. we would need $a_2({\bf k})$ and $b_2({\bf k})$ to vanish. And that would mean that $a_{00}({\bf k})$ and $b_{00}({\bf k})$ would have to vanish as well, and thus not only would $A_{00}$ have to vanish but so would all the other components of $A_{\mu\nu}$ as well. A decomposition theorem would thus require that
%
\begin{eqnarray}
(\nabla_{\alpha}\nabla^{\alpha}-2H^2)F_{\mu\nu}=0,
\label{6.46}
\end{eqnarray}
%
for all components of $F_{\mu\nu}$. To look for a non-trivial solution to (\ref{6.46}) in order to show that the decomposition theorem does in fact have a solution, we note that in a plane wave (\ref{6.46}) reduces to 
%
\begin{eqnarray}
\ddot{F}_{00}-\frac{2}{\tau}\dot{F}_{00}+k^2F_{00}=0,
\label{6.47}
\end{eqnarray}
%
for the representative $F_{00}$ component. The non-trivial solution to (\ref{6.47}) is of the form 
%
\begin{eqnarray}
F_{00}=\sum_{\bf k} k^2\tau^2[c_{00}({\bf k})j_1(k\tau)+d_{00} ({\bf k})y_1(k\tau)]e^{i{\bf k}\cdot {\bf x}}.
\label{6.48}
\end{eqnarray}
% 
The form for $F_{00}$ given in (\ref{6.48}) and its $F_{\mu\nu}$ analogs together with $\chi=0$ thus constitute a non-trivial solution that corresponds to the decomposition theorem, so in this sense the decomposition theorem can be recovered, as it is a specific solution to the full evolution equations. However, there is no compelling reason to restrict the solutions to (\ref{6.35a}) to the trivial $A_{\mu\nu}=0$, with it being (\ref{6.38a}), (\ref{6.42}), (\ref{6.45}) and (\ref{6.48}) that provide the most general solution in the $F_{00}$ sector and its analogs according to 
%
\begin{eqnarray}
F_{00}&=&\sum_{\bf k}\frac{k^2}{2H^2} [-a_{00}({\bf k})\cos(k\tau)+b_{00} ({\bf k})\sin(k\tau)]e^{i{\bf k}\cdot {\bf x}}
\nonumber\\
&&+\sum_{\bf k} k^2\tau^2[c_{00}({\bf k})j_1(k\tau)+d_{00} ({\bf k})y_1(k\tau)]e^{i{\bf k}\cdot {\bf x}},
\label{6.49}
\end{eqnarray}
% 
while at the same time (\ref{6.26a}) is the most general solution in the $\chi$ sector as constrained by (\ref{6.41}). Moreover, in this solution we can choose the coefficients in (\ref{6.41}) so that $F_{\mu\nu}$ and $\chi$ are localized in space. Thus no spatially asymptotic boundary coefficient could affect them. In fact suppose that we could have constrained the solutions by an asymptotic condition. We would need one that would force $A_{\mu\nu}$ to have to vanish in $(\nabla_{\alpha}\nabla^{\alpha}-4H^2)A_{\mu\nu}=0$ while not at the same time forcing $F_{\mu\nu}$ to have to vanish in $(\nabla_{\alpha}\nabla^{\alpha}-2H^2)F_{\mu\nu}=0$, something that would not obviously appear possible to achieve. Thus as we see, in this general solution the decomposition theorem does not hold. And just as we noted in Sec. \ref{ss:decomp_svt4_basis}, in the SVT4 case asymptotic boundary conditions do not force us to the decomposition theorem, to thus provide a completely solvable cosmological model in which the decomposition theorem does not hold. However, we should point out that while we could not make $\chi$ vanish through spatial boundary conditions it would be possible to force $\chi$ to vanish at all times by judiciously choosing initial  conditions at an initial time.  However, there would not appear to be any compelling rationale for doing so, and  to nonetheless do so would appear to be quite contrived. Thus absent any compelling rationale for such a judicious choice or for any other choice at all for that matter (i.e., no compelling rationale that would force $\chi$ to vanish) the decomposition theorem would not hold for SVT4 fluctuations around a de Sitter background.


\subsubsection{Defining the SVT4 Fluctuations With a Conformal Factor}
\label{sss:defining_svt4_without_conformal_factor}

In a 
%
\begin{align}
ds^2=\frac{1}{(\tau H)^2}(d\tau^2-dx^2-dy^2-dz^2)
\label{6.50}
\end{align}
%
de Sitter background with fluctuations of the form 
%
\begin{align}
h_{\mu\nu}=\frac{1}{(\tau H)^2}(-2g_{\mu\nu}\chi+2\tilde{\nabla}_{\mu}\tilde{\nabla}_{\nu}F
+ \tilde{\nabla}_{\mu}F_{\nu}+\tilde{\nabla}_{\nu}F_{\mu}+2F_{\mu\nu}), 
\label{6.51}
\end{align}
%
where $\tilde{\nabla}^{\mu}F_{\mu}=0$, $\tilde{\nabla}^{\nu}F_{\mu\nu}=0$, $g^{\mu\nu}F_{\mu\nu}=0$, and where, as per (\ref{6.2}), the $\tilde{\nabla}_{\mu}$ denote derivatives with respect to the flat Minkowski $\eta_{\mu\nu}dx^{\mu}dx^{\nu}$ metric, we write the fluctuation Einstein tensor as 
%
\begin{eqnarray}
\delta G_{00}&=& -6 \dot{\chi} \tau^{-1} - 2 \tau^{-1} \tilde{\nabla}^2\dot{F} - 2 \tilde{\nabla}^2\chi -2 \tau^{-1} \tilde{\nabla}^2F_{0}- \overset{..}{F}_{00} - 2 \dot{F}_{00} \tau^{-1} + \tilde{\nabla}^2F_{00},
\nonumber\\ 
\delta G_{0i}&=& -2 \tau^{-1} \tilde{\nabla}_{i}\overset{..}{F} + 6 \tau^{-2} \tilde{\nabla}_{i}\dot{F} - 2 \tilde{\nabla}_{i}\dot{\chi} - 2 \tau^{-1} \tilde{\nabla}_{i}\chi +3 \dot{F}_{i} \tau^{-2} 
\nonumber\\
&&- 2 \tau^{-1} \tilde{\nabla}_{i}\dot{F}_{0} + 3 \tau^{-2} \tilde{\nabla}_{i}F_{0}- \overset{..}{F}_{0i}
 + 6 F_{0i} \tau^{-2} +  \tilde{\nabla}^2F_{0i} - 2 \tau^{-1} \tilde{\nabla}_{i}F_{00},
\nonumber\\ 
\delta G_{ij}&=& -2 \overset{..}{\chi}\delta_{ij} + 6 \overset{..}{F}\delta_{ij} \tau^{-2} - 2 \overset{...}{F}\delta_{ij} \tau^{-1} + 2 \dot{\chi}\delta_{ij} \tau^{-1} + 2\delta_{ij} \tau^{-1} \tilde{\nabla}^2\dot{F} 
\nonumber\\
&&+ 2\delta_{ij} \tilde{\nabla}^2\chi - 2 \tau^{-1} \tilde{\nabla}_{j}\tilde{\nabla}_{i}\dot{F}
 + 6 \tau^{-2} \tilde{\nabla}_{j}\tilde{\nabla}_{i}F - 2 \tilde{\nabla}_{j}\tilde{\nabla}_{i}\chi +6 \dot{F}_{0}\delta_{ij} \tau^{-2} 
 \nonumber\\
 &&- 2 \overset{..}{F}_{0}\delta_{ij} \tau^{-1} + 2\delta_{ij} \tau^{-1} \tilde{\nabla}^2F_{0} + 3 \tau^{-2} \tilde{\nabla}_{i}F_{j} 
 + 3 \tau^{-2} \tilde{\nabla}_{j}F_{i} - 2 \tau^{-1} \tilde{\nabla}_{j}\tilde{\nabla}_{i}F_{0}
 \nonumber\\
 &&- \overset{..}{F}_{ij} + 6 F_{ij} \tau^{-2} + 6 F_{00}\delta_{ij} \tau^{-2} +2 \dot{F}_{ij} \tau^{-1} +\tilde{\nabla}^2F_{ij}
- 2 \tau^{-1} \tilde{\nabla}_{i}F_{0j} 
\nonumber\\
&&- 2 \tau^{-1} \tilde{\nabla}_{j}F_{0i},
\nonumber\\
g^{\mu\nu}\delta G_{\mu\nu} &=& 18 H^2 \overset{..}{F} - 6 H^2 \overset{...}{F} \tau + 12 H^2 \dot{\chi} \tau - 6 H^2 \overset{..}{\chi} \tau^2 + 6 H^2 \tau \tilde{\nabla}^2\dot{F} + 6 H^2 \tilde{\nabla}^2F 
\nonumber \\ 
&& + 6 H^2 \tau^2 \tilde{\nabla}^2\chi +24 H^2 \dot{F}_{0} - 6 H^2 \overset{..}{F}_{0} \tau + 6 H^2 \tau \tilde{\nabla}_{k}^2F_{0}+24 H^2 F_{00}.
\label{6.52}
\end{eqnarray}
%
Here the dot denotes the derivative with respect to the conformal time $\tau$ and $\tilde{\nabla}^2=\delta^{ij}\tilde{\nabla}_i\tilde{\nabla}_j$. With a $3H^2h_{\mu\nu}$ perturbation  the fluctuation equations take the form
%
\begin{eqnarray}
\Delta_{00}&=& -6 \overset{..}{F} \tau^{-2} - 6 \dot{\chi} \tau^{-1} - 6 \tau^{-2} \chi - 2 \tau^{-1} \tilde{\nabla}^2\dot{F} - 2 \tilde{\nabla}^2\chi -6 \dot{F}_{0} \tau^{-2} 
\nonumber\\
&&- 2 \tau^{-1} \tilde{\nabla}^2F_{0}
- \overset{..}{F}_{00} 
- 6 F_{00} \tau^{-2} - 2 \dot{F}_{00} \tau^{-1} + \tilde{\nabla}^2F_{00}=0,
\nonumber\\ 
\Delta_{0i}&=& -2 \tau^{-1} \tilde{\nabla}_{i}\overset{..}{F} - 2 \tilde{\nabla}_{i}\dot{\chi} - 2 \tau^{-1} \tilde{\nabla}_{i}\chi -2 \tau^{-1} \tilde{\nabla}_{i}\dot{F}_{0}- \overset{..}{F}_{0i} + \tilde{\nabla}^2F_{0i} 
\nonumber\\
&&- 2 \tau^{-1} \tilde{\nabla}_{i}F_{00}=0,
\nonumber\\ 
\Delta_{ij}&=& -2 \overset{..}{\chi}\delta_{ij} + 6 \overset{..}{F}\delta_{ij} \tau^{-2} - 2 \overset{...}{F}\delta_{ij} \tau^{-1} + 2 \dot{\chi}\delta_{ij} \tau^{-1} + 6\delta_{ij} \tau^{-2} \chi + 2\delta_{ij} \tau^{-1} \tilde{\nabla}^2\dot{F} 
\nonumber\\
&&+ 2\delta_{ij} \tilde{\nabla}^2\chi 
 - 2 \tau^{-1} \tilde{\nabla}_{j}\tilde{\nabla}_{i}\dot{F} - 2 \tilde{\nabla}_{j}\tilde{\nabla}_{i}\chi +6 \dot{F}_{0}\delta_{ij} \tau^{-2} - 2 \overset{..}{F}_{0}\delta_{ij} \tau^{-1} 
 \nonumber\\
 &&+ 2\delta_{ij} \tau^{-1} \tilde{\nabla}^2F_{0}  - 2 \tau^{-1} \tilde{\nabla}_{j}\tilde{\nabla}_{i}F_{0}- \overset{..}{F}_{ij} + 6 F_{00}\delta_{ij} \tau^{-2} + 2 \dot{F}_{ij} \tau^{-1} + \tilde{\nabla}^2F_{ij}
 \nonumber\\
 && -2 \tau^{-1} \tilde{\nabla}_{i}F_{0j} - 2 \tau^{-1} \tilde{\nabla}_{j}F_{0i}=0,
\nonumber\\
g^{\mu\nu}\Delta_{\mu\nu} &=& 24 H^2 \overset{..}{F} - 6 H^2 \overset{...}{F} \tau + 12 H^2 \dot{\chi} \tau - 6 H^2 \overset{..}{\chi} \tau^2 + 24 H^2 \chi + 6 H^2 \tau \tilde{\nabla}^2\dot{F} 
\nonumber \\ 
&& + 6 H^2 \tau^2 \tilde{\nabla}^2\chi +24 H^2 \dot{F}_{0} - 6 H^2 \overset{..}{F}_{0} \tau + 6 H^2 \tau \tilde{\nabla}^2F_{0}+24 H^2 F_{00}=0,
\nonumber\\
\label{6.53}
\end{eqnarray}
%
where $\Delta_{\mu\nu}=\delta G_{\mu\nu}+8\pi G \delta T_{\mu\nu}$.
On introducing $\alpha=\dot{F}+\tau\chi+F_0$ the perturbative equations simplify to 
%
\begin{eqnarray}
\Delta_{00}&=& -6 \dot{\alpha} \tau^{-2} - 2 \tau^{-1} \tilde{\nabla}^2\alpha - \overset{..}{F}_{00}  - 6 F_{00} \tau^{-2} - 2 \dot{F}_{00} \tau^{-1} + \tilde{\nabla}^2F_{00}=0,
\nonumber\\ 
\Delta_{0i}&=& -2 \tau^{-1} \tilde{\nabla}_{i}\dot{\alpha}- \overset{..}{F}_{0i} +  \tilde{\nabla}^2F_{0i} - 2 \tau^{-1} \tilde{\nabla}_{i}F_{00}=0,
\nonumber\\ 
\Delta_{ij}&=&\delta_{ij} \left[- 2 \ddot{\alpha}\tau^{-1}+  6 \dot{\alpha} \tau^{-2}  + 2\tau^{-1} \tilde{\nabla}^2\alpha + 6 F_{00} \tau^{-2}\right]-2\tau^{-1} \tilde{\nabla}_{i}\tilde{\nabla}_{j}\alpha
\nonumber\\
&& - \overset{..}{F}_{ij}  + 2 \dot{F}_{ij} \tau^{-1} + \tilde{\nabla}^2F_{ij} -2 \tau^{-1} \tilde{\nabla}_{i}F_{0j} - 2 \tau^{-1} \tilde{\nabla}_{j}F_{0i}=0,
\nonumber\\
H^{-2}g^{\mu\nu}\Delta_{\mu\nu} &=& 24\dot{\alpha} - 6  \overset{..}{\alpha} \tau + 6  \tau \tilde{\nabla}^2\alpha +24 F_{00}=0.
\label{6.54}
\end{eqnarray}
%
We thus see that $\alpha$ and $F_{\mu\nu}$ are gauge invariant for a total of six (one plus five) gauge-invariant components, just as needed. (In passing we note that the gauge invariant  $\alpha=\dot{F}+\tau\chi+F_0$ actually mixes scalars and vectors.)

While we have written $\Delta_{\mu\nu}$ in the non-manifestly covariant form given (\ref{6.54}) as this will be convenient for actually solving $\Delta_{\mu\nu}=0$ below, since the SVT4 approach is covariant we are able to write the rank two  tensor $\Delta_{\mu\nu}$ in a manifestly covariant form. To do so we introduce a unit  timelike four-vector $U^{\mu}$ whose only non-zero component is $U^{0}$. In terms of this $U^{\mu}$ the gauge-invariant $\alpha$ is now given by the manifestly general coordinate scalar $\alpha=U^{\mu}\partial_{\mu}F+\chi/H\Omega+U^{\mu}F_{\mu}$, while the $F_{00}$ term in $g^{\mu\nu}\Delta_{\mu\nu}$ can be written as $U^{\mu}U^{\nu}F_{\mu\nu}$.


If there is to be a decomposition theorem then (\ref{6.54}) would have to break up into 
%
\begin{align}
-6 \dot{\alpha} \tau^{-2} - 2 \tau^{-1} \tilde{\nabla}^2\alpha=0, \quad - \overset{..}{F}_{00}  - 6 F_{00} \tau^{-2} - 2 \dot{F}_{00} \tau^{-1} + \tilde{\nabla}^2F_{00}&=0,
\nonumber\\ 
-2 \tau^{-1} \tilde{\nabla}_{i}\dot{\alpha}=0,\quad - \overset{..}{F}_{0i} +  \tilde{\nabla}^2F_{0i} - 2 \tau^{-1} \tilde{\nabla}_{i}F_{00}&=0,
\nonumber\\ 
\delta_{ij} \left[- 2 \ddot{\alpha}\tau^{-1}+  6 \dot{\alpha} \tau^{-2}  + 2\tau^{-1} \tilde{\nabla}^2\alpha \right]-2\tau^{-1} \tilde{\nabla}_{i}\tilde{\nabla}_{j}\alpha&=0,
\nonumber\\
6 \delta_{ij}F_{00} \tau^{-2} - \overset{..}{F}_{ij}  + 2 \dot{F}_{ij} \tau^{-1} + \tilde{\nabla}^2F_{ij} - 2 \tau^{-1} \tilde{\nabla}_{i}F_{0j} - 2 \tau^{-1} \tilde{\nabla}_{j}F_{0i}&=0,
\nonumber\\
24\dot{\alpha} - 6  \overset{..}{\alpha} \tau + 6  \tau \tilde{\nabla}^2\alpha=0,\quad 24 F_{00}&=0,
\label{6.55}
\end{align}
%
to then yield
%
\begin{eqnarray}
&& \dot{\alpha}=0,\quad \tilde{\nabla}_{i}\tilde{\nabla}_{j}\alpha=0,\quad \tilde{\nabla}^2\alpha=0,\quad F_{00}=0,\quad  - \ddot{F}_{0i} +  \tilde{\nabla}^2F_{0i} =0,
\nonumber\\ 
&&  - \overset{..}{F}_{ij}  + 2 \dot{F}_{ij} \tau^{-1} + \tilde{\nabla}^2F_{ij} - 2 \tau^{-1} \tilde{\nabla}_{i}F_{0j} - 2 \tau^{-1} \tilde{\nabla}_{j}F_{0i}=0,
\label{6.56}
\end{eqnarray}
%
with the $\epsilon^{ijk}\tilde{\nabla}_j\Delta_{0k}=0$ condition not being needed as it is satisfied identically. The solution to (\ref{6.56}) is the form 
%
\begin{eqnarray}
\nonumber\\ 
&& \alpha=0,\quad F_{00}=0,\quad F_{0i}=\sum _{\bf k}f_{0i}({\bf k})e^{i{\bf k}\cdot {\bf x}-ik\tau},\quad  ik^jf_{0j}({\bf k})=0,
\nonumber\\
&& F_{ij}=\sum _{\bf k}[f_{ij}({\bf k})+\tau\hat{f}_{ij}({\bf k})]e^{i{\bf k}\cdot {\bf x}-ik\tau},
\nonumber\\
&&-ikf_{ij}({\bf k})+\hat{f}_{ij}({\bf k})=ik_jf_{0i}({\bf k})+ik_if_{0j}({\bf k}),
\\
&&\delta^{ij}f_{ij}({\bf k})=0,\quad
\delta^{ij}\hat{f}_{ij}({\bf k})=0,\quad ik^jf_{ij}({\bf k})=-ikf_{0i}({\bf k}),\quad ik^j\hat{f}_{ij}({\bf k})=0,
\nonumber
\label{6.57}
\end{eqnarray}
%
and while the most general solution for  $\alpha$ would be  a constant,  we have imposed an asymptotic spatial boundary condition, which sets the constant to zero.


We now solve the full (\ref{6.54}) exactly  to determine whether and under what conditions (\ref{6.57}) might hold. Eliminating $F_{00}$ between the $\Delta_{00}=0$ and $g^{\mu\nu}\Delta_{\mu\nu}=0$ equations in (\ref{6.54}) yields
%
\begin{eqnarray}
-\frac{\tau}{4}\left(\frac{\partial ^2}{\partial \tau^2}-\tilde{\nabla}^2\right)\left(\frac{\partial ^2}{\partial \tau^2}-\tilde{\nabla}^2\right)\alpha=0,
\label{6.58}
\end{eqnarray}
%
with general solution
%
\begin{eqnarray}
\alpha=\sum_{\bf k}\left(a_{\bf k}+\tau b_{\bf k}\right)e^{i{\bf k}\cdot {\bf x}-ik\tau},
\label{6.59}
\end{eqnarray}
%
where $a_{\bf k}$ and $b_{\bf k}$ are independent of ${\bf x}$ and $\tau$. Given $\alpha$, $F_{00}$ then evaluates to 
%
\begin{eqnarray}
F_{00}=\sum_{\bf k}\left[a_{00}({\bf k})+\tau b_{00}({\bf k})\right]e^{i{\bf k}\cdot {\bf x}-ik\tau},
\label{6.60}
\end{eqnarray}
%
where
%
\begin{eqnarray}
a_{00}({\bf k})=ika_{\bf k}-b_{\bf k},\quad b_{00}({\bf k})=\frac{ik}{2}b_{\bf k}.
\label{6.61}
\end{eqnarray}
%
Inserting these solutions for $\alpha$ and $F_{00}$ into $\Delta_{0i}=0$ then yields
%
\begin{eqnarray}
\ddot{F}_{0i}-\tilde{\nabla}^2F_{0i}=-\sum_{\bf k}kk_ib_{\bf k}e^{i{\bf k}\cdot {\bf x}-ik\tau},
\label{6.62}
\end{eqnarray}
%
with solution 
%
\begin{eqnarray}
F_{0i}=\sum_{\bf k}\left[a_{0i}({\bf k})+\tau b_{0i}({\bf k})\right]e^{i{\bf k}\cdot {\bf x}-ik\tau},
\label{6.63}
\end{eqnarray}
%
where
%
\begin{eqnarray}
b_{0i}({\bf k})=-\frac{ik_i}{2}b_{\bf k}.
\label{6.64}
\end{eqnarray}
%
With $F_{0i}$ obeying the transverse condition $\partial^iF_{0i}-\dot{F}_{00}=0$, we obtain 
%
\begin{eqnarray}
ik^ia_{0i}({\bf k})=k^2a_{\bf k}+\frac{3ik}{2}b_{\bf k}.
\label{6.65}
\end{eqnarray}
%
Finally, from $\Delta_{ij}=0$ we obtain 
%
\begin{eqnarray}
\ddot{F}_{ij}  - \frac{2}{\tau} \dot{F}_{ij} - \tilde{\nabla}^2F_{ij}=
\sum_{\bf k}\left[\delta_{ij}ikb_{\bf k}+2k_ik_ja_{\bf k}-2ik_ia_{0j}-2ik_ja_{0i}\right]\frac{1}{\tau}e^{i{\bf k}\cdot {\bf x}-ik\tau}.
\nonumber\\
\label{6.66}
\end{eqnarray}
%
We can thus set 
%
\begin{eqnarray}
F_{ij}=\sum_{\bf k}\left[a_{ij}({\bf k})+\tau b_{ij}({\bf k})\right]e^{i{\bf k}\cdot {\bf x}-ik\tau},
\label{6.67}
\end{eqnarray}
%
where 
%
\begin{eqnarray}
2ika_{ij}({\bf k})-2b_{ij}({\bf k})=\delta_{ij}ikb_{\bf k}+2k_ik_ja_{\bf k}-2ik_ia_{0j}-2ik_ja_{0i}.
\label{6.68}
\end{eqnarray}
%
With $F_{ij}$ obeying the transverse and traceless conditions $\partial^{j}F_{ij}=\dot{F}_{0i}$, $\delta^{ij}F_{ij}-F_{00}=0$, we obtain 
%
\begin{eqnarray}
&&ik^ja_{ij}({\bf k})=-ika_{0i}({\bf k})-\frac{ik_i}{2}b_{\bf k},\quad ik^jb_{ij}({\bf k})=-\frac{kk_i}{2}b_{\bf k}, 
\nonumber\\
&&
\delta^{ij}a_{ij}({\bf k})=ika_{\bf k}-b_{\bf k},
\quad \delta^{ij}b_{ij}({\bf k})=\frac{ik}{2}b_{\bf k}.
\label{6.69}
\end{eqnarray}
%
Equations (\ref{6.58}) to (\ref{6.69}) provide us with the most general solution to (\ref{6.54}).


Having now obtained the exact solution, we see that  we do not get the decomposition theorem solution given in (\ref{6.57}). If we want to get the exact solution to reduce to the decomposition theorem solution we would need to set $\alpha$ and $F_{00}$ to zero, and this would be a particular solution to the fluctuation equations. However, there is no reason to set them to zero, and certainly no spatial asymptotic condition that could do so. And even if there were to be one, then such an asymptotic condition would have to suppress $\alpha$ and $F_{00}$ while at the same time not suppressing $F_{0i}$ and $F_{ij}$, even though though all of the fluctuation components have precisely the same asymptotic spatial behavior. We could possibly set $\alpha$ and $F_{00}$ to zero at all times via judiciously chosen initial conditions, but there would not appear to be any compelling reason for doing that either. As we had seen in our study of SVT4 without a conformal factor we would only be able to recover the decomposition theorem solution if we were to set $\chi$ to zero, and just as with wanting to set $\alpha$ and $F_{00}$ to zero, for $\chi$ there is also no reason  to do so. Thus in parallel with our analysis of SVT4 with no conformal factor, we find that similarly for SVT4 with a conformal factor no decomposition theorem is obtained in the de Sitter background case.


\subsubsection{Some General Comments}
\label{sss:some_general_comments}

While we have discussed SVT4 fluctuations around a de Sitter background as this is a rich enough system to show that one does not in general get a decomposition theorem, this discussion is not the one that is relevant to the early universe  inflationary model since that model is not described by an explicit cosmological constant but by a scalar field instead. Specifically, if we have a scalar field $S(x)$ with a Lagrangian density $L(S)=K(S)-V(S)$, then at the $S=S_0$ minimum of the $V(S)$ potential with constant $S_0$ the potential acts as a cosmological constant $V(S_0)$ and one has a background de Sitter geometry. If we now perturb the background the potential will change to $V(\delta S)$ even though $V(S_0)$ will not change at all. With there also being a change $K(\delta S)$ in  the scalar field kinetic energy, all of the terms in the background $T_{\mu\nu}=\partial_{\mu}S\partial_{\nu}S-g_{\mu\nu}L(S)$ will be perturbed and $\delta T_{\mu\nu}$ will not be of the form $\delta T_{\mu\nu}=\delta g_{\mu\nu}V(S_0)$ that we studied above. Nonetheless, it would not appear that there would obviously be an SVT4 decomposition theorem in  this more general scalar field case.


To conclude this section we note that in the above study of Einstein gravity SVT4 fluctuations around a de Sitter background we found in the no conformal prefactor case that the tensor fluctuations obeyed (\ref{6.31a}), viz.  
%
\begin{eqnarray}
(\nabla_{\alpha}\nabla^{\alpha}-4H^2)(\nabla_{\alpha}\nabla^{\alpha}-2H^2)F_{\mu\nu}=0.
\label{6.70}
\end{eqnarray}
%
Even though (\ref{6.70}) was obtained in Einstein gravity, this very same structure for $F_{\mu\nu}$ also appears in conformal gravity. In \cite{mannheim_2012}  the perturbative conformal gravity Bach tensor $\delta W_{\mu\nu}$ was calculated for fluctuations around a de Sitter background of the form $h_{\mu\nu}=K_{\mu\nu}+g_{\mu\nu}g^{\alpha\beta}h_{\alpha\beta}/4$ (i.e. a traceless but not necessarily transverse $K_{\mu\nu}$), and was found to take the  form
%
\begin{align}
\delta W_{\mu\nu}&=\frac{1}{2}[\nabla_{\alpha}\nabla^{\alpha}-4H^2][\nabla_{\beta}\nabla^{\beta}-2H^2]K_{\mu\nu}
\nonumber\\
&
-\frac{1}{2}[\nabla_{\beta}\nabla^{\beta}-4H^2][
\nabla_{\mu}\nabla_{\lambda}K^{\lambda}_{\phantom{\lambda}\nu}
+\nabla_{\nu}\nabla_{\lambda}K^{\lambda}_{\phantom{\lambda}\mu}]
\nonumber\\
&+\frac{1}{6}[g_{\mu\nu}\nabla_{\alpha}\nabla^{\alpha}+2\nabla_{\mu}\nabla_{\nu}
-6H^2g_{\mu\nu}]\nabla_{\kappa}\nabla_{\lambda}K^{\kappa\lambda}.
\label{6.71}
\end{align}
%
Evaluating $\delta W_{\mu\nu}$ for the fluctuation $h_{\mu\nu}$ given in (\ref{6.1}) in the same de Sitter background is found to yield 
%
\begin{eqnarray}
\delta W_{\mu\nu}= (\nabla_{\alpha}\nabla^{\alpha}-4H^2)(\nabla_{\alpha}\nabla^{\alpha}-2H^2)F_{\mu\nu},
\label{6.72}
\end{eqnarray}
%
i.e. the same structure that we would have obtained from (\ref{6.71}) had we made $K_{\mu\nu}$ transverse and replaced it by $2F_{\mu\nu}$ (even though the relation of $F_{\mu\nu}$ to  $h_{\mu\nu}$ is not the same as that of $K_{\mu\nu}$ to $h_{\mu\nu}$). We recognize the structure of the conformal gravity (\ref{6.72}) as being none other than that of the standard gravity (\ref{6.70}). The transverse-traceless sector of standard gravity (viz. gravity waves) thus has a conformal structure.


Now in a geometry that is conformal to flat such as de Sitter, the background $W_{\mu\nu}$ vanishes identically. Thus from the conformal gravity equation of motion (\ref{AP3}) for the Bach tensor it follows that the background $T_{\mu\nu}$ also vanishes identically. In the absence of a new source $\delta \bar{T}_{\mu\nu}$, for conformal gravity fluctuations around de Sitter  we can thus set 
%
\begin{eqnarray}
4\alpha_g\delta W_{\mu\nu}-\delta T_{\mu\nu}=0,\quad 4\alpha_g\delta W_{\mu\nu}=0,
\label{6.73}
\end{eqnarray}
%
since $\delta T_{\mu\nu}=0$. And since $\delta T_{\mu\nu}$ is zero,  it follows that  $\delta W_{\mu\nu}$ is gauge invariant all on its own in a background that is conformal to flat. And with it being traceless, the five degree of freedom $\delta W_{\mu\nu}$ can only depend on the five degree of freedom $F_{\mu\nu}$, just as we see in (\ref{6.72}). 

Now from  (\ref{6.19}) we can identify $(\nabla_{\alpha}\nabla^{\alpha}-2H^2)F_{\mu\nu}$ as the transverse-traceless piece of $\delta G_{\mu\nu}-3H^2h_{\mu\nu}$ in a de Sitter background, and thus can set 
%
\begin{eqnarray}
(\nabla_{\alpha}\nabla^{\alpha}-4H^2)(\delta G_{\mu\nu}+\delta T_{\mu\nu})^{T\theta}=(\nabla_{\alpha}\nabla^{\alpha}-4H^2)(\nabla_{\alpha}\nabla^{\alpha}-2H^2)F_{\mu\nu}
\label{6.74}
\end{eqnarray}
%
for the transverse ($T$) traceless ($\theta$) sector of $\delta G_{\mu\nu}+\delta T_{\mu\nu}$. Thus given (\ref{6.72}) we can set
%
\begin{eqnarray}
\delta W_{\mu\nu}=(\nabla_{\alpha}\nabla^{\alpha}-4H^2)(\delta G_{\mu\nu}+\delta T_{\mu\nu})^{T\theta}.
\label{6.75}
\end{eqnarray}
%
We thus generalize the flat space fluctuation relation $\delta W_{\mu\nu}=\nabla_{\alpha}\nabla^{\alpha}\delta G_{\mu\nu}^{T\theta}$ to the de Sitter case. Finally, we note that if the  $\delta\bar{T}_{\mu\nu}$ source is present, then its tracelessness in the conformal case restricts its form in (\ref{6.20}) to $\delta \bar{T}_{\mu\nu}=\bar{F}_{\mu\nu}$, with the conformal gravity fluctuation equation in a de Sitter background then taking the form 
%
\begin{eqnarray}
4\alpha_g(\nabla_{\alpha}\nabla^{\alpha}-4H^2)(\nabla_{\alpha}\nabla^{\alpha}-2H^2)F_{\mu\nu}=\bar{F}_{\mu\nu}.
\label{6.76}
\end{eqnarray}
%
Thus with or without $\delta \bar{T}_{\mu\nu}$, in the conformal gravity SVT4 de Sitter case $\delta W_{\mu\nu}$ depends on $F_{\mu\nu}$ alone, and with there being no dependence on $\chi$ the decomposition theorem is automatic. 
%%%%%%%%%%%%%%%%%%%%%%%%%%%%%%%%%%%%%%%%%%%%
\subsection{General Robertson Walker}
\label{ss:general_rw_SVT4}
%%%%%%%%%%%%%%%%%%%%%%%%%%%%%%%%%%%%%%%%%%%%

\subsubsection{The Background}
\label{sss:the_background_svt4_rw}

Let us take the background metric and the 3-space Ricci tensor to be of the form 
%
\begin{eqnarray}
ds^2 &=&-g_{\mu\nu}dx^{\mu}dx^{\nu}=\Omega^2(\tau)\left(d\tau^2 -\tilde{\gamma}_{ij} dx^i dx^j\right),\quad \tilde{R}_{ij} = -2k \tilde{\gamma}_{ij}.
\label{12.1}
\end{eqnarray}
%
Given the symmetry of the 4-geometry, the 4-space Ricci tensor and the 4-space Einstein tensor can be written as 
%
\begin{eqnarray}
R_{\mu\nu} &=& (A+B)U_\mu U_\nu + g_{\mu\nu}B,
\nonumber\\
G_{\mu\nu}&=& \tfrac{1}{2} A g_{\mu \nu } -  \tfrac{1}{2} B g_{\mu \nu } + A U_{\mu } U_{\nu } + B U_{\mu } U_{\nu },
\label{12.2}
\end{eqnarray}
%
where $A$ and $B$ are functions of $\tau$ alone and $U^{\mu}$ is a unit 4-vector that obeys $g_{\mu\nu}U^{\mu}U^{\nu}=-1$. With a background perfect fluid radiation era or matter era source of the form
%
\begin{eqnarray} 
T_{\mu\nu} &=& (\rho+p)U_\mu U_\nu +  p g_{\mu\nu},
\label{12.3}
\end{eqnarray}
%
where $\rho$ and $p$ are functions of $\tau$, the background Einstein equations are of the form
%
\begin{eqnarray}
\Delta_{\mu\nu}^{(0)}&=&\tfrac{1}{2} A g_{\mu \nu } -  \tfrac{1}{2} B g_{\mu \nu } + g_{\mu \nu } p + A U_{\mu } U_{\nu } + B U_{\mu } U_{\nu } + p U_{\mu } U_{\nu } 
\nonumber\\
&&+ U_{\mu } U_{\nu } \rho=0,
\label{12.4}
\end{eqnarray}
%
with solution 
%
\begin{eqnarray}
A &=& -\tfrac{1}{2} (3p+\rho)= -3 \dot{\Omega}^2 \Omega^{-4} + 3 \overset{..}{\Omega} \Omega^{-3}, 
\nonumber\\
B&=& \tfrac{1}{2}(p-\rho)=- \dot{\Omega}^2 \Omega^{-4} -  \overset{..}{\Omega} \Omega^{-3} - 2 k \Omega^{-2}, 
\nonumber\\
\rho &=& \tfrac{1}{2} (- A - 3 B)= 3 \dot{\Omega}^2 \Omega^{-4} + 3 k \Omega^{-2},
\nonumber\\
 p &=& \tfrac{1}{2} (- A + B)
= \dot{\Omega}^2 \Omega^{-4} - 2 \overset{..}{\Omega} \Omega^{-3} -  k \Omega^{-2}.
\label{12.5}
\end{eqnarray}
%

\subsubsection{The SVT4 Fluctuations}
\label{sss:fluctuations_svt4}

While we have incorporated a prefactor of $\Omega^2(\tau)$ in the background metric, we have found it more convenient to not include such a prefactor in the fluctuations. We thus take the background plus fluctuation metric to be of the form
%
\begin{eqnarray}
ds^2 &=&-[g_{\mu\nu}+h_{\mu\nu}]dx^{\mu}dx^{\nu},
\nonumber\\
 h_{\mu\nu}&=& -2 g_{\mu\nu}\chi + 2\nabla_\mu \nabla_\nu F +\nabla_\mu F_\nu +\nabla_\nu F_\mu+ 2F_{\mu\nu},
\label{12.6}
\end{eqnarray}
%
where the $\nabla_{\mu}$ derivatives are with respect to the full background $g_{\mu\nu}$, with respect to which $\nabla^{\mu}F_{\mu}=0$, $\nabla^{\mu}F_{\mu\nu}=0$. In analog to our discussion of SVT3 Robertson-Walker fluctuations given above, we set
%
\begin{eqnarray}
\delta U_{\mu} &=& (V_\mu + \nabla_\mu V) + U_\mu U^\alpha(V_\alpha + \nabla_\alpha V)-U_\mu\left(\tfrac{1}{2} U^\alpha U^\beta h_{\alpha\beta}\right), \quad Q_\mu = F_\mu + \nabla_\mu F, 
\nonumber\\
\hat{V}&=& V-U^\alpha Q_\alpha,
\nonumber\\
\delta \hat{\rho}{} &=& \delta \rho-(\rho+p)( Q^{\alpha } U_{\alpha } \nabla_{\beta }U^{\beta }-Q^{\alpha } U^{\beta } \nabla_{\alpha }U_{\beta }),
\nonumber\\
\delta \hat{p}{} &=& \delta p - \tfrac{1}{3} Q^{\alpha } \nabla_{\alpha }(3p+\rho) +  \tfrac{1}{3} (\rho+p) Q^{\alpha } U_{\alpha } \nabla_{\beta }U^{\beta }.
\label{12.7}
\end{eqnarray}
%
With these definitions and quite a bit of algebra we find that we can write the fluctuation equation $\Delta_{\mu\nu}=0$ as
%
\begin{eqnarray}
&&\Delta_{\mu\nu}= (g_{\mu \nu } + U_{\mu } U_{\nu }) \delta \hat{p}{} + U_{\mu } U_{\nu } \delta \hat{\rho}{} + \bigl((A -  B) g_{\mu \nu } + 2 (A + B) U_{\mu } U_{\nu }\bigr) \chi \nonumber \\ 
&& - 2 (A + B) U_{\mu } U_{\nu } U^{\alpha } \nabla_{\alpha }\hat{V}{} + 2 g_{\mu \nu } \nabla_{\alpha }\nabla^{\alpha }\chi 
\nonumber\\
&&-  (A + B) U_{\nu } \nabla_{\mu }\hat{V}{} -  (A + B) U_{\mu } \nabla_{\nu }\hat{V}{}  - 2 \nabla_{\nu }\nabla_{\mu }\chi -2 (A + B) U_{\mu } U_{\nu } U^{\alpha } V_{\alpha }
\nonumber\\
&& -  (A + B) U_{\nu } V_{\mu }  -  (A + B) U_{\mu } V_{\nu }+2 (A + B) U_{\mu } U_{\nu } U^{\alpha } U^{\beta } F_{\alpha \beta } 
\nonumber\\
&&+ 2 (A + B) U_{\nu } U^{\alpha } F_{\mu \alpha } + (\tfrac{1}{3} A + B) F_{\mu \nu }  + 2 (A + B) U_{\mu } U^{\alpha } F_{\nu \alpha } 
\nonumber\\
&&+ \nabla_{\alpha }\nabla^{\alpha }F_{\mu \nu }=0,
\nonumber\\ 
\nonumber\\
&&g^{\mu\nu}\Delta_{\mu\nu}= 3 \delta \hat{p}{} -  \delta \hat{\rho}{} + 2 (A - 3 B) \chi + 6 \nabla_{\alpha }\nabla^{\alpha }\chi +2 (A + B) U^{\alpha } U^{\beta } F_{\alpha \beta }=0,
\nonumber\\
\label{12.8}
\end{eqnarray}
%
or as
%
%
\begin{eqnarray}
\Delta_{\mu\nu}&=& (g_{\mu \nu } + U_{\mu } U_{\nu }) \delta \hat{p}{} + U_{\mu } U_{\nu } \delta \hat{\rho}{} + \bigl(-2 p g_{\mu \nu } - 2 (p + \rho) U_{\mu } U_{\nu }\bigr) \chi 
\nonumber\\
&&+ 2 (p + \rho) U_{\mu } U_{\nu } U^{\alpha } \nabla_{\alpha }\hat{V}{}  + 2 g_{\mu \nu } \nabla_{\alpha }\nabla^{\alpha }\chi + (p + \rho) U_{\nu } \nabla_{\mu }\hat{V}{} 
\nonumber\\
&&+ (p + \rho) U_{\mu } \nabla_{\nu }\hat{V}{} - 2 \nabla_{\nu }\nabla_{\mu }\chi +2 (p + \rho) U_{\mu } U_{\nu } U^{\alpha } V_{\alpha } \nonumber \\ 
&& + (p + \rho) U_{\nu } V_{\mu } + (p + \rho) U_{\mu } V_{\nu }-2 (p + \rho) U_{\mu } U_{\nu } U^{\alpha } U^{\beta } F_{\alpha \beta } 
\nonumber\\
&&- 2 (p + \rho) U_{\nu } U^{\alpha } F_{\mu \alpha } -  \tfrac{2}{3} \rho F_{\mu \nu }  - 2 (p + \rho) U_{\mu } U^{\alpha } F_{\nu \alpha }
\nonumber\\
&& + \nabla_{\alpha }\nabla^{\alpha }F_{\mu \nu }=0,
\nonumber\\ 
\nonumber\\
g^{\mu\nu}\Delta_{\mu\nu}&=& 3 \delta \hat{p}{} -  \delta \hat{\rho}{} + (-6 p + 2 \rho) \chi + 6 \nabla_{\alpha }\nabla^{\alpha }\chi 
\nonumber\\
&&-2 (p + \rho) U^{\alpha } U^{\beta } F_{\alpha \beta }=0.
\label{12.9}
\end{eqnarray}
%
As written, $\Delta_{\mu\nu}$ only depends on the metric fluctuations $F_{\mu\nu}$ and $\chi$  and  the source fluctuations $\delta \hat{\rho}$,  $\delta \hat{p}$, $\hat{V}$ and $V_i$. Comparing with the SVT3 (\ref{9.13}) to (\ref{9.17})  where there are $\alpha$, $\gamma$, $B_i-\dot{E}_i$ and $E_{ij}$ metric fluctuations and the same set of source fluctuations, we find, just as in the de Sitter background case, that  in a general Robertson-Walker background the SVT4 formalism is far more compact than the SVT3 formalism. 

As a check on our result we note that in a background de Sitter geometry with $\rho=-p=3H^2$, $k=0$, $\Omega=1/\tau H$, $\delta \hat{\rho}=\delta \rho=0$,  $\delta \hat{p}=\delta p=0$, (\ref{12.9}) reduces to 
%
\begin{eqnarray}
\Delta_{\mu\nu}&=& 6 H^2 g_{\mu \nu } \chi + 2 g_{\mu \nu } \nabla_{\alpha }\nabla^{\alpha }\chi - 2 \nabla_{\nu }\nabla_{\mu }\chi -2 H^2 F_{\mu \nu } + \nabla_{\alpha }\nabla^{\alpha }F_{\mu \nu }=0,
\nonumber\\ 
g^{\mu\nu}\Delta_{\mu\nu}&=& 24 H^2 \chi + 6 \nabla_{\alpha }\nabla^{\alpha }\chi=0. 
\label{12.10}
\end{eqnarray}
%
We recognize (\ref{12.10}) as (\ref{6.24}) and (\ref{6.25a}), just as required.

Finally, since the SVT4 fluctuation equations involve the $\nabla_{\alpha }\nabla^{\alpha }$ operator with its curved space harmonic basis functions, as before there will again be no decomposition theorem unless we choose some judicious initial conditions.


%%%%%%%%%%%%%%%%%%%%%%%%%%%%%%%%%%%%%%%%%%%%
\subsection{$\delta W_{\mu\nu}$ Conformal to Flat}
\label{ss:deltaw_conformal_flat_svt4}
%%%%%%%%%%%%%%%%%%%%%%%%%%%%%%%%%%%%%%%%%%%%
For conformal gravity SVT4 fluctuations associated with the metric $g_{\mu\nu}+h_{\mu\nu}$ where the background metric $g_{\mu\nu}$ is of the conformal to flat form given in (\ref{13.7}), we recall that for completely arbitrary conformal factor $\Omega(x)$ the fluctuation $\delta W_{\mu\nu}$ is given by the remarkably simple expression  \cite{amarasinghe_2019}

%
\begin{eqnarray}
\delta W_{\mu\nu}&=&\frac{1}{2}\Omega^{-2}\bigg{(}\partial_{\sigma}\partial^{\sigma}\partial_{\tau}\partial^{\tau}[\Omega^{-2}K_{\mu\nu}]
-\partial_{\sigma}\partial^{\sigma}\partial_{\mu}\partial^{\alpha}[\Omega^{-2}K_{\alpha\nu}]
-\partial_{\sigma}\partial^{\sigma}\partial_{\nu}\partial^{\alpha}[\Omega^{-2}K_{\alpha\mu}]
\nonumber\\
&+&\frac{2}{3}\partial_{\mu}\partial_{\nu}\partial^{\alpha}\partial^{\beta}[\Omega^{-2}K_{\alpha\beta}]+\frac{1}{3}\eta_{\mu\nu}\partial_{\sigma}\partial^{\sigma}\partial^{\alpha}\partial^{\beta}[\Omega^{-2}K_{\alpha\beta}]\bigg{)},
\label{13.18}
\end{eqnarray} 
%
where all derivatives are four-dimensional derivatives with respect to a flat Minkowski metric, and where $K_{\mu\nu}$ is given by $K_{\mu\nu}=h_{\mu\nu}-(1/4)g_{\mu\nu}g^{\alpha\beta}h_{\alpha\beta}$. If we now make the SVT4 expansion
%
\begin{eqnarray}
h_{\mu\nu}=\Omega^2(x)\left[-2\eta_{\mu\nu}\chi+2\partial_{\mu}\partial_{\nu}F
+ \partial_{\mu}F_{\nu}+\partial_{\nu}F_{\mu}+2F_{\mu\nu}\right],
\label{13.19}
\end{eqnarray}
%
where the derivatives and the transverse and tracelessness  $\partial^{\mu}F_{\mu}=0$, $\partial^{\nu}F_{\mu\nu}=0$, $\eta^{\mu\nu}F_{\mu\nu}=0$ conditions are with respect to a flat Minkowski background, we find that (\ref{13.18}) reduces to
%
\begin{eqnarray}
\delta W_{\mu\nu}&=&\Omega^{-2}\partial_{\sigma}\partial^{\sigma}\partial_{\tau}\partial^{\tau}F_{\mu\nu}.
\label{13.20}
\end{eqnarray} 
%
This expression is remarkable not just in its simplicity but in the fact that all components of $F_{\mu\nu}$ are completely decoupled from each other, with (\ref{13.20}) being diagonal in the $\mu,\nu$ indices. Since (\ref{13.20}) only contains $F_{\mu\nu}$ with none of $\chi$, $F$ or $F_{\mu}$ appearing  in it, unlike in the Einstein gravity SVT4 case where one needs initial conditions to establish the decomposition theorem, in the conformal gravity SVT4 case the decomposition theorem is automatic.

\chapter{Construction and Imposition of Gauge Conditions}
\label{c:constructing_gauge_conditions}
In the context of conformal gravity, we continue work done in obtaining solutions to the cosmological fluctuation equations \cite{mannheim_2012} by constructing a gauge condition that is invariant under conformal transformations. Referred to as the conformal gauge, we find that in backgrounds that are conformal to flat, the conformal gravity cosmological fluctuation equations can be brought to an exceedingly simple form, comprised only of a single term. This calculation requires a series of many steps which are given in detail within the chapter, consisting of first motivating and constructing the conformal gauge, composing the fluctuation equations and using curvature identities and covariant derivatives to reduce its form, and finally imposing the conformal gauge itself, with expansion of covariant derivatives into flat Minkowski partial derivatives to yield \eqref{AP61}, one of the seminal results of this thesis. Using the conformal properties detailed in Sec. \ref{ss:conformal_invariance}, we find the extra degree of symmetry afforded by conformal invariance provides significant simplifications regarding the trace and moreover allows a very straightforward treatment of the entire cosmological fluctuation equations.

Such a streamlined approach may be contrasted with the Einstein fluctuations, which are also studied here in Sec. \ref{s:compact_expressions_ein} within differing choices of gauges. Specifically, we form a generalized gauge constraint and vary the coefficients in order to cast $\delta G_{\mu\nu}$ into as reduced and compact form possible, permitting readily solvable integral solutions to be obtained in some simple choices of background geometry (de Sitter and Minkowski). We find that is the coupling of the trace of the fluctuations that prevents the Einstein fluctuation equations from being able to be completely decoupled, unlike the case of conformal gravity where the trace is efficiently isolated as a consequence of conformal invariance.

To demonstrate a concrete application of the fluctuation equations in conformal gravity, in Sec. \ref{s:rw_radiation_conformal_gravity_sol} we evaluate and solve the fluctuations in a $k=-1$ Robertson Walker radiation era cosmology. Here we find that gravitational perturbations naturally have substantial growth, with a leading order time behavior $\propto t^4$ (with $t$ the comoving time). Such may be contrasted with comparatively damped radiation era $t^{1/2}$ leading order behavior one obtains in standard Einstein gravity \cite{weinberg_2008}.

%%%%%%%%%%%%%%%%%%%%%%%%%%%%%%%%%%%%%%%%%%%%
\section{The Conformal Gauge and General Solutions in Conformal Gravity}
\label{s:conformal_gauge_sols}
%%%%%%%%%%%%%%%%%%%%%%%%%%%%%%%%%%%%%%%%%%%%

%%%%%%%%%%%%%%%%%%%%%%%%%%%%%%%%%%%%%%%%%%%%
\subsection{The Conformal Gauge}
\label{ss:conformal_gauge}
%%%%%%%%%%%%%%%%%%%%%%%%%%%%%%%%%%%%%%%%%%%%
As discussed in Sec. \ref{ss:fluctuations_around_flat_in_the_harmonic_gauge}, we recall that under an infinitesimal transformation of coordinates of the form
%
\begin{eqnarray}
x^{\mu}\rightarrow x^{\mu}+\epsilon^{\mu}(x),
\label{liecoord}
\end{eqnarray}
%
the metric perturbation $h_{\mu\nu}$ will transform as
%
\begin{eqnarray}
h_{\mu\nu}-\nabla_{\nu}\epsilon_{\mu}-\nabla_{\mu}\epsilon_{\nu},
\label{covarh}
\end{eqnarray}
%
with contravariant components transforming similarly as
%
\begin{eqnarray}
h^{\mu\nu}-\nabla^{\nu}\epsilon^{\mu}-\nabla^{\mu}\epsilon^{\nu}.
\label{contrah}
\end{eqnarray}
%
Here all covariant derivatives are taken with respect to the background metric $g_{\mu\nu}^{(0)}$. For every solution $h_{\mu\nu}$ that satisfies the fluctuation equations (i.e. $\delta G_{\mu\nu} = \delta T_{\mu\nu}$ or $\delta W_{\mu\nu} = \delta T_{\mu\nu}$), there exists a transformed $h'_{\mu\nu}$ that will also serve as a solution. Thus with the freedom of fixing the four possible arbitrary space-time functions $\epsilon^\mu(x)$, one may eliminate the four gauge degrees of freedom within $h_{\mu\nu}$ by imposing a gauge condition satisfying four equations. Within Ch. \ref{c:formalism} we have already explored the form of the Einstein fluctuation equations in the harmonic gauge as well as the conformal gravity fluctuation equations in the transverse gauge.

In continuing the discussion within conformal gravity, we also recall that in background that are conformal to flat, the gravitational sector fluctuation $\delta W_{\mu\nu}$ (the Bach tensor, analogous to the Einstein tensor) depends only upon the trace free contribution of $h_{\mu\nu}$, to thus be able to be expressed entirely in terms of the 9 component $K_{\mu\nu}$, obeying $K^{\nu\sigma}g^{(0)}_{\nu\sigma}=0$. Our focus is then to determine the most appropriate gauge condition as applied to $K_{\mu\nu}$. We have seen in Sec. \ref{ss:fluctuations_around_flat_in_the_tranverse_gauge} that imposition of the transverse gauge (i.e. $\nabla^\mu K_{\mu\nu} = 0$) was effective in reducing the fluctuation equations into a readily solvable form within a flat space background. However, the analogous imposition of the transverse gauge in curved conformal to flat backgrounds does not provide the same degree of reduction. Rather, in the more general background, the fluctuations take a form where tensor components are tightly coupled, and no simple solution is immediately available. 

In an attempt to remedy the situation, we seek to find a gauge condition that makes contact with the desirable behavior within the flat space background. That is, we want a gauge condition that reduces to the transverse gauge if the background geometry is flat. As the content of this work concerns cosmology, even more specifically we desire a gauge condition that is suitable for conformal to flat backgrounds. The space of all conformal flat backgrounds is in fact quite large, with cosmologically relevant geometries only comprising a small subspace of all possible conformal flat metrics. The above gauge constraint criteria will in fact be satisfied if we can find a gauge condition that is conformally covariant (i.e. invariant under conformal transformation up to an overall scale factor). 

To this end, we first note that under the first order coordinate transform of \eqref{liecoord}, $K^{\mu\nu}$ transforms as
%
\begin{eqnarray}
K^{\mu\nu}\rightarrow K^{\mu\nu}-\nabla^{\nu}\epsilon^{\mu}-\nabla^{\mu}\epsilon^{\nu}+\frac{1}{2}g_{(0)}^{\mu\nu}\nabla_{\alpha}\epsilon^{\alpha}.
\label{AP17}
\end{eqnarray}
% 
With the transverse gauge serving as a starting point, we find its behavior under coordinate transformation, which takes the from
%
\begin{eqnarray}
\nabla_{\nu}K^{\mu\nu}=
\partial_{\nu}K^{\mu\nu}
+K^{\nu\sigma}g_{(0)}^{\mu\rho}\partial_{\nu}g^{(0)}_{\rho\sigma}
-\frac{1}{2}K^{\nu\sigma}g_{(0)}^{\mu\rho}\partial_{\rho}g^{(0)}_{\nu\sigma}
+\frac{1}{2}K^{\mu\sigma}g_{(0)}^{\nu\rho}\partial_{\sigma}g^{(0)}_{\rho\nu},
\label{AP18}
\end{eqnarray}
% 
Under a conformal transformation $\nabla_{\nu}K^{\mu\nu}$ transforms as
%
\begin{eqnarray}
\nabla_{\nu}K^{\mu\nu}\rightarrow \Omega^{-2}\nabla_{\nu}K^{\mu\nu}+4\Omega^{-3}K^{\mu\sigma}\partial_{\sigma}\Omega.
\label{AP19}
\end{eqnarray}
% 
Hence, as alluded to, the transverse gauge condition $\nabla_{\nu}K^{\mu\nu}=0$ is not conformal invariant. To determine such a coordinate gauge condition that is conformal invariant, we refer to \cite{amarasinghe_2019} and note that under a conformal transformation the quantity $K^{\mu\nu}g_{(0)}^{\alpha\beta}\partial_{\nu}g^{(0)}_{\alpha\beta}$ transforms as
%
\begin{eqnarray}
K^{\mu\nu}g_{(0)}^{\alpha\beta}\partial_{\nu}g^{(0)}_{\alpha\beta}
\rightarrow \Omega^{-2}K^{\mu\nu}g_{(0)}^{\alpha\beta}\partial_{\nu}g^{(0)}_{\alpha\beta}
+8\Omega^{-3}K^{\mu\nu}\partial_{\nu}\Omega.
\label{AP20}
\end{eqnarray}
%
Hence, it then follows that 
%
\begin{eqnarray}
\nabla_{\nu}K^{\mu\nu}-\frac{1}{2} K^{\mu\nu}g_{(0)}^{\alpha\beta}\partial_{\nu}g^{(0)}_{\alpha\beta}&\rightarrow &\Omega^{-2}\left[\nabla_{\nu}K^{\mu\nu}
-\frac{1}{2} K^{\mu\nu}g_{(0)}^{\alpha\beta}\partial_{\nu}g^{(0)}_{\alpha\beta}\right]
\nonumber\\
&=&\overline{\nabla_{\nu}K^{\mu\nu}}-\frac{1}{2} \bar{K}^{\mu\nu}\bar{g}_{(0)}^{\alpha\beta}\partial_{\nu}\bar{g}^{(0)}_{\alpha\beta}.
\label{AP21}
\end{eqnarray}
% 
Here, we clarify that $\overline{\nabla_{\nu}K^{\mu\nu}}$ is to be evaluated in a geometry with metric $\bar{g}^{(0)}_{\mu\nu}$ according to 
%
\begin{eqnarray}
\overline{\nabla_{\nu}K^{\mu\nu}}=
\partial_{\nu}\bar{K}^{\mu\nu}
+\bar{K}^{\nu\sigma}\bar{g}_{(0)}^{\mu\rho}\partial_{\nu}\bar{g}^{(0)}_{\rho\sigma}
-\frac{1}{2}\bar{K}^{\nu\sigma}\bar{g}_{(0)}^{\mu\rho}\partial_{\rho}\bar{g}^{(0)}_{\nu\sigma}
+\frac{1}{2}\bar{K}^{\mu\sigma}\bar{g}_{(0)}^{\nu\rho}\partial_{\sigma}\bar{g}^{(0)}_{\rho\nu}.
\label{AP22}
\end{eqnarray}
% 
We see that we have achieved the desired result, with the quantity $\nabla_{\nu}K^{\mu\nu}- K^{\mu\nu}g_{(0)}^{\alpha\beta}\partial_{\nu}g^{(0)}_{\alpha\beta}/2$ conformal covariance. Expressed in three equivalent forms, the gauge condition
%
\begin{eqnarray}
&&\nabla_{\nu}K^{\mu\nu}=\frac{1}{2}K^{\mu\nu}g_{(0)}^{\alpha\beta}\partial_{\nu}g^{(0)}_{\alpha\beta},
\nonumber\\
&&\partial_{\nu}K^{\mu\nu}+\Gamma^{\mu(0)}_{\nu\sigma}K^{\sigma\nu}
+\Gamma^{\nu(0)}_{\nu\sigma}K^{\mu\sigma}=K^{\mu\nu}\Gamma^{\alpha(0)}_{\alpha\nu},
\nonumber\\
&&\partial_{\nu}K^{\mu\nu}+\Gamma^{\mu(0)}_{\nu\sigma}K^{\sigma\nu}=0,
\label{AP23}
\end{eqnarray}
%
is aptly referred to as the conformal gauge.

As a check, we note that when the background is flat Minkowski ($g^{(0)}_{\alpha\beta}=\eta_{\alpha\beta}$), (\ref{AP23}) indeed reduces to the transverse condition $\partial_{\nu}K^{\mu\nu}=0$. Hence we may construct fluctuations around a conformal to flat background in the conformal gauge by conformally transforming fluctuations around a flat background in the transverse gauge. Such a method is not shared within standard Einstein gravity, but in conformal gravity it will be prove to be very beneficial in simplifying the fluctuation equations. 

%%%%%%%%%%%%%%%%%%%%%%%%%%%%%%%%%%%%%%%%%%%%
\subsection{$\delta W_{\mu\nu}$ in an Arbitrary Background}
\label{ss:fluctuation_eqns_around_arb_background_cgauge}
%%%%%%%%%%%%%%%%%%%%%%%%%%%%%%%%%%%%%%%%%%%%

%%%%%%%%%%%%%%%%%%%%%%%%%%%%%%%%%%%%%%%%%%%%
\subsubsection{Composing the Fluctuation Equations}
\label{sss:setting_up_fluctuation_eqns_cgauge}
%%%%%%%%%%%%%%%%%%%%%%%%%%%%%%%%%%%%%%%%%%%%

With the conformal gauge in hand, we proceed in a sequence of steps in order to implement it with the fluctuation equations. Prior to perturbing $W_{\mu\nu}$ we present a useful identity
%
\begin{eqnarray}
\nabla_{\beta}\nabla_{\nu}T_{\lambda \mu}=\nabla_{\nu}\nabla_{\beta}T_{\lambda \mu}+R_{\lambda\sigma\nu\beta}T^{\sigma}_{\phantom{\sigma}\mu}-R_{\sigma\mu\nu\beta}T_{\lambda}^{\phantom{\lambda}\sigma},
\label{AP41}
\end{eqnarray}
%
which holds for any rank two tensor. We then express $W^{\mu\nu}$ as 
%                                                                               
\begin{eqnarray}
W_{\mu \nu}&=&
-\frac{1}{6}g_{\mu\nu}\nabla_{\beta}\nabla^{\beta}R^{\alpha}_{\phantom{\alpha}\alpha}
+\nabla_{\beta}\nabla^{\beta}R_{\mu\nu}                    
-\frac{1}{3}\nabla_{\mu}\nabla_{\nu}R^{\alpha}_{\phantom{\alpha}\alpha}  
-R^{\beta\sigma} R_{\sigma\mu\beta\nu}   
\nonumber\\
&-&R^{\beta\sigma} R_{\sigma\nu\beta\mu}  
+\frac{1}{2}g_{\mu\nu}R_{\alpha\beta}R^{\alpha\beta}                                            
+\frac{2}{3}R^{\alpha}_{\phantom{\alpha}\alpha}R_{\mu\nu}                              
-\frac{1}{6}g_{\mu\nu}(R^{\alpha}_{\phantom{\alpha}\alpha})^2.
\label{AP42}
\end{eqnarray}                                 
%
We set the metric as the most general $g_{\mu\nu}+h_{\mu\nu}$, where $g_{\mu\nu}$ denotes an arbitrary background metric and $\delta g_{\mu\nu}=h_{\mu\nu}$ an arbitrary and general fluctuation. Upon perturbing $W^{\mu\nu}$ we then obtain
%
\begin{eqnarray}
&&\delta W_{\mu\nu}(h_{\mu\nu})=\tfrac{1}{2} h_{\mu \nu} R_{\alpha \beta} R^{\alpha \beta} -  g_{\mu \nu} h^{\alpha \beta} R_{\alpha}{}^{\gamma} R_{\beta \gamma} -  \tfrac{2}{3} h^{\alpha \beta} R_{\alpha \beta} R_{\mu \nu} + \tfrac{1}{3} g_{\mu \nu} h^{\alpha \beta} R_{\alpha \beta} R
\nonumber\\
&& -  \tfrac{1}{6} h_{\mu \nu} R^2 
+ h^{\alpha \beta} R_{\alpha}{}^{\gamma} R_{\mu \beta \nu \gamma} + h^{\alpha \beta} R_{\alpha}{}^{\gamma} R_{\mu \gamma \nu \beta} -  \tfrac{1}{6} h_{\mu \nu} \nabla_{\alpha}\nabla^{\alpha}R -  h^{\alpha \beta} \nabla_{\beta}\nabla_{\alpha}R_{\mu \nu} 
\nonumber\\
&&+ \tfrac{1}{6} g_{\mu \nu} h^{\alpha \beta} \nabla_{\beta}\nabla_{\alpha}R + \tfrac{1}{6} g_{\mu \nu} h^{\alpha \beta} \nabla_{\gamma}\nabla^{\gamma}R_{\alpha \beta} 
+ \tfrac{1}{3} h^{\alpha \beta} \nabla_{\mu}\nabla_{\nu}R_{\alpha \beta}
+\tfrac{1}{3} R \nabla_{\alpha}\nabla^{\alpha}h_{\mu \nu}
\nonumber\\
&& + R_{\mu \beta \nu \gamma} \nabla_{\alpha}\nabla^{\gamma}h^{\alpha \beta} + R_{\mu \gamma \nu \beta} \nabla_{\alpha}\nabla^{\gamma}h^{\alpha \beta} -  \tfrac{1}{3} R \nabla_{\alpha}\nabla_{\mu}h_{\nu}{}^{\alpha} -  \tfrac{1}{3} R \nabla_{\alpha}\nabla_{\nu}h_{\mu}{}^{\alpha} 
\nonumber\\
&&-  \tfrac{1}{6} \nabla_{\alpha}h_{\mu \nu} \nabla^{\alpha}R 
+ \tfrac{1}{6} g_{\mu \nu} \nabla^{\alpha}R \nabla_{\beta}h_{\alpha}{}^{\beta} -  \nabla_{\alpha}h^{\alpha \beta} \nabla_{\beta}R_{\mu \nu} -  \tfrac{2}{3} R_{\mu \nu} \nabla_{\beta}\nabla_{\alpha}h^{\alpha \beta} 
\nonumber\\
&&+ \tfrac{1}{3} g_{\mu \nu} R \nabla_{\beta}\nabla_{\alpha}h^{\alpha \beta} + \tfrac{1}{2} R_{\nu}{}^{\alpha} \nabla_{\beta}\nabla_{\alpha}h_{\mu}{}^{\beta} 
-  R^{\alpha \beta} \nabla_{\beta}\nabla_{\alpha}h_{\mu \nu} 
+ \tfrac{1}{2} R_{\mu}{}^{\alpha} \nabla_{\beta}\nabla_{\alpha}h_{\nu}{}^{\beta} 
\nonumber\\
&&-  \tfrac{1}{2} R_{\nu}{}^{\alpha} \nabla_{\beta}\nabla^{\beta}h_{\mu \alpha} -  \tfrac{1}{2} R_{\mu}{}^{\alpha} \nabla_{\beta}\nabla^{\beta}h_{\nu \alpha} + \tfrac{1}{2} \nabla_{\beta}\nabla^{\beta}\nabla_{\alpha}\nabla^{\alpha}h_{\mu \nu} 
-  \tfrac{1}{2} \nabla_{\beta}\nabla^{\beta}\nabla_{\alpha}\nabla_{\mu}h_{\nu}{}^{\alpha} 
\nonumber\\
&&
-  \tfrac{1}{2} \nabla_{\beta}\nabla^{\beta}\nabla_{\alpha}\nabla_{\nu}h_{\mu}{}^{\alpha} -  \tfrac{1}{2} R_{\nu}{}^{\alpha} \nabla_{\beta}\nabla_{\mu}h_{\alpha}{}^{\beta} + R^{\alpha \beta} \nabla_{\beta}\nabla_{\mu}h_{\nu \alpha} -  \tfrac{1}{2} R_{\mu}{}^{\alpha} \nabla_{\beta}\nabla_{\nu}h_{\alpha}{}^{\beta} 
\nonumber\\
&&+ R^{\alpha \beta} \nabla_{\beta}\nabla_{\nu}h_{\mu \alpha} 
+ \nabla_{\alpha}R_{\nu \beta} \nabla^{\beta}h_{\mu}{}^{\alpha} 
-  \nabla_{\beta}R_{\nu \alpha} \nabla^{\beta}h_{\mu}{}^{\alpha} + \nabla_{\alpha}R_{\mu \beta} \nabla^{\beta}h_{\nu}{}^{\alpha} 
\nonumber\\
&&-  \nabla_{\beta}R_{\mu \alpha} \nabla^{\beta}h_{\nu}{}^{\alpha} -  g_{\mu \nu} R^{\alpha \beta} \nabla_{\gamma}\nabla_{\beta}h_{\alpha}{}^{\gamma} 
+ \tfrac{2}{3} g_{\mu \nu} R^{\alpha \beta} \nabla_{\gamma}\nabla^{\gamma}h_{\alpha \beta} 
-  R_{\mu \alpha \nu \beta} \nabla_{\gamma}\nabla^{\gamma}h^{\alpha \beta} 
\nonumber\\
&&+ \tfrac{1}{6} g_{\mu \nu} \nabla_{\gamma}\nabla^{\gamma}\nabla_{\beta}\nabla_{\alpha}h^{\alpha \beta} + \tfrac{1}{3} g_{\mu \nu} \nabla_{\gamma}R_{\alpha \beta} \nabla^{\gamma}h^{\alpha \beta} -  \nabla_{\beta}R_{\nu \alpha} \nabla_{\mu}h^{\alpha \beta} 
+ \tfrac{1}{6} \nabla^{\alpha}R \nabla_{\mu}h_{\nu \alpha} 
\nonumber\\
&&
-  \tfrac{1}{6} R^{\alpha \beta} \nabla_{\mu}\nabla_{\nu}h_{\alpha \beta} -  \nabla_{\beta}R_{\mu \alpha} \nabla_{\nu}h^{\alpha \beta} + \tfrac{1}{3} \nabla_{\mu}R_{\alpha \beta} \nabla_{\nu}h^{\alpha \beta} + \tfrac{1}{6} \nabla^{\alpha}R \nabla_{\nu}h_{\mu \alpha} 
\nonumber\\
&&+ \tfrac{1}{3} \nabla_{\mu}h^{\alpha \beta} \nabla_{\nu}R_{\alpha \beta} 
-  \tfrac{1}{2} R^{\alpha \beta} \nabla_{\nu}\nabla_{\mu}h_{\alpha \beta} 
+ \tfrac{1}{3} \nabla_{\nu}\nabla_{\mu}\nabla_{\beta}\nabla_{\alpha}h^{\alpha \beta}
+\tfrac{2}{3} R_{\mu \nu} \nabla_{\alpha}\nabla^{\alpha}h 
\nonumber\\
&&-  \tfrac{1}{3} g_{\mu \nu} R \nabla_{\alpha}\nabla^{\alpha}h + \tfrac{1}{2} \nabla_{\alpha}\nabla^{\alpha}\nabla_{\nu}\nabla_{\mu}h 
-  \tfrac{1}{12} g_{\mu \nu} \nabla_{\alpha}h \nabla^{\alpha}R + \tfrac{1}{2} \nabla_{\alpha}R_{\mu \nu} \nabla^{\alpha}h 
\nonumber\\
&&
+ \tfrac{1}{2} g_{\mu \nu} R^{\alpha \beta} \nabla_{\beta}\nabla_{\alpha}h -  \tfrac{1}{6} g_{\mu \nu} \nabla_{\beta}\nabla^{\beta}\nabla_{\alpha}\nabla^{\alpha}h -  R_{\mu \alpha \nu \beta} \nabla^{\beta}\nabla^{\alpha}h + \tfrac{1}{3} R \nabla_{\nu}\nabla_{\mu}h 
\nonumber\\
&&-  \tfrac{1}{3} \nabla_{\nu}\nabla_{\mu}\nabla_{\alpha}\nabla^{\alpha}h.
\label{AP43}
\end{eqnarray}
%
Here we clarify that all covariant derivatives are evaluated with respect to the background $g_{\mu\nu}$ (and use a more compact notation where the Ricci scalar $R$ denotes $R^{\alpha}_{\phantom{\alpha}\alpha}$.) Inspecting (\ref{AP43}), we observe a total of 62 different terms, with ten of these depending on the trace $h=g^{\mu\nu}h_{\mu\nu}$. 

We now substitute $h_{\mu\nu}=K_{\mu\nu}+(1/4)g_{\mu\nu}h$ in (\ref{AP43}) and, as anticipated and discussed in Sec. \ref{ss:conformal_invariance}, we find that $\delta W_{\mu\nu}(h_{\mu\nu})$ breaks into two pieces; one that depends only on $K_{\mu\nu}$ comprising 52 terms and one that depends only on $h=g_{\mu\nu}h^{\mu\nu}$ with 19 total terms. These two separate components have been obtained and take the form
%
\begin{eqnarray}
&&\delta W_{\mu\nu}(K_{\mu\nu})=\tfrac{1}{2} K_{\mu \nu} R_{\alpha \beta} R^{\alpha \beta} -  g_{\mu \nu} K^{\alpha \beta} R_{\alpha}{}^{\gamma} R_{\beta \gamma} -  \tfrac{2}{3} K^{\alpha \beta} R_{\alpha \beta} R_{\mu \nu} 
\nonumber\\
&&+ \tfrac{1}{3} g_{\mu \nu} K^{\alpha \beta} R_{\alpha \beta} R -  \tfrac{1}{6} K_{\mu \nu} R^2 
+ K^{\alpha \beta} R_{\alpha}{}^{\gamma} R_{\mu \beta \nu \gamma} 
+ K^{\alpha \beta} R_{\alpha}{}^{\gamma} R_{\mu \gamma \nu \beta} -  \tfrac{1}{6} K_{\mu \nu} \nabla_{\alpha}\nabla^{\alpha}R 
\nonumber\\
&&-  K^{\alpha \beta} \nabla_{\beta}\nabla_{\alpha}R_{\mu \nu} + \tfrac{1}{6} g_{\mu \nu} K^{\alpha \beta} \nabla_{\beta}\nabla_{\alpha}R + \tfrac{1}{6} g_{\mu \nu} K^{\alpha \beta} \nabla_{\gamma}\nabla^{\gamma}R_{\alpha \beta} 
+ \tfrac{1}{3} K^{\alpha \beta} \nabla_{\mu}\nabla_{\nu}R_{\alpha \beta}
\nonumber\\
&&
+\tfrac{1}{3} R \nabla_{\alpha}\nabla^{\alpha}K_{\mu \nu} + R_{\mu \beta \nu \gamma} \nabla_{\alpha}\nabla^{\gamma}K^{\alpha \beta} + R_{\mu \gamma \nu \beta} \nabla_{\alpha}\nabla^{\gamma}K^{\alpha \beta} -  \tfrac{1}{3} R \nabla_{\alpha}\nabla_{\mu}K_{\nu}{}^{\alpha} 
\nonumber\\
&&-  \tfrac{1}{3} R \nabla_{\alpha}\nabla_{\nu}K_{\mu}{}^{\alpha} 
-  \tfrac{1}{6} \nabla_{\alpha}K_{\mu \nu} \nabla^{\alpha}R 
+ \tfrac{1}{6} g_{\mu \nu} \nabla^{\alpha}R \nabla_{\beta}K_{\alpha}{}^{\beta} -  \nabla_{\alpha}K^{\alpha \beta} \nabla_{\beta}R_{\mu \nu} 
\nonumber\\
&&-  \tfrac{2}{3} R_{\mu \nu} \nabla_{\beta}\nabla_{\alpha}K^{\alpha \beta} + \tfrac{1}{3} g_{\mu \nu} R \nabla_{\beta}\nabla_{\alpha}K^{\alpha \beta} + \tfrac{1}{2} R_{\nu}{}^{\alpha} \nabla_{\beta}\nabla_{\alpha}K_{\mu}{}^{\beta} 
-  R^{\alpha \beta} \nabla_{\beta}\nabla_{\alpha}K_{\mu \nu} 
\nonumber\\
&&
+ \tfrac{1}{2} R_{\mu}{}^{\alpha} \nabla_{\beta}\nabla_{\alpha}K_{\nu}{}^{\beta} -  \tfrac{1}{2} R_{\nu}{}^{\alpha} \nabla_{\beta}\nabla^{\beta}K_{\mu \alpha} -  \tfrac{1}{2} R_{\mu}{}^{\alpha} \nabla_{\beta}\nabla^{\beta}K_{\nu \alpha} + \tfrac{1}{2} \nabla_{\beta}\nabla^{\beta}\nabla_{\alpha}\nabla^{\alpha}K_{\mu \nu} 
\nonumber\\
&&-  \tfrac{1}{2} \nabla_{\beta}\nabla^{\beta}\nabla_{\alpha}\nabla_{\mu}K_{\nu}{}^{\alpha} 
-  \tfrac{1}{2} \nabla_{\beta}\nabla^{\beta}\nabla_{\alpha}\nabla_{\nu}K_{\mu}{}^{\alpha} -  \tfrac{1}{2} R_{\nu}{}^{\alpha} \nabla_{\beta}\nabla_{\mu}K_{\alpha}{}^{\beta} + R^{\alpha \beta} \nabla_{\beta}\nabla_{\mu}K_{\nu \alpha} 
\nonumber\\
&&-  \tfrac{1}{2} R_{\mu}{}^{\alpha} \nabla_{\beta}\nabla_{\nu}K_{\alpha}{}^{\beta} 
+ R^{\alpha \beta} \nabla_{\beta}\nabla_{\nu}K_{\mu \alpha} 
+ \nabla_{\alpha}R_{\nu \beta} \nabla^{\beta}K_{\mu}{}^{\alpha} 
-  \nabla_{\beta}R_{\nu \alpha} \nabla^{\beta}K_{\mu}{}^{\alpha} 
\nonumber\\
&&+ \nabla_{\alpha}R_{\mu \beta} \nabla^{\beta}K_{\nu}{}^{\alpha} -  \nabla_{\beta}R_{\mu \alpha} \nabla^{\beta}K_{\nu}{}^{\alpha} -  g_{\mu \nu} R^{\alpha \beta} \nabla_{\gamma}\nabla_{\beta}K_{\alpha}{}^{\gamma} 
+ \tfrac{2}{3} g_{\mu \nu} R^{\alpha \beta} \nabla_{\gamma}\nabla^{\gamma}K_{\alpha \beta} 
\nonumber\\
&&
-  R_{\mu \alpha \nu \beta} \nabla_{\gamma}\nabla^{\gamma}K^{\alpha \beta} + \tfrac{1}{6} g_{\mu \nu} \nabla_{\gamma}\nabla^{\gamma}\nabla_{\beta}\nabla_{\alpha}K^{\alpha \beta} + \tfrac{1}{3} g_{\mu \nu} \nabla_{\gamma}R_{\alpha \beta} \nabla^{\gamma}K^{\alpha \beta} 
\nonumber\\
&&-  \nabla_{\beta}R_{\nu \alpha} \nabla_{\mu}K^{\alpha \beta} 
+ \tfrac{1}{6} \nabla^{\alpha}R \nabla_{\mu}K_{\nu \alpha} 
-  \tfrac{1}{6} R^{\alpha \beta} \nabla_{\mu}\nabla_{\nu}K_{\alpha \beta} 
-  \nabla_{\beta}R_{\mu \alpha} \nabla_{\nu}K^{\alpha \beta}  
\nonumber\\
&&+ \tfrac{1}{3} \nabla_{\mu}R_{\alpha \beta} \nabla_{\nu}K^{\alpha \beta} 
+ \tfrac{1}{6} \nabla^{\alpha}R \nabla_{\nu}K_{\mu \alpha} + \tfrac{1}{3} \nabla_{\mu}K^{\alpha \beta} \nabla_{\nu}R_{\alpha \beta} 
-  \tfrac{1}{2} R^{\alpha \beta} \nabla_{\nu}\nabla_{\mu}K_{\alpha \beta}
\nonumber\\
&&+ \tfrac{1}{3} \nabla_{\nu}\nabla_{\mu}\nabla_{\beta}\nabla_{\alpha}K^{\alpha \beta},
\label{AP44}
\end{eqnarray}
%
%
\begin{eqnarray}
&&\delta W_{\mu\nu}(h)=- \tfrac{1}{8} g_{\mu \nu} R_{\alpha \beta} R^{\alpha \beta} h -  \tfrac{1}{6} R_{\mu \nu} R h + \tfrac{1}{24} g_{\mu \nu} R^2 h + \tfrac{1}{2} R^{\alpha \beta} R_{\mu \alpha \nu \beta} h
\nonumber\\
&& -  \tfrac{1}{4} h \nabla_{\alpha}\nabla^{\alpha}R_{\mu \nu} + \tfrac{1}{24} g_{\mu \nu} h \nabla_{\alpha}\nabla^{\alpha}R 
+ \tfrac{1}{12} h \nabla_{\nu}\nabla_{\mu}R
+\tfrac{1}{4} \nabla_{\alpha}\nabla^{\alpha}\nabla_{\nu}\nabla_{\mu}h
\nonumber\\
&& -  \tfrac{1}{4} \nabla_{\alpha}R_{\mu \nu} \nabla^{\alpha}h -  \tfrac{1}{2} R_{\mu \alpha \nu \beta} \nabla^{\beta}\nabla^{\alpha}h + \tfrac{1}{4} \nabla_{\mu}R_{\nu \alpha}\nabla^{\alpha}h  -  \tfrac{1}{4} \nabla_{\alpha}R_{\nu}{}^{\alpha} \nabla_{\mu}h
\nonumber\\
&& + \tfrac{1}{4} R_{\nu}{}^{\alpha} \nabla_{\mu}\nabla_{\alpha}h
 + \tfrac{1}{4} \nabla_{\nu}R_{\mu \alpha} \nabla^{\alpha}h + \tfrac{1}{8} \nabla_{\nu}R\nabla_{\mu}h  -  \tfrac{1}{4} \nabla_{\alpha}R_{\mu}{}^{\alpha} \nabla_{\nu}h + \tfrac{1}{8} \nabla_{\mu}R \nabla_{\nu}h +
 \nonumber\\
 && \tfrac{1}{4} R_{\mu}{}^{\alpha} \nabla_{\nu}\nabla_{\alpha}h  
-  \tfrac{1}{4} \nabla_{\nu}\nabla_{\mu}\nabla_{\alpha}\nabla^{\alpha}h.
\label{AP45}
\end{eqnarray}
%
%%%%%%%%%%%%%%%%%%%%%%%%%%%%%%%%%%%%%%%%%%%%
\subsubsection{Trace Properties}
\label{sss:decoupling_trace_cgauge}
Within Sec. \ref{ss:conformal_invariance}, we illustrated an important property of the trace of the fluctation, here given within \eqref{AP45}, where we have demonstrated its form being able to represent as proportional to the background $W_{\mu\nu}$. To show and verify such a form, we first note the identity 
%
\begin{eqnarray}
\nabla_{\kappa}\nabla_{\nu}V_{\lambda}-\nabla_{\nu}\nabla_{\kappa}V_{\lambda}=V^{\sigma}R_{\lambda\sigma\nu\kappa}
\label{AP46}
\end{eqnarray}
%
which holds for any vector $V_{\lambda}$. Upon setting $V_{\lambda}=\nabla_{\lambda}h$ and $T_{\lambda\mu}=\nabla_{\lambda}\nabla_{\mu}h$ in (\ref{AP41}) we obtain
%
\begin{eqnarray}
\nabla_{\nu}\nabla_{\mu}\nabla_{\alpha}\nabla^{\alpha}h
&=&g^{\alpha\beta}\nabla_{\nu}[\nabla_{\alpha}\nabla_{\mu}\nabla_{\beta}h
+R_{\beta\sigma\alpha\mu}\nabla^{\sigma}h]
\nonumber\\
&=&g^{\alpha\beta}\nabla_{\nu}[\nabla_{\alpha}\nabla_{\beta}\nabla_{\mu}h
+R_{\beta\sigma\alpha\mu}\nabla^{\sigma}h]
\nonumber\\
&=&g^{\alpha\beta}[\nabla_{\alpha}\nabla_{\nu}\nabla_{\beta}\nabla_{\mu}h
+R_{\beta\sigma\alpha\nu}\nabla^{\sigma}\nabla_{\mu}h
-R_{\sigma\mu\alpha\nu}\nabla_{\beta}\nabla^{\sigma}h
\nonumber\\
&&
+R_{\beta\sigma\alpha\mu}\nabla_{\nu}\nabla^{\sigma}h
+\nabla_{\nu}R_{\beta\sigma\alpha\mu}\nabla^{\sigma}h]
\nonumber\\
&=&g^{\alpha\beta}[\nabla_{\alpha}[\nabla_{\beta}\nabla_{\nu}\nabla_{\mu}h
+R_{\mu\sigma\beta\nu}\nabla^{\sigma}h]
+R_{\beta\sigma\alpha\nu}\nabla^{\sigma}\nabla_{\mu}h
\nonumber\\
&&
-R_{\sigma\mu\alpha\nu}\nabla_{\beta}\nabla^{\sigma}h
+R_{\beta\sigma\alpha\mu}\nabla_{\nu}\nabla^{\sigma}h
+\nabla_{\nu}R_{\beta\sigma\alpha\mu}\nabla^{\sigma}h].
\label{AP47}
\end{eqnarray}
%
Recalling the curvature relation 
%
\begin{eqnarray}
\nabla^{\nu}R_{\nu\mu\kappa\eta}=\nabla_{\kappa}R_{\mu\eta}-\nabla_{\eta}R_{\mu\kappa},
\label{AP48}
\end{eqnarray}
%
it then follows that 
%
\begin{eqnarray}
&&\nabla_{\nu}\nabla_{\mu}\nabla_{\alpha}\nabla^{\alpha}h-\nabla_{\alpha}\nabla^{\alpha}\nabla_{\nu}\nabla_{\mu}h
=R_{\mu\sigma\alpha\nu}\nabla^{\alpha}\nabla^{\sigma}h
+\nabla_{\mu}R_{\nu\sigma}\nabla^{\sigma}h
-\nabla_{\sigma}R_{\nu\mu}
\nabla^{\sigma}h\nonumber\\
&&+R_{\sigma\nu}\nabla^{\sigma}\nabla_{\mu}h
-R_{\sigma\mu\alpha\nu}\nabla^{\alpha}\nabla^{\sigma}h
+R_{\sigma\mu}\nabla_{\nu}\nabla^{\sigma}h
+\nabla_{\nu}R_{\sigma\mu}\nabla^{\sigma}h.
\label{AP49}
\end{eqnarray}
%
Finally, using the Bianchi identity $\nabla^{\alpha}R_{\mu\alpha}=(1/2)\nabla _{\mu}R$ we thus observe that all twelve terms within (\ref{AP45}) that contain the gradient of $h$ must all cancel. For those seven terms that remain, we can further observe that they are directly equivalent to the definition of the background $W_{\mu\nu}$. Thus, the trace component $\delta W_{\mu\nu}(h)$ reduces to the extremely compact form of
%
\begin{eqnarray}
\delta W_{\mu\nu}(h)=-\frac{1}{4}W_{\mu\nu}h.
\label{AP50}
\end{eqnarray}
%
To gain further insight into the form of \eqref{AP50}, it is instructive to consider the fluctuation of the Weyl tensor itself. In evaluating the perturbed Weyl tensor around an arbitrary background, iwe obtain $\delta C_{\lambda\mu\nu\kappa}=\delta C_{\lambda\mu\nu\kappa}(K_{\mu\nu})+\delta C_{\lambda\mu\nu\kappa}(h)$, where
%
\begin{eqnarray}
&&\delta C_{\lambda\mu\nu\kappa}(K_{\mu\nu})=- \tfrac{1}{6} g_{\kappa \mu} g_{\lambda \nu} K^{\alpha \beta} R_{\alpha \beta} + \tfrac{1}{6} g_{\kappa \lambda} g_{\mu \nu} K^{\alpha \beta} R_{\alpha \beta} + \tfrac{1}{2} K_{\mu \nu} R_{\kappa \lambda} -  \tfrac{1}{2} K_{\lambda \nu} R_{\kappa \mu}  
\nonumber\\
&&-  \tfrac{1}{2} K_{\kappa \mu} R_{\lambda \nu} 
+ \tfrac{1}{2} K_{\kappa \lambda} R_{\mu \nu} -  \tfrac{1}{6} g_{\mu \nu} K_{\kappa \lambda} R 
+ \tfrac{1}{6} g_{\lambda \nu} K_{\kappa \mu} R + \tfrac{1}{6} g_{\kappa \mu} K_{\lambda \nu} R -  \tfrac{1}{6} g_{\kappa \lambda} K_{\mu \nu} R 
\nonumber\\
&& + K_{\lambda}{}^{\alpha} R_{\kappa \nu \mu \alpha} + \tfrac{1}{4} g_{\mu \nu} \nabla_{\alpha}\nabla^{\alpha}K_{\kappa \lambda} -  \tfrac{1}{4} g_{\lambda \nu} \nabla_{\alpha}\nabla^{\alpha}K_{\kappa \mu} 
-  \tfrac{1}{4} g_{\kappa \mu} \nabla_{\alpha}\nabla^{\alpha}K_{\lambda \nu}  
\nonumber\\
&&
+ \tfrac{1}{4} g_{\kappa \lambda} \nabla_{\alpha}\nabla^{\alpha}K_{\mu \nu} 
-  \tfrac{1}{4} g_{\mu \nu} \nabla_{\alpha}\nabla_{\kappa}K_{\lambda}{}^{\alpha} + \tfrac{1}{4} g_{\lambda \nu} \nabla_{\alpha}\nabla_{\kappa}K_{\mu}{}^{\alpha} -  \tfrac{1}{4} g_{\mu \nu} \nabla_{\alpha}\nabla_{\lambda}K_{\kappa}{}^{\alpha}  
\nonumber\\
&&
 + \tfrac{1}{4} g_{\kappa \mu} \nabla_{\alpha}\nabla_{\lambda}K_{\nu}{}^{\alpha} 
+ \tfrac{1}{4} g_{\lambda \nu} \nabla_{\alpha}\nabla_{\mu}K_{\kappa}{}^{\alpha} -  \tfrac{1}{4} g_{\kappa \lambda} \nabla_{\alpha}\nabla_{\mu}K_{\nu}{}^{\alpha} + \tfrac{1}{4} g_{\kappa \mu} \nabla_{\alpha}\nabla_{\nu}K_{\lambda}{}^{\alpha}  
\nonumber\\
&&-  \tfrac{1}{4} g_{\kappa \lambda} \nabla_{\alpha}\nabla_{\nu}K_{\mu}{}^{\alpha} 
-  \tfrac{1}{6} g_{\kappa \mu} g_{\lambda \nu} \nabla_{\beta}\nabla_{\alpha}K^{\alpha \beta} 
+ \tfrac{1}{6} g_{\kappa \lambda} g_{\mu \nu} \nabla_{\beta}\nabla_{\alpha}K^{\alpha \beta} 
-  \tfrac{1}{2} \nabla_{\kappa}\nabla_{\lambda}K_{\mu \nu}  
\nonumber\\
&&+ \tfrac{1}{2} \nabla_{\kappa}\nabla_{\mu}K_{\lambda \nu} 
+ \tfrac{1}{2} \nabla_{\kappa}\nabla_{\nu}K_{\lambda \mu}  
- \tfrac{1}{2} \nabla_{\nu}\nabla_{\kappa}K_{\lambda \mu} + \tfrac{1}{2} \nabla_{\nu}\nabla_{\lambda}K_{\kappa \mu} 
-  \tfrac{1}{2} \nabla_{\nu}\nabla_{\mu}K_{\kappa \lambda}
\nonumber\\
\label{AP51}
\end{eqnarray}
%
and
%
\begin{eqnarray}
\delta C_{\lambda\mu\nu\kappa}(h)&=&\bigg[\tfrac{1}{8} g_{\mu \nu}  R_{\kappa \lambda} -  \tfrac{1}{8} g_{\lambda \nu} R_{\kappa \mu} -  \tfrac{1}{8} g_{\kappa \mu}  R_{\lambda \nu} +\tfrac{1}{8} g_{\kappa \lambda}  R_{\mu \nu}  
\nonumber\\
&&+ \tfrac{1}{24} g_{\kappa \mu} g_{\lambda \nu}  R -  \tfrac{1}{24} g_{\kappa \lambda} g_{\mu \nu}  R -  \tfrac{1}{4}  R_{\kappa \nu \lambda \mu}\bigg]h
\nonumber\\
&=&\frac{1}{4}hC_{\lambda\mu\nu\kappa}.
\label{AP52}
\end{eqnarray}
%
Inspection of \eqref{AP52} reveals that if the background Weyl tensor $C_{\lambda\mu\nu\kappa}$ is zero then it follows that $\delta C_{\lambda\mu\nu\kappa}$ has no dependence upon $h$. Given a vanishing background Weyl tensor, we may also observe that from (\ref{AP3}) $\delta W^{\mu\nu}$ can therebfore be expressed as
%
\begin{eqnarray}
\delta W^{\mu\nu}=2\nabla_{\kappa}\nabla_{\lambda}\delta C^{\mu\lambda\nu\kappa}-
R_{\kappa\lambda}\delta C^{\mu\lambda\nu\kappa},
\label{AP53}
\end{eqnarray}
%
and thereby also be independent of $h$. 

%%%%%%%%%%%%%%%%%%%%%%%%%%%%%%%%%%%%%%%%%%%%
\subsubsection{Differential Commutations}
\label{sss:imposing_cgauge}
%%%%%%%%%%%%%%%%%%%%%%%%%%%%%%%%%%%%%%%%%%%%

Before we can express (\ref{AP44}) as a form ready for application of the conformal gauge condition, we must first commute the differential operators as per (\ref{AP41}) and (\ref{AP46}).  On performing the commutations for $\delta W_{\mu\nu}^{}(K_{\mu\nu})$ we obtain
%
\begin{eqnarray}
&&\delta W_{\mu\nu}^{}(K_{\mu\nu})=\tfrac{1}{2} K_{\mu \nu} R_{\alpha \beta} R^{\alpha \beta} -  \tfrac{1}{2} K_{\nu}{}^{\alpha} R_{\alpha \beta} R_{\mu}{}^{\beta} -  \tfrac{2}{3} K^{\alpha \beta} R_{\alpha \beta} R_{\mu \nu} + K^{\alpha \beta} R_{\mu \alpha} R_{\nu \beta}  
\nonumber\\
&&-  \tfrac{1}{2} K_{\mu}{}^{\alpha} R_{\alpha \beta} R_{\nu}{}^{\beta} + \tfrac{1}{3} g_{\mu \nu} K^{\alpha \beta} R_{\alpha \beta} R 
+ \tfrac{1}{3} K_{\nu}{}^{\alpha} R_{\mu \alpha} R + \tfrac{1}{3} K_{\mu}{}^{\alpha} R_{\nu \alpha} R -  \tfrac{1}{6} K_{\mu \nu} R^2  
\nonumber\\
&&-  g_{\mu \nu} K^{\alpha \beta} R^{\gamma \kappa} R_{\alpha \gamma \beta \kappa} -  \tfrac{2}{3} K^{\alpha \beta} R R_{\mu \alpha \nu \beta} -  K_{\nu}{}^{\alpha} R^{\beta \gamma} R_{\mu \beta \alpha \gamma} + 2 K^{\alpha \beta} R_{\alpha}{}^{\gamma} R_{\mu \gamma \nu \beta} 
\nonumber\\
&&+ 2 K^{\alpha \beta} R_{\alpha \gamma \beta \kappa} R_{\mu}{}^{\gamma}{}_{\nu}{}^{\kappa} -  K_{\mu}{}^{\alpha} R^{\beta \gamma} R_{\nu \beta \alpha \gamma} + \tfrac{1}{3} R \nabla_{\alpha}\nabla^{\alpha}K_{\mu \nu} -  \tfrac{1}{6} K_{\mu \nu} \nabla_{\alpha}\nabla^{\alpha}R  
\nonumber\\
&&+ \tfrac{1}{2} R_{\nu}{}^{\alpha} \nabla_{\alpha}\nabla_{\beta}K_{\mu}{}^{\beta} + \tfrac{1}{2} R_{\mu}{}^{\alpha} \nabla_{\alpha}\nabla_{\beta}K_{\nu}{}^{\beta} 
-  \tfrac{1}{6} \nabla_{\alpha}K_{\mu \nu} \nabla^{\alpha}R + \tfrac{1}{6} g_{\mu \nu} \nabla^{\alpha}R \nabla_{\beta}K_{\alpha}{}^{\beta} 
\nonumber\\
&& -  \nabla_{\alpha}K^{\alpha \beta} \nabla_{\beta}R_{\mu \nu} -  \tfrac{2}{3} R_{\mu \nu} \nabla_{\beta}\nabla_{\alpha}K^{\alpha \beta} + \tfrac{1}{3} g_{\mu \nu} R \nabla_{\beta}\nabla_{\alpha}K^{\alpha \beta} -  R^{\alpha \beta} \nabla_{\beta}\nabla_{\alpha}K_{\mu \nu} 
\nonumber\\
&&-  K^{\alpha \beta} \nabla_{\beta}\nabla_{\alpha}R_{\mu \nu} + \tfrac{1}{6} g_{\mu \nu} K^{\alpha \beta} \nabla_{\beta}\nabla_{\alpha}R + \tfrac{1}{2} K_{\nu}{}^{\alpha} \nabla_{\beta}\nabla^{\beta}R_{\mu \alpha} + \tfrac{1}{2} K_{\mu}{}^{\alpha} \nabla_{\beta}\nabla^{\beta}R_{\nu \alpha}  
\nonumber\\
&&+ \tfrac{1}{2} \nabla_{\beta}\nabla^{\beta}\nabla_{\alpha}\nabla^{\alpha}K_{\mu \nu} 
-  \tfrac{1}{2} \nabla_{\beta}\nabla^{\beta}\nabla_{\mu}\nabla_{\alpha}K_{\nu}{}^{\alpha} -  \tfrac{1}{2} \nabla_{\beta}\nabla^{\beta}\nabla_{\nu}\nabla_{\alpha}K_{\mu}{}^{\alpha}  
\nonumber\\
&&-  g_{\mu \nu} R^{\alpha \beta} \nabla_{\beta}\nabla_{\gamma}K_{\alpha}{}^{\gamma}  
+ \nabla_{\alpha}R_{\nu \beta} \nabla^{\beta}K_{\mu}{}^{\alpha} + \nabla_{\alpha}R_{\mu \beta} \nabla^{\beta}K_{\nu}{}^{\alpha} 
+ \tfrac{2}{3} g_{\mu \nu} R^{\alpha \beta} \nabla_{\gamma}\nabla^{\gamma}K_{\alpha \beta} 
\nonumber\\
&& - 2 R_{\mu \alpha \nu \beta} \nabla_{\gamma}\nabla^{\gamma}K^{\alpha \beta} + \tfrac{1}{6} g_{\mu \nu} K^{\alpha \beta} \nabla_{\gamma}\nabla^{\gamma}R_{\alpha \beta} -  K^{\alpha \beta} \nabla_{\gamma}\nabla^{\gamma}R_{\mu \alpha \nu \beta}  
\nonumber\\
&&+ \tfrac{1}{6} g_{\mu \nu} \nabla_{\gamma}\nabla^{\gamma}\nabla_{\beta}\nabla_{\alpha}K^{\alpha \beta} 
+ \tfrac{1}{3} g_{\mu \nu} \nabla_{\gamma}R_{\alpha \beta} \nabla^{\gamma}K^{\alpha \beta} - 2 \nabla_{\gamma}R_{\mu \alpha \nu \beta} \nabla^{\gamma}K^{\alpha \beta} 
\nonumber\\
&&+ R_{\mu \beta \nu \gamma} \nabla^{\gamma}\nabla_{\alpha}K^{\alpha \beta} + R_{\mu \gamma \nu \beta} \nabla^{\gamma}\nabla_{\alpha}K^{\alpha \beta} -  \nabla_{\beta}R_{\nu \alpha} \nabla_{\mu}K^{\alpha \beta} 
+ \tfrac{1}{6} \nabla^{\alpha}R \nabla_{\mu}K_{\nu \alpha} 
\nonumber\\
&&-  \tfrac{1}{3} R \nabla_{\mu}\nabla_{\alpha}K_{\nu}{}^{\alpha} -  \tfrac{1}{2} R_{\nu}{}^{\alpha} \nabla_{\mu}\nabla_{\beta}K_{\alpha}{}^{\beta} + R^{\alpha \beta} \nabla_{\mu}\nabla_{\beta}K_{\nu \alpha} -  \nabla_{\beta}R_{\mu \alpha} \nabla_{\nu}K^{\alpha \beta}  
\nonumber\\
&&+ \tfrac{1}{3} \nabla_{\mu}R_{\alpha \beta} \nabla_{\nu}K^{\alpha \beta} 
+ \tfrac{1}{6} \nabla^{\alpha}R \nabla_{\nu}K_{\mu \alpha} + \tfrac{1}{3} \nabla_{\mu}K^{\alpha \beta} \nabla_{\nu}R_{\alpha \beta} -  \tfrac{1}{3} R \nabla_{\nu}\nabla_{\alpha}K_{\mu}{}^{\alpha}  
\nonumber\\
&&-  \tfrac{1}{2} R_{\mu}{}^{\alpha} \nabla_{\nu}\nabla_{\beta}K_{\alpha}{}^{\beta} + R^{\alpha \beta} \nabla_{\nu}\nabla_{\beta}K_{\mu \alpha} -  \tfrac{2}{3} R^{\alpha \beta} \nabla_{\nu}\nabla_{\mu}K_{\alpha \beta} 
+ \tfrac{1}{3} K^{\alpha \beta} \nabla_{\nu}\nabla_{\mu}R_{\alpha \beta} 
\nonumber\\
&& + \tfrac{1}{3} \nabla_{\nu}\nabla_{\mu}\nabla_{\beta}\nabla_{\alpha}K^{\alpha \beta}.
\label{AP54}
\end{eqnarray}
%

Comprising 59 terms, we note that when evaluated within a flat background, (\ref{AP54}) reduces to 
\begin{eqnarray}
\delta W_{\mu\nu}^{}(K_{\mu\nu})&=&\tfrac{1}{2} \nabla_{\beta}\nabla^{\beta}\nabla_{\alpha}\nabla^{\alpha}K_{\mu \nu} -  \tfrac{1}{2} \nabla_{\beta}\nabla^{\beta}\nabla_{\mu}\nabla_{\alpha}K_{\nu}{}^{\alpha} -  \tfrac{1}{2} \nabla_{\beta}\nabla^{\beta}\nabla_{\nu}\nabla_{\alpha}K_{\mu}{}^{\alpha}  
\nonumber\\
&&+ \tfrac{1}{6} g_{\mu \nu} \nabla_{\gamma}\nabla^{\gamma}\nabla_{\beta}\nabla_{\alpha}K^{\alpha \beta}+ \tfrac{1}{3} \nabla_{\nu}\nabla_{\mu}\nabla_{\beta}\nabla_{\alpha}K^{\alpha \beta}.
\end{eqnarray}
Hence, \eqref{AP54} reduces to the result of \eqref{AP24}, namely the simplified flat fluctuation equations without any imposition of a gauge condition - such a result provides an affirmative check on our calculation thus far.


%%%%%%%%%%%%%%%%%%%%%%%%%%%%%%%%%%%%%%%%%%%%
\subsection{$\delta W_{\mu\nu}$ in a Conformal to Flat Minkowski Background}
\label{ss:fluctuations_around_conformal_flat_cgauge}
%%%%%%%%%%%%%%%%%%%%%%%%%%%%%%%%%%%%%%%%%%%%

%%%%%%%%%%%%%%%%%%%%%%%%%%%%%%%%%%%%%%%%%%%%
\subsubsection{Implementing the Conformal Gauge Condition}
\label{sss:implementing_cgauge}
%%%%%%%%%%%%%%%%%%%%%%%%%%%%%%%%%%%%%%%%%%%%

With \eqref{AP54} now being ready for insertion of the conformal gauge, we evaluate (\ref{AP54}) in the conformal to flat background given in (\ref{AP6}) recalling that we take $\Omega(x)$ to be a completely general and arbitrary spacetime function. In the  $g_{\mu\nu}=\Omega^2(x)\eta_{\mu\nu}$ background  the gauge condition $\nabla_{\nu}K^{\mu\nu}=\frac{1}{2}K^{\mu\nu}g^{\alpha\beta}\partial_{\nu}g_{\alpha\beta}$ takes the form
%
\begin{eqnarray}
\nabla_{\nu}K^{\mu\nu}-\frac{1}{2}K^{\mu\nu}\Omega^{-2}\eta^{\alpha\beta}\eta_{\alpha\beta}\partial_{\nu}\Omega^2&=&\nabla_{\nu}K^{\mu\nu}-4\Omega^{-1}K^{\mu\nu}\partial_{\nu}\Omega 
\nonumber\\
&=&\partial_{\nu}K^{\mu\nu}+6\Omega^{-1}K^{\mu\nu}\partial_{\nu}\Omega-4\Omega^{-1}K^{\mu\nu}\partial_{\nu}\Omega
\nonumber\\
&=&\partial_{\nu}K^{\mu\nu}+2\Omega^{-1}K^{\mu\nu}\partial_{\nu}\Omega=\Omega^{-2}\partial_{\nu}(\Omega^{2}K^{\mu\nu}) 
\nonumber\\
&=&0. 
\label{AP55}
\end{eqnarray}
%
We can factor out a contribution of $\Omega^2(x)$ from the fluctuation by setting $K^{\mu\nu}=\Omega^{-2}(x)k^{\mu\nu}$ and $K_{\mu\nu}=\Omega^{2}(x)k_{\mu\nu}$. For clarification, here indices on $k^{\mu\nu}$ and $k_{\mu\nu}=\eta_{\mu\alpha}\eta_{\nu\beta}k^{\alpha\beta}$ are raised and lowered with $\eta_{\mu\nu}$. As a result, (\ref{AP55}) can thus be expressed as the familiar transverse form $\partial_{\nu}k^{\mu\nu}=0$, noting that the conformal gauge condition is such that there is no longer a dependence upon the conformal factor. 
%%%%%%%%%%%%%%%%%%%%%%%%%%%%%%%
	\footnote{Within (\ref{AP55}) we note that we have taken $\Omega(x)$ to be a general function of the coordinates so that we additionally encompass the special case of Robertson-Walker geometries with arbitrary spatial curvature $k$. As given in Appendix \ref{ab:cosmologies}, $\Omega(x)$ will only depend on the time coordinate $t$ for $k=0$, while for $k\ne 0$, $\Omega(x)$ will depend on both $t$ and the radial coordinate $r$.}
%%%%%%%%%%%%%%%%%%%%%%%%%%%%%%%
Before we alas evaluate the (\ref{AP54}) within a conformal Minkowski background in the conformal gauge given in (\ref{AP55}), we observe that the gauge condition 
%
\begin{eqnarray}
\nabla_{\nu}K^{\mu\nu}=4\Omega^{-1}K^{\mu\nu}\partial_{\nu}\Omega
\label{AP56}
\end{eqnarray}
% 
in fact possesses the same form of a covariant gauge condition for a background metric $\Omega^2(x) g_{\mu\nu}$ with any $g_{\mu\nu}$. Hence our treatment is fully covariant. It follows that when (\ref{AP56}) is imposed in a conformal to flat but not necessarily Minkowski background (e.g. polar coordinates of the form $ds^2=dt^2-dr^2-r^2d\theta^2-r^2\sin^2\theta d\phi^2$) then (\ref{AP54}) will take the form
%
\begin{eqnarray}
&&\delta W_{\mu\nu}=\frac{1}{2}\Omega^{-4}\tilde{\nabla}_{\beta}\tilde{\nabla}^{\beta}\tilde{\nabla}_{\alpha}\tilde{\nabla}^{\alpha}K_{\mu \nu}-  4\Omega^{-5} \tilde{\nabla}_{\beta}\tilde{\nabla}_{\alpha}K_{\mu \nu} \tilde{\nabla}^{\beta}\tilde{\nabla}^{\alpha}\Omega 
\nonumber\\
&&- 2\Omega^{-5}  \tilde{\nabla}_{\alpha}\tilde{\nabla}^{\alpha}\Omega \tilde{\nabla}_{\beta}\tilde{\nabla}^{\beta}K_{\mu \nu}-  4 \Omega^{-5}\tilde{\nabla}^{\alpha}\Omega \tilde{\nabla}_{\beta}\tilde{\nabla}^{\beta}\tilde{\nabla}_{\alpha}K_{\mu \nu}  
\nonumber\\
&&-  \Omega^{-5}K_{\mu \nu} \tilde{\nabla}_{\beta}\tilde{\nabla}^{\beta}\tilde{\nabla}_{\alpha}\tilde{\nabla}^{\alpha}\Omega -  4\Omega^{-5} \tilde{\nabla}_{\alpha}K_{\mu \nu} \tilde{\nabla}_{\beta}\tilde{\nabla}^{\beta}\tilde{\nabla}^{\alpha}\Omega + 6\Omega^{-6} \tilde{\nabla}_{\alpha}\Omega \tilde{\nabla}^{\alpha}\Omega \tilde{\nabla}_{\beta}\tilde{\nabla}^{\beta}K_{\mu \nu}  
\nonumber\\
&&+ 12\Omega^{-6} \tilde{\nabla}^{\alpha}\Omega \tilde{\nabla}_{\beta}\tilde{\nabla}_{\alpha}K_{\mu \nu} \tilde{\nabla}^{\beta}\Omega
+ 3\Omega^{-6} K_{\mu \nu} \tilde{\nabla}_{\alpha}\tilde{\nabla}^{\alpha}\Omega \tilde{\nabla}_{\beta}\tilde{\nabla}^{\beta}\Omega  
\nonumber\\
&&+ 12 \Omega^{-6}\tilde{\nabla}_{\alpha}K_{\mu \nu} \tilde{\nabla}^{\alpha}\Omega \tilde{\nabla}_{\beta}\tilde{\nabla}^{\beta}\Omega 
+ 24\Omega^{-6}  \tilde{\nabla}^{\alpha}\Omega \tilde{\nabla}_{\beta}K_{\mu \nu} \tilde{\nabla}^{\beta}\tilde{\nabla}_{\alpha}\Omega 
\nonumber\\
&&
+ 6\Omega^{-6} K_{\mu \nu} \tilde{\nabla}_{\beta}\tilde{\nabla}_{\alpha}\Omega \tilde{\nabla}^{\beta}\tilde{\nabla}^{\alpha}\Omega 
+ 12\Omega^{-6} K_{\mu \nu} \tilde{\nabla}^{\alpha}\Omega \tilde{\nabla}_{\beta}\tilde{\nabla}^{\beta}\tilde{\nabla}_{\alpha}\Omega  
\nonumber\\
&&-  24 \Omega^{-7}K_{\mu \nu} \tilde{\nabla}_{\alpha}\Omega \tilde{\nabla}^{\alpha}\Omega \tilde{\nabla}_{\beta}\tilde{\nabla}^{\beta}\Omega 
-  48\Omega^{-7}  \tilde{\nabla}_{\alpha}\Omega \tilde{\nabla}^{\alpha}\Omega \tilde{\nabla}_{\beta}K_{\mu \nu} \tilde{\nabla}^{\beta}\Omega 
\nonumber\\
&&
-  48\Omega^{-7}  K_{\mu \nu} \tilde{\nabla}^{\alpha}\Omega \tilde{\nabla}_{\beta}\tilde{\nabla}_{\alpha}\Omega \tilde{\nabla}^{\beta}\Omega 
+ 60\Omega^{-8} K_{\mu \nu} \tilde{\nabla}_{\alpha}\Omega \tilde{\nabla}^{\alpha}\Omega \tilde{\nabla}_{\beta}\Omega \tilde{\nabla}^{\beta}\Omega.
\label{AP57}
\end{eqnarray}
%
Here we have introduced $\tilde{\nabla}_{\alpha}$ to denote the covariant derivative with respect to the flat (but not necessarily Minkowski) background $g_{\mu\nu}$ such that $\tilde{\nabla}^{\alpha}$ is equal to $g^{\alpha\beta}\tilde{\nabla}_{\alpha}$. Observing \eqref{AP57}, we remarkably find that the 17 terms can be factored into one single compact expression
%
\begin{eqnarray}
\delta W_{\mu\nu}(K_{\mu\nu})=\frac{1}{2}\Omega^{-2}\tilde{\nabla}_{\alpha}\tilde{\nabla}^{\alpha}\tilde{\nabla}_{\beta}\tilde{\nabla}^{\beta}(\Omega^{-2}K_{\mu\nu})
=\frac{1}{2}\Omega^{-2}\tilde{\nabla}_{\alpha}\tilde{\nabla}^{\alpha}\tilde{\nabla}_{\beta}\tilde{\nabla}^{\beta}k_{\mu\nu},
\label{AP58}
\end{eqnarray}
%
where we we again note that $k_{\mu\nu}=\Omega^{-2}(x)K_{\mu\nu}$. Thus (\ref{AP58}) embodies the representation of fluctuations around a geometry that is conformal to an arbitrary flat background metric, within the  $\nabla_{\nu}K^{\mu\nu}=4\Omega^{-1}K^{\mu\nu}\partial_{\nu}\Omega$ gauge. Upon setting $\Omega(x)=1$,  (\ref{AP58}) reconciles our result within Sec. \ref{ss:fluctuations_around_flat_in_the_tranverse_gauge} as a representation of the fluctuation equations around a flat background geometry in the transverse gauge $\nabla_{\nu}K^{\mu\nu}=0$ as 
%
\begin{eqnarray}
\delta W_{\mu\nu}(K_{\mu\nu})=\frac{1}{2}\tilde{\nabla}_{\alpha}\tilde{\nabla}^{\alpha}\tilde{\nabla}_{\beta}\tilde{\nabla}^{\beta}K_{\mu\nu}.
\label{AP59}
\end{eqnarray}
%

%%%%%%%%%%%%%%%%%%%%%%%%%%%%%%%%%%%%%%%%%%%%
\subsubsection{Further Fluctuation Reduction}
\label{sss:obtaining_fluctuation_eqns_in_cgauge}
%%%%%%%%%%%%%%%%%%%%%%%%%%%%%%%%%%%%%%%%%%%%

Although the forms of (\ref{AP58}) and (\ref{AP59}) are exceedingly simple, from a practical perspective, these are not of straightforward use since they involve covariant derivatives that mix the various components of $K_{\mu\nu}$. To fix this issue, we note that (\ref{AP58}) and (\ref{AP59}) also apply in the gauge given in (\ref{AP55}). Hence within the conformal gauge, fluctuations around a conformal to Minkowski geometry take the partial derivative form
%
\begin{eqnarray}
&&\delta W_{\mu\nu}(K_{\mu\nu})=\frac{1}{2}\Omega^{-4}\partial_{\beta}\partial^{\beta}\partial_{\alpha}\partial^{\alpha}K_{\mu \nu}-  4\Omega^{-5} \partial_{\beta}\partial_{\alpha}K_{\mu \nu} \partial^{\beta}\partial^{\alpha}\Omega 
\nonumber\\
&&- 2\Omega^{-5}  \partial_{\alpha}\partial^{\alpha}\Omega \partial_{\beta}\partial^{\beta}K_{\mu \nu}-  4 \Omega^{-5}\partial^{\alpha}\Omega \partial_{\beta}\partial^{\beta}\partial_{\alpha}K_{\mu \nu}  
-  \Omega^{-5}K_{\mu \nu} \partial_{\beta}\partial^{\beta}\partial_{\alpha}\partial^{\alpha}\Omega 
\nonumber\\
&& -  4\Omega^{-5} \partial_{\alpha}K_{\mu \nu} \partial_{\beta}\partial^{\beta}\partial^{\alpha}\Omega 
+ 6\Omega^{-6} \partial_{\alpha}\Omega \partial^{\alpha}\Omega \partial_{\beta}\partial^{\beta}K_{\mu \nu} + 12\Omega^{-6} \partial^{\alpha}\Omega \partial_{\beta}\partial_{\alpha}K_{\mu \nu} \partial^{\beta}\Omega 
\nonumber\\
&&+ 3\Omega^{-6} K_{\mu \nu} \partial_{\alpha}\partial^{\alpha}\Omega \partial_{\beta}\partial^{\beta}\Omega + 12 \Omega^{-6}\partial_{\alpha}K_{\mu \nu} \partial^{\alpha}\Omega \partial_{\beta}\partial^{\beta}\Omega+ 24\Omega^{-6}  \partial^{\alpha}\Omega \partial_{\beta}K_{\mu \nu} \partial^{\beta}\partial_{\alpha}\Omega  
\nonumber\\
&&+ 6\Omega^{-6} K_{\mu \nu} \partial_{\beta}\partial_{\alpha}\Omega \partial^{\beta}\partial^{\alpha}\Omega 
+ 12\Omega^{-6} K_{\mu \nu} \partial^{\alpha}\Omega \partial_{\beta}\partial^{\beta}\partial_{\alpha}\Omega -  24 \Omega^{-7}K_{\mu \nu} \partial_{\alpha}\Omega \partial^{\alpha}\Omega \partial_{\beta}\partial^{\beta}\Omega  
\nonumber\\
&&-  48\Omega^{-7}  \partial_{\alpha}\Omega \partial^{\alpha}\Omega \partial_{\beta}K_{\mu \nu} \partial^{\beta}\Omega-  48\Omega^{-7}  K_{\mu \nu} \partial^{\alpha}\Omega \partial_{\beta}\partial_{\alpha}\Omega \partial^{\beta}\Omega 
\nonumber\\
&&
+ 60\Omega^{-8} K_{\mu \nu} \partial_{\alpha}\Omega \partial^{\alpha}\Omega \partial_{\beta}\Omega \partial^{\beta}\Omega.
\label{AP60}
\end{eqnarray}
%
Inspection of \eqref{AP60} reveals that it remarkably simplifies to
%
\begin{eqnarray}
\delta W_{\mu\nu}(K_{\mu\nu})=\frac{1}{2}\Omega^{-2}\eta^{\sigma\rho}\eta^{\alpha\beta}\partial_{\sigma}\partial_{\rho} \partial_{\alpha}\partial_{\beta}(\Omega^{-2}K_{\mu\nu})
=\frac{1}{2}\Omega^{-2}\eta^{\sigma\rho}\eta^{\alpha\beta}\partial_{\sigma}\partial_{\rho} \partial_{\alpha}\partial_{\beta}k_{\mu\nu}.
\label{AP61}
\end{eqnarray}
%
As written, (\ref{AP61}) may be recognized as being of the exact form given within Sec. \ref{ss:conformal_invariance} based on grounds on conformal invariance and transformation properties of the Bach tensor. Regardless of the $g_{\mu\nu}=\Omega^2(x)\eta_{\mu\nu}$ background not being flat, in (\ref{AP60}) and (\ref{AP61}) all derivatives are flat Minkowski (i.e. associated with the metric $ds^2=-\eta_{\alpha\beta}dx^{\alpha}dx^{\beta}=dt^2-dx^2-dy^2-dz^2$). In terms of these partial derivatives, we observe a significant feature in that (\ref{AP60}) and (\ref{AP61}) are diagonal in the $(\mu,\nu)$ indices (i.e. there is no mixing of the components of $k_{\mu\nu}$ from the different operator).  Consequently, the reductive significance of conformal symmetry is exemplified in our starting point with a the 62 term $\delta W_{\mu\nu}(h_{\mu\nu})$ given in (\ref{AP43}) and arriving at the single term (\ref{AP61}).

\subsection{Calculation Summary}
\label{ss:summary_cgauge}

As we have covered numerous steps in the derivation of the fluctuations within conformal gravity, we provide a summary overview of the procedure given. We began with a general $W_{\mu\nu}$ in the form of in (\ref{AP42}) and then perturbed $W_{\mu\nu}$ to first order around a general background  with metric $g_{\mu\nu}$ with a perturbed metric of the form  $g_{\mu\nu}+\delta g_{\mu\nu}=g_{\mu\nu}+h_{\mu\nu}$. With the identity 
%
\begin{eqnarray}
\delta R^{\lambda}_{\phantom{\lambda}\mu\nu\kappa}&=&
\partial \delta\Gamma^{\lambda}_{\mu\nu}/\partial x^{\kappa}
+\Gamma^{\lambda}_{\kappa\sigma}\delta\Gamma^{\sigma}_{\mu\nu}
-\Gamma^{\sigma}_{\mu\kappa}\delta\Gamma^{\lambda}_{\nu\sigma}
-\partial \delta\Gamma^{\lambda}_{\mu\kappa}/\partial x^{\nu}
-\Gamma^{\lambda}_{\nu\sigma}\delta\Gamma^{\sigma}_{\mu\kappa}
+\Gamma^{\sigma}_{\mu\nu}\delta\Gamma^{\lambda}_{\kappa\sigma}
\nonumber\\
&=&
\nabla_{\kappa}\delta\Gamma^{\lambda}_{\mu\nu}
-\nabla_{\nu}\delta\Gamma^{\lambda}_{\mu\kappa},
\end{eqnarray}
%
where $\delta\Gamma^{\lambda}_{\mu\nu}=(1/2)g^{\lambda \rho}[\nabla_{\nu}\delta g_{\rho\mu}+\nabla_{\mu}\delta g_{\rho\nu}-\nabla_{\rho}\delta g_{\mu\nu}]$, we then evaluate $\delta W_{\mu\nu}$ to lowest order in $\delta g_{\mu\nu}$ to obtain (\ref{AP43}). 

To address trace simplifications, in (\ref{AP43}) we set  $h_{\mu\nu}=K_{\mu\nu}+(1/4)g_{\mu\nu}h$ where $h=g^{\mu\nu}h_{\mu\nu}$ and $g^{\mu\nu}K_{\mu\nu}=0$, from which it follows that $\delta W_{\mu\nu}$ may be expressed as two contributions viz. $\delta W_{\mu\nu}(K_{\mu\nu})$ in (\ref{AP44}) and $\delta W_{\mu\nu}(h)$ in (\ref{AP45}). Commuting covariant derivatives we observed and confirmed that $\delta W_{\mu\nu}(h)=-(1/4)W_{\mu\nu}h$ as shown in (\ref{AP50}). Consequently, this has established that $\delta W_{\mu\nu}(h)$ will vanish if the background $W_{\mu\nu}$ vanishes (a result that applied to the backgrounds of Robertson-Walker and de Sitter cosmologies).

For geometries in which the background $W_{\mu\nu}$ vanishes, $\delta W_{\mu\nu}$ reduces to $\delta W_{\mu\nu}(K_{\mu\nu})$, with $\delta W_{\mu\nu}$ to then only be dependent on the traceless fluctuation $K_{\mu\nu}$ as given in (\ref{AP44}). With further commutation of covariant derivatives we then expressed (\ref{AP44}) in the form given in (\ref{AP54}). Now in a form ready for conformal gauge implementation, we apply condition (\ref{AP23}) within a conformal flat background to yield $\delta W_{\mu\nu}(K_{\mu\nu})$ given as (\ref{AP61}), the main result of this section.


%%%%%%%%%%%%%%%%%%%%%%%%%%%%%%%%%%%%%%%%%%%%
\section{Imposing Specific Gauges within $\delta G_{\mu\nu}$}
\label{s:compact_expressions_ein}
%%%%%%%%%%%%%%%%%%%%%%%%%%%%%%%%%%%%%%%%%%%%

With  background plus fluctuation metric of the form $ds^2=\Omega^2(x)[dt^2-\delta_{ij}dx^idx^j]-\Omega^2(x)f_{\mu\nu}dx^{\mu}dx^{\nu}$, the fluctuation in the Einstein tensor is given by 
%
\begin{eqnarray}
&&\delta G_{\mu\nu}=- \tfrac{1}{2}\tilde{\nabla}_{\alpha}\tilde{\nabla}_{\mu}f_{\nu}{}^{\alpha} -  \tfrac{1}{2} \tilde{\nabla}_{\alpha}\tilde{\nabla}_{\nu}f_{\mu}{}^{\alpha} -  \eta^{\alpha \beta} \eta_{\mu \nu} \Omega^{-1}\tilde{\nabla}_{\alpha}f \tilde{\nabla}_{\beta}\Omega + \eta^{\alpha \beta} \Omega^{-1} \tilde{\nabla}_{\alpha}f_{\mu \nu} \tilde{\nabla}_{\beta}\Omega
\nonumber\\
&& + \eta^{\beta \alpha} f_{\mu \nu} \Omega^{-2}\tilde{\nabla}_{\alpha}\Omega \tilde{\nabla}_{\beta}\Omega 
-  \tfrac{1}{2} \eta^{\alpha \beta} \eta_{\mu \nu} \tilde{\nabla}_{\beta}\tilde{\nabla}_{\alpha}f + \tfrac{1}{2} \eta_{\mu \nu}\tilde{\nabla}_{\beta}\tilde{\nabla}_{\alpha}f^{\alpha \beta} + \tfrac{1}{2} \eta^{\alpha \beta} \tilde{\nabla}_{\beta}\tilde{\nabla}_{\alpha}f_{\mu \nu}
\nonumber\\
&& + 2 \eta_{\mu \nu} f^{\alpha \beta} \Omega^{-1} \tilde{\nabla}_{\beta}\tilde{\nabla}_{\alpha}\Omega - 2 \eta^{\alpha \beta} f_{\mu \nu} \Omega^{-1} \tilde{\nabla}_{\beta}\tilde{\nabla}_{\alpha}\Omega 
+ 2 \eta^{\alpha \gamma} \eta_{\mu \nu} \Omega^{-1} \tilde{\nabla}_{\beta}f_{\alpha}{}^{\beta} \tilde{\nabla}_{\gamma}\Omega 
\nonumber\\
&&-  \eta^{\alpha \gamma} \eta^{\beta \kappa} \eta_{\mu \nu} f_{\alpha \beta} \Omega^{-2} \tilde{\nabla}_{\gamma}\Omega\tilde{\nabla}_{\kappa}\Omega -  \eta^{\alpha \beta} \Omega^{-1}\tilde{\nabla}_{\beta}\Omega \tilde{\nabla}_{\mu}f_{\nu \alpha} -  \eta^{\alpha \beta} \Omega^{-1}\tilde{\nabla}_{\beta}\Omega \tilde{\nabla}_{\nu}f_{\mu \alpha} 
\nonumber\\
&&+ \tfrac{1}{2} \tilde{\nabla}_{\nu}\tilde{\nabla}_{\mu}f,
\label{AP76}
\end{eqnarray}
%
where $\Omega(x)$ is an arbitrary function of $x_{\mu}$. In our exploration of various gauges below, we make use of the trace free contribution of the conformally factored metric perturbation $k_{\mu\nu}=f_{\mu\nu}-(1/4)\eta^{\alpha\beta}f_{\alpha\beta}$ where the tracelessness is defined with respect to the background $\eta^{\mu\nu}k_{\mu\nu}=0$. With the substitution of $k_{\mu\nu}$, we may express (\ref{AP76}) as 
%
\begin{eqnarray}
\delta G_{\mu\nu}&=&- \tfrac{1}{4} \eta^{\alpha \beta} \eta_{\mu \nu} \Omega^{-1} \partial_{\alpha}\Omega \partial_{\beta}f + \eta^{\alpha \beta} \Omega^{-1} \partial_{\alpha}k_{\mu \nu} \partial_{\beta}\Omega + \eta^{\beta \alpha} k_{\mu \nu} \Omega^{-2} \partial_{\alpha}\Omega \partial_{\beta}\Omega  
\nonumber\\
&&+ \tfrac{1}{2} \eta^{\alpha \beta} \partial_{\beta}\partial_{\alpha}k_{\mu \nu} -  \tfrac{1}{4} \eta^{\alpha \beta} \eta_{\mu \nu} \partial_{\beta}\partial_{\alpha}f 
- 2 \eta^{\alpha \beta} k_{\mu \nu} \Omega^{-1} \partial_{\beta}\partial_{\alpha}\Omega -  \tfrac{1}{2} \eta^{\alpha \beta} \partial_{\beta}\partial_{\mu}k_{\nu \alpha}  
\nonumber\\
&&-  \tfrac{1}{2} \eta^{\alpha \beta} \partial_{\beta}\partial_{\nu}k_{\mu \alpha} + 2 \eta^{\alpha \beta} \eta^{\gamma \kappa} \eta_{\mu \nu} \Omega^{-1} \partial_{\beta}\Omega \partial_{\kappa}k_{\alpha \gamma} 
-  \eta^{\alpha \gamma} \eta^{\beta \kappa} \eta_{\mu \nu} k_{\alpha \beta} \Omega^{-2} \partial_{\gamma}\Omega \partial_{\kappa}\Omega 
\nonumber\\
&& + \tfrac{1}{2} \eta^{\alpha \beta} \eta^{\gamma \kappa} \eta_{\mu \nu} \partial_{\kappa}\partial_{\beta}k_{\alpha \gamma} + 2 \eta^{\alpha \beta} \eta^{\gamma \kappa} \eta_{\mu \nu} k_{\alpha \gamma} \Omega^{-1} \partial_{\kappa}\partial_{\beta}\Omega -  \eta^{\alpha \beta} \Omega^{-1} \partial_{\beta}\Omega \partial_{\mu}k_{\nu \alpha} 
\nonumber\\
&&-  \eta^{\alpha \beta} \Omega^{-1} \partial_{\beta}\Omega \partial_{\nu}k_{\mu \alpha} -  \tfrac{1}{4} \Omega^{-1} \partial_{\mu}\Omega \partial_{\nu}f -  \tfrac{1}{4} \Omega^{-1} \partial_{\mu}f \partial_{\nu}\Omega + \tfrac{1}{4} \partial_{\nu}\partial_{\mu}f,
\label{D1}
\end{eqnarray}
%
with $f$ denoting $\eta^{\alpha\beta}f_{\alpha\beta}$. As our goal is to reduce the \eqref{D1} into as compact form as possible, we explore a most general set of possible gauges with variable coefficients in the form of the gauge condition
%
\begin{eqnarray}
\eta^{\alpha\beta}\partial_{\alpha}k_{\beta\nu} = \Omega^{-1} J \eta^{\alpha\beta}k_{\nu\alpha}\partial_\beta \Omega + P \partial_\nu f+ R \Omega^{-1} f\partial_\nu \Omega.
\label{D2}
\end{eqnarray}
%
Here $J$, $P$, and $R$ represent the constant coefficients of which we will vary. On taking  $J = -2$, $P = 1/2$, $R = 0$ the gauge condition becomes
%
\begin{eqnarray}
\eta^{\alpha\beta}\partial_{\alpha}k_{\beta\nu} = -2 \Omega^{-1}  \eta^{\alpha\beta}k_{\nu\alpha}\partial_\beta \Omega + \tfrac{1}{2} \partial_\nu f.
\label{D3}
\end{eqnarray}
%
If we elect to take an $\Omega$  that is strictly dependent upon conformal time $\tau$, then evaluation of $\delta G_{\mu\nu}$ leads to
%
\begin{eqnarray}
\delta G_{00}&=&
(3 \Omega^{-2} \dot{\Omega}^2 -  \Omega^{-1} \ddot{\Omega} + \tfrac{1}{2} \eta^{\mu \nu} \partial_{\mu} \partial_{\nu} -  \Omega^{-1} \dot{\Omega} \partial_{0}) k_{00} - \tfrac{1}{4} ( \Omega^{-1} \dot{\Omega} \partial_{0} + \partial_{0} \partial_{0}) f,
\nonumber\\
\delta G_{0i}&=&
(\Omega^{-1} \ddot{\Omega} + \tfrac{1}{2} \eta^{\mu \nu} \partial_{\mu} \partial_{\nu} -  \Omega^{-1} \dot{\Omega} \partial_{0}) k_{0i} - \tfrac{1}{4}  (\Omega^{-1} \dot{\Omega} \partial_{i} +\partial_{i} \partial_{0}) f,
\nonumber\\
\delta G_{ij}&=&
\delta_{ij}(-2 \Omega^{-2} \dot{\Omega}^2 + \Omega^{-1} \ddot{\Omega}) k_{00} + (- \Omega^{-2} \dot{\Omega}^2 + 2 \Omega^{-1} \ddot{\Omega} + \tfrac{1}{2} \eta^{\mu \nu} \partial_{\mu} \partial_{\nu}  
\nonumber\\
&&-  \Omega^{-1} \dot{\Omega} \partial_{0}) k_{ij}, 
- \tfrac{1}{4} (\delta_{ij} \Omega^{-1} \dot{\Omega} \partial_{0} + \partial_{i} \partial_{j}) f,
\nonumber\\
\eta^{\mu\nu}\delta G_{\mu\nu}&=&(-10\Omega^{-2} \dot{\Omega}^2 +6  \Omega^{-1} \ddot{\Omega})k_{00}- \tfrac{1}{4}  (2\Omega^{-1} \dot{\Omega} \partial_{0} +\eta^{\mu\nu}\partial_{\mu}\partial_{\nu}) f,
\label{D4}
\end{eqnarray}
%
where as usual a dot denotes a derivative with respect to $\tau$, and where $\partial_0$ denotes $\partial_{\tau}$. We see that $\delta G_{\mu\nu}$ is now expressed in a form that nearly decouples each tensor component, with it being the $k_{00}$ and $f$ dependence  $\delta G_{\mu\nu}$ that prevents it from doing so. However, one can solve the equations exactly once $\delta T_{\mu\nu}$ is specified, as one can use the $\delta G_{00}$ and $\eta^{\mu\nu}\delta G_{\mu\nu}$ equations to can determine $k_{00}$ and $f$, and then from $\delta G_{\mu\nu}$ one can determine all remaining components of $k_{\mu\nu}$. 


To illustrate an example, if the background is the de Sitter geometry, i.e. where we set $\Omega(\tau)=1/H\tau$ with $H$ constant, then (\ref{D4}) takes the form
%
\begin{eqnarray}
\delta G_{00}&=&
(\tau^{-2} + \tfrac{1}{2} \eta^{\mu \nu} \partial_{\mu} \partial_{\nu} + \tau^{-1} \partial_{0}) k_{00} + \tfrac{1}{4} (\tau^{-1} \partial_{0} -   \partial_{0} \partial_{0}) f,
\nonumber\\
\delta G_{0i}&=&
(2 \tau^{-2} + \tfrac{1}{2} \eta^{\mu \nu} \partial_{\mu} \partial_{\nu} + \tau^{-1} \partial_{0}) k_{0i} + \tfrac{1}{4} ( \tau^{-1} \partial_{i} -  \partial_{i} \partial_{0}) f,
\nonumber\\
\delta G_{ij}&=&
(3 \tau^{-2} + \tfrac{1}{2} \eta^{\mu \nu} \partial_{\mu} \partial_{\nu} + \tau^{-1} \partial_{0}) k_{ij} + \tfrac{1}{4} (\delta_{ij} \tau^{-1} \partial_{0} -   \partial_{i} \partial_{j}) f,
\nonumber\\
\eta^{\mu\nu}\delta G_{\mu\nu}&=&2\tau^{-2}k_{00}+ \tfrac{1}{4}  (2\tau^{-1}\partial_{0} -\eta^{\mu\nu}\partial_{\mu}\partial_{\nu}) f.
\label{D5}
\end{eqnarray}
%
These equations form a compact set of reduced terms, a form which was possible by the explicit incorporation of the conformal factor $\Omega(\tau)$ into both the background and the fluctuation (which also holds for (\ref{D4}). Hence, in this gauge the $\delta G_{\mu\nu}=-8\pi G \delta T_{\mu\nu}$ fluctuation equations facilitate a readily integrable solution.

Another convenient decomposition of $\delta G_{\mu\nu}$ corresponds to the gauge choice $J=-4$, $R=2P-3/2$, with $P$ arbitrary. In the de Sitter background, the fluctuations take the form
%
\begin{eqnarray}
\delta G_{00}&=&(-2 \tau^{-2}
+ \tfrac{1}{2} \eta^{\mu \nu} \partial_{\mu} \partial_{\nu}
+ 3 \tau^{-1} \partial_{0}) k_{00}
+ \big[(\tfrac{3}{4} -  P) \tau^{-2} 
\nonumber\\
&&
+ \tfrac{1}{4}(1-2P) \eta^{\mu \nu} \partial_{\mu} \partial_{\nu}
+ P \tau^{-1} \partial_{0}
+( \tfrac{1}{4} -P)\partial^2_{0}\big]f,
\nonumber\\
\delta G_{0i}&=&\tau^{-1} \partial_{i} k_{00}
+ (\tau^{-2}
+ \tfrac{1}{2} \eta^{\mu \nu} \partial_{\mu} \partial_{\nu}
+ 2 \tau^{-1} \partial_{0}) k_{0i}
+ \big[(P- \tfrac{1}{2}) \tau^{-1} \partial_{i} 
\nonumber\\
&&
+ (\tfrac{1}{4}-P) \partial_{i} \partial_{0} \big]f,
\nonumber\\
\delta G_{ij}&=&\delta_{ij}\tau^{-2} k_{00}
+\tau^{-1} \partial_{j} k_{0i}
+ \tau^{-1} \partial_{i} k_{0j}
+ (3 \tau^{-2}
+ \tfrac{1}{2} \eta^{\mu \nu} \partial_{\mu} \partial_{\nu}
+ \tau^{-1} \partial_{0}) k_{ij}
\nonumber\\
&+& \delta_{ij}\left[(\tfrac{3}{4}-P) \tau^{-2}
+ \tfrac{1}{4}(2P-1) \eta^{\mu \nu} \partial_{\mu} \partial_{\nu}
+(P-1)\tau^{-1} \partial_{0}\right]f 
\nonumber\\
&&
+ (\tfrac{1}{4}-P) \partial_{i} \partial_{j}f.
\\
\eta^{\alpha\beta}\delta G_{\alpha\beta}&=&(P-\tfrac{3}{4})( \eta^{\alpha\beta}\partial_{\alpha}\partial_{\beta}f +4\tau^{-1}\partial_0 f-6\tau^{-2}f)=(P-\tfrac{3}{4})\tau^2 \eta^{\alpha\beta}\partial_{\alpha}\partial_{\beta}(\tau^{-2}f).
\nonumber
\label{D6}
\end{eqnarray}
%
We see conveniently that $\eta^{\mu\nu}\delta G_{\mu\nu}=\Omega^2g^{\mu\nu}\delta G_{\mu\nu}$ depends only upon the trace $f$ of the fluctuation. Hence we can foremost solve for $f$ and then proceed to the other components of the fluctuation in turn. We may further observe the component $\eta^{\alpha\beta}\delta G_{\alpha\beta}$ takes the form of the flat space free massless particle wave operator acting on $\tau^{-2} f$. Hence, we may solve $g^{\mu\nu}\delta G_{\mu\nu}=-8\pi G g^{\mu\nu}\delta T_{\mu\nu}$ via integration by the $D^{(4)}(x-y)$ Green's function obeying $\eta^{\alpha\beta}\partial_{\alpha}\partial_{\beta}D^{(4)}(x-y)=\delta^4(x-y)$. We thus can form the exact integral solution as
%
\begin{eqnarray}
f=\eta^{\mu\nu}f_{\mu\nu}=-\frac{8\pi G}{(P-\tfrac{3}{4})}\tau^2(x) \int d^4yD^{(4)}(x-y)\tau^{-2}(y)\eta^{\mu\nu}\delta T_{\mu\nu}(y),
\label{D7}
\end{eqnarray}
%
where it is implied that $\tau(x)=x^0$, $\tau(y)=y^0$.

%%%%%%%%%%%%%%%%%%%%%%%%%%%%%%%%%%%%%%%%%%%%
\section{Robertson Walker Radiation Era Conformal Gravity Solution}
\label{s:rw_radiation_conformal_gravity_sol}
%%%%%%%%%%%%%%%%%%%%%%%%%%%%%%%%%%%%%%%%%%%%

Motivated by the excellent agreement between the fits to the galactic rotation curves of 138 spiral galaxies presented in \cite{mannheim_2011,mannheim_2012, obrien_mannheim_2012} and to fits of the accelerating universe Hubble plot data presented in \cite{mannheim_2006,mannheim_2017}, we thus consider conformal gravity fluctuations in Robertson-Walker cosmologies with negative $k$.

In the conformal gravity fitting of the current era Hubble plot, we note from \cite{mannheim_1990} that the cosmological constant term is found to dominate over the perfect fluid contribution. However, in the early universe, the radiation era perfect fluid is the dominant contribution since $a(t)$ is small and $A/a^4(t)$ is large. Moreover, if the $A/a^4(t)$ radiation contribution is dominant, then since $k$ is given by $k=-\dot{a}^2-2A/a^2S_0^2$ when the $\alpha$ contribution in (\ref{A18}) is negligible, we are phenomenologically steered towards negative $k$. Thus for studying fluctuation growth in the early universe the only relevant solution for $a(t)$ is that of $a(t,\alpha=0,k<0,A>0)$. Using this solution, upon setting $k=-1/L^2$, $d^2=2AL^4/S_0^2$, and $L^2a^2(t)=(d^2+t^2)$, we obtain
%
\begin{eqnarray}
\tau=L\int_0^t \frac{dt}{(d^2+t^2)^{1/2}}=L~{\rm arc sinh}\left(\frac{t}{d}\right),\qquad t=d\sinh p,
\label{B1}
\end{eqnarray}
%
where $p=\tau/L$. We have seen that from (\ref{AP29}), fluctuations around a flat background grow linearly in the relevant time variable, which according to the $k<0$ (\ref{A14}) is $p^{\prime}$. Upon setting $\Omega(p,\chi)=La(p)(\cosh p+\cosh \chi)$, we may express (\ref{A14}) in the form
%
\begin{eqnarray}
ds^2=\Omega^2(p,\chi)\left[dp^{\prime 2}-dx^{\prime 2} -dy^{\prime 2} -dz^{\prime 2}\right],
\label{B2}
\end{eqnarray}
%
i.e. the conformal to flat form. From Sec. \ref{s:conformal_gauge_sols}, we recall that the conformal gravity fluctuation (\ref{AP61}) takes the form
%
\begin{eqnarray}
\delta W_{\mu\nu}=(1/2)\Omega^{-2}\eta^{\sigma\rho}\eta^{\alpha\beta}\partial_{\sigma}\partial_{\rho} \partial_{\alpha}\partial_{\beta}k_{\mu\nu}
\end{eqnarray}
%
where $k_{\mu\nu}=\Omega^{-2}(x)K_{\mu\nu}$. Solving this fourth-order wave equation in terms of momentum eiginstates as done in Sec. \ref{ss:fluctuations_around_flat_in_the_tranverse_gauge} (cf. (\ref{AP29})), we express the solution to $\delta W_{\mu\nu}=0$ in the primed-variable form
%
\begin{eqnarray}
k_{\mu\nu}=A^{\prime}_{\mu\nu}e^{ik^{\prime}\cdot x^{\prime}}+(n^{\prime}\cdot x^{\prime})B^{\prime}_{\mu\nu}e^{ik^{\prime}\cdot x^{\prime}}+A^{\prime *}_{\mu\nu}e^{-ik^{\prime}\cdot x^{\prime}}+(n^{\prime}\cdot x^{\prime})B^{\prime *}_{\mu\nu}e^{-ik^{\prime}\cdot x^{\prime}},
\label{B3}
\end{eqnarray}
%
where $\eta^{\mu\nu}k^{\prime}_{\mu}k^{\prime}_{\nu}=0$. Within this coordinate basis, the conformal gauge condition takes the form $\partial^{\prime}_{\nu}k^{\mu\nu}=0$, with $k^{\mu\nu}=\eta^{\mu\alpha}\eta^{\nu\beta}k_{\alpha\beta}$. Substituting \eqref{B3} into the conformal gauge, it follows that
%
\begin{eqnarray}
&&ik^{\prime \nu}\left[A^{\prime}_{\mu\nu}e^{ik^{\prime}\cdot x^{\prime}}+(n^{\prime}\cdot x^{\prime})B^{\prime}_{\mu\nu}e^{ik^{\prime}\cdot x^{\prime}}-A^{\prime *}_{\mu\nu}e^{-ik^{\prime}\cdot x^{\prime}}-(n^{\prime}\cdot x^{\prime})B^{\prime *}_{\mu\nu}e^{-ik^{\prime}\cdot x^{\prime}}\right]
\nonumber\\
&&+n^{\prime\nu}\left[B^{\prime}_{\mu\nu}e^{ik^{\prime}\cdot x^{\prime}}+B^{\prime *}_{\mu\nu}e^{-ik^{\prime}\cdot x^{\prime}}\right]=0.
\label{B4}
\end{eqnarray}
%
For these expression to hold for all $x'$, it then follows that we must have
%
\begin{eqnarray}
&&ik^{\prime \nu}A^{\prime}_{\mu\nu}+n^{ \prime\nu}B^{\prime}_{\mu\nu}=0,\quad ik^{\prime \nu}B^{\prime}_{\mu\nu}=0, 
\nonumber\\
&&
-ik^{\prime \nu}A^{\prime *}_{\mu\nu}+n^{ \prime\nu}B^{\prime *}_{\mu\nu}=0,\quad -ik^{\prime \nu}B^{\prime *}_{\mu\nu}=0.
\label{B5}
\end{eqnarray}
%
Inspection of \eqref{B3} show that the $B^{\prime}_{\mu\nu}$ term has leading order large time behavior via its $n^{\prime}\cdot x^{\prime}$ prefactor. Consequently, continuing to work to leading order, we ignore the non-leading $A^{\prime}_{\mu\nu}$ modes and may take the $B^{\prime}_{\mu\nu}$ modes to obey the transverse momentum space conditions $ik^{\prime \nu}B^{\prime}_{\mu\nu}=0$ and $-ik^{\prime \nu}B^{\prime *}_{\mu\nu}=0$. It follows then that components of the $B^{\prime}_{\mu\nu}$ modes have the same $(n^{\prime}\cdot x^{\prime})e^{ik^{\prime}\cdot x^{\prime}}$ leading behavior in coordinate space. More specifically, since $n^{\prime \mu}=(1,0,0,0)$ the fluctuations grow linearly in the time variable $p^{\prime}$ associated with (\ref{B2}).

At present, we have expressed the solutions in terms of the conformal to flat coordinate system; however, to make contact with their leading order behavior in terms of the canonical comoving coordinates $t$ and $r$, we need to both reexpress $\Omega(p,\chi)$ in terms of the comoving coordinates $(t,r)$ and transform the components of the fluctuation $K_{\mu\nu}$ to the comoving coordinates. Starting first with $\Omega(p,\chi)$, we note that from $K_{\mu\nu} = \Omega^2 k_{\mu\nu}$ and \eqref{AP61} that the $K_{\mu\nu}$ fluctuations grow as $\Omega^2(p,\chi)p^{\prime}$. Making use of (\ref{A14}) and (\ref{A11}) from Appendix \ref{ab:cosmologies}, and coordinate transformation (\ref{B1}) one calculates $\Omega^2(p,\chi)p^{\prime}$ to obtain
%
\begin{eqnarray}
\Omega^2(p,\chi)p^{\prime}&=&L^2a^2(p)(\cosh p+\cosh \chi)^2p^{\prime}=L^2a^2(p)\sinh p (\cosh p+\cosh \chi)
\nonumber\\
&=&(d^2+t^2)\frac{t}{d}\left[\left(1+\frac{t^2}{d^2}\right)^{1/2}+\left(1+\frac{r^2}{L^2}\right)^{1/2}\right].
\label{B6}
\end{eqnarray}
%
Analysis of \eqref{B6} reveals that $\Omega^2(p,\chi)p^{\prime}$ for $t\ll d$ grows linearly in $t$, for later times and grows as $t^4$ when $t\gg d$. Consequently, in the given conformal to flat Minkowski coordinate system of (\ref{A14}), we see that the leading time behavior ($t \gg d$ specifically) of all the components of $K_{\mu\nu}$ is proportional to $t^4$.

The second task that remains is to transform the fluctuation $K_{\mu\nu}$ itself. To this end, one can note that in transforming from (\ref{A14}) to (\ref{A13}), there is no alteration of the behavior of $p^{\prime}$, with all dependence relegated to the conformal factor. However, one must additional be careful to note that the final transformation between spatial Cartesian coordinates (\ref{A14})  and polar coordinates (\ref{A13}) does in fact introduce a dependence on $r^{\prime}$. One can observe that, any angular component induces a dependence on the comoving $t$. For instance, in terms of the $K_{x^{\prime}x^{\prime}}$ type fluctuations in the (\ref{A14}) coordinate system, we have the following transformation relations in the (\ref{A13}) coordinate system
%
\begin{align}
K_{\theta\theta}&=(r^{\prime}\cos\theta\cos\phi)^2K_{x^{\prime}x^{\prime}}+(r^{\prime}\cos\theta\sin\phi)^2K_{y^{\prime}y^{\prime}}+(r^{\prime}\sin\theta)^2K_{z^{\prime}z^{\prime}}
\nonumber\\
&=\frac{\sinh^2\chi}{(\cosh p+\cosh \chi)^2}[\cos^2\theta\cos^2\phi K_{x^{\prime}x^{\prime}}+\cos^2\theta\sin^2\phi K_{y^{\prime}y^{\prime}}+\sin^2\theta K_{z^{\prime}z^{\prime}}]
\nonumber\\
&=\frac{r^2d^2}{[L(d^2+t^2)^{1/2}+d(L^2+r^2)^{1/2}]^2}\times
\nonumber\\
&\qquad[\cos^2\theta\cos^2\phi K_{x^{\prime}x^{\prime}}+\cos^2\theta\sin^2\phi K_{y^{\prime}y^{\prime}}+\sin^2\theta K_{z^{\prime}z^{\prime}}],
\nonumber\\
K_{r^{\prime}\theta}&=r^{\prime}\cos\theta\sin\theta\cos^2\phi K_{x^{\prime}x^{\prime}}+r^{\prime}\cos\theta\sin\theta\sin^2\phi K_{y^{\prime}y^{\prime}}-r^{\prime}\sin\theta\cos\theta K_{z^{\prime}z^{\prime}}
\nonumber\\
&=\frac{\sinh\chi}{(\cosh p+\cosh \chi)}\times 
\nonumber\\
&\qquad[\cos\theta\sin\theta\cos^2\phi K_{x^{\prime}x^{\prime}}+\cos\theta\sin\theta\sin^2\phi K_{y^{\prime}y^{\prime}}-\sin\theta\cos\theta K_{z^{\prime}z^{\prime}}],
\nonumber\\
K_{p^{\prime}\theta}&=r^{\prime}\cos\theta\cos\phi K_{p^{\prime}x^{\prime}}+r^{\prime}\cos\theta\sin\phi K_{p^{\prime}y^{\prime}}-r^{\prime}\sin\theta K_{p^{\prime}z^{\prime}}
\nonumber\\
&=\frac{\sinh\chi}{(\cosh p+\cosh \chi)}[\cos\theta\cos\phi K_{p^{\prime}x^{\prime}}+\cos\theta\sin\phi K_{p^{\prime}y^{\prime}}-\sin\theta K_{p^{\prime}z^{\prime}}]. 
\label{B7}
\end{align}
%
We may continue to compute the analogous expressions for $K_{\theta\phi}$, $K_{\phi\phi}$, $K_{r^{\prime}\phi}$ and $K_{p^{\prime}\phi}$. Within \eqref{B7}, we see that the $r^{\prime}=\sinh \chi/(\cosh p +\cosh \chi)$ prefactor has leading comoving time behavior as $t^0$ if $p=\chi$, (i.e. $t=r$ with both $t$ and $r$ large) or as $t^{-1}$ if $p\gg \chi$ (i.e. $t \gg r$). These correspond to the lightlike and timelike modes respectively. 

In transforming from from (\ref{A13}) to (\ref{A4}), we note that the angular sector is unaffected by the transformation. Thus the angular sector fluctuations $K_{\theta\theta}$, $K_{\theta  \phi}$, $K_{\phi\phi}$ associated with the comoving Robertson-Walker geometry given in (\ref{A4}) will have leading order growth as $t^4$ when including the contribution of the prefactor in (\ref{B7}) for lightlike coordinates and as $t^2$ for the timelike case. Moreover, since the lightlike $ds^2=0$ is both general coordinate invariant and conformal invariant, lightlike modes associated with the (\ref{A14}) metric will transform into lightlike modes associated with the metric (\ref{A4}). Finally, we note that a $t^4$ growth for the lightlike modes is a rather significant growth rate, one that cannot be obtained from standard Einstein gravity when using the same radiation matter source.

To further address the non-angular modes, we recall that the full coordinate transformation of the fluctuation is given by
%
\begin{eqnarray}
K_{\mu\nu}=\frac{\partial x^{\prime \alpha}}{\partial x^{\mu}}\frac{\partial x^{\prime\beta}}{\partial x^{\nu}}K^{\prime}_{\alpha\beta},
\label{B8}
\end{eqnarray}
%
and transformations between the $(p^{\prime}, r^{\prime})$, $(p,\chi)$ and $(t,r)$ take the form
%
\begin{eqnarray}
&&\frac{\partial p^{\prime }}{\partial p}=\frac{\partial r^{\prime }}{\partial \chi}=\frac{1+\cosh p\cosh\chi}{[\cosh p+\cosh\chi]^2},\qquad
\frac{\partial p^{\prime }}{\partial \chi}=\frac{\partial r^{\prime }}{\partial p}=-\frac{\sinh p\sinh\chi}{[\cosh p+\cosh\chi]^2}, 
\nonumber\\
&& \frac{\partial p}{\partial t}=\frac{1}{La(t)},\qquad \frac{\partial \chi}{\partial r}=\frac{1}{L\cosh\chi}.
\label{B9}
\end{eqnarray}
%

To interpret (\ref{B9}), we must determine the relation between $p$ and $\chi$, namely whether $p=\chi$ or $p\gg \chi$. Inspecting (\ref{B1}), we may observe that $t=d\sinh p$ when $La(t)=(d^2+t^2)^{1/2}$. Hence for large $t$ with $p=\chi$ also large, we obtain
%
\begin{eqnarray}
\frac{\partial p^{\prime }}{\partial p}=\frac{\partial r^{\prime }}{\partial \chi}=1,\qquad
\frac{\partial p^{\prime }}{\partial \chi}=\frac{\partial r^{\prime }}{\partial p}=1,\qquad \frac{\partial p}{\partial t}=\frac{1}{t},\qquad \frac{\partial \chi}{\partial r}=\frac{d}{Lt}.
\label{B10}
\end{eqnarray}
%
Consequently, as we go from the  $(p^{\prime},r^{\prime})$ coordinates to $(p,\chi)$ coordinates, the leading time behavior is unchanged. In further transforming (\ref{A10}) to the comoving (\ref{A4}) ($(p,\chi)$ to $(t,r)$) we obtain $1/t^2$ suppression in the $K_{tt}$, $K_{tr}$ and $K_{rr}$ sectors, a $1/t$ suppression for $K_{t\theta}$, $K_{t\phi}$, $K_{r\theta}$ and $K_{r\phi}$, and no suppression for $K_{\theta\theta}$, $K_{\theta\phi}$ and $K_{\phi\phi}$.

Lastly, we must include the contribution of the $\Omega^2(p,\chi)p^{\prime}\propto t^4$ prefactor and obtain $K_{tt}$, $K_{tr}$ and $K_{rr}$  growing as $t^2$, $K_{t\theta}$, $K_{t\phi}$, $K_{r\theta}$ and $K_{r\phi}$ growing as $t^3$, and $K_{\theta\theta}$, $K_{\theta\phi}$ and $K_{\phi\phi}$ growing as $t^4$. For this large $t$ and large $p=\chi$ scenario, the $K_{tt}$, $K_{tr}$, $K_{rr}$, $K_{t\theta}$, $K_{t\phi}$, $K_{r\theta}$ and $K_{r\phi}$ are suppressed with respect to $K_{\theta\theta}$, $K_{\theta\phi}$ and $K_{\phi\phi}$, so the leading growth will be the $t^4$ growth associated with the angular $K_{\theta\theta}$, $K_{\theta\phi}$ and $K_{\phi\phi}$. 


Now investigating the leading order behavior for large $p\gg \chi$, the transformations take the form
%
\begin{eqnarray}
&&\frac{\partial p^{\prime }}{\partial p}=\frac{\partial r^{\prime }}{\partial \chi}=\frac{d\cosh\chi}{t},\qquad
\frac{\partial p^{\prime }}{\partial \chi}=\frac{\partial r^{\prime }}{\partial p}=-\frac{d\sinh\chi}{t}, 
\nonumber\\
&& \frac{\partial p}{\partial t}=\frac{1}{t},\qquad \frac{\partial \chi}{\partial r}=\frac{1}{L\cosh\chi}.
\label{B11}
\end{eqnarray}
%
As we go from the  $(p^{\prime},r^{\prime})$ coordinates to $(p,\chi)$ coordinates, incorporating the $t^4$ dependence of $\Omega^2(p,\chi)p^{\prime}$ and including the prefactor in (\ref{B7}), we proceed analogous to before and determine growth of $K_{tt}\sim t^0$, $K_{tr}\sim t^1$, $K_{t\theta}\sim t^1$, $K_{t\phi}\sim t^1$, while $K_{rr}\sim t^2$, $K_{r\theta}\sim t^2$, $K_{r\phi}\sim t^2$,  $K_{\theta\theta}\sim t^2$, $K_{\theta\phi}\sim t^2$ and $K_{\phi\phi}\sim t^2$.
%%%%%%%%%%%%%%%%%
\footnote{Such suppression in this case can be viewed as the following: if $\chi$ is negligible then so is $r$, and thus the spatial part of the metric in (\ref{A4}) effectively becomes flat.}
%%%%%%%%%%%%%%%.
Hence, for large $p \gg \chi$,  $K_{tr}$, $K_{t\theta}$ and $K_{t\phi}$ all have $t^1$ leading order time behavior, and $K_{rr}$, $K_{r\theta}$, $K_{r\phi}$,  $K_{\theta\theta}$, $K_{\theta\phi}$ and $K_{\phi\phi}$  all have $t^2$ leading order behavior, with $t^2$ being the overall leading growth.




\chapter{Conclusions}
\label{C7}

In the analysis of cosmological perturbations, we have delved into the two methods of constructing and solving the equations of motion: the SVT decomposition of the fluctuations and the imposition of specific gauge constraints. As regards the former, we have seen that the SVT basis provides a very convenient formalism in expressing the 10 degrees of freedom within the metric perturbation $h_{\mu\nu}$. With the SVT3 quantities being defined as spatial integrals of $h_{\mu\nu}$, the SVT construction itself is thus intrinsically non-local, with the existence of the integrals themselves requiring proper asymptotic convergence. Upon implementing the SVT basis within the cosmological fluctuation equations as applied to de Sitter and Roberston Walker backgrounds, one can readily form a set of six gauge invariant combinations comprised of two scalars, a two-component transverse vector, and a traceless rank two tensor. 

Expressing the fluctuation equations solely in terms of these gauge invariants, one can solve them exactly and achieve an decoupling of the gauge invariants by application of higher derivatives which serve to project out the longitudinal and transverse components. By solving these higher derivative equations of motion exactly, we are able to assess the necessary conditions required for the decomposition theorem - a theorem asserting that scalars, vectors, and tensors decouple in the equations of motion themselves - to hold. We find that solely by imposing the same asymptotic boundary conditions at $r=\infty$ that are needed to set up the scalar, vector, tensor basis in the first place, one then indeed does get the decomposition theorem for fluctuations around a flat background, around a de Sitter background or around a spatially flat Robertson-Walker background with $k=0$. However, for fluctuations around a Robertson-Walker background with $k\neq 0$ one additionally has to require that fluctuations be well-behaved at $r=0$ in order to get the decomposition theorem to hold. The distinction between these two classes of solution is that in the first three (flat, de Sitter, $k=0$ Robertson-Walker) the background metric can be written as a conformal to flat metric with a conformal factor is solely time dependent, i.e. $\Omega(\tau)$. In this class, the SVT gauge invariant combinations are $\phi + \psi + \dot B - \ddot E$,  $B_i-\dot{E}_i$, $E_{ij}$, and $-\psi+\Omega^{-1}\dot{\Omega}(B-\dot{E})$ (up to a minor variation given the selected linear combination of scalars). In the second class ($k\neq 0$ Robertson-Walker) the background metric can be written as a conformal to flat metric with the conformal factor depending on both $r$ and $\tau$, i.e. $\Omega(r,\tau)$. Here the gauge invariant combinations are $\phi + \psi + \dot B - \ddot E$,  $B_i-\dot{E}_i$, $E_{ij}$, and $\psi-\Omega^{-1}\dot{\Omega}(B-\dot{E})-\Omega^{-1}\delta^{ij}\partial_j\Omega(E_i+\partial_iE)$ (up to a minor variation given the selected linear combination of scalars), where the evident intertwining of SVT scalars and vectors thus prevents establishing a decomposition theorem. Hence, in conformal to flat backgrounds asymptotic boundary conditions alone will in general only secure the decomposition theorem if the conformal factor depends solely on the conformal time. However using the underlying gauge freedom in the theory one can find gauges in which the scalars and vectors are not intertwined to thus recover the decomposition theorem.

Given that the SVT3 basis is not manifestly covariant with, for instance, scalars transforming into vectors under the full four dimensional transformations, we have introduced an alternative SVT4 formalism (and more generally developed the $D$ dimensional SVTD basis) that appropriately matches the underlying transformation group. When contrasted with SVT3, implementation of the SVT4 formalism provides significant simplification, with the fluctuation equations being able to be expressed in more compact and readily solvable forms. Similar to SVT3, one can apply higher derivatives in order to separate the scalar, vector, and tensor sectors at the level of the equations of motion. However, to obtain a decomposition theorem, one need not only impose good asymptotic spatial behavior but also must require constraints via initial conditions as well. 

As regards the second method of solution where one directly imposes gauge conditions (without enacting a decomposition), we have constructed a gauge condition appropriate to conformal gravity that remains invariant under conformal transformations. Referred to as the conformal gauge, upon implementing it within a generical conformal to flat background, one can express the entire perturbative Bach tensor succinctly as
%
\begin{eqnarray}
\delta W_{\mu\nu}(K_{\mu\nu})&=&\frac{1}{2}\Omega^{-2}\eta^{\sigma\rho}\eta^{\alpha\beta}\partial_{\sigma}\partial_{\rho} \partial_{\alpha}\partial_{\beta}(\Omega^{-2}K_{\mu\nu})
\nonumber\\
&=&\frac{1}{2}\Omega^{-2}\eta^{\sigma\rho}\eta^{\alpha\beta}\partial_{\sigma}\partial_{\rho} \partial_{\alpha}\partial_{\beta}k_{\mu\nu}.
\nonumber
\end{eqnarray}
%
with $K_{\mu\nu}$ being the traceless component of the general metric perturbation $h_{\mu\nu}$. With the $k=-1$ Roberston Walker geometry being the relevant cosmology within conformal gravity, we solve the radiation era fluctuation equations exactly. In terms of momentum eigenstates of the fourth order equation of motion, we find that the solutions grow as $t^4$, with $t$ the comoving time. Such can be contrasted with the $t^{1/2}$ growth one obtains in standard Einstein radiation era cosmology. Finally, we evaluate a large set of possible gauges within Einstein gravity with a de Sitter background, finding compact forms that facilitate exact integral solutions, with coupling only occurring between the trace and the remaining metric components. 




%%%%%%%%%%%%%%%%%%%%%%%%%%%%%%%%%%%%%%%%%%%%%%%%%%%%%%%%%%%%%%%%%%%%%%%%%%%%%%%%%%%%%%%%%%%%%%%%%%%%%
arise in regards to spatially asymptotic


at hand that one may use to solve the fluctuation equations, within this work we instill both. 

the perturbative equations of motion have historically been acknowledged as forming a rather complex set of coupled non-linear differential tensor equations. For instance, within Weinberg's cosmology book \cite{weinberg_2008}, he notes that even after performing additional simplifications that these equations are still ``fearsomely complicated''. 






extensive methods of simplifying the equations of motion and eliminating non-physical gauge modes are required in order to construct the perturbative solutions
%%%%%%%%%%%%%%%%%%%%%%%%%%%%%%%%%%%%%%%%%%%%%%%%%%%%%%%%%%%%%%%%%%%%%%%%%%%%%%%%%%%%%%%%%%%%%%%%%%%%%


*touch on decomposition theorem. maybe a little mock example
*conformal gravity - while we do not dig into details on qm here, has been proposed as a ... citations. dope as properties
*overview section
*imposing gauges or for intrinsic gauge invariants. difficult problem to solve, complicated fluctuations. as we move to curved spacetimes, solving not straightforward even with gauge constraints
** SVT3, scalar modes, CMB power spectrum. then motivate SVT4. projectors


%%%%%%%%%%%%%%%%%%%%%%%%%%%%%%%%%%%%%%%%%%%%%%%%%%%%%%%%%%%%%%%%%%%%%%%%%%%%%%%%%%%%%%%%%%%%%%%%%%%%%
solve equations, Determine conditions required for decomposition theorem. Does not hold unless further input. by going to higher derivatives. See if we can impose asymptotic boundary conditions. 

start introducing perfect fluid source. RW k=0 radiation. Determine matter + gravitaional gauge invariants. $\tau^2 e_ij$ sector goes as $t^{1/2}$, decomp follows w/ spatially asympt. bc's.

generalize to all RW, perturb perfect fluid, requires equation of state. identify gauge invariants, use many vuarvature relations, commutations 4th order. seek help from svt in terms of $h_{\mu\nu}$ to determine gauge invariants here. In order to solve, need to determine $\Omega(\tau)$ and reduce 11 dof's to 10. We then determine the form of $\Omega(\tau)$ in all curvatures in radiation and matter dominated. Reduce from 11 to 10 by specifying equation of state. We interpolate btween radiation and matter: transition between two eras is complicationed, but propto $p=w\rho$ in high temp (radiation) and low temp (matter). Transition era = recombination. $p=w\rho$ not always valid. Solve by suming over complete basis of modes associated with propagation of spinless massive particle in chosen $g_{\mu\nu}$ background. Complicated, but not done here generally. 

k=-1 RW general As the implications of boundary conditions are very sensitive to the sign of the coefficient of $k$, and we will need to monitor both positive and negative coefficient cases below. In implementing evolution equations that involve products of derivative operators such as the generic $(\tilde{\nabla}^2+\alpha)(\tilde{\nabla}^2+\beta)F=0$ . scalar sector checks out given good behavior (bounded) at infinity and origin. Seem to find vector that is bounded, well behaved at both, but does not obey decomp theorem. see end of vector section. Same for tensor sector.

n Sec. \ref{ss:rw_k=-1_svt3} we have seen that there are realizations of the evolution equations in the scalar, vector, and tensor sectors that would not lead to a decomposition theorem in those sectors. However, equally there are other realizations that given the boundary conditions would lead to a decomposition theorem. Thus we need to determine which realizations are the relevant ones. To this end we look not at the individual higher-derivative equations obeyed by the separate scalar, vector, and tensor sectors, but at how these various sectors interface with each other in the original second-order $\Delta_{\mu\nu}=0$ equations themselves. Any successful such interface would require that all the terms in $\Delta_{\mu\nu}=0$ would have to have the same $\chi$ behavior. Noting that the scalar modes appear with two $\tilde{\nabla}$ derivatives in $\Delta_{ij}=0$, the vector sector appears with one $\tilde{\nabla}$ derivative and the tensor appears with none, we need to compare derivatives of scalars with vectors and derivatives of vectors with tensors. 

If we force bc that vector and tensor modes vanish at $\chi=\infty$ instead of limiting to a constant value, then decomp holds. 

compute svt3 conformal gravity, instead working in conformal flat. Imposing boundary conditions leads to simple evolution equations. We can invert svt3 quantities in terms of $\delta W_{\mu\nu}$, to serve as alternative integral relations in the RW background. 

SVT4 minkowski, delta G is purely gauge invariant in zero background. Evaluate in ds4 w/o conformal factor. Make use of SVTD in constant 3 space. For scalar $\chi$ to obey decomp theorem, require very particular solution. General solution not at all forced to $\chi =0$; specific solution to the full evolution equations. No compelling reason to choose so. $F_{\mu\nu}$ and $\chi$ can still be localized in space, thus no spatially asymptotic bc could affect them. Completely solvable though. Could enforce decomp theorem with judiciously choosing IC's at an intial time. No compelling rationale for doing so. 

ds4 with a conformal factor. GI mixes scalars and vectors. Introduce $U^\mu$ to express covariantly. Find exact solutions. Again, same story. 

Do SVT4 desitter in conformal gravity. Simple structure, in fact TT sector has same form as standard Einstein gravity. Find relation between the two. Below, and also for flat space. Decomp is automatic, only $F_{\mu\nu}$. 
\begin{eqnarray}
\delta W_{\mu\nu}=(\nabla_{\alpha}\nabla^{\alpha}-4H^2)(\delta G_{\mu\nu}+\delta T_{\mu\nu})^{T\theta}.
\end{eqnarray}

By introducing timelike $U^\mu$ can express generalized SVT4 RW flucations in compact covariant form. Same story with decomp. 

dW conformal to flat. Beautiful super simple. 
\begin{eqnarray}
\delta W_{\mu\nu}&=&\Omega^{-2}\partial_{\sigma}\partial^{\sigma}\partial_{\tau}\partial^{\tau}F_{\mu\nu}.
\end{eqnarray} 

by looking at ds4 svt4 we saw mixing, and thus one shold only look for decomp theorem for gauge invariants and not for seaprate scalar vector, tensors sects and gauage invariance can in general intertwine them. We present example occuring in SVT3 to show not just artifact of SVT4. 

ads svt3 mixing. 
\begin{eqnarray}
\delta = \phi -\psi + \dot B - \ddot E + \frac{2}{z}(\tilde\nabla_3 E + E_3),
\end{eqnarray}

General conformal to flat. $eta=\psi -\Omega^{-1}\dot{\Omega}(B-\dot E)+\Omega^{-1}\tilde\nabla^i\Omega(E_i+\tilde\nabla_i E)$ Spatially dependent $\Omega(x)$ leads to inseperable gauge invariant not foudn in non-conformal geoms. Procedure is one must first determine GI's, then separate, not other way around. For some geoms no choice of coords can undo intertwiing (if conformal factor pulled out). To express the $\eta$ in terms of a curvature invariant, one cannot make recourse to $\delta W_{\mu\nu}$, however, in traceless radiation we can take $\delta(g_{\mu\nu}G^{\mu\nu}$ to determine $\eta$ in terms of the $h_{\mu\nu}$. 

Using gauge freedom, imposing gauge, we can decouple intertwining in gauge invariants.

% All the references are located in here
\chapter*{Bibliography}
\label{chap:References}
\addcontentsline{toc}{chapter}{Bibliography}

\bibliographystyle{uconn_unsrtnat}
\singlespacing\bibliography{bibliography} %link to bibtex file here


% Appendices: Place them after references
\appendix

\chapter{SVT by Projection}
\label{aa:svt_projection}

%%%%%%%%%%%%%%%%%%%%%%%%%%%%%%%%%%%%%%%%%%%%
\section{The $3+1$ Decomposition}
\label{aas:3_1_decomp}
%%%%%%%%%%%%%%%%%%%%%%%%%%%%%%%%%%%%%%%%%%%%

For geometries with constant spatial curvature, (e.g. Roberston Walker, Minkowski) it is especially convenient to utilize projections that decouple the time and spatial components of symmetric rank two tensors. Thus, we introduce the standard covariant $3+1$ decomposition of a symmetric rank two tensor $T_{\mu\nu}$ in a 4-dimensional geometry with metric $g_{\mu\nu}$. To facilitate decomposition, we need only make use of a 4-vector $U^{\mu}$ obeying $g_{\mu\nu}U^{\mu}U^{\nu}=-1$ and a projector 
%
\begin{eqnarray}
P_{\mu\nu}=g_{\mu\nu}+U_{\mu}U_{\nu}
\label{E1}
\end{eqnarray}
%
that obeys
%
\begin{eqnarray}
U_{\mu}P^{\mu\nu}=0, \qquad P_{\mu\nu}P^{\mu\nu}=g_{\mu\nu}P^{\mu\nu}=3,\qquad P_{\mu\sigma}P^{\sigma}_{\phantom{\sigma}\nu}=P_{\mu\nu}.
\label{E2}
\end{eqnarray}
%
In terms of the projector we can write
%
\begin{eqnarray}
T_{\mu\nu}&=&g_{\mu}^{\phantom{\mu}\sigma}g_{\nu}^{\phantom{\nu}\tau}T_{\sigma\tau}=
P_{\mu}^{\phantom{\mu}\sigma}P_{\nu}^{\phantom{\nu}\tau}T_{\sigma\tau}
-U_{\mu}U^{\sigma}P_{\nu}^{\phantom{\nu}\tau}T_{\sigma\tau}
\nonumber\\
&&
-P_{\mu}^{\phantom{\mu}\sigma}U_{\nu}U^{\tau}T_{\sigma\tau}
+U_{\mu}U_{\nu}U^{\sigma}U^{\tau}T_{\sigma\tau}.
\label{E3}
\end{eqnarray}
%
On introducing
%
\begin{eqnarray}
\rho&=&U^{\sigma}U^{\tau}T_{\sigma\tau},\qquad p=\frac{1}{3}P^{\sigma\tau}T_{\sigma\tau},\qquad 
q_{\mu}=-P_{\mu}^{\phantom{\mu}\sigma}U^{\tau}T_{\sigma\tau},
\nonumber\\
\pi_{\mu\nu}&=&\left[\frac{1}{2}P_{\mu}^{\phantom{\mu}\sigma}P_{\nu}^{\phantom{\nu}\tau}
+\frac{1}{2}P_{\nu}^{\phantom{\nu}\sigma}P_{\mu}^{\phantom{\mu}\tau}
-\frac{1}{3}P_{\mu\nu}P^{\sigma\tau}\right]T_{\sigma\tau},
\label{E4}
\end{eqnarray}
%
which obey
%
\begin{eqnarray}
U^{\mu}q_{\mu}=0,\qquad U^{\nu}\pi_{\mu\nu}=0,\qquad \pi_{\mu\nu}=\pi_{\nu\mu},\qquad g^{\mu\nu}\pi_{\mu\nu}=P^{\mu\nu}\pi_{\mu\nu}=0,
\label{E5}
\end{eqnarray}
%
we can rewrite $T_{\mu\nu}$ as
%
\begin{eqnarray}
T_{\mu\nu}=(\rho+p)U_{\mu}U_{\nu}+pg_{\mu\nu}+U_{\mu}q_{\nu}+U_{\nu}q_{\mu}+\pi_{\mu\nu},
\label{E6}
\end{eqnarray}
%
a familiar form that may for instance be found in \cite{ellis_maartens_maccallum_2009}. As constructed, the ten-component $T_{\mu\nu}$ has been covariantly decomposed into two one-component 4-scalars, one three-component  4-vector that is orthogonal to $U_{\mu}$ and one five-component traceless, rank two tensor that is also orthogonal  to $U_{\mu}$.

%%%%%%%%%%%%%%%%%%%%%%%%%%%%%%%%%%%%%%%%%%%%
\section{Vector Fields}
\label{aas:vector_fields}
%%%%%%%%%%%%%%%%%%%%%%%%%%%%%%%%%%%%%%%%%%%%

We recall from Sec. \ref{s:einstein_gravity} that the general $h_{\mu\nu}$ has ten components and with the freedom to impose four coordinate transformations, these may be reduced to six physical components. In terms of the SVT decompositions, the SVT3 and SVT4 formalisms yield fluctuation equations that only depend on six SVT combinations in the SVT3 or SVT4 expansions of $h_{\mu\nu}$. We recall that the SVT3 and SVT4 components are related to the components of $h_{\mu\nu}$ via integral relations such as that given in (\ref{2.12}), for example
%
\begin{eqnarray}
B=\int d^3yD^{(3)}(\mathbf{x}-\mathbf{y})\tilde{\nabla}_y^ih_{0i},\quad B_i=h_{0i}-\tilde{\nabla}_i\int d^3yD^{(3)}(\mathbf{x}-\mathbf{y})\tilde{\nabla}_y^ih_{0i}.
\label{A.1a}
\end{eqnarray}
%
From \eqref{A.1a} one can observe that components are intrinsically non-local, with their very existence requiring that the associated integrals exist. Consequently, we see that asymptotic boundary conditions are a necessary and irreducible component in their construction. As discussed in \cite{mannheim_2005} and \cite{amarasinghe_2019}, we introduce another method to implement gauge invariance using non-local operators, namely the projection operator approach, with such an approach being equivalent to the SVT formalism used in Ch. \ref{c:scalar_vector_tensor_basis} .

To discuss the application of the projection operator approach to rank two tensors such as $h_{\mu\nu}$ we first apply it to a four-dimensional gauge field $A_{\mu}$. Thus in analog to (\ref{A.1a}) we set
%
\begin{eqnarray}
A_{\mu}&=&A^T_{\mu}+\partial_{\mu}\int d^4x^{\prime}D(x-x^{\prime})\partial^{\alpha}A_{\alpha}=A^T_{\mu}+A^L_{\mu},
\\
A^T_{\mu}&=&A_{\mu}-\partial_{\mu}\int d^4x^{\prime}D(x-x^{\prime})\partial^{\alpha}A_{\alpha},\quad A_{\mu}^L=\partial_{\mu}\int d^4x^{\prime}D(x-x^{\prime})\partial^{\alpha}A_{\alpha},
\nonumber
\label{A.2a}
\end{eqnarray}
%
where $\partial_{\mu}\partial^{\mu}D(x-x^{\prime})=\delta^4(x-x^{\prime})$, where $A^T_{\mu}$ obeys the transverse condition $\partial^{\mu}A^T_{\mu}=0$, and where $A_{\mu}^L$ is longitudinal. The utility of this expansion is that under  $A_{\mu}\rightarrow A_{\mu}+\partial_{\mu}\chi$ the transverse $A^T_{\mu}$ transforms as 
%
\begin{eqnarray}
A^T_{\mu}\rightarrow&& A_{\mu}+\partial_{\mu}\chi-\partial_{\mu}\int d^4x^{\prime}D(x-x^{\prime})\partial^{\alpha}A_{\alpha}
-\partial_{\mu}\int d^4x^{\prime}D(x-x^{\prime})\partial^{\alpha}\partial_{\alpha}\chi 
\nonumber\\
&&=A_{\mu}^T,
\label{A.3a}
\end{eqnarray}
%
where we perform an integration by parts. Thus with integration by parts the transverse $A^T_{\mu}$ is automatically gauge invariant. In addition we note that $A_{\mu}^T$ obeys 
%
\begin{eqnarray}
\partial_{\nu}\partial^{\nu}A^T_{\mu}=\partial_{\nu}\partial^{\nu}A_{\mu}-\partial_{\mu}\partial^{\nu}A_{\nu}=\partial^{\nu}F_{\nu\mu}.
\label{A.4a}
\end{eqnarray}
%
Thus just as with the use of the non-local SVT formalism for gravity, the use of the non-local $A_{\mu}^T$ enables us to write the Maxwell equations entirely in terms of gauge-invariant quantities. With $A_{\mu}^L$ being the derivative of a scalar function it is pure gauge, and thus cannot appear in the gauge-invariant Maxwell equations. Moreover, while there may be an integration by parts issue for $A_{\mu}^T$, there is none for $\partial_{\nu}\partial^{\nu}A^T_{\mu}$ as it is equal to the gauge-invariant quantity $\partial_{\nu}\partial^{\nu}A_{\mu}-\partial_{\mu}\partial^{\alpha}A_{\alpha}$, just as it must be since the Maxwell equations are gauge invariant.
In the SVT language, with (\ref{A.1a}) and (\ref{A.2a}) only involving scalars and vectors, we can think of (\ref{A.1a}) as an SV3 decomposition of the 3-component $h_{0i}$, and (\ref{A.2a}) as an SV4 decomposition of the 4-component $A_{\mu}$.

An alternate way of understanding these results is to introduce a projection operator
%
\begin{equation}
\Pi_{\mu\nu}=\eta_{\mu\nu}-\frac{\partial}{\partial x^{\mu}}\int
d^4x^{\prime}D(x-x^{\prime})\frac{\partial}{\partial x^{\prime \nu}},
\label{A.5a}
\end{equation}
%
as we can then rewrite $A_{\mu}^T$  as 
%
\begin{equation}
A_{\mu}^{T}=\Pi_{\mu\nu}A^{\nu}.
\label{A.6a}
\end{equation}
%
As introduced, $\Pi_{\mu\nu}$ obeys the projector algebra relations
%
\begin{align}
\Pi_{\mu\nu}\Pi^{\nu}_{\phantom{\sigma}\sigma}
&=\Pi_{\mu\sigma},
\nonumber \\
\Pi_{\mu\nu}A^{T \nu}&= A^{T}_{\mu}-\partial_{\mu}\int
d^4x^{\prime}D(x-x^{\prime})
\partial_{\nu}A^{T\nu}(x^{\prime})=A^{T}_{\mu},
\nonumber \\
\Pi_{\mu\nu}A^{L \nu}&=\partial_{\mu}\int
d^4x^{\prime}D(x-x^{\prime})\partial_{\nu}A^{\nu}(x^{\prime})
-\partial_{\mu}\int
d^4x^{\prime}D(x-x^{\prime})\times
\nonumber\\
&\qquad
\partial_{\nu}\partial^{\nu}\int
d^4x^{\prime\prime}D(x^{\prime}-x^{\prime\prime})
\partial_{\sigma}A^{\sigma}(x^{\prime\prime})=0.
\label{A.7a}
\end{align}
%
In the SVT4 language we set  $A_{\mu}=A_{\mu}^T+\partial_{\mu}A$, and can thus identify 
%
\begin{eqnarray}
A_{\mu}^T=  \Pi_{\mu\nu}A^{\nu},\quad A_{\mu}^L=\partial_{\mu}A=(\eta_{\mu\nu}-\Pi_{\mu\nu})A^{\nu}.
\label{A.8a}
\end{eqnarray}
%
For vector fields the SVT formalism is thus equivalent to the projector formalism.  Having now established this equivalence for vector fields, we turn now to tensor fields.

%%%%%%%%%%%%%%%%%%%%%%%%%%%%%%%%%%%%%%%%%%%%
\section{Transverse and Longitudinal Projection Operators for Flat Spacetime Tensor Fields}
\label{aas:tt_long_flat_proj}
%%%%%%%%%%%%%%%%%%%%%%%%%%%%%%%%%%%%%%%%%%%%

For tensor fields we introduce 4-dimensional flat  spacetime transverse and longitudinal projection operators \cite{mannheim_2005,amarasinghe_2019}: 
%
\begin{eqnarray}
T_{\mu\nu\sigma\tau}&=&\eta_{\mu\sigma}\eta_{\nu\tau}
-\partial_{\mu}\int d^4x^{\prime}D(x-x^{\prime})
\eta_{\nu\tau}\partial_{\sigma}
-\partial_{\nu}\int d^4x^{\prime}D(x-x^{\prime})
\eta_{\mu\sigma}\partial_{\tau}
\nonumber \\
&+&\partial_{\mu}\partial_{\nu}\int
d^4x^{\prime}D(x-x^{\prime})\partial_{\sigma}\int
d^4x^{\prime\prime}D(x^{\prime}-x^{\prime\prime})
\partial_{\tau},
\nonumber\\
L_{\mu\nu\sigma\tau}&=&\partial_{\mu}\int d^4x^{\prime}D(x-x^{\prime})
\eta_{\nu\tau}\partial_{\sigma}
+\partial_{\nu}\int d^4x^{\prime}D(x-x^{\prime})
\eta_{\mu\sigma}\partial_{\tau}
\nonumber \\
&-&\partial_{\mu}\partial_{\nu}\int
d^4x^{\prime}D(x-x^{\prime})\partial_{\sigma}\int
d^4x^{\prime\prime}D(x^{\prime}-x^{\prime\prime})
\partial_{\tau}.
\label{A.9a}
\end{eqnarray}
%
As constructed, these projectors obey a standard projector algebra
%
\begin{eqnarray}
&&T_{\mu\nu\sigma\tau}T^{\sigma\tau}_{\phantom{\sigma\tau}\alpha\beta}=
T_{\mu\nu\alpha\beta},\quad
L_{\mu\nu\sigma\tau}L^{\sigma\tau}_{\phantom{\sigma\tau}\alpha\beta}
=L_{\mu\nu\alpha\beta},
\nonumber \\
&&T_{\mu\nu\sigma\tau}L^{\sigma\tau}_{\phantom{\sigma\tau}\alpha\beta}=
0,\quad
L_{\mu\nu\sigma\tau}T^{\sigma\tau}_{\phantom{\sigma\tau}\alpha\beta}
=0,\quad L_{\mu\nu\sigma\tau}
+T_{\mu\nu\sigma\tau}
=\eta_{\mu\sigma}\eta_{\nu\tau}.
\label{A.10a}
\end{eqnarray}
% 
In terms of these projectors we define transverse and longitudinal components $h^{T}_{\mu\nu}$ and $h^{L}_{\mu\nu}$ of $h_{\mu\nu}$ according to
% 
\begin{eqnarray}
T_{\mu\nu\sigma\tau}h^{\sigma\tau}&=&h^{T}_{\mu\nu}=h_{\mu\nu}
-\partial_{\mu}\int
d^4x^{\prime}D(x-x^{\prime})\partial_{\sigma}
h^{\sigma}_{\phantom{\sigma}\nu}(x^{\prime})  
\nonumber\\
&&
-\partial_{\nu}\int d^4x^{\prime}D(x-x^{\prime})
\partial_{\kappa}h^{\kappa}_{\phantom{\kappa}\mu}(x^{\prime})
\nonumber \\
&&+\partial_{\mu}\partial_{\nu}\int
d^4x^{\prime}D(x-x^{\prime})\partial_{\sigma}\int
d^4x^{\prime\prime}D(x^{\prime}-x^{\prime\prime})
\partial_{\kappa}h^{\sigma\kappa}(x^{\prime\prime}),
\nonumber\\
L_{\mu\nu\sigma\tau}h^{\sigma\tau}&=&h^{L}_{\mu\nu}=\partial_{\mu}\int
d^4x^{\prime}D(x-x^{\prime})\partial_{\sigma}
h^{\sigma}_{\phantom{\sigma}\nu}(x^{\prime}) 
+\partial_{\nu}\int d^4x^{\prime}D(x-x^{\prime})
\partial_{\kappa}h^{\kappa}_{\phantom{\kappa}\mu}(x^{\prime})
\nonumber \\
&&-\partial_{\mu}\partial_{\nu}\int
d^4x^{\prime}D(x-x^{\prime})\partial_{\sigma}\int
d^4x^{\prime\prime}D(x^{\prime}-x^{\prime\prime})
\partial_{\kappa}h^{\sigma\kappa}(x^{\prime\prime}).
\label{A.11a}
\end{eqnarray}
%
Assuming integration by parts these components obey
%
\begin{eqnarray}
\partial_{\nu}h^{T\mu\nu}
=0,\quad 
\partial_{\nu}h^{L\mu\nu}=\partial_{\nu}h^{\mu\nu}.
\label{A.12a}
\end{eqnarray}
% 
With $h^{T}_{\mu\nu}$ transforming as $h^{T}_{\mu\nu}\rightarrow h^{T}_{\mu\nu}$ under $h_{\mu\nu}\rightarrow h_{\mu\nu}-\partial_{\mu}\epsilon_{\nu}-\partial_{\nu}\epsilon_{\mu}$ as long as we can integrate by parts, we see that, as introduced, $h^{T}_{\mu\nu}$ is both transverse and gauge invariant. 

On evaluation we obtain 
%
\begin{align}
&&\frac{1}{2}[\partial_{\mu}\partial_{\nu}h^{T}
+\partial_{\alpha}\partial^{\alpha}h_{\mu\nu}^{T}]
-\frac{1}{2}\eta_{\mu\nu}\partial_{\sigma}\partial^{\sigma}
h^{T}
=\frac{1}{2}[\partial_{\mu}\partial_{\nu}h
-\partial_{\mu}\partial_{\lambda}h^{\lambda}_{\phantom{\lambda}\nu}
-\partial_{\nu}\partial_{\lambda}h^{\lambda}_{\phantom{\lambda}\mu} 
\nonumber\\
&&\qquad
+\partial_{\alpha}\partial^{\alpha}h_{\mu\nu}]
-\frac{1}{2}\eta_{\mu\nu}[\partial_{\alpha}\partial^{\alpha}h
-\partial_{\sigma}\partial_{\lambda}h^{\sigma\lambda}],
\label{A.13a}
\end{align}
%
where $h^{T}$ is given by
%
\begin{equation}
h^{T}=\eta^{\alpha\beta}h_{\alpha\beta}^{T}
=h -\partial_{\nu}\int
d^4x^{\prime}D(x-x^{\prime})\partial_{\sigma}
h^{\sigma\nu}(x^{\prime}),
\label{A.14a}
\end{equation}
%
with $h=\eta^{\alpha\beta}h_{\alpha\beta}$. On recognizing the right-hand side of  (\ref{A.13a}) as $\delta R_{\mu\nu}-\frac{1}{2}\eta_{\mu\nu}\delta R=\delta G_{\mu\nu}$,
we obtain 
%
\begin{eqnarray}
&&\delta G_{\mu\nu}=\tfrac{1}{2}[\partial_{\mu}\partial_{\nu}h^{T}
+\partial_{\alpha}\partial^{\alpha}h_{\mu\nu}^{T}]
-\frac{1}{2}\eta_{\mu\nu}\partial_{\sigma}\partial^{\sigma}
h^{T}.
\label{A.15a}
\end{eqnarray}
%
We thus write the perturbed Einstein tensor entirely in terms of the non-local, gauge invariant, six degree of freedom $h_{\mu\nu}^T$.

To make contact with the SVT4 expansion we insert
%
\begin{eqnarray}
h_{\mu\nu}=-2\eta_{\mu\nu}\chi+2\partial_{\mu}\partial_{\nu}F
+ \partial_{\mu}F_{\nu}+\partial_{\nu}F_{\mu}+2F_{\mu\nu}
\label{A.16a}
\end{eqnarray}
%
into $h_{\mu\nu}^T$,  to obtain
%
\begin{eqnarray}
h^T_{\mu\nu}=-2\eta_{\mu\nu}\chi+2F_{\mu\nu}+2\partial_{\mu}\partial_{\nu}\int d^4D(x-x^{\prime})\chi(x^{\prime}),\quad h^T=-6\chi.
\label{A.17a}
\end{eqnarray}
%
With $\delta G_{\mu\nu}$ being written in terms of the projected $h^T_{\mu\nu}$, we see that it is written in terms of the SVT4 $F_{\mu\nu}$ and $\chi$. However as written, $h_{\mu\nu}^T$ contains an integral term in (\ref{A.17a}). To eliminate it we extend transverse projection to transverse-traceless projection.

%%%%%%%%%%%%%%%%%%%%%%%%%%%%%%%%%%%%%%%%%%%%
\section{Transverse-Traceless Projection Operators for Flat Spacetime Tensor Fields}
\label{aas:tt_proj}
%%%%%%%%%%%%%%%%%%%%%%%%%%%%%%%%%%%%%%%%%%%%

In \cite{mannheim_2005} and \cite{amarasinghe_2019} two further projectors were introduced
%
\begin{eqnarray}
Q_{\mu\nu\sigma\tau}&=&\frac{1}{3}\left[\eta_{\mu\nu}
-\partial_{\mu}\partial_{\nu}\int d^4x^{\prime}D(x-x^{\prime})\right]
\left[\eta_{\sigma\tau}-\partial^{\prime}_{\sigma}\int
d^4x^{\prime\prime}D(x^{\prime}-x^{\prime\prime})\partial^{\prime\prime}_{\tau}\right],
\nonumber\\
P_{\mu\nu\sigma\tau}&=&T_{\mu\nu\sigma\tau}-Q_{\mu\nu\sigma\tau}.
\label{A.18a}
\end{eqnarray}
%  
They obey the projector algebra
%
\begin{eqnarray}
T_{\mu\nu\sigma\tau}Q^{\sigma\tau}_{\phantom{\sigma\tau}\alpha\beta}
&=&Q_{\mu\nu\alpha\beta},\quad
Q_{\mu\nu\sigma\tau}T^{\sigma\tau}_{\phantom{\sigma\tau}\alpha\beta}
=Q_{\mu\nu\alpha\beta},\quad
Q_{\mu\nu\sigma\tau}Q^{\sigma\tau}_{\phantom{\sigma\tau}\alpha\beta}
=Q_{\mu\nu\alpha\beta}, 
\nonumber\\
P_{\mu\nu\sigma\tau}Q^{\sigma\tau\alpha\beta}&=&0,\quad
Q_{\mu\nu\sigma\tau}P^{\sigma\tau\alpha\beta}=0,\quad
P_{\mu\nu\sigma\tau}P^{\sigma\tau}_{\phantom{\sigma\tau}\alpha\beta}
=P_{\mu\nu\alpha\beta}.
\label{A.19a}
\end{eqnarray}
%
The projector $P_{\mu\nu\sigma\tau}$ projects out the traceless piece of $h^T_{\mu\nu}$, while $Q_{\mu\nu\sigma\tau}$ projects out its complement, and they implement
%
\begin{eqnarray}
P_{\mu\nu}^{\phantom{\mu\nu}\sigma\tau}h^T_{\sigma\tau}=h^{T\theta}_{\mu\nu},\quad 
Q_{\mu\nu}^{\phantom{\mu\nu}\sigma\tau}h^T_{\sigma\tau}
=h^T_{\mu\nu}-h^{T\theta}_{\mu\nu},
\label{A.20a}
\end{eqnarray}
% 
with $h^{T\theta}_{\mu\nu}$ being both traceless and transverse. With $Q_{\mu\nu}^{\phantom{\mu\nu}\sigma\tau}$ implementing $Q_{\mu\nu}^{\phantom{\mu\nu}\sigma\tau}h^L_{\sigma\tau}=0$, $P_{\mu\nu}^{\phantom{\mu\nu}\sigma\tau}$ implements $P_{\mu\nu}^{\phantom{\mu\nu}\sigma\tau}h^L_{\sigma\tau}=0$ as well, to thus implement 
%
\begin{eqnarray}
P_{\mu\nu}^{\phantom{\mu\nu}\sigma\tau}h_{\sigma\tau}=h^{T\theta}_{\mu\nu}.
\label{A.21a}
\end{eqnarray}
%
$P_{\mu\nu\sigma\tau}$ is thus a traceless projector not just for the transverse $h_{\mu\nu}^T$ but for the full $h_{\mu\nu}$ as well. We can thus introduce its complementary projection operator $U_{\mu\nu\sigma\tau}=\eta_{\mu\sigma}\eta_{\nu\tau}-P_{\mu\nu\sigma\tau}$, as it obeys
%
\begin{eqnarray}
P_{\mu\nu\sigma\tau}U^{\sigma\tau\alpha\beta}&=&0,\quad
U_{\mu\nu\sigma\tau}P^{\sigma\tau\alpha\beta}=0,\quad
U_{\mu\nu\sigma\tau}U^{\sigma\tau}_{\phantom{\sigma\tau}\alpha\beta}
=U_{\mu\nu\alpha\beta},
\nonumber\\
U_{\mu\nu}^{\phantom{\mu\nu}\sigma\tau}h_{\sigma\tau}&=&h_{\mu\nu}-h^{T\theta}_{\mu\nu}=
h^{L\theta}_{\mu\nu}+\frac{1}{3}\eta_{\mu\nu}\eta^{\sigma\tau}h_{\sigma\tau} 
\nonumber\\
&&
-\frac{1}{3}\partial_{\mu}\partial_{\nu}\int d^4y D(x-y)\eta^{\sigma\tau}h_{\sigma\tau},
\label{A.22a}
\end{eqnarray}
% 

Given (\ref{A.20a}) and (\ref{A.18a}) we obtain 
%
\begin{eqnarray}
h^{T\theta}_{\mu\nu}= h^{T}_{\mu\nu}-\frac{1}{3}\eta_{\mu\nu}\eta^{\sigma\kappa}h^{T}_{\sigma\kappa}
+\frac{1}{3}\partial_{\mu}\partial_{\nu}\int d^4y D(x-y)\eta^{\sigma\kappa}h^{T}_{\sigma\kappa},
\label{A.23a}
\end{eqnarray}
%
Inserting (\ref{A.17a}) into (\ref{A.23a}) yields
%
\begin{eqnarray}
h^{T\theta}_{\mu\nu}=2F_{\mu\nu},
\label{A.24a}
\end{eqnarray}
%
with $\chi$ dropping out. Finally, in terms of $h^{T\theta}_{\mu\nu}$ we can rewrite (\ref{A.15a}) as 
%
\begin{eqnarray}
&&\delta G_{\mu\nu}=
\tfrac{1}{2}\partial_{\alpha}\partial^{\alpha}h_{\mu\nu}^{T\theta}
-\tfrac{1}{3}\eta_{\mu\nu}\partial_{\sigma}\partial^{\sigma}h^{T}
+\tfrac{1}{3}\partial_{\mu}\partial_{\nu}h^{T}.
\label{A.25a}
\end{eqnarray}
%
Then with 
%
\begin{eqnarray}
F_{\mu\nu}=\tfrac{1}{2}h_{\mu\nu}^{T\theta}, \quad \chi=-\tfrac{1}{6}h^{T},
\label{A.26a}
\end{eqnarray}
%
we can rewrite (\ref{A.25a}) as 
%
\begin{eqnarray}
\delta G_{\mu\nu}&=&\partial_{\alpha}\partial^{\alpha}F_{\mu\nu}+2\eta_{\mu\nu}\partial_{\alpha}\partial^{\alpha}\chi-2\partial_{\mu}\partial_{\nu}\chi.
\label{A.27a}
\end{eqnarray}
%
We recognize (\ref{A.27a}) as the expression for $\delta G_{\mu\nu}$ as given in (\ref{3.10}) when $D=4$, and with $h^T_{\mu\nu}$ and thus $h^{T\theta}_{\mu\nu}$ and $h^T$ being gauge invariant, we confirm that given integration by parts $F_{\mu\nu}$ and $\chi$ are gauge invariant, just as noted in Sec. \ref{s:svtd}. Thus with (\ref{A.26a})
we establish the equivalence of the  SVT4 decomposition and the projection operator technique.

As a further example of this equivalence we note that for conformal gravity fluctuations around a flat spacetime background (\ref{13.18}) takes the form
%
\begin{eqnarray}
\delta W_{\mu\nu}&=&\frac{1}{2}\bigg{(}\partial_{\sigma}\partial^{\sigma}\partial_{\tau}\partial^{\tau}K_{\mu\nu}
-\partial_{\sigma}\partial^{\sigma}\partial_{\mu}\partial^{\alpha}K_{\alpha\nu}
-\partial_{\sigma}\partial^{\sigma}\partial_{\nu}\partial^{\alpha}K_{\alpha\mu} 
\nonumber\\
&&
+\frac{2}{3}\partial_{\mu}\partial_{\nu}\partial^{\alpha}\partial^{\beta}K_{\alpha\beta}+\frac{1}{3}\eta_{\mu\nu}\partial_{\sigma}\partial^{\sigma}\partial^{\alpha}\partial^{\beta}K_{\alpha\beta}\bigg{)},
\label{A.28a}
\end{eqnarray} 
%
where all derivatives are four-dimensional derivatives with respect to a flat Minkowski metric, and where $K_{\mu\nu}$ is given by $K_{\mu\nu}=h_{\mu\nu}-(1/4)\eta_{\mu\nu}\eta^{\alpha\beta}h_{\alpha\beta}$. Inserting (\ref{A.11a}) and (\ref{A.23a}) into (\ref{A.28a}) yields
%
\begin{eqnarray}
\delta W_{\mu\nu}&=&\frac{1}{2}\partial_{\sigma}\partial^{\sigma}\partial_{\tau}\partial^{\tau}h^{T\theta}_{\mu\nu}.
\label{A.29a}
\end{eqnarray} 
%
With the insertion of (\ref{A.16a}) into (\ref{A.28a}) yielding 
%
\begin{eqnarray}
\delta W_{\mu\nu}&=&\partial_{\sigma}\partial^{\sigma}\partial_{\tau}\partial^{\tau}F_{\mu\nu},
\label{A.30a}
\end{eqnarray} 
%
(cf. (\ref{13.20}) with $\Omega=1$), we recover (\ref{A.24a}), and again confirm the equivalence of the  SVT4 decomposition and the projection operator technique.

%%%%%%%%%%%%%%%%%%%%%%%%%%%%%%%%%%%%%%%%%%%%
\section{Transverse and Longitudinal Projection Operators for Curved Spacetime Tensor Fields}
\label{aas:tt_long_curved_proj}
%%%%%%%%%%%%%%%%%%%%%%%%%%%%%%%%%%%%%%%%%%%%

For curved spacetime with background metric $g_{\mu\nu}$ it is convenient to  define a 2-index propagator
%
\begin{equation} 
[g^{\nu}_{\phantom{\nu}\beta}\nabla_{\tau}\nabla^{\tau}
+\nabla_{\beta}\nabla^{\nu}]D^{\beta}_{\phantom{\beta}\sigma}
(x,x^{\prime}) =g^{\nu}_{\phantom{\nu}\sigma}(-g)^{-1/2}\delta^4
(x-x^{\prime}).
\label{A.31a}
\end{equation}
%
In terms of it we introduce \cite{mannheim_2005}
% 
\begin{eqnarray} 
T_{\mu\nu\sigma\tau}&=&g_{\mu\sigma}g_{\nu\tau}- \nabla_{\mu}\int
d^4x^{\prime}(-g)^{1/2}
D_{\nu\sigma}(x,x^{\prime})
\nabla_{\tau}  
\nonumber\\
&&
-\nabla_{\nu}\int
d^4x^{\prime}(-g)^{1/2}
D_{\mu\sigma}(x,x^{\prime})\nabla_{\tau},
\nonumber \\
L_{\mu\nu\sigma\tau}&=&\nabla_{\mu}\int
d^4x^{\prime}(-g)^{1/2}
D_{\nu\sigma}(x,x^{\prime})
\nabla_{\tau} 
+\nabla_{\nu}\int
d^4x^{\prime}(-g)^{1/2}
D_{\mu\sigma}(x,x^{\prime})
\nabla_{\tau}.
\nonumber\\
\label{A.32a}
\end{eqnarray}
%
These projection operators close on the projector algebra given in (\ref{A.10a}). As such, they effect
$T_{\mu\nu\sigma\tau}h^{\sigma\tau}=
h^{T}_{\mu\nu}$ and $L_{\mu\nu\sigma\tau}h^{\sigma\tau}=
h^{L}_{\mu\nu}$, where
%
\begin{equation} 
h^{T}_{\mu\nu}=h_{\mu\nu}-\nabla_{\mu}\int
d^4x^{\prime}(-g)^{1/2}
D^{\nu}_{\phantom{\nu}\sigma}(x,x^{\prime})
\nabla_{\tau}h^{\sigma\tau}  
-\nabla_{\nu}\int
d^4x^{\prime}(-g)^{1/2}
D^{\mu}_{\phantom{\mu}\sigma}(x,x^{\prime})
\nabla_{\tau}h^{\sigma\tau},
\label{A.33a}
\end{equation}
% 
%
\begin{equation} 
h^{L}_{\mu\nu}=\nabla_{\mu}\int
d^4x^{\prime}(-g)^{1/2}
D^{\nu}_{\phantom{\nu}\sigma}(x,x^{\prime})
\nabla_{\tau}h^{\sigma\tau} 
+\nabla_{\nu}\int
d^4x^{\prime}(-g)^{1/2}
D^{\mu}_{\phantom{\mu}\sigma}(x,x^{\prime})
\nabla_{\tau}h^{\sigma\tau}.
\label{A.34a}
\end{equation}
% 

The utility of constructing these projected states is that under a gauge transformation $h_{\mu\nu}$ transforms into $h_{\mu\nu}-\nabla_{\mu}\epsilon_{\nu}-\nabla_{\nu}\epsilon_{\mu}$. However, we see that this is precisely the structure of $h^{L}_{\mu\nu}$. The longitudinal component of $h_{\mu\nu}$ can thus be removed by a gauge transformation, and the fluctuation Einstein equations can only depend on the 6-component $h^{T}_{\mu\nu}$. However, unlike the flat background case where one can write $\delta G_{\mu\nu}$ itself entirely in terms of $h^T_{\mu\nu}$, in the curved background case there must be a background $T_{\mu\nu}$, and thus it is only in the full $\delta G_{\mu\nu}+8\pi G \delta T_{\mu\nu}$ that the metric fluctuations can be described entirely by $h^T_{\mu\nu}$. If we introduce a quantity $\delta T^T_{\mu\nu}$ in which the dependence on $\epsilon_{\mu}$ has been excluded (i.e. under a gauge transformation $\delta T_{\mu\nu}\rightarrow \delta T^T_{\mu\nu}$ plus a function of $\epsilon_{\mu}$, and this function of $\epsilon_{\mu}$ cancels against an identical function of $\epsilon_{\mu}$ in $\delta G_{\mu\nu}$), then following the commuting of some derivatives,  the fluctuation equations take the form \cite{mannheim_2005}
%
\begin{eqnarray} 
\delta G_{\mu\nu}+8\pi G \delta T_{\mu\nu}
&=&\frac{1}{2}[\nabla_{\mu}\nabla_{\nu}h^{T}
+R^{\sigma}_{\phantom{\sigma}\mu}h_{\sigma\nu}^{T}
+R^{\sigma}_{\phantom{\sigma}\nu}h_{\sigma\mu}^{T}
-2R_{\mu\lambda\nu\sigma}h^{T\lambda\sigma}
+\nabla_{\alpha}\nabla^{\alpha}h_{\mu\nu}^{T}]
\nonumber \\
&-&\frac{1}{2}R^{\sigma}_{\phantom{\sigma}\sigma}h^{T}_{\mu\nu}
+\frac{1}{ 2}g_{\mu\nu}R_{\alpha\beta}h^{T\alpha\beta}
-\frac{1}{2}g_{\mu\nu}\nabla_{\alpha}\nabla^{\alpha}h^{T}
+8\pi G \delta T^T_{\mu\nu}=0.
\nonumber\\
\label{A.35a}
\end{eqnarray}
% 
The SVT4 fluctuations around a de Sitter background as given in (\ref{6.16}) to (\ref{6.19}) and around a general Robertson-Walker background as given in (\ref{12.9}) are special cases of (\ref{A.35a}), with the only metric fluctuations that appear in (\ref{6.19}) and (\ref{12.9}) being $F_{\mu\nu}$ and $\chi$, viz. just the six degrees of freedom associated with $h^T_{\mu\nu}$.


%%%%%%%%%%%%%%%%%%%%%%%%%%%%%%%%%%%%%%%%%%%%
\section{D-dimensional SVTD Transverse-Traceless Projection Operators for Curved Spacetime Tensor Fields}
\label{aas:tt_curved_proj}
%%%%%%%%%%%%%%%%%%%%%%%%%%%%%%%%%%%%%%%%%%%%

Rather than generalize the general curved spacetime transverse and longitudinal projection technique to the general transverse-traceless case, we have instead  found it more convenient to generalize the SVTD discussion given in Secs. \ref{s:svtd} and \ref{ss:ds4_svt4} to general curved spacetime background fluctuations. To this end we take $h_{\mu\nu}$ to be of the form:
%
\begin{eqnarray}
h_{\mu\nu} &=& 2F_{\mu\nu}+W_{\mu\nu}+S_{\mu\nu},
\label{A.36a}
\end{eqnarray}
%
where
%
\begin{align}
W_{\mu\nu} &=\nabla_\mu W_\nu + \nabla_\nu W_\mu - \frac{2}{D}g_{\mu\nu}\nabla^\alpha W_\alpha,
\nonumber\\
S_{\mu\nu}&=\frac{1}{D-1}\left( g_{\mu\nu}\nabla_\alpha \nabla^\alpha - \nabla_\mu\nabla_\nu\right)\int d^Dx^{\prime}[-g(x^{\prime})]^{1/2}D^{(D)}(x,x^{\prime}) h(x^{\prime}),
\label{A.37a}
\end{align}
%
with $D(x,x^{\prime})$ obeying
%
\begin{eqnarray}
\nabla_\alpha \nabla^\alpha D^{(D)}(x,x^{\prime}) =[-g(x)]^{-1/2}\delta^{(D)}(x-x^{\prime}).
\label{A.38a}
\end{eqnarray}
%
From (\ref{A.37a}) we obtain
%
\begin{eqnarray}
g^{\mu\nu}W_{\mu\nu}=0,\quad g^{\mu\nu}S_{\mu\nu}=h,
\label{A.39a}
\end{eqnarray}
%
%
\begin{eqnarray}
\nabla^\nu h_{\mu\nu} &=& \nabla^\nu W_{\mu\nu} + \nabla^\nu S_{\mu\nu}
\label{A.40a}
\end{eqnarray}
%
as the conditions that $F_{\mu\nu}$ be transverse and traceless. From (\ref{A.40a}) we obtain 
%
\begin{align}
\left[g_{\nu\alpha} \nabla_\beta \nabla^\beta +\nabla_\alpha \nabla_\nu - \frac{2}{D}\nabla_\nu\nabla_\alpha\right] W^\alpha
&=\nabla^\alpha h_{\alpha\nu} - \frac{1}{D-1}(\nabla_\nu \nabla_\alpha\nabla^\alpha 
\nonumber\\
&- \nabla_\alpha\nabla^\alpha \nabla_\nu)\times
\nonumber\\
&
\int d^Dx^{\prime}[-g(x^{\prime})]^{1/2} D^{(D)}(x,x^{\prime}) h(x^{\prime}),
\label{A.41a}
\end{align}
%
and by commuting derivatives can rewrite (\ref{A.41a}) as
%
\begin{align}
\left[g_{\nu\alpha}\nabla_\beta\nabla^\beta + \left(\frac{D-2}{D}\right)\nabla_\nu \nabla_\alpha - R_{\nu\alpha}\right]&W^\alpha
= \nabla^\alpha h_{\alpha\nu} - \frac{1}{D-1}R_{\nu\alpha}\nabla^\alpha \times
\nonumber\\
&\int d^Dx^{\prime}[-g(x^{\prime})]^{1/2}D^{(D)}(x,x^{\prime}) h(x^{\prime}).
\label{A.42a}
\end{align}
%

To solve for $W_{\mu}$ it is convenient to use the bitensor formalism in which we define $G_{\alpha}^{(D)\beta}(x,x^{\prime})=e^a_{\alpha}(x)e^{\beta}_a(x^{\prime})$ where the D-dimensional $e^a_{\alpha}(x)$ vierbeins obey $g_{\mu\nu}(x)=\eta_{ab}e^{a}_{\mu}(x)e^{b}_{\nu}(x)$, with $a$ and $b$ referring to a fixed D-dimensional basis. With this bitensor definition $e^a_{\alpha}(x)$ and $e^{\beta}_a(x^{\prime})$ are acting in separate spaces, but  at $x=x^{\prime}$ we obtain $G_{\alpha}^{(D)\beta}(x,x)=g_{\alpha}^{\phantom{\alpha}\beta}(x)$. On the introducing the propagator that satisfies 
%
\begin{eqnarray}
\left[g_{\nu\alpha}\nabla_\beta\nabla^\beta + \left(\frac{D-2}{D}\right)\nabla_\nu \nabla_\alpha - R_{\nu\alpha}\right]D_{(D)}^{\alpha\gamma}(x,x^{\prime}) &=& G_{\nu}^{(D)\gamma}(x,x^{\prime}) [-g(x^{\prime})]^{-1/2}\times
\nonumber\\
&& \delta^{(D)}(x-x^{\prime}),
\label{A.43a}
\end{eqnarray}
%
we can solve for $W_{\mu}$ as
%
\begin{eqnarray}
W_{\mu}(x) &=& \int d^Dx^{\prime}[-g(x^{\prime})]^{1/2} D_{\mu}^{(D)\sigma}(x,x^{\prime})\bigg[ \nabla^{\rho}_{x^{\prime}} h_{\sigma\rho}(x^{\prime})-
\frac{1}{D-1}R_{\sigma\rho}(x^{\prime})\nabla^{\rho}_{x^{\prime}} \times
\nonumber\\
&&\int d^Dx^{\prime\prime}[-g(x^{\prime\prime})]^{1/2} D^{(D)}(x^{\prime},x^{\prime\prime}) h(x^{\prime\prime})\bigg].
\label{A.44a}
\end{eqnarray}
%

Next we decompose $W_{\mu}$ into transverse and longitudinal components viz.
%
\begin{eqnarray}
W_{\mu} &=&W^T_{\mu}+W^L_{\mu}=F_{\mu}+\nabla_{\mu}H,\quad  \nabla^{\mu}F_{\mu}=0,
\nonumber\\
 H&=&\int d^Dx^{\prime}[-g(x^{\prime})]^{1/2}D^{(D)}(x,x^{\prime})\nabla^\sigma W_\sigma(x^{\prime}),
\label{A.45a}
\end{eqnarray}
%
with $h_{\mu\nu}$ then taking the form
%
\begin{align}
h_{\mu\nu}&= 2F_{\mu\nu} + \nabla_\mu F_\nu + \nabla_\nu F_\mu + 2 \nabla_\mu\nabla_\nu H - \frac{2}{D}g_{\mu\nu}\nabla_\alpha \nabla^\alpha H 
\nonumber\\
&+\frac{1}{D-1}\left( g_{\mu\nu}\nabla_\alpha \nabla^\alpha - \nabla_\mu\nabla_\nu\right)\int d^Dx^{\prime}[-g(x^{\prime})]^{1/2} D^{(D)}(x,x^{\prime}) h(x^{\prime}).
\label{A.46a}
\end{align}
%
Upon further defining
%
\begin{align}
F &= H - \frac{1}{2(D-1)} \int d^Dx^{\prime}[-g(x^{\prime})]^{1/2} D^{(D)}(x,x^{\prime}) h(x^{\prime}),
\nonumber\\
\chi &= \frac{1}{D}\nabla_\alpha\nabla^\alpha H - \frac{1}{2(D-1)}\nabla_\alpha\nabla^\alpha\int d^Dx^{\prime}[-g(x^{\prime})]^{1/2} D^{(D)}(x,x^{\prime}) h(x^{\prime}),
\label{A.47a}
\end{align}
%
we may express $h_{\mu\nu}$ in the SVTD form:
%
\begin{eqnarray}
h_{\mu\nu} &=& -2g_{\mu\nu}\chi + 2\nabla_\mu\nabla_\nu F + \nabla_\mu F_\nu + \nabla_\nu F_\mu + 2F_{\mu\nu}
\label{A.48a},
\end{eqnarray}
%
where
%
\begin{align}
\chi &= \frac{1}{D}\nabla^\sigma W_{\sigma}  - \frac{1}{2(D-1)}h,
\nonumber\\
F_{\mu} &= W_{\mu}^T=W_{\mu} -\nabla_\mu \int d^Dx^{\prime}[-g(x^{\prime})]^{1/2} D^{(D)}(x,x^{\prime})\nabla^{\sigma}W_\sigma(x^{\prime}),
\nonumber\\
F &= \int d^Dx^{\prime}[-g(x^{\prime})]^{1/2} D^{(D)}(x,x^{\prime}) \left(\nabla^\sigma W_{\sigma}(x^{\prime})  - \frac{1}{2(D-1)}h(x^{\prime})\right),
\nonumber\\
2F_{\mu\nu} &= h_{\mu\nu}+2g_{\mu\nu}\chi - 2\nabla_\mu\nabla_\nu F - \nabla_\mu F_\nu - \nabla_\nu F_{\mu}.
\label{A.49a}
\end{align}
%
We thus generalize the SVTD approach to the arbitrary D-dimensional curved spacetime background.




\chapter{Conformal to Flat Cosmological Geometries}
\label{ab:cosmologies}

%%%%%%%%%%%%%%%%%%%%%%%%%%%%%%%%%%%%%%%%%%%%
\section{Robertson-Walker $k=0$}
\label{abs:rw_k=0}
%%%%%%%%%%%%%%%%%%%%%%%%%%%%%%%%%%%%%%%%%%%%

In order to apply (\ref{AP61}) to cosmology we need to write the Robertson-Walker and de Sitter background geometries in a conformal to flat Minkowski form. For a $k=0$ Robertson-Walker background the comoving coordinate system metric takes the form
% 
\begin{eqnarray}
ds^2({\rm comoving})=dt^2-a^2(t)[dx^2+dy^2+dz^2].
\label{A}
\end{eqnarray}
%
The straightforward introduction of the conformal time
% 
\begin{eqnarray}
d\tau=\int \frac{dt}{a(t)}
\label{A2}
\end{eqnarray}
%
then allows us to write the conformal time metric as
% 
\begin{eqnarray}
ds^2({\rm conformal~time})=a^2(\tau)[d\tau^2-dx^2-dy^2-dz^2].
\label{A3}
\end{eqnarray}
%
%%%%%%%%%%%%%%%%%%%%%%%%%%%%%%%%%%%%%%%%%%%%
\section{Robertson-Walker $k>0$}
\label{abs:rw_klt0}
%%%%%%%%%%%%%%%%%%%%%%%%%%%%%%%%%%%%%%%%%%%%

For a $k>0$ or a $k<0$ Robertson-Walker background the comoving and conformal time coordinate system metrics take the form
% 
\begin{eqnarray}
ds^2({\rm comoving})&=&dt^2-a^2(t)\left[\frac{dr^2}{1-kr^2}+r^2d\theta^2+r^2\sin^2\theta d\phi^2\right],
\nonumber\\
ds^2({\rm conformal~time})&=&a^2(\tau)\left[d\tau^2-\frac{dr^2}{1-kr^2}-r^2d\theta^2-r^2\sin^2\theta d\phi^2\right].
\label{A4}
\end{eqnarray}
%


To bring the RW geometries with non-zero $k$ to a conformal to flat form requires coordinate transformations that involve both $\tau$ and $r$. For the $k>0$ case first, it is convenient to set $k=1/L^2$, and introduce $\sin \chi=r/L$, with the conformal time metric given in (\ref{A4}) then taking the form
%
\begin{eqnarray}
ds^2=L^2a^2(p)\left[dp^2-d\chi^2 -\sin^2\chi d\theta^2-\sin^2\chi \sin^2\theta d\phi^2\right],
\label{A5}
\end{eqnarray}
%
where $p=\tau/L$. Following e.g. \cite{mannheim_2012} we introduce
%
\begin{eqnarray}
p^{\prime}+r^{\prime}&=&\tan[(p+\chi)/2],\qquad p^{\prime}-r^{\prime}=\tan[(p-\chi)/2],
\nonumber\\
p^{\prime}&=&\frac{\sin p}{\cos p+\cos \chi},\qquad r^{\prime}=\frac{\sin \chi}{\cos p+\cos \chi},
\label{A6}
\end{eqnarray}
%
so that
%
\begin{eqnarray}
dp^{\prime 2}-dr^{\prime 2}&=&\frac{1}{4}[dp^2-d\chi^2]\sec^2[(p+\chi)/2]\sec^2[(p-\chi)/2],
\\
\frac{1}{4}(\cos p +\cos \chi)^2&=&\cos^2[(p+\chi)/2]\cos^2[(p-\chi)/2]
\nonumber\\
&=&\frac{1}{[1+(p^{\prime}+r^{\prime})^2][1+(p^{\prime}-r^{\prime})^2]}.
\label{A7}
\end{eqnarray}
%
With these transformations the $k>0$ line element then takes the conformal to flat form
%
\begin{eqnarray}
ds^2=\frac{4L^2a^2(p)}{[1+(p^{\prime}+r^{\prime})^2][1+(p^{\prime}-r^{\prime})^2]}\left[dp^{\prime 2}-dr^{\prime 2} -r^{\prime 2}d\theta^2-r^{\prime 2} \sin^2\theta d\phi^2\right].
\label{A8}
\end{eqnarray}
%
To bring the spatial sector  of (\ref{A8}) to Cartesian coordinates we set  $x^{\prime}=r^{\prime}\sin\theta\cos\phi$, $y^{\prime}=r^{\prime}\sin\theta\sin\phi$, $z^{\prime}=r^{\prime}\cos\theta$ and thus bring the line element to the form  
%
\begin{eqnarray}
ds^2=L^2a^2(p)(\cos p+\cos \chi)^2\left[dp^{\prime 2}-dx^{\prime 2} -dy^{\prime 2} -dz^{\prime 2} \right],
\label{A9}
\end{eqnarray}
%
where now $r^{\prime}=(x^{\prime 2}+ y^{\prime 2}+z^{\prime 2})^{1/2}$. With these transformations (\ref{A9}) is now in the form given in (\ref{AP6}).

%%%%%%%%%%%%%%%%%%%%%%%%%%%%%%%%%%%%%%%%%%%%
\section{Robertson-Walker $k<0$}
\label{abs:rw_kgt0}
%%%%%%%%%%%%%%%%%%%%%%%%%%%%%%%%%%%%%%%%%%%%


For the $k<0$ case, it is convenient to set $k=-1/L^2$, and introduce ${\rm sinh} \chi=r/L$, with the conformal time metric given in (\ref{A4}) then taking the form
%
\begin{eqnarray}
ds^2=L^2a^2(p)\left[dp^2-d\chi^2 -{\rm sinh}^2\chi d\theta^2-{\rm sinh}^2\chi \sin^2\theta d\phi^2\right],
\label{A10}
\end{eqnarray}
%
where $p=\tau/L$. Next we introduce
%
\begin{eqnarray}
p^{\prime}+r^{\prime}&=&\tanh[(p+\chi)/2],\qquad p^{\prime}-r^{\prime}=\tanh[(p-\chi)/2],
\nonumber\\
 p^{\prime}&=&\frac{\sinh p}{\cosh p+\cosh \chi},\qquad r^{\prime}=\frac{\sinh \chi}{\cosh p+\cosh \chi},
\label{A11}
\end{eqnarray}
%
so that
%
\begin{eqnarray}
dp^{\prime 2}-dr^{\prime 2}&=&\frac{1}{4}[dp^2-d\chi^2]{\rm sech}^2[(p+\chi)/2]{\rm sech}^2[(p-\chi)/2],
\nonumber\\
\frac{1}{4}(\cosh p+\cosh \chi)^2&=&{\rm \cosh}^2[(p+\chi)/2]{\rm \cosh}^2[(p-\chi)/2]
\nonumber\\
&=&\frac{1}{[1-(p^{\prime}+r^{\prime})^2][1-(p^{\prime}-r^{\prime})^2]}.
\label{A12}
\end{eqnarray}
%
With these transformations the line element takes the conformal to flat form
%
\begin{eqnarray}
ds^2=\frac{4L^2a^2(p)}{[1-(p^{\prime}+r^{\prime})^2][1-(p^{\prime}-r^{\prime})^2]}\left[dp^{\prime 2}-dr^{\prime 2} -r^{\prime 2}d\theta^2-r^{\prime 2} \sin^2\theta d\phi^2\right].
\label{A13}
\end{eqnarray}
%
The spatial sector can then be written in Cartesian form
%
\begin{eqnarray}
ds^2=L^2a^2(p)(\cosh p+\cosh \chi)^2\left[dp^{\prime 2}-dx^{\prime 2} -dy^{\prime 2} -dz^{\prime 2}\right],
\label{A14}
\end{eqnarray}
%
where again $r^{\prime}=(x^{\prime 2}+ y^{\prime 2}+z^{\prime 2})^{1/2}$.  We note that in transforming from (\ref{A4}) to (\ref{A9}) or to (\ref{A14}) we have only made coordinate transformations and not made any conformal transformation. 

%%%%%%%%%%%%%%%%%%%%%%%%%%%%%%%%%%%%%%%%%%%%
\section{$dS_4$ and $AdS_4$ Background Solutions}
\label{abs:ds4}
%%%%%%%%%%%%%%%%%%%%%%%%%%%%%%%%%%%%%%%%%%%%

While the conformal to flat Minkowski structures given in (\ref{A3}), (\ref{A9}) and (\ref{A14}) are purely kinematical, the explicit form of $a(t)$ can be determined once a dynamics has been specified. Thus in regard to a de Sitter or anti-de Sitter cosmology, a de Sitter or an anti-de Sitter geometry is  just a particular case of a Robertson-Walker geometry in which $a(t)$ has a specific assigned value for each possible choice of  spatial 3-curvature $k$. On writing the  maximally 4-symmetric geometry condition $R_{\mu\nu}=-3\alpha g_{\mu\nu}$ in Robertson-Walker form one obtains 
%
\begin{eqnarray} 
\dot{a}^2(t) +k=\alpha  a^2(t).
\label{A15}
\end{eqnarray}
%
(In terms of the scalar field model described in (\ref{AP32}) -- (\ref{AP35}) we have $K=\alpha =-2\lambda_{S}S^2_0$.) Here $\alpha$ is positive for de Sitter and negative for anti-de Sitter. Allowable solutions to (\ref{A15}) depend on the values of $\alpha$ and $k$, and are of the form (see e.g. \cite{mannheim_2006})
%
\begin{eqnarray}
a(t,\alpha>0,k<0)&=&\left(-\frac{k}{\alpha}\right)^{1/2}
\sinh(\alpha^{1/2}t),
\nonumber \\
a(t,\alpha>0,k=0)&=&a(t=0)\exp(\alpha^{1/2}t),
\nonumber \\
a(t,\alpha>0,k>0)&=&\left(\frac{k}{\alpha}\right)^{1/2}\cosh(\alpha^{1/2}t),
\nonumber \\
a(t,\alpha=0,k<0)&=&(-k)^{1/2}t,
\nonumber \\
a(t,\alpha<0,k<0)&=&\left(\frac{k}{\alpha}\right)^{1/2}\sin((-\alpha)^{1/2}t).
\label{A16}
\end{eqnarray}
%
In these solutions (\ref{A3}), (\ref{A9}), and (\ref{A14}) all apply  to a de Sitter or an anti-de Sitter cosmology.

%%%%%%%%%%%%%%%%%%%%%%%%%%%%%%%%%%%%%%%%%%%%
\section{$dS_4$ and $AdS_4$ Background Solutions - Radiation Era}
\label{abs:ds4_ads4_radiation}
%%%%%%%%%%%%%%%%%%%%%%%%%%%%%%%%%%%%%%%%%%%%

For Robertson-Walker cosmologies we note that with slight modification we can extend the scalar field model given above to include a perfect fluid, with the energy-momentum tensor then being given by \cite{mannheim_1998}
%                                                                               
\begin{eqnarray}
T_{\rm S}^{\mu \nu}&=&(\rho+p)U_{\mu}U_{\nu}+pg_{\mu\nu} 
-\frac{1}{6} S_0^2\left(R^{\mu\nu}-\frac{1}{2}g^{\mu\nu}
R^\alpha_{\phantom{\alpha}\alpha}\right)-g^{\mu\nu}\lambda_S S_0^4,
\label{A17}
\end{eqnarray}                                 
%
with the background conformal cosmology still obeying $T_{\rm S}^{\mu \nu}=0$ since the background  Robertson-Walker geometry continues to obey $W_{\mu\nu}=0$.  On taking the perfect fluid energy-momentum tensor to be traceless radiation  (viz. $\rho=3p$, $\rho=A/a^4(t)$, $A>0$) as needed in the early universe, and with $\alpha =-2\lambda_{S}S^2_0$ as before, the evolution equation takes the form
%                                                                               
\begin{eqnarray}
\dot{a}^2+k&=&\alpha a^2-\frac{2A}{S_0^2a^2},
\label{A18}
\end{eqnarray}                                 
% 
with allowed solutions to the cosmology being given by  \cite{mannheim_1998} 
%
\begin{eqnarray}
a(t,\alpha>0,k<0,A>0)&=&\left(-\frac{k(\beta-1)}{2\alpha}-\frac{k\beta}{\alpha}\sinh^2(\alpha^{1/2}t)\right)^{1/2},
\nonumber \\
a(t,\alpha>0,k=0,A>0)&=&\left(-\frac{A}{\lambda_S S_0^4}\right)^{1/4}\cosh^{1/2}(2\alpha^{1/2}t),
\nonumber \\
a(t,\alpha>0,k>0,A>0)&=&\left(\frac{-k(\beta-1)}{2\alpha}+\frac{k\beta}{\alpha}\cosh^2(\alpha^{1/2}t)\right)^{1/2},
\nonumber\\
a(t,\alpha=0,k<0,A>0)&=&\left(-\frac{2A}{kS_0^2}-kt^2\right)^{1/2},
\nonumber \\
a(t,\alpha<0,k<0,A>0)&=&\left(-\frac{k(\beta-1)}{2\alpha}+\frac{k\beta}{\alpha}\sin^2((-\alpha)^{1/2}t)\right)^{1/2},
\label{A19}
\end{eqnarray}
%
where $\beta=(1+8A\alpha/k^2S_0^2)^{1/2}$.

%\input{Chapters/Table.tex}

\end{document}






