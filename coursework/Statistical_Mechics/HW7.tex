\documentclass[11pt,letterpaper]{article}
\usepackage{mymacros}

\title{\begin{spacing}{1.2}Statistical Mechanics\\HW 7\end{spacing}}
\author{Matthew Phelps}
\date{Due: Oct. 28}

\begin{document}
\maketitle

\benum
% 7.1 ----------------------------------------------------------------------------------------------------------------------
  	\item[\textbf{7.1}]
	Prove the relationship for fermions that we wrote down in class:
	\[
		U = \sum_i \braket{n_i}\epsilon_i = \frac{3}{2}\frac{gVk_bT}{\lambda_{dB}^3}f_{\frac{5}{2}}(z)
	\]
	What is the explicit relationship with the grand potential? How would the same problem look like
	for bosons?
	\\
	\\
	The internal energy can be computed with 
	\[	
		E = \sum_i \braket{n_i}\epsilon_i.
	\]
	The eigenstates of the free Hamiltonian are momentum eigenstates. These can be normalized
	via a box normalized. In three dimensions, the possible values of momentum are quantized
	\[
		\vect K = \frac{2\pi}{L}(k_1\vecth e_1+ k_2 \vecth e_2+k_3 \vecth e_3)
	\]
	where $k_1,k_2,k_3 = 0,\pm 1,\pm 2..$. 
	The energies are then
	\[
		\epsilon_k = \frac{\h^2|k|^2}{2m}.
	\]
	Since the fermions have spin, the eigenstates are labeled by wavevector $\vect k$ and spin $m_s$
	\[
		\ket{i} = \ket{\vect k,m_s}.
	\]
	Since the spectrum is degenerate we have for degeneracy $g$
	\[
		E = \sum_i \braket{n_i}\epsilon_i = g\sum_\vect k \braket{n_i(\epsilon_k)}\epsilon_k
	\]
	Next we take the continuum limit of possible momentum eignstates
	\[
		E = \frac{gV}{(2\pi)^3}\int d^3k\ \frac{\epsilon_k}{z^{-1}e^{\beta\epsilon_k}+1}
	\]
	where we have used for fermions
	\[
		\braket{n_i} = \frac{1}{e^{\beta(\epsilon_i-\mu)}+1}.
	\]
	Putting in the explicit for of $\epsilon_k$ and doing a change of variable we arrive at
	\ba
		E &= \frac{gV}{2\pi^2}\frac{\h^2}{2m}
		\pfrac{2mkT}{\h^2}^{5/2} \int dx\ x^4 \frac{ze^{-x^2}}{1+ze^{-x^2}} \\
		 & = \frac{gV}{2\pi^2}\frac{\h^2}{2m}
		\pfrac{2mkT}{\h^2}^{5/2} \int dx\ x^4 \plr{ze^{-x^2}-z^2e^{-2x^2}+z^3e^{-3x^2}}
	\ea
	For the Gaussian
	\[
		\int_0^\infty dx\ x^4 e^{-\lambda x^2} = \frac{3\sqrt\pi}{8\lambda^{5/2}}.
	\]
	This may be used to construct a function for the integral, namely
	\[
		f_\alpha(z)=\sum_{l=1}^\infty \frac{(-)^{l+1}z^l}{l^\alpha}.
	\]
	We can now express the whole integral as
	\[
		E = \frac{3}{2}\frac{gVk_bT}{\lambda_{dB}^3}f_{\frac{5}{2}}(z)
	\]
	where $\lambda = \sqrt{\frac{2\pi\h^2}{mkT}}$.
	\\
	\\
	The grand potential is given as $\Omega = -kT\ln \mathcal Z$ and for fermions there are only
	two occupation numbers 
	$\mathcal Z = 1+ze^{-\beta\epsilon_i}$. Thus
	\[
		\ln\mathcal Z = \sum_i \ln(1+ze^{-\beta\epsilon_i}) = \frac{gV}{(2\pi)^3}\int d^3k\ 
		\ln(1+ze^{\beta\epsilon_k}).
	\]
	We can expand this natural log as
	\[
		\ln\mathcal Z = \frac{gV}{2\pi^2}\int dk\  k^2(ze^{-
		\beta\epsilon_k}-\frac{z^2}{2}e^{-2\beta\epsilon_k}+\frac{z^3}{3}e^{-3\beta\epsilon_k}-...)
	\]
	This time the Gaussian is
	\[
		\int_0^\infty dx\ x^2e^{-\lambda x^2} = \frac{\sqrt\pi}{4\lambda^{3/2}}
	\]
	and so we have the same special function as the energy
	\[
		\ln\mathcal Z = \frac{gV}{\lambda^3}f_{5/2}(z).
	\]
	Thus the grand potential is
	\[
		\Omega = -\frac{gVkT}{\lambda^3}f_{5/2}(z).
	\]
	Looking back at the energy we see that 
	\[
		\Omega = -\frac{2}{3} U
	\]
	\\
	\\
	For Bosons we must use the occupation number
	\[
		\braket{n_i} = \frac{1}{e^{\beta(\epsilon_i-\mu)}-1}.
	\]
	The same analysis proceeds up to the integration with a slightly different expansion
	\ba
		E &= \frac{gV}{(2\pi)^2}\frac{\h^2}{2m}
		\pfrac{2mkT}{\h^2}^{5/2} \int dx\ x^4 \frac{ze^{-x^2}}{1-ze^{-x^2}} \\
		 & = \frac{gV}{(2\pi)^2}\frac{\h^2}{2m}
		\pfrac{2mkT}{\h^2}^{5/2} \int dx\ x^4 \plr{ze^{-x^2}+z^2e^{-2x^2}-z^3e^{-3x^2}}.
	\ea
	For Bosons the special function is
	\[
		g_\alpha(z) = \sum_{l=1}^\infty \frac{z^l}{l^\alpha}
	\]
	and the energy is thus
	\[
		E = \frac{3}{2}\frac{gVk_bT}{\lambda_{dB}^3}g_{\frac{5}{2}}(z).
	\]
	The grand partition function for bosons goes as
	\[
		\ln\mathcal Z = \sum_i \ln(1-ze^{-\beta\epsilon_i}) .
	\]
	Hence we may propagate the minus sign only to find the same result
	\[
		\Omega = -\frac{2}{3} U
	\]
	\\
	\\
% 7.2 -----------------------------------------------------------------------------------------------------------------------
	\item[\textbf{7.2}]
	Take a system of $n$ adsorption sites, each of which may bind a molecule: putting a molecule to a site lowers the 
	energy by $\epsilon$ compared to the energy of a molecule floating around in the environment. 
	\\
	\benum
	\item
	Given an environment where the molecules have temperature $T$ and chemical potential $\mu$, the 
	grand partition
	function for the molecules of the lattice gas is
	\[
		\mathcal Z = (1+e^{\beta(\epsilon+\mu)})^n
	\]
	Why?
	\\
	\\
	Each adsorption site is distinguishable and it may only take e may take on are $0$ and $-\epsilon$.
	The grand partition function is given as
	\[
		\mathcal Z= \sum_N z^NZ_N.
	\]
	For the canonical partition function, we may treat each site independently such that the
	partition function factorizes:
	\[	
		Z = \prod_n Z_n = (1+e^{\beta \epsilon})^n.
	\]
	So
	\[
		\mathcal Z = \sum_N z^N(1+e^{\beta\epsilon})^n.
	\]
    	
	\[
		\mathcal Z = \prod_n (1+e^{\beta( \epsilon+\mu)}) = (1+e^{\beta( \epsilon+\mu)})^n
	\]
	\item
	Show that the probability that a site is occupied by a molecule is $p = (1+e^{-\beta(\epsilon+\mu)})^{-1}$
	\\
	\\
	We know the probability that the system is in specified state with energy $E_n$ in a canonical
	ensemble is given by
	\[
		p_n = \frac{1}{Z}e^{-\beta E_n}.
	\]
	I will assume the same form for the grand canonical ensemble; however, to find the probability that 
	only a single site is occupied, we must use the grand partition function 
	of the single independent system, $\mathcal Z_i = (1+e^{\beta(\epsilon+\mu)})$.  
	Thus
	\[
		p = \frac{e^{-\beta(-\epsilon-\mu)}}{\mathcal Z_i} 
		= \frac{e^{\beta(\epsilon+\mu)}}{1+e^{\beta(\epsilon+\mu)}} = \frac{1}{1+e^{-\beta(\epsilon+\mu)}}
	\]
	Therefore
	\[
		p = (1+e^{-\beta(\epsilon+\mu)})^{-1}
	\]
	\\
	
	\eenum
% 7.3 ------------------------------------------------------------------------------------------------------------------------
	\item[\textbf{7.3}]
	A system consists of two particles, each of which has two possible quantum states with energies
	$\pm \epsilon/2$. Write down the canonical partition function if (a) the particles are distinguishable
	; (b) the particles are bosons; (c) the particles are fermions.
	\\
	\\
	\benum
	\item
	For two distinguishable particles, the partition function factorizes as
	\[
		Z = Z_1Z_2 = \plr{e^{-\beta \epsilon/2}+e^{\beta \epsilon/2}}^2. 
	\] \\
	
	\item
	For bosons, the particles are indistinguishable so it is the sum of all 4 distinct microstates
	\[
		Z = e^{-\beta\epsilon}+1+e^{\beta\epsilon}.
	\]\\
	
	\item
	For fermions, the particles are distinguishable and cannot occupy the same state
	\[
		Z = 1+1 = 2
	\]
	\\
	\eenum
% 7.4 ------------------------------------------------------------------------------------------------------------------------
	\item[\textbf{7.4}]
	Calculate the integrals
	\[
		I_1 = \int_0^\infty dx\ x^n e^{\lambda x},\qquad I_2= \int_0^\infty dx\ x^ne^{-\lambda x^2}
	\]
	For the purpose of this exercise, it is enough to find an algorithm for the values of the integrals in terms of 
	$n = 0,1,2,...$ assuming $\lambda >0$. The results are good, however, for complex $\lambda$ with 
	$Re(\lambda)>0$. 
	\\
	\\
	For $n=0$ 
	\[
		I_1 = \frac{1}{\lambda}\qquad I_2 = \frac{\sqrt \pi}{2\sqrt \lambda}
	\]
	$n=1$
	\[
		I_1 = \frac{1}{\lambda^2}\qquad I_2 = \frac{1}{2 \lambda}
	\]
	$n=2$
	\[
		I_1 = \frac{2}{\lambda^3}\qquad I_2= \frac{\sqrt \pi}{4\lambda^{3/2}}
	\]	
	$n=3$
	\[
		I_1 = \frac{6}{\lambda^4}\qquad  I_2= \frac{1}{2\lambda^2}
	\]
	$n=4$
	\[
		I_1 = \frac{24}{\lambda^5}\qquad  I_2= \frac{3\sqrt\pi}{8\lambda^{5/2}}.
	\]
	Looking at these results and continuing onward we find that $I_2$ actually has the same dependence 
	as the gamma function, where $\Gamma(1/2) =\sqrt\pi$. 
	\[
		I_1 = \frac{n!}{\lambda^{n+1}}
	\]
	and
	\[
		I_2= \frac{1}{2}\lambda^{\frac{-n+1}{2}}\Gamma\pfrac{n+1}{2}.
	\]
	\\
% 7.5 -------------------------------------------------------------------------------------------------------------------------
	\item[\textbf{7.5}]
	
	Show (e.g., by starting from the grand partition function) that the entropies of Bose-Einstein and Fermi-	Dirac
	ideal gases are
	\[
		S = -k_b\sum_i\blr{\braket{n_i}\ln\braket{n_i}\mp(1\pm\braket{n_i})\ln(1\pm\braket{n_i})}
	\]
	\\
	\\
	We know that
	\[
		\Omega = -kT\ln\mathcal Z = 
	\]
	and that
	\[
		-\plr{\pdiff[\Omega]{T}}_{\mu,V} = S.
	\]
	So
	\[
		S = k\ln \mathcal Z +kT\pdiff[(\ln\mathcal Z)]{T}
	\]
	Also,
	\[
		\pdiff[]{T} = \pdiff[\beta]{T}\pdiff[]{\beta}\quad\to\quad \pdiff[]{T} = -\frac{\beta}{T}\pdiff[]{\beta}.
	\]
	Forming $S$, 
	\ba
	S &= k\sum_i \mp\ln(1\pm ze^{-\beta\epsilon_i})+k\beta \pdiff[]{\beta}\sum_i \pm\ln(1\pm ze^{-\beta
	\epsilon_i})\\
	& = k\sum_i \mp\ln(1\pm ze^{-\beta\epsilon_i})+
	k\beta\sum_i \frac{\pm(\epsilon_i-\mu)}{1\pm ze^{-\beta\epsilon_i}}\\
	& = k\sum_i \mp\ln(1\pm ze^{-\beta\epsilon_i})+
	k\sum_i \frac{\pm\ln(1/\braket{n_i}-1)}{1\pm ze^{-\beta\epsilon_i}}
	\ea
	Continuing to use the occupation number for fermions
	\[
		\braket{n_i} = \frac{1}{e^{\beta(\epsilon_i-\mu)}+1}.
	\]
	and bosons
	\[
		\braket{n_i} = \frac{1}{e^{\beta(\epsilon_i-\mu)}-1}.
	\]
	we can finally arrive at
	\[
		S = -k_b\sum_i\blr{\braket{n_i}\ln\braket{n_i}\mp(1\pm\braket{n_i})\ln(1\pm\braket{n_i})}.
	\]
	
\eenum
\end{document}