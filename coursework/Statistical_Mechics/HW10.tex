\documentclass[11pt,letterpaper]{article}
\usepackage{macroshw}

\title{\begin{spacing}{1.2}Statistical Mechanics\\HW 10\end{spacing}}
\author{Matthew Phelps}
\date{Due: Dec. 2}

\begin{document}
\maketitle

\benum
% 10.1 ----------------------------------------------------------------------------------------------------------------------
  	\item[\textbf{10.1}]
	Read (and work through) the remainder of chapter 7 in the manuscript. This is, basically after
	7.3, everything after equation (7.62). We will discuss this next class and I will assume you know it
	(modulo asking questions during next class).
 	\\
	
% 10.2 -----------------------------------------------------------------------------------------------------------------------
	\item[\textbf{10.2}]
	(7.10 in script)
	\benum
		% (a)
		\item
		Consider a simple chemical reaction $A+B \leftrightarrow AB$. As the double arrow indicates,
		all the chemical reactions go both ways. Show that in a mixture of reactants  the condition for 
		chemical equilibrium is expressed in terms of chemical potentials as $\mu_A+\mu_B = \mu_{AB}$.
		\\ \\
		In thermodynamic equilibrium, any variation of the total entropy must be zero. The
		total entropy is the entropy of the reactants and products
		\[
			\delta S = \delta S_A+\delta S_B +\delta S_{AB}.
		\]
		However we also know that 
		\[
			dN_A+dN_B = d_{N_{AB}} \equiv dN.
		\]
		Assuming the variation in entropy comes solely from the change in particle number, we have
		\[
			\delta S = \plr{\pdiff[S_A]{N} +\pdiff[S_B]{N} - \pdiff[S_{AB}]{N}}dN = 0
		\]
		and so our criteria for chemical equilibrium is
		\[
			 \pdiff[S_A]{N} +\pdiff[S_B]{N} - \pdiff[S_{AB}]{N} = 0
		\]
		However we also have the maxwell relation
		\[
			\plr{\pdiff[S]{N}}_{U,V} = \mu
		\]
		and so finally, for chemical equilibrium we have
		\[
			\mu_A+\mu_B = \mu_{AB}.
		\]
		% (b)
		\item
		Show that in an ideal gas where the only relevant degree of freedom is the center-of-mass
		motion of the atoms or molecules, the chemical potential in the classical (high temperature/
		low density) limit is $\mu = kT\ln(n\lambda^3)$, where $\lambda = (2\pi\h^2/mkT)^{1/2}$ is 
		the usual thermal de Broglie wavelength. 
		\\ \\
		With the center of mass motion being the only relevant degree of freedom, we have just a free ideal 
		gas with 
		\[
			\mathcal H = \sum_i^N \frac{\vect p^2}{2m}. 
		\]
		We then compute the partition function, as in the script, as 
		\[
			Z = \frac{1}{N!}\pfrac{V}{\lambda^3}^N
		\]
		with the deBroglie wavelength $\lambda$ defined as in the problem. Now we can make a 
		connection to thermodynamics using $F = -kT\ln Z$. Hence
		\[
			F = -kT\ln\blr{ \frac{1}{N!}\pfrac{V}{\lambda^3}^N} \simeq-NkT\plr{\ln\frac{V}{\lambda^3}
			-\ln N+1}.
		\]
		To find the chemical potential, we recall
		\[
			\mu = \plr{\pdiff[F]{N}}_{T,V}
		\]
		and so
		\[
			\mu = -kT\ln\frac{V}{\lambda^3}+\ln N+1 = kT\ln\plr{n\lambda^3}
		\]
		\\ \\
		% (c)
		\item
		Consider the formation of diatomic molecules $A_2$ out of atoms $A$. Assume that the binding
		energy of the molecule is $I$, i.e., the difference in internal energy between a molecule and a pair
		of atoms is $-I$. Show that in the limit when the atoms and molecules may be regarded as classical
		ideal gasses, except for the formation of the molecules, the equilibrium density of the atoms and
		molecules satisfy $n_{A_2}/n^2_{A} = \sqrt 8 \lambda_A^3 e^{\frac{I}{kT}}$. The chemical
		equilibrium density of atoms and molecules depend manifestly on the de Broglie wavelength, 
		i.e., on quantum mechanics, even at room temperature!
		\\
		\\
		Based on part (a), the condition for chemical equilibrium is
		\[
			2\mu_A = \mu_{A_2}.
		\]
		We know from (b) that
		\[
			\mu_A =  kT\ln\plr{n_A\lambda_A^3}
		\]
		and
		\[
			\mu_{A_2} =  kT\ln\plr{n_{A_2}\lambda_{A_2}^3}
		\]
		or
		\[
			\exp\plr{\frac{\mu_A}{kT}} = n_A\lambda_A^3.
		\]
		and
		\[
			\exp\plr{\frac{\mu_{A_2}}{kT}} = n_{A_2}\lambda_{A_2}^3.
		\]
		If we form the ratio of equilibrium densities
		\[
			\frac{n_{A_2}}{n^2_A} = \frac{\lambda_A^6}{\lambda_{A_2}^3}
			\exp\plr{\frac{\mu_{A_2}-2\mu_A}{kT}}.
		\]
		Perhaps if we ignore the formation of molecules, then the condition of part (a) does not 
		apply and we may write the difference in chemical potential as 
		\[
			\mu_{A_2} - 2\mu_A = I\quad (?)
		\]
		and thus
		\[
			\frac{n_{A_2}}{n^2_A} =  \frac{\lambda_A^6}{\lambda_{A_2}^3}\exp\plr{\frac{I}{kT}}.
		\]
		If we can show that
		\[
			\lambda_A =\sqrt 2 \lambda_{A_2}\quad (?)
		\]
		(maybe from the partition function and that two atoms make up one molecule)
		we can show that
		\[
			\frac{n_{A_2}}{n^2_{A}} = \sqrt 8 \lambda_A^3 e^{\frac{I}{kT}}.
		\]
	\eenum 
% 10.3 ------------------------------------------------------------------------------------------------------------------------
	\item[\textbf{10.3}]
	(7.14 in script)
	\\ \\
	Study a trap for atoms. We assume that the gas is effectively homogeneous along the trap axis,
	so that the direction along the trap axis may be separated out and we have a two-dimensional 
	situation. The trapping potential is the rotationally symmetric
	 $V(\vect x) = \frac{1}{2}m\omega^2(x_1^2+x_2^2)$. Let us introduce the 3 component of the
	 angular momentum $L_3 = x_1p_2-x_2p_1$, and write
	 \[
	 	f(\vect x,\vect p) = K\exp\blr{-\frac{1}{kT}\plr{\frac{\vect p^2}{2m}+
		\frac{m\omega^2\vect x^2}{2}-\Omega L_3}}
	\]
	with some constant $\Omega$.
	
	\benum
		% (a)
		\item
            	Show that the convective derivative acting on $f(\vect x,\vect p)$ give zero. Since everything inside
		the exponential is conserved locally in a collision, $f(\vect x,\vect p)$ is therefore also a 
		stationary solution to the Boltzmann equation.
            	\\
		\\
		\\
		Convective derivative on phase space density:
		\[
			\plr{\pdiff{t} +\vect v\cdot \pdiff{\vect r}+\vect F\cdot \pdiff{\vect p}}f(\vect r,\vect p,t) = 0.
		\]
		In this particular trap, there is no explicit time dependence. Also, we have
		\[
			\vect F = -\del V(x_1,x_2) = -m\omega^2(x_1\vecth x_1+x_2\vecth x_2).
		\]
		So
		\[
			\frac{\vect p}{m} \cdot\pdiff{\vect r}f(\vect r,\vect p) = \del V(x_1,x_2)\cdot \pdiff{\vect p}
			f(\vect r,\vect p)\tag{1}.
		\]
		For the position derivative
		\[
			\pdiff{\vect r}f(\vect r,\vect p) = -\blr{ \plr{ m\omega^2 x_1-\Omega p_2}\vecth x_1+
			 \plr{ m\omega^2 x_2+\Omega p_1}\vecth x_2}\frac{f(\vect r,\vect p)}{kT}
		\]
		and momentum derivative
		\[
			\pdiff{\vect p}f(\vect r,\vect p) = -\blr{ \plr{ \frac{p_1}{m}+\Omega x_2}\vecth x_1 + 
			\plr{ \frac{p_2}{m}-\Omega x_1}\vecth x_2 }\frac{f(\vect r,\vect p)}{kT}.
		\]
		Now forming (1)
		\ba
			\plr{ p_1\omega^2 x_1-\Omega\frac{p_1p_2}{m}} + 
			\plr{ p_2\omega^2 x_2 + \Omega\frac{p_1p_2}{m}} &= 
			\plr{ \omega^2 p_1x_1+m\omega^2\Omega x_1x_2} +
			\plr{ \omega^2 x_2p_2-m\omega^2\Omega x_1x_2} \\
			\omega^2(x_1p_1+x_2p_2) &= \omega^2(x_1p_1+x_2p_2).
		\ea
		Thus the convective derivative is zero.
		\\
		\\
		% (b)
		\item
		Find the local flow velocity 
		\[
			\vect u(\vect x) = \frac{\int d^2p\ f(\vect x,\vect p)(\vect p/m)}
			{\int d^2p\ f(\vect x,\vect p)},
		\]
		and explain what the constant $\Omega$ means.
		\\ \\
		Finding the average velocity (without yet normalizing)
		\[
			\braket{\vect v} = \int_{-\infty}^{\infty} dp_1\ \int_{-\infty}^{\infty} dp_2\ 
			\exp\blr{-\frac{1}{kT}\plr{\frac{p_1^2+p_2^2}{2m}+
			\frac{m\omega^2(x_1^2+x_2^2)}{2}-\Omega(x_1p_2-x_2p_1)}}\frac{1}{m}
			(p_1\vecth x_1+p_2\vecth x_2)
		\]
		The integrals to be evaluated are of the form
		\ba
			I_1 &= \int_{-\infty}^{\infty} dx\ \exp\blr{-\plr{ Ax^2+Bx +C}}D \\
			& = D \sqrt\frac{\pi}{A} \exp\plr{\frac{B^2}{4A}-C}
		\ea
		and
		\ba
			I_2 &= \int_{-\infty}^{\infty} dx\ \exp\blr{-\plr{ Ax^2+Bx +C}}Dx \\
			& = -\frac{DB}{2}\sqrt\frac{\pi}{A^3}  \exp\plr{\frac{B^2}{4A}-C}.
		\ea
		First we integrate in the $\vecth x_1$ direction. The integral over $p_2$ is of the form $I_1$ 
		in which we get
		\[
			v_1(\vect r)= -2\pi mkT\Omega x_2 \exp\blr{-\frac{m}{2kT}(x_1^2+x_2^2)(\omega^2-		
			\Omega^2)}.
		\]
		The $\vecth x_2$ integral has the form of $I_2$ and we get something similar
		\[
			v_2(\vect r) = 2\pi mkT\Omega x_1 \exp\blr{-\frac{m}{2kT}(x_1^2+x_2^2)(\omega^2-		
			\Omega^2)}.
		\]
		As for the normalization
		\[
			\int d^2p\ f(\vect x,\vect p) = 2\pi mkT \exp\blr{-\frac{m}{2kT}(x_1^2+x_2^2)(\omega^2-	
			\Omega^2)}.
		\]
		Hence
		\[
			\vect u(\vect x) = \Omega\plr{ -x_2 \vecth x_1 + x_1\vecth x_2}
		\]
		\\ \\
		The factor of $I_3\Omega$ in the phase space density function (and thus in the Hamiltonian)
		represents rotational motion about the $\vecth x_3$ axis at frequency $\Omega$; it is the 
		phase space density in a rotating frame of reference. Thus, we see the local flow velocity is
		dependent on this rate of rotation. I would guess in this problem it represents a rotating cylinder
		of the trapped ions, i.e a centrifuge. \\ \\
            	% (c)
		\item 
            	How does the root-mean-square radius $r = \sqrt{\braket{\vect x^2}}$ of the gas behave with
		the parameter $\Omega$? Please explain. 
		\\
		\\
		If we find the expectation value of $\braket{\vect x^2}$ we have (without normalization)
		\ba
			\braket{\vect x^2} &= \int_{-\infty}^{\infty} dx_1\ \int_{-\infty}^{\infty} dx_2\ 
			\exp\blr{-\frac{1}{kT}\plr{\frac{p_1^2+p_2^2}{2m}+
			\frac{m\omega^2(x_1^2+x_2^2)}{2}-\Omega(x_1p_2-x_2p_1)}}(x_1^2+x_2^2) \\
			& = \frac{2\pi kT}{m^3\omega^6}(2kTm\omega^2+(p_1^2+p_2^2)\Omega^2)
			\exp\blr{-\frac{1}{2kTm\omega^2}(p_1^2+p_2^2)(\omega^2-\Omega^2)}.
		\ea
		With the normalization
		\[
			\int d^2x f(\vect x,\vect p) = \frac{2\pi kT}{m\omega^2}
			\exp\blr{-\frac{1}{2kTm\omega^2}(p_1^2+p_2^2)(\omega^2-\Omega^2)}
		\]
		we finally have
		\[
			\braket{\vect x^2} = \frac{2kTm\omega^2+(p_1^2+p_2^2)\Omega^2}{m^2\omega^4}
		\]
		or 
		\[
			r = \sqrt{2kTm\omega^2+(p_1^2+p_2^2)}\pfrac{\Omega}{m\omega^2}.
		\]
		The radius depends linearly on the angular velocity $\Omega$. As seen from a rotating 
		reference frame, a centrifugal force acts on the ions as
		\[
			F_c = \Omega\times(\Omega\times \vect r).
		\]
		We can see that the directions work out
		appropriately when we use the general form 
		\[
			\diff[\vect r]{t} = \plr{\diff[\vect r]{t}}_{rotation} + \vect \Omega\times \vect r.
		\]
		For our trap this equates to
		\[
			\Omega (\vecth x_3\times \vect r) = \Omega(-x_2\vecth x_1+x_1 \vecth x_2).
		\]
		The force is quadratic in $\Omega$, but I'm not sure how intuitively this translates to average position 
		(since the expectation position we found is linear in $\Omega$). 
            	\eenum
	\eenum
\end{document}