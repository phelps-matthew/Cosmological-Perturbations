\documentclass[11pt,letterpaper]{article}
\usepackage{macroshw}

\title{\begin{spacing}{1.2}Statistical Mechanics\\HW 11\end{spacing}}
\author{Matthew Phelps}
\date{Due: Dec. 9}

\begin{document}
\maketitle

\benum
% 11.3 ----------------------------------------------------------------------------------------------------------------------
  	\item[\textbf{11.3}]
	(9.1 in script)
	By taking the derivative of the self-consistency condition of a 1D Ising model with respect to the
	magnetic field, show that the susceptibility diverges like $|T-T_C|^{-1}$ on both sides of the critical
	temperature.
 	\\ \\ 
	It is helpful to deduce the behavior of the magnetization when $H=0$ and $T<T_C \rightarrow x>1$.
	Expanding the self-consistency condition for $L\ll 1$ we have
	\[
		L = xL -\frac{1}{3}(xL^3)
	\]
	\[
		L = \sqrt{3\pfrac{x-1}{x^3}}.
	\]
	From here we expand the fraction in the radical about the parameter 
	\[
		\Delta x = \frac{T_C-T}{T_C}  = \frac{x-1}{x}
	\]
	thus
	\[
		\frac{x-1}{x^3} = \Delta x - 2(\Delta x)^2 +(\Delta x)^3 \approx \Delta x.
	\]
	Now the magnetization may be approximated as
	\[
		L = \sqrt{ 3\pfrac{x-1}{x}} = \sqrt{3\plr{1-\frac{1}{x}}} = \sqrt{3\pfrac{T_C-T}{T_C}} 
		\propto (T_C-T)^{1/2}.
	\]
	On the other hand, when we approach the critical temperature from above $T>T_C$ then 
	$x<1$ and the only solution for $H=0$ is $L=0$. Hence
	\[
		L = \begin{cases} \sqrt{3\pfrac{T_C-T}{T_C}}&\qquad T<T_C \\
			0&\qquad T>T_C
			\end{cases}
	\]
	\\
	Now as for the susceptibility, we expand the consistency equation for small $L$ and $H$. It turns
	out that $L\to 0$ much slower than $H$ and so we only keep the first order term in $H$:
	\[
		L= \tanh \plr{\beta H + Lx} \simeq \beta H + Lx -\frac{1}{3}(Lx)^3.
	\]
	Using
	\[
		\chi  = \plr{\pdiff[L]{H}}_{H=0}
	\]
	we have
	\[
		\chi =  \beta +\chi x-\chi x^3 L^2
	\]
	or
	\[
		\chi = \frac{\beta}{1-x+x^3L^2}.
	\]
	From here, we may use our equation for $L$ close to the critical temperature
	\[
		\chi = \begin{cases}\ds \frac{\beta}{1-x+3x^3\pfrac{x-1}{x} }\approx \frac{\beta}{1-
		(1+\Delta x)+3\Delta x} = \frac{\beta}{2\pfrac{T_C-T}{T_C}} \propto |T_C-T|^{-1}
		 &\qquad T<T_C \\
		\ds \frac{\beta}{1-x}\approx \frac{\beta}{-\Delta x} = \frac{\beta}{\pfrac{T-T_C}{T_C}} \propto
		|T_C-T|^{-1}
		&\qquad T>T_C
		\end{cases}
	\]
	We conclude that
	\[
		\chi \propto |T_C-T|^{-1}
	\]
	on either side of the critical temperature.
	\\ \\
	
	
% 10.2 -----------------------------------------------------------------------------------------------------------------------
	\item[\textbf{11.4}]
	(9.2 in script)
	
	In the Bragg-Williams mean field approximation, the Hamiltonian for the one-dimensional Ising
	model is
	\[
		H_{MF} = -(\gamma \epsilon\braket s +H)\sum_i s_i +\frac{1}{2}\gamma\epsilon N\braket s^2
	\]
	where $\gamma = 2$, $\epsilon >0$ is a constant characterizing the spin-spin interaction, $H$ is the
	applied magnetic field, $N$ is the number of lattice sites $i$ containing the spins $s_i = \pm 1$, and
	$\braket s = \braket{s_i}$ is the mean field. 
	
	\benum
		% (a)
		\item
		Show that the free energy is
		\[
			G = -k_B TN\ln(2\cosh(\beta(\gamma\epsilon \braket s +H))) +\frac{1}{2}\gamma
			\epsilon N\braket{s}^2
		\]
		It is a subtlety of the magnetic system that when you calculate the free energy according to
		$-k_BT\ln Z$, the result is indeed Gibbs, not Helmholtz!
		\\ \\
		We look to compute
		\[
			G = -kT\ln Z.
		\]
		The mean field conveniently makes the Hamiltonian non-interacting, and thus the
		partition function is simply
		\ba
			Z &= \sum_i e^{-\beta E_i} =
			\exp\blr{-\beta(\gamma \epsilon \braket s +H)^N-\beta \frac{1}{2}\gamma\epsilon
			N\braket s^2}+
			\exp\blr{\beta(\gamma \epsilon \braket s +H)^N-\beta \frac{1}{2}\gamma\epsilon
			N\braket s^2} \\
			& = \exp\plr{-\beta \frac{1}{2}\gamma\epsilon N\braket{s}^2}\plr{
			\exp\blr{-\beta(\gamma \epsilon \braket s +H)}^N
			+\exp\blr{\beta(\gamma \epsilon \braket s +H)}^N}
			\\ &=\exp\plr{-\beta \frac{1}{2}\gamma\epsilon N\braket{s}^2}
			2\cosh\blr{\beta(\gamma \epsilon \braket s +H)}^N.
		\ea
		Now we can compute the Gibbs free energy using 
		\[
			G = -kTN\ln\blr{2\cosh\plr{\beta(\gamma \epsilon \braket s +H)}}
			+\frac{1}{2}\gamma\epsilon N\braket{s}^2
		\]
		\\ \\
		% (b)
		\item
		Now regard the mean field as an independent parameter, and find the extremum of free energy
		with respect to the value of the mean field. Show that the result is the usual self-consistency condition 
		of the mean field theory, $\braket s = \tanh(\beta(\gamma\epsilon \braket s +H))$.
		\\ \\
		Now we take
		\[
			\pdiff[G]{\braket s} = 0
		\]
		and compute:
		\ba
			\pdiff[G]{\braket s} &= -\pfrac{1}{\beta}\frac{2\sinh\plr{\beta(\gamma \epsilon \braket s +H)}}	
			{2\cosh\plr{\beta(\gamma \epsilon \braket s +H)}}N\beta\gamma\epsilon
			+
			\gamma\epsilon N\braket s  \\
			\gamma\epsilon N\braket s & = \tanh(\beta(\gamma\epsilon\braket s +H)) 
			\gamma \epsilon N \\
			\braket s & =  \tanh(\beta(\gamma\epsilon\braket s +H)) 
		\ea
		\\ \\
		% (c)
		\item
		In the absence of the magnetic field, below the critical temperature, the system has a spontaneous
		magnetization $\braket s \ne 0$. Show that the nontrivial solution, in fact, gives minimum of
		free energy, and the trivial solutions $\braket s = 0$ is a maximum of free energy.
		\\
		\\
		We can determine whether we are at a maximum or minimum by taking the second
		derivative (at $H=0$ and $x>1$)
		\[
			\pdifff{G}{*2\braket s} = \gamma\epsilon N\plr{1-
			\frac{\beta\gamma\epsilon}{\cosh^2(\beta\gamma\epsilon\braket s)}}
		\]
		For the nontrivial case, we may use the relation
		\[
			\beta\gamma\epsilon\braket s = \tanh^{-1}\braket s.
		\]
		Substituting this into the second derivative
		\ba
			\elr{\pdifff{G}{*2\braket s} }_{\braket s\ne 0} & =  \gamma\epsilon N\plr{1-
			\frac{\beta\gamma\epsilon}{\cosh^2( \tanh^{-1}\braket s)}} \\
			& =  \gamma\epsilon N\plr{1-\beta\gamma\epsilon
			(1-\braket{s}^2)}\\
			& = \gamma\epsilon N\plr{1+\beta\gamma\epsilon(\braket{s}^2-1)} \\
			& = \gamma\epsilon N\plr{1+x(\braket s^2 -1)}
		\ea
		Assuming 
		\[
			1-\braket{s}^2 >0
		\]
		which it seems it must in order for 
		\[
			\cosh(\tanh^{-1}\braket s) = \frac{1}{\sqrt{1-\braket s^2}}
		\]
		to be real. Thus the nontrivial magnetization value is a minimum of the free energy.
		\\
		\\
		For the trivial case, we substitute $\braket s = 0$ into the second derivative
		\ba
			\elr{\pdifff{G}{*2\braket s}}_{\braket s =0} &= \gamma\epsilon N\plr{1-
			\frac{\beta\gamma\epsilon}{\cosh^2(0)}} \\
			& = \gamma\epsilon N\plr{1-\beta\gamma\epsilon}\\
			& = \gamma \epsilon N (1-x) < 0
		\ea
		Hence, the trivial magnetization is a maximum of Gibbs energy. 
		\\ \\
		% (d)
		\item
		In the absence of the magnetic field, below the critical temperature, the system has a 
		spontaneous magnetization $\braket s \ne 0$; besides, if $\braket s$ is a possible
		magnetization, then so is $-\braket s$. Show, however, that even the tiniest external
		field $H\ne 0$ suffices to determine which way the equilibrium magnetization points for
		$T\to T_C$. 
		\\
		\\
		We can break the symmetry by taking the limit as $H\to 0$
		\[
			\braket s = \lim_{H\to 0}\lim_{N\to \infty} \frac{\sum_{\{s_i\}}
			\blr{\plr{\frac{1}{N}\sum_{i=1}^N s_i}e^{-\beta(H_0+H_I)}}}{\sum_{\{s_i\}}
			e^{-\beta(H_0+H_I)}} \ne 0
		\]
		\\
		% (e)
		\item
		Argue that the ferromagnetic phase transition in this system is continuous at the critical point
		$T_C$, $H=0$, but first order for any $H\ne 0$. 
		\\
		\\
		For temperature above $T_C$, the only solution for $\braket s$ is the trivial solution. At
		zero field $H=0$, we find that $\braket s=0$ and thus we have a continuous phase transition 
		at the critical temperature. However, when we have a small first order field of $H\ne 0$, then
		at the critical temperature we in fact have $\braket s \propto H^{1/3}$. In this sense,
		the magnetization jumps from zero to a finite first order value as we approach 
		the critical temperature. 
		skgj
		\eenum
\eenum
\end{document}