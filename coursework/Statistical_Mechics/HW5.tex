\documentclass[11pt,letterpaper]{article}
\usepackage{macroshw}

\title{\begin{spacing}{1.2}Statistical Mechanics\\HW 5\end{spacing}}
\author{Matthew Phelps}
\date{Due: Oct. 5}

\begin{document}
\maketitle

\benum
% 5.1 ----------------------------------------------------------------------------------------------------------------------
  	\item[\textbf{5.1}]
	Show that any density operator defined as
	\[
		\rho = \sum_i p_i\ket{\psi_i}\bra{\psi_i}
	\]
	for any set of normalized (but not necessarily orthogonal) states $\{\ket{\psi_i}\}$ can also be written as
	\[
		\rho = \sum_n p_n'\ket{n}\bra{n}
	\]
	where $\{\ket{n}\}$ is an orthonormal basis, and the $p'_n$ are also probabilities. 
	\\
	\\\\
	As a hermitian operator, the eigenvalues of $\rho$ are real and the eigenvectors form an orthogonal basis. Let's
	denote these normalized eigenvectors as 
	\[
		\rho \ket{n} = \lambda \ket n
	\]
	Using the identity operator twice, the matrix representation of $\rho$ is 
	\ba
		\rho &= \sum_i p_i\sum_{n,m} \ket{n}\braket{n|\psi_i}\braket{\psi_i|m}\bra m\\
		& = \sum_{n,m} \ket{n}\bra m\sum_{i} \braket{n|p_i|\psi_i}\braket{\psi_i|m}\\
		& = \sum_{n,m}\braket{n|\rho|m}\ket{n}\bra m.
	\ea
	Since the basis is the diagonal basis, this becomes
	\[
		\rho = \sum_n \lambda_n \braket{n|\rho|n}\ket n\bra n.
    	\]
	The expectation value of the density operator is real due to hermiticity and is positive definite by
	\[
		\braket{n|\rho|n} = \sum_i p_i |\braket{\psi|n}|^2 =\lambda_n \ge 0.
	\]
	Therefore we may write $\rho$ in an orthogonal basis with probabilities $p_n'=\lambda_n^2$
	\[
		\rho = \sum_n p_n'\bra n\ket n.
	\]
	
		
% 5.2 -----------------------------------------------------------------------------------------------------------------------
	\item[\textbf{5.2}]
	Given an entangled state of $\ket\phi = (\ket{000}+\ket{111})/\sqrt 2$ of three qubits,
	what is the state of one of the individual subsystems?
	\\
	\\
	\\
	As an entangled state, it cannot be written as a tensor product of three pure state. \emph{However we may write it as a 	density matrix, which suggests the individual subsystems are actually ensembles?}
	\\
	\\
	Start with the entangled pure state
	\[
		\rho = \ket\phi\bra\phi.
	\]
	To find the ensemble state of say subsystem $A$, we must trace out the other two subsystem $B$ and $C$:
	\[
		\sum_{b,c} \bra b_B\bra c_C\ket\phi\bra\phi \ket c_C\ket b_B = \sum_{b,c}\braket{bc|\phi}\braket{\phi|bc}.
	\]
	Now applied to the given system,
	\[
		\braket{\phi|cb} = \frac{1}{\sqrt 2} \delta_{c,b}\bra{b}
	\]
	thus
	\[
		\braket{bc|\phi} = \frac{1}{\sqrt 2} \delta_{c,b} \ket{b}
	\]
	so
	\[
		\sum_{b,c}\braket{bc|\phi}\braket{\phi|bc} = \sum_{b,c}\frac{1}{2}\delta_{c,b}\ket b\bra b = \frac{1}{2}(
		\ket 0\bra 0+\ket 1\bra 1).
	\]
	Therefore, the state of each individual subsystem $i$ may be written as
	\[
		\rho_i =  \frac{1}{2}(\ket 0\bra 0+\ket 1\bra 1).
	\]
	\\
% 5.3 ------------------------------------------------------------------------------------------------------------------------
	\item[\textbf{5.3}]
	Take a qubit with the two states $\ket\pm$. Write down the density operators in matrix form corresponding
	to the superposition state $\ket\psi = \frac{1}{\sqrt 2}\plr{\ket ++\ket -}$ and to the 50/50 mixture of the 
	states $\ket +$ and $\ket -$. 
	\\
	\\
	The superposition state is a pure state
	\[
		\rho = \ket{\psi}\bra\psi = \sum_{n,m}\braket{n|\rho|m}\ket n\bra m = 
		\sum_{n,m}\braket{n|\psi}\braket{\psi|m}\ket n\bra m.
	\]
	The elements are easily worked out
	\[
		\braket{\pm|\psi} = \frac{1}{\sqrt 2}
	\]
	and our matrix
	\[
		\rho= \frac{1}{2}\bpm 1&1\\1&1\epm
	\]
	For the ensemble, our density operator is
	\[
		\rho = \frac{1}{2}\ket +\bra++\frac{1}{2}\ket -\bra -
	\]
	so our matrix is
	\[
		\rho = \frac{1}{2}\bpm 1&0\\0&1\epm
	\]
	\\
	\\
% 5.4 -------------------------------------------------------------------------------------------------------------------------
	\item[\textbf{5.4}]
	
	Show that the relation that we used in class is correct:
	\[
		p_m = \tr\plr{P_m\rho P_m},
	\]
	where $p_m$ is the probability of being in eigenstate $\ket m$ and $P_m$ is the projector onto this state.
	\\
	\\
	\\
	Forming the trace of the density operator $\rho = \sum_i p_i\ket{\psi_i}\bra{\psi_i}$ in the basis of $\ket m$ (which 
	should be orthogonal considering they are eigenstates)
	\ba
		\tr\plr{P_m\rho P_m} &= \sum_i p_i \sum_n \braket{n|m}\braket{m|\psi_i}\braket{\psi_i|m}\braket{m|n}\\
		& = \sum_i p_i |\braket{m|\psi_i}|^2 \\
		& =  p_m
	\ea
	If the density operator is a pure state, then $\rho = \ket\psi\bra\psi$ and the trace reduces to
	\[
		\tr\plr{P_m\rho P_m} =  |\braket{m|\psi}|^2
	\]
	which is the familiar probability of obtaining eigenstate $\ket m$ for a known pure state. 
		
\eenum
\end{document}