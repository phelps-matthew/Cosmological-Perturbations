\documentclass[10pt,letterpaper]{article}
\usepackage{macroshw}

\title{\begin{spacing}{1.3}QFT\\ Ch 3: The Dirac Field\end{spacing}}
\author{Matthew Phelps}
\date{}
\begin{document}
\maketitle

\benum
% #1 -----------------------------------------------------------------------------------------------------------------------------------------------------------------
  	 \item[3.4]{\bf{Majorana fermions}}
	 
	Recall from Eq. (3.40) that one can write a relativistic equation for a massless 2-component fermion field that transforms
	as the upper two components of a Dirac spinor ($\psi_L$). Call such a 2-component field $\chi_a(x)$, $a=1,2$.
	\benum
	% (a)
	\item
	Show that it is possible to write an equation for $\chi(x)$ as a massive field in the following way:
	\[
		i\bar\sigma\cdot \pd\chi - im\sigma^2\chi^* = 0.
	\]
	That is, show, first, that this equation is relativistically invariant and, second, that it implies the Klein-Gordan
	equation, $(\pd^2+m^2)\chi = 0$. This form of the fermion mass is called a Majorana mass term.
	\\ \\
	In the reduced Lorentz representation of two component left handed Weyl spinors, $SL(2,\mathbb C)$, such a
	spin transforms as
	\[
		\chi \to \Lambda_{L}\chi(\Lambda^{-1}x)
	\]
	\[
		\chi^* \to \Lambda_{L}^*\chi^*(\Lambda^{-1}x)
	\]
	with
	\[
		\Lambda_{L} = e^{-\frac{i}{2}\omega_{\mu\nu}S^{\mu\nu}},
	\]
	 where $S^{\mu\nu}$ is the antisymmetric tensor representing boosts and rotations
	\[
		S^{0i} = -\frac{i}{2}\sigma^i;\qquad S^{ij} = \frac{1}{2}\epsilon^{ijk}\sigma^k.
	\]
	The 4x4 matrix $\Lambda$ is the real four vector Lorentz representation
	\[
		\Lambda^\mu_\nu = e^{-\frac{i}{2}\omega_{\mu\nu}\mathcal J^{\mu\nu}}.
	\]
	In order to determine how $\psi_L$ transforms under a Lorentz boost, we will need an analogue to eq. 3.29.
	First note
	\[
		\gamma^\mu = \bpm 0 &\sigma^\mu \\ \bar\sigma^\mu & 0 \epm.
	\]
	In terms of the left and right spinor operators, we can write
	\[
		\Lambda_{1/2} = \bpm \Lambda_L& 0\\0&\Lambda_R \epm;\qquad 
		\Lambda^{-1}_{1/2} = \bpm \Lambda^{-1}_L& 0\\0&\Lambda^{-1}_R \epm.
	\]
	Now we make take eq. 3.29 
	\[
		\Lambda^{-1}_{1/2}\gamma^\mu\Lambda_{1/2} = \Lambda_\nu^\mu \gamma^\nu
	\]
	\[
		\Rightarrow
		\bpm \Lambda_L^{-1} & 0 \\0 & \Lambda_R^{-1} \epm
		\bpm 0 &\sigma^\mu \\ \bar\sigma^\mu & 0 \epm
		\bpm \Lambda_L& 0\\0&\Lambda_R \epm
		= \Lambda^\mu_\nu  \bpm 0 &\sigma^\nu \\ \bar\sigma^\nu & 0 \epm
	\]
	Since $\Lambda$ only acts on vector indices, we conclude that
	\[
		\Lambda^{-1}_L\sigma^\mu\Lambda_{R} = \Lambda_\nu^\mu \sigma^\nu
	\]
	\[
		\Lambda^{-1}_R\bar\sigma^\mu\Lambda_{L} = \Lambda_\nu^\mu \bar\sigma^\nu.
	\]
		
	
	This is due to the block diagonal form of $S^{\mu\nu}$ in the Dirac representation. The last ingredient we will need 
	is how the complex conjugate $\psi_L^*$ transforms. Using the identity (three vectors here)
	\[
		\sigma^2\vect \sigma^* = -\vect \sigma \sigma^2 
	\]
	we can show
	\[
		\sigma^2 \psi_L^* \to \sigma^2 \Lambda_{L}^*\psi_L^* = \Lambda_R \sigma^2\psi_L^*.
	\]
	Now compute the Majorana equation under a boost
	\ba
		i\bar\sigma^\mu\pd_\mu\chi -im\sigma^2\chi^* 
		&\to i\bar\sigma^\mu \Lambda_L \plr{\Lambda^{-1}}^\nu_\mu\pd_\nu\chi(\Lambda^{-1}x) 
		-im\sigma^2\Lambda_L^*\chi^*(\Lambda^{-1}x) \\
		& = i\Lambda_R \Lambda_R^{-1} \bar\sigma^\mu \Lambda_L \plr{\Lambda^{-1}}^\nu_\mu\pd_\nu\chi(\Lambda^{-1}x) 
		-im\Lambda_R\sigma^2 \chi^*(\Lambda^{-1}x) \\
		& = i\Lambda_R \Lambda^\mu_\sigma \bar\sigma^\sigma \plr{\Lambda^{-1}}^\nu_\mu\pd_\nu\chi(\Lambda^{-1}x) 
		-im\Lambda_R\sigma^2 \chi^*(\Lambda^{-1}x)\\
		& = \Lambda_R\blr{ i\bar\sigma^\nu \pd_\nu \chi(\Lambda^{-1}x)-im\sigma^2\chi^*(\Lambda^{-1}x)}\\
		& = 0.
	\ea
	\\ \\
	Before we show the Klein-Gordon field, note the extension of the identity 
	\[
		\sigma^2\sigma^{*\mu} = \bar\sigma^\mu \sigma^2
	\]
	\[
		\sigma^2\sigma^\mu = \bar\sigma^{*\mu} \sigma^2.
	\]
	Now take the field and its conjugate
	\be\label{1}
		\bar\sigma^\mu\pd_\mu\chi = m\sigma^2\chi^*
	\ee
	\be\label{2}
		\bar\sigma^{*\nu}\pd_{\nu}\chi^* = -m\sigma^2\chi.
	\ee
	Solve for $\chi^*$ from \eqref 1
	\[
		\chi^* = \frac{\sigma^2}{m}\bar\sigma^\mu\pd_\mu\chi
	\]
	insert into \eqref 2
	\[
		\bar\sigma^{*\nu}\sigma^2 \bar\sigma^\mu\pd_\nu \pd_\mu\chi = -m^2\sigma^2\chi
	\]
	use identity and eliminate $\sigma^2$
	\[
		\sigma^\nu \bar\sigma^\mu\pd_\nu \pd_\mu\chi = -m^2\chi.
	\]
	The matrices acting on derivatives sum to
	\[
		\sigma^\nu\bar\sigma^\mu\pd_\nu\pd_\mu = g^{\nu\mu}\pd_\nu\pd_\mu,
	\]
	this can be found by antisymmetry via the anticommutator $\{\sigma^i,\sigma^j\} = 2\delta_{ij}$. Thus we conclude
	\[
		\sigma^\nu\bar\sigma^\mu \pd_\nu\pd_\mu = g^{\nu\mu}\pd_\nu\pd_\mu = \pd^\mu \pd_\mu
	\]
	and we have
	\[
		(\pd^\mu\pd_\mu+m^2)\chi = 0.
	\]
	\\ 
	% (b)
	\item
	Does the Majorana equation follow from a Lagrangian? The mass term would seem to be the variation of 
	$(\sigma)^2_{ab}\chi_a^*\chi_b^*$; however, since $\sigma^2$ is antisymmetric, this expression would 
	vanish if $\chi(x)$ were an ordinary c-number field. When we go to quantum field theory, we know that 
	$\chi(x)$ will become an anticommuting quantum field. Therefore, it makes sense to develop its
	classical theory by considering $\chi(x)$ as a classical anticommuting field, that is, as a field that takes as
	values \emph{Grassmann numbers} which satisfy
	\[
		\alpha\beta = -\beta\alpha\qquad\text{for any}\ \alpha,\beta.
	\]
	Note that this relation implies that $\alpha = 0$ A Grassmann field $\zeta(x)$ can be expanded in a basis of functions
	as 
	\[
		\zeta(x) = \sum_n \alpha_n\phi_n(x),
	\]
	where the $\phi_n(x)$ are orthogonal c-number functions and the $\alpha_n$ are a set of independent Grassmann
	numbers. Define the complex conjugate of a product of Grassman numbers to reverse the order:
	\[
		(\alpha\beta)^* \equiv \beta^*\alpha^* = -\alpha^*\beta^*.
	\]
	This rule imitates the Hermitian conjugation of quantum fields. Show that the classical action,
	\[
		S = \int d^4x\ \blr{\chi^\dag i\bar\sigma\cdot\pd\chi +\frac{im}{2}\plr{ \chi^T\sigma^2\chi
		-\chi^\dag\sigma^2\chi^*}},
	\]
	(where $\chi^\dag = (\chi^*)^T$) is real ($S^*= S$), and that varying this $S$ with respect to $\chi$ 
	and $\chi^*$ yields the Majorana equation.
	% (c)
	\item
	C
	% (d)
	\item
	D	
	\eenum
	\eenum
\end{document}