\documentclass[10pt,letterpaper]{article}
\usepackage{macroshw}

\title{\begin{spacing}{1.3}QFT\\ Ch 2: The Klein-Gordan Field\end{spacing}}
\author{Matthew Phelps}
\date{}
\begin{document}
\maketitle

\benum
% #1 -----------------------------------------------------------------------------------------------------------------------------------------------------------------
  	 \item[2.2]{\bf{The complex scalar field}}
	 
	Consider the field theory of a complex-valued scalar field obeying the the Klein-Gordon equation. The action of
	this theory is
	\[
		S = \int d^4x\ (\partial_\mu \phi^*\partial^\mu \phi-m^2\phi^*\phi)
	\]
	It is easiest to analyze this theory by considering $\phi(x)$ and $\phi^*(x)$, rather than the real and imaginary
	parts of $\phi(x)$, as the basic dynamical variables. 
	
	\benum
	% (a)
	\item
	Find the conjugrate momenta to $\phi(x)$ and $\phi^*(x)$ and the canonical commutation relations. Show that the
	Hamiltonian is 
	\[
		H = \int d^3x\ (\pi^*\pi +\del \phi^*\cdot \del \phi +m^2 \phi^*\phi).
	\]
	Compute the Heisenberg equation of motion for $\phi(x)$ and show that it is indeed the Klein-Gordon equation.
	
	% (b)
	\item
	Diagonalize $H$ by introducing creation and annihilation operators. Show that the theory contains two sets of 
	particles of mass $m$.
	
	% (c)
	\item
	Rewrite the conserved charge 
	\[
		Q = \int d^3x\ \frac{i}{2}(\phi^*\pi^*-\pi\phi)
	\]
	in terms of creation and annihilation operators, and evaluate the charge of the particles of each type. 
	% (d)
	\item
	Consider the case of two complex Klein-Gordon fields with the same mass. Label the fields as
	$\phi_a(x)$, where $a=1,2$. Show that there are now four conserved charges, one given by the
	generalization of part (c), and the other three given by
	\[
		Q_i = \int d^3x\ \frac{i}{2}\plr{\phi^*_a(\sigma^i)_{ab}\pi^*_b-\pi_a(\sigma^i)_{ab}\phi_b},
	\]
	where $\sigma^i$ are the Pauli sigma matrices. Show that these three charges have the commutation 
	relations of angular momentum ($SU(2)$). Generalize these results to the case of $n$ identical
	complex scalar fields. 
	\\ \\
	\eenum
	\pagebreak
	\benum
	\item 
	% (a)
	With
	\[
		\pi(\vect x) = \pdiff[\mathcal L]{\dot\phi(\vect x)}
	\]
	we form the conjugate momenta from our Lagrangian
	\[
		\pi(\vect x) = \dot\phi^*;\qquad \pi^*(\vect x) = \dot\phi
	\]
	From the canonical commutation relation of a field and its conjugate momenta,
	\[
		[\phi(\vect x),\pi(\vect x')] = i\delta(\vect x-\vect x')
	\]
	and taking the conjugate
	\[
		[\phi(\vect x)^*,\pi(\vect x')^*] = -[\phi(\vect x),\pi(\vect x')]^\dag = i\delta(\vect x-\vect x').
	\]
	
	The other usual commutation relations should also hold
	\[
		[\phi(\vect x),\phi^*(\vect x')] = [\pi(\vect x),\pi^*(\vect x')] = [\phi(\vect x),\pi^*(\vect x')]= 0.
	\]
	To find the Hamiltonian, we may use
	\[
		H = \int d^3x\ \plr{\sum_i \pi_i(\vect x)\dot\phi_i(\vect x) - \mathcal L} = \int d^3x\ \mathcal H
	\]
	thus
	\ba
		\mathcal H &= \pdiff[\mathcal L]{\dot\phi}\dot\phi + \pdiff[\mathcal L]{\dot\phi^*}\dot\phi^* -\mathcal L\\
		& = \dot\phi^*\dot\phi + m^2\phi\phi^*+\del\phi\cdot\del\phi^* \\
		& = \pi^*\pi + \del\phi^*\cdot\del\phi +m^2\phi^*\phi
	\ea
	To find the operators' time dependence, we compute the Heisenberg equation of motion
	\ba
		i\pdiff{t}\phi(x) &= [\phi(x),H]  \\
		& = \blr{ \phi(x),\int d^3x'\ \pi^*(x')\pi(x') + \del\phi^*(x')\cdot\del\phi(x') +m^2\phi^*(x')\phi(x') }\\
		& = \int d^3x'\ \plr{ \pi^*(x')[\phi(x),\pi(x')]+\del\phi^*(x')\cdot [\phi(x),\del\phi(x')] + 
		m^2\phi^*(x')[\phi(x),\phi(x')]} \\
		& = \int d^3x'\ i\pi^*(x')\delta^3(\vect x-\vect x') \\
		& = i\pi^*(\vect x,t).
	\ea
	Similarly
	\[
		\pdiff{t}\phi^*(x) = \pi(x).
	\]
	For the conjugate momenta
	\ba
	 [\pi(x),H]  & = \blr{ \pi(x),\int d^3x'\ \pi^*(x')\pi(x') + \del\phi^*(x')\cdot\del\phi(x') +m^2\phi^*(x')\phi(x') }\\
		& = \int d^3x'\ \plr{ \pi^*(x')[\pi(x),\pi(x')]+\del\phi^*(x')\cdot [\pi(x),\del\phi(x')] + 
		m^2\phi^*(x')[\pi(x),\phi(x')]} \\
	\ea
	Looking at the commutator with the gradient
	\[
		[\pi(x),\del \phi(x')] = \del_{x'}[\pi(x),\phi(x')] = -i\del \delta^3(\vect x-\vect x')
	\]
	with the derivative defined as (for arbitrary $\vect f$)
	\ba
		\vect f\cdot \del \delta^3(\vect x-\vect x') = -(\del\cdot \vect f) \delta^3(\vect x-\vect x').
	\ea
	Back to the EOM for $\pi(x)$
	\ba
		[\pi(x),H] &= \int d^3x'\ \blr{i\del^2\phi^*(x')\delta^3(\vect x-\vect x) 
		- im^2\phi^*(x')\delta^3(\vect x-\vect x')} \\
		\pdiff[\pi(x)]{t} & = \del^2\phi^*(x)-m^2\phi^*(x)
	\ea
	and likewise for the conjugate
	\[
		\pdiff[\pi^*(x)]{t} = \del^2\phi(x)-m^2\phi(x).
	\]
	Substituting $\pi = \dot\phi$ we arrive at the Klein-Gordon equation for each field
	\[
		\pd_\mu\pd^\mu \phi +m^2\phi = 0;\quad \pd_\mu\pd^\mu \phi^* +m^2\phi^* =0
	\]
	\pagebreak
	
	% (b)
	\item
	We can express $\phi(x)$ in momentum space as
	\[
		\phi(x) = \int \frac{d^3p}{(2\pi)^3} e^{i\vect p\cdot\vect x} \phi(\vect p,t)
	\]
	and then apply the Klein-Gordon equation to arrive at
	\[
		\plr{\pdifff{}{*2t}+\vect p\cdot \vect p + m^2}\phi(\vect p,t) = 0
	\]
	which has solutions
	\[
		\phi(\vect p,t) = a(\vect p) e^{-i\omega_\vect p t}+b(\vect p) e^{i\omega_\vect p t};\qquad 
		\omega_\vect p = \sqrt{|\vect p|^2+m^2}.
	\]
	This applies to both fields $\phi(x)$ and $\phi^*(x)$. In the case of the real field, we choose the coefficients
	of momentum in analogy with the harmonic oscillator raising/lowering operators as the following
	\[
		a(\vect p) \to \frac{1}{\sqrt{2\omega_\vect p}} a_\vect p
	\]
	\[
		b(\vect p) \to \frac{1}{\sqrt{2\omega_\vect p}}a^\dag_{-\vect p}
	\]
	thus
	\[
		\phi(x) = \int \frac{d^3p}{(2\pi)^3} \frac{1}{\sqrt{2\omega_\vect p}} e^{i\vect p\cdot\vect x}
		(a_\vect p e^{-i\omega_{\vect p}t}+a^\dag_\vect{-p} e^{i\omega_{\vect p}t})
	\]
	and from the Heisenberg equation of motion $\pi(x) = \dot\phi(x)$
	\[
		\pi(x) = \int \frac{d^3p}{(2\pi)^3}(-i) \sqrt{\frac{\omega_\vect p}{2}}e^{i\vect p\cdot\vect x}\plr{
		a_\vect p e^{-i\omega_\vect p t}-a^\dag_{-\vect p}e^{i\omega_\vect p t}}.
	\]
	Imposing $[\phi(x),\pi(x')]_{x_0=x'_0} = i\delta^3(\vect x-\vect x')$ leads to 
	\[
		[a_\vect p,a^\dag_{\vect p'}] = (2\pi)^3\delta(\vect p-\vect p').
	\]
	This last commutation relation aligns with our expectation of a harmonic oscillator. The specific form 
	of the coefficients was chosen so that $\phi(x)$ is self adjoint i.e. $\phi^\dag(\vect p,t) = \phi(-\vect p,t)$. 
	This is important because for the complex field, $\phi(x)$ is no longer self adjoint. As such we 
	choose coefficients as
	\[
		a(\vect p) \to \frac{1}{\sqrt{2\omega_\vect p}} a_\vect p
	\]
	\[
		b(\vect p) \to \frac{1}{\sqrt{2\omega_\vect p}}b^\dag_{-\vect p}.
	\]
	Recalling that the Heisenberg e.o.m. are now $\pi = \dot\phi^\dag$ we have the following:
	\[
		\phi(x) = \int \frac{d^3p}{(2\pi)^3} \frac{1}{\sqrt{2\omega_\vect p}} e^{i\vect p\cdot\vect x}
		(a_\vect p e^{-i\omega_{\vect p}t}+b^\dag_\vect{-p} e^{i\omega_{\vect p}t})
	\]
	\[
		\pi(x) = \int \frac{d^3p}{(2\pi)^3}i \sqrt{\frac{\omega_\vect p}{2}}e^{i\vect p\cdot\vect x}\plr{
		a^\dag_{-\vect p} e^{i\omega_\vect p t}-b_{\vect p}e^{-i\omega_\vect p t}}.
	\]
	\[
		\phi^\dag(x) = \int \frac{d^3p}{(2\pi)^3} \frac{1}{\sqrt{2\omega_\vect p}} e^{i\vect p\cdot\vect x}
		(a^\dag_{-\vect p} e^{i\omega_{\vect p}t}+b_\vect{p} e^{-i\omega_{\vect p}t})
	\]
	\[
		\pi^\dag(x) = \int \frac{d^3p}{(2\pi)^3}(-i) \sqrt{\frac{\omega_\vect p}{2}}e^{i\vect p\cdot\vect x}\plr{
		a_\vect p e^{-i\omega_\vect p t}-b^\dag_{-\vect p}e^{i\omega_\vect p t}}.
	\]	
	In maintaining $[\phi(\vect x),\pi(\vect x')] = [\phi^\dag(\vect x),\pi^\dag(\vect x')] = i\delta(\vect x-\vect x')$ 
	we find the following commutation relation (denote $c\equiv e^{i\omega_\vect p t_0}$)
	\ba
		i\delta^3(\vect x-\vect x') &= 
		\int \frac{d^3pd^3p'}{(2\pi)^6}\pfrac{i}{2}e^{i(\vect p\cdot x+\vect p'\vect\cdot \vect x')}
		\blr{ (ca_\vect p+c^*b^\dag_{-\vect p}),(c'^*a^\dag_{-\vect p'}-c'b_{\vect p'}) }\\
		&= \int \frac{d^3pd^3p'}{(2\pi)^6}\pfrac{i}{2}e^{i(\vect p\cdot x+\vect p'\vect\cdot \vect x')}
		\clr{ cc'^*[a_\vect p,a^\dag_{-\vect p'}]+c'c^*[b_{\vect p'},b^\dag_{-\vect p}]-cc'[a_\vect p,b_{\vect p'}]
		+c^*c'^*[b^\dag_{-\vect p},a^\dag_{\vect -p'}] }\\
		&= \int \frac{d^3pd^3p'}{(2\pi)^6}\pfrac{i}{2}e^{i(\vect p\cdot x+\vect p'\vect\cdot \vect x')}
		\plr{ cc'^*[a_\vect p,a^\dag_{-\vect p'}]+c'c^*[b_{\vect p'},b^\dag_{-\vect p}]}
	\ea
	which leads to
	\[
		[a_\vect p,a^\dag_{\vect p'}] = [b_\vect p,b^\dag_{\vect p'}] = (2\pi)^3\delta(\vect p-\vect p').
	\]
	All other combinations commute. Note that it seemed to helpful to single out which commutators are relevant
	by keeping the phase factors in; those that were not conjugates were discarded for many (hopefully
	self evident) reasons. 
	\\ \\

	Now we form the Hamiltonian
	\ba
		H &= \int d^3x\ (\pi^*\pi + \del\phi^*\cdot\del\phi +m^2\phi^*\phi)\\
		&= \int \frac{d^3x}{(2\pi)^3}e^{i(\vect{p}+\vect{p}')\cdot \vect x} \int \frac{ d^3pd^3p'}{(2\pi)^3}
		\bigg\{
		\frac{\sqrt{\omega_{\vect p}\omega_{\vect p'}}}{2}
				(a_{\vect{p}'}-b^\dag_{-\vect{p}'})(a^\dag_{-\vect{p}}-b_{\vect{p}})\\
				&\qquad\qquad\qquad\qquad\qquad\qquad\qquad\qquad\qquad
		+\frac{-\vect p\cdot\vect p'+m^2}{2\sqrt{\omega_{p}\omega_{p'}}}
		(a^\dag_{-\vect p'}+b_{\vect p'})(a_{\vect p}+b^\dag_{-\vect p})
		\bigg\}\\
		& = \int \frac{ d^3pd^3p'}{(2\pi)^3} \delta^3(\vect p+\vect p')
		\bigg\{
		\frac{\sqrt{\omega_{\vect p}\omega_{\vect p'}}}{2}
				(a_{\vect{p}'}-b^\dag_{-\vect{p}'})(a^\dag_{-\vect{p}}-b_{\vect{p}})\\
				&\qquad\qquad\qquad\qquad\qquad\qquad\qquad\qquad\qquad
		+\frac{-\vect p\cdot\vect p'+m^2}{2\sqrt{\omega_{p}\omega_{p'}}}
		(a^\dag_{-\vect p'}+b_{\vect p'})(a_{\vect p}+b^\dag_{-\vect p})
		\bigg\}\\
		& = \int \frac{d^3p}{(2\pi)^3}\pfrac{\omega_\vect p}{2}\clr{
		(a_{-\vect{p}}-b^\dag_{\vect{p}})(a^\dag_{-\vect{p}}-b_{\vect{p}})
		+
		(a^\dag_{\vect p}+b_{-\vect p})(a_{\vect p}+b^\dag_{-\vect p})
		}\\
		& = \int \frac{d^3p}{(2\pi)^3}\pfrac{\omega_\vect p}{2}\clr{
		(a_{-\vect{p}}-b^\dag_{\vect{p}})(a^\dag_{-\vect{p}}-b_{\vect{p}})
		+
		(a^\dag_{-\vect p}+b_{\vect p})(a_{-\vect p}+b^\dag_{\vect p})
		}\\
		& = \int \frac{d^3p}{(2\pi)^3} \pfrac{\omega_\vect p}{2} \plr{
			a_{-\vect p}a^\dag_{-\vect p}+a^\dag_{-\vect p}a_{-\vect p}
			+b_{\vect p}b^\dag_{\vect p} +b_{\vect p}^\dag b_\vect p}\\
		& = \int \frac{d^3p}{(2\pi)^3} \omega_\vect p \plr{
			a^\dag_\vect p a_\vect p + b^\dag_\vect p b_\vect p 
			+\frac{1}{2}[a_\vect p,a^\dag_\vect p]+\frac{1}{2}[b_\vect p,b^\dag_\vect p]
			}
	\ea
	It appears we have creation/annihilation operators for particles of type $a$ and $b$, with the same
	relativistic energy and thus same mass. 
	\pagebreak
	
	% (c)
	\item
	The Lagrangian is invariant under the transformation
	\[
		\phi\to e^{i\alpha}\phi;\qquad \phi^* \to e^{-i\alpha}\phi^*
	\]
	which in infinitesimal form amounts to the variation $\Delta \phi = i\alpha\phi$. The conserved current
	is then
	\[
		j^\mu = \pdiff[\mathcal L]{(\pd_\mu\phi)}\Delta\phi = i\alpha\blr{
		(\pd_\mu\phi^*) \phi-(\pd_\mu\phi) \phi^*}.
	\]
	The conserved charge is then $Q= \int d^3x\ j^0$ and so (for $\alpha = 1/2$)
	\[
		Q = \int d^3x\ \frac{i}{2}\plr{ \phi^*\pi^* -\pi\phi}
	\]
	I suspect we write it in this form so that $Q^\dag = Q$. Continuing, 
	\ba
		Q &= \int \frac{d^3x}{(2\pi)^3} e^{i(\vect p+\vect p')\cdot \vect x}
		\int \frac{d^3pd^3p'}{(2\pi)^3} \pfrac{1}{4}\sqrt\frac{\omega_\vect p'}{\omega_\vect p}
		\bigg\{
		\plr{a^\dag_{-\vect p} e^{i\omega_{\vect p}t}+b_\vect{p} e^{-i\omega_{\vect p}t}}
		\plr{ a_{\vect p'} e^{-i\omega_\vect p' t}-b^\dag_{-\vect p'}e^{i\omega_\vect p' t}}\\
				&\qquad\qquad\qquad\qquad\qquad\qquad\qquad\qquad\qquad
		+\plr{a^\dag_{-\vect p'} e^{i\omega_\vect p' t}-b_{\vect p'}e^{-i\omega_\vect p' t}}
		\plr{a_\vect p e^{-i\omega_{\vect p}t}+b^\dag_\vect{-p} e^{i\omega_{\vect p}t}}
		\bigg\}\\
		& = \int \frac{d^3p}{(2\pi)^3}\pfrac{1}{4}\bigg\{
		\plr{a^\dag_{-\vect p} e^{i\omega_{\vect p}t}+b_\vect{p} e^{-i\omega_{\vect p}t}}
		\plr{ a_{-\vect p} e^{-i\omega_\vect p t}-b^\dag_{\vect p}e^{i\omega_\vect p t}}\\
		&\qquad\qquad\qquad\qquad\qquad\qquad\qquad\qquad\qquad
		+\plr{a^\dag_{\vect p} e^{i\omega_\vect p t}-b_{-\vect p}e^{-i\omega_\vect p t}}
		\plr{a_\vect p e^{-i\omega_{-\vect p}t}+b^\dag_{-\vect{p}} e^{i\omega_{\vect p}t}}\bigg\}\\
		& =  \int \frac{d^3p}{(2\pi)^3}\pfrac{1}{2}
		\clr{ a^\dag_{\vect p}a_{\vect p}-b_{\vect p}b^\dag_{\vect p}}\\
		& =   \int \frac{d^3p}{(2\pi)^3}\pfrac{1}{2}
		\clr{ a^\dag_{\vect p}a_{\vect p}-b^\dag_{\vect p}b_{\vect p}+\delta^3(0)}
	\ea
	Not accounting for the infinite constant, we see that the corresponding charges are
	$\pm \frac{1}{2}q$, where $\alpha = \frac{1}{2}q$ is some charge constant. 
	\pagebreak
	
	% (d)
	\item
	To maintain the Klein-Gordon equation for two complex fields (4 independent fields), we may add on a similar 
	Lagrangian so that
	\ba
		\mathcal L &= \pd_\mu \phi_1\pd^\mu\phi^*_1 +\pd_\mu \phi_2\pd^\mu\phi^*_2
		-m^2(\phi_1\phi^*_1+\phi_2\phi^*_2) \\
		& = |\pd_\mu \phi_1|^2+ |\pd_\mu \phi_2|^2 -m^2\plr{|\phi_1|^2+|\phi_2|^2}.
	\ea
	Based on the form of the conserved current, we need to write this in terms of vectors. Denote
	\[
		\vec \Phi = \bpm \phi_1 \\ \phi_2 \epm;\qquad \Phi_i = \phi_i
	\]
	and now
	\[
		\mathcal L = \pd^\mu\Phi_i^\dag\pd_\mu\Phi^i - m^2\plr{ \Phi_i^\dag\Phi^i}.
	\]
	The Lagrangian is invariant under $SU(2)$ transformation
	\[
		\vec \Phi \to e^{i\vec\sigma\cdot\vec \alpha}\vec \Phi
	\]
	infinitesimally
	\[
		\alpha^i \Delta \Phi_j = i\alpha^i (\sigma_{jk})^i\Phi^k.
	\]
	We may also include the identity matrix, which amounts to the same transformation as earlier ($U(1)$)
	\[
		\phi_i \to e^{i\alpha}\phi_i.
	\]
	Then we may combine everything into a conserved tensor of four currents (define $\sigma^0 = \mathds 1$) 
	\ba
		T^{\mu \nu} &= \pdiff[\mathcal L]{(\pd_\mu \Phi^i)}\alpha^\nu \Delta\Phi_i+
					\pdiff[\mathcal L]{(\pd_\mu \Phi^{\dag i})}\alpha^\nu \Delta\Phi_i^\dag \\
		& = \pdiff[\mathcal L]{(\pd_\mu \Phi^i)}i\alpha^\nu (\sigma_{ij})^\nu\Phi^j-
					\pdiff[\mathcal L]{(\pd_\mu \Phi^{\dag i})}i\alpha^\nu (\sigma_{ij}^\dag)^\nu\Phi^{\dag j}
	\ea
	The four conserved charges are then (with $\alpha = -1/2$)
	\[
		Q^\nu = \int d^3x\ T^{0\nu} = \int d^3x\ \frac{i}{2}\blr{\pi_a^\dag (\sigma_{ab}^\dag)^\nu\phi^{\dag}_b-
		 \pi_a (\sigma_{ab})^\nu\phi_b} 
	\]
	There is an implied summation over $a,b$ here. I suppose as we require $Q = Q^\dag$ we 
	rewrite this as
	\[
		Q^\nu =  \int d^3x\ \frac{i}{2}\blr{\phi_a^\dag (\sigma_{ab})^\nu\pi^{\dag}_b-
		 \pi_a(\sigma_{ab})^\nu\phi_b}.
	\]
	Now forming the commutator
	\ba
		[Q^i,Q^j] &= \int d^3x\ d^3x' \pfrac{i}{2}^2
		\blr{
		\plr{\phi_a^\dag (\sigma_{ab})^i\pi^{\dag}_b-\pi_a(\sigma_{ab})^i\phi_b},
		\plr{\phi_{a'}^\dag (\sigma_{a'b'})^j\pi^{\dag}_{b'}-\pi_{a'}(\sigma_{a'b'})^j\phi_{b'}}
		}
	\ea
	\ba
		\blr{
		\plr{\phi_a^\dag (\sigma_{ab})^i\pi^{\dag}_b-\pi_a(\sigma_{ab})^i\phi_b},
		\plr{\phi_{a'}^\dag (\sigma_{a'b'})^j\pi^{\dag}_{b'}-\pi_{a'}(\sigma_{a'b'})^j\phi_{b'}}
		}&\\=
		\delta_{aa',bb'}\blr{\phi_a^\dag (\sigma_{ab})^i\pi^{\dag}_b,\phi_{a'}^\dag (\sigma_{a'b'})^j\pi^{\dag}_{b'}}
		+\delta_{aa',bb'}\blr{\pi_a(\sigma_{ab})^i\phi_b,\pi_{a'}(\sigma_{a'b'})^j\phi_{b'}}&\\
		=\blr{\phi_a^\dag(\vect x) (\sigma_{ab})^i\pi^{\dag}_b(\vect x),\phi_{a}^\dag(\vect x')
		(\sigma_{ab})^j\pi^{\dag}_{b}(\vect x')}
		+\blr{\pi_a(\vect x)(\sigma_{ab})^i\phi_b(\vect x),\pi_{a}(\vect x')(\sigma_{ab})^j\phi_{b}(\vect x')}&\\
		=\phi^\dag_a\pi^\dag_b[(\sigma_{ab})^i,(\sigma_{ab})^j]+\pi_a\phi_b[(\sigma_{ab})^i,(\sigma_{ab}^j)]&\\
		=\plr{ \phi_a^\dag\pi_b^\dag+\pi_a\phi_b}[(\sigma_{ab})^i,(\sigma_{ab})^j] (wrong)&
	\ea
	\\ 
	Looking at part of the commutator
	\[
		[(\sigma_{ab})^i\phi_a(\vect x)\pi_b(\vect x),(\sigma_{ab})^j\phi_a(\vect x')\pi_b(\vect x')]
		=2i\delta^3(\vect x-\vect x')\phi_a\pi_b\sigma_{ab}^i\sigma_{ab}^j.
	\]
	The conjugate expression is then
	\[
		[(\sigma_{ab})^i\phi_a(\vect x)\pi_b(\vect x),(\sigma_{ab})^j\phi_a(\vect x')\pi_b(\vect x')]^\dag
		=-2i\delta^3(\vect x-\vect x')\pi_a^\dag\phi_b^\dag\sigma_{ab}^j\sigma_{ab}^i.
	\]	
	Back to the integral
	\[
		[Q^i,Q^j] = \int d^3x \pfrac{i}{2} \plr{ \phi_a^\dag\pi_b^\dag-\pi_a\phi_b}[(\sigma_{ab})^i,(\sigma_{ab})^j] = 
		\epsilon_{ijk}Q^k.
	\]
	In this form it should be easy to generalize to $n$ independent fields by replacing the sigma matrices
	by the general $SU(n)$ commutation relation
	\[
	 	\left[\lambda_i, \lambda_j \right] = i \sum_{k=1}^{n^2 -1}{f_{ijk} \lambda_k} \,    .
	\]
	where $f_{ijk}$ are the structure constants. As such, there will be $n^2$ conserved charges associated with $n$
	independent complex fields (at least for those due to the $SU(n)$) symmetry. 
	\eenum
	\eenum
\end{document}