\documentclass[11pt,letterpaper]{article}
\usepackage{macroshw}

\title{\begin{spacing}{1.2}Quantum Mechanics I\\HW 4\end{spacing}}
\author{Matthew Phelps}
\date{Due: March 26}

\begin{document}
\maketitle

\benum
% #1 --------------------------------------------------------------------------------------------------------------------------------------------------------------------------------------
  	\item 
 	 Consider the 3-dimensional Schrodinger equation
 	 \[
  		i\h\pdiff t\psi(\vect r,t) = H\psi(\vect r,t) = \blr{-\frac{\h^2}{2m}\plr{\pdifff{}{*2x}+\pdifff{}{*2y}+\pdifff{}{*2z}}+V(\vect r)}\psi(\vect r,t)
 	 \] 
	 For a given time independent potential $V(\vect r)$ the Hamiltonian $H$ possesses a complete set of stationary solutions $\chi_j(\vect r,t)
	 = \chi_i(\vect r)e^{-i\omega_j t}$ with energies $E_j = \h \omega_j$, where the eigenfunctions $\chi_j(\vect r)$ form an orthonormal set. In
	 terms of the Green's functions $G(\vect r',t',\vect r,t)$ any general wave function at time $t'$ may be related to that at time $t$ according 
	 to 
	 
	 \[
	 	\psi(\vect r',t') = i\int d^3x\,G(\vect r',t',\vect r,t)\psi(\vect r,t).
	\]
	
	\benum
		% (a)
		\item
		For the above Schrodinger equation show that the function 
		\[
			G(\vect r',t',\vect r,t) = -i\sum_j e^{-i \omega_j(t'-t)} \chi_j(\vect r')\chi_j^*(\vect r)
		\]
		can serve as the Green's function.
		\\
		\\
		On one hand, we can choose to substitute the given Green's function into the integral above and show that the RHS does indeed 
		reduce to the LHS.
		\[
			\psi(\vect r',t') = \int d^3\vect r\,\sum_j e^{-i \omega_j(t'-t)} \chi_j(\vect r')\chi_j^*(\vect r)\psi(\vect r,t).
		\]
		In dirac notation, this becomes
		\ba
			\braket{\vect x'|\psi,t;t'} &= \int d^3\vect x\, \sum_a e^{-\frac{iE_a}{\h}(t'-t)}\braket{\vect x'|a}
			\braket{a|\vect x}\braket{\vect x|\psi, t}\\
			&= \sum e^{-\frac{iE_a}{\h}(t'-t)}\braket{\vect x'|a}\braket{a|\psi,t}\\
			&= \bra{\vect x'}e^{\frac{-iH}{\h}(t'-t)}\plr{\sum \ket{a}\braket{a|\psi, t}}\\
			&= \braket{\vect x'|U(t',t)|\psi, t}\\
			& = \braket{\vect x'|\psi,t;t'}
		\ea
		where we have denoted the energy eigenkets $H\ket a = E_a\ket a$. I prefer dirac notation, but this relation could have also been
		shown by substituting 
		\[
			\psi(\vect r, t) = \sum_j c_j\chi_j(\vect r)e^{-i\omega_j t}
		\]
		into the integral and using the orthogonality condition
		\[
			\int d^3\vect r\, \chi_j(\vect r)\chi^*_i(\vect r)	= \delta_{ij}.
		\]		
		\\
		\\
		\item
		In terms of the Green's function it is convenient to define a \emph{retarded} Green's function which only propagates forward in time 
		(viz. $t'>t$) according to 
		\[
			G^+(\vect r',t',\vect r,t) = \theta(t'-t)G(\vect r',t',\vect r, t)
		\]
		where $\theta$ is the usual Heaviside unit step function: $\theta(\tau) = 1$, $\tau > 0$; $\theta(\tau) = 0$, $\tau < 0$ which can be 
		represented as the $\epsilon \to 0^+$ limit of the integral
		\[
			\theta(\tau) = -\frac{1}{2\pi i}\int_{-\infty}^{\infty} d\omega\, \frac{e^{i\omega \tau}}{\omega +i\epsilon}.
		\]
		Show that the Fourier transform of the retarded Green's function which is defined via
		\[
			G^+(\vect r',t',\vect r,t) = \frac{1}{2\pi}\int_{-\infty}^\infty d\omega\, G_\omega^+(\vect r',\vect r)e^{-i\omega(t'-t)}
		\]	
		can be written as
		\[
			G_\omega^+(\vect r',\vect r) = \sum_j\frac{\chi_i(\vect r')\chi_i^*(\vect r)}{\omega-\omega_j+i\epsilon}
		\]	
		where $\epsilon$ is small and positive.
		\\
		\\
		To find $G^+_\omega(\vect r',\vect r)$ we should be able to take the inverse transform of $G^+(\vect r',t',\vect r,t)$. That is
		\ba
			G^+_\omega(\vect r',\vect r) &= \frac{1}{2\pi}\int_{-\infty}^\infty d(t'-t)\,G^+(\vect r',t',\vect r,t)e^{i\omega(t'-t)}\\
			& =  \frac{1}{2\pi}\int_{-\infty}^\infty d(t'-t)\, \theta(t'-t)\plr{-i\sum_j e^{-i \omega_j(t'-t)} \chi_j(\vect r')\chi_j^*(\vect r)}e^{i\omega(t'-
			t)}.
		\ea
		where we notice that our Green's function is only a function of the difference in time $t'-t$, hence the choice of integration variable. 	
		From the step function, the integrand vanishes for $t'-t<0$ and is equal to unity for $t'-t>0$ thus we have
		\ba
			G^+_\omega(\vect r',\vect r) &=  \frac{1}{2\pi}\int_{0}^\infty d(t'-t)\, \plr{-i\sum_j e^{-i \omega_j(t'-t)} \chi_j(\vect r')
			\chi_j^*(\vect r)}e^{i\omega(t'-t)}\\
			&=\frac{-i}{2\pi}\sum_j \chi_j(\vect r')\chi_j^*(\vect r)\int_0^\infty d(t'-t)\,\exp\blr{i(\omega-\omega_j)(t'-t)}.
		\ea
		This is just an infinitely oscillating integrand that does not converge. We can however impose convergence by allowing $\omega$ to 
		be complex. Specifically, $\omega \to \omega +i\epsilon$ where $\epsilon$ is small and \emph{positive}. In effect, our oscillating 
		integrand is now modulated by a decaying exponential and thus will converge at infinity. 
		\ba
			&\int_0^\infty d(t'-t)\,\exp\blr{i(\omega+i\epsilon-\omega_j)(t'-t)}\\
			&= \elr{\frac{\exp\blr{i(\omega-\omega_j+i\epsilon)(t'-t)}}{i(\omega-\omega_j+i\epsilon)}}_0^\infty\\
			&= \frac{-1}{i(\omega-\omega_j+i\epsilon)}.
		\ea
		Substituting this result back in, for real $\omega$ we have
		\[
			G^+_\omega(\vect r',\vect r) = \lim_{\epsilon\to 0^+}\frac{1}{2\pi}\sum_j \frac{\chi_j(\vect r')\chi_j^*(\vect r)}{\omega - \omega_j+i
			\epsilon}.
		\]
		\\
		\emph{Could not figure out how to rid this $1/2\pi$.}
		\\
			
			
		% (c)
		\item
		Using the result of part (b), show that for a free particle case, where $V(\vect r) =0$, the retarded Green's function can be written in 
		the closed form 
		\[
			G^+(\vect r',t',\vect r, t) = -i\theta(t'-t)\blr{\frac{m}{2\pi i\h(t'-t)}}^{3/2}\exp{\blr{\frac{im|\vect r-\vect r'|^2}{2\h (t'-t)}}}.
		\]
		\\
		\\
		I couldn't quite figure out how to implement part (b), but here I will try a different method. The retarded Green's function is very 
		closely related to the propagator given in Sakurai (eq. 2.6.10) by
		\[
			G^+(\vect r',t',\vect r, t) = -i\theta(t'-t)K(\vect r',t',\vect r,t)
		\]
		where 
		\[
			K(\vect r',t',\vect r,t) = \sum_j e^{-i \omega_j(t'-t)} \chi_j(\vect r')\chi_j^*(\vect r).
		\]
		$K$ can also be cast into a very elegant form of
		\[
			K(\vect x',t',\vect x,t) = \braket{\vect x'|\exp\blr{\frac{-iH(t'-t)}{\h}}|\vect x}
		\]
		which is what we will use here. 
		\\
		\\
		For the free particle, the momentum states are eigenstates of the Hamiltonian. Therefore, we can easily form the retarded Green's 
		function in the momentum representation,
		\ba
			G^+(\vect p',t',\vect p,t) &= \theta(t'-t) \braket{\vect p'|\exp\blr{\frac{-iH(t'-t)}{\h}}|\vect p}\\
			&= \theta(t'-t) \exp\blr{\frac{-i\vect p\cdot\vect p(t'-t)}{2m\h}}\delta(\vect p'-\vect p)\\
			& = \theta(t'-t) \exp\blr{\frac{-i\vect p'\cdot\vect p'(t'-t)}{2m\h}}.
		\ea
		To find $G^+(\vect x',t',\vect x,t)$ we seek to find $\braket{\vect x'|\exp\blr{\frac{-iH(t'-t)}{\h}}|\vect x}$ which is related to $G^+$
		in the $\vect p$ representation by a Fourier transform. To show this
		\ba
			\braket{\vect x'|\exp\blr{\frac{-iH(t'-t)}{\h}}|\vect x} &= \int d\vect p\, \braket{\vect x'|\vect p}\braket{\vect p|
			\exp\blr{\frac{-iH(t'-t)}{\h}}|\vect x}\\
			& = \int d\vect p\, \braket{\vect x'|\vect p}\braket{\vect p|\vect x}\exp\blr{\frac{-i|\vect{p}|^2(t'-t)}{2m\h}} \\
			&=\frac{1}{(2\pi\h)^{3}}\int d\vect p\,  \exp\blr{\frac{i\vect p\cdot(\vect x'-\vect x)}{\h}}\exp\blr{\frac{-i|\vect{p}|^2(t'-t)}{2m
			\h}}\\
			&=\frac{1}{(2\pi)^{3}}\int d\vect k\, \exp\blr{i\vect k\cdot (\vect x'-\vect x)}\exp\blr{\frac{-i|\vect k|^2(t'-t)\h}{2m}}
		\ea
		
		Here we can see that this integral is in its Fourier form. For some reason I had to convert to wave-vector space in order for the $\h$
		in the second exponential argument to come out correctly. If we define 
		\[
			a^2 \equiv \frac{i\h(t'-t)}{2m};\quad \vect \xi \equiv p_i
		\]
		we can use the Fourier transform pair, defined in one dimension as
		\[
			\exp(-a^2x^2) = (a\sqrt 2)^{-1}\exp(-\xi^2/4a^2).
		\]
		When we apply the Fourier transform to all three dimensions, we have 
		\[
			\blr{\frac{m}{i\h(t'-t)}}^{3/2}\exp\blr{\frac{i(\vect x'-\vect x)^2 m}{2\h(t'-t)}}.
		\]
		Since the correct normalization constant for a 3-D Fourier transform is $\frac{1}{(2\pi)^{3/2}}$ we finish up with the correct answer of
		\[
			\braket{\vect x'|\exp\blr{\frac{-iH(t'-t)}{\h}}|\vect x} = \blr{\frac{m}{i2\pi\h(t'-t)}}^{3/2}\exp\blr{\frac{i|\vect x'-\vect x|^2 m}{2\h(t'-t)}}.
		\]
		Attaching the step function we finally have
		\[
			G(\vect r',t',\vect r,t) = -i\theta(t'-t)\blr{\frac{m}{i2\pi\h(t'-t)}}^{3/2}\exp\blr{\frac{i|\vect r'-\vect r|^2 m}{2\h(t'-t)}}.
		\]
	\eenum
		

% #2 -------------------------------------------------------------------------------------------------------------------------------------------------------------------------
	\item 
	The expectation value of an operator $\Omega(\vect r,\vect p)$ in a state $\psi(\vect r,t)$ is given by 
	\[
		\braket{\Omega} = \int d^3r\, \psi^*(\vect r,t)\,\Omega\,\psi(\vect r,t).
	\]
		
	\benum
		% (a)
		\item
		Show for any $\psi(\vect r, t)$ which obeys the Scrhodinger equation associated with the Hamiltonian $H = \frac{p^2}{2m} +V(\vect r)\
		$ with real $V(\vect r)$ that Ehrenfest's theorem holds, i.e.
		\[
			\diff t\braket{\vect r} = \frac{\braket{\vect p}}{m};\quad \diff t\braket{\vect p}=\braket{\vect F} = -\braket{\diff[V]{r}}.
		\]
		\\
		\\
		We can prove Ehrenfest's theorem in integral form, but I prefer the method via Heisenberg's equation of motion (Sakurai p. 86)
		\[
			\diff[A]{t} = \frac{1}{i\h}[A,H].
		\]
		For the Hamiltonian
		\[
			H = \frac{\vect p^2}{2m}+V(\vect x)
		\]
		we have
		\[
			\diff[\vect x]{t} = \frac{1}{i\h}\frac{1}{2m}[\vect x,\vect p^2] = \frac{1}{i\h}\frac{1}{2m}\del \vect p^2 [\vect x,\vect p] = \frac{\vect p}
			{m}
		\]
		\[
			\diff[\vect p]{t} = \frac{1}{i\h}[\vect p,V(\vect x)] = \frac{1}{i\h}\del V(\vect x) [\vect p,\vect x] = -\del V(\vect x)
		\]
		\[
			m\difff{\vect x}{*2t} = \diff[\vect p]{t}=-\del V(\vect x).
		\]
		Taking the expectation value of both sides with respect to a stationary Heisenberg state ket we then form
		\[
			m\difff{\braket{\vect x}}{*2t} = \diff[\braket{\vect p}]{t}=-\braket{\del V(\vect x)}
		\]
		which can now be interpreted in either the Schrodinger or Heisenberg picture since expectation values are equivalent in either 
		representation. 
		\\
		% (b)
		\item
		In the presence of electromagnetism an appropriate Hamiltonian is 
		\[
			H = \frac{(\vect p -e\vect A)^2}{2m} - e\phi
		\]
		where $\vect A(\vect r,t)$ and $\phi(\vect r,t)$ are vector and scalar electromagnetism potentials. Derive the appropriate Ehrenfest's 
		theorem for this case to explicitly obtain the Lorentz force law.
		\\
		\\
		Classically, since we have a velocity dependent potential the mechanical momentum is not equivalent to the canonical momentum.
		We expect the same to be true for the given Hamiltonian, which can easily be demonstrated by
		\[
			\frac{dx_i}{dt} = \frac{1}{i\h}[x_i,H] = \frac{(p_i-eA_i)}{m}.
		\] 
		To depict this difference, we denote the kinematical/mechanical momentum by
		\[
			\vect \Pi \equiv m\diff[\vect x]{t} = \vect p-e\vect A
		\]
		which have the commutation relation
		\[
			[\Pi_i,\Pi_j] = (i\h e)\epsilon_{ijk}B_k.
		\]
		We can use these to alternatively express our Hamiltonian as
		\[
			H = \frac{\vect \Pi^2}{2m}+e\phi.
		\]
		Okay, now onto finding the Lorentz force law. Beginning with
		\ba
			m\diff[x_i]{t} & = \frac{1}{i\h}[x_i, \frac 12\Pi_i^2+me\phi]\\
			& = \frac{1}{i\h}\Pi_i[x,\Pi_i]\\
			& = \Pi_i.
		\ea
		Continuing
		\ba
			m\difff{x_i}{*2t} &=  \diff[\Pi_i]{t} = \frac 1{i\h}[\Pi_i,\frac 1{2m} \sum_{j=1}^3\Pi_j^2+e\phi]- e \pdiff[A_i]{t}\\
			&= \frac{1}{2mi\h}[\Pi_i,\sum_{j=1}^3\Pi_j^2]+\frac{1}{i\h}[\Pi_i,e\phi(x,t)]- e\pdiff[A_i]{t}.
		\ea
		Lets look at the terms separately. First off we have
		\[
			\frac{1}{i\h}[\Pi_i,e\phi(x,t)] = \frac{e}{i\h}[p_i,\phi(x_i,t)] = \frac{e}{i\h}[p_i,x_i]\del_i\phi = -e\del_i\phi.
		\]
		Next up
		\ba
			\frac{1}{2mi\h}[\Pi_i,\sum_{j=1}^3\Pi_j^2] &= \sum_{j=1}^3\frac{1}{2mi\h}[\Pi_i,\Pi_j^2] \\
			&=  \sum_{j=1}^3\frac{1}{2mi\h}\plr{[\Pi_i,\Pi_j]\Pi_j+\Pi_j[\Pi_i,\Pi_j]}\\
			&= \sum_{j=1}^3\frac{e}{2m}(\epsilon_{ijk}B_k\Pi_j+\epsilon_{ijk}\Pi_jB_k)\\
			& =  \sum_{j=1}^3\frac{e}{2}(\epsilon_{ijk}B_k\diff[x_j]{t}+\epsilon_{ijk}\diff[x_j]{t}B_k)\\
			& = \frac e2 \plr{\diff[\vect x]{t}\times\vect B-\vect B\times\diff[\vect x]{t}}_i
		\ea
		Putting them all together we have
		\[
			m\difff{x_i}{*2t} =e\blr{-\del_i\phi -\pdiff[A_i]{t}+\frac 12 \plr{\diff[\vect x]{t}\times\vect B-\vect B\times\diff[\vect x]{t}}_i}
		\]
		We recognize the left term on the RHS as the $i$th component of the electric field. Therefore, we can write this in total as
		\[
			m\difff{\vect x}{*2t} = e\blr{\vect E+\frac 12 \plr{\diff[\vect x]{t}\times\vect B-\vect B\times\diff[\vect x]{t}}}
		\]
		and form the appropriate Ehrenfest theorem by taking the expectation values of each side ($\vect E$ and $\vect B$ are position 			dependent, therefore they too act as operators)
		\[
			m\difff{\braket{\vect x}}{*2t} = e\blr{\braket{\vect E}+\frac 12 \plr{\diff[\braket{\vect x}]{t}\times\braket{\vect B}-\braket{\vect B}
			\times\diff[\braket{\vect x}]{t}}}.
		\]
	\eenum
	
% #3 --------------------------------------------------------------------------------------------------------------------------------------------------------------------------
	\item 
	Consider the harmonic oscillator with
	\[
		L = \frac 12 m\dot{x}^2-\frac 12 m\omega^2x^2.
	\]
	For a general solution of classical equations of motion $x = A\cos(\omega t)+B\sin(\omega t)$, express the energy in terms of $A$ and
	$B$. Now choose energy such that $x(0) =x_1$ and $x(\tau) = x_2$, and write energy in terms of $x_1$, $x_2$ and $\tau$. Show that the 	action for the classical trajectory is 
	\[
		S_{cl}(x_1,x_2,\tau) = \frac{m\omega}{2\sin(\omega \tau)}\blr{(x_1^2+x_2^2)\cos(\omega \tau)-2x_1x_2}.
	\]
	Now show that the propagator for the quantum harmonic oscillator is given by
	\[
		K(x',t,x,0) = A(t)\exp{\clr{\frac{im\omega}{2\h\sin(\omega \tau)}\blr{(x_1^2+x_2^2)\cos(\omega \tau)-2x_1x_2}}}
	\]
	where $A(t)$ is a function of time.
	\\
	\\
	Using Euler-Lagrange equations
	\[
		\mathcal{L} = \frac 12 m\dot x^2-\frac 12 m\omega^2x^2;\quad \diff{t}\pdiff[\mathcal L]{\dot x}  = \pdiff[\mathcal L]{x}
	\]
	\[
		m\ddot x = -\omega^2 x
	\]
	\[
		x(t) = A\cos(\omega t)+B\sin(\omega t)
	\]
	\[
		\dot x(t) = -A\omega\sin(\omega t)+B\omega\cos(\omega t)
	\]
	\[
		E = \frac 12 m\dot x^2+\frac 12 m\omega^2x^2 = \frac 12 m \omega^2\plr{A^2+B^2}
	\]
	\[
		x(0) = A = x_1;
	\]
	\[
		x(\tau) = x_1\cos(\omega \tau)+B\sin(\omega \tau) = x_2
	\]
	\[
		B = \frac{x_2-x_1\cos(\omega \tau)}{\sin(\omega \tau)}
	\]
	\ba
		E &= \frac{1}{2}m\omega^2\blr{x_1^2+\frac{(x_2-x_1\cos(\omega \tau))^2}{\sin^2(\omega \tau)}}\\
		&= \frac{1}{2}m\omega^2\blr{\frac{x_1^2(\sin^2(\omega \tau)+\cos^2(\omega \tau))-2x_1x_2\cos(\omega\tau) +x_2^2}{\sin^2(\omega
		\tau)}}\\
		& = \frac{1}{2}m\omega^2\blr{\frac{x_1^2+x_2^2-2x_1x_2\cos(\omega \tau)}{\sin^2(\omega\tau)}}
	\ea
	To find the classical action, we need to evaluate the integral
	\ba
		\int_0^\tau dt\, \mathcal{L} &= \int_0^\tau dt\,  \blr{\frac 12 m\dot x_{cl}^2-\frac 12 m\omega^2x_{cl}^2}\\
		& = \frac{m\omega^2}{2}\int_0^\tau dt\, \clr{(A^2-B^2)[\sin^2(\omega t)-\cos^2(\omega t)]-4AB\sin(\omega t)\cos(\omega t)}\\
		& = \frac{m\omega}{2}\blr{(A^2-B^2)\sin(\omega \tau)\cos(\omega\tau)-2AB\sin^2(\omega \tau)}\\
		& = \frac{m\omega}{2}\plr{\frac{x_1^2(\sin^2(\omega\tau)-\cos^2(\omega\tau))-x_2^2+2x_1x_2\cos(\omega \tau)}{\sin(\omega
		\tau)}}\cos(\omega\tau)\\
		&\qquad+\frac{m\omega}{2}\plr{-2x_1x_2-2x_1^2\sin(\omega \tau)\cos(\omega\tau)}\\
		& = \frac{m\omega}{2\sin(\omega\tau)}[x_1^2\sin^2(\omega\tau)\cos(\omega\tau)-x_1^2\cos^3(\omega\tau)\\
		&\qquad-x_2^2\cos(\omega\tau)-2x_1x_2\cos^2(\omega\tau) - 2x_1x_2\sin^2(\omega\tau)+2x_1^2\sin^2(\omega\tau)\cos(\omega
		\tau)]\\
		& = \frac{m\omega}{2\sin(\omega\tau)}\blr{\plr{x_1^2+x_2^2}\cos(\omega\tau)-2x_1x_2}.
	\ea
	Therefore we have
	\[
		S_{cl}(x_1,x_2,\tau) = \frac{m\omega}{2\sin(\omega\tau)}\blr{(x_1^2+x_2^2)\cos(\omega\tau)-2x_1x_2}.
	\]
	\\
	\\
	To find the propagator of the harmonic oscillator
	\ba
		K(x'',t'';x',t') &= \braket{x''|\exp{\blr{-\frac{iH(t''-t')}{\h}}}|x'}\\
		& = \sum_a\braket{x''|\exp\blr{\frac{-Ht''}{\h}}|a}\braket{a|\exp\blr{\frac{iHt'}{\h}}|x'}\\
		& = \braket{x'',t''|x',t'}
	\ea
	we note that the last expression here is that of a path integral. That is
	\[
		\braket{x_N,t_n|x_1,t_1} = \int_{x_1}^{x_N}\mathcal{D}[x(t)]\,\exp\blr{\frac{i}{\h} \int_{t_1}^{t_N} dt\, \mathcal{L}_{classical}(x,\dot x)}
	\]
	where $\mathcal{D}[x(t)]$ is the path integral operator. Solving this path integral does not amount to simply substituting the classical action
	we computed earlier because we must integrate over all possible paths. However, the best way to accomplish this will be to express all
	paths relative to our classical trajectory:
	\[
		x(t) \to x_{cl}(t) +\eta(t),
	\]
	where the deviation from the classical trajectory $\eta(t)$ vanishes at the endpoints. As such, we can segregate our action integral as 
	follows:
	\ba
		\int_0^\tau dt\, \mathcal{L}_{classical} & = \int_0^\tau dt\, \blr{\frac{1}{2}m(\dot x_{cl}+\dot\eta)^2-\frac{1}{2}m\omega^2
		(x_{cl}+\eta)^2}\\
		& = \int_0^\tau dt\, \blr{\frac{1}{2}m\dot x_{cl}^2-\frac{1}{2}m\omega^2x_{cl}^2}+\int_0^\tau dt\, \blr{\frac{1}{2}m\dot \eta^2-\frac{1}{2}
		m\omega^2\eta^2}\\
		&\quad + \int_0^\tau dt\, m\dot\eta\dot x_{cl}-m\omega^2\eta x_{cl}
	\ea 
	Integration by parts on the last integral yields
	\[
		\int_0^\tau dt\, m\dot\eta\dot x_{cl}-m\omega^2\eta x_{cl} = \elr{m\eta\dot x_{cl}}_0^\tau -\int_0^\tau dt\, m\ddot x_{cl}\eta -\int_0^\tau
		dt\,m\omega^2x_{cl}\eta.
	\]
	Since $\eta(t)$ vanishes at the endpoints, we are left with
	\[
		\int_0^\tau dt\, \blr{\frac{1}{2}m\dot x_{cl}^2-\frac{1}{2}m\omega^2x_{cl}^2}+\int_0^\tau dt\, \blr{\frac{1}{2}m\dot \eta^2-\frac{1}{2}
		m\omega^2\eta^2} + \int_0^\tau dt\, \eta(m\ddot x_{cl}-m\omega^2x_{cl}).
	\]
	Interestingly, the last integrand is the classical equation of motion for the harmonic oscillator, which we know must be precisely zero. As 
	such, our action separates into a classical action (which we already computed) and that of the deviation from it
	\[
		\int_0^\tau dt\, \mathcal{L}_{classical} = \int_0^\tau dt\, \blr{\frac{1}{2}m\dot x_{cl}^2-\frac{1}{2}m\omega^2x_{cl}^2}+\int_0^\tau dt\, 	
		\blr{\frac{1}{2}m\dot \eta^2-\frac{1}{2}m\omega^2\eta^2}.
	\]
	Our path integral is now
	\[
		\braket{x_2,\tau|x_1,0} = \exp\blr{\frac{i}{\h}S_{classical}}\int_{\eta(0)}^{\eta(\tau)}\mathcal{D}[\eta(t)]\,\exp\blr{\frac{i}{\h} \int_{0}^{\tau} 
		dt\, \plr{\frac{1}{2}m\dot \eta^2-\frac{1}{2}m\omega^2\eta^2}}.
	\]
	By factoring out our classical action, we notice that the remaining path integral is independent of position. Rather it is a function of time 
	$\tau$ only (with an $m$ and $k$ as well). Therefore, we can denote
	\[
		A(\tau) \equiv \int_{\eta(0)}^{\eta(\tau)}\mathcal{D}[\eta(t)]\,\exp\blr{\frac{i}{\h} \int_{0}^{\tau} dt\, \plr{\frac{1}{2}m\dot \eta^2
		-\frac{1}{2}m\omega^2\eta^2}}.
	\]
	In conclusion, we can now express the propagator of the harmonic oscillator as
	\[
		K(x_2,\tau;x_1,0) =  A(\tau)\exp{\clr{\frac{im\omega}{2\h\sin(\omega \tau)}\blr{(x_1^2+x_2^2)\cos(\omega \tau)-2x_1x_2}}}
	\]
% #4 ---------------------------------------------------------------------------------------------------------------------------------------------------------------------------
	\item
	Sakurai 2.28: Consider an electron confined to the \emph{interior} of a hollow cylindrical shell whose axis coincides with the $z$-axis. The 
	wave function is required to vanish on the inner and outer walls, $\rho = \rho_a$ and $\rho_b$, and also at the top and bottom, $z=0$ and 
	$L$.
	 
	 \benum
	 	% (a)
	 	\item
		Find the energy eigenfunctions. (Do not bother with normalization.) Show that the energy eigenvalues are given by 
                \[
                		E_{lmn} = \left(\frac{\h^2}{2m_e}\right)\left[k^2_{mn}+\left(\frac{l\pi}{L}\right)^2\right]\quad(l=1,2,3,..\ m=0,1,2,..)
		\]
                where $k_{mn}$ is the $n$th root of the transcendental equation
                \[
                		J_m(k_{mn}\rho_b)N_m(k_{mn}\rho_a)-N_m(k_{mn}\rho_b)J_m(k_{mn}\rho_a)=0.
		\]
                \\ 
                \\ 
                In this problem, we have no potential and thus the TISE will have the form
                \[
                		-\frac{\h^2}{2m}\del^2\psi -E\psi = 0
		\]
                which can easily be cast into the form of the Helmholtz equation, namely
                \[
                		\del^2\psi +j^2\psi = 0
		\]
                where 
                \[
                		j\equiv \sqrt{\frac{2mE}{\h^2}}.
		\]
                Our task amounts to solving the TISE for the specified boundary conditions. To solve this particular PDE (cf. Arfken p. 421) we first 
                write it down in cylindrical coordinates:
                \[
                		\frac 1\rho \frac{\partial}{\partial \rho}\left(\rho\frac{\partial \psi}{\partial\rho}\right)+\frac{1}{\rho^2}\frac{\partial^2\psi}{\partial
			\phi^2}+ \frac{\partial^2\psi}{\partial z^2}+j^2\psi = 0.
		\]
                Observing that our differential operators are all additive, we try separation of variables,
                \[
                		\psi(\rho,\phi,z) = P(\rho)\Phi(\phi)Z(z).
		\]
                Substituting this in, we have
                \[
                		\frac {\Phi Z}{\rho} \frac{d}{d \rho}\left(\rho\frac{dP}{d\rho}\right)+\frac{PZ}{\rho^2}\frac{d^2\Phi}{d\phi^2}+P\Phi\frac{d^2Z}
			{dz^2}+j^2P\Phi Z = 0
		\]
                \[
                		\frac{1}{\rho P}\frac{d}{d\rho}\left(\rho\frac{dP}{d\rho}\right)+\frac{1}{\rho^2\Phi}\frac{d^2\Phi}{d\phi^2}+j^2 = -\frac{1}{Z}
			\frac{d^2Z}{dz^2}.
		\]
                Since we have isolated some function $f(z)$ on the RHS, we choose to set it equal to $h^2$. I don't like the choices of separation 
                variables here, but to conform to the final answer given in the text we will have to use some non-customary ones. The decision to 
                choose $h^2$ to be positive requires some foresight on our boundary conditions. First observe that 
                \[
                		\frac{d^2Z}{dz^2} = \pm h^2Z.
		\]
                A positive coeffecient leads to a hyperbolic function while a negative leads to an oscillating exponential. Since the B.C.'s require 
                $Z(z)$ to vanish at \emph{two} faces of the cylinder, we can certainly rule out the hyperbolic functions. Hence the choice of $h^2$ on 
                the RHS of the D.E. As such we now have
                \[
                		Z(z)\propto e^{\pm ihz}.
		\]
                If we now denote a new constant 
                \[
                		k^2\equiv j^2-h^2
		\]
                we can reform our equation as
                \[
                		\frac{\rho}{P}\frac{d}{d\rho}\left(\rho\frac{dP}{d\rho}\right)+k^2\rho^2 = -\frac{1}{\Phi}\frac{d^2\Phi}{d\phi^2}.
		\]
                Lets use the separation constant $m^2$ for this one, in which we have
                \[
                		\frac{d^2\Phi}{d\phi^2} = -m^2\Phi;\quad\Phi(\phi)\propto e^{\pm im\phi}.
		\]
                In using $\phi$ as the azimuthal angle, we impose the B.C. of periodicity and find that $m$ must be an integer:
                \[
                		\Phi(\phi) = \Phi(\phi+2\pi)\rightarrow e^{\pm im\phi}=e^{\pm i(m+2\pi)\phi} = e^{\pm im\phi}+e^{\pm im2\pi}.
		\]
                For the $\rho$ dependence we are left with 
                \begin{equation}\label{bde}\rho\frac{d}{d\rho}\left(\rho\frac{dP}{d\rho}\right)+(k^2\rho^2-m^2)P = 0.
                \end{equation}
                If we use a change of variable $x\equiv k\rho,$ this becomes Bessel's differential equation! The solutions to this equation are of 
                course the Bessel functions of order $m$. The Bessel functions of the first, second, and third kind are denoted $J_m(x)$, $N_m(x)$, 
                and $H_m(x)$ respectively. The last two are at times referred to as the Neumann and Hankel functions. The solution to our D.E. can 
                formed as 
                \[
                		P(x) = C_1J_m(x)+C_2N_m(x)
		\]
                in which it can be shown that $J_m(x)$ and $N_m(x)$ are linearly independent. All together we now have
                \[
                		\psi(\rho,\phi,z) = (C_1J_m(k\rho)+C_2N_m(k\rho))e^{\pm ihz}e^{\pm im\phi}
		\]
                remembering that
                \[
                		j^2-h^2 = k^2\quad\text{and}\quad m=0,1,2,3,..
		\]
                Our boundary conditions state that $\psi$ must vanish at all surfaces of our cylinder:
              	  \begin{enumerate}
               		 \item $\psi(\rho,\phi,0) = \psi(\rho,\phi,L) = 0$
              		  \item $\psi(\rho_a,\phi,z)=\psi(\rho_b,\phi,z) = 0$
              	  \end{enumerate}
               	We may want to note that we expect the wavefunction to be symmetric in $\phi$ from the cylindrical symmetry present. To be 	
		precise, the square of the probability amplitude should be constant in $\phi$
		\[
                		|\Phi(\phi)|^2 = C.
		\]
		This can be accomplished if we specifically choose $e^{+im\phi}$ \emph{or} $e^{-im\phi}$ but not a linear combination of 
		both. Moving onto the first B.C listed. our function for $z$ goes as
                \[
                		Z(z) = C_4e^{ihz}+C_5e^{-ihz}.
		\]
                In order to satisfy both B.C.'s, we will have to quantize $h$ and form a single trig function. That is,
                \[
                		C_4 = -C_5;\quad h = \frac{\pi l}{L}\quad\text{where}\quad l=0,1,2,3..
		\]
                which yields 
                \[
                		Z(z) = C_4\sin\left(\frac{l\pi}{L}z\right).
		\]
                For the last boundary condition, we simply need
                \[
                		P(k\rho_a) = P(k\rho_b) = 0
		\]
                or
                \[
                		C_1J_m(k\rho_a)+C_2N_m(k\rho_a) = 0
		\]
                \[
                		C_1J_m(k\rho_b)+C_2N_m(k\rho_b) = 0.
		\]
                To satisfy these conditions, either we must fix $k$ to use the zero's of the Bessel functions or we must vary the coefficients. Since 
                the zero's would be different for each equation (and for Bessel eqs. of the second kind), we conclude that we must vary the 
                coefficients. For a non-zero solution of these two homogenous equations (functions of $C_1$ and $C_2$) to exist, the determinant 
                must vanish (one equation must be linearly dependent on the other)
                \[
                		J_m(k\rho_b)N_m(k\rho_a)-J_m(k\rho_a)N_m(k\rho_b) = 0.
		\]
                Under what conditions will this vanish? The only factor we have not yet fixed is $k$ and so if $k$ is a root to this transcendental 
                equation, then the determinant will vanish. In fact, we expect that there may be many roots to such an equation and so we can 
                denote them as $k_{mn}$ where $n$ is the $n$th root and $m$ is the integer order of the Bessel functions. Going back to our B.C.'s 
                we have
                \[
                		C_1J_m(k_{mn}\rho_a)+C_2N_m(k_{mn}\rho_a) = 0
		\]
                \[
                		C_1J_m(k_{mn}\rho_b)+C_2N_m(k_{mn}\rho_b) = 0.
		\]
                Due to the linear dependence we created, the second equation is simply a constant multiple of the first; satisfying the first equation 
                automatically takes care of the second. A feature of our homogenous system is that we have one equation and two unknowns. Thus 
                our solutions will be parameterized (although here we will simply choose a constant and not deal with a parameterization variable, 
                as we expect normalization to fix it anyway). Lets denote
                \[
                		C_1 = A
		\]
                in which we see that
                \[
                		C_2 = -A\frac{J_m(k_{mn}\rho_a)}{N_m(k_{mn}\rho_a)}.
		\]
                We now have the wavefunction. As for energy, remember that
                \[
                		j\equiv\sqrt{\frac{2mE}{\h^2}};\quad j^2-h^2 = k^2
		\]
                and so we have
                \[
                		\frac{2mE}{\h^2} = \left(\frac{l\pi }{L}\right)^2+k_{mn}^2
		\]
                \[
                		 E = \left(\frac{\h^2}{2m_e}\right)\left[k_{mn}^2+\left(\frac{l\pi}{L}\right)^2\right].
		\]
                Finally we can express everything in full form as 
                \[
                		\psi(\rho,\phi,z) = \left[AJ_m(k_{mn}\rho)-A\tfrac{J_m(k_{mn}\rho_a)}{N_m(k_{mn}\rho_a)}N_m(k_{mn}\rho)\right]\sin\left(\frac{l	
			\pi}{L}z\right)e^{\pm im\phi}
		\]
                \[
                		E = \left(\frac{\h^2}{2m_e}\right)\left[k_{mn}^2+\left(\frac{l\pi}{L}\right)^2\right]
		\]
                where 
                \[
                		l=0,1,2,3,..\quad m=0,1,2,3,..
		\]
                and $k_{mn}$ is the $n$th root of the transcendental equation
                \[
                		J_m(k_{mn}\rho_b)N_m(k_{mn}\rho_a)-J_m(k_{mn}\rho_a)N_m(k_{mn}\rho_b) = 0.
		\]
                
                % (b)
                \item
                Repeat the same problem when there is a uniform magnetic field $\vect B = B\vecth z$ for $0<\rho<\rho_a$. Note that the energy 
                eigenvalues are influenced by the magnetic field even though the electron never "touches" the magnetic field. 
                \\
		\\
		We start with the Hamiltonian for a time-independent magnetic field (no $\vect E$ field) defined as
		\[
			H = \frac{1}{2m_e}\plr{\vect p-\frac{e\vect A}{c}}^2.
		\]
		In interpreting such a Hamiltonian, we must remember that we require it to be hermitian. Therefore it can be written as
		\[
			H = \frac{1}{2m_e}\blr{p^2-\frac ec\plr{\vect p\cdot\vect A+\vect A\cdot\vect p}+\pfrac{e}{c}^2A^2}.
		\]
		Starting with the TISE we have
		\[
			\braket{\vect x|H|\psi} = E\braket{\vect x|\psi}.
		\]
		Putting it into wavefunction form
		\ba
			\braket{\vect x|H|\psi} &= \frac{1}{2m_e}\braket{\vect x|\blr{p^2-\frac ec\plr{\vect p\cdot\vect A+\vect A\cdot\vect p}+
			\pfrac{e}{c}^2A^2}|\psi}\\
			&= \frac{1}{2m_e}\blr{-\h^2\del^2+\pfrac{e}{c}^2A^2}\braket{\vect x|\psi}-\frac{e}{c}\braket{\vect x|\vect p\cdot\vect A+\vect A
			\cdot\vect p|\psi}\\
			&= \frac{1}{2m_e}\blr{-\h^2\del^2+\pfrac{e}{c}^2A^2+i\h\frac{e}{c}\plr{\del\cdot\vect A+\vect A\cdot\del}}\braket{\vect x|\psi}
		\ea
		where 
		\[
			\vect A = \pfrac{B\rho_a^2}{2\rho}\vecth \phi
		\]
		is the vector potential appropriate for $\vect B = B\vecth z$, which is given by Sakurai 2.7.62 and can be obtained via Stokes law. 
		In cylindrical coordinates, the Hamiltonian operator becomes
		\ba
			H &= \frac{1}{2m_e}\blr{-\h^2\del^2+\pfrac{eB\rho_a^2}{2c\rho}^2+i\h\frac{e}{c}\plr{\frac{1}{\rho}\pdiff{\phi}\pfrac{B\rho_a^2}
			{2\rho}+\pfrac{B\rho_a^2}{2\rho}\frac{1}{\rho}\pdiff{\phi}}}\\
			& = \frac{1}{2m_e}\blr{-\h^2\del^2+\pfrac{eB\rho_a^2}{2c\rho}^2+i\h\frac{e}{c}\pfrac{B\rho_a^2}{\rho^2}\pdiff{\phi}}\\
			& =  \frac{-\h^2}{2m_e}\blr{\frac 1\rho \frac{\partial}{\partial \rho}\left(\rho\frac{\partial }{\partial\rho}\right)+\frac{1}
			{\rho^2}\frac{\partial^2}{\partial\phi^2}+ \frac{\partial^2}{\partial z^2}}+\frac{1}{2m_e}\pfrac{eB\rho_a^2}{2c\rho}^2+\frac{i\h}
			{2m_e}\frac{e}{c}\pfrac{B\rho_a^2}{\rho^2}\pdiff{\phi}.
		\ea
		Notice that since $\vect A$ does not depend on $\phi$, $\del\cdot\vect A = \vect A\cdot\vect\del$. Now for the full Schrodinger 		
		equation
		\[
			H\psi = E\psi
		\]
		\[
			\frac{-\h^2}{2m_e}\blr{\frac 1\rho \frac{\partial}{\partial \rho}\left(\rho\frac{\partial\psi}{\partial\rho}\right)+\frac{1}
			{\rho^2}\frac{\partial^2\psi}{\partial\phi^2}+ \frac{\partial^2\psi}{\partial z^2}}+\frac{1}{2m_e}\pfrac{eB\rho_a^2}{2c\rho}^2\psi
			+\frac{i\h}{2m_e}\frac{e}{c}\pfrac{B\rho_a^2}{\rho^2}\pdiff[\psi]{\phi}=E\psi
		\]
		\[
			\frac 1\rho \frac{\partial}{\partial \rho}\left(\rho\frac{\partial\psi}{\partial\rho}\right)+\frac{1}
			{\rho^2}\frac{\partial^2\psi}{\partial\phi^2}+ \frac{\partial^2\psi}{\partial z^2}-\frac{1}{\h^2}\pfrac{eB\rho_a^2}{2c\rho}^2\psi+
			\frac{2m_e}{\h^2}E\psi-\frac{i}{\h}\frac{e}{c}\pfrac{B\rho_a^2}{\rho^2}\pdiff[\psi]{\phi}=0
		\]
		\[
			\frac 1\rho \frac{\partial}{\partial \rho}\left(\rho\frac{\partial\psi}{\partial\rho}\right)+\frac{1}
			{\rho^2}\frac{\partial^2\psi}{\partial\phi^2}+ \frac{\partial^2\psi}{\partial z^2}-\blr{\frac{1}{\h^2}\frac{\alpha^2}{\rho^2}-
			\frac{2m_e}{\h^2}E}\psi-\frac{i}{\h}\frac{2\alpha}{\rho^2}\pdiff[\psi]{\phi}=0
		\]
		where for convenience we denote
		\[
			\alpha \equiv \frac{e}{c}\pfrac{B\rho_a^2}{2}.
		\]
		Our only hope for solving this system is with separation of variables, which should be valid since no partial derivatives are coupled
		together. We try a solution of the form
		\[
			\psi(\rho,\phi,z) = P(\rho)\Phi(\phi)Z(z)
		\]
		in which our TISE becomes
		\ba
			\frac {\Phi Z}{\rho} \frac{\partial}{\partial \rho}\left(\rho\frac{\partial P}{\partial\rho}\right)+\frac{PZ}
			{\rho^2}\frac{\partial^2\Phi}{\partial\phi^2}+ P\Phi\frac{\partial^2Z}{\partial z^2}-\blr{\frac{1}{\h^2}\frac{\alpha^2}{\rho^2}-
			\frac{2m_e}{\h^2}E}P\Phi Z\\
			-\frac{i}{\h}PZ\pfrac{2\alpha}{\rho^2}\pdiff[\Phi]{\phi}=0.
		\ea
		
		Now we divide by $\psi$
		
		\ba
			\frac {1}{\rho P} \frac{\partial}{\partial \rho}\left(\rho\frac{\partial P}{\partial\rho}\right)+\frac{1}
			{\rho^2\Phi}\frac{\partial^2\Phi}{\partial\phi^2}+ \frac{1}{Z}\frac{\partial^2Z}{\partial z^2}-\frac{1}{\h^2}\frac{\alpha^2}{\rho^2}
			+\frac{2m_e}{\h^2}E\\
			-\frac{i}{\h}\frac{1}{\Phi}\pfrac{2\alpha}{\rho^2}\pdiff[\Phi]{\phi}=0
		\ea
		and isolate $Z$
		\[
			\frac {1}{\rho P} \frac{\partial}{\partial \rho}\left(\rho\frac{\partial P}{\partial\rho}\right)+\frac{1}
			{\rho^2\Phi}\frac{\partial^2\Phi}{\partial\phi^2}-\frac{i}{\h}\frac{1}{\Phi}\frac{2\alpha}{\rho^2}\pdiff[\Phi]{\phi} -\frac{1}
			{\h^2}\frac{\alpha^2}{\rho^2}+\frac{2m_e}{\h^2}E = -\frac{1}{Z}\frac{\partial^2Z}{\partial z^2}.
		\]
		Setting the separation constant equal to $h^2$ we have for $Z(z)$
		\[
			-\frac{1}{Z}\frac{\partial^2Z}{\partial z^2} = h^2;\quad \difff{Z}{*2z} = -h^2 Z;\quad Z \propto e^{\pm ihz}.
		\]
		We chose $h^2$ such that we get a trigonometric form, just like part (a). Now we have 
		\[
			\frac {1}{\rho P} \frac{\partial}{\partial \rho}\left(\rho\frac{\partial P}{\partial\rho}\right)+\frac{1}
			{\rho^2\Phi}\frac{\partial^2\Phi}{\partial\phi^2}-\frac{i}{\h}\frac{1}{\Phi}\frac{2\alpha}{\rho^2}\pdiff[\Phi]{\phi} -\frac{1}
			{\h^2}\frac{\alpha^2}{\rho^2}+\frac{2m_e}{\h^2}E - h^2 = 0.
		\]
		Lets multiply by $\rho^2$
		\[
			\frac {\rho}{P} \frac{\partial}{\partial \rho}\left(\rho\frac{\partial P}{\partial\rho}\right)+\frac{1}
			{\Phi}\frac{\partial^2\Phi}{\partial\phi^2}-\frac{i}{\h}\frac{2\alpha}{\Phi}\pdiff[\Phi]{\phi} -\frac{\alpha^2}
			{\h^2}+\rho^2\plr{\frac{2m_e}{\h^2}E - h^2} = 0
		\]
		and isolate $\phi$
		\[
			\frac {\rho}{P} \frac{\partial}{\partial \rho}\left(\rho\frac{\partial P}{\partial\rho}\right) - \frac{\alpha^2}
			{\h^2}+\rho^2\plr{\frac{2m_e}{\h^2}E - h^2} = -\frac{1}
			{\Phi}\frac{\partial^2\Phi}{\partial\phi^2}+\frac{i}{\h}\frac{2\alpha}{\Phi}\pdiff[\Phi]{\phi}.
		\]
		Let's denote the separation constant by $b^2$. Contrary to part(a) we now have a different differential equation for $\Phi(\phi)$:
		\[
			-\frac{1}{\Phi}\frac{d^2\Phi}{d\phi^2}+\frac{i}{\h}\frac{2\alpha}{\Phi}\diff[\Phi]{\phi}= b^2
		\]
		\[
			\frac{d^2\Phi}{d\phi^2}-\frac{i}{\h}2\alpha\diff[\Phi]{\phi}+ b^2\Phi=0.
		\]
		This can be solved by an exponential 
		\[
			\Phi(\phi) \propto e^{\pm ia\phi}
		\]
		Starting with $+ia$
		\[
			-a^2+ a\frac{2\alpha}{\h}+b^2 = 0;\quad a^2-a\frac{2\alpha}{\h}-b^2 = 0.
		\]
		Solving for $a$ we have
		\ba
			a &= \frac{1}{2}\plr{\frac{2\alpha}{\h}\pm\sqrt{\pfrac{2\alpha}{\h}^2+4b^2}}\\
			& = \frac{\alpha}{\h}\pm\sqrt{\alpha^2/\h^2+b^2}.
		\ea
		Requiring periodic boundary conditions
		\[
			e^{\pm ia\phi} = e^{\pm ia(\phi+2\pi)}
		\]
		\[
			e^{ia2\pi} = 1 \to 2\pi a = 2\pi j\quad\text{for}\quad j = 0,\pm1,\pm2,\pm3..,
		\]
		which allows us to solve for $b^2$ as
		\[
			j = \frac{\alpha}{\h}\pm\sqrt{\alpha^2/\h^2+b^2}
		\]
		\[
			\plr{j-\frac{\alpha}{\h}}^2 = \frac{\alpha^2}{\h^2}+b^2
		\]
		\[
			b^2 = j^2 -2j\frac{\alpha}{h}.
		\]
		We can express $a$ in terms of $j$ now as
		\[
			a = \frac{\alpha}{\h}\pm\sqrt{\alpha^2/\h^2+j^2 -2j\frac{\alpha}{h}}\quad\text{for}\quad j = 0,\pm1,\pm2,\pm3..
		\]
		As before, we must choose either $e^{+ia\phi}$ or $e^{-ia\phi}$. Thus our solution for $\Phi(\phi)$ is then
		\[
			\Phi(\phi) \propto \exp\blr{\pm i \plr{\frac{\alpha}{\h}\pm\sqrt{\alpha^2/\h^2+j^2 -2j\frac{\alpha}{h}}}\phi}.
		\]
		Back to the TISE we have
		\[
			\frac {\rho}{P} \frac{\partial}{\partial \rho}\left(\rho\frac{\partial P}{\partial\rho}\right) - \frac{\alpha^2}
			{\h^2}+\rho^2\plr{\frac{2m_e}{\h^2}E - h^2} -j^2 +2j\frac{\alpha}{h}= 0 
		\]
		\[
			\frac {\rho}{P} \frac{\partial}{\partial \rho}\left(\rho\frac{\partial P}{\partial\rho}\right)+\rho^2\plr{\frac{2m_e}{\h^2}E - h^2}
			-\plr{j-\frac{\alpha}{h}}^2= 0 
		\]
		\[
			\frac {\rho}{P} \frac{\partial}{\partial \rho}\left(\rho\frac{\partial P}{\partial\rho}\right)+\rho^2\plr{\frac{2m_e}{\h^2}E - h^2}
			-\plr{j-\frac{\alpha}{h}}^2= 0 .
		\]
		As a function of $\rho$ only, we seek to bring this to a form of Bessel's equation. Thus we denote
		\[
			k^2 \equiv \frac{2m_e}{\h^2}E-h^2;\quad m \equiv j-\frac{\alpha}{\h}.
		\]
		Now multiply by $P$
		\[
			\rho^2 \frac{\partial^2 P}{\partial \rho^2}+\rho\frac{\partial P}{\partial\rho}+P(\rho^2k^2-m^2)= 0 .
		\]
		If we make the substitution 
		\[
			x\equiv \rho k;\quad \to \quad \pdiff{\rho} = \pdiff{x}\pdiff[x]{\rho}=\pdiff{x}k
		\]
		we successfully arrive at Bessel's equation:
		\[
			x^2 \frac{\partial^2 P}{\partial x^2}+x\frac{\partial P}{\partial x}+P(x^2-m^2)= 0 .
		\]
		Notice that $m$ is not an integer here. Despite this, the solutions can still be written as
                \[
                		\psi(\rho,\phi,z) = (C_1J_m(k\rho)+C_2N_m(k\rho))e^{\pm ihz}e^{\pm ia\phi}.
		\]
		Nowe we impose boundary conditions:
		  \begin{enumerate}
               		 \item $\psi(\rho,\phi,0) = \psi(\rho,\phi,L) = 0$
              		  \item $\psi(\rho_a,\phi,z)=\psi(\rho_b,\phi,z) = 0$
              	  \end{enumerate}
		For the first B.C. our function for $z$ goes as
                \[
                		Z(z) = C_4e^{ihz}+C_5e^{-ihz}.
		\]
                Just as before, we will have to quantize $h$ and form a single trig function. That is,
                \[
                		C_4 = -C_5;\quad h = \frac{\pi l}{L}\quad\text{where}\quad l=0,1,2,3..
		\]
                which yields 
                \[
                		Z(z) = C_4\sin\left(\frac{l\pi}{L}z\right).
		\]
                For the last boundary condition, we need
                \[
                		P(k\rho_a) = P(k\rho_b) = 0
		\]
                or
                \[
                		C_1J_m(k\rho_a)+C_2N_m(k\rho_a) = 0
		\]
                \[
                		C_1J_m(k\rho_b)+C_2N_m(k\rho_b) = 0.
		\]
		As before, we require the determinant to vanish
		\[
                		J_m(k\rho_b)N_m(k\rho_a)-J_m(k\rho_a)N_m(k\rho_b) = 0.
		\]
		We can denote the roots to this transcendental equation as $k_{mn}$ where $n$ is the $n$th root and $m$ is the noninteger order 
		of the Bessel functions. This yields our energy spectrum. Since the determinant is forced to be linearly dependent we denote
                \[
                		C_1 = A
		\]
                in which we see that
                \[
                		C_2 = -A\frac{J_m(k_{mn}\rho_a)}{N_m(k_{mn}\rho_a)}.
		\]
		We can now write our wavefunction as 
		\[
                		\psi(\rho,\phi,z) = \left[AJ_m(k_{mn}\rho)-A\tfrac{J_m(k_{mn}\rho_a)}{N_m(k_{mn}\rho_a)}N_m(k_{mn}\rho)\right]\sin\left(\frac{l	
			\pi}{L}z\right)e^{\pm ia\phi}
		\]
		where 		
		\ba
			a &= \frac{\alpha}{\h}\pm\sqrt{\alpha^2/\h^2+j^2 -2j\frac{\alpha}{h}};\quad \alpha \equiv \frac{e}{c}\pfrac{B\rho_a^2}{2}\\
			m& = j-\frac{e}{c}\frac{B\rho_a^2}{2\h};\quad\text{for}\quad j = 0,\pm1,\pm2,\pm3..
		\ea
		
		To find our energy eigenvalues we recall our relations,
		\[
			k^2 \equiv \frac{2m_e}{\h^2}E-h^2;\quad h = \frac{\pi l}{L}\quad\text{where}\quad l=0,1,2,3..
		\]
		Therefore
		\[
                		E = \left(\frac{\h^2}{2m_e}\right)\left[k_{mn}^2+\left(\frac{l\pi}{L}\right)^2\right]
		\]
		where once again $k_{mn}$ is the $n$th root of the transcendental equation of order $m$ defined by
		\[
			m = j-\frac{e}{c}\frac{B\rho_a^2}{2\h}\quad \text{where}\quad j = 0,\pm1,\pm2,\pm3,...
		\]
		The energy eigenvalues appear to be same as before, however the disparity in the order $m$ of the Bessel function has caused a 
		definite shift in the energy spectrum. 
		\\
                % (c)
                \item
                Compare, in particular, the ground state of the $B=0$ problem with that of the $B\ne 0$ problem. Show that if we require the ground-
                state energy energy to be unchanged in the presence of $B$, we obtain ``flux quantization" 
                \[
                		\pi\rho_a^2B = \frac{2\pi N\h c}{e},\quad (N = 0,\pm 1,\pm 2,...).
		\]
		\\
		The ground state of part (a) is given by the smallest root corresponding to the $m=0$ order transcendental equation (in terms of $
		\phi$, this corresponds to the smallest number of oscillations). In order for the energy to be unchanged in the presence of the 
		magnetic field, we require $m=0$ for the energy eigenvalues of part (b). Once $m=0$ is satisfied, the roots will be equivalent and 			so the energies will also be the same. Our requirement is
		\[
			0=m = j-\frac{e}{c}\frac{B\rho_a^2}{2\h}\quad \text{where}\quad j = 0,\pm1,\pm2,\pm3,...
		\]
		or
		\[
			\pi\rho_a^2B = \frac{2\pi N\h c}{e},\quad (N = 0,\pm 1,\pm 2,...).
		\]
		If the magnitude flux is equal to these quantized values, the ground state energies with and without the magnetic field will coincide. 
		
	\eenum
% 5 ---------------------------------------------------------------------------------------------------------------------------------------------------------------------------------------
	\item
	Sakurai 2.39: An electron moves in the presence of a uniform magnetic field in the $z$-direction $(\vect B = B\vecth z)$.
	
	\benum
		% (a)
		\item
		Evaluate
		\[
			[\Pi_x,\Pi_y],
		\]
		where
		\[
			\Pi_x \equiv p_x-\frac{eA_x}{c},\quad\Pi_y \equiv p_y-\frac{eA_y}{c}
		\]
		\\
		\\
		Using the relations
		\[
			[\Pi_i,\Pi_j] = \pfrac{i\h e}{c}\epsilon_{ijk}B_k
		\]
		we can easily see that the only non-zero commutator for $\vect B = B\vecth z$ is
		\[
			[\Pi_x,\Pi_y]  = \pfrac{i\h e}{c}B.
		\]
		% (b)
		\item 
		By comparing the Hamiltonian and the commutation relation obtained in (a) with those of the one-dimensional oscillator problem, 
		show how we can immediately write the energy eigenvalues as
		\[
			E_{k,n} = \frac{\h^2k^2}{2m}+\plr{\frac{|eB|\h}{mc}}\plr{n+\frac 12}
		\]
		where $\h k$ is the continuous eigenvalue of the $p_z$ operator and $n$ is a nonnegative integer including zero. 
		\\
		\\
		The Hamiltonian for this system is 
		\[
			H = \frac{1}{2m}(\Pi_x^2+\Pi_y^2+\Pi_z^2).
		\]
		Since the only non-zero commutation of the mechanical momentum operators is $[\Pi_x,\Pi_y]$, we can see that 
		\[
			[\Pi_z,\Pi_x] = [\Pi_z,\Pi_y] = [\Pi_z,\Pi_z]= 0
		\]
		and thus
		\[
			[\Pi_z,f(\vect \Pi)] = 0.
		\]
		Therefore we have
		\[
			[\Pi_z,H] = 0
		\]
		which means that eigenkets of $\Pi_z$ are proportional to the energy eigenkets. Looking at the one-dimensional harmonic oscillator, 
		we need to try to find some pair of operators that we can express our Hamiltonian with. Taking after the form of $a$ and $a^\dag$ of 
		the harmonic oscillator, we find that the operators that do the job are 
		\[
			a= \sqrt{\frac{c}{2\h eB}}\plr{\Pi_x+i\Pi_y};\quad a^\dag =  \sqrt{\frac{c}{2\h eB}}\plr{\Pi_x-i\Pi_y}.
		\]
		Taking their product we have
		\ba
			aa^\dag &= \frac{c}{2\h eB}\plr{\Pi_x^2+\Pi_y^2 +i[\Pi_y,\Pi_x]}\\
				     & =  \frac{c}{2\h eB}\plr{\Pi_x^2+\Pi_y^2 +\frac{\h e}{c}B}\\
				     & = \frac{c}{2\h eB}\plr{\Pi_x^2+\Pi_y^2} +\frac{1}{2}
		\ea
		as well as
		\[
			a^\dag a = \frac{c}{2\h eB}\plr{\Pi_x^2+\Pi_y^2} -\frac{1}{2}.
		\]
		Notice that 
		\[
			[\Pi_z,aa^\dag] = [\Pi_z,a^\dag a] = 0
		\]
		and that the commutation relation for our operators is just like the S.H.O., 
		\[
			[a,a^\dag] = 1.
		\]
		By S.H.O. convention, we can denote 
		\[
			a^\dag a \equiv N
		\]
		and we can now form our Hamiltonian as
		\[
			H = \frac{\h e B}{cm}\plr{ N  +\frac{1}{2}}+\frac{1}{2m}\Pi_z^2
		\]
		Since we have shown that
		\[
			[\Pi_z,N] = [\Pi_z,H] = [N,H] = 0
		\]
		we have essentially found a set of three compatible operators. Therefore, we can label our simultaneous eigenstates by three 
		different indices $\ket{E_n,n,p_z}$. To find the energy eigenvalues, lets first denote the eigenvalue for $N$ as
		\[
			N\ket{E_n,n,p_z} = n\ket{E_n,n,p_z}.
		\]
		At this point $n$ is still undetermined in value. In order to show that $n$ is an integer, lets develop some more relationships. Notice
		that
		\[
			[N,a^\dag] = [a^\dag a,a^\dag] = a^\dag[a,a]+[a^\dag,a]a = -a
		\]
		and similarly
		\[
			[N,a^\dag] = a^\dag.
		\]
		To show the effect of creation and annihilation we form
		\[
			Na^\dag\ket{E_n,n,p_z} = ([N,a^\dag]+a^\dag N)\ket{E_n,n,p_z} = (n+1)a^\dag \ket{E_n,n,p_z}
		\]
		and
		\[
			Na\ket{E_n,n,p_z} = ([N,a]+a N)\ket{E_n,n,p_z} = (n-1)a\ket{E_n,n,p_z}
		\]
		which implies that $a\ket{E_n,n,p_z}$ is an eigenket of $N$ with eigenvalue decreased by one. Therefore we deduce that
		\[
			a\ket{E_n,n,p_z} = c\ket{E_n,n-1,p_z}.
		\]
		By requiring $\ket{E_n,n,p_z}$ and $\ket{E_n,n-1,p_z}$ to be normalized, we can determine $c$:
		\[
			\braket{E_n,n,p_z|a^\dag a|E_n,n,p_z} = |c|^2
		\]
		\[
			\braket{E_n,n,p_z|a^\dag a|E_n,n,p_z} = \braket{E_n,n,p_z|N|E_n,n,p_z} = n 
		\]
		\[	
			n = |c|^2.
		\]
		If we take $c$ to be real and positive (which still retains the original definition of $|c|^2$) we then have
		\[
			c=\sqrt{n}. 
		\]
		Thus the effect of the annihilation operator is
		\[
			a\ket{E_n,n,p_z} = \sqrt{n}\ket{E_n,n-1,p_z}.
		\]
		As such, we could continue annihilating indefinitely. However, we have the requirement the inner product be positive
		\[
			n=\braket{E_n,n,p_z|N|E_n,n,p_z}= (\bra{E_n,n,p_z}a^\dag)\cdot(a\ket{E_n,n,p_z}) \ge 0
		\]
		which forces us to conclude that $n$ must be a non-negative integer. A similar analysis can be done for the creation operator. 
		\\
		\\
		Notice that the eigenkets of the Hamiltonian have been labeled by the $p_z$ index. What has allowed us to do so? Remembering 
		that
		\[
			\Pi_z = p_z-\frac{e}{c}A_z,
		\]
		we note that simplest form of $\vect A$ that yields $\vect B = B\vecth z$ is 
		\[
			\vect A = (-By,0,0).
		\]
		In this form
		\[
			\Pi_z = p_z
		\] 
		and thus we have
		\[
			\Pi_z \ket{E_n,n,p_z} = p_z\ket{E_n,n,p_z}
		\]
		as desired. However, we know that we can compose the same magnetic field under any arbitrary gauge transformation, most 
		especially those in which $\Pi_z \ne p_z$. Fortunately this does not pose as serious of a problem as it seems since we can show
		that the energy eigenvalues are gauge invariant, as we expect they must be. To show this, lets denote the Hamiltonian before the 
		gauge transformation as 
		\[
			H\ket a = E_a\ket a
		\]
		and the Hamiltonian after a gauge transformation $\vect A \to \vect A +\del \Lambda(\vect x)$ as
		\[
			H'\ket{a'} = E_a'\ket{a'}.
		\]
		Using Sakurai eq. 2.7.47 we can relate our energy eigenstate before and after a gauge transformation by
		\[
			\ket{a'} = U\ket{a} = \exp\blr{\frac{ie\Lambda(\vect x)}{\h c}}\ket a.
		\]
		Since $U$ is a unitary operator, we can form
		\[
			UHU^{-1}U\ket a = E_aU\ket a
		\]
		which shows that $U\ket a=  \ket{a'}$ is an eigenket of the unitary transform of $H$, $UHU^{-1}$. Since the eigenvalue is in fact 			that of an energy eigenvalue, $E_a$, we can deduce that $H$ and $H'$ are related by
		\[
			H' = UHU^{-1}.
		\]
		Thus we have shown that unitary equivalent observables have identical spectra! In conclusion we have
		\[
			H'\ket{a'} = E_a\ket{a'}.
		\]
		Essentially, we have the freedom to choose any gauge in our Hamiltonian since the energy eigenvalues are 
		invariant under gauge transformations. Being a physical quantity, we expect energy to be gauge invariant in this situation. Now we 
		can evaluate the eigenvalue of our Hamiltonian as
		\ba
			H\ket{E_n,n,p_z} &=  \blr{\frac{\h e B}{cm}\plr{ N  +\frac{1}{2}}+\frac{1}{2m}\Pi_z^2}\ket{E_n,n,p_z}\\
			& =  \blr{\frac{\h e B}{cm}\plr{ n  +\frac{1}{2}}+\frac{1}{2m}p_z^2}\ket{E_n,n,p_z}.
		\ea
		Using DeBroglie's relationship for momentum, the energy eigenvalues for a charged particle in a uniform magnetic field are given
		as
		\[
			E_{k,n} = \frac{\h^2k^2}{2m} + \pfrac{eB\h}{mc}\plr{n+\frac 12}
		\]
		where $p_z = \h k$ is continuous and $n = 0,1,2,3,..$. 
	\eenum 
	
 \eenum
 
 \end{document}