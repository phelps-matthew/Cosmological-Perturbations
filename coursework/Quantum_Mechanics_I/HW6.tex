\documentclass[11pt,letterpaper]{article}
\usepackage{macroshw}

\title{\begin{spacing}{1.2}Quantum Mechanics I\\HW 6\end{spacing}}
\author{Matthew Phelps}
\date{Due: April 9}

\begin{document}
\maketitle

\benum
% #1 --------------------------------------------------------------------------------------------------------------------------------------------------------------------------------------
  	\item 
	
	\benum
		\item 
		% (a)
		Consider the angular momentum operator $\vect L = \vect r\times \vect p$. Evaluate the commutator
		$[L_x,az^2p_y^2+bx^2r^2]$, where $a$, $b$ are pure numbers and $r^2 = x^2+y^2+z^2$. 
		\\
		\\
		The $x$ component of orbital angular momentum can be expressed as
		\[
			L_x = yp_z - zp_y
		\]
		in which the commutator becomes
		\[
			[yp_z-zp_y, az^2p_y^2+bx^2(x^2+y^2+z^2)].
		\]
		To evaluate, the only commutation relation we need to keep in mind here is 
		\[
			[x_i,p_j] = i\h \delta_{ij};
		\]
		everything else commutes. Start with
		\ba
			[yp_z,zp_y] &= [yp_z,z]p_y + z[yp_z,p_y]\\
				& = ([y,z]p_z+y[p_z,z])p_y + z([y,p_y]p_z+y[p_z,p_y])\\
				& = -i\h yp_y+i\h zp_z\\
				& = i\h(zp_z - yp_y).
		\ea
		Using this result we have
		\ba	
			[yp_z,z^2p_y^2] &= [yp_z,zp_y]zp_y+zp_y[yp_z,zp_y]\\
				& = i\h(zp_z-yp_y)zp_y + i\h zp_y(zp_z-yp_y)\\
				& = i\h\{zp_z-yp_y,zp_y\}.
		\ea
		For the first commutator then
		\[
			[yp_z-zp_y,az^2p_y^2] = ai\h\{zp_z-yp_y,zp_y\}.
		\]
		As for the second part, only select terms contribute
		\ba
			[yp_z-zp_y,x^2(x^2+y^2+z^2)]  &=  [yp_z,x^2z^2]-[zp_y,x^2y^2]\\
			& = x^2y[p_z,z^2]-x^2z[p_y,y^2]\\
			& = -2i\h x^2yz +2i\h x^2yz\\
			& = 0.
		\ea
		Thus we finally have
		\[
			[L_x,az^2p_y^2+bx^2r^2] = ai\h\{zp_z-yp_y,zp_y\}
		\]
		\\
		\\
		
		\item
		% (b)
		Consider the addition of two angular momentum operators according to $\vect L_1 + \vect L_2 = \vect L$. Eigenstates
		$\ket{l_1,m_1}$ are associated with the operators $\vect L_1^2$ and $L_{1z}$, eigenstates $\ket{l_2,m_2}$ are associated
		with the operators $\vect L_2^2$ and $L_{2z}$, and eigenstates $\ket{L,M}$ are associate with the operators 
		$\vect L^2$ and $L_z$. In terms of the quantum numbers $(l_1,m_1)$ and $(l_2,m_2)$ determine (derive and state the
		answer) the values which are allowed for the quantum numbers $(L,M)$. Express the $\ket{L,M}$ eigenstate with the largest
		$M$ value in terms of the $\ket{l_1,m_1}$ and $\ket{l_2,m_2}$ eigenstates. 
		\\
		\\
		For the addition of two angular momentum operators, 
		\[
			\vect L = \vect L_1 \otimes 1 +1 \otimes  \vect L_2
		\] 
		we can expand the base kets in terms of two different sets of maximally compatible oberservables. To distinguish which observables 
		are compatible with each other, we note the following commutation relations:
		\[
			[J_{1i}, J_{2j}] = 0 \quad (\text{all } i,j)
		\]
		\[
			[\vect J^2,\vect J_1^2] = [\vect J^2,\vect J_2^2] = 0
		\]
		\[
			[\vect J^2,J_{1z}] \ne 0
		\] 
		\[
			[\vect J^2,J_{2z}] \ne 0.
		\]
		Commutation relations between angular momentum in the same subspace have been left out as they follow the familiar
		rules. From these commutation relations, we can form kets that are eigenstates of $\vect J_1^2$, $\vect J_2^2$, $J_{1z}$, 
		$J_{2z}$ or kets that are eigenstates of $\vect J^2$, $J_z$, $J_1^2$, and $J_2^2$. We will denote them respectively as
		\[
			\ket{l_1,l_2,m_1,m_2}
		\]
		and
		\[
			\ket{l_1,l_2,l,m}.
		\]
		To relate the two base kets to each other (for fixed $l_1$, $l_2$) we form the expansion
		\[
			\ket{l_1,l_2,l,m} = \sum_{m_1,m_2} \ket{l_1,l_2,m_1,m_2}\braket{l_1,l_2,m_1,m_2|l_1,l_2,l,m}.
		\]
		where the inner products (matrix elements)
		\[
			\braket{l_1,l_2,m_1,m_2|l_1,l_2,l,m}
		\]
		are called the Clebsch-Gordan coefficients. The question of what values are allowed for $l,m$ for a given $l_1,m_1$ and $l_2,m_2$
		is equivalent to asking which Clebsch-Gordan coeffiecents are non-vanishing. Put another way, if we have a state of definite $l_1$, 
		$l_2$, $m_1$, and $m_2$ and we apply the total angular momentum operator $\vect L^2$ to this this state, the possible values
		that are returned depend on the expansion of $\ket{l_1,l_2,m_1,m_2}$ in the $\ket{l_1,l_2,l,m}$ basis - aka the 
		nonvanishing Clebsch-Gordan coefficients. By convention, the Clebsch-Gordan coefficients are real. As a consequence, the
		transformation is orthogonal $U= U^T$ and thus the nonvanishing requirement applies equally regardless of which basis we start 
		with. 
		\\
		\\
		To find the requirement on which $z$-components of total angular momentum are allowed, we can do the following routine. 
		Start with
		\[
			(L_z-L_{1z}-L_{2z})\ket{l_1,l_2,l,m} = 0 
		\]
		since 
		\[
			L_z = L_{1z}+L_{2z}.
		\]
		Now multiply by the other base ket on the left
		\[
			\braket{l_1,l_2,m_1,m_2|(L_z-L_{1z}-{L_2z})|l_1,l_2,l,m} = 0 
		\]
		\[
			(m-m_1-m_2)\braket{l_1,l_2,m_1,m_2|l_1,l_2,l,m} = 0.
		\]
		We observe that only nonvanishing coffecients occur for
		\[
			m = m_1+m_2.
		\]
		As for the condition on the angular momentum itself it is easy to imagine that
		\[
			|l_1-l_2|\le l \le l_1+l_2
		\]
		if we view it as the sum of two angular momentum vectors. For a more formal proof, however, we can show that the 
		dimensionality of the space spanned by either basis is equivalent only if $|l_1-l_2|\le l \le l_1+l_2$. Given fixed
		$l_1$ and $l_2$, the dimensionality of the space is 
		\[
			N = (2j_1+1)(2j_2+1).
		\]
		In terms of the other basis, for each $l$ there are $2l+1$ states. If we assume our rule for angular momentum holds true,
		we find
		\[
			\sum_{|l_1-l_2|}^{l_1+l_2} (2l+1) = (2l_1+1)(2l_2+1)= N.
		\]
		Therefore we conclude that for a given $(l_1,m_1)$ and $(l_2,m_2)$ the allowed $(l,m)$ are
		\[
			m = m_1+m_2
		\]
		\[
			|l_1-l_2|\le l \le l_1+l_2.
		\]
		For a given $l$, the maximum $m$ value the state $\ket{l_1,l_2,l,m}$ can have is 
		\[
			m = l = l_1+l_2.
		\]
		If we combine this with the requirement that
		\[
			m = m_1+m_2
		\]
		we must conclude that $\ket{l_1,l_2,l,m_{max}}$ can only be expressed by one state in the $\ket{l_1,l_2,m_1,m_2}$ basis. That 
		state is 
		\[
			\ket{l_1,l_2,l_1,l_2}.
		\]
		Thus for maximum $m$ the conversion between the two basis kets is
		\[
			\ket{l_1,l_2,l,m_{max}} = \ket{l_1,l_2,l_1,l_2}.
		\]
		\\
		
		\item
		% (c)
		Consider a general ket $\ket{l,m}$ where $l$ designates the orbital angular momentum eigenvalue and $m$ its $z$ 
		component. Consider a specific ket $\ket{2,1}$. Determine for which $\ket{l,m}$ values the matrix elements
		\[
			\braket{2,1|r^2|l,m},\quad \braket{2,1|r\vect r|l,m}
		\]
		are non-zero, and determine their values (you can give your answer in terms of closed form integrals, there is no need to
		calculate the integrals themselves). 
		\\
		\\
		To apply $r^2$, we will need to expand $\ket{l,m}$ in a position basis. Since the $\ket{l,m}$ deal only with angular positions, 
		we expand in terms of $\ket{\vecth n}$. For the first matrix element,
		\ba
			\braket{2,1|r^2|l,m} &= \iint \braket{2,1|\vecth n'}\braket{\vecth n'|r^2|\vecth n}\braket{\vecth n|l,m} d\Omega\, d\Omega'
			\\ & = \iint \braket{2,1|\vecth n'}\braket{\vecth n'|\vecth n}\braket{\vecth n|l,m} d\Omega\, d\Omega' \\
			& = \int \braket{2,1|\vecth n}\braket{\vecth n|l,m} d\Omega\\
			& = \int_0^{2\pi}\int_0^\pi Y^*_{21} Y_{lm} \sin\theta d\theta \, d\phi\\
			& = \delta_{2,l}\delta_{1,m}.
		\ea
		Thus the only nonvanishing matrix element is
		\[
			\braket{2,1|r^2|2,1} = 1.
		\]
		Similarly for the other element we form
		\ba
			\braket{2,1|r\vect r|l,m} &= \int_0^{2\pi}\int_0^\pi Y^*_{21} Y_{lm}\vecth n \sin\theta d\theta \, d\phi\\
			& = \vecth r \, \delta_{2,l}\delta_{1,m}.
		\ea
		Therefore the only nonvanishing matrix element is
		\[
			\braket{2,1|r\vect r|2,1} = \vecth r = \vecth n.
		\]
		\\
	\eenum
	
% #3 ----------------------------------------------------------------------------------------------------------------------------------------------------------------------------------
	\item
	A particle of mass $M$ is constrained to move on the surface of a sphere of radius $r$. Its dynamics can be described by a free 
	Hamiltonian $H_0$ and a set of free eigenstates $Y_l^m(\theta,\phi)$. The sphere is then embedded in a uniform gravitational 
	field with acceleration $g$ directed along the $-z$ axis, so that the particle experiences the potential
	\[
		V(\theta,\phi) = mgr\cos\theta.
	\]
	
	\benum
		% (a)
		\item 
		Compute the values of all non-zero matrix elements of the potential operator $V(\theta,\phi)$ in a basis consisting of those 				$Y_l^m(\theta,\phi)$ that have $l=0,1$ and $2$. Do not attempt to work out every case - there are 81 of them. Use symmetry 
		arguments to save work.
		\\
		\\
		In the position representation, the matrix values go as
		\[
			mgr \int_{0}^{2\pi}\int_{0}^\pi (Y_{l'}^{m'})^*Y_l^m\cos\theta\sin\theta d\theta \, d\phi.
		\]
		Notice that since $m$ and $m'$ are integers, the $\phi$ dependence of the integration is proportional to
		\[
			\int_0^{2\pi} e^{im\phi}e^{-im'\phi}  \propto \delta_{m,m'}.
		\]
		Therefore, only values with the same $z$ component of angular momentum have nonvanishing matrix elements. For the $\theta$ 
		dependence of the integral, $\cos\theta$ is odd about $\pi/2$ while $\sin\theta$ is even about $\pi/2$. Therefore, if a function
		$f(\theta)$ in
		\[
			\int_0^\pi f(\theta)\cos\theta\sin\theta\, d\theta\
		\]
		is even (symmetric) about $\pi/2$, then the integral will vanish. Here is a list of the 
		spherical harmonics up to $l=2$. 
		\ba
			Y_0^0  & = \pfrac{1}{4\pi}^{1/2} \\
			Y_1^0  & =\pfrac{3}{4\pi}^{1/2}\cos\theta \\
			Y_1^1  & = -\pfrac{3}{8\pi}^{1/2}\sin\theta e^{i\phi} \\
			Y_2^0  & = \pfrac{5}{16\pi}^{1/2}(3\cos^2\theta-1) \\
			Y_2^1  & = -\pfrac{15}{8\pi}^{1/2}\sin\theta\cos\theta e^{i\phi}\\
			Y_2^2 & = \pfrac{15}{32\pi}^{1/2}\sin^2\theta e^{2i\phi}
		\ea
		The rest can be composed using $Y_l^{-m} = (-1)^m(Y_l^m)^*$. We observe that each spherical harmonic, being composed
		of combinations of sines and cosines, is even or odd about our integration bounds. Therefore we can exclude harmonics
		with $l'=l$ since their product $f(\theta)$ will be even. Now our task amounts to finding two spherical harmonics with
		$m=m'$, $l\ne l'$ and who's product is an odd function (it turns out that last requirement holds given the first two). Listing
		them out as $\braket{l',m'| |l,m}$,  we have the possible combinations of states
		\ba
			 &\braket{1,0   | |   0,0}\quad &\braket{0,0   | |   1,0} \\
			 &\braket{2,0   | |   1,0}\quad &\braket{1,0   | |   2,0} \\
			 &\braket{2,-1   | |   1,-1}\quad &\braket{1,-1   | |   2,-1} \\
			& \braket{2,1   | |   1,1}\quad &\braket{1,1   | |   2,1} 
		\ea
		Since the transpose of the matrix elements for $m'=m$ are equivalent, we have the relationship
		\[
			\braket{l',m'| |l,m} = \braket{l,m||l'm'}.
		\]
		Therefore we only need to compute four integrals in total (3 actually since $\braket{2,1   | |   1,1} = \braket{2,-1   | |   1,-1}$.
		Denoting $V = mgr\cos\theta$ we have
		\ba	
			\braket{1,0   |V |   0,0} &=  \pfrac{\sqrt 3}{2} \int_0^\pi \cos^2\theta\sin\theta d\theta = \frac{\sqrt 3}{3}\\
			\braket{2,0   |V |   1,0} &=  \pfrac{\sqrt 15}{4} \int_0^\pi (3\cos^2\theta-1)\cos^2\theta\sin\theta d\theta  =\frac{2\sqrt{15}}{15}\\
			\braket{2,-1   |V |   1,-1} &=  \pfrac{3\sqrt 5}{4} \int_0^\pi \cos^2\theta\sin^3\theta d\theta  =  \frac{\sqrt 5}{5}\\
			\braket{2,1   |V |   1,1} &=  \braket{2,-1   |V |   1,-1}  = \frac{\sqrt 5}{5} 
		\ea
		All matrix elements should be multiplied by $mgr$. 
		\\
		% (b) 
		\item
		Consider matrix elements of the full Hamiltonian between the states in this basis and the free ground state. Identify for which 
		particular states in the basis these matrix elements are nonzero. Reduce the dimension of the basis to just these particular 
		states and write down the full Hamiltonian as a matrix in this reduced basis. 
		\\
		\\
		Which basis is this even referring to? I am going to assume it is the basis $Y_l^m(\theta,\phi)$ for $l=0$, $1$, $2$. 
		The full Hamiltonian is
		\[
			H = H_0+V(\theta) = \frac{\vect p^2}{2m}+mgr\cos\theta. 
		\]
		We are looking for elements such that
		\[
			\frac{1}{2m}\braket{l,m|\vect p^2|0,0} + \braket{l,m|V|0,0} \ne 0
		\]
		for $l=0,1,2$. We have already computed all nonvanishing matrix elements for $V$, so it remains to find the matrix elements
		for $\vect p^2$. Expanding in a position basis, 
		\[
			\braket{l,m|\vect p^2|0,0} = \int d\vect x\, \braket{l,m|\vect x}\braket{\vect x|\vect p^2|0,0}.
		\]
		For the last term, we can use the relation 
		\[
			\braket{\vect x|\vect p^2|\alpha} = -\h^2\plr{\pdifff{}{*2r}\braket{\vect x|\alpha}+\frac{2}{r}\pdiff{r}\braket{\vect x|\alpha}
			-\frac{1}{\h^2r^2}\braket{\vect x|\vect L^2|\alpha}}.
		\]
		As there is no radial depedence in our ket $\ket\alpha$, the only term remaining is then
		\[
			\braket{\vect x|\vect p^2|\alpha} = \frac{1}{r^2}\braket{\vect x|\vect L^2|\alpha}.
		\]
		However we note
		\[
			\braket{\vect x|\vect L^2|0,0} = 0
		\]
		so all matrix elements must vanish. So it seems only nonvanishing elements of $\braket{l,m|V|0,0}$ can be used. Looking
		at part (a) this only leaves 
		\[
			\braket{1,0|V|0,0} = mgr\frac{\sqrt 3}{3} =  \braket{0,0|V|1,0}.
		\]
		In the basis
		\[
			\ket{0,0} = \begin{pmatrix}1\\0\end{pmatrix},\quad \ket{1,0} = \begin{pmatrix}0\\1\end{pmatrix}
		\]
		we have
		\[
			H = mgr\frac{\sqrt 3}{3} \begin{pmatrix}0&1\\1&0\end{pmatrix}
		\]
		\\
		\\
	\eenum
% #3 ------------------------------------------------------------------------------------------------------------------------------------------------------------------------------
	\item
	Find the Clebsch-Gordan coefficients for combining two spins, $s_1 = 1$ and $s_2 = 1$. 
	\\
	\\
	The spin number of a particle distinguishes the dimension of the Hilbert space spanned by the spin states. We seek to find the 
	representation of a state $\ket{j_1=1,j_2=1,j,m}$ in terms of the states $	
	\ket{j_1=1,m_1,j_2=1,m_2}$, where
	the former are eigenstates of the total angular momentum operators ($\vect J^2$, $J_z$) according to
	\[
		\vect J = \vect J_1\otimes1 + 1\otimes \vect J_2
	\]
	while the latter are eigenstates of $\vect J_1^2$, $\vect J_2^2$, $J_{1z}$, and $J_{2z}$. Since the two particles always have the 
	same individual spin number $j_1=j_2= 1$ we may denote the states as $\ket{j,m}$ and $\ket{m_1,m_2}$. The procedure 
	for finding the Clebsch-Gordan coefficients can be achieved by the following steps:
		\begin{enumerate}[label=\Roman*.]
		\item Find the maximal total angular momentum and $z$ angular momentum state 
		\[
			\ket{j_{max},m_{max}} = \ket{m_1=j_1,m_2=j_2}.
		\]
		\item Repeatedly apply the lowering operator $J_-=J_{1-}+J_{2-}$ to generate the set of $m$ states. 
		\item Obtain the next maximal spin state  $\ket{j_{max}-1,m=j_{max}-1}$ by requiring it to be orthogonal to 
		$\ket{j_{max},j_{max}-1}$. 
		\item Repeat II and III until $N=(2j_1+1)(2j_2+1)$ states are found.
		\eenum 
		
	As per the procedure, the maximal state, given in the notation of $\ket{j,m} = \sum \ket{m_1,m_2}$, is
	\[
		\ket{2,2} = \ket{1,1}.
	\]
	Now applying the lowering operator (omitting $\h$ as they all cancel)
	\ba
		J_-\ket{2,2} &= (J_{1-}+J_{2-})\ket{1,1}\\
		2 \ket{2,1} & = \sqrt{2}(\ket{0,1}+\ket{1,0})\\
		\ket{2,1}& = \frac{1}{\sqrt 2}(\ket{0,1}+\ket{1,0})\\ 
		\\
		J_-\ket{2,1} &= \frac{1}{\sqrt 2}(J_{1-}+J_{2-})(\ket{0,1}+\ket{1,0})\\
		\sqrt{6} \ket{2,0} & = \frac{1}{\sqrt 2}[\sqrt 2 (\ket{-1,1}+\ket{1,-1})+\sqrt {2}(\ket{0,0}+\ket{0,0})\\
		\ket{2,0} &= \frac{1}{\sqrt 6}\plr{\ket{-1,1}+\ket{1,-1}+2\ket{0,0}}\\
		\\
		J_-\ket{2,0} &= \frac{1}{\sqrt 6}(J_{1-}+J_{2-})(\ket{-1,1}+\ket{1,-1}+2\ket{0,0})\\
		\sqrt 6\ket{2,-1} &= \frac{1}{\sqrt 6}[\sqrt 2(\ket{0,-1}+\ket{-1,0}+2\ket{-1,0}+2\ket{0,-1})]\\
		\ket{2,-1} & = \frac{1}{\sqrt 2}(\ket{0,-1}+\ket{-1,0})\\
		\\
		J_-\ket{2,-1} &= \frac{1}{\sqrt 2}(J_{1-}+J_{2-})(\ket{0,-1}+\ket{-1,0})\\
		2\ket{2,0} & = \frac{1}{\sqrt 2}(2\sqrt 2(\ket{-1,-1})\\
		\ket{2,0} & = \ket{-1,-1}
	\ea
	The next maximal state $\ket{j_{max}-1,m_{max}-1} = \ket{1,1}$ can be found by making it orthogonal to 
	$\ket{2,1}$. That is
	\[
		\ket{1,1}_{j,m} = \frac{1}{\sqrt 2}(\ket{0,1}-\ket{1,0})_{m_1,m_2}.
	\]
	Applying the lowering operator
	\ba
		J_-\ket{1,1} &= \frac{1}{\sqrt 2}(J_{1-}+J_{2-})(\ket{0,1}-\ket{1,0})\\
		\sqrt 2\ket{1,0} &= \frac{1}{\sqrt 2}[\sqrt 2(\ket{-1,1}-\ket{1,-1}+\ket{0,0}-\ket{0,0})]\\
		\ket{1,0} & = \frac{1}{\sqrt 2}(\ket{-1,1}-\ket{1,-1})\\
		\\
		J_-\ket{1,0} &= \frac{1}{\sqrt 2}(J_{1-}+J_{2-})(\ket{-1,1}-\ket{1,-1})\\
		\sqrt 2\ket{1,-1} & = \frac{1}{\sqrt 2}[\sqrt 2(-\ket{0,-1}+\ket{-1,0})]\\
		\ket{1,-1} & = \frac{1}{\sqrt 2}(\ket{-1,0}-\ket{0,-1}).
	\ea
	To find the last state, $\ket{0,0}$ we must make it orthogonal to both $\ket{2,0}$ and $\ket{1,0}$. This can also be seen
	as finding a state such that it vanishes when applied by either the raising or lowering operator. Thus
	\[
		\ket{0,0} = \frac{1}{\sqrt 3}(\ket{-1,1}+\ket{1,-1}-\ket{0,0}).
	\]
	All together then we have ($\ket{j,m} = \sum \ket{m_1,m_2}$)
	\ba
		\ket{2,2} & = \ket{1,1} \\
		\ket{2,1} & = \frac{1}{\sqrt 2}(\ket{0,1}+\ket{1,0}) \\
		\ket{2,0} & = \frac{1}{\sqrt 6}(\ket{-1,1}+\ket{1,-1}+2\ket{0,0}) \\
		\ket{2,-1} & = \frac{1}{\sqrt 2}(\ket{0,-1}+\ket{-1,0}) \\
		\ket{2,-2} & = \ket{-1,-1} \\
		\ket{1,1} & = \frac{1}{\sqrt 2}(\ket{0,1}-\ket{1,0}) \\
		\ket{1,0} & = \frac{1}{\sqrt 2}(\ket{-1,1}-\ket{1,-1}) \\
		\ket{1,-1} & = \frac{1}{\sqrt 2}(\ket{-1,0}-\ket{0,-1}) \\
		\ket{0,0} & = \frac{1}{\sqrt 3}(\ket{-1,1}+\ket{1,-1}-\ket{0,0})
	\ea	
	\\
% #4 ------------------------------------------------------------------------------------------------------------------------------------------------------------------------------------
	\item
	A beam of massive spin-1 particles passes through a Stern-Gerlach apparatus and splits into three output beams, each one 
	corresponding to one of the allowed projections of the particle spins onto a direction defined by the magnetic field inside the 
	apparatus. Unless stated otherwise, all directions in this problem are with respect to a fixed laboratory coordinate system.
	\benum
		% (a)
		\item
		Using the standard raising and lowering operators of angular momentum $J_\pm = J_x\pm iJ_y$ such that 
		$J_\pm\ket{j,m} = \sqrt{j(j+1)-m(m+1)}\ket{jm\pm1}$ show that the three matrices below form a valid representation
		of the spin operators for these particles
		\[
			S_x = \frac{\h}{\sqrt 2}\begin{pmatrix}0&1&0\\1&0&1\\0&1&0\end{pmatrix};\ 
			S_y = \frac{\h}{\sqrt 2}\begin{pmatrix}0&-i&0\\i&0&-i\\0&i&0\end{pmatrix};\ 
			S_z = \h\begin{pmatrix}1&0&0\\0&0&0\\0&0&-1\end{pmatrix}
		\]
		\\
		In this problem $j$ is fixed at $j=1$. We may first form the raising and lowering operators in terms of the matrices given. Let us
		define the kets in the matrix representation 
		\[
			\begin{pmatrix}1\\0\\0\end{pmatrix} = \ket{j=1,m=1},\ 
			\begin{pmatrix}0\\1\\0\end{pmatrix} = \ket{j=1,m=0},\ 
			\begin{pmatrix}0\\0\\1\end{pmatrix} = \ket{j=1,m=-1}.
		\]
		As a quick check to confirm that we put these in the right order, we can see that $S_z$ acting on each column
		matrix gives the correct value of $m$ back as an eigenvector. 
		\\
		\\
		Lets construct the raising and lowering operator in the matrix representation:
		\ba
			J_- = J_x\pm iJ_y \quad \to \quad  J_- &=  \frac{\h}{\sqrt 2} \begin{pmatrix}0&1&0\\1&0&1\\0&1&0\end{pmatrix}
			-i\frac{\h}{\sqrt 2} \begin{pmatrix}0&-i&0\\i&0&-i\\0&i&0\end{pmatrix}\\
			& = \sqrt 2\h \begin{pmatrix}0&0&0\\1&0&0\\0&1&0\end{pmatrix} .
		\ea
		Using $J_+ = (J_-)^\dag$ we have
		\[
			J_- = \sqrt 2\h \begin{pmatrix}0&0&0\\1&0&0\\0&1&0\end{pmatrix},\quad
			J_+ = \sqrt 2\h \begin{pmatrix}0&1&0\\0&0&1\\0&0&0\end{pmatrix}
		\]
		In matrix representation, we must confirm the same results as we know hold true in Dirac notation:
		\[
			J_-\ket{1,1} = \sqrt 2\h\ket{1,0} \to  \sqrt 2\h \begin{pmatrix}0&0&0\\1&0&0\\0&1&0\end{pmatrix}
			\begin{pmatrix}1\\0\\0\end{pmatrix} = \sqrt 2\h \begin{pmatrix}0\\1\\0\end{pmatrix}
		\]
		\[
			J_-\ket{1,0} = \sqrt 2\h\ket{1,-1} \to  \sqrt 2\h \begin{pmatrix}0&0&0\\1&0&0\\0&1&0\end{pmatrix}
			\begin{pmatrix}0\\1\\0\end{pmatrix} = \sqrt 2\h \begin{pmatrix}0\\0\\1\end{pmatrix}
		\]
		\[
			J_-\ket{1,-1} = 0  \to  \sqrt 2\h \begin{pmatrix}0&0&0\\1&0&0\\0&1&0\end{pmatrix}
			\begin{pmatrix}0\\0\\1\end{pmatrix} = 0 .
		\]
		Similarly for $J_+$ we have
		\[
			J_+\ket{1,1} = 0 \to \sqrt 2\h \begin{pmatrix}0&1&0\\0&0&1\\0&0&0\end{pmatrix}
			\begin{pmatrix}1\\0\\0\end{pmatrix} = 0
		\]
		\[
			J_+\ket{1,0} = \sqrt 2\h\ket{1,1} \to \sqrt 2\h \begin{pmatrix}0&1&0\\0&0&1\\0&0&0\end{pmatrix}
			\begin{pmatrix}0\\1\\0\end{pmatrix} = \sqrt 2\h \begin{pmatrix}1\\0\\0\end{pmatrix}
		\]
		\[
			J_+\ket{1,-1} = \sqrt 2\h\ket{1,0} \to \sqrt 2\h \begin{pmatrix}0&1&0\\0&0&1\\0&0&0\end{pmatrix}
			\begin{pmatrix}0\\0\\1\end{pmatrix} = \sqrt 2\h \begin{pmatrix}0\\1\\0\end{pmatrix}.
		\]
		Thus the spin matrices given form a valid representation of the spin operators. 
		\\
		\\
		% (b)
		\item
		Suppose the incoming beam is initially completely unpolarized, so its density operator in the matrix 
		representation given in part (a) is 
		\[
			\rho_0 = \frac{1}{3}\begin{pmatrix}1&0&0\\0&1&0\\0&0&1\end{pmatrix}.
		\]
		The density operator may be written in terms of the three eigenstates of the spin in the $z$ direction as
		\be\label{1}
			\rho_0 = \frac{1}{3}\blr{\ket +\bra ++\ket 0\bra 0 +\ket -\bra -}
		\ee
		not only in the laboratory frame, but in any Cartesian coordinate system obtained from the laboratory frame with an arbitrary 
		rotation. Why?
		\\
		\\
		Because the beam is unpolarized, it is a completely random ensemble and there is no preferred direction for the spin
		orientation. In other words, the ensemble given above can be also be expressed as an ensemble of the spin projection
		 $S_x$ operator all with equal weights
		\[
			\rho_0 = \frac{1}{3}\blr{\ket{S_x,+}\bra{S_x,+}+\ket{S_x,0}\bra{S_x,0}+\ket{S_x,-}\bra{S_x,-}}.
		\]
		Even more generally, $\rho_0$ can be expressed as
		\be\label{2}
			\rho_0 = \frac{1}{3}\blr{\ket{S\cdot\vecth n,+}\bra{S\cdot\vecth n,+}+\ket{S\cdot\vecth n,0}\bra{S\cdot\vecth n,0}
			+\ket{S\cdot\vecth n,-}\bra{S\cdot\vecth n,-}}
		\ee
		where $\vecth n$ is the direction of the SG apparatus. In regards to rotation, our unpolarized beam will pass through the SG 
		apparatus and always emerge as three beams of equal intensity along the $S\cdot\vecth n$ direction solely because
		there is \emph{no preferred spin direction}. Since the probability weight of the density operator is evenly split, we may
		always represent \eqref 2 in the form of \eqref 1 at any arbitrary angle of rotation. 
		\\
		\\
		% (c)
		\item
		What fraction of the particles in the unpolarized beam will pass through a Stern-Gerlach filter set up in such a way that
		only the particles with the component of the spin equal to 0 in the $x$ direction are selected?
		\\
		\\
		From \eqref 2 for $\vecth n = \vecth x$ we see that fraction of particles that pass through the SG filter that have
		$m = 0$ is $1/3 = 33\%$. 
	\eenum	
% #5 ------------------------------------------------------------------------------------------------------------------------------------------------------------------------
	\item 
	Four massive spin 1/2 particles are fixed on the vertices of a regular tetrahedron. The Hamiltonian of this system consists of a sum
	of spin-spin interactions over each of the six pairs as follows
	\[
		H = \alpha\sum_{i\ne j} \vect S_i\vect S_j
	\]
	\benum
		% (a)
		\item
		Show that all three components of the total spin $\vect J = \sum\nolimits_{i} \vect S_i$ of the system commutes with $H$. 
		\\
		\\
		Each of the four spin 1/2 particles resides in its own spin space such that we may use the formalism of addition 
		of angular momentum to describe how the operators act; i.e. $S_{1i}$ may only act on the spin ket corresponding
		to particle 1. The state of the system may be written as (since total angular momentum is fixed at $j=1/2$ for each 
		particle) 
		\[
			\ket{m_1,m_2,m_3,m_4} = \ket{m_1}\otimes\ket{m_2}\otimes\ket{m_3}\otimes\ket{m_4}
		\]
		and an operator such as $S_{1x}$ should be interpreted as
		\[
			S_{1x} = S_{1x}\otimes 1 \otimes 1 \otimes 1. 
		\]
		By restricting operators to only act within certain spaces, we automatically are implying that the commutator between
		different spaces must be 0
		\[
			[S_{li},S_{mj}] = 0 \ \ \text{if}\ \  l\ne m. 
		\]
		Operators within the same spin particle space are to be evaluated as usual.
		\\
		\\
		Now we are in a position to evaluate the commutator
		\[
			[H,J_i].
		\]
	 	Beginning with $J_x$, 
		\[
			J_x = S_{1x}+S_{2x}+S_{3x}+S_{4x}
		\]
		\[
			[J_x, \vect S_1\cdot\vect S_2] = [S_{1x}+S_{2x}+S_{3x}+S_{4x},S_{1x}S_{2x}+S_{1y}S_{2y}+S_{1z}S_{2z}].
		\]
		Term by term,
		\ba
			[S_{1i},S_{1j}S_{2j}] &= [S_{1i},S_{1j}]S_{2j}+S_{1j}[S_{1i},S_{2j}]\\
			& = [S_{1i},S_{1j}]S_{2j}+0\\
			& = i\h\epsilon_{ijk}S_{1k}S_{2j} .
		\ea	
		Using this relation, we can see that
		\ba
			[J_x,\vect S_1\cdot\vect S_2] =&[S_{1x}+S_{2x},S_{1x}S_{2x}+S_{1y}S_{2y}+S_{1z}S_{2z}] \\
			=&i\h( S_{1z}S_{2y}- S_{1y}S_{2z})+i\h(S_{1y}S_{2z}-S_{1z}S_{2y})\\
			=& 0
		\ea
		By both cyclic permutation and commutation in the same spin particle space, it should now be clear that the following relation holds 
		for any $J_i$ as well as any pair of spin interactions $\vect S_j\cdot \vect S_k$
		\[
			[J_i,\vect S_j\cdot \vect S_k] = 0 \quad\text{for}\quad j\ne k.
		\]
		Consequently, we arrive at the desired result,
		\[
			[J_i,H] = 0 \quad\text{for}\quad i=1,2,3.
		\]
		\\
		\\
		% (b)
		\item
		List all of the allowed energy levels for the system and the degeneracy factors of each level. Please note, you do not have
		to construct any eigenstates of energy explicitly. 
		\\
		\\
		Since we have
		\[
			[J_i,H] = 0
		\]
		it follows that
		\[
			[\vect J^2,H] = 0.
		\]
		However, since
		\[
			[J_i,J_j] \ne 0\quad\text{for}\quad i\ne j
		\]
		eigenstates of the Hamiltonian are simultaneous eigenstates of $\vect J^2$ and only one component of $\vect J$. Following 
		convention, we choose to diagonalize in the $z$-direction. Thus the energy eigenstates may be denoted as
		\[
			H\ket{j,m} = E\ket{j,m}
		\]
		where the same eigenvalue equations apply as we had for the addition of two angular momentum, i.e.
		\[
			J_z\ket{j,m} = m\h \ket{j,m}
		\]
		\[
			\vect J^2\ket{j,m} = j(j+1)\h^2\ket{j,m}.
		\]
		There is actually one more set of operators that commute with the Hamiltonian and $\vect J$. It can be shown that
		\[
			[S_i^2,H] = [S_i^2, \vect J^2] = [S_i^2,J_z] = 0.
		\]
		So we could denote the common eigenstates by four indices instead
		\[
			\ket{j_1,j_2,j_3,j_4,j,m}
		\]
		with the eigenvalue equation
		\[
			S_i^2\ket{j_1,j_2,j_3,j_4,j,m} = j_i(j_i+1)\h^2\ket{j_1,j_2,j_3,j_4,j,m}.
		\]
		Typically we do not label eigenstates with these numbers because the total spin (per particle) is fixed. In our case
		it is fixed at $s=1/2$ so that
		\ba
			S_i^2\ket{j_1,j_2,j_3,j_4,j,m} &= \frac{1}{2}\plr{\frac{1}{2}+1}\h^2\ket{j_1,j_2,j_3,j_4,j,m}\\
			&= \frac{3}{4}\h^2\ket{j_1,j_2,j_3,j_4,j,m}. 
		\ea
		Keeping this in mind, we will continue to denote our energy eigenstates as $\ket{j,m}$. The reason the set of operators
		$S_i^2$ are important is because they allow us to write out Hamiltonian as (dropping the $\alpha$ for now)
		\[
			H = \vect J^2 - (S_1^2+S_2^2+S_3^2+S_4^2).
		\]
		This enables us to solve the allowed energy values
		\ba
			H\ket{j,m} &= E\ket{j,m}\\
			& = [\vect J^2 - (S_1^2+S_2^2+S_3^2+S_4^2)]\ket{j,m}\\
			& = [j(j+1)\h^2 - 3\h^2]\ket{j,m}.
		\ea
		The next question is then what are the allowed values of $j$ and what is their degeneracy? For $j_1=j_2=j_3=j_4=1/2$, the 
		dimensionality of the space spanned by $\ket{m_1,m_2,m_3,m_4}$ is
		\[
			N = \blr{2\pfrac{1}{2}+1}^4 = 16. 
		\]
		In the other basis, however, thing are a little different. By adding four particles of spin 1/2, we expect the angular momentum to 
		range from $j=0$ to $j=2$. Since the total angular momentum must be separated by an integer, the three values of $j$ must be
		\[
			j=0,1,2. 
		\]
		Naively summing $(2j+1)$ for $j=1,2,3$ will not give the correct dimensionality. A good way to proceed would be to build up
		to the addition four angular momenta. Starting with just the addition of two angular momenta, 
		\[
			\vect S = \vect S_1 +\vect S_2
		\]
		we know the allowed values of $j$ are $j = 0,1$. Now lets add another spin 1/2,
		\[
			\vect S = \vect{S}_1+\vect{S}_2+\vect{S}_3.
		\]
		The allowed values of $j$ are now $j = \frac{1}{2},\frac{3}{2}$. However we see that there are two ways that we can arrive at
		$j=\frac{1}{2}$. In other words, there are two linearly independent representations of $j=\frac{1}{2}$. Lastly, we add on
		one more angular momentum, $\vect S_4$
		\[
			\vect S = \vect S_1+\vect S_2+\vect S_3 +\vect S_4.
		\]
		Listing the values of $j$ for each independent representation we see that we have
		\[
			j = 0,1,2,1,0,1.
		\]
		To check that the dimensionality is correct, we take the sum of $(2j+1)$ for each $j$ and representation to find
		\[
			N = 1+3+5+3+1+3 = 16.
		\]
		Thus we have found that $j=0$ has two representations, $j=1$ has three representations, and $j=2$ has one representation. 
		In terms of degeneracy
		\ba
			j &= 0 \to 2 \\
			j &= 1 \to 9 \\
			j &= 2 \to 5 
		\ea 
		These show the degeneracy at each value of $j$. Regarding the energy, we have
		\[
			E = \alpha\h^2[j(j+1)-3]\quad\text{for}\quad j = 0,1,2
		\]
		which the degeneracies for each value of $j$ listed above. 
		\\
		\\
	\eenum
	
% #6 --------------------------------------------------------------------------------------------------------------------------------------------------------------------------
	\item
	An electron in a Coulomb field of a proton is in the state 
	\[
		\ket\psi = \frac{4}{5}\ket{1,0,0}+\frac{3i}{5}\ket{2,1,1,}
	\]
	where $\ket{n,l,m}$ are the standard energy eigenstates of hydrogen.
	\\
	\benum
		% (a)
		\item 
		What is the expectation value of energy in this state? What are $\braket{\vect L^2}$ and $\braket{L_z}$?
		\\
		\\
		Since the Hamiltonian of the state is the same as that of the hydrogen atom, the state is an energy eigenstate
		$H\ket\psi = E\ket\psi$. For the states $\ket{n,l,m}$ the energy can be shown to be
		\[
			H\ket{n,l,m} = -\frac{mc^2\alpha^2}{2}\frac{1}{n^2}\ket{n,l,m} \quad\text{for}\quad n = 1,2,3...
		\]
		where $\alpha \equiv e^2/hc \approx 1/37$ is the fine structure constant. For a given value of $n$, the angular
		momentum $l$ ranges from $l=0,1...n-1$ and we also have the usual degeneracy in $m = -l,-l+1...l-1,l$.  The 
		expectation value of the energy is
		\[
			\braket{\psi|H|\psi} = \frac{16}{25}\braket{1,0,0|H|1,0,0}+\frac{9}{25}\braket{2,1,1|H|2,1,1}
		\]
		where orthogonality between states with different quantum numbers has been implemented. Continuing on,
		\ba
			\braket{\psi|H|\psi} &= -\frac{mc^2\alpha^2}{2}\plr{\frac{16}{25}+\frac{9}{25}\pfrac{1}{4}}\\
			& =  -\frac{mc^2\alpha^2}{2}\pfrac{73}{100}.
		\ea
		In more physical terms,
		\[
			\braket{E} = -13.6eV\pfrac{73}{100}.
		\]
		The expectation values for $\vect L^2$ and $L_z$ are found similarly, since $\ket{\psi}$ is an eigenstate of both.
		\ba
			\braket{\psi|\vect L^2|\psi} &=  \frac{16}{25}\braket{1,0,0|\vect L^2|1,0,0}+\frac{9}{25}\braket{2,1,1|\vect L^2|2,1,1}\\
			& = \frac{16}{25}(0)\h^2+\frac{9}{25}(2)\h^2\\
			& = \frac{18}{25}\h^2.
		\ea
		\[
			\braket{\vect L^2} = \frac{18}{25}\h^2
		\]
		\ba
			\braket{\psi|L_z|\psi} & = \frac{16}{25}\braket{1,0,0|L_z|1,0,0}+\frac{9}{25}\braket{2,1,1|L_z|2,1,1}\\
			& = \frac{9}{25}\h
		\ea
		\[
			\braket{L_z} = \frac{9}{25}\h
		\]
		% (b)
		\item
		What is $\ket{\psi(t)}$? What, if any, of the expectation values in (a) vary with time?
		\\
		\\
		Since $\ket{\psi}$ is an energy eigenstate, we have a very simple time dependence. Setting $E_0 \equiv -13.6eV$,
		\[
			\ket{\psi(t)} = e^{-\frac{i}{\h}Ht}\ket{\psi} = \frac{4}{5}e^{-\frac{i}{\h}E_0t}\ket{1,0,0}
			+\frac{3i}{5}e^{-\frac{i}{\h}\frac{E_0}{4}t}\ket{2,1,1}.
		\]
		Because the states $\ket{1,0,0}$ and $\ket{2,1,1}$ are eigenstates of $H$, $\vect L^2$, and $L_z$, the matrix elements
		of operators between any two differing states will always be zero, i.e.
		\[	
			\braket{2,1,1|H,\vect L^2,L_z|1,0,0} = 0.
		\]
		As such, only the same states will have a non-vanishing contribution to the expectation values. However, the time dependence
		between like-states will cancel and thus all expectation values are independent of time.
		\\
		\\
	\eenum

% #7 --------------------------------------------------------------------------------------------------------------------------------------------------------------------------
	\item
	Sakurai 3.10: 
	\benum
		% (a)
		\item
		Consider a pure ensemble of identically prepared spin 1/2 systems. Suppose the expectation values $\braket{S_x}$
		and $\braket{S_z}$ and the sign of $\braket{S_y}$ are known. Show how we may determine the state vector. Why is it 
		unnecessary to know the magnitude of $\braket{S_y}$? 
		\\
		\\
		As a pure ensemble, every particle is within the same spin 1/2 state, and the density operator can be formed as
		\[
			\rho = \ket{\alpha}\bra\alpha.
		\]
		This problem is probably most easily treated in the spinor formalism. Our yet to be determined state is given by
		the spinor
		\[
			\ket{\alpha} \doteq \bpm a\\b\epm	
		\]
		For a given state, only the relative phase difference is of importance in the outplay of physical quantities. Therefore, we may 
		choose to make $a$ real and $b = |b|e^{i\delta}$. 
		\ba
			\braket{S_x} &= \frac{\h}{2} \bpm a& b^* \epm \bpm 0&1\\1&0\epm \bpm a\\b\epm \\
			& = \frac{\h}{2}a(b+b^*) 
		\ea
		
		\ba
			\braket{S_y} &= \frac{\h}{2} \bpm a& b^*\epm \bpm 0&-i\\i&0\epm \bpm a\\b\epm \\
			& = \frac{\h}{2}a(b^*-b)i
		\ea
		
		\ba
			\braket{S_z} &= \frac{\h}{2} \bpm a& b^*\epm \bpm 1&0\\0&-1\epm \bpm a\\b\epm \\
			& = \frac{\h}{2}(a^2-|b|^2).
		\ea
		We shall make things a little simpler by denoting
		\[
			S_i' = \braket{S_i}\frac{2}{\h}.
		\]
		From our normalization condition, we know
		\[
			a^2+|b|^2 = 1.
		\]
		Using this with $S_z'$ we see that
		\[
			S_z'+1 = 2a^2;\quad a = \sqrt{\frac{S_z'+1}{2}}
		\]
		With $a$ in terms of $S_z'$ we can also solve for the magnitude $|b|$ now as
		\[
			|b| = \sqrt{1-a^2} = \sqrt{1-\frac{S_z'+1}{2}}.
		\]
		Now if we look at $S_x'$ we see
		\[
			S_x' = a|b|(e^{i\delta}+e^{-i\delta}) = a|b|2\cos\delta 
		\]
		\[
			\cos\delta = \frac{S_x'}{2a|b|}.
		\]
		Here we have determined the magnitude of our last quantity $\delta$, but the sign of $\delta$ still remains
		unknown. If we look at $S_y'$ we see
		\[
			S_y' = -ia|b|(e^{i\delta}-e^{-i\delta}) = 2a|b|\sin\delta = 2a|b|sign(\delta)\sin(|\delta|)
		\]
		By only knowing the sign of $S_y$ we can determine the sign of $\delta$ by the equation above in conjunction with 
		$\cos\delta = \frac{S_x'}{2a|b|}$. 
		\\
		\\
		Therefore, we have shown that given $\braket{S_x}$, $\braket{S_z}$ and 
		$sign(\braket{S_y})$ we can determine exactly what state $\ket\alpha = a\ket+ + b\ket-$ we have. 
		\\
		\\
		% (b)
		\item
		Consider a mixed ensemble of spin 1/2 systems. Suppose the ensemble averages $[S_x]$, $[S_y]$, and $[S_z]$ are
		all known. Show how we may construct the $2\times 2$ density matrix that characterizes the ensemble.
		\\
		\\
		The ensemble average is defined by 
		\[
			[A] = \tr(\rho A)
		\]
		where $A$ is any operator. But we also know the trace is independent of representation, so the ensemble 
		average can be calculated in any convenient basis. An additional property is that the denisty operator always
		has $\tr(\rho) = 1$ - all probability weights must add to unity and it can be seen to be Hermitian. With this
		information in hand, the most general density operator in spin 1/2 space is written as the hermitian matrix
		\[
			\rho = \bpm a&c\\c^*&b \epm.
		\]
		where $a$ and $b$ are real. Lets now form the ensemble average of the spin operators:
		\ba
			[S_x] &= \frac{\h}{2}\tr(\rho\sigma_1) \\
			& = \frac{\h}{2}\tr\blr{
			\bpm a&c\\c^*&b \epm \bpm 0&1\\1&0 \epm} \\
			&= \frac{\h}{2}(c+c^*)
		\ea
		Similarly, for the other two ensemble averages we find
		\[
			[S_y] = \frac{\h}{2}i(c-c^*)
		\]
		\[
			[S_z] = \frac{\h}{2}(a-b)
		\]
		Lets denote $c = c_1+ic_2$ and $[S_i]'= \frac{2}{\h}[S_i]$. If we include the trace condition for the density matrix,
		we all together have
		\ba
			& a+b = 1\\
			& [S_x]' = 2c_1\\
			&[S_y]' = -2c_2\\
			&[S_z]' = a-b
		\ea
		Then we see
		\[
			a = \frac{[S_z]'+1}{2}
		\]
		\[
			b = 1-\frac{[S_z]'+1}{2}
		\]
		\[
			c_1 = \frac{[S_x]'}{2}	
		\]
		\[
			c_2 = -\frac{[S_y]'}{2}.
		\]
		This completes the construction of the density matrix in terms of $[S_x]$, $[S_y]$, and $[S_z]$. 
		
	\eenum
	
% #8 -------------------------------------------------------------------------------------------------------------------------------------------------------------------------
	\item
	Sakurai 3.12: Consider an ensemble of spin 1 systems. The density matrix is now a $3\times3$ matrix. How many independent (real) 
	parameters are needed to characterize the ensemble completely?
	\\
	\\
	A $3\times 3$ Hermitian matrix may be represented as
	\[
		\bpm a_1&b^*&c^*\\b&a_2&d^*\\c&d&a_3 \epm
	\]
	where $a_1$, $a_2$, $a_3$ are all real. This matrix is characterized by $3+2(3) = 9$ real components. However, 
	from the normalization condition $\tr(\rho) = 1$, we may eliminate one real parameter. Thus a spin 1 density matrix
	is characterized by 8 real parameters. 
	\\
% #9 ------------------------------------------------------------------------------------------------------------------------------------------------------------------------
	\item 
	Sakurai 3.30
	\benum
		% (a)
		\item
		Construct a spherical tensor of rank 1 out of two different vectors $\vect U = (U_x,U_y,U_z)$ and
		$\vect V = (V_x,V_y,V_z)$. Explicitly write $T^{(0)}_{\pm1,0}$ in terms of $U_{x,y,z}$ and $V_{x,y,z}$.
		\\
		\\
		A vector in the spherical tensor basis goes as
		\[
			\vect V = (V_{-1},V_0,V_{1})
		\]
		where the components are related the cartesian vector by
		\[
			V_{-1} = \frac{1}{\sqrt 2}(V_x-iV_y),\qquad V_0 = V_z,\qquad V_1 = -\frac{1}{\sqrt 2}(V_x+iV_y).
		\]
		Although there are other (easier) ways of constructing the rank 1 spherical tensor from two vectors, we shall
		use the most formal way which is based upon the following equation:
		\[
			T_q^{(k)} = \sum_{q_1}\sum_{q_2}\braket{k_1k_2;q_1q_2|k_1k_2;kq}X_{q_1}^{(k_1)}Z_{q_2}^{(k_2)}.
		\]
		If we use $X^{(1)}_{q_1} = U$ and $Z_{q_2}^{(1)} = V$, then we can form a rank 1 spherical tensor by
		\[
			T^{(1)}_q = \sum_{q_1}\sum_{q_2}\braket{11;q_1q_2|11;1q}U_{q_1}V_{q_2}.
		\]
		For $T^{(1)}_{0}$ we have
		\ba
			T_0^{(1)} =& \braket{11;-11|11;10}U_{-1}V_1 \\
			&+ \braket{11;1-1|11;10}U_{1}V_{-1}\\
			&+\braket{11;00|11;10}U_{0}V_0.
		\ea
		The coefficients are the same as the Clebsch-Gordan coefficients. In the sum we have omitted terms in 
		which $q_1+q_2 \ne q$. The last coefficient turns out to be zero and the remaining two have been referenced 
		from a table. The result is
		\[
			T_0^{(1)} = -\frac{1}{\sqrt 2}U_{-1}V_1+\frac{1}{\sqrt 2}U_1V_{-1}.
		\]
		Inputting the vector components in their cartesian form, and temporarily suppressing the normalization constant
		\ba
			T_0^{(1)} &= U_1V_{-1}-U_{-1}V_1 \\
			& = -\frac{1}{2}(U_x+iU_y)(V_x-iV_y)+\frac{1}{2}(U_x-iU_y)(V_x+iV_y)\\
			& = i(U_xV_y-U_yV_x)\\
			& = i(\vect U\times\vect V)_z.
		\ea
		Next, we have $T^{(1)}_1$
		\ba
			T^{(1)}_1 &= \braket{11;10|11;11}U_1V_0+\braket{11;01|11;11}U_0V_1 \\
			& = \frac{1}{\sqrt 2}U_1V_0-\frac{1}{\sqrt 2}U_0V_1.
		\ea
		Converting to cartesian 
		\ba
			T^{(1)}_1 &= U_1V_0-U_0V_1\\
			& = -\frac{1}{\sqrt 2}(U_x+iU_y)V_z+\frac{1}{\sqrt 2}(V_x+iV_y)U_z\\
			& = -\frac{i}{\sqrt 2}[(\vect U\times\vect V)_x+i(\vect U\times\vect V)_y]
		\ea
		Lastly, 
		\ba
			T^{(1)}_{-1} &= \braket{11;-10|11;11-1}U_{-1}V_0+\braket{11;0-1|11;1-1}U_0V_{-1} \\
			& = -\frac{1}{\sqrt 2}U_{-1}V_0+\frac{1}{\sqrt 2}U_0V_{-1}\\
			&\to \frac{1}{\sqrt 2}[-(U_x-iU_y)V_z+U_z(V_x-iV_y)]\\
			& = \frac{i}{\sqrt 2}[(\vect U\times\vect V)_x-i(\vect U\times \vect V)_y].
		\ea
		If we like, we could define $\vect Z = i\vect U\times\vect V$. Then we have
		\[
			T^{(1)}_{\pm 1} = \mp \frac{1}{\sqrt 2}(Z_x+iZ_y),\qquad T_0^{(1)} = Z_z.
		\]
		\\

		% (b)
		\item
		Construct a spherical tensor of rank 2 out of two different vectors $\vect U$ and $\vect V$. Write down explicitly
		$T^{(2)}_{\pm2,\pm1,0}$ in terms of $U_{x,y,z}$ and $V_{x,y,z}$. 
		\\
		\\
		Proceeding in the same manner as part (a), we form the sum, evaluate the coefficients and arrive at (suppressing the 
		common normalization factor)
		\ba
			T^{(2)}_2 &= U_1V_1 \\
			& = \frac{1}{2}(U_x+iU_y)(V_x+iV_y)\\
			\\
			T^{(2)}_1 &= \frac{1}{\sqrt 2}(U_1V_0+U_0V_1) \\
			& = -\frac{1}{2}[(U_x+iU_y)V_z+(V_x+iV_y)U_z]\\
			\\
			T^{(2)}_0 &= \frac{1}{\sqrt 6}(U_1V_{-1}+U_{-1}V_1+2U_0V_0) \\
			& = -\frac{1}{2\sqrt 6}[(U_x+iU_y)(V_x-iV_y)+(U_x-iU_y)(V_x+iV_y)-4U_zV_z]\\
			\\
			T^{(2)}_{-1} &= \frac{1}{\sqrt 2}(U_{-1}V_0+U_0V_{-1}) \\
			& = \frac{1}{2}[(U_x-iU_y)V_z+(V_x-iV_y)U_z]\\
			\\
			T^{(2)}_{-2} &= U_{-1}V_{-1} \\
			& = \frac{1}{2}(U_x-iU_y)(V_x-iV_y)\\
		\ea
			
	\eenum 
	
% #10 --------------------------------------------------------------------------------------------------------------------------------------------------------------------
	\item
	Sakurai 3.32: 
	\benum
		% (a)
		\item 
		Write $xy$, $xz$, and $(x^2-y^2)$ as components of a spherical (irreducible) tensor of rank 2.
		\\
		\\
		In part (a) we formed a rank 1 and rank 2 spherical tensor from two vectors $\vect U$ and $\vect V$. If we 
		chose instead to make $\vect U =\vect V = \vect r$, we can show that the new rank 2 spherical tensor 
		can be expressed in cartesian components as
		\ba
			T^{(2)}_2 & = \frac{1}{2}(x^2-y^2 +2i xy)\\
			T^{(2)}_1 & = -(x+iy)z \\
			T^{(2)}_0 & = \frac{1}{\sqrt 6}(3z^2-r^2)\\
			T^{(2)}_1 & = (x-iy)z \\
			T^{(2)}_2 & = \frac{1}{2}(x^2-y^2-2ixy).
		\ea
		If we play around with combinations of these tensors we find that we can form the desired quantities with
		\ba
			xy &= -\frac{i}{2}(T_2^{(2)}-T_{-2}^{(2)})\\
			xz &= \frac{1}{2}(T_{-1}^{(2)}-T_{1}^{(2)})\\
			x^2-y^2 &= T_2^{(2)}+T_{-2}^{(2)}
		\ea

		%(b)
		\item
		The expectation value 
		\[
			Q \equiv e\braket{\alpha,j,m=j|(3z^2-r^2)|\alpha,j,m=j}
		\]
		is known as the quadrupole moment. Evaluate
		\[
			e\braket{\alpha,j,m'|(x^2-y^2)|\alpha,j,m=j}
		\]
		where $m' = j,j-1,j-2,...$ in terms of $Q$ and appropriate Clebsch-Gordan coefficients. 
		\\
		\\
		First let us notice that
		\[	
			\frac{1}{\sqrt 6}(3r^2-z^2) = T_0^{(2)}
		\]
		so 
		\[
			Q \equiv e\sqrt 6\braket{\alpha,j,m=j|T_0^{(2)}|\alpha,j,m=j}.
		\]
		This means that any expectation values involving $T_0^{(2)}$ may be written as
		\[
			\braket{\alpha,j,j|T_0^{(2)}|\alpha,j,j} =\frac{Q}{e\sqrt 6}.
		\]
		To evaluate
		\[
			e\braket{\alpha,j,m'|(x^2-y^2)|\alpha,j,m=j} = e\braket{\alpha,j,m'|T_2^{(2)}+T_{-2}^{(2)}|\alpha,j,m=j}
		\]
		we implement the Wigner Eckart theorem. This theorem is as follows:
		\[
			\braket{\alpha',j',m'|T_q^{(k)}|\alpha,j,m} = \braket{j,k;m,q|j,k;j',m'}\frac{\braket{\alpha'j'||T^{(k)}||\alpha j}}{\sqrt{
			2j+1}}
		\]
		The double bar matrix element is completely independent of $m'$ and $q$. What this means for our problem
		is that we can use $Q$, a defined value, as our double bar matrix element and solve the desired matrix element in 
		terms of it. Putting this into action, we have
		\ba
			&e\braket{\alpha,j,m'|T_2^{(2)}+T_{-2}^{(2)}|\alpha,j,j} \\
			&= \frac{e\braket{\alpha,j|T^{(2)}|\alpha,j}}{\sqrt{2j+1}}\blr{\braket{j,2;j,-2|j,2;j,m'}+\braket{j,2;j,2|j,2;j,m'}}\\
			& = \frac{Q}{\sqrt 6\sqrt{2(j+1)}}\blr{\braket{j,2;j,-2|j,2;j,m'}+\braket{j,2;j,2|j,2;j,m'}}
		\ea
		In evaluating the different $m' = j,j-1,j-2,..$ we note that the condition for nonvanishing Clebsch-Gordan 
		coefficients is that 
		\[
			m' = m+q \Rightarrow j\pm2. 
		\]
		However $m'$ may not exceed $j$ so the only remaining coefficient corresponds to $m' = j-2$
		and thus our result may be written in final form as
		\[
			e\braket{\alpha,j,m'|(x^2-y^2)|\alpha,j,m=j} =  \frac{Q}{\sqrt 6\sqrt{2(j+1)}}\braket{j,2;j,-2|j,2;j,j-2}.
		\]
	\eenum 
		
		
	 
\eenum
\end{document}