\documentclass[10pt,letterpaper]{article}
\usepackage{macroshw}

\title{\begin{spacing}{1.2}Quantum Mechanics III\\HW 3\end{spacing}}
\author{Matthew Phelps}
\date{Due: Feb. 8 }

\begin{document}
\maketitle

\benum
% #1 -----------------------------------------------------------------------------------------------------------------------------------------------------------------
  	 \item[2.6]

	\benum
	% (a)
	\item
	Show that all eigenvalues of a unitary operator have unit modulus.
	% (b)
	\item
	Show that an operator $U$ is unitary if and only if there is a hermitian operator $A$ such that $U = e^{iA}$. 
	\\ \\
	\eenum
	
	\benum
	\item 
	% (a)
	Unitary operators are normal $[U,U^\dag] = 0$ and thus have a spectral representation
	\[
		U = \sum_n c_n \ket n\bra n.
	\]
	By definition of $UU^\dag = U^\dag U = \mathds 1$
	\ba
		 \sum_n c_n \ket n\bra n\plr{\sum_m c_m^* \ket m\bra m} &= \sum_{n,m} c_nc^*_m \ket n\bra m \braket{n|m}\\
		 & = \sum_{n,m} \delta_{nm}  c_nc^*_m \ket n\bra m \braket{n|m} \\
		 & = \sum_n |c_n|^2 \ket n\bra n = \mathds 1
	\ea
	thus the eigenvalues are of unit modulus $|c_n|^2 = 1$. We can take the adjoint of this expression $(UU^\dag)^\dag$
	to find the same result.  \\
	\item 
	% (b)
	Using the spectral theorem, any unitary operator $U$ can always be written as
	\[
		\sum_n e^{i a_n} \ket{a_n}\bra{a_n}.
	\]
	where, of course, the set $\{\ket{a_n}\}$ is an orthonormal basis and $a_n\in \mathbb R$. Now define 
	the diagonalized operator
	\[
		A = \sum_n a_n \ket{a_n} \bra{a_n}.
	\]
	Since the eigenvalues of this diagonalized operator are real, it is hermitian. Then, we have the relation
	\[
		e^{iA} = \sum_m \frac{(iA)^m}{m!} = \sum_m \frac{i^m}{m!}\sum_n \plr{a_n^m\ket{a_n}\bra{a_n}} 
		= \sum_n e^{ia_n}\ket{a_n}\bra{a_n} = U
	\]
	To prove that any operator in this form must be unitary, we note
	\[
		UU^\dag = U^\dag U = e^{iA}e^{-iA^\dag} =e^{-iA^\dag} e^{iA} = e^{-iA} e^{iA} = \mathds 1
	\]
	Since \emph{every} unitary operator can be represented by $e^{iA}$ with $A=A^\dag$, if there does not exist
	a hermitian operator $A$ such that $U = e^{iA}$, then $U$ cannot be unitary.
	\\ \\
	\eenum 
	
	% 2 ----------------------------------------------------------------------------------------------------------------------------------------------------
	\item[2.7]
	
	Take a subspace $\mathscr S$ of a Hilbert space $\mathscr H$. Suppose we have defined an operator $U$
	with the property that $(U\psi,U\phi) = (\psi,\phi)$ for all vectors $\phi$ and $\phi$ in the subspace
	$\mathscr S$. Show that the operator $U$ can be extended from $\mathscr S$ to the whole Hilbert space
	$\mathscr H$ in such a way that the result is a unitary operator on $\mathscr H$. As always in (my version)
	of QM, you may assume that an arbitrary orthonormal set may be completed to an orthonormal basis. 
	\\ \\
	The preservation of the inner product on a subspace implies, by definition, that the operator $U$ is unitary (on 
	$\mathscr S$):
	\[
		\braket{\psi_\mathscr S|U^\dag U|\phi_\mathscr S} = \braket{\psi_\mathscr S|\phi_\mathscr S}
		\Rightarrow U^\dag U =  \mathds 1
	\]
	\[
		\plr{\braket{\psi_\mathscr S|U^\dag U|\phi_\mathscr S}}^\dag = \braket{\phi_\mathscr S|UU^\dag|\psi_\mathscr S} 
		= \braket{\phi_\mathscr S|\psi_\mathscr S} \Rightarrow UU^\dag = \mathds 1
	\]
	\[
		U^\dag U = U U^\dag = \mathds 1.
	\]
	As such, it is normal and can be spectral decomposed on the subspace $\mathscr S$
	\be\label{1}
		U_\mathscr S = \sum_{n\in\mathscr S} e^{in}\ket n\bra n.
	\ee
	Given the orthonormal subset $\{ \ket n \in \mathscr S\}$ defined in \eqref 1, let us complete this set
	so that it forms an orthonormal basis on the Hilbert space $\mathscr H$. Likewise, let's extend the 
	unitary operator in the same form as
	\[
		U = \sum_{n\in \mathscr H} e^{in}\ket n\bra n.
	\]
	To clarify, the eigenvalues and eigenvectors of the subset $\ket n \in \mathscr S$ remain the same. Now
	the extended operator $U$ is also unitary on the entire Hilbert space. Take two arbitrary vectors 
	$\ket \psi,\ket \phi \in \mathscr H$ 
	\ba
		\braket{\psi|U^\dag  U|\phi} &= 
		\braket{\psi| \sum_{n} e^{in}\ket n\bra n\plr{
		\sum_{m} e^{-im}\ket m\bra m}|\phi}\\
		&= \braket{\psi |\sum_{n }\ket n\bra n|\phi} \\
		& = \braket{\psi |\phi}.
	\ea
	Take the adjoint of this argument as we see that
	\[
		U^\dag U = UU^\dag = \mathds 1.
	\]
	\\ \\
% 3 ------------------------------------------------------------------------------------------------------------------------------------------------------
	\item[2.9]
	Consider a unitary transformation of quantum mechanics defined by the unitary operator $U$ that may
	in the general case depend explicitly on time. 
	
	\benum
	% (a)
	\item
	Show that the time evolution of the transformed state vector is generated by the modified or 
	"effective" Hamiltonian
	\[
		H_E = UHU^\dag + i\h \diff[U]{t}U^\dag.
	\]
	
	% (b)
	\item
	Starting from the preceding result, show that in the Heisenberg picture, the effective Hamiltonian equals
	the zero operator.
	
	% (c)
	\item 
	Suppose the Hamiltonian is of the form $H = H_0+H'$. The transformation generated by $U = e^{iH_0t/\h}$
	leads to what is commonly referred to as the interaction picture. What is the effective Hamiltonian?
	\\
	\eenum
	\benum
	% (a)
	\item
	Under the unitary transformation $U$, states transform as
	\[
		\ket \psi \to \ket{\tilde\psi} = U\ket \psi
	\]
	so
	\[
		\ket \psi = U^\dag \ket{\tilde\psi}.
	\]
	Taking the time evolution equation
	\ba
		i\h \pdiff{t}\plr{ U^\dag \ket{\tilde\psi}} &= H U^\dag\ket{\tilde\psi}\\
		i\h U^\dag \pdiff{t}\ket{\tilde\psi} &= -i\h \pdiff[U^\dag]{t}\ket{\tilde\psi} + HU^\dag \ket{\tilde\psi}\\
		i\h\pdiff{t}\ket{\tilde\psi} & = -i\h U\pdiff[U^\dag]{t}\ket{\tilde\psi}+UHU^\dag\ket{\tilde\psi}\\
		i\h\pdiff{t}\ket{\tilde\psi}& = \plr{-i\h U\pdiff[U^\dag]{t}+UHU^\dag}\ket{\tilde\psi}
	\ea
	hence
	\[
		H_E = -i\h U\pdiff[U^\dag]{t}+UHU^\dag.
	\]
	As all unitary (and possibly time dependent) operators  may be written as 
	\[
		U= e^{iA(t)}\qquad\text{with}\qquad A(t)=A^\dag(t)
	\]
	then
	\[
		H_E = -\h UU^\dag \pdiff[A(t)]{t}+UHU^\dag = -hU^\dag U \pdiff[A(t)]{t}+UHU^\dag = H_E^\dag.
	\]
	Thus we may alternatively express the effective Hamiltonian as
	\[
		H_E = i\h \diff[U]{t}U^\dag + UHU^\dag.
	\]
	\\ \\
	% (b)
	\item 
	Starting with
	\[
		i\h\pdiff{t} \ket{\tilde\psi} = H_E\ket{\tilde\psi}
	\]
	we can choose to represent the time evolution through a unitary operator as:
	\[
		\ket{\tilde\psi(t+\delta t)} = \plr{1-\frac{i\delta tH_E}{\h}}\ket{\tilde\psi(t)}
	\]
	and apply $n$ infinitesimal steps
	\[
		\ket{\tilde\psi(t+n\delta t)} = \plr{1-\frac{i\delta tH_E}{\h}}^n\ket{\tilde\psi(t)}.
	\]
	Now start from initial time $t_0 = 0$ take $n$ steps $\delta t = t/n$ and take the limit
	\[
		\ket{\tilde\psi(t)} = \lim_{n\to\infty}\plr{1-\frac{iH_Et}{\h n}}^n\ket{\tilde \psi(0)} = e^{-\frac{iH_Et}{\h}}\ket{\tilde\psi(0)}.
	\]
	Thus the unitary time evolution operator is
	\[
		U(t) = e^{-\frac{iH_Et}{\h}};\qquad \ket{\tilde\psi(t)} = U(t)\ket{\tilde\psi(0)}.
	\]
	For the rest of the discussion denote the transformed state of part (a) as $\ket{\psi}$. In the Heisenberg
	picture, we effectively apply a unitary transformation that is the adjoint of the time 
	evolution operator: $U'(t) = e^{\frac{iH_Et}{\h}}$. With that,
	\[
		\ket\psi \to \ket{\tilde\psi} = U'(t)\ket{\psi};\qquad \ket{\psi} = U'^\dag\ket{\tilde\psi}.
	\]
	Now we utilize the time evolution equation for the Hamiltonian $H_E$ of part (a)
	\ba
		i\h\pdiff{t}\plr{U'^\dag \ket{\tilde\psi}} &= H_EU'^\dag\ket{\tilde\psi}\\
		i\h U'^\dag\pdiff{t}\ket{\tilde\psi} &= \plr{-H_EU'^\dag+H_EU'^\dag}\ket{\tilde\psi}\\
		i\h\pdiff{t}\ket{\tilde\psi} &= \plr{-U'H_EU'^\dag+U'H_EU'^\dag}\ket{\tilde\psi}\\
		i\h\pdiff{t}\ket{\tilde\psi} &=0 = H_{E'}\ket{\tilde\psi}
	\ea
	Hence in the Heisenberg picture, the effective Hamiltonian is the zero operator. \\ \\
	% (c)
	\item 
	Define the transformation $U = e^{\frac{iH_0t}{\h}}$ with total Hamiltonian $H = H_0+H'$
	\[
		\ket\psi \to \ket{\tilde\psi} = U\ket\psi;\qquad \ket\psi = U^\dag\ket{\tilde\psi}.
	\]
	Then we have the time evolution
	\ba
		i\h\pdiff{t}\plr{U^\dag\ket{\tilde\psi}} &= (H_0+H')U^\dag\ket{\tilde\psi}\\
		i\h  U^\dag\pdiff{t}\ket{\tilde\psi} & = \plr{-i\h\pdiff[U^\dag]{t}+(H_0+H')U^\dag}\ket{\tilde\psi}\\
		i\h \pdiff{t}\ket{\tilde\psi} &= \plr{-UH_0U^\dag+U(H_0+H')U^\dag}\ket{\tilde\psi}\\
		i\h \pdiff{t}\ket{\tilde\psi} &= UH'U^\dag\ket{\tilde\psi}.
	\ea
	Thus in the interaction picture, the unitary transform of the perturbation is the effective Hamiltonian
	\[
		H_E = UH'U^\dag.
	\]
	\\ \\
	\eenum 
% 4 ------------------------------------------------------------------------------------------------------------------------------------------------------
	\item[3.3]
	\benum
	% (a)
	\item
	Take two distinct quantum systems 1 and 2 evolving completely independently of one another with the Hamiltonians
	$H_1$ and $H_2$. Show that time evolution operator for the joint system is the tensor product of the 
	time evolution operators of individual systems.
	% (b)
	\item
	Use the result of part (a) to argue that the Hamiltonian for the joint system is $H = H_1+H_2$.
	% (c)
	\item
	Now apply the above observations to two spins 1/2. Suppose the spin are initially in the entangled
	state 
	\[
		\ket{\psi(t=0)} = \frac{1}{\sqrt 2}\plr{\ket{\uparrow_z}_1\ket{\downarrow_z}_2-\ket{\downarrow_z}_1
					\ket{\uparrow_z}_2}
	\]
	and that each spin evolves in a magnetic field that points in the $x$ direction,
	\[
		H = \h\omega(S_{1,x}+S_{2,x}).
	\]
	What is the state at an arbitrary time $t$?
	\eenum 
	\benum
	% (a)
	\item
	The time evolution of a state in system 1 and system 2 is
	\[
		\ket{\psi(t)}_1 = U_1(t,t_0)\ket{\psi(t_0)}_1;\quad \ket{\psi(t)}_2 = U_2(t,t_0)\ket{\psi(t_0)}_2.
	\]
	Taking $t_0=0$ for convenience, the combined two state system is given as a tensor product of 
	the two states
	\ba
		\ket{\psi(t)}_1\ket{\psi(t)}_2 &= \ket{\psi(t)}_1\otimes\ket{\psi(t)}_2 \\
		&= \plr{U_1(t)\ket{\psi(t_0)}_1} \otimes \plr{U_2(t)\ket{\psi(t_0)}_2 \ket{\psi(t)}_2} \\
		&=U_1(t)\otimes U_2(t)\ket{\psi(0)}_1\ket{\psi(0)}_2.
	\ea
	We may also express this as
	\ba
		\ket{\psi(t)}_1\ket{\psi(t)}_2 &=\plr{ e^{-\frac{i H_1 t}{\h}} }\otimes 
		\plr{e^{-\frac{i H_2 t}{\h}} }\ket{\psi(0)}_1\ket{\psi(0)}_2\\
		& = \plr{ e^{-\frac{i H_1 t}{\h}} }
		\plr{e^{-\frac{i H_2 t}{\h}} }\ket{\psi(0)}_1\ket{\psi(0)}_2
	\ea
	where it understood that $H_1$ only acts on system 1, i.e. $H_1 = (H_1\otimes \mathds 1)$ and likewise for $H_2$. 
	\\ \\
	% (b)
	\item
	Denote the total Hamiltonian of the joint system as
	\[
		\vect H = \vect{H_1}+\vect{H_2} = (H_1\otimes \mathds 1) + (\mathds 1\otimes H_2).
	\]
	Now form the unitary operator
	\ba
		e^{-\frac{i \vect Ht}{\h}} &= e^{-\frac{it}{\h}\blr{(H_1\otimes \mathds 1) + (\mathds 1\otimes H_2)}} \\
		& = e^{-\frac{it}{\h}(H_1\otimes \mathds 1)}e^{-\frac{it}{\h}(H_2\otimes \mathds 1)}
		e^{\frac{it}{2\h}[(H_1\otimes \mathds 1),(H_2\otimes \mathds 1)]}\\
		&=  e^{-\frac{it}{\h}(H_1\otimes \mathds 1)}e^{-\frac{it}{\h}(H_2\otimes \mathds 1)}\\
		& = e^{-\frac{i \vect{H_1} t}{\h}}e^{-\frac{i \vect{H_2} t}{\h}}.
	\ea
	Comparison to part (a) verifies that we may express the time evolution of the joint system by
	a single unitary operator with total hamiltonian
	\[
		\vect H = \vect{H_1}+\vect{H_2}.
	\] \\ \\
	% (c)
	\item
	First, let's determine the evolution of a single spin 1/2 particle in a magnetic system as described earlier:
	\[
		\ket{\uparrow_x} = \frac{1}{\sqrt 2}\plr{ \ket{\uparrow_z}+\ket{\downarrow_z}};\quad 
		\ket{\downarrow_x} = \frac{1}{\sqrt 2}\plr{ \ket{\uparrow_z}-\ket{\downarrow_z}}
	\]
	\[
		\ket{\uparrow_z} = \frac{1}{\sqrt 2}\plr{ \ket{\uparrow_x}+\ket{\downarrow_x}};\quad 
		\ket{\downarrow_z} = \frac{1}{\sqrt 2}\plr{ \ket{\uparrow_x}-\ket{\downarrow_x}}
	\]
	\[
		S_x\ket{\uparrow_z} = \frac{1}{\sqrt 2}\pfrac{\h}{2}\plr{ \ket{\uparrow_x}-\ket{\downarrow_x}} =
		 \frac{\h}{2} \ket\downarrow_z
	\]
	\[
		S_x\ket{\downarrow_z} = \frac{1}{\sqrt 2}\pfrac{\h}{2}\plr{ \ket{\uparrow_x}+\ket{\downarrow_x}} =
		 \frac{\h}{2} \ket\uparrow_z
	\]
%	With $H = \omega S_z$, we have 
%	\ba
%		\ket{\uparrow_z(t)} = U(t)\ket{\uparrow_z(0)} &= e^{-\frac{i\omega S_xt}{\h}}\ket{\uparrow_z}\\
%		&= e^{-\frac{i\omega S_x t}{\h}} \frac{1}{\sqrt 2}\plr{ \ket{\uparrow_x}+\ket{\downarrow_x}} \\
%		&=\frac{1}{\sqrt 2}\plr{ e^{-\frac{i\omega t}{2}}\ket{\uparrow_x}+e^{\frac{i\omega t}{2}}\ket{\downarrow_x}}
%	\ea
%	Similarly,
%	\ba
%		\ket{\downarrow_z(t)} = U(t)\ket{\downarrow_z(0)} &= e^{-\frac{i\omega S_xt}{\h}}\ket{\downarrow_z}\\
%		&= e^{-\frac{i\omega S_x t}{\h}} \frac{1}{\sqrt 2}\plr{ \ket{\uparrow_x}-\ket{\downarrow_x}} \\
%		&=\frac{1}{\sqrt 2}\plr{ e^{-\frac{i\omega t}{2}}\ket{\uparrow_x}-e^{\frac{i\omega t}{2}}\ket{\downarrow_x}}
%	\ea
	Now we apply the unitary operator with Hamiltonian $H=\omega(S_{1,x}+S_{2,x})$ on the entangled system
	\ba
		U(t)\ket{\psi(0)} &= e^{-\frac{i H t}{\h}}\ket\psi \\
		&= \plr{e^{-\frac{i\omega S_{1x}}{\h}}e^{-\frac{i\omega S_{2x}}{\h}}}
		\frac{1}{\sqrt 2}\plr{\ket{\uparrow_z}_1\ket{\downarrow_z}_2-\ket{\downarrow_z}_1
					\ket{\uparrow_z}_2}\\
		& = \plr{e^{-\frac{i\omega S_{1x}}{\h}}e^{-\frac{i\omega S_{2x}}{\h}}}
		\frac{1}{2\sqrt 2}\blr{
		\plr{ \ket{\uparrow_x}_1+\ket{\downarrow_x}_1}\plr{ \ket{\uparrow_x}_2-\ket{\downarrow_x}_2}
		-\plr{ \ket{\uparrow_x}_1-\ket{\downarrow_x}_1}\plr{ \ket{\uparrow_x}_2+\ket{\downarrow_x}_2}
		}\\
		& =  \plr{e^{-\frac{i\omega S_{1x}}{\h}}e^{-\frac{i\omega S_{2x}}{\h}}}
		\frac{1}{\sqrt 2}\plr{ \ket{\downarrow_x}_1\ket{\uparrow_x}_2-\ket{\uparrow_x}_1\ket{\downarrow_x}_2}\\
		& = \frac{1}{\sqrt 2}\plr{ \ket{\downarrow_x}_1\ket{\uparrow_x}_2-\ket{\uparrow_x}_1\ket{\downarrow_x}_2}\\
		& = \ket{\psi(0)}
	\ea
		
	\eenum
	\eenum

\end{document}