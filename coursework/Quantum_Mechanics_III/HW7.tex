\documentclass[10pt,letterpaper]{article}
\usepackage{macroshw}

\title{\begin{spacing}{1.2}Quantum Mechanics III\\HW 7\end{spacing}}
\author{Matthew Phelps}
\date{Due: Mar. 7}

\begin{document}
\maketitle

\benum
% #1 ---------------------------------------------------------------------------------------------------------------------------------------------------------
  	 \item[6.4]
	Starting from equations (6.9) and (6.10), show that the creation and annihilation operators are hermitian conjugates,
	and verify equations (6.11), (6.13), and (6.15). To avoid inessential complications, assume one boson mode only.
	\\ \\
	To show the creation/annihilation operators are hermitian conjugates, denote a single boson mode of $n$ particles as
	the orthonormal states $\{ \ket n\}$, along with the annihilation operator $b$ 
	\[
		b\ket n = \sqrt n\ket{n-1}.
	\]
	Now form the inner product
	\ba
		\braket{n-1|b|n} &= (n-1,bn)  = (b^\dag(n-1),n)\\
		& =\sqrt n.
	\ea
	Since the states are orthonormal, this must imply
	\ba
		&b^\dag\ket{n-1} = \sqrt n \ket n\\
		&\Rightarrow b^\dag\ket n = \sqrt{n+1}\ket{n+1}.
	\ea
	The operator equation above, equivalent to the hermitian conjugate
	of the annihilation operator, is the definition of the creation operator (6.10). \\ \\
	For a single mode, we have
	\[
		[b,b^\dag]\ket n = (n+1)\ket n-n\ket n = \ket n\Rightarrow [b,b^\dag]=1,
	\]
	thus
	\[
		[b,b]=[b^\dag,b^\dag]=0,\quad [b,b^\dag] = 1.
	\]
	As for the number operator
	\[
		\hat n\ket n = b^\dag b\ket n = b^\dag \sqrt n\ket{n-1} = n\ket n.
	\]
	Lastly, we may form any $n$ particle state $\ket n$ of a single boson mode by acting $b^\dag$ on the vacuum:
	\[
		(b^\dag)^n\ket 0 = (b^\dag)^{n-1}\ket 1 = (b^\dag)^{n-2}\sqrt{2}\ket 2 = (b^\dag)^{n-3}\sqrt 3\sqrt{2}\ket 3 
		= ... = \sqrt{n!}\ket n
	\]
	So to get just the state $\ket n$ we divide by $\sqrt{n!}$
	\[
		\frac{(b^\dag)^n}{\sqrt{n!}} = \ket n.
	\]
	\pagebreak
	
% 2 ----------------------------------------------------------------------------------------------------------------------------------------------------
	\item[6.6]
	Show by direct calculation of the relevant commutator that the particle number $\hat N  = \sum_{\vect k\sigma}
	\hat n_{\vect k\sigma}$ is a constant of motion for our interacting electron gas, Eq. (6.47).
	\\ \\
	From the Heisenberg equation of motion for the number operator $\hat N$,
	\[
		-i\h \diff{t}\hat N = [H,\hat N]\overset ! = 0.
	\]
	In computing the commutator
	\ba
		[H,\hat N] & = \blr{\plr{
		\sum_{\vect k,\sigma}\ep_k c^\dag_{\vect k\sigma}c_{\vect k\sigma}
		+\frac{e^2}{2\epo V}\sum_{\vect k\vect p\vect q,\lambda \sigma}
		\frac{1}{q^2}c^\dag_{\vect k+\vect q \lambda}c^\dag_{\vect p-\vect q \sigma}c_{\vect p\sigma}c_{\vect k\lambda}}
		, \sum_{\vect k'\sigma'}c^\dag_{\vect k'\sigma'}c_{\vect k'\sigma'}}.
	\ea 
	we must make use of the anti-commutation relations
	\[
		[c_k,c_l]_+ = [c_k^\dag,c_l^\dag]_+ = 0 \Rightarrow c_kc_l = -c_lc_k,\quad c_k^\dag c_l^\dag = -c_l^\dag c_k^\dag
	\]
	\[
		[c_k,c_l^\dag]_+= \delta_{kl} \Rightarrow c_kc_l^\dag = \delta_{kl} - c_l^\dag c_k
	\]
	along with the commutator of the number operator
	\[
		[\hat n_{\vect k\sigma},\hat n_{\vect k'\sigma'}] = [c^\dag_{\vect k\sigma}c_{\vect k\sigma},
		c^\dag_{\vect k'\sigma'}c_{\vect k'\sigma'}]= 0.
	\]
	The kinetic term in the Hamiltonian consists of number operators (multiplied by associated energies), so
	\ba
		\blr{\sum_{\vect k,\sigma}\ep_k c^\dag_{\vect k\sigma}c_{\vect k\sigma},\hat N} &=
		\sum_{\vect k\sigma \vect k'\sigma'} \ep_k
		\blr{c^\dag_{\vect k\sigma}c_{\vect k\sigma},c^\dag_{\vect k'\sigma'}c_{\vect k'\sigma'}}\\
		& = \sum_{\vect k\sigma \vect k'\sigma'} \ep_k
		\blr{\hat n_{\vect k\sigma},\hat n_{\vect k'\sigma'}}= 0.
	\ea
	As for the interaction term, we cut it down to the form
	\ba
		\blr{\plr{\sum_{\vect k\vect p\vect q,\lambda \sigma}\frac{1}{q^2}
		c^\dag_{\vect k+\vect q \lambda}c^\dag_{\vect p-\vect q \sigma}c_{\vect p\sigma}c_{\vect k\lambda}},
		 \sum_{\vect k'\sigma'}c^\dag_{\vect k'\sigma'}c_{\vect k'\sigma'}}
		 & = \sum_{\vect k\vect p\vect q,\lambda \sigma,\vect k' \sigma'}\frac{1}{q^2}
		 \blr{c^\dag_{\vect k+\vect q \lambda}c^\dag_{\vect p-\vect q \sigma}c_{\vect p\sigma}c_{\vect k\lambda},
		 c^\dag_{\vect k'\sigma'}c_{\vect k'\sigma'}}\\
		&\sim c^\dag_{\vect k+\vect q \lambda}c^\dag_{\vect p-\vect q \sigma}c_{\vect p\sigma}c_{\vect k\lambda}
		 c^\dag_{\vect k'\sigma'}c_{\vect k'\sigma'}
		-  c^\dag_{\vect k'\sigma'}c_{\vect k'\sigma'}
		c^\dag_{\vect k+\vect q \lambda}c^\dag_{\vect p-\vect q \sigma}c_{\vect p\sigma}c_{\vect k\lambda}.
	\ea
	Before we go further, it will be helpful to use the following relation between operators
	\ba
		c_i c_j c_k^\dag&= c_i(-c_k^\dag c_j+\delta_{jk})\\
		& = -c_ic_k^\dag c_j+c_i\delta_{jk}\\
		& = -(-c_k^\dag c_i+\delta_{ik})c_j+c_i\delta_{jk}\\
		 & = c_k^\dag c_ic_j-\delta_{ik}c_j+\delta_{jk}c_i
	\ea
	and similarly
	\[
		c_i c_j^\dag c_k^\dag = c_j^\dag c_k^\dag c_i-\delta_{ik}c_j^\dag +\delta_{ij}c^\dag_k.
	\]
	Now put it to use
	\ba
	&c^\dag_{\vect k+\vect q \lambda}c^\dag_{\vect p-\vect q \sigma}c_{\vect p\sigma}c_{\vect k\lambda}
		 c^\dag_{\vect k'\sigma'}c_{\vect k'\sigma'}
		-  c^\dag_{\vect k'\sigma'}c_{\vect k'\sigma'}
		c^\dag_{\vect k+\vect q \lambda}c^\dag_{\vect p-\vect q \sigma}c_{\vect p\sigma}c_{\vect k\lambda}\\
		& =\{ c^\dag_{\vect k+\vect q \lambda}c^\dag_{\vect p-\vect q \sigma}
		(c^\dag_{\vect k'\sigma'}c_{\vect p\sigma}c_{\vect k\lambda}-\delta_{\vect p\sigma,\vect k'\sigma'}
		c_{\vect k\lambda}+\delta_{\vect k\lambda,\vect k'\sigma'}c_{\vect p\sigma})c_{\vect k'\sigma'}\}
		\\
		&\  -\{  c^\dag_{\vect k'\sigma'}(c_{\vect k+\vect q\lambda}^\dag c^\dag_{\vect p-\vect q\sigma}
		c_{\vect k'\sigma'}-\delta_{\vect p-\vect q\sigma,\vect k'\sigma '}c^\dag_{\vect k+\vect q\lambda}
		+\delta_{\vect k+\vect q\lambda,\vect k'\sigma'}c^\dag_{\vect p-\vect q\sigma})
		c_{\vect p\sigma}c_{\vect k\lambda}\}.
	\ea
	By virtue of $[c_k,c_l]_+ = [c_k^\dag,c_l^\dag]_+ = 0$, we can commute operators in the terms without any delta 			functions to find that they both cancel. We are then left with
	\ba
		 &\quad\{- \delta_{\vect p\sigma,\vect k'\sigma'}c^\dag_{\vect k+\vect q \lambda}c^\dag_{\vect p-\vect q \sigma}
		 c_{\vect k\lambda}c_{\vect k'\sigma'}
		 +  \delta_{\vect k\lambda,\vect k'\sigma'}c^\dag_{\vect k+\vect q \lambda}c^\dag_{\vect p-\vect q \sigma}
		c_{\vect p\sigma}c_{\vect k'\sigma'}\}\\
		 &\qquad -\{-\delta_{\vect p-\vect q\sigma,\vect k'\sigma '}c^\dag_{\vect k'\sigma'}c^\dag_{\vect k+\vect q\lambda}
		 c_{\vect p\sigma}c_{\vect k\lambda}+\delta_{\vect k+\vect q\lambda,\vect k'\sigma'}c^\dag_{\vect k'\sigma'}
		c^\dag_{\vect p-\vect q\sigma}
		c_{\vect p\sigma}c_{\vect k\lambda}\}\\%asdfasd
		& = \{- c^\dag_{\vect k+\vect q \lambda}c^\dag_{\vect p-\vect q \sigma}
		 c_{\vect k\lambda}c_{\vect p\sigma}
		 + c^\dag_{\vect k+\vect q \lambda}c^\dag_{\vect p-\vect q \sigma}
		c_{\vect p\sigma}c_{\vect k\lambda}\}\\%fgh
		 &\qquad -\{-c^\dag_{\vect p-\vect q'\sigma}c^\dag_{\vect k+\vect q\lambda}
		 c_{\vect p\sigma}c_{\vect k\lambda}+c^\dag_{\vect k+\vect q\lambda}
		c^\dag_{\vect p-\vect q\sigma}
		c_{\vect p\sigma}c_{\vect k\lambda}\}\\
		& = \{-c^\dag_{\vect p-\vect q'\sigma}c^\dag_{\vect k+\vect q\lambda}
		 c_{\vect p\sigma}c_{\vect k\lambda}
		 + c^\dag_{\vect k+\vect q \lambda}c^\dag_{\vect p-\vect q \sigma}
		c_{\vect p\sigma}c_{\vect k\lambda}\}\\%fgh
		 &\qquad -\{-c^\dag_{\vect p-\vect q'\sigma}c^\dag_{\vect k+\vect q\lambda}
		 c_{\vect p\sigma}c_{\vect k\lambda}+c^\dag_{\vect k+\vect q\lambda}
		c^\dag_{\vect p-\vect q\sigma}
		c_{\vect p\sigma}c_{\vect k\lambda}\}\\
		& = 0.
	\ea
	Now that my eyes hurt, we can confirm that
	\[
		\diff{t}\hat N = \frac{i}{\h}[H,\hat N] = 0
	\]
	and so $\hat N$, the total number of particles, is a constant of motion. \\ \\
		
		
% 3 ------------------------------------------------------------------------------------------------------------------------------------------------------
	\item[6.12]		
	The boson-fermion model is specified by the Hamiltonian
	\[
		H/\h = \omega a^\dag a+\omega_0 c^\dag c+(\Omega c^\dag a+\Omega^* a^\dag c),
	\]
	where $a$ and $c$ are boson and fermion operators and $\omega$, $\omega_0$, $\Omega$ are appropriate
	constants. Show that the number of excitations $\hat N = a^\dag a+c^\dag c$ is a constant of the motion.
	\\ \\
	Since the spin space of bosons and fermions are fundamentally different (integer vs half integer), it should follow
	that boson operators and fermion operators commute
	\[
		[a,c] = [a^\dag,c^\dag] = [a,c^\dag] = [a^\dag,c] = 0.
	\]
	Boson operators $a$ obey the commutation relation
	\be\label{1}
		[a,a^\dag] = 1
	\ee
	while the fermion operators $c$ obey anti-commutation relation
	\be\label{2}
		[c,c^\dag]_+ = 1.
	\ee
	To find the time dependence of $\hat N$, we commute with the Hamiltonian as before
	\ba
		[H,\hat N] &= \blr{ \omega a^\dag a+\omega_0 c^\dag c+(\Omega c^\dag a+\Omega^* a^\dag c),
		a^\dag a+c^\dag c}\\
		& = \blr{\Omega c^\dag a+\Omega^* a^\dag c,
		a^\dag a+c^\dag c}\\
		& =\Omega  c^\dag[a, a^\dag a]+\Omega a[c^\dag,c^\dag c]+\Omega^* c[a^\dag,a^\dag a]+
		\Omega^* a^\dag[ c,c^\dag c].
	\ea
	The commutations with the number operator are solved using \eqref 1 and \eqref 2
	\[
		[a,a^\dag a] = [a,N_a] = a,\qquad [a^\dag,a^\dag a] =[a^\dag,N_a] = -a^\dag
	\]
	while for fermions
	\[
		[c,c^\dag c] = [c,N_c] = c,\qquad [c^\dag, c^\dag c] = [c^\dag, N_c] = -c^\dag.
	\]
	Altogether then, we have
	\[
		[H,\hat N] = \Omega c^\dag a +\Omega a(-c^\dag)+\Omega^* c(-a^\dag)+\Omega^* a^\dag c = 0
	\]
	Thus $\hat N$ is a constant of motion. \pagebreak
% 4 ------------------------------------------------------------------------------------------------------------------------------------------------------
	\item[6.15]
	Let us study the electron gas. As we know from statistical mechanics, if there were no electron-electron interactions,
	at zero temperature the thermal equilibrium would be the Fermi sea: one-particle states with the wave number $k$
	less than the Fermi wave number $k_F$ are occupied and the rest of the states are empty. Of course, the ground state
	energy of the noninteracting gas is $\frac{3}{5}N\ep_F$. 
	\benum
	% (a)
	\item
	Regarding the electron-electron interactions as a perturbation, find the leading correction to the energy in 
	integral form.\\ \\
	The integral can be done analytically, but you do not need to do it to answer the following questions:
	% (b)
	\item
	In the perturbative limit, do the repulsive electron-electron interactions increase or decrease the energy?
	
	% (c)
	\item
	The interaction energy per electron is obviously a function of density $n$, and at $T=0$ there is no other independent
	intensive thermodynamic variable. How does the interaction energy scale as a function of density?
	\\ \\ Finally:\\ 
	
	% (d)
	\item
	Find the interaction energy explicitly, including the dimensionless multiplicative constant. \\ \\
	\eenum 
	\benum
	% (a)
	\item
	Recall that the first order correction to the energy due to perturbation $H^1$ is
	\[
		E^1_n=\braket{n^0|H^1|n^0}
	\]
	where $\ket{n^0}$ is the unperturbed eigenstate of $H^0$ with energy $E^0_n$. In this case,
	the ground state ($T=0$) is the fermi sea, where all states are occupied up to 
	\[
		k_F = \pfrac{6\pi^2 n}{g}^{1/3}
	\]
	with particle density $n$ and degeneracy $g = 2s+1 =2$. We denote the ground state as 
	\[
		\ket{GS} = \prod_{\sigma, \vect k:|\vect k| < k_F} c^\dag_{\vect k\sigma}\ket 0.
	\]
	The first order contribution is then the expectation of the electron-electron interaction potential $V$
	\[
		E^1 = \braket{GS|V|GS} = 
		\frac{e^2}{2\epo V}\sum_{\vect k\vect p\vect q,\lambda \sigma}
		\frac{1}{q^2}
		\braket{GS|
		c^\dag_{\vect k+\vect q \lambda}c^\dag_{\vect p-\vect q \sigma}c_{\vect p\sigma}c_{\vect k\lambda}
		|GS}.
	\]
	Focusing on the inner product term,
	\[
		\braket{GS|
		c^\dag_{\vect k+\vect q \lambda}c^\dag_{\vect p-\vect q \sigma}c_{\vect p\sigma}c_{\vect k\lambda}
		|GS},
	\]
	 all first order perturbations are zero unless
	\[
		c^\dag_{\vect k+\vect q \lambda}c^\dag_{\vect p-\vect q \sigma}c_{\vect p\sigma}c_{\vect k\lambda}
		\ket{GS} = a\ket{GS}
	\]
	for some eigenvalue $a$. This is due to the $\braket{GS|\psi} = \delta_{GS,\psi}$ where $\ket\psi$ is a single state in the 		Fock space. Since
	$\vect q\ne 0$, the above may only be satisfied given 
	\ba
		c^\dag_{\vect k+\vect q \lambda}c^\dag_{\vect p-\vect q \sigma}c_{\vect p\sigma}c_{\vect k\lambda}
		\ket{GS}& \to \delta_{\lambda,\sigma}\delta_{\vect k+\vect q,\vect p}
		c^\dag_{\vect k+\vect q \lambda}c^\dag_{\vect p-\vect q \sigma}c_{\vect p\sigma}c_{\vect k\lambda}
		\ket{GS}\\
		& = c^\dag_{\vect k+\vect q\sigma}c^\dag_{\vect k\sigma}c_{\vect k+\vect q\sigma}c_{\vect k\sigma}
		\ket{GS}\\
		&= -c^\dag_{\vect k+\vect q\sigma}c_{\vect k+\vect q\sigma}c^\dag_{\vect k\sigma}c_{\vect k\sigma}
		\ket{GS}\\
		& = -\hat n_{\vect k+\vect q\sigma}\hat n_{\vect k\sigma}\ket{GS}
	\ea
	where in the third line, we note that $\vect k\ne \vect k+\vect q$ for the anti-commutation. Thus our summation is now 		only over three indices
	\[
		E^1 = \braket{GS|V|GS} = 
		\frac{e^2}{2\epo V}\sum_{\sigma,\vect k\vect q}
		-\frac{1}{q^2}
		\braket{GS|
		\hat n_{\vect k+\vect q\sigma}\hat n_{\vect k\sigma}
		|GS}
	\]
	The number operators will return a $1$ if $|\vect k+\vect q|,|\vect k| < k_F$ and zero otherwise, and the so the 				expectation may be expressed as step functions.
	 In the thermodynamic
	limit, as we take the length of our box $L\to \infty$, the lattice spacing of our quantized $\vect k$ values becomes 			infinitesimal and we may perform an integration
	\ba
		\frac{e^2}{2\epo V}\sum_{\sigma,\vect k\vect q}
		-\frac{1}{q^2}
		\braket{GS|
		\hat n_{\vect k+\vect q\sigma}\hat n_{\vect k\sigma}
		|GS} &\to -\frac{e^2}{2\epo V}\pfrac{V}{(2\pi)^3}^2 \int d^3k\ d^3q\ \frac{1}{q^2}\theta(k_F-|\vect k|)
		\theta(k_F-|\vect k+\vect q|)
	\ea
	For each direction of $\vect q$, we vary $\vect k$ over all space. Under integration over $k$, the term $|\vect k+\vect q|$ 		does not
	depend on the \emph{direction} of $\vect q$, but rather on the magnitude. So we write the following integral
	\[
		E^1 = -\frac{e^2}{2\epo V}\pfrac{V}{(2\pi)^3}^28\pi^2 \int dq\ dk\ d(\cos\theta_k)\ k^2
		\ \theta(k_F-\sqrt{k^2+q^2+2kq\cos\theta_k})\theta(k_F-|\vect k|)
	\]
	Note that $k$ is bound
	to a maximum of $k_F$, so $k<k_F$. From this we deduce that $0<q<2k_F$. Moreover
	\ba
		k^2+q^2+2kq\cos\theta_k &\le k_F^2 \\
		q\cos\theta_k &\le \frac{k_F^2 - k^2 - q^2}{2k}\\
		q\cos\theta_k &\le -\frac{q^2}{2k_F} \\
		\cos\theta_k &\le -\frac{q}{2k_F}\\
		1\ge\cos\theta_k &\ge \frac{q}{2k_F}
	\ea
	where we have added the upper bound of unity. Thus $ \frac{q}{2\cos\theta_k}<k<k_F$. The integral may then be formed 		as
	\ba
		E^1&=  -\frac{e^2}{2\epo V}\pfrac{V}{(2\pi)^3}^28\pi^2
		 \int_0^{2k_F}dq\ \int_{\frac{q}{2k_F}}^1\ d(\cos\theta_k)\ \int_\frac{q}{2\cos\theta_k}^{k_F} dk\ k^2\\
		 & = -\frac{e^2}{4\pi\epo}V\frac{k_F^4}{4\pi^3}\\
		 & =-\frac{e^2}{4\pi\epo}\frac{V}{4\pi^3} \plr{3\pi^2n}^{4/3}
	\ea
	\\ \\
	
	% (b)
	\item
	From the integration over $\vect q$ and $\vect k$, we see that the integrand is positive and the overall sign of
	the perturbation due to e-e interaction is negative (can also be seen by explicit evaluation of integral). So the interaction 		energy decreases the total energy 
	in the limit of first order perturbation.
	\\ \\
	
	% (c)
	\item[(c),(d)]
	The interaction energy per particle can be found from part (a) as
	\ba
		\frac{E^1}{N} &= -\frac{e^2}{4\pi\epo}\frac{1}{4\pi^3} \plr{3\pi^2}^{4/3}n^{1/3}\\
		&\sim -n^{1/3}
	\ea
	\eenum
\eenum
\end{document}