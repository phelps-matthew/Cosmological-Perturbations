\documentclass[10pt,letterpaper]{article}
\usepackage{macroshw}

\title{\begin{spacing}{1.2}Quantum Mechanics III\\HW 8\end{spacing}}
\author{Matthew Phelps}
\date{Due: Mar. 21}

\begin{document}
\maketitle

\benum
% #1 ---------------------------------------------------------------------------------------------------------------------------------------------------------
\item[6.7]
Consider Bosons as an example. By studying the expression $\hat n(\vect R)\psi^\dag(\vect r)\ket 0$, argue that the 
effect of the field operator $\psi^\dag(\vect r)$ on the vacuum is to put a boson in the point $\vect r$ in space. 
Multiple operators $\psi^\dag$ with different position arguments obviously work analogously, but how about
$\psi^\dag(\vect r)\psi^\dag(\vect r)$?
\\ \\
Expanding out the operators in terms of creation/annihilation operators in terms of orthonormal basis $\{u_k\}$
\ba
		\hat n(\vect R)\psi^\dag(\vect r)\ket 0 &= \sum_{k_1, k_2, k_3}
		u^*_{k_3}(\vect R)u_{k_2}(\vect R)u^*_{k_1}(\vect r)b_{k_3}^\dag b_{k_2}b^\dag_{k_1}\ket 0\\
		& = \sum_{k_1, k_2, k_3}
		u^*_{k_3}(\vect R)u_{k_2}(\vect R)u^*_{k_1}(\vect r)b_{k_3}^\dag \delta_{k_1,k_2}\ket 0\\
		&= \sum_{k_2, k_3}
		u^*_{k_3}(\vect R)\bigg(u_{k_2}(\vect R)u^*_{k_2}(\vect r)\bigg)b_{k_3}^\dag \ket 0\\
		&= \sum_{k_3}
		u^*_{k_3}(\vect R)\delta(\vect R-\vect r)b_{k_3}^\dag \ket 0\\
		&=\delta(\vect R-\vect r) \sum_k u_k^*(\vect R)b_k^\dag\ket 0\\
		& = \delta(\vect R-\vect r)\psi^\dag(\vect R)\ket 0.
\ea
Hence $\psi(\vect r)\ket 0$ is a non-zero eigenstate of the particle density operator only at $\vect R = \vect r$. Moreover, note that $b_k^\dag \ket 0=\ket{u_k}$ (a single particle state with all other $u_k$ unoccupied) and contract the expression with the position eigenstate $\bra{\vect r'}$
\ba
	\bra{\vect r'} \psi^\dag(\vect r)\ket 0 &=  \bra{\vect r'} \sum_k u_k^*(\vect r)b_k^\dag\ket 0\\
	&=\sum_k u_k^*(\vect r)\braket{\vect r'|u_k}\\
	& = \sum_k u_k^*(\vect r)u_k(\vect r')\\
	& = \delta(\vect r-\vect r')\\
	& = \braket{\vect r'|\vect r}
\ea
Thus  $\psi^\dag(\vect r)\ket 0$ represents a single localized state at position $\vect r$. If we take our basis 
$\{ u_k\} $ 
as the eigenstates of momentum $\{ \ket{\vect k}\}$ (quantized in a cubic box with periodic b.c.'s for example), we identity 
$\psi^\dag(\vect r)\ket 0 = \sum_{\vect k} \braket{\vect k|\vect r}\ket{\vect k}$ as the familiar Fourier expansion of the position operator
in terms of momentum, viz. a localized state is a linear superposition of all momentum states. \\ \\
As for the next part,
\ba
	\hat n(\vect r')\psi^\dag(\vect r)\psi^\dag(\vect r)\ket 0 &= \sum_{k_1,k_2,k_3,k_4}
	u^*_{k_4}(\vect r')u_{k_3}(\vect r')u^*_{k_2}(\vect r)u^*_{k_1}(\vect r) 
	b_{k_4}^\dag b_{k_3}b^\dag_{k_2}b^\dag_{k_1}\ket 0\\
	&= 
	\sum_{k_1,k_2,k_3,k_4}
	u^*_{k_4}(\vect r')u_{k_3}(\vect r')u^*_{k_2}(\vect r)u^*_{k_1}(\vect r) 
	b_{k_4}^\dag(\delta_{k_2,k_3}+b^\dag_{k_2}b_{k_3})b^\dag_{k_1}\ket 0\\
	&= 
		\sum_{k_1,k_2,k_3,k_4}
	u^*_{k_4}(\vect r')u_{k_3}(\vect r')u^*_{k_2}(\vect r)u^*_{k_1}(\vect r)b_{k_4}^\dag \bigg(
	\delta_{k_2,k_3}b_{k_1}^\dag \ket 0+b^\dag_{k_2}b_{k_3}b^\dag_{k_1}\ket 0\bigg )\\
	&= 
		\sum_{k_1,k_2,k_4}
	u^*_{k_4}(\vect r')u_{k_2}(\vect r')u^*_{k_2}(\vect r)u^*_{k_1}(\vect r)b^\dag_{k_4}b^\dag_{k_1}\ket 0\\
	&\qquad +\sum_{k_1,k_2,k_4}u^*_{k_4}(\vect r')u_{k_1}(\vect r')u^*_{k_2}(\vect r)u^*_{k_1}(\vect r)
	b_{k_4}^\dag b^\dag_{k_2}\ket 0\\
	&= 
		\delta(\vect r'-\vect r)\sum_{k_1,k_4}
	u^*_{k_4}(\vect r')u^*_{k_1}(\vect r)b_{k_4}^\dag b_{k_1}^\dag \ket 0\\
	&\qquad +\delta(\vect r'-\vect r)\sum_{k_2,k_4}u^*_{k_4}(\vect r')u^*_{k_2}(\vect r)
	b_{k_4}^\dag b^\dag_{k_2}\ket 0\\
	&= 2\delta(\vect r'-\vect r)\sum_{k_1,k_2} u_{k_2}^*(\vect r')u_{k_1}^*(\vect r)b_{k_2}^\dag b_{k_1}^\dag \ket 0\\
	&= 2\delta(\vect r'-\vect r) \psi^\dag(\vect r')\psi^\dag(\vect r)\ket 0
\ea
We have formed an eigenstate of the particle density operator with eigenvalue 2 if and only if $\vect r'= \vect r$. This creates
precisely two particles at position $\vect r$. \\ \\
% #2 ---------------------------------------------------------------------------------------------------------------------------------------------------------
\item[6.8]
Consider a spin-0 Bose gas; for a multicomponent gas there are analogous results, but we do not go into this. Just as one
may write the Heisenberg equation of motion for the field operator, one can do it also for the particle density operator
$\hat n(\vect r)$. Now, if it is possible to identify a many-body operator $\vecth j(\vect r)$ in such a way that the equation
of continuity $\pdiff{t} \hat n(\vect r)+\del\cdot\vecth j(\vect r) = 0$ is valid, $\vecth j(\vect r)$ is evidently the quantum
operator for the particle current density. Find it!
\\ \\
If we can find $\diff{t}\hat n$ then hopefully we can express this as a divergence of some vector quantity, thus finding
the current density. Under second quantization, we may express our Hamiltonian 
\[
	H = \int d^3r\ \psi^\dag(\vect r)\blr{ -\frac{\h^2}{2m}\del^2+U(\vect r)}\psi(\vect r)+
	\frac{2\pi\h^2 a}{m}\int d^3r\ \psi^\dag(\vect r)\psi^\dag(\vect r)\psi(\vect r)\psi(\vect r).
\]
We include the two body operator of the contact interaction model for added generality, though we will find it is independent of
 $\diff{t}\hat n$.
%Now take the commutator
%\[
%	 [\hat n,H] = \blr{\psi^\dag(\vect r)\psi(\vect r),\psi^\dag(\vect r')\plr{ -\frac{\h^2}{2m}\del'^2+U(\vect r')}\psi(\vect r')
%	 +\frac{2\pi\h^2 a}{m}\psi^\dag(\vect r')\psi^\dag(\vect r')\psi(\vect r')\psi(\vect r')}
%\]
%The first term is
%\ba
%	 \blr{\psi^\dag(\vect r)\psi(\vect r),\psi^\dag(\vect r')\plr{ -\frac{\h^2}{2m}\del'^2+U(\vect r')}\psi(\vect r')}&=
%	 \blr{\psi^\dag(\vect r)\psi(\vect r),\psi^\dag(\vect r')}\plr{ -\frac{\h^2}{2m}\del'^2+U(\vect r')}\psi(\vect r')\\
%	 &\quad+
%	\psi^\dag(\vect r')  \blr{\psi^\dag(\vect r)\psi(\vect r),\plr{ -\frac{\h^2}{2m}\del'^2+U(\vect r')}\psi(\vect r')}\\
%	&=  \blr{\psi^\dag(\vect r)\psi(\vect r),\psi^\dag(\vect r')}\plr{ -\frac{\h^2}{2m}\del'^2+U(\vect r')}\psi(\vect r')\\
%	 &\quad+
%	 \psi^\dag(\vect r') \plr{ -\frac{\h^2}{2m}\del'^2+U(\vect r')} \blr{\psi^\dag(\vect r)\psi(\vect r),\psi(\vect r')}
%\ea
%Note that $U(\vect r')$ is not an operator here. It is just a scalar function of position, originating inside the Hamiltonian
%integral. Additionally, $\del'^2$ passes through $\psi(\vect r)$ and $\psi^\dag(\vect r)$. The two relavent commutors are
%calculated as
%\[
%	 [\psi^\dag(\vect r)\psi(\vect r),\psi^\dag(\vect r')]= \psi^\dag(\vect r)[\psi(\vect r),\psi^\dag(\vect r')] = 
%	 \psi^\dag(\vect r)\delta(\vect r-\vect r') 
%\]
%and
%\[
%	[\psi^\dag(\vect r)\psi(\vect r),\psi(\vect r')]= [\psi^\dag(\vect r),\psi(\vect r')]\psi(\vect r)
%	= -\delta(\vect r-\vect r')\psi(\vect r).
%\]
%These results together give use the first commutator term
%\ba
%	 \blr{\psi^\dag(\vect r)\psi(\vect r),\psi^\dag(\vect r')\plr{ -\frac{\h^2}{2m}\del'^2+U(\vect r')}\psi(\vect r')}&=
%	 \psi^\dag(\vect r)\plr{ -\frac{\h^2}{2m}\del^2+U(\vect r)}\psi(\vect r) 
%	 - \psi^\dag(\vect r)\plr{ -\frac{\h^2}{2m}\del'^2+U(\vect r')}\psi(\vect r)
%\ea	
%Thus
%\[
%	-i\h \pdiff{t}\psi^\dag (\vect r)=
%	\blr{\frac{-\h^2}{2m}\del^2+U(\vect r)}\psi^\dag(\vect r)+\frac{4\pi\h^2 a}{m}
%	\psi^\dag(\vect r)\hat n^\dag(\vect r)
%\]
%\ba
%	 [\hat n,H] &= \delta(\vect r-\vect r')\psi^\dag(\vect r)\plr{ -\frac{\h^2}{2m}\del'^2+U(\vect r')}\psi(\vect r')
%	 -\psi^\dag(\vect r')\plr{ -\frac{\h^2}{2m}\del'^2+U(\vect r')}\delta(\vect r-\vect r')\psi(\vect r) \\
%	 &= \psi^\dag(\vect r)\plr{ -\frac{\h^2}{2m}\del^2+U(\vect r)}\psi(\vect r)-
%	 \psi^\dag(\vect r)\plr{ -\frac{\h^2}{2m}\del^2+U(\vect r)}\psi(\vect r) 
%\ea
Using the result derived in the script for $\pdiff{t}\psi(\vect r)$ along with its conjugate
\ba
	i\h\pdiff{t}(\psi^\dag(\vect r)\psi(\vect r)) &= i\h\plr{ \pdiff{t}\psi^\dag(\vect r)\psi(\vect r) +\psi^\dag(\vect r)
	\pdiff{t}\psi(\vect r)}\\
	&= -\blr{\frac{-\h^2}{2m}\del^2+U(\vect r)}\psi^\dag(\vect r)\psi(\vect r)-\frac{4\pi\h^2 a}{m}
	\psi^\dag(\vect r)\hat n^\dag(\vect r)\psi(\vect r)\\
	&\qquad + \psi^\dag(\vect r)\blr{\frac{-\h^2}{2m}\del^2+U(\vect r)}\psi(\vect r)+\frac{4\pi\h^2 a}{m}
	\psi^\dag(\vect r)\hat n(\vect r)\psi(\vect r)
\ea
$\ ^\dag$Since $U(\vect r)$ commutes with the $\psi,\psi^\dag$ operators and $\hat n = \hat n^\dag$, this leaves us with
\[
	i\h\diff{t}\hat n = -\frac{\h^2}{2m}\bigg[\psi^\dag(\vect r)\del^2\psi(\vect r)-(\del^2\psi^\dag(\vect r))\psi(\vect r)\bigg].
\]
This can be more symmetrically arranged by noting that $\psi$ and $\psi^\dag$ are both evaluated at $\vect r$, 
so their commutator vanishes
\[
	i\h\diff{t}\hat n = -\frac{\h^2}{2m}\bigg[\psi^\dag(\vect r)\del^2\psi(\vect r)-\psi(\vect r)\del^2\psi^\dag(\vect r)\bigg].
\]
We can express the right hand side in terms of a divergence
\[
	\del\cdot \plr{ \psi^\dag(\vect r)\del\psi(\vect r)-\psi(\vect r)\del\psi^\dag(\vect r)}=
	\psi^\dag(\vect r)\del^2\psi(\vect r)-\psi(\vect r)\del^2\psi^\dag(\vect r).
\]
Cross terms from the derivative cancel from again $[\psi(\vect r),\psi^\dag(\vect r)] = 0$. Thus the particle 
current density is 
\[
	\vecth j = -\frac{i\h}{2m}\plr{ \psi^\dag(\vect r)\del\psi(\vect r)-\psi(\vect r)\del\psi^\dag(\vect r)}.
\]
\\ \\
$\ ^\dag$$U(\vect r)$ is actually just a scalar as a function of position. It originated in the integrand of the Hamiltonian 
and the actual operators reside in the fields $\psi\sim a_{\vect k}$ and $\psi^\dag\sim a^\dag_{\vect k}$.
\\ \\
% #3 ---------------------------------------------------------------------------------------------------------------------------------------------------------
\item[6.10]	 
In terms of plane-wave states labeled by the wave number $\vect k$, the Hamiltonian of a single-component Bose gas
under the contact interaction model with scattering length $a$ reads
\[
	H = \sum_k \ep_k b_{\vect k}^\dag b_{\vect k}+\frac{2\pi\h^2 a}{mV}
	\sum_{\substack{\vect k_1,\vect k_2,\vect k_3,\vect k_4\\ \vect k_1+\vect k_2 = \vect k_3+\vect k_4}}
	b_{\vect k_1}^\dag b_{\vect k_2}^\dag b_{\vect k_3}b_{\vect k_4}
\]
In a noninteracting zero-temperature Bose-Einstein condensate all $N\gg 1$ atoms are in the state with $\vect k =0$.
\benum
% (a)
\item
Use perturbation theory to calculate the energy of the condensate to first order in the atom-atom interaction strength.
\\ \\
At $T=0$, all $N$ particles lie in the $\vect k=0$ state, which can be expressed as the Fock state $\ket\psi$
\[
	\ket\psi = \ket{N,0,0,...}.
\]
Treating the gas nonrelativistically (appropriate for a finite number of $N$ atoms with large masses), the dispersion 
relation is
\[
	\ep_k = \frac{\h^2 k^2}{2m}
\]
and, consequently, the noninteracting contribution from the Hamiltonian is zero
\[
	H^0\ket\psi = \ep_0 \hat n_0 \ket\psi = 0\Rightarrow E^0 = 0.
\]
However, taking the contact interaction as a perturbation, we have the first order correction to the energy
\[
	E^1 = \frac{2\pi\h^2 a}{mV}
	\sum_{\substack{\vect k_1,\vect k_2,\vect k_3,\vect k_4\\ \vect k_1+\vect k_2 = \vect k_3+\vect k_4}}
	 \braket{\psi|b_{\vect k_1}^\dag b_{\vect k_2}^\dag b_{\vect k_3}b_{\vect k_4}|\psi}=
	 \sum_{\vect k_1,\vect k_2,\vect k_3,\vect k_4}\delta_{\vect k_1+\vect k_2,\vect k_3+\vect k_4}
	 \braket{\psi|b_{\vect k_1}^\dag b_{\vect k_2}^\dag b_{\vect k_3}b_{\vect k_4}|\psi}
\]
The expectation term
\[
	 \braket{\psi|b_{\vect k_1}^\dag b_{\vect k_2}^\dag b_{\vect k_3}b_{\vect k_4}|\psi}
	 =  \braket{...,0,0,N||b_{\vect k_1}^\dag b_{\vect k_2}^\dag b_{\vect k_3}b_{\vect k_4}|N,0,0,...}
\]
can only be nonzero given that 
\[
	b^\dag_{\vect k_1}b^\dag_{\vect k_2}b_{\vect k_3}b_{\vect k_4}\ket{\psi} \to 
	\delta_{\vect k_1,0}\delta_{\vect k_2,0}\delta_{\vect k_3,0}\delta_{\vect k_4,0}
	b^\dag_{\vect k_1}b^\dag_{\vect k_2}b_{\vect k_3}b_{\vect k_4}\ket{\psi}.
\]
This still preserves the relation $\vect k_1+\vect k_2 = \vect k_3+\vect k_4$. Now our sum reduces to
\ba
	E^1 &=  \frac{2\pi\h^2 a}{mV}\braket{...,0,0,N| 
	b^\dag_{0}b^\dag_{0}b_{0}b_{0}
	|N,0,0,...} \\
	&=  \frac{2\pi\h^2 a}{mV} \sqrt N\sqrt{N-1}\sqrt{N-1}\sqrt{N}\braket{...,0,0,N|N,0,0,...}\\
	 &= \frac{2\pi\h^2 a}{mV} N(N-1)
\ea
% (b)
\item
What is the pressure of this weakly interacting condensate?
\\ \\
Although the usual thermodynamic quantities like entropy and pressure do not exist in a BEC condensate (in the thermodynamic limit), when we 
add in interactions, they may. Proceeding on the assumption they do, we have the Helmholtz free energy $F(T,V)$,
\[
	F = U-TS
\]
and its relation to pressure
\[
	p=-\plr{\pdiff[F]{V}}_T.
\]
At $T=0$ then, $F=U$ and $p = -\pdiff[U]{V}$. Thus
\[
	p =  \frac{2\pi\h^2 a}{mV^2}(N^2-N) .
\]
For $N \gg 1$ this may be approximated as
\[
	p = \frac{2\pi\h^2 a}{mV^2}N^2 =  \frac{2\pi\h^2 a}{m}n^2.
\]
\eenum 
\eenum
\end{document}