\documentclass[10pt,letterpaper]{article}
\usepackage{macroshw}

\title{\begin{spacing}{1.2}Quantum Mechanics III\\HW 9\end{spacing}}
\author{Matthew Phelps}
\date{Due: Mar. 28}

\begin{document}
\maketitle

\benum
% #1 ---------------------------------------------------------------------------------------------------------------------------------------------------------
\item[7.1]
\benum
\item
% (a)
Write down the density operator of the electromagnetic field at temperature $T$ and chemical potential $\mu =0$.
\\ \\
\[
	\rho = \elr{\frac{1}{\mathcal Z}e^{-\beta(\hat H-\mu \hat N)}}_{\mu=0} = \frac{e^{-\beta \hat H}}{\mathcal Z}
\]
where
\[
	\mathcal Z = \elr{\tr\  e^{-\beta(\hat H-\mu \hat N)}}_{\mu = 0} = \tr\ e^{-\beta\hat H}.
\]
The Hamiltonian of the electromagnetic field, given in a basis of the Fock space characterized by free photons with
mode energy $\h\omega_n$, is
\[
	\hat H = \sum_n \h\omega_n a_n^\dag a_n.
\]
Here $n$ denotes a general index of the photon mode; it could, for example, stand for a momentum index $\vect k$ and polarization vector $\lambda$. This defines the density operator given above. \\ \\
\item
% (b)
Show that the grand partition function is
\[
	\mathcal Z = \prod_n \frac{1}{1-e^{-\beta\h\omega_n}},
\]
where the product runs over all photon modes.
\\ \\
\[
	\mathcal Z = \tr\plr{ e^{-\beta\sum_n \h\omega_n a_n^\dag a_n }} 
\]
To trace over all possible photon modes, we may first sum over a fixed particle number $N$ and then sum over all
possible permutations of the occupation numbers $n_i$ of single mode states $i$
\ba
	\mathcal Z &= \sum_N\sum_{\{\{n_i\}:\sum_i n_i= N\}} e^{-\sum_i \beta\h\omega_i n_i }\\
	&= \sum_{\{ n_i\}} e^{-\sum_i  \beta\h\omega_i n_i }\\
	&= \sum_{\{n_i\}} \plr{ \prod_i e^{-\beta\h\omega_i n_i }}\\
	&= \prod_i \plr{\sum_{n_i}e^{- \beta\h\omega_i n_i }}\\
	& = \prod_i \frac{1}{1-e^{-\beta \h\omega_i}}
\ea
Summing over all possible sets $\{\{n_i\}:\sum_i n_i= N\}$ and then summing over $N$ is equivalent to
carrying out a sum over all possible sets of $\{ n_i\}$, unconstrained. In addition, as we sum over all 
sets ${\{ n_i\}}$, the term
$\sum_{\{ n_i\}}\prod_i$ can be rearranged to a product of a sum, analogous to factorization. \\ \\
\eenum
% #2 ---------------------------------------------------------------------------------------------------------------------------------------------------------
\item[7.2]
As we have discussed, in atom-field interactions two forms of the interaction Hamiltonian are in common use.
Let us denote these by
\[
	H_{\vect d\cdot \vect E} = -q\vect r\cdot \vect E(t),\quad H_{\vect p\cdot\vect A} = -\frac{q}{m}\vect p\cdot\vect A(t).
\]
The term quadratic in $\vect A$ is immaterial in the present context, as it does not involve the quantum variables
$\vect r$ or $\vect p$ and can be removed with a trivial unitary transformation.
\benum
\item 
% (a)
By studying the commutator $[x,H_A]$ show that the matrix elements of position between the eigenstates $\ket n$ of
the atomic Hamiltonian $H_A$ satisfy $\braket{m|\vect p|n} = im\omega_{mn}\braket{m|\vect r|n}$, where
$\omega_{nm} = (E_m-E_n)/\h$ is the frequency difference between the states $\ket m$ and $\ket n$. 
\item
% (b)
Suppose the atom is driven by electromagnetic fields such that $\vect A(t)$ tends to zero smoothly with $t\to \pm\infty$.
Consider transitions from some initial $(t=-\infty)$ state of the atom $\ket n$ to the other states using first-order
time dependent perturbation theory. Show that the transition probabilities after the fields have turned back to zero
$(t=\infty)$, are the same for the two choices of the interaction Hamiltonian. \\ \\
\eenum
\benum
% (a)
\item 
Given the atomic Hamiltonian
\[
	H_A = \frac{\vect p^2}{2m}+V(\vect r)
\]
and its commutator with position
\[
	[\vect r,H_A] = \frac{1}{2m}[\vect r,\vect p^2] = \frac{\vect p}{m}[\vect r,\vect p] = i\h \frac{\vect p}{m},
\]
take the commutator between the expectation of eigenstates $\ket m$ and $\ket n$ of $H_A$
\ba
	\braket{m|[H_A,\vect r]|n} &= \braket{m|H_A\vect r-\vect rH_A|n}\\
	&= \h\omega_{mn}\braket{m|\vect r|n}\\
	&= -\frac{i\h}{m} \braket{m|\vect p|n}.
\ea
Thus
\[
	im\omega_{mn}\braket{m|\vect r|n} = \braket{m|\vect p|n}.
\]
\\ \\

\item
% (b)
To first order, the probability of transition from initial state $\ket n$ to state $\ket m$ is given by the time dependent 
coefficient $c_m(t)$ (more specifically the modulus square)
\ba
	c_m^1(t) &= \frac{-i}{\h} \int^\infty_{-\infty} dt\ \braket{m|H_{int}|n} e^{i\omega_{mn}t}.
\ea
We evaluate with both forms of the interaction Hamiltonian
\ba
	c_m^1(t) &= \frac{-i}{\h} \int^\infty_{-\infty} dt\ \braket{m|H_{int}|n} e^{i\omega_{mn}t}\\
	&= \frac{iq}{\h}\int dt\ \braket{m|\vect r\cdot\vect E(t)|n}e^{i\omega_{mn}t}\\
	&=  \frac{iq}{\h}\int dt\ \braket{m|\vect r|n}\cdot\vect E(t) e^{i\omega_{mn}t}\\
	&= -\frac{iq}{\h} \int dt\ \braket{m|\vect r|n}\cdot \pdiff[\vect A(t)]{t} e^{i\omega_{mn}t}\\
	&= -\frac{q}{\h m\omega_{mn}}\int dt\ \braket{m|\vect p|n}\cdot\pdiff[\vect A(t)]{t} e^{i\omega_{mn}t}\\
	&= \frac{i\omega_{mn}q}{\h m\omega_{mn}}\int dt\ \braket{m|\vect p|n}\cdot \vect A(t)e^{i\omega_{mn}t}\\
	&= \frac{iq}{\h m}\int dt\ \braket{m|\vect p\cdot \vect A(t)|n}e^{i\omega_{mn}t}
\ea
where in the sixth line, we integrate by parts with a vanishing surface term 
\[
	\braket{m|\vect p|n}\cdot\vect A(t)e^{i\omega_{mn}t}|_{-\infty}^\infty =0.
\]
We see our result for $c^1_m(t)$ is the same for either interacting Hamiltonian. \\ \\
\eenum
% #3 ---------------------------------------------------------------------------------------------------------------------------------------------------------
\item[7.3]
Study a transition in an atom where the lower state has the angular momentum $J=0$ and the upper state the 
angular momentum $J'=1$. Using the Wigner-Eckart theorem, show that the dipole matrix elements between the  state 
$J=0$, $m=0$ and $J'=1$, $m'=0$ may be chosen to be a real vector that points along the quantization $(z)$ axis.
\\ \\
The dipole matrix elements between states $\ket{jm}$ that we seek are
\[
	\braket{00|\vect d|10} = q\braket{00|\vect r|10}.
\]
Meanwhile, according the Wigner-Eckart theorem
\[
	\braket{00|T^{(k)}_q|10} = \braket{10;kq|00}\braket{0||T^{(k)}||1}
\]
where $T^(k)_q$ is an irreducible spherical tensor of rank $k$, $\braket{10;kq|00}$ is a Clebsch-Gordon coefficient, and
$\braket{0||T^{(k)}||1}$ is proportionality factor independent of the geometric features. To compute the dipole elements,
we may form a rank 1 spherical tensor, with a representation along quantization axis $z$
\[
	z = T^1_0,\qquad y = \frac{i(T_{-1}^1+T_1^1)}{\sqrt 2},\qquad x = \frac{T^1_{-1}-T^1_1}{\sqrt 2}.
\]
Putting the theorem to use, we then have (using a shorthand for the Clebsch-Gordon coefficients)
\[
	\braket{00|\vect d_x|10}= \frac{q}{\sqrt 2}\braket{0||T^{(1)}||1}(C^{00}_{1(-1)10}-C^{00}_{1110}) = 
	 \frac{q}{\sqrt 2}\braket{0||T^{(1)}||1}(0-0) =0
\]
\[
	\braket{00|\vect d_y|10}= i\frac{q}{\sqrt 2}\braket{0||T^{(1)}||1}(C^{00}_{1(-1)10}+C^{00}_{1110}) = 
	 i\frac{q}{\sqrt 2}\braket{0||T^{(1)}||1}(0+0) =0
\]
\[
	\braket{00|\vect d_z|10}= q\braket{0||T^{(1)}||1}C^{00}_{1010}
\]
Treating each matrix element as a vector component, we may represent $\braket{00|\vect d|01}$ as a real vector $\vect a$ of which only one component is nonzero, the component along the quantization axis. Note that the common factor of the double bar matrix element (reduced matrix) is set through normalization. 

\eenum

\end{document}