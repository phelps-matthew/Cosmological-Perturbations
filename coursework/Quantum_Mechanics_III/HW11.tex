\documentclass[10pt,letterpaper]{article}
\usepackage{macroshw}

\title{\begin{spacing}{1.2}Quantum Mechanics III\\HW 11\end{spacing}}
\author{Matthew Phelps}
\date{Due: April 11}

\begin{document}
\maketitle

\benum
% #1 --------------------------------------------------------------------------------------------------------------------------------------------------------
\item[7.9]
Let us study a single mode $a$ of an electromagnetic field under the usual cavity damping, so that the master
equation reads
\[
	\dot\rho = \gamma(2a\rho a^\dag - a^\dag a\rho - \rho a^\dag a).
\]
Ignore the Hamiltonian evolution, which just unnecessarily complicates the argument. \\
\benum
\item
Show that the normally ordered moments of the electromagnetic field $M_{p,q}(t) = \tr[\rho(t)a^{p\dag}a^q]$,
with $p$ and $q$ being nonnegative integers, evolve in time according to $M_{p,q}(t) = 
e^{-(p+q)\gamma t}M_{p,q}(0)$. \\
\item
According to a famous theorem by Glauber (paraphrased), the one and only state of the electromagnetic field such
that $M_{p,q} = \alpha^{p*}\alpha^q$ for some complex number $\alpha$ and all nonnegative integers $p$, $q$ is the 
coherent state $\ket \alpha$. Use this theorem to demonstrate that evolution according to the master equation (7.189)
preserves a coherent state; i.e., a coherent state at one time, a coherent state at all others.
\\ \\
\eenum
\benum
\item
We can express $M_{p,q}(t)$ as a first order ODE by taking its time derivative. This should lead to the form desired
\ba
	\dot M_{p,q}(t) &= \tr[\dot\rho(t)a^{p\dag}a^q]\\
	&= \gamma \tr[2a\rho a^\dag a^{p\dag}a^q - a^\dag a\rho a^{p\dag}a^q - \rho a^\dag a a^{p\dag}a^q]\\
	&= \gamma \tr[2a\rho a^\dag a^{p\dag}a^q - a\rho a^{p\dag}a^q a^\dag - \rho a^\dag a a^{p\dag}a^q]\\
	&= \gamma\tr[2a\rho a^\dag a^{p\dag} a^q -\rho a^\dag[a,a^{p\dag}]a^q-a\rho a^{p\dag}[a^q,a^\dag]
	- 2a\rho a^\dag a^{p\dag}a^q]\\
	&= \gamma\tr[-q\rho a^{p\dag}a^q-p\rho a^{p\dag}a^q]\\
	& = -\gamma(p+q)M_{p,q}(t)
\ea
Now we may solve the differential equation for $M_{p,q}(t)$ to find
\[
	M_{p,q}(t) = e^{-(p+q)\gamma t}M_{p,q}(0)
\]
\\
\item
According to Glauber's theorem, we have
\[
	\dot M_{p,q}(t) = 0\quad\text{for}\quad \rho = \ket\alpha\bra\alpha.
\]
\[
	\Rightarrow 0 = -\gamma(p+q)\alpha^{*p}\alpha^q.
\]
Now for integers $p,q >0$ and $\alpha \ne 0$, this implies $\gamma = 0$, and thus $\dot\rho = 0$ for a coherent state.
In the case that $\alpha = 0$, each term cancels in the master equation and still $\dot\rho = 0$. Coherent
state is persevered.
\eenum
% #2 ---------------------------------------------------------------------------------------------------------------------------------------------------------
\item[7.12]
In parametric downconversion photons from an incoming field are split in two. A stylistic but serviceable Hamiltonian
for downconversion to a photon mode with the annihilation operator $a$ reads
\[
	H = \frac{\h\xi}{2}(aa+a^\dag a^\dag)
\]
where the parameter $\xi$ proportional to the amplitude of the driving field is real.
\benum
\item
Write down the Heisenberg equations of motion for the operators $a(t)$ and $a^\dag(t)$.
\item
Solve these equations with the ansatz $a(t) = f(t)a(0)+g^*(t)a^\dag(0)$.
\item
Suppose the photon mode starts at $t=0$ in the vacuum state. Show that all time the photon
is in a squeezed vacuum state, an eigenstate of an operator of the form (7.120) with eigenvalue 0. \\ \\
\eenum
\benum
\item
\[
	\dot a = -\frac{i}{\h}[a,H] = -\frac{i\xi}{2}[a,a^\dag a^\dag] = -\frac{i\xi}{2}(2a^\dag) = -i\xi a^\dag
\]
\[
	\Rightarrow \dot a^\dag = i\xi a
\]
\item
The equation from (a) may be decoupled as
\[
	\difff{a}{*2t} = \xi^2a,\qquad \difff{a^\dag}{*2t} = \xi^2a^\dag.
\]
Using the ansatz we have
\[
	\difff{f(t)}{*2t}a(0)+ \difff{g^*(t)}{*2t}a^\dag(0) = \xi^2( f(t)a(0)+g^*(t)a^\dag(0)).
\]
Viewing differentiation as a linear operator, this implies
\[
	\Rightarrow \difff{f(t)}{*2t} = \xi^2 f(t)
\]
and likewise for $g^*(t)$. The general solution is
\[
	f(t) = A e^{\xi t} + Be^{-\xi t}
\]
and likewise 
\[
	g^*(t) = C e^{\xi t} + De^{-\xi t}.
\]
The coefficients from $f(t)$ may be related to those of $g(t)$ by 
the first derivative relations of the ansatz
\[
	\diff[f(t)]{t} = -i\xi g(t),\qquad \diff[g^*(t)]{t} = -i\xi f^*(t).
\]
\[
	\Rightarrow A = -iC^* \qquad B = -iD^*.
\]
\[
	\Rightarrow C = -iA^* \qquad D = -iB^*.
\]
Initial conditions 
\[
	f(0) = 1,\qquad g(0) = 0.
\]
Now our solution is
\[
	f(t) = A e^{\xi t} + Be^{-\xi t},\qquad g^*(t) = i(A^* e^{\xi t} + B^*e^{-\xi t}).
\]
We may also take the commutator
\ba
	[a,a^\dag] &= [f(t)a(0)+g^*(t)a^\dag(0), f(t)^*a^\dag(0)+g(t)a(0)]\\
	&= |f(t)|^2-|g(t)|^2\\
	& = 
\ea
\[
	a(t) = (A e^{\xi t} + Be^{-\xi t})a(0) + i(A^* e^{\xi t} + B^*e^{-\xi t})a^\dag(0)
\]
.....
The final solution is
\[
	a(t) = a(0)\cosh(\xi t) - ia^\dag(0)\sinh(\xi t).
\] \\
\item
The time evolution of the vacuum state is given as
\[
	\ket{\psi(t)} = e^{-iH t/\h}\ket 0
\]
and in the Schrodinger picture we have
\[	
	a = e^{-iH t/\h}a(t)e^{iH t/\h}.
\]
\emph{Unitary transformations preserve eigen-equations}\\ \\
Using this, we act upon the time dependent state with the annihilation operator. If it annihilates, the state remained
in the vacuum state.
\ba
	a\ket{\psi(t)} &= e^{-iH t/\h}a(t)e^{iH t/\h}e^{-iH t/\h}\ket 0\\
	& = e^{-iH t/\h}a(t)\ket 0\\
	& = 0
\ea
Thus the squeezed vacuum state 
\[
	\ket{\psi(t)} = e^{-iH t/\h}\ket 0 = \ket{0,\xi t}
\]
($A=0$ squeezed vacuum)\\ \\
is an eigenstate of our $a$ with zero eigenvalue and matches the form of 7.120.
\\
\eenum
% #3 ---------------------------------------------------------------------------------------------------------------------------------------------------------
\item[7.14]
\benum
\item
Take an operator $F$ that satisfied $[F,F^\dag] = -C$. Show that the expectation value of the operator $F^\dag F$
in any state satisfies $\braket{F^\dag F} \ge \braket C$. 
\\ \\
In a Heisenberg-like picture, a phase-insensitive linear amplifier of a boson mode (say, single mode of light) is 
represented by the equation $a_0 = Ga_i+F$, where $a_i$ and $a_0$ are the input and output mode operators, the 
amplitude gain satisfied $G>1$, and $F$ is an operator having to do with the internal degrees of freedom of the 
amplifier. This being a Heisenberg-like picture, the state of the system $\ket\psi$, for the input mode and the
internal degrees of freedom, is constant. Let us assume that the input and the amplifier are initially 
uncorrelated, so that $\ket \psi$ is a direct product of the state of the input and of the amplifier. Finally,
in order that the internal degrees of freedom do not generate a false signal at the output, let us assume that
$\braket F = 0$.\\
\item
Show that, given a coherent input state $\ket\alpha$, the expectation value of the output is $G\alpha$. This is, 
of course, as it should be for an amplifier with gain $G$.
\item
Now, the output operator $a_0$ must be a boson operator. Show that a non-trivial ``noise operator" $F$ 
\emph{must} be present, and satisfies $[F,F^\dag] = 1-G^2$.
\item
Show that, for any input, the boson number at the output satisfies $\braket{a_0^\dag a_0}\ge
G^2\braket{a^\dag_i a_i}+G^2-1$.
\\ \\
Under the present conditions (basically, no correlation between the input and the initial state of the 
amplifier), the output is not just an amplified copy of the input, but the process of amplification invariably adds
at least $G^2-1$ uncorrelated bosons to worsen the signal to noise ratio.\\ \\
\eenum
\benum
\item
Take the expectation value in an arbitrary normalized state $\ket \psi$
\[
		\braket{\psi|[F,F^\dag]|\psi} = -\braket{\psi|C|\psi} = -\braket{C}
\]
\be\label{1}
	\Rightarrow \braket{F^\dag F} = \braket{C}+\braket{FF^\dag}
\ee
By virtue of the hermiticity of $[F,F^\dag]$, $\braket{C} \in \mathbb R$. Likewise $FF^\dag$ is hermitian so 
$\braket{FF^\dag} \in \mathbb R$.  Moreover, $FF^\dag$ is a positive operator by 
\[
	(\psi,FF^\dag\psi) = (F^\dag\psi,F^\dag\psi) \equiv (\phi,\phi) \ge 0.
\]
Thus from \eqref 1 we have
\[
	\braket{F^\dag F} \ge \braket C.
\]
\\ \\
\item
The state $\ket\psi$ with coherent input $\ket \alpha$ may be written as $\ket\alpha\ket\phi$
where $\ket\phi$ represents the (normalized and uncorrelated) internal degrees of freedom of the state. Taking the
expectation value
\ba
	\bra\phi\bra\alpha a_0 \ket\alpha\ket\phi &= \bra\phi\bra\alpha(Ga_i+F) \ket\alpha\ket\phi \\
	&= G\braket{\alpha|a_i|\alpha}+\braket{\phi|F|\phi} = G\alpha.
\ea
\\ \\
\item 
As a boson operator, $a_0$ must satisfy commutation relation
\ba
	[a_0,a_0^\dag] &= 1\\
	& = [Ga_i+F,Ga_i^\dag+F^\dag]\\
	& = G^2+[F,F^\dag].
\ea
Without the noise operator $F$ present in the definition of $a_0$, the output boson commutation relation cannot hold for $G\ne1$. Also note that photon operators and internal operators $F$ commute.  Thus we have
\[
	[F,F^\dag] = 1-G^2.
\]\\ \\
\item
Given an arbitrary input, construct the general expectation value
\ba
	\braket{a_0^\dag a_0} &= \braket{(Ga_i^\dag+F^\dag)(Ga_i+F)} \\
	&= G^2\braket{a_i^\dag a_i}+G\braket{a_i^\dag F}+G\braket{a_i F^\dag} +\braket{F^\dag F}\\
	& = G^2\braket{a_i^\dag a_i}+G\braket{a_i^\dag F}+G\braket{a_i F^\dag} +\braket{FF^\dag - [F,F^\dag]}\\
	&=G^2\braket{a_i^\dag a_i}+G\braket{a_i^\dag F}+G\braket{a_i F^\dag} +\braket{FF^\dag}+G^2-1.
\ea
As the input is not correlated to the initial state of the amplifier, we may utilize the assumption $\braket F = 0$
to allow the cross terms to vanish
\[
	\braket{F} = 0 = \braket{F^\dag}^*  \Rightarrow \braket{F^\dag} = 0
\]
\[
	\Rightarrow \braket{a_iF^\dag} = 0,\qquad \braket{a_i^\dag F} = 0.
\]
Just to be clear, because of no correlation, the expectation is between a product of states, which could be defined as
arbitrary initial boson state $\ket \xi$ and internal state $\ket\phi$, so
\[
	\braket{a_iF^\dag} = \braket{\xi|a_i|\xi}\braket{\phi|F^\dag|\phi} = 0.
\]
Now eliminating the vanishing products in the expectation of the output number 
\ba
	\braket{a_0^\dag a_0} &= G^2\braket{a_i^\dag a_i}+\braket{FF^\dag}+G^2-1\\
	& \ge G^2\braket{a_i^\dag a_i}+G^2-1.
\ea
by virtue of $FF^\dag$ as a positive operator.
\eenum
\eenum
\end{document}