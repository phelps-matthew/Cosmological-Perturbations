\documentclass[10pt,letterpaper]{article}
\usepackage{macroshw}

\title{\begin{spacing}{1.2}Quantum Mechanics III\\HW 2\end{spacing}}
\author{Matthew Phelps}
\date{Due: Feb. 3 }

\begin{document}
\maketitle

\benum
% #1 -----------------------------------------------------------------------------------------------------------------------------------------------------------------
  	 \item[1.12]
	 
	Consider measurements of a spin 1/2 particle in the direction $\vecth n = \cos\alpha \vecth e_z + \sin\alpha 
	\vecth e_x$, so that we are measuring the operator $S(\alpha)$ with the corresponding
	unit-normalized eigenvectors $\chi_\pm(\alpha)$ corresponding to the eigenvalues $\pm \h/2$:
	\[
		S(\alpha) = \frac{\h}{2}\bbm \cos\alpha & \sin\alpha \\ \sin\alpha & -\cos\alpha\ebm;\ 
		\chi_+(\alpha) = \bbm \cos\frac{\alpha}{2} \\ \sin\frac{\alpha}{2} \ebm,\ 
		\chi_-(\alpha) = \bbm -\sin\frac{\alpha}{2} \\ \cos\frac{\alpha}{2} \ebm
	\]
	
	\benum
	% (a)
	\item
	Suppose we prepare the system initially in the state $\chi_+(0)$ (along the $z$ axis), then measure
	successively in the directions $\alpha,\ 2\alpha,\ ...,\ n\alpha$. Show that the probability that the result is 
	$+\h/2$ everytime is $\blr{\cos\frac{\alpha}{2}}^{2n}$. 
	% (b)
	\item
	Now make the angle $\alpha$ smaller and the number of measurements larger in such a way
	that $n\alpha = \pi$ remains constant. What do you achieve in the limit $n\to\infty$?
	\\ \\
	\eenum
	
	\benum
	\item 
	% (a)
	After each measurement, the state collapses into the eigenstate of $S(\alpha)$ with eigenvalue $\h/2$. In 
	order to find probabilities of measuring these eigenvalues, we expand the state in question in the
	eigenbasis of our operator $S(\alpha)$. An arbitrary state $\xi(\alpha')$ can be expanded in said basis as
	\[
		\xi(\alpha') = \chi^\dag_+(\alpha) \xi(\alpha') \chi_+(\alpha) 
		+  \chi^\dag_-(\alpha) \xi(\alpha') \chi_-(\alpha).
	\]
	with probability of measuring eigenvalue $\h/2$
	\[
		| \chi^\dag_+(\alpha) \xi(\alpha')|^2.
	\]
	Now, if we start off with the initial state $\chi_+(0)$ and perform a measurement at angle $\alpha$, the probabilty
	is
	\ba
		| \chi^\dag_+(\alpha) \chi(0)|^2 &= (\chi^\dag_+(\alpha) \chi_+(0))^2 \\
		&  = \plr{
		\bbm \cos\frac{\alpha}{2} & \sin\frac{\alpha}{2}\ebm
		\bbm 1 \\ 0 \ebm }^2 \\
		& = \plr{\cos\frac{\alpha}{2}}^2
	\ea
	The system is now in state $\chi_+(\alpha)$. If we perform a measurement at $2\alpha$ we have probability
	\ba
		 (\chi^\dag_+(2\alpha) \chi_+(\alpha))^2 & =
		  \plr{
		\bbm \cos\frac{2 \alpha}{2} & \sin\frac{2\alpha}{2}\ebm
		\bbm \cos\frac{\alpha}{2} \\ \sin\frac{\alpha}{2} \ebm }^2.
	\ea
	If we perform $n$ measurements in intervals of $\alpha$ the probability is 
	\ba
		 \blr{\chi^\dag_+(n\alpha) \chi_+((n-1)\alpha)}^2 &=
		  \plr{
		\bbm \cos\frac{n \alpha}{2} & \sin\frac{n\alpha}{2}\ebm
		\bbm \cos\frac{(n-1)\alpha}{2} \\ \sin\frac{(n-1)\alpha}{2} \ebm }^2 \\
		& = \blr{\cos\pfrac{n\alpha}{2}\cos\pfrac{(n-1)\alpha}{2}+\sin\pfrac{n\alpha}{2}\sin\pfrac{(n-1)\alpha}{2}}^2\\
		& = \clr{ \cos\plr{\frac{n\alpha}{2}-\frac{(n-1)\alpha}{2}}}^2\\
		& = \plr{\cos\frac{\alpha}{2}}^2
	\ea
	The final probability is the product of individual probabilities, thus
	\[
		P(n) = \plr{\cos\frac{\alpha}{2}}^{2n}.
	\]
	\item 
	% (b)
	For $n\alpha = \pi$ we may write our probability as
	\ba
		P &= \lim_{n\to\infty} \blr{\cos\pfrac{\pi}{2n}}^{2n} \\
		& =  \lim_{n\to\infty} \exp\blr{2n \ln\plr{\cos\pfrac{\pi}{2n}}} \\
		& = \exp\clr{ \lim_{n\to\infty} \pfrac{\ln \plr{\cos\pfrac{\pi}{2n}}}{1/2n}} \\
		& \Rightarrow \exp\clr{ \lim_{n\to\infty} \pfrac{ -\tan\pfrac{\pi}{2n} \frac{\pi}{2n^2}}{1/2n^2}} \\
		& =  \exp\clr{ \lim_{n\to\infty} \tan\pfrac{-\pi}{2n}\pi} \\
		& = 1
	\ea
	So it seems that given this type of successive measurement, our state will always be in a spin up eigenstate. 
	\\ \\
	\eenum 
	
	% 2 ----------------------------------------------------------------------------------------------------------------------------------------------------
	\item[2.1]
	
	Verify the completeness relation in Eq. (2.3) from the notions of orthonormality and completeness as discussed in 
	Ch. 1.\\ \\
	
	Given an arbitrary vector $\ket v$ in $\mathscr H$ space, if we have a complete orthonormal basis  $\{ \ket n\}$, 
	then $\ket v$ may be expanded as a linear combination of  $\{ \ket n\}$ with coefficients $c_n = (n,v)$
	\[
		\ket v = \sum_n c_n \ket n = \sum_n \braket{n|v}\ket n =\plr{ \sum_n {\ket n\bra n}}\ket v = \mathds 1 \ket v.
	\]
	 Or perhaps if we apply the completeness relation the vector $\ket v$ (which we expand in the same basis 
	 labeled by $\{\ket m\}$)
	 
	\[
		\plr{\sum_n \ket n\bra n} \ket v = \sum_{n,m} \ket n \braket{n|m}\braket{m|v} = \sum_m \ket m\braket{m|v} 
		=\ket v.
	\]
	\\
	Thus we verify that $\sum_n \ket n\bra n = \mathds 1$. 
	\\ \\
% 3 ------------------------------------------------------------------------------------------------------------------------------------------------------
	\item[2.3]
	Any operator with the property that $P^2 = P$ is called a projection operator or projector. This property, in fact, 
	does not guarantee that $P$ is hermitian, but in quantum mechanics only hermitian projection operators
	really figure. Thus, assume that $P^2 = P$ and $P = P^\dag$. Show that such an operator is always an
	orthogonal projection to some subspace of the Hilbert space. \\ \\
	As always, take is as given that any orthonormal set in the quantum mechanical Hilbert space can be completed
	into an orthonormal basis. \\ \\ \\
	
	Since $P=P^\dag$, its eigenvectors form an orthogonal set, which may be completed to an orthonormal
	basis $\{\ket n\}$ for a particular subspace $\mathscr S$. We apply the completeness relation
	\[
		\sum_{n \in \mathscr S} \ket n \bra n = \mathds 1
	\]
	to the operator $P$
	\[
		\sum_{n,m \in \mathscr S} \ket n \bra m \braket{n|P|m} = \sum_{n,m \in \mathscr S} \lambda_n
		\ket n \bra m \braket{n|m} = \sum_{n \in \mathscr S} \lambda_n
		\ket n \bra n
	\]
	where $\lambda_n$ denote the eigenvalues $P\ket n = \lambda_n \ket n$. Now we take $P^2$
	\ba
		P^2 &= \plr{\sum_{n \in \mathscr S} \lambda_n \ket n \bra n}
		\plr{\sum_{m \in \mathscr S} \lambda_m \ket m \bra m} \\
		& = \sum_{n,m\in\mathscr S} \lambda_n\lambda_m \delta_{nm}\ket n\bra m \\
		& = \sum_{n\in \mathscr S} \lambda_n^2 \ket n\bra n \overset{!}{=} 
		\sum_{n \in \mathscr S} \lambda_n \ket n \bra n = P
	\ea
	We have the equality $P^2 = P$ only if the eigenvalues $\lambda^2_n = \lambda_n$. Hence $\lambda_n = 1$
	(or trivially $0$). Therefore
	\[
		P = \sum_{n\in \mathscr S} \ket n\bra n
	\]
	which is the definition of the orthogonal projection on a subspace in $\mathscr H$. 
	 \\
	
% 4 ------------------------------------------------------------------------------------------------------------------------------------------------------
	\item[2.4]
	An alternative approach to normal operators: Take it as given that two hermitian operators may be diagonalized
	simultaneously if and only if they commute.
	\benum
	% (a)
	\item
	As already noted, every operator $A$ may be decomposed trivially in the form $A = A_1+iA_2$,
	where $A_1$ and $A_2$ are hermitian. Suppose we have a normal operator $N$ with the 
	corresponding components $N_1$ and $N_2$. Verify the following items: (i) $[N_1,N_2] = 0$. 
	(ii) $N$ may be diagonalized.
	% (b)
	\item
	Conversely, suppose that an operator $N$ can be diagonalilzed, with the eigenvalues $c_n$ (not necessarily
	real) and the orthonormal eigenvectors $u_n$. Verify the following items: (i) $(u_n,N^\dag u_m) = 
	c_n^*\delta_{nm}$. (ii) $N^\dag u_m = c_m^* u_m$. Therefore, $N^\dag$ can also be diagonalized, eigenvalues
	and eigenvectors $c_n^*$ and $u_n$. (iii) $N$ is normal.
	\\ \\
	We have again, the result that normal, and only normal, operators can be diagonalized.
	\eenum 
	\benum
	% (a)
	\item
	\[
		N = N_1+iN_2;\quad N^\dag = N_1-iN_2
	\]
	\[
		[N,N^\dag] = [N_1+iN_2,N_1-iN_2] = 2i[N_2,N_1] = 0
	\]
	
	\[
		\Rightarrow [N_1,N_2] = 0.
	\]
	Since $[N_1,N_2] = 0$ both $N_1$ and $N_2$ (hermitian operators) can be simultaneously diagonalized. As
	$N = N_1 +iN_2$ it is the sum of two diagonalized operators and thus is itself diagonal (but not strictly 
	hermitian). Therefore normal operators are diagonalizable.
	\\ 
	% (b)
	\item
	\begin{enumerate}[label=\roman*).]
	\item
	\[
		(u_n,N^\dag u_m) = (Nu_n,u_m) = (c_nu_n,u_m) = c_n^*\delta_{nm}
	\]
	\item
	\[
		N^\dag = \sum_{n,m} \ket n\braket{n|N^\dag|m}\bra m = \sum_n c_n^*\ket n\bra n
	\]
	\[
		N^\dag \ket n = \plr{\sum_n c_n^*\ket n\bra n} \ket n = c_n^* \ket n.
	\]
	\item
	\[
		[N,N^\dag]\ket n = (|c_n|^2-|c_n|^2)\ket{n} = 0
	\]
	\[
		\Rightarrow [N,N^\dag] = 0
	\]
	\eenum 
	\eenum
	\eenum

\end{document}