\documentclass[10pt,letterpaper]{article}
\usepackage{macroshw}

\title{\begin{spacing}{1.2}Quantum Mechanics III\\HW 12\end{spacing}}
\author{Matthew Phelps}
\date{Due: April 20}

\begin{document}
\maketitle

\benum
% #1 --------------------------------------------------------------------------------------------------------------------------------------------------------
\item[8.2]
\benum
\item
Show that the spectrum of the solutions $\omega$ to the eigenvalue equations (8.49) and (8.50) is unchanged
if there is a global change in the phase of the zeroth order solution $\phi$, $\phi\to e^{i\psi}\phi$ for
a constant (in space and time) $\psi$. \\
\item
The physics of a BEC is said to be invariant under time reversal if the field $\phi$ may be chosen to be real. Show
that in such a case the eigenvalues come in pairs: if $\omega$ is an eigenvalue, then so is $-\omega$.\\ \\
\eenum
\benum
\item %(a)
As eigenvalue equations, we may put (8.49) and (8.50) into a more illustrative form according the following definitions:
\[
 	\mathcal L = \plr{ -\frac{\h^2}{2m}\del^2 +U+2g|\phi|^2},\qquad
	A = g\phi^2.
\]
Yes, $\mathcal L$ is dependent on $A$, but for the argument here this is not important. With these definitions,
we may now construct our eigenvalue matrix equation
\[
	\bpm \mathcal L - (\mu+\omega)& A \\ -A^*&-(\mathcal L-(\mu-\omega)) \epm\bpm u\\ v\epm 
\]
The transformation $\phi \to e^{i\psi}\phi$ is effectively carried out by $A\to e^{2i\psi}A$, thus
\[
	\bpm \mathcal L - (\mu+\omega)& A \\ -A^*&-(\mathcal L-(\mu-\omega)) \epm \to
	\bpm \mathcal L - (\mu+\omega)& Ae^{2i\psi} \\ -A^*e^{-2i\psi}&-(\mathcal L-(\mu-\omega)) \epm.
\]
The spectrum of eigenvalues are defined by the characteristic equation resulting from the determinant of a matrix.
We see that the determinant here is unaffected by the transformation, making $\{\omega\}$ invariant.
\\ \\ Now that I'm skeptical on the validity of taking the determinant of differential operators, perhaps we may answer this
question in a different manner. Note that if we define $v' =  ve^{2i\psi}$, we may write both our eigenvalue equations 
(that is, after a transformation $\phi \to \phi e^{i\psi}$) as
\[
	\plr{-\frac{\h^2}{2m}\del^2+U+2g|\phi|^2}u+g\phi^2 v' = (\mu +\omega)u
\]
\[
	\plr{-\frac{\h^2}{2m}\del^2+U+2g|\phi|^2}v'+g\phi^{*2} u = (\mu -\omega)v'.
\]
The second eq. was accomplished by multiplying by an overall phase factor. Now we see that the eigenvalue
equations are of the exact same form as before, now defined in terms of variables $u,v'$. Therefore, the spectrum of eigenvalues $\{\omega\}$ has not changed. \\ \\
\item 
For $\phi$ real, the variable change $\omega \to -\omega$ may be effectively carried out by taking 
$u\to u'= v$ and $v\to v'= u$ in eqs. (8.49) and (8.50). Now we have the same two eigenvalue equations in
terms of variables $v',u'$, and therefore $-\omega$ also serves as valid eigenvalue. \\ \\
\eenum
% #2 ---------------------------------------------------------------------------------------------------------------------------------------------------------
\item[8.5]
Let us model a torus containing a BEC as one-dimensional motion of the atoms with periodic boundary conditions
over the circumference $L$. In order to keep the dimensions of the various quantities under control, let us also
quietly assume that the gas is confined in the transverse directions to a small area $A$ in such a way that the density
is only a function of the longitudinal coordinate $x$.\\ \\
The torus is rotated about its axis so that the linear velocity of the rim is $v$, and we study the gas in a frame rotating
with the torus. A simple piece of classical mechanics shows that in the rotating frame the Hamiltonian gets 
amended with the term $H_R = -v\hat p$.
\benum
\item
Find the constant-density solutions of the time independent GPE and the corresponding wave numbers in the
rotating frame.
\item
Suppose the potential along the torus is not constant, but has a little dimple that rotates with the torus. Intuition
then tells us that the thermodynamical equilibrium state must be one that is stationary in the rotating frame.
Generalizing, the thermodynamical equilibrium state should be one with the smallest chemical potential in the rotating frame. Show that the thermodynamical equilibrium is the state with the wave number
\[
	k = \frac{2\pi}{L}\pfrac{v}{v_0},
\]
where $v_0 =  2\pi\h/(mL)$ is kind of a quantum of rotation of velocities, and $[x]$ stands for the integer
closest to $x$.
\item
Now, the instantaneous (at a fixed $t$) spatial variation of the wave function is the same in the rotating and 
in the stationary frame. Incredible as it may sound, this correctly suggests that in the laboratory frame the flow
of velocity of the gas $V$ is given in terms of the wave number in the rotating frame $k$ as 
$V = \h k/m$. Thus, sketch the variation of the equilibrium flow velocity of the gas in the lab frame
as a function of the velocity at which the torus is rotated. \\ \\
\eenum
\benum
\item %(a)
The density of the condensate is 
\[
	n(x,t) = \psi(x,t)\psi^*(x,t).
\]
With
\[
	\psi(x,t) = \phi(x)e^{-i\mu t/ \h}
\]
the GPE equation becomes (in the rotating frame)
\[
	\plr{-\frac{\h^2}{2m}\difff{}{*2x} + U(x)+i\h v\diff{x}}\phi(x) +\frac{4\pi\h^2 a}{m}\phi^*(x)\phi(x)\phi(x) = \mu \phi(x).
\]
Assuming no external potential and accounting for proper normalization ($\int dx |\phi(x)|^2 = N$), this equation has the solution
\[
	\phi(x) = \sqrt ne^{ikx}
\]
under the energy constraint
\[
	\ep_k - \h kv +\frac{4\pi\h^2 an}{m} = \mu.
\]
Additionally, our periodic boundary conditions restrict our values of $k$:
\[
	\phi(x) = \phi(x+L) \Rightarrow k = \frac{2\pi}{L}j
\]
where $j = 0,1,2,3..$.\\
\item % (b)
Following the logic as explained in the question, we seek to find wavenumbers that minimize the chemical potential, i.e.
\[
	\pdiff[\mu]{k} = 0.
\]
Minimizing our energy equation with respect to $k$
\ba
	\pdiff[\mu]{k}&= \pdiff{k}\plr{\frac{\h^2k^2}{2m} - \h kv+\frac{4\pi\h^2 an}{m}}\\
	&=\frac{\h^2 k}{m}-\h v\\
	\Rightarrow k &= \frac{mv}{\h}
\ea
Meanwhile this must conform to our periodic boundary conditions
\[
	\frac{2\pi}{L} j = \frac{mv}{\h} \Rightarrow j = \blr{\frac{mvL}{2\pi \h}} = \blr{\frac{v}{v_0}}.
\]
Thus
\[
	k = \frac{2\pi}{L}\blr{\frac{v}{v_0}}.
\]
where $[..]$ denotes nearest integer. \\ \\
\item % (c)
The equilibrium velocity flow of the gas in the lab frame $V$ as a function of the rotation velocity $v$ is
\[
	V = \frac{\h k}{m} = \frac{ 2\pi \h}{L m}\blr{\frac{v}{v_0}}.
\]
The behavior is interesting because we must take the nearest integer, so it appears the flow velocity is quantized.
\figg[width=150mm]{12_1.png}
\eenum
\eenum
\end{document}