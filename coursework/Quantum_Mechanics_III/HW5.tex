\documentclass[10pt,letterpaper]{article}
\usepackage{macroshw}

\title{\begin{spacing}{1.2}Quantum Mechanics III\\HW 5\end{spacing}}
\author{Matthew Phelps}
\date{Due: Feb. 22}

\begin{document}
\maketitle

\benum
% #1 ---------------------------------------------------------------------------------------------------------------------------------------------------------
  	 \item[4.1]
	Starting from the Liouville-von Neumann equation, show that the property ``$\rho$ is a pure state" or ``$\rho$ is
	a mixed state" is preserved in the time evolution of a closed system. In particular, if the system is in a mixed state at
	some point in time, it will always be in a mixed state and can never be described by a state vector.  \\ \\
	Form the Liouville-von Neumann equation for $\rho^2$
	\ba
		i\h \diff{t}\rho^2 = i\h(\dot\rho\rho +\rho\dot\rho) = \{[H,\rho],\rho\}.
	\ea
	Now take the trace
	\ba
		i\h \tr\plr{ \diff{t}\rho^2} &= \tr\plr{ [H,\rho]\rho } + \tr\plr{\rho [H,\rho] }\\
		& = 2\tr([H,\rho]\rho)\\
		& = 2\tr(H\rho^2)-2\tr(\rho H\rho)\\
		& = 2\tr(H\rho^2)-2\tr(H\rho^2)\\
		& = 0.
	\ea
	From $\tr(\diff{t}(\rho^2)) = 0$, we have for an infinitesimal variation in time $\delta t$
	\[
		\tr\pfrac{ \rho^2(t+\delta t)-\rho^2(t)}{\delta t} = 0
	\]
	\[
		\Rightarrow \tr(\rho^2(t+\delta t)) = \tr(\rho^2(t))\quad \forall t.
	\]
	Hence, $\tr(\rho^2)$ is invariant under infinitesimal variations in time and so, for any reasonable smooth function
	$\rho^2(t)$,
	\[
		\tr(\rho^2(t_0)) = \tr(\rho^2(t))\quad\forall t.
	\]
	Thus a pure state with $\tr(\rho^2) = 1$ will remain pure and a mixed state with $\tr(\rho^2)\ne 1$ will remain mixed.
	\\ \\
% 2 ----------------------------------------------------------------------------------------------------------------------------------------------------
	\item[4.3]
	
	Imagine two spins prepared in a singlet state
	\[
		\ket\psi = \frac{1}{\sqrt 2}( \ket\uparrow_1 \ket\downarrow_2 - \ket\downarrow_1\ket\uparrow_2).
	\]
	The spins are then transported far apart, but so that the entangled state is preserved, and the value of the first spin is
	measured ($\ket\downarrow$ or $\ket\uparrow$). Assume that the measurement result is not known at the site of
	the second spin, for instance because at the time the site is outside the light cone of the the measurement. What
	is the effect of the measurement on the state of the other spin?
	\\ \\
	Denote the projective measurements of up or down acting on spin system 1 as
	\[
		P_{1\uparrow} = \sum_{n=\uparrow,\downarrow} \ket\uparrow_1\ket n_2 \bra n_2\bra\uparrow_1
	\]
	\[
		P_{1\downarrow} = \sum_{n=\uparrow,\downarrow} \ket\downarrow_1\ket n_2 \bra n_2\bra\downarrow_1.
	\]
	The probabilities of such measurements giving us up or down spin in system 1 are
	\ba
		P(\uparrow_1) &= \tr(\rho P_{1\uparrow}) \overset{\text{pure state}}= \braket{\psi|P_{1\uparrow}|\psi} = \frac{1}{\sqrt 
		2}\bra\psi(\ket\uparrow_1\ket\downarrow_2) = \frac{1}{2} \\
	\ea
	and similarly,
	\[
		P(\downarrow_1) = \tr(\rho P_{1\downarrow}) = \frac{1}{2}.
	\]
	After a measurement on system 1 is made, the state transforms to 
	\[
		\rho \to \rho' = \sum_{k = \uparrow_1,\downarrow_1}P(k)\rho_k
	\]
	where
	\[
		\rho_k = \frac{\tr_1(P_k\rho P_k)}{P(k)}.
	\]
	Earlier we showed that
	\[
		P_{1\uparrow}\ket{\psi} = \frac{1}{\sqrt 2} \ket\uparrow_1\ket\downarrow_2
	\]
	and since $P_{1\uparrow}^\dag = P_{1\uparrow}$ 
	\[
		P_{1\uparrow}\ket\psi\bra\psi P_{1\uparrow} = \frac{1}{2}\ket\uparrow_1\ket\downarrow_2\bra\downarrow_2
		\bra\uparrow_1.
	\]
	Similarly,
	\[
		P_{1\downarrow}\ket\psi\bra\psi P_{1\downarrow} = \frac{1}{2}\ket\downarrow_1\ket\uparrow_2\bra\uparrow_2
		\bra\downarrow_1.
	\]
	Thus
	\[
		\rho_{\uparrow_1}=\frac{\tr_1(P_{1\uparrow}\rho P_{1\uparrow})}{P(\uparrow_1)} = 2\pfrac{1}{2}
		\sum_{n = \uparrow,\downarrow} \bra n_1
		(\ket\downarrow_1\ket\uparrow_2\bra\uparrow_2 \bra\downarrow_1)\ket n_1 = \ket\uparrow_2
		\bra\uparrow_2
	\]
	similarly
	\[
		\rho_{\downarrow_1} = \ket\downarrow_2\bra\downarrow_2.
	\]
	Therefore the new state after measurement is
	\[
		\rho' = \frac{1}{2}\plr{ \ket\uparrow_2\bra\uparrow_2+\ket\downarrow_2\bra\downarrow_2}
	\]
	Without knowing the outcome of measurement on system 1, the state of the system 2 transforms into a mixed
	ensemble of equal probabilities for being in state spin up/down. \\ \\
% 3 ------------------------------------------------------------------------------------------------------------------------------------------------------
	\item[4.4]		
	\benum
	% (a)
	\item
	Suppose the observables $A$ and $B$ commute. Denote by $P_a$ and $P_b$ the orthogonal projections
	to the subspaces associated with the eigenvalues $a$ of $A$ and $b$ of $B$. Show that $[P_a,P_b]=0$.
	\\ \\
	Define the operators $A$ and $B$ in terms of their respective projectors and eigenvalues
	\[
		A = \sum_a aP_a;\qquad B = \sum_b bP_b
	\]
	where here we are summing over all the possible eigenvalues $a$ and $b$ (respectively). 
	For
	\[
		[A,B]=0
	\]
	we have
	\[
		\sum_{a,b}ab[P_a,P_b] = 0
	\]
	To retain the equality for general operators $A$ and $B$ the commutator must vanish identically
	\[
		[P_a,P_b] = 0.
	\] \\ \\
	% (b)
	\item
	Think of projective measurements of the observables $A$ and $B$ in succession. Show that the probability to
	obtain the (eigen)value $a$ for $A$ and $b$ for $B$ is the same whether you measure $A$ first and then $B$,
	or $B$ first and then $A$. 
	\\ \\
	This motivates the saying that commuting observables may be measured simultaneously. Note the relation between
	joint and conditional probabilities of events $E_1$ and $E_2$, $P(E_1,E_2) = P(E_2|E_1)P(E_1)$.
	\\ \\
	Let's measure $A$ first. The probability of measuring eigenvalue a in an arbitrary state $\rho$ is given by
	\[
		P(a) = \tr(\rho P_a).
	\]
	Upon measurement, the state transforms into
	\[
		\rho \to \tilde\rho = \sum_a P(a)\rho_a
	\]
	where
	\[
		\rho_a = \frac{P_a \rho P_a}{P(a)}.
	\]
	This simplifies to
	\[
		\tilde \rho= \sum_a P_a\rho P_a.
	\]
	However, this is only true if we discard the result of measurement $A$. When an actual measurement is made
	that gives us the eigenvalue $a$, the state must transform specifically to 
	\[
		\rho \to\rho' = P_a\rho P_a
	\]
	and the probability of obtaining such a result is $P(a)= \tr(\rho P_a)$. Now 
	we make a measurement of $B$. The probability of measuring eigenvalue $b$ is then
	\[
		P(b) = \tr(\rho' P_b) = \tr(P_a\rho P_aP_b).
	\]
	Since the measurement of $B$ cannot affect the result of prior measurement $A$ (the state has already collapsed), the 		probabilities
	are independent and the joint probability is given by the product 
	\[
		P_{(a,b)} = P(a)P(b) = \tr(\rho P_a)\tr\plr{ P_a\rho P_aP_b}.
	\]
	Similarly, the probability of measuring $b$ then $a$ is
	\[
		P_{(b,a)} = P(b)P(a) =  \tr(\rho P_b)\tr\plr{P_b\rho P_bP_a}
	\]
	Using the cyclic property of the trace, the projector property $P_n^2 = P_n$, and $[P_a,P_b] = 0$ we may rearrange 		each result
	\[
		P_{(a,b)} = \tr(\rho P_a)\tr(\rho P_b) = \tr(\rho P_b)\tr(\rho P_a) = P_{(b,a)}
	\]
	Thus the probabilities are the same in either order and so a pair of commuting observables cannot influence one another.
	\\ \\
	\eenum
	
	
		
% 4 ------------------------------------------------------------------------------------------------------------------------------------------------------
	\item[4.8]
	\benum
	% (a)
	\item
	Suppose $E$ is a positive operator with the property that $\braket{\psi|E|\psi} \le 1$ for every normalized state
	$\ket\psi$. Show that under these conditions $\braket{\psi|E|\psi} = 1$ actually implies $E\ket\psi = \ket\psi$.
	\\ \\
	Expand $\ket\psi$ in an orthonormal basis $\{ \ket n\}$ and take the inner product $\braket{\psi|E|\psi}$
	\[
		\braket{\psi|E|\psi} = \bra\psi \plr{E\sum_n \braket{n|\psi}\ket n} =  \sum_n \braket{\psi|E|n}\braket{n|\psi}
		=1.
	\]
	The sum of the expansion coefficients of $\ket\psi$ in the $\{\ket n\}$ basis must also add to unity, so
	\[
		\sum_n |\braket{\psi|n}|^2 = 1 = \sum_n \braket{\psi|E|n}\braket{n|\psi}.
	\]
	Thus we have the relation
	\[
		\braket{\psi|E|n} = \braket{\psi|n}.
	\]
	For this to hold for any choice of orthonormal basis $\{\ket n\}$, $E$ must be the identity operator
	\[
		E\ket\psi = \ket\psi.
	\]
	\\ \\
	% (b)
	\item
	Take two (of course, normalized) states $\ket\psi_1$ and $\ket\psi_2$. Assume that it is possible to construct an
	experiment that can distinguish between these states without fail. A moment of thought will show that then a POVM
	$\{ E_1,E_2\}$ must exist with the properties that $\braket{\psi_1|E_1|\psi_1} = 1$, $\braket{\psi_2|E_1|\psi_2} = 0$,
	$\braket{\psi_1|E_1|\psi_1} = 0$, and $\braket{\psi_2|E_2|\psi_2} = 1$. Starting from this observation, show that
	$\ket{\psi_1}$ and $\ket{\psi_2}$ must be orthogonal. Hint: Considering an arbitrary superposition
	$\ket V = \lambda_1\ket{\psi_1} +\lambda_2\ket{\psi_2}$ may be helpful. 
	\\ \\
	For $\braket{\psi_1|E_1|\psi_1} = 1$, we have from part (a) that $E_1\ket{\psi_1} = \ket{\psi_1}$ and similarly for 
	$E_2$. Also, as a POVM (with no missing measurements) we have $E_1+E_2 = 1$. Take the state 
	$\ket V$ as defined above to be normalized: $|\lambda_1|^2+|\lambda_2|^2 = 1$. Now form
	\ba
		\braket{\psi_1|V} &= \braket{\psi_1|(E_1+E_2)|V} \\
		\lambda_1 +\lambda_2 \braket{\psi_1|\psi_2} & = 
		\lambda_1\braket{\psi_1|E|\psi_1} +\lambda_2\braket{\psi_1|E_1|\psi_2} +
		\lambda_1\braket{\psi_1|E_2|\psi_1} +\lambda_2\braket{\psi_1|E_2|\psi_2}
	\ea
	Using the hermiticity (POVM) of $E_i$ and $E_i\ket{\psi_i} = \ket{\psi_i}$ we may reduce the above to
	\ba
		\lambda_1 + \lambda_2 \braket{\psi_1|\psi_2} &= \lambda_1\braket{\psi_1|\psi_2} + 2\lambda_2
		\braket{\psi_1|\psi_2}\\
		\braket{\psi_1|\psi_2} &= 2\braket{\psi_1|\psi_2}
	\ea
	This above line may only be true if 
	\[
		\braket{\psi_1|\psi_2} = 0
	\]
	i.e. $\ket\psi_1$ and $\ket\psi_2$ are orthogonal. 
	\eenum
\eenum

\end{document}