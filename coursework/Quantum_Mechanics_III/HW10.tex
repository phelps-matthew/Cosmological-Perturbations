\documentclass[10pt,letterpaper]{article}
\usepackage{macroshw}

\title{Quantum Mechanics III\\HW 10}
\author{Matthew Phelps}
\date{Due: April 4}

\begin{document}
\maketitle

\benum
% #1 ---------------------------------------------------------------------------------------------------------------------------------------------------------
\item[7.4]
Consider the two-level atom embedded in the big environment of the quantized electromagnetic field along the lines 
of section 7.5 using the interaction Hamiltonian (7.59). Show that the relaxation terms of the form (5.6) ensue. There
will also be a level shift called \emph{Lamb Shift}, but at the present stage of the modeling it is infinite and you 
should ignore it. In a full analysis the Lamb shift is ``renormalized" so that it becomes finite.\\ \\
The atom-field interaction Hamiltonian is
\[
	H_{AR} = -\sum_{\vect k\lambda}\sqrt\frac{\h\omega_k}{2\epo V}\vect D\cdot\vecth e_{\vect\lambda}(\vect k)
	(\ket 2\bra 1 a_{\vect k\lambda}+a_{\vect k\lambda}^\dag \ket 1\bra 2).
\]
We will work with a combined index $n$ and shorten the notation to
\[
	H_{AR} = \sum_{n}G_{n}
	(\ket 2\bra 1 a_{n}+a_{n}^\dag \ket 1\bra 2)
\]
where
\[
	G_n(\vect D,\lambda,\vect k) = -\sqrt\frac{\h\omega_k}{2\epo V}\vect D\cdot\vecth e_{\vect\lambda}(\vect k).
\]
Converting to the interaction picture with respect to $H_0$
\[
	H_0 = H_A + H_E = \h\omega_1 \ket 1\bra 1+\h\omega_2\ket 2\bra 2
	+\sum_{n}\h\omega_n a^\dag_n a_n
\]
we have for the interaction perturbation
\[
	H^I_{AR} = e^{iH_0t/\h}H_{AR}e^{-iH_0t/\h}.
\]
To compute this, first take the result from the atomic hamiltonian
\[
	e^{iH_At/\h}H_{AR}e^{-iH_At/\h} = \sum_n G_n (
	e^{i(\omega_2-\omega_1)t}\ket 2\bra 1 a_n+e^{-i(\omega_2-\omega_1)t}\ket 1\bra 2a_n^\dag).
\]
Now we need to take this result between the exponentials of the environment
\ba
	e^{iH_E}\blr{ \sum_n G_n (
	e^{i(\omega_2-\omega_1)t}\ket 2\bra 1 a_n+e^{-i(\omega_2-\omega_1)t}\ket 1\bra 2a_n^\dag)}e^{-iH_E}.
\ea 
Some steps necessary to compute this:
\[
	[a_i, n_{j}] = \delta_{ij}a_k,\qquad [a_i^\dag,n_{j}] = -\delta_{ij}a^\dag_i
\]
\qquad\qquad BCH formula
\[
	e^{X}Ye^{-X} = Y+[X,Y]+\frac{1}{2!}[X,[X,Y]]+\frac{1}{3!}[X[X[X,Y]]]+...
\]
\[
	e^{iH_Et/\h}a_ke^{-iH_Et/\h} = e^{-i\omega_k t}a_k,
	\qquad e^{iH_Et/\h}a^\dag_ke^{-iH_Et/\h} = e^{i\omega_k t}a_k^\dag.
\]
All together then we have the interaction potential 
\[
	H^I_{AR} =  \sum_n G_n (
	e^{i(\omega_{21}-\omega_n)t}\ket 2\bra 1 a_n+e^{-i(\omega_{21}-\omega_n)t}\ket 1\bra 2a_n^\dag).
\]
The time dependence of the density operator $\rho(t)$ corresponding to the combined system may be written out as an perturbative expansion of the Liouville-von Neumann equation
\be\label{1}
	\rho(t+\delta t) = \rho(t)-\frac{i}{\h}\int_t^{t+\delta t}dt_1\ [V(t_1),\rho(t)] +
	\pfrac{-i}{\h}^2\int_t^{t+\delta t}dt_1\ \int_{t}^{t_1}dt_2\ [V(t_1),[V(t_2),\rho(t)]]+...
\ee
We assume the density operator starts in the vacuum state of the environment
\[
	\rho(t) = \rho_A(t)\otimes \ket 0\bra 0_E
\]
and we choose to trace away the environment to get the time evolution of the system. With this we compute the
trace of each term in \eqref 1 (denoting $\rho_A(t)\equiv \rho(t)$). 
The first order term vanishes
\[
	\tr_E([V(t_1),\rho(t)]) = 0
\]
The second order term can be split into four parts. One part is
\ba
	\tr_E(V(t_1)V(t_2)\rho(t)) =\sum_{nn'}G_nG_{n'}\bigg( \bra0|\blr{
	e^{i(\omega_{21}-\omega_n)t_1}\ket 2\bra 1 a_n+e^{-i(\omega_{21}-\omega_n)t_1}\ket 1\bra 2a_n^\dag}
	\\ \quad\times\blr{
	e^{i(\omega_{21}-\omega_{n'})t_2}\ket 2\bra 1 a_{n}'+e^{-i(\omega_{21}-\omega_{n'})t_2}\ket 1\bra 2a^		\dag_{n'}}\ket 0\bigg) \rho(t)
\ea

The annihilation operators annihilate the vacuum and only $n=n'$ terms survive. We must also note the 
orthogonality of the atomic states. We are left with a trace of
\[
	\tr_E(V(t_1)V(t_2)\rho(t)) = \sum_{n}G_n^2e^{i(\omega_{21}-\omega_n)(t_1-t_2)}\ket 2\bra 2\rho(t)
\]
The remaining three terms of the commutator from the perturbation are evaluated similarly. At this point, we note that our results take a similar form as that in the script. The
result of the trace of the integration of each term is
\[
	\pfrac{-i}{\h}^2\tr_E \blr{ \int_t^{t+\delta t}dt_1\ \int_{t}^{t_1}dt_2\ V(t_1)V(t_2)\rho(t)}
	\approx \delta t\blr{-(\gamma-i\delta)\ket 2\bra 2\rho(t)}
\]
\[
	\pfrac{-i}{\h}^2\tr_E \blr{ \int_t^{t+\delta t}dt_1\ \int_{t}^{t_1}dt_2\ \rho(t)V(t_2)V(t_1)}
	\approx \delta t\blr{-(\gamma+i\delta)\rho(t)\ket 2\bra 2}
\]
\[
	\pfrac{-i}{\h}^2\tr_E \blr{ \int_t^{t+\delta t}dt_1\ \int_{t}^{t_1}dt_2\ V(t_1)\rho(t)V(t_2)}
	\approx \delta t\blr{(\gamma-i\delta)\ket 1\braket{2|\rho(t)|2}\bra 1}
\]
\[
	\pfrac{-i}{\h}^2\tr_E \blr{ \int_t^{t+\delta t}dt_1\ \int_{t}^{t_1}dt_2\ V(t_2)\rho(t)V(t_1)}
	\approx \delta t\blr{(\gamma+i\delta)\ket 1\braket{2|\rho(t)|2}\bra 1}
\]
where $\gamma$ and $\delta$ follow the definitions in the text. \\ \\
Summing these results
\[
	\rho(t+\delta t) = \rho(t) +\delta t\bigg(-(\gamma-i\delta)\ket 2\bra 2\rho(t)-
	(\gamma+i\delta)\rho(t)\ket 2\bra 2+2\gamma\ket 1\braket{2|\rho(t)|2}\bra 1 \bigg)
\]
which may be brought to the form
\[
	\dot\rho = i\delta[\ket 2\bra 2,\rho]+\gamma\big(2\ket 1\braket{2|\rho|2}\bra 1-\ket 2\bra 2\rho-\rho
	\ket 2\bra 2\big).
\]
We have the Liouville-von Neumann term for a modified Hamiltonian and the relaxation terms in the Lindblad form.
Taking expectation values between the two atomic states, we may easily confirm eq. (5.9) for $\Gamma = 2\gamma$.
\[
	\dot\rho_{22} = -\Gamma\rho_{22},\ \dot\rho_{11} = \Gamma\rho_{22},\ \dot\rho_{21} = -\frac{1}{2}
	\Gamma\rho_{21},\ \dot\rho_{12} = -\frac{1}{2}\Gamma\rho_{12}.
\]
\emph{For some reason I find that only the state $\ket 2\bra 1\rho(t)\ket 1\bra 2a_na^\dag_n$ is nonvanishing
in the interaction term. But this doesn't follow the proper Lindblad form of $\ket 1\bra 2\rho(t)\ket 2\bra 1$ for a two level system.. }
\\ \\
% #2 ---------------------------------------------------------------------------------------------------------------------------------------------------------
\item[7.6]
By attempting to construct an eigenstate of the boson creation operator $a^\dag$ in the Fock state basis, show
that it does not have any eigenstates at all. \\ \\
Eigenvalue equation for $a^\dag$:
\[
	a^\dag \ket \psi = \lambda \ket \psi.
\]
Expand $\ket \psi$ in a basis of single mode Fock states $\ket n$ (we may expand over all modes and all particle
numbers, but $a^\dag_{\vect k}$ acting on any other momentum of $\vect k\ne \vect k'$ vanishes, so 
we restrict our space to a single mode)
\ba
	a^\dag \ket \psi &= \lambda \ket \psi\\
	a^\dag \sum_{n=0}^\infty c_n \ket n &= \lambda  \sum_{n=0}^\infty c_n \ket n\\
	\sum_{n=0}^\infty \sqrt{n+1}c_n\ket{n+1} &= \lambda \sum_{n=0}^\infty c_n\ket n\\
	\sum_{n=1}^\infty \sqrt nc_{n-1}\ket n &= \lambda \sum_{n=0}^\infty c_n \ket n\\
	\sum_{n=1}^\infty \sqrt nc_{n-1}\ket n &= \lambda c_0 \ket 0 +\lambda \sum_{n=1}^\infty \sqrt nc_{n}\ket n.
\ea
This leads to
\[
	\lambda c_0\ket 0 + \sum_{n=1}^\infty (\lambda c_n -\sqrt n c_{n-1})\ket n = 0
\]
As the states are orthogonal, each coefficient must vanish independently
\ba
	\lambda c_0 &= 0\\
	\lambda c_1 &= \sqrt 1 c_0\\
	\lambda c_2 &= \sqrt 2 c_1\\
	&.....\\
	\lambda c_n &= \sqrt n c_{n-1}
\ea
For a nonzero eigenvalue, $\lambda \ne 0$, we must have $c_0 = 0$. However, through the recursion it 
then follows that all $c_n=0$. Thus a non-zero eigenstate of the creation operator cannot exist.
\newpage
% #3 ---------------------------------------------------------------------------------------------------------------------------------------------------------
\item[7.7]
Consider a beam splitter that combines two incoming photon modes $a_1$ and $a_2$ into two new photon modes
$A_1$ and $A_2$. 
\benum
\item
% (a)
Assume that the transformation between the modes is linear, so that $A_1 = u_{11}a_1+u_{12}a_2$ and
$A_2 = u_{21}a_1+u_{22}a_2$. Show that the matrix with the element $u_{ij}$ must be unitary. 
\\ \\
Now consider an ideal 50/50 beam splitter, for which the (unitary) transformation is $A_1 = \frac{1}{\sqrt 2}
a_1+\frac{1}{\sqrt 2}a_2$, $A_2 = -\frac{1}{\sqrt 2}a_1+\frac{1}{\sqrt 2}a_2$. \\
\item 
% (b)
Suppose that we put one photon in each incoming mode $a_1$ and $a_2$. Show that at the output both photons
will be observed to come out from the same port, either in the mode $A_1$ or $A_2$.
\\ \\
Hint: A state with one photon in the modes $a_1$ and $a_2$ may be written $a_1^\dag a_2^\dag\ket 0$. \\ \\
\eenum
\benum
\item
Recall the commutation relations for boson modes
\[
	[a_i,a_j^\dag] = \delta_{ij},\quad [a_i,a_j] = [a_i^\dag,a_j^\dag] = 0.
\]
This also applies to the $A$ modes. Using the linear composition of $A$ modes in terms of $a$ modes,
we equate them to the commutator
\ba
	[A_1,A_2^\dag] &= [u_{11}a_1+u_{12}a_2, u_{21}^*a_1^\dag+u_{22}^*a_2^\dag]\\
	&= u_{11}u_{21}^*[a_1,a_1^\dag] + u_{12}u_{22}^*[a_2,a_2^\dag] \\
	&= u_{11}u_{21}^*+u_{12}u_{22}^*\\
	&= 0.
\ea
Thus
\be\label{2}
	u_{11}u_{21}^* = -u_{12}u_{22}^*
\ee
and it follows
\be\label{3}
	u_{11}^*u_{21} = - u_{12}^*u_{22}.
\ee
Repeat the same procedure for $A_1$ and $A_2$, yielding
\be\label{4}
	[A_1,A_1^\dag] \Rightarrow |u_{11}|^2 + |u_{12}|^2 = 1
\ee
\be\label{5}
	[A_2,A_2^\dag] \Rightarrow |u_{21}|^2 +|u_{22}|^2 = 1
\ee
Now, a 2x2 unitary matrix must satisfy 
\ba
	U^\dag &= U^{-1}\\  
	\bpm u_{11}^* &u_{21}^*\\ u_{12}^*& u_{22}^* \epm &= \frac{1}{u_{11}u_{22}-u_{12}u_{21}}
	\bpm u_{22}&-u_{12}\\-u_{21}&u_{11}\epm
\ea
Now we relate all the elements using \eqref 2, \eqref 3,\eqref 4, and \eqref 5. 
\ba
	u_{22}|u_{11}|^2+u_{22}|u_{12}|^2 &= u_{22}\\
	u_{22}|u_{11}|^2-u_{21}u_{11}^*u_{12} &= u_{22}
\ea
\[
	\Rightarrow u_{11}^* = \frac{u_{22}}{u_{22}u_{11}-u_{21}u_{12}}
\]
\[
	\Rightarrow (U^\dag)_{11} = (U^{-1})_{11}
\]
\newpage
\ba
	-u_{12}|u_{22}|^2-u_{12}|u_{21}|^2 &= -u_{12}\\
	-u_{12}|u_{21}|^2+u_{11}u_{21}^*u_{22} &= -u_{12}
\ea
\[
	\Rightarrow u_{21}^* = -\frac{u_{12}}{u_{22}u_{11}-u_{21}u_{12}}
\]
\[
	\Rightarrow (U^\dag)_{21} = (U^{-1})_{21}
\]
\[
	u_{12}^*u_{11}u_{22}-|u_{11}|^2u_{21} = -u_{21}
\]
\[
	\Rightarrow u_{12}^* = -\frac{u_{21}}{u_{22}u_{11}-u_{21}u_{12}}
\]
\[
	\Rightarrow (U^\dag)_{12} = (U^{-1})_{12}
\]
\[
	-u_{22}^*u_{12}u_{21}+|u_{21}|^2u_{11} = -u_{21}
\]
\[
	\Rightarrow u_{22}^* = \frac{u_{11}}{u_{22}u_{11}-u_{21}u_{12}}
\]
\[
	\Rightarrow (U^\dag)_{22} = (U^{-1})_{22}.
\]
Thus we establish
\[
	U^\dag = U^{-1}.
\]
\\ \\
\item
we have for the state $a_1^\dag a_2^\dag \ket 0 = \ket \psi$
\[
	A_1 = \frac{1}{\sqrt 2}a_1+\frac{1}{\sqrt 2}a_2 = \alpha_1 a_1 +\beta a_2
\]
\[
	A_2 = -\frac{1}{\sqrt 2}a_1+\frac{1}{\sqrt 2}a_2 = \alpha_2 a_1 +\beta a_2
\]
The measurement probability for sequence $A_iA_j$ in the state $\ket{11} = a_1^\dag a_2^\dag \ket{0}$
is
\ba
	P(A_i A_j) &= \frac{1}{2}\braket{0|a_2a_1A_j^\dag A_i^\dag A_i A_ja_1^\dag a_2^\dag|0}\\
	&=\frac{1}{2}\braket{0|a_2a_1A_j^\dag A_i^\dag [A_i A_j,a_1^\dag a_2^\dag]|0}\\
	&=\frac{1}{2}\braket{0|a_2a_1A_j^\dag A_i^\dag A_i [A_j,a_1^\dag a_2^\dag]|0}\\
	&=\frac{1}{2}\braket{0|a_2a_1A_j^\dag A_i^\dag A_i(a_1^\dag [A_j, a_2^\dag]+[A_j,a_1^\dag]a_2^\dag)|0}\\
\ea
The results above follow from annihilation of the state from annihilation operators. Looking at the last
two commutators
\ba
	[A_i,a_1^\dag] &= [\alpha_i a_1,a_1^\dag] +[\beta a_2,a_1^\dag]\\
	& = \alpha_i
\ea
and likewise
\[
	[A_i,a_2^\dag] = \beta.
\]
Thus
\ba
	P(A_iA_j)
	&= \frac{1}{2}\braket{0|a_2a_1A_j^\dag A_i^\dag\beta(\alpha_i a_1a_1^\dag+\alpha_j a_2a_2^\dag)|0}\\
	& = \frac{1}{2}\braket{0|a_2a_1A_j^\dag A_i^\dag\beta( \alpha_i+\alpha_j)|0}
\ea
Now noting from the earlier assignment that 
\[
	\alpha_1+\alpha_2 = 0,\quad 2\alpha_1 = 2\beta,\quad 2\alpha_2 = -2\beta
\]
we see that unless $i=j$, the probability $P(A_iA_j) = 0$. Therefore, at the output both photons
will be observed to come from the same port, either in the mode $A_1$ or $A_2$.
\eenum
\eenum
\end{document}