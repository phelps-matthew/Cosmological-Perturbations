\documentclass[10pt,letterpaper]{article}
\usepackage{mymacros}

\title{Astrophysics \& Cosmology\\HW 10}
\author{Matthew Phelps}
\date{}

\begin{document}
\maketitle

\benum

% 1------------------------------------------------------------------------------------
\item[14.5]
From the virial theorem 
\[
	E_{kinetic} = -\frac12 E_{grav}.
\]
For $N$ bodies of mass $m$ with average separation between bodies $R$, the graviational potential of the total $N(N-1)/2$ pairings is
\[
	E_{grav} = -\frac{GN(N-1)m^2}{2R}
\]
and the total kinetic energy is 
\[
	E_{kinetic} = \frac12 mv^2.
\]
Now using the virial theorem
\[
	\frac{Nmv^2}{2} = \frac{GN(N-1)m^2}{4R}.
\]
This rearranges to 
\[
	(N-1)m = \frac{2Rv^2}{G}.
\]
For $N\gg 1$ and $M=Nm$, the above reduces to
\[
	M = \frac{2Rv^2}{G}.
\]
For a galaxy cluster, on average $v^2 = 3V^2$, where $V$ is the mean line-of-sight velocity displacement of a galaxy with respect to the center. In this case,
\[
	M = \frac{6RV^2}{G}
.\]
Contrast this with the approximate mass of a single galaxy, as a function of radius $r$ from the center
\[
	M(r) = \frac{rv^2}{G}
\]
and we see the mass of a spherical cluster of galaxies scales by a numerical factor of $6$ for mean line-of-sight velocity and a factor of $2$ for mean random velocity. \\ \\
For $V = 10^8\ \text{cm s}^{-1}$ and $R = 2.8\times 10^{24}\ \text{cm}$, the total mass is
\[
	M = \frac{6RV^2}{G} = 2.55\times 10^{48}\ {g} = 1.3\times 10^{15} M_{sun}.
\]
Problem 14.4 assumes a stellar disk of $\mu$ corresponding to $10^{11} M_{sun}$, with radius $R_{disk} = 10^{-2} R$. At the same radius then, the mass of the galaxy cluster would be $M \approx 10^{13} M_{sun}$. Thus a spherical galaxy distribution is only about 100 times more massive than the stellar gas disk mass. 
\\ \\
\item[14.10]
From the Hubble law
\[
	v = H_0 r
\]
it follows that
\[
	z = \frac{H_0 r}{c}.
\]
If the hubble constant is
\[
	H_0 = 20\ \text{km/sec/million lyr}
\]
and we spot a galaxy moving at redshift $z = 0.02$, then its distance would be
\[
	r = \frac{zc}{H_0} = 300\ \text{million lyr} = 2.8\times10^{26}\ \text{cm}.
\]
Small redshifts can be caused by gravitational interactions between objects and not that due to the expansion of space and therefore may not always be a reliable metric for computing distances. 
\eenum
\end{document}