\documentclass[10pt,letterpaper]{article}
\usepackage{mymacros}

\title{Astrophysics \& Cosmology\\HW 11}
\author{Matthew Phelps}
\date{}

\begin{document}
\maketitle

\benum

% 1------------------------------------------------------------------------------------
\item[15.1]
The energy of a select galaxy of mass $m$ lying at the edge of sphere of radius $r$ of mass $M = 4/3\pi r^3 \rho$ is
\[
	E = \frac12 m\dot r^2 - \frac{4\pi G m r^2}{3}.
\]
Setting $E=0$ and using $\dot r = H_0 r$ we have
\[
	H_0^2  = \frac{8\pi G}{3}\rho.
\]
or
\[
	\rho = \frac{3H_0^2}{8\pi G}.
\]
Since this is the energy density such that $E=0$, this is the critical energy density $\rho = \rho_c$. This density may be calculated at present time given 
\[
	H_0 = 20\ \text{km/sec/million lyr}
\]
which yields in CGS units
\[
	\rho_c \approx 8\times 10^{-30}\ \text{g cm}^{-3}.
\]
\\ \\
To make an estimate of comparison of the energy density, we take the mass of the Local Group 
\[
	M = 4\times 10^{11}M_{sun}
\]
spread over a volume of the M81 group
\[
	V= (15\times 10^6)^3\ \text{lyr}^3.
\]
This yields
\[
	\rho = \frac{M}{V} = 2.8\times 10^{-31}\ \text{g cm}^{-3}
\]
which is very close but about a factor of $10$ smaller than the critical density today. 
\\ \\
\item[15.2]
Setting
\[
	r \approx 10^{10}/H_0 \approx 10^{10}/10^{-18}\ \text{cm}\approx 10^{10}\ \text{lyr}.
\]
Referring to figure 14.23, distant ScI galaxies are at most $6\times 10^8 \ \text{lyr}$. This is still an appreciable $100$ times less than velocities required to begin approaching the speed of light. Thus it is likely that at these distances, $H_0$ still provides a good estimation for the distance as $v\ll c$. 
\\ \\
\item[15.3]
 For $E=0$ we have
\[
	0= \frac12 m\dot r^2 - \frac{ GmM}{r}
\]
in which $m$ cancels. Then we rearrange
\be
	\frac{dr}{dt} = \sqrt{ \frac{2GM}{r}}.
\ee
Since we measure red shifts and not blue shifts, the radius $r$ is increasing over time and thus $\dot r>0$. Hence the positive sign on the square root. Integrating (1) we have
\ba	
	\int_0^r dr\ \sqrt r &= \int_0^t dt\ (2GM)^{1/2}\\
	\frac23 r^{3/2} &= (2GM)^{1/2}t
\ea
which implies
\[
	r = \frac32 (2GM)^{1/3}t^{2/3}.
\]
It follow that at present time
\[
	H_0 r_0 = \dot r_0 = \plr{2GM/r_0}^{1/2} = H_0 \frac32 (2GM)^{1/3}t^{2/3}
\]
or
\[
	H_0(3/2 t_0)^{2/3}(2GM)^{1/3} = (2GM)^{1/3}(3/2 t_0)^{1/3}
\]
\[
	\frac32 H_0t_0 = 1
\]
\[
	t_0 = 2/3 H_0^{-1}.
\]
For $\dot r > \pfrac{2GM}{r}^{1/2}$ we have $E>0$. Since $t \propto r^{3/2}$ and $r$ increases at a faster rate than the marginally bound $E=0$ we know that we must have
\[
	\frac 23 H_0^{-1} < t_0 < t_{max}\qquad E>0
\]
where $t_{max}$ is shown explicitly in Box 15.2 to take the value $H_0^{-1}$. For $E<0$, $\dot r <  \pfrac{2GM}{r}^{1/2}$ and the bound universe may only decrease to a lower bound of $t_0 = 0$. Thus
\[
	0 < t_0 < \frac23 H_0^{-1}\qquad E<0.
\]
\\ \\
\item[15.4]
Using $v = r_0H_0$ and $M = 4/3\pi r_0^3\rho$ the energy equation may be written as
\[
	E = m(H_0^2/2-4\pi G\rho /3)r_0^2.
\]
Setting this equal to 15.1 at arbitrary $r$ (energy is conserverd)
\ba
	m(H_0^2/2-4\pi G\rho /3)r_0^2 &= \frac12m\dot r^2 - \frac{GmM}{r}\\
	(1-\Omega_0)&= \pfrac{ \dot r}{H_0 r_0}^2 - \frac{2GM}{H_0^2 r_0^2 r}\\
	(1-\Omega_0)&= \pfrac{ \dot r}{H_0 r_0}^2 - \frac{\Omega_0r_0}{r}
\ea
With substitution $D = r/r_0$ and $\tau^* = H_0 t$ it follows that
\[
	\frac{dD}{d\tau^*} = \frac{1}{r_0H_0}\dot r
\]
and the energy equation yields
\be
	(1-\Omega_0)= \pfrac{dD}{d\tau^*}^2 - \frac{\Omega_0}{D}.
\ee
We make another substitution
\[
	\xi = (|1-\Omega_0|/\Omega_0)D
\]
and
\[
	\tau = (|1-\Omega_0|^{3/2}/\Omega_0)\tau^*.
\]
We may rearrage (2) as
\[
	1 =  \pfrac{dD}{d\tau^*}^2\frac{1}{(1-\Omega_0)}-\frac{\Omega_0}{(1-\Omega_0)D} 
\]
Now with
\[
	\frac{d\xi}{d\tau} = \frac{1}{|1-\Omega_0|^{1/2}}\frac{dD}{d\tau^*}
\]
we can see that
\[
	\pfrac{d\xi}{d\tau}^2 - \frac{1}{\xi} = 1.
\]
Here we have assumed that $0<\Omega_0 < 1$. For $\Omega_0 > 1$, we have an overall change in sign, thus we can take absolute magnitudes and write the expression as
\be
	\pfrac{d\xi}{d\tau}^2 - \frac{1}{\xi} = \pm 1
\ee
without any restriction on $\Omega_0$. (3) yields the differential equation
\[
	d\tau = \plr{ \pm 1 +1/\xi}^{-1/2}d\xi = \pfrac{ \xi}{1\pm \xi}^{1/2} d\xi
\]
or
\[
	\tau = \int_0^\xi d\xi\  \pfrac{ \xi}{1\pm \xi}^{1/2}
\]
for $\xi(0) = 0$. For $1-\xi$ in the denominator $\xi \le 1$ and we make substitution
\[
	\xi = \sin^2(\eta/2)
\]
such that
\[
	d\xi = \sin(\eta/2)\cos(\eta/2)d\eta
\]
and
\ba
	\tau &= \int d\eta\ \sin(\eta/2)\cos(\eta/2) \pfrac{\sin^2(\eta/2)}{\cos^2(\eta/2)}^{1/2}\\
&= \int d\eta\ \sin^2(\eta/2)\\
&= \int d\eta\ \frac12 (1-\cos\eta)\\
\tau &= \frac12(\eta - \sin\eta)
\ea
where we also note that $\xi(\eta=0) = 0$.\\ \\
For the positive denominator, $1+\xi$ has $\xi >0$ we have an unbound universe and we use the substitution
\[
	\xi = \sinh^2(\eta/2)
\]
such that
\[
	d\xi = \sinh(\eta/2)\cosh(\eta/2)d\eta
\]
and
\ba
	\tau &= \int d\eta\ \sinh(\eta/2)\cosh(\eta/2) \pfrac{\sinh^2(\eta/2)}{\cosh^2(\eta/2)}^{1/2}\\
&= \int d\eta\ \sinh^2(\eta/2)\\
&= \int d\eta\ \frac12 (\cosh\eta -1)\\
\tau &= \frac12(\sinh\eta -\eta )
\ea
where again $\xi(\eta=0) = 0$. \\ \\
For the bound universe $\xi = \frac12 (1-\cos\eta)$ we see that at $\eta = \pi$, $\xi$ is maximum, and assumes the value of one. Thus
\[
	\xi = 1 = \frac{|\Omega_0-1|}{\Omega_0}\frac{r_{max}}{r_0}
\]
which leads to 
\[
	r_{max} = \frac{\Omega_0}{\Omega_0-1}r_0.
\]
From the figures below, we will find that the behavior is the same towards the origin because a series expansion around $\eta = 0$ up to leading order in $\eta$ gives the same for the bounded and unbounded universe, that is
\[
	\xi \approx \eta^2
\]
\[
	\tau \approx \frac{x^3}{3}.
\]
Below we have included sample tables, though the graphs generated used many more points.
	\figg[width=100mm]{15_4a.pdf}
	\[
	\begin{array}{ccc}
 \tau  & \xi  & \eta  \\
 0. & 0. & 0. \\
 0.0792645 & 0.229849 & 1. \\
 0.545351 & 0.708073 & 2. \\
 1.42944 & 0.994996 & 3. \\
 2.3784 & 0.826822 & 4. \\
 2.97946 & 0.358169 & 5. \\
 3.13971 & 0.0199149 & 6. \\
 3.17151 & 0.123049 & 7. \\
 3.50532 & 0.57275 & 8. \\
 4.29394 & 0.955565 & 9. \\
 5.27201 & 0.919536 & 10. \\
\end{array}
\]
		\figg[width=100mm]{15_4b.pdf}
		\[
		\begin{array}{ccc}
 \tau  & \xi  & \eta  \\
 0. & 0. & 0. \\
 0.0876006 & 0.27154 & 1. \\
 0.81343 & 1.3811 & 2. \\
 3.50894 & 4.53383 & 3. \\
 11.645 & 13.1541 & 4. \\
 34.6016 & 36.605 & 5. \\
 97.8566 & 100.358 & 6. \\
 270.658 & 273.659 & 7. \\
 741.239 & 744.74 & 8. \\
 2021.27 & 2025.27 & 9. \\
 5501.62 & 5506.12 & 10. \\
\end{array}
\]
\\ \\ \\

\item[15.8]
Starting with the Freidmann equation
\[
	\frac{\dot R^2}{R^2} - \frac{8\pi G\rho}{3}  = -\frac{c^2}{R^2}
\]
we substitute $\rho = \frac{M}{2\pi ^2 R^3}$ and multiply through by $\frac12 R^2$ to arrive at
\[
	\frac12 \dot R^2-\frac{2G}{3\pi}\plr{2\pi^2\rho R^2} = -\frac12 c^2
\]
or
\be
	\frac12 \dot R^2 - \frac{2GM}{3\pi R}=-\frac12 c^2.
\ee
Compared to eq 15.1
\[
	E = \frac 12 m\dot r^2 - \frac{GMm}{r}
\]
both involve time derivatives of a radius $R$ governed by a gravtiational interaction, however, the Freidmann equation makes no reference to the intrinsic mass of an object and differs in coeffiecents of kinetic and potential - like terms. Also, the energy constant is given by $c^2$ in the Freidmann equation. We now introduce auxially variable
\[
	\xi = \frac{3\pi c^2}{4GM}R
\]
\[
	\tau = \frac{3\pi c^3}{4GM}t
\]
such that 
\[
	\frac{d\xi}{d\tau} = \frac{\dot R}{c}.
\]
Multiplying (4) through by $2/c^2$, we now see that (4) may be written in terms of $\xi$ and $\tau$ as
\[
	\pfrac{d\xi}{d\tau}^2 - \frac{1}{\xi} = -1.
\]
This is the same differential equation solved in 15.4, with solutions
\[
	\xi = \frac12 (1-\cos\eta),\qquad \tau = \frac12 (\eta-\sin\eta).
\]
We see that $\xi$ takes its maximum value at $\xi(\pi) = 1$. It then follows that the maximum radius is
\[
	R_{max} = \frac{4GM}{3\pi c^2}.
\]
The time elapsed from big bang to big squeeze is $\tau(\xi = 0)$. Since $\xi$ reaches its second zero at $\eta = 2\pi$, this yields
\[
	\tau(2\pi) = \pi = \frac{3\pi c^3}{4GM}t
\]
or
\[
	t= \frac{4GM}{3c^3}.
\]
For $M = 10^{24}M_{sun}$ we have
\[
	R_{max} \approx 6\times 10^{10}\ \text{lyr}
\]
\[
	t_{squeeze} = 6.5\times 10^{18}\ \text{sec} = 2\times 10^{11}\ \text{yr}.
\]
We may use the hubble relation
\[
	H_0 = \frac{\dot R_0}{R_0}
\]
to express (4) as
\[
	H_0^2 - \frac{4GM}{3\pi}\frac{1}{R_0^3} = -\frac{c^2}{R_0^2}.
\]
Solving this cubic equation in terms of $H_0$ and other constants given, we have
\[
	R_0 = 2.04\times 10^{28}\ \text{cm} \approx 2\times 10^{10}\ \text{lyr}.
\]
This yields a $\xi_0$ of
\[
	\xi_0 =  \frac{3\pi c^2}{4GM}R_0 = 0.326.
\]
With $\xi_0 = \frac12(1-\cos\eta_0)$ this gives
\[
	\eta_0 = 1.21536
\]
and thus
\[
	\tau_0 = \frac12(\eta_0-\sin\eta_0) = 0.1389
\]
and finally
\[
	t_0 = \frac{4GM}{3\pi c^3}\tau_0 =2.9\times 10^{17}\ \text{sec} = 9.2\times 10^9\ \text{yr}.
\] 
Compare this to 
\[
	t_0 = \frac23 H_0^{-1} = 10^{10}\ \text{yr}
\]
and we see that the estimate above is very close the numerical calculation given by GR. 
\\ \\
\item[15.9]
The invariant volume element (for spatial metrics) is
\[
	dV = \sqrt{g} d^n x
\]
where $n$ denotes the dimension and $g$ is the determinant from the matrix $g_{ij}$ in the generalized line element
\[
	dl^2 = g_{ij} dx^i dx^j.
\]
Using this form, we may find the ``volume'' of a closed surface in one, two, or three dimensions. \\For $n=1$
\[
	g_{ij} = R^2 \qquad g = R^2
\]
\[
	\int\ dx^1\sqrt g = \int_0^{2\pi} d\phi R = 2\pi R.
\]
For $n=2$
\[
	dl^2 = R^2(d\theta^2 + \sin^2\theta d\phi^2)
\]
\[
	g = R^2\sin^2\theta
\]
\[
		\int\ dx^1dx^2\sqrt g = \int_0^\pi d\theta\ R \int_0^{2\pi} d\phi\ R\sin\theta= 4\pi R^2
\]
For $n=3$
\[
	dl^2 = R^2[ d\psi^2+\sin^2\psi d\theta^2 + \sin^2\psi\sin^2\theta d\phi^2]
\]
\[
	g = R^2(sin^2\psi\sin^2\psi\sin^2\theta)
\]
\ba
		\int\ dx^1dx^2dx^3\sqrt g &= \int_0^\pi d\psi\ R \int_0^{\pi} d\phi\ R\sin\theta \int_0^{2\pi}d\psi\ R\sin\psi\sin\theta\\
		&= (R\pi)(2\pi R^2) = 2\pi^2R^3.
\ea
We note that in the last problem, we used
\[
	\rho = M/V = M/(2\pi^2 R^3)
\]
which is the proper volume to use in a space with positive curvature. \\ 
\\
To put our line element in a more suitiable form, we make the coordinate transformation
\[
	r = R\sin\psi
\]
such that
\[
	dr = R\cos\psi d\psi.
\]
Now
\[
	(1-r^2/R^2)^{-1} dr^2 = (1-\cos^2\psi)^{-1} R^2\cos^2\psi d\psi^2 = R^2 d\psi^2.
\]
Thus we replace $R^2d\psi^2$ in the line element
\[
	dl^2 = R^2[ d\psi^2+\sin^2\psi d\theta^2 + \sin^2\psi\sin^2\theta d\phi^2]
\]
to yield
\[
	dl^2 = \frac{dr^2}{1-r^2/R^2}+r^2(d\theta^2+\sin^2\theta d\phi^2).
\]
This is the line element of 3-space with uniform positive curvature. 
\eenum
\end{document}