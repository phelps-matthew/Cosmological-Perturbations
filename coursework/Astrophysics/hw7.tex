\documentclass[10pt,letterpaper]{article}
\usepackage{mymacros}

\title{Astrophysics \& Cosmology\\HW 8}
\author{Matthew Phelps}
\date{}

\begin{document}
\maketitle

\benum

% 1------------------------------------------------------------------------------------
\item[11.1]
Since the mean free path between collision is $l$, the volume swept out by a dust particle with cross sectional area $\pi R^2$ is
\[
	V = l\pi R^2.
\]
Since only one collision is expected to occur within such a volume, we have
\[
	n/N = n = l\pi R^2
\]
or
\[
	l = \frac{1}{n\pi R^2}.
\]
Given $l = 3000\ \text{ly}$ and $R = 10^{-5}\ \text{cm}$, we have a density
\[
	n = \frac{1}{l\pi R^2} = 33.6\ \text{stars cm}^{-3}.
\]
Given a football stadium volume of $10^{12}\ \text{cm}^3$ this given the number of stars as
\[
	N = 33.65(10^{12}) = 3.36 \times 10^{12}
\]
\eenum

\end{document}