\documentclass[10pt,letterpaper]{article}
\usepackage{mymacros}

\title{Astrophysics \& Cosmology\\HW 2}
\author{Matthew Phelps}
\date{}

\begin{document}
\maketitle

\benum

% 1------------------------------------------------------------------------------------
\item[4.1]
Starting with pressure as force per area, we have
\be
	P = \frac{F}{A} = \frac{1}{A} \frac{\Delta p}{\Delta t} 
\ee
Within a time $\Delta t$, the momentum transferred to a given wall is the sum of momentum transfers from individual particles
\[
	\Delta p = \sum_i \Delta p_i.
\]
The particles will follow some velocity distribution, but for a large number of particles and for long times compared to the RMS collision time, the change in momentum will be an average steady-state value (at fixed $T$ and $V$) and thus
\[
	\Delta p \equiv \braket{N\Delta p} = \braket{nV p} = n A\Delta t\braket{vp}.
\]
In the last step, molecules travel an average distance $\braket v \Delta t$ and hit a wall with area $A$, thus spanning a volume $V = A\Delta t \braket v$. Also we set $\Delta p = p$ for simplicity. Substituing the above into (1) we have
\[
	P = n\braket{vp}.
\]
Note that the above applies to a single wall with particle velocities perpendicular to that wall, $v \equiv \vect v\cdot \vect n$. 
For an isotropic distribution, $\braket{v_x} = \braket{v_y}= \braket{v_z}$, and using 
\[
	\braket{vp} =\braket{ \vect v \cdot \vect p} = \braket{v_x p_x + v_y p_y + v_z p_z}= \braket{v_xp_x}+\braket{v_y p_y} + \braket{v_z p_z}
\]
we may write the pressure acting on any wall as 
\be
	P = \frac13 n\braket{vp}.
\ee
where now $v$ is the average velocity in any direction. \\ \\
\item[4.2]
With $p=mv$ we substitute this into (2) to obtain
\[
	P = \frac23 n\braket{\frac12 mv^2} = \frac23n \plr{\frac32 kT} =  nkT
\]
where we have used the result from statistical mechanics relating averaging kinetic energy to temperature $\braket{\frac12 mv^2} = \frac32 kT$. \\ \\
\item[4.3]
Starting with
\[
	\braket{mv^2} = \frac{ \int_0^\infty mv^2 \exp\plr{-mv^2/2kT}4\pi v^2\ dv}{\int_0^\infty \exp\plr{-mv^2/2kT}4\pi v^2\ dv}
\]
take $x = mv^2/2kT$ with $dx = mv/kT dv$ and $dv = kT/mv dx = \plr{kT/2m}^{1/2} x^{-1/2} dx$
\ba
	\braket{mv^2} &= \frac{ \int_0^\infty (2kTx) \exp\plr{-x}4\pi (2kT/m)x\ \plr{kT/2m}^{1/2} x^{-1/2} dx}{\int_0^\infty \exp\plr{-x}(2kT/m) x\ \plr{kT/2m}^{1/2} x^{-1/2} dx}\\
	&= 2kT \frac{ \int_0^\infty x \exp\plr{-x} x\ \plr{kT/2m}^{1/2} x^{-1/2} dx}{\int_0^\infty \exp\plr{-x} x\ \plr{kT/2m}^{1/2} x^{-1/2} dx}\\
	&= 2kT\frac{ \int_0^\infty x^{3/2}e^{-x}\ dx}{x^{1/2} e^{-x}\ dx}.
\ea
Looking at the top integral, we may write
\[
	x^{3/2}e^{-x} = -\frac{d}{dx}\plr{ x^{3/2}e^{-x}}+\frac32 x^{1/2}e^{-x}
\]
and thus
\[
	 \int_0^\infty x^{3/2}e^{-x}\ dx = \frac32 \int_0^\infty x^{1/2}e^{-x}\ dx - \elr{\plr{ x^{3/2}e^{-x}}}_0^\infty = \frac32 \int_0^\infty x^{1/2}e^{-x}\ dx.
\]
Noting that this is the same integral in the denominator, we now have
\[
	\braket{mv^2} = 2kT\pfrac32 = 3kT.
\]
\\ \\
\item[4.15]
The luminosity of the sun is constant and gives the total power radiated. Taken at a distance $r$, which is the distance from sun to earth, the total power per area (flux) may be found by dividing by the area of a sphere at $r$
\[
	f = \frac{L_{sun}}{4\pi r^2}.
\]
The amount of luminous power intercepted at the earth is then just the flux times the cross sectional area of the earth
\[
	P = \pi R^2 f = L_{sun}\pfrac{\pi R^2}{4\pi r^2}
\]
where $R$ is the cross-sectional area of the earth. \\ \\
As the earth rotates, different parts of the surface of earth are exposed to the sun. On average, the luminous power per area intercepted by a rotating earth is the luminous power intercepted by the cross section of earth, $P$, spread out over the entire surface area of the Earth $4\pi R^2$. Thus the average flux density intercepted by a rotating earth is
\[
	\bar f = P/4\pi R^2 = L_{sun}/16\pi r^2.
\]
Now, assume only 61\% of $\bar f$ is absorbed by the Earth, with the rest being reflected. This flux will heat the surface of the earth to a tempertaure $T$, in which the Earth will then behave roughly as a blackbody and will re-radiate out energy at the rate 
\[
	\sigma T^4.
\]
Equating this to the flux absorbed, we have
\[
	\sigma T^4 = 0.61 \bar f,
\]
or
\be
	T =  \pfrac{0.61 \bar f}{\sigma}^{1/4}= \pfrac{0.61  L_{sun}}{16\pi r^2\sigma}^{1/4} = 246.8 [\text K].
\ee
\\
\item[5.1]
If we enclose the sun in a sphere of radius $r$, the flux intercepted by this sphere must be the total luminous power over the area $f = L_{sun}/A$. However, if we move out to a further radius $r'$, again the flux is $f' = L_{sun}/A'$, but now the flux is less since the area is larger. Yet we know the luminous power must be constant over any intercepting surface, since energy is conserved. So it is in fact the product $fA$ that is constant and equal to the luminous power. Thus
\[
	L = f(4\pi r^2)
\]
\[
	L = (1.36\times 10^6)4\pi (1.5\times 10^{13})^2=3.85\times 10^{33}\  [\text{erg s}^{-1}]
\]
\\ \\
\item[5.2]
\[
	100\ [\text W] = 10^9\ [\text{erg s}^{-1}]
\]
If we divide the surface area of the earth by $30\ [\text{cm}^2]$, we find the number of bulbs on the surface of the earth. The total luminous power emitted from such an earth would then be
\[
	L_{earth} = \pfrac{4\pi R^2}{30}10^9 = 1.7\times 10^{26}\ [\text{erg s}^{-1}].
\]
This luminosity differs from the sun by about a factor of $2\times 10^{7}$ - a large difference. \\ \\
\item[5.3]
We may find the linear size of the sun by forming a triangle and using trigonometry
\[
	R_{sun} = r \tan\blr{ \frac12\pfrac{32}{60} \pfrac{\pi}{180}} = 6.98\times 10^{10}\ [\text{cm}].
\]
Now we may find the temperature by treating the Sun as a blackbody, recalling that the flux at the surface must be equal to the radiated power per area
\[
	f = \sigma T^4
\]
\[
	\frac{L_{sun}}{4\pi R^2} = \sigma T^4
\]
or 
\[
	T = \pfrac{L_{sun}}{4\pi\sigma R^2}^{1/4} = 5.8\times 10^3 [\text K].
\]
\\ \\
\item[5.4]
If we treat the motion of the earth around the sun as circular, we have the centripital force
\[
	m \frac{v^2}{r} = \frac{mMG}{r^2}
\]
or
\[
	M = \frac{r v^2}{G}.
\]
With $r=1\text{AU}$ and 
\[
	v = \frac{2\pi r}{P} = 2.98\times 10^6\ [\text{cm sec}^{-1}]
\]
we verify that
\[
	M = 2.0 \times 10^{33}\ [\text{gm}].
\]
Now take the total energy 
\[
	E = \frac12 mv^2 - \frac{GmM}{r}
\]
and substitute the centripital velocity
\[
	v = \pfrac{GM}{r}^{1/2}
\]
so that
\[
	E = \frac12 \frac{GmM}{r} - \frac{GmM}{r} = -\frac12 \frac{GmM}{r} = \frac12 V.
\]
\\ \\
\item[5.7]
Gauss's law states that the  flux through a surface is equal to the divergence of the field within the surface. Given a gravitational field $\vect g$, we have
\be
	\oint_S \vect g\cdot \vect n \ dA = \int_V \del\cdot \vect g\ dV.
\ee
From the Poisson equation for a Newtonian gravitational field 
\[
	\del\cdot g = - 4\pi G\rho
\]
we substitute this into (4)
\[
	\oint_S \vect g\cdot \vect n \ dA = -4\pi G \int_V \rho\ dV = -4\pi G M.
\]
The above states that the graviational field perpendicular to a closed surface (and thus by extension the gravitational force $\vect F = m\vect g$) depends only upon the mass within the surface, i.e only the internal mass contributes. 
 \\ \\
 \item[5.9]
 As pressure is the force per unit area, we may calculate the force acting on a unit area at a radius $r$ in the sun by finding the gravitational force acting from a column of mass centered above area $A$ and radius $r$. To start, let us first find the differential force acting on a differential volume element at a given radius $r$
 \[
 	dV = Adr
\]
\[
	dm = \rho(r)dV = A\rho(r)dr
\]
\[
	dF = \frac{GM(r)dm}{r^2} = A\frac{G\rho(r)M(r)}{r^2} dr.
\]
The mass of the sun at a radius $r$ can be found by breaking it up into shells of differential volume $dV = 4\pi r^2 dr$
\[
	M(r) = \int_0^r\ dr' 4\pi r'^2 \rho(r').
\]
Then
\[
	dF = AG \frac{\rho(r)}{r^2} \int_0^r\ dr' 4\pi r'^2 \rho(r').
\]
To find the total force acting on a unit area at $r$, we must integrate up to the surface
\[
	F(r) = AG \int_r^R dr_1\  \frac{\rho(r_1)}{r_1^2} \int_0^{r_1}\ dr_2 \ 4\pi r_2^2 \rho(r_2)
\]
and the pressure is then
\[
	P(r) = G \int_r^R dr_1\  \frac{\rho(r_1)}{r_1^2} \int_0^{r_1}\ dr_2 \ 4\pi r_2^2 \rho(r_2).
\]
If we assume the density is constant throughout (similar to an averaging procedure) then $\rho = 1.4\ [\text{gm cm}^{-3}]$ and we find the pressure at the center of the sun by taking $r=0$
\ba
	P&= \frac{4\pi}{3} G\rho^2 \int_0^R dr_1 r_1\\
	&= \frac{4\pi}{3} G \rho^2 \pfrac{R^2}{2}\\
	&= \frac{GM_{sun}^2}{R^4} \pfrac{ 3}{8\pi}\\
	&= 0.12  \pfrac{GM_{sun}^2}{R^4}\\
	&= 1.35 \times 10^{15}\  [\text{gm cm}^{-1}\text{sec}^{-2}].
\ea
I was unable to verify the factor of 19?\\ \\
If we look at the average gravitational field $g$ felt by a column of mass above sea level $\mu$ we may find the pressure at sea level as
\[
	P = \mu g = (1.03\times 10^3)(9.80 \times 10^2)= 1.0\times 10^6\   [\text{gm cm}^{-1}\text{sec}^{-2}].
\]
As expected, the pressure at the sun is extraordinarily greater than that of the sun -  a factor of $10^9$ according to the calculation. \\ \\
\item[5.10]
From the ideal gas law
\[
	P = nkT
\]
or 
\[
	T = \frac{P}{nk}.
\]
Looking at the central pressure of the sun, with $n = \rho/m = 10^26\ [\text{cm}^{-3}]$ and $P_c$ calculated as $P_c = 2.1\times 10^{17}\ [\text{erg cm}^{-3}]$, this yields
\[
	T = \frac{2.1\times 10^{17}}{(10^{26})1.38\times10^{-16}} = 1.5\times 10^7\ [\text K].
\]
\\ \\
\item[5.11]
In random walk, the position after $N+1$ interactions is recursively related to its position prior by taking equal probabilities of traveling a mean free path length $l$ in either direction
\ba
	\braket{x_{N+1}^2} &=\frac12 \braket{(x_N-l)^2} +\frac12 \braket{(x_N+l)^2}\\
	&= \braket{x_N^2} -\braket{x_N l} + \braket{x_N l} +\braket{l^2}\\
	&= \braket{x_N^2} + l^2
\ea
Looking at the above formula, we see that the base case of zero interactions $N=0$ and starting at the origin leads to 
\[
	\braket{x_1^2} = l^2
\]
which agrees with our mean free path length. Now let us take an arbitrary value of $N$ and substitute $N+1$ :
\ba
	\braket{x_{N+2}} &= \braket{x_{N+1}^2} + l^2\\
	&= \braket{ (x_N^2+ l^2)} + l^2\\
	&= \braket{x_N^2}+2l^2
\ea
If we start with the root case $N=0$, $\braket{x_N^2} = 0$ we see that at arbitrary $N$ 
\[
	\braket{x_N^2} = Nl^2.
\]
The number of steps to cover a root mean square distance is then
\[
	N= \frac{\braket{x_N^2}}{l^2}.
\]
In three dimensions, the photon has equal probablity of traveling in any direction and so the number of interactions scales by a factor of 3. To escape the sufrace of the sun, then
\[
	N = \frac{3R^2}{l^2}
\]
where $R$ is the radius of the sun. Traveling at the speed of light, the photon travels a distance $Nl$
\[
	ct = Nl
\]
and solving for $t$
\[
	t = \frac{Nl}{c} = \frac{3R^2}{lc}.
\]
Assuming $l = 0.5\ [\text{cm}]$, the time it takes to get out of the sun is
\[
	t = \frac{3(6.96\times 10^{10})}{0.5(3\times 10^{10})} =9.7\times 10^{11} \ [\text {sec}].
\]
Compare this to a free streaming photon
\[
	t = \frac{R}{c} = 2.32 \ [\text{sec}].
\]
\\ \\
\item[5.12]
To calculate the luminosity of the sun, we will need to find the energy of radiated emission per second. Assuming the sun to lie at a temperature $T$, we know the energy density is given by 
\[
	\ep = aT^4
\]
where $a = 7.56\times 10^{-15}\ [\text{erg cm}^{-3}\text K^{-4}]$. If we multiple the energy density by the total volume of the sun, we get the total energy due to thermal radiation. However, it takes the radiation an average of $t = \frac{Nl}{c}$ seconds to escape to the surface and emit into vacuum. Thus the energy per unit time is found by simply dividing by the travel time $t$
\[
	L = \frac{4\pi R^3 a T^4 lc}{9 R^2} .
\]
Evaluating this at $T = 4.5\times 10^6\ [\text K]$ we have
\[
	L = 4.52 \times 10^{33}\ [\text{erg sec}^{-1}].
\]
This is actually very close to the observed value of $3.9 \times 10^{33}\ [\text{erg sec}^{-1}]$. \\ \\
\item[5.13]
We may calculate the ratio of plasma energy density to that due to thermal radiation at the center of the sun. From statistical mechanical kintetic theory
\[
	\frac{3}{2}kT = E
\]
and from the ideal gas law we also know
\[
	\frac32 nkT = \frac32 P.
\]
Since $n$ is the number of particles per volume, the quantity
\[
	\ep = \frac32 P
\]
gives the energy density of the gas (in this case plasma). Meanwhile, the energy density due to thermal radiation is
\[
	\ep =aT^4.
\]
The ratio of energy densities is then
\[
	\frac{\ep_p}{\ep_r} = \frac{3/2 P}{aT^4}.
\]
Taking the pressure at the center of the sun $P= 2.1\times 10^{17}$ and the temperature at the center $T = 1.5\times 10^7$, we have
\[
	\frac{\ep_p}{\ep_r} = 823 \approx 10^3.
\]
\\ \\
\item[5.14]
For the gravitational potential energy of the sun
\[
	W = -2\frac{GM^2}{R}.
\]
The time at which such gravitational potential energy could fuel a luminosity $L$ of the sun is
\[
	t = -\frac12 \frac{W}{L}.
\]
Taking the current luminosity $L = 3.9\times 10^{33}$ this equates to
\[
	t = 9.73\times 10^{14}\ [\text{sec}]  = 3.09\times 10^7 \ [\text{yr}].
\]
Thus graviational contraction could only have sustained such luminosity for $3\times 10^7$ years, in conflict with the fact that life has been on Earth for $3\times 10^9$ years. 
\eenum
\end{document}