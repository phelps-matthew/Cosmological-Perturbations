\documentclass[10pt,letterpaper]{article}
\usepackage{mymacros}

\title{Astrophysics \& Cosmology\\HW 8}
\author{Matthew Phelps}
\date{}

\begin{document}
\maketitle

\benum

% 1------------------------------------------------------------------------------------
\item[11.1]
Since the mean free path between collision is $l$, the volume swept out by a dust particle with cross sectional area $\pi R^2$ is
\[
	V = l\pi R^2.
\]
Since only one collision is expected to occur within such a volume, we have
\[
	n/N = n = l\pi R^2
\]
or
\[
	l = \frac{1}{n\pi R^2}.
\]
Given $l = 3000\ \text{ly}$ and $R = 10^{-5}\ \text{cm}$, we have a density
\[
	n = \frac{1}{l\pi R^2} = 33.6\ \text{stars cm}^{-3}.
\]
Given a football stadium volume of $10^{12}\ \text{cm}^3$ this given the number of stars as
\[
	N = 33.65(10^{12}) = 3.36 \times 10^{12}\ \text{stars}.
\]\\ \\
\item[12.1]
The number of stars with $f>f_0$ is all stars with $r<r_0$. Thus the total number of stars will lie in a volume spanned by the limit of $r_0$, i.e.
\[
	V = \frac{4}{3}\pi r^3.
\]
Multiplying by the density
\[
	N = n(L)\frac{4}{3}\pi r^3.
\]
From the luminosity relation
\[
	L = 4\pi r^2 f_0
\]
we have
\[
	\frac{4}{3}\pi r_0^3  = \frac{L^{3/2}f_0^{-3/2}}{3(4\pi)^{1/2}}
\]
and thus the total number of stars is
\[
	N_L(f>f_0) =  \frac{n(L)L^{3/2}f_0^{-3/2}}{3(4\pi)^{1/2}}.
\]
\\ \\
\item[12.3]
Equating the pressure due to gravity to the pressure from kinetic energy
\ba
	P& = F/A = \rho_k \bar g_z H\\
	&= P_k = \rho_k v_z^2
\ea
this  leads to
\[
	\bar g_z = \frac{H}{v_z^2}.
\]
Poisson eq:
\[
	\del \cdot \vect g = -4\pi G\rho.
\]
Divergence theorem:
\[
	\int \del \cdot \vect g\ d^3x = \oint \vect g \cdot \vecth n\ dA = -4\pi G\int d^3x\ \rho
\]
\[
	\oint \vect g\cdot \vecth n\ dA = -4\pi GM.
\]
Taking a surface as a rectange of height $2H$ centered at $z=0$, the only flux of the field through the suface is that at $z=\pm H$. In this case, the gravitational field is in opposite direction to the normal of the surface, thus
\[
	\oint \vect g\cdot\vecth n\ dA = -2Ag_z.
\]
\[
	-2Ag_z = -4\pi GM
\]
\[
	\mu = \frac{M}{A} = \frac{g_z}{2\pi G}.
\]
Given that $g_z = 2\bar g_z$ and from the above $\bar g_z = H/v_z^2$, this yields
\[
	\mu = \frac{H}{v^2\pi G}.
\]
\item[12.4]
\ba
	A-B &= -\frac{r}{2} \frac{d\Omega}{dr} +\frac{1}{2r}\frac{d}{dr}(r^2\Omega)\\
	&=  -\frac{r}{2} \frac{d\Omega}{dr} + \frac{1}{2r}\plr{ 2r\Omega + r^2\frac{d\Omega}{dr}}\\
	&= \Omega 
\ea
If $A=0$, it follows that $\Omega = -B$. \\ \\
With 
\[
	(1-A/B)^{1/2} = 1.6
\]
and 
\[
	A = 0.005\ \text{km/sec/lyr}
\]
it follows that 
\[
 	B = -A/1.56 = -.0032\ \text{km/sec/lyr}.
\]
Then the period is
\[
	T=2\pi/\Omega = 2\pi/(A-B) = 765.8\ \text{s}^{-1}
\]
At a radius of $r=3\times 10^4\ \text{lyr}$, this implies a linear velocity of
\[
	v = r\Omega = 3\times 10^4 (0.0082) = 246.15\ \text{ km s}^{-1}.
\]
From the mass galaxy formula
\[
	M_G = \frac{rv^2}{G} 
\] 
we then have
\[
	M_G = 2\times 10^12\ \text{g} = 10^{-21} M_{sun}
\]
\\ \\
I must have made an error somewhere within the period $T = 2\pi/\Omega$?
\\ \\
\item[12.5]
From Gauss's law, let us take our surface to enclose the mass at radius $r = 30000\ \text{lyr}$. Then
\[
	\oint \vect g\cdot \vecth n \ dA = -4\pi G M
\]
\[
	-4\pi r^2 g(r) - 4\pi GM
\]
\[
	M = \frac{r^2g(r)}{G}.
\]
At the radius $r$, if in circular orbit, the centripetal accleration is that due to gravtiation
\[
	mg(r) = m \frac{v^2}{r}
\]
or
\[
	g(r) = \frac{v^2}{r}.
\]
Thus we have
\[
	M = \frac{v^2r}{G}.
\]
For $v= 300\ \text{km s}^{-1}$ and $r = 6\times 10^4\ \text{lyr}$ this yields
\[
	M_G = 5.1\times 10^{37} \text{g} = 10^5 M_{sun}.
\]
\eenum
\end{document}