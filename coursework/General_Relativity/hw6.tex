\documentclass[10pt,letterpaper]{article}
\usepackage{mymacros}

\title{General Relativity\\HW 6}
\author{Matthew Phelps}
\date{Due: Nov 29}

\begin{document}
\maketitle

\benum
% 1------------------------------------------------------------------------------------------------------------------------
\item
If we are to try to describe relativistic gravity in terms of a scalar function, one that could then be compared to that of Newtonian gravity, let form a general metric in terms of such a scalar function as
\[
	d\tau^2 = \exp\pfrac{2\phi}{c^2}dt^2 - dl^2.
\]
Here the spatial part $dl$ may also include non-trivial metric components, but our focus is upon $g_{00}$ only, as it is the dominant term that appears in the non-relativistic limit of the Einstein field equations, i.e. the Newtonian Poisson equation, as $T_{00} = \rho$ implies the usage of $\del^2 g_{00} = -8\pi GT_{00}$. 
\\ \\
For the Schwarzchild metric,
\[
	\exp\pfrac{2\phi}{c^2} = \plr{ 1-\frac{2GM}{rc^2}}
\]
and thus
\[
	\phi_S  = \frac{c^2}{2} \ln\plr{1-\frac{2GM}{rc^2}}.
\]
We define this $\phi$ as the general relativistic potential. If $\phi = 0$, we recover Minkowski space. Now, we may expand the potential as a series in $-2GM/rc^2$
\[
	 \ln\plr{1-\frac{2GM}{rc^2}} = -\frac{2GM}{rc^2} - \frac12 \pfrac{2GM}{rc^2}^2 - \frac13 \pfrac{2GM}{rc^2}^3 - ...
\]
and so the potential expansion is 
\[
	\phi = -\frac{GM}{r} - \frac{1}{c^2}\pfrac{GM}{r}^2-\frac43\frac{1}{c^4} \pfrac{GM}{r}^3-...
\]
We see that the first order approximation to the Schwarzchild metric is given by the Newtonian potential
\[
	\phi_{N} = -\frac{GM}{R}.
\]
However, as $GM/r$ becomes larger, i.e. for strong gravitational fields, the Newtonian potential begins to fall short of effectively describing gravitation. In such a regime, we must have the full Schwarzchild potential given by the logarithm. \\ \\
The leading order difference between the Newtonian and Schwarzchild potential is the quadratic term, so
\[
	\frac{|\phi_N-\phi_S|}{\phi_N} =\frac{(GM/rc^2)^2}{GM/rc^2} = \frac{GM}{rc^2} =0.01.
\]
This $1\%$ difference occurs at a radius of 
\[
	r= 100GM/c^2.
\]
\\ \\
The min and max radii are
\[
	r_{min} = 2MG/c^2,\qquad r_{max} = 100MG/c^2.
\]
The volume in which stars may occupy is then
\[
	\frac43 \pi (r_{max}^3-r_{min}^3) = \frac43\pi[ (100MG/c^2)^3-(2MG/c^2)^2\approx \frac43\pi 10^6MG/c^2
\]
For a star of $M = 10^6 M_{\odot}$,
\[
	V_1 = 1.34\times 10^{34} \text m^3
\]
while for $M = 10^9 M_{\odot}$
\[
	V_2 = 10^9 V_1.
\]
Dividing these volumes by the radius of a normal star $r_{\odot} = 10^10 \text m$,
we have as the number of stars that could fill such a volume
\[
	\frac{V_1}{V_{\odot}} = 3200 \approx 10^3\ \text{stars}
\]
\[
	\frac{V_2}{V_{\odot}} =  10^9 \frac{V_1}{V_{\odot}}\approx 10^{12} \ \text{stars}.
\]
% 2------------------------------------------------------------------------------------------------------------
\item 
\benum
\item
% a
Schwarzchild metric:
\[
	d\tau^2 = \plr{ 1-\frac{2MG}{r}}dt^2-\plr{ 1-\frac{2MG}{r}}^{-1}dr^2 - r^2 d\theta^2 - r^2\sin^2\theta d\phi^2
\]
For circular orbit $dr = d\theta = 0$, and we may choose coordinates such that the plane of precession has $\sin^2\theta = 1$. The remaining metric factors are
\[
	g_{00} = -\plr{1-\frac{2MG}{r}},\qquad g_{\phi\phi} = r^2.
\]
As we seek to integrate over one period, we need to relate $d\phi$ to $dt$. This can be accomplished by
\be
	\frac{d\phi}{d\tau}\frac{d\tau}{dt} = \frac{U^\phi}{U^0}.
\ee
Defining angular momentum $L\equiv U_\phi$ and energy $E \equiv -U_0$, (1) is
\be
	\frac{d\phi}{d\tau}\frac{d\tau}{dt}=\frac{p^\phi}{p^0} = \frac{L}{E}\frac{g^{\phi\phi}}{-g^{00}} = \frac{L}{E}\plr{\frac{1}{r^2\plr{1-2M/R}}}.
\ee
To determine $L/E$, we may use the energy relation 
\be
	g_{\mu\nu}p^\mu p^\nu = -m^2.
\ee
For the moment we generalize (3) to include radial motion - this will be useful to find orbital conditions. (3) yields an equation for $dr/d\tau$
\be
	\pfrac{dr}{d\tau}^2 = E^2- \plr{1-\frac{2M}{r}}\plr{1+\frac{L^2}{r^2}}
\ee
This equation looks quite similar to the total energy in a classical circular orbit, but here with effective potential
\[
	V(r) = \plr{1-\frac{2M}{r}}\plr{1+\frac{L^2}{r^2}}.
\]
For a stable circular orbit, we must have $E^2 = V$ and we must be at a minimum of $V(r)$, so
\[
	\frac{d}{dr}V(r)= 0,\qquad \frac{d^2}{dr^2}V(r) >0.
\]
This occurs at 
\[
	r = \frac{L^2}{2M}\plr{1+\sqrt{1-\frac{12M^2}{L^2}}},
\]
or in terms of $L$
\be
	L^2 = \frac{Mr}{1-3M/r}.
\ee 
Finally, using (4) and (5) we may solve for $L/E$ and substitute this into (2)
\be
	\frac{d\phi}{dt} = \pfrac{M}{r^3}^{1/2}.
\ee
Now we are able to integrate over one full period
\ba
	\tau &= \int d\tau = \int_0^{2\pi} \blr{ \plr{1-2M/r}dt^2-r^2d\phi^2}^{1/2}\\
	&=  2\pi\blr{ \plr{1-2M/r}\frac{r^3}{M}-r^2}^{1/2}\\
\ea
For $r=10M$ this is
\[
	\tau = 20\pi M\sqrt 7.
\]
\item
% (b)
For a distant observer, the time elapsed between orbits is just the coordinate time $dt$. From (6)
\[
	dt = \pfrac{r^3}{M}^{1/2}d\phi
\]
and so, at $r=10M$, this is
\[
	t = 2\pi \sqrt{ 10^3 M^2} = 20\pi M\sqrt{10}.
\]
\item
% (c)
For another observer at rest but at the same orbtial radius, $d\phi =0$ and its proper time interval is given as
\[
	\tau = \int d\tau = \int dt\plr{1-\frac{2M}{r}}^{1/2}.
\]
Here $d\tau$ represents time oberserved in such a rest frame, while $dt$ represents the coordinate time as measured by a distant observer. However, from part (b) we know exactly that the coordinate time elapsed was $t = 20\pi M\sqrt{10}$: this serves as our integration bounnds.
\[
	\tau = \int_0^{20\pi M\sqrt{10}} dt \plr{1-\frac{2M}{r}}^{1/2} = 20\pi M \sqrt{10\pfrac45} = 40\pi \sqrt 2.
\]
\eenum
% 3------------------------------------------------------------------------------------------------------------
\item 
\benum
\item
% (a)
For a particle moving only radially, we may use (4) to find ($G=1$ units)
\[
	\frac{dr}{d\tau} = \sqrt{E^2-\plr{1-\frac{2M}{r}}}.
\]
To find the total proper time elapsed we have
\be
	\tau = \int d\tau = \int dr \blr{ E^2 - \plr{1-\frac{2M}{r}}}^{-1/2}
\ee
Rescaling this as $r' = r/M$ and for $E = 0.95$ we have
\[
	\tau =M \int_2^3 dr'\blr{E^2-\plr{1-\frac{2}{r'}}}^{-1/2} = 1.19M.
\]
\item 
% (b)
Using (7) for $0$ to $2M$, we have
\[
	\tau = 1.37M.
\]
Although the distance is twice as far, the rate at which time elapses increases towards the center of the black hole. 
\item
% (c)
From earlier, 
\[
	U^r = \frac{dr}{d\tau} = \sqrt{E^2-\plr{1-\frac{2M}{r}}}
\]
and we also have
\[
	U^0 = \frac{dt}{d\tau} = g^{00}(-U_0) = \plr{1-\frac{2M}{r}}^{-1}E,
\]
with the other components vanishing for radial motion. At $r = 2.001 M$, these evaluate to
\[
	U^r = 0.95,\qquad U^0 = 1900.95.
\]
We see that an infinite coordinate time elapses when the particle finally reaches the event horizon. \\ \\
\item
% (d)
To find the total redshift observed by a distant observer we must find the shift due to the doppler effect from the radial motion, as well as the graviational redshift due to spacetime curvature.\\
\\
The doppler shift is given by 
\[
	\frac{\lambda_o}{\lambda_e} = \pfrac{1+v}{1-v}^{1/2}
\]
where $\lambda_o$ is the observed wavelength, $\lambda_e$ is emitted, and $v$ is the receding relative (radial) velocity. These are $c=1$ units. From (c) we have 
\[
	v=\frac{dr}{dt} = 0.95/1900.95 = 5\times 10^{-4}\text m
\]
thus
\[
	\frac{\lambda_o}{\lambda_e} = 1.0005.
\]
Meanwhile, the gravitational redshift is given by
\[
	\frac{\lambda_o}{\lambda_e}  = \pfrac{g_{00}(x_o)}{g_{00}(x_e=2.001MG)}^{1/2}
\]
If we take our observation point to be far enough away that we reside in Minkowski space, then the above is
\[
	\frac{\lambda_o}{\lambda_e} = \plr{1-\frac{2M}{r_e}}^{-1/2} = 44.73.
\]
In term of the redshift then, we have
\[
	z = z_{\text{doppler}}+z_{\text{graviation}} = 0.0005+43.73 \approx 44.
\]
The gravitational redshift is extreme, but according to what we saw earlier (infinte time dilation at event horizon), this seems reasonable given a radius very near the event horizon. 
\\
\eenum
% 4------------------------------------------------------------------------------------------------------------
\item 
The equation for geodesic deviation is
\[
	\frac{D^2\xi^\lambda}{D\tau^2} = R^\lambda_{\nu\mu\kappa}\xi^\mu U^\nu U^\kappa.
\]
If we choose a coordinate system that is always locally inertial, i.e. freely falling, then we know that $g_{\mu\nu}=\eta_{\mu\nu}$ and its first derivatives vanish and hence the affine connections also vanish. This means that in such a frame we may rewrite the deviation equation with covariant derivatives reducing to ordinary derivatives
\[
	\frac{d^2\xi^\lambda}{d\tau^2} = R^\lambda_{\nu\mu\kappa}\xi^\mu U^\nu U^\kappa.
\]
We should also note that in the local inertial frame, we may specifically choose among our Lorentz frames ie where the particle is at rest, that is
\[
	U^\alpha = U^0 = 1.
\]
In addition, we start in a frame where we are located intially at $\xi^\lambda = \xi^r$, all other components vanishing. \\ \\
The condition on the four-velocity implies
\[
	d\tau = cdt,
\] 
and thus our deviation equation simplies to (using $\xi^\lambda = \xi^r\equiv \xi$)
\[
	\frac{d^2\xi^\lambda}{dt^2} = R^\lambda_{0r0}\xi c^2.
\]
Since the tidal force in the radial direction will be the largest, we restrict our attention to only this component
\[
	\frac{d^2\xi}{dt^2} = R^r_{0r0}\xi c^2.
\]	
Now for the Schwarzchild metric, we may calculate the Riemann tensor component
\[
	R^r_{0r0} = \frac{2MG}{r^3}
\]
and so we finally have an equation for the radial tidal force acting on a body at a radius $r$ from the center of a blackhole
\[
	\frac{d^2\xi}{dt^2} =  \frac{2MG}{r^3}\xi .
\]	
At the horizon $r = 2MG/c^2$, the force is
\[
	F = \frac{c^6}{4M^2G^2}\xi .
\]
For simplicity, let us take $\xi = 1\text m$, about half the height of a human. Equating this with the limiting force gradient
\[
	400= \frac{c^6}{4M^2G^2}
\]
or
\[
	M^2 = \frac{c^6}{1600 G^2}.
\]
This gives
\[
	M = 1.01\times 10^{34} \approx 5085 M_{\odot} \approx (10^3\text-10^4) M_{\odot}.
\]
\\ \\
% 5------------------------------------------------------------------------------------------------------------
\item 
\benum
\item 
% a
In geometrized units, we have $G=1$ and $c=1$. To convert from kg to m, we multiply by a factor of $G/c^2$. In these units
\[
	M_{\odot} = 1485\text m,\qquad \omega = 2\pi/T = 2.9\times 10^{-6}/c = 9.67 \times 10^{-15}\text m^{-1}
\]
The angular momentum is then
\[
	J = I\omega = \frac25 M_{\odot}R_{\odot}^2 = 2.814 \times 10^{6}\text m{}^2
\]
\item
% (b)
\[
	a = J/M = 1894.95\text m
\]
\[
	a/M = 1.276 \ >1
\]
No black hole.
\item
% (c)
With the angular momentum of the electron arising solely from is quantum mechanical spin 1/2 angular momentum, we have
\[
	J = \h/2 = 2.612/2 \times 10^{-70} = 1.306 \times 10^{-70} \text m^2
\]
\[
	m_e = 6.764\times 10^{-58}\text m
\]
\[
	a/m_e = J/m_e^2 = 2.855 \times 10^{44} \ \gg 1.
\]
Need much larger mass or less angular momentum to become a black hole. 
\eenum
\eenum



\end{document}