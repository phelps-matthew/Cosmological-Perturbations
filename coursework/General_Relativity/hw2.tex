\documentclass[10pt,letterpaper]{article}
\usepackage{mymacros}

\title{General Relativity\\HW 2}
\author{Matthew Phelps}
\date{Due: September 29}

\begin{document}
\maketitle

\benum
% 1------------------------------------------------------------------------------------------------------------------------
\item
\be
	\pd_\alpha F^{\alpha \beta} = J^\beta
\ee
\be
	\pd_\alpha F_{\beta\gamma} + \pd_\beta F_{\gamma\alpha} + \pd_\gamma F_{\beta\alpha} = 0
\ee
\\ \\
\[
	J^\alpha = (\rho,J^i), \quad F^{i0} = E^{i0}, F^{ij} = -\epsilon^{ijk}B_k
\]
\[
	\pd_i F^{i0} = J^0 \to \boxed{ \del \cdot \vect E = \rho}
\]
\ba
	\pd_\alpha F^{\alpha i} &= \pdiff[t]F^{0i} + \pd_j F^{ji} = J^i\\
	\pdiff[F^{0i}]{t} +  \epsilon^{jik}\pd_jB_k &= J^i \to \boxed{-\pdiff[E^i]{t}+(\del\times \vect B)^i = J^i}
\ea
For $\alpha =1,\beta =2,\gamma =3$, (2) takes the form
\ba
	\pdiff{x}F_{23} + \pdiff{y}F_{31} + \pdiff{z} F_{21} &= 0\\
	\pdiff{x}B_x + \pdiff{y}B_y + \pdiff{z} B_z &= 0\\
	\to \boxed{ \del\cdot \vect B =0}.
\ea
For $\alpha =0,\beta =i,\gamma =j$, (2) takes the form
\ba
	\pd_0 F_{ij} + \pd_i F_{j0}j + \pd_jF_{0i} &= 0\\
	\pdiff{t}F_{ij} + \pd_i E_j + \pd_j E_i &=0
\ea
Now if we sum over all values of $i$ and $j$ (noting the antisymmetry) we have
\[
	\boxed{\pdiff{t}\vect B + \del\times \vect E = 0}
\]

% 2------------------------------------------------------------------------------------------------------------
\item 
\[
	x = r\cos\phi,\qquad y = r\sin\phi
\]
Line element:
\[
	dl^2 = dx^2 + dy^2 = dr^2 + r^2d\phi^2.
\]
Metric:
\[
	g_{\mu\nu} = \bpm 1 & 0\\0&r^2\epm.
\]
Connection:
\[
	\Gamma^{\sigma}_{\lambda\mu} = \frac12 g^{\nu\sigma}\blr{ \pd_\lambda g_{\nu\mu} + \pd_{\mu}g_{\lambda\nu} - \pd_\nu g_{\lambda\mu}}.
\]
Since $g_{\mu\nu}$ is diagonal, from the above we must have $g^{\nu\sigma} = g^{\sigma\sigma}$. Also, we see the only non-zero derivative of $g_{\mu\nu}$ is $\pd_0 g_{11} = 2r$. It follows that $\lambda=1$ and/or $\mu=1$. If either is $1$ then $\sigma = 1$, but if both $\mu=\lambda=1$, then $\sigma = 0$. So we are left with two (due to symmetry) we need to compute:
\[
	\Gamma^{0}_{11} = -\frac12 g^{00}(\pd_o g_{11}) = -r
\]
\[
	\Gamma^{1}_{01} = \frac12 g^{11}\pd_0 g_{11} = \frac1r
\]
Equation of motion:
\[
	\frac{d^2x^\lambda}{d\tau^2} + \Gamma^{\lambda}_{\mu\nu} \frac{dx^\mu}{d\tau}\frac{dx^\nu}{d\tau} = 0
\]
leads to
\[
	\frac{d^2r}{d\tau^2} -r\pfrac{d\phi}{d\tau}^2 = 0
\]
\[
	\frac{d^2\phi}{d\tau^2} + \frac2r \frac{d\phi}{d\tau}\frac{dr}{d\tau} = 0.
\]
Adding these together
\[
	\frac{d^2r}{d\tau^2}+\frac{d^2\phi}{d\tau^2}(1-r)+ \frac2r \frac{d\phi}{d\tau}\frac{dr}{d\tau}=0.
\]
% 3 -------------------------------------------------------------------------------------------------------------------------------------------------------
\item
\[
	x = r\cos\phi,\qquad y = r\sin\phi,\qquad r = \sqrt{x^2+y^2},\qquad \phi = \tan^{-1}\pfrac{y}{x}
\]
\[
	\pdiff[r]{x} = \cos\phi
\]
\[
	\pdiff[r]{y} = \sin\phi
\]
\[
	\pdiff[\phi]{x} = -\frac{1}{x^2}\pfrac{y}{1+(y/x)^2} = -\frac{y}{r^2} = -\frac{\sin\phi}{r}
\]
\[
	\pdiff[\phi]{y} = \frac{1}{x}\pfrac{1}{1+(y/x)^2} = \frac{x}{r} = \frac{\cos\phi}{r}
\]
\[
	\pdiff[x'^\mu]{x^\nu} = \bpm \cos\phi & \sin\phi \\ -\frac{\sin\phi}{r} & \frac{\cos\phi}{r} \epm.
\]
% 4 -------------------------------------------------------------------------------------------------------------------------------------------------------
\item
\benum 
% (a)
\item 
In cartesian coordinates, the connection vanishes and so
\ba
	V^\mu{}_{;\nu} &= \pd_\nu V^{\mu}.
\ea
\[
	\pd_0 V^0 = 2x
\]
\[
	\pd_0 V^1=3
\]
\[
	\pd_1 V^0=3
\]
\[
	\pd_1 V^1=2y.
\]
So in matrix form 
\[
	V^\mu{}_{;\nu} = \bpm 2x & 3 \\ 3 & 2y \epm
\]
%(b)
\item
Under a change from cartesian to polar coordinates,$x\to x'$, the mixed rank tensor transforms as
\[
	V'^{\mu}{}_{;\nu} = \pdiff[x'^\mu]{x^\rho}\pdiff[x^\sigma]{x'^\nu}V^{\rho}{}_{;\sigma}.
\]
In matrix form
\ba
	V'^\mu{}_{;\nu} &= \bpm \cos\phi & \sin\phi \\ -\frac{\sin\phi}{r} & \frac{\cos\phi}{r} \epm
	\bpm 2x & 3 \\ 3 & 2y \epm
	\bpm \cos\phi & -r\sin\phi \\ \sin\phi & r\cos\phi \epm\\
	&=  \bpm \cos\phi & \sin\phi \\ -\frac{\sin\phi}{r} & \frac{\cos\phi}{r} \epm
	\bpm (2x\cos\phi + 3\sin\phi)& (-2xr\sin\phi + 3r\cos\phi)\\(3\cos\phi+2y\sin\phi)&(-3r\sin\phi +2yr\cos\phi)\epm\\\
	&=\bpm  (2x\cos^2\phi + 3\sin\phi\cos\phi+3\sin\phi\cos\phi+2y\sin^2\phi)&(-2xr\sin\phi\cos\phi+3r\cos^2\phi-3r\sin^2\phi + 2yr\sin\phi\cos\phi)\\
	\frac1r (-2x\sin\phi\cos\phi - 3\sin^2\phi + 3\cos^2\phi + 2y\sin\phi\cos\phi)&
	\frac1r (2xr\sin^2\phi -3r\sin\phi\cos\phi -3r\sin\phi\cos\phi +2yr\cos^2\phi) \epm.
\ea
Since that got cutoff,
\ba
	V^0{}_{;0} &= 2x\cos^2\phi + 3\sin\phi\cos\phi+3\sin\phi\cos\phi+2y\sin^2\phi\\
	&=2r\cos^2\phi +6\sin\phi\cos\phi +2r\sin^2\phi
\ea
\ba
	V^0{}_{;1} &=-2xr\sin\phi\cos\phi+3r\cos^2\phi-3r\sin^2\phi + 2yr\sin\phi\cos\phi\\
	&=-2r^2\sin\phi\cos^2\phi + 3r\cos^2\phi -3r\sin^2\phi +2r^2\sin^2\phi\cos\phi
\ea
\ba
	V^1{}_{;0} &=\frac1r (-2x\sin\phi\cos\phi - 3\sin^2\phi + 3\cos^2\phi + 2y\sin\phi\cos\phi)\\
	&=-2\sin\phi\cos^2\phi + \frac3r\cos^2\phi -\frac3r\sin^2\phi +2\sin\phi\cos^2\phi
\ea
\ba
	V^1{}_{;1} &=\frac1r (2xr\sin^2\phi -3r\sin\phi\cos\phi -3r\sin\phi\cos\phi +2yr\cos^2\phi) \\
	&=2r\cos\phi\sin^2\phi - 6\sin\phi\cos\phi +2r\sin\phi\cos^2\phi
\ea
% c
\item
\[
	V^\mu{}_{;\nu} = \pd_\nu V^\mu + \Gamma^\mu_{\lambda\nu}V^\lambda
\]
From question 2, we have
\[
	\Gamma^0_{11} = -r,\qquad \Gamma^1_{01} = \frac1r.
\]
We must also convert $V^\mu$ to polar 
\[
	V^\mu = (r^2\cos^2\phi+3r\sin\phi, r^2\sin^2\phi +3r\cos\phi).
\]
Taking each appropriate derivative with respective Christoffel symbol (summing over $\lambda$), we find
\[
	V^0{}_{;0} = 2r\cos^2\phi +6\sin\phi\cos\phi +2r\sin^2\phi
\]
\[
	V^0{}_{;1} = =-2r^2\sin\phi\cos^2\phi + 3r\cos^2\phi -3r\sin^2\phi +2r^2\sin^2\phi\cos\phi
\]
\[
	V^1{}_{;0}=-2\sin\phi\cos^2\phi + \frac3r\cos^2\phi -\frac3r\sin^2\phi +2\sin\phi\cos^2\phi
\]
\[
	V^1{}_{;1}=2r\cos\phi\sin^2\phi - 6\sin\phi\cos\phi +2r\sin\phi\cos^2\phi,
\]
which coincides with what we calculated in part b.
\eenum
% 5------------------------------------------------------------------------------------
\item
\benum
\item
% (a)
In the inertial system, the proper time interval is given as (constrained to 1D motion)
\[
	d\tau^2 = dt^2 - dx^2.
\]
With the coordinate transformation parametrized by $\lambda$
\[
	t(\lambda) = a \sinh(\lambda),\qquad x(\lambda)= a\cosh(\lambda)
\]
it follows that
\[
	dt = a\cosh (\lambda) d\lambda,\qquad dx = a\sinh(\lambda) d\lambda
\]
and thus
\[
	d\tau^2 = a^2 d\lambda^2 [\cosh^2(\lambda) - \sinh^2(\lambda)] = a^2 d\lambda^2.
\]
For a finite proper time, taking $\tau_0 = 0$, we integrate
\[
	\tau = a\lambda.
\]
\item
% (b)
For an interial observer, a spacelike line can be expressed as
\[
	t = xb,\qquad -1<b<1.
\]
For the accelerated observer, his coordinates satisfy the relation
\[
	x^2 - t^2 = a^2
\]
for any fixed $\lambda$. Different values of $a$ will generate different hyperbolic curves in the $x-t$ plane. Solving for the point of intersection between these two worldlines,
\[
	x_0 = \frac{a}{\sqrt{1-b^2}},\qquad  t_0 = \frac{ab}{\sqrt{1-b^2}}.
\]
The two lines will be perpendicular if their derivatives are negative inverses. For the inertial observer,
\[
	\frac{dt}{dx} = b
\]
while for the accelerated observer
\[
	2x - 2t \frac{dt}{dx} = 0 \to \frac{dt}{dx} = \elr{\frac{x}{t}}_{t_0,x_0} = \frac1b.
\]
This would show they were orthogonal if the slopes were of opposite sign... \\
\item
% c
The coordinate and inverse coordinate transformation is
\[
	t = a\sinh \lambda,\qquad x= a\cosh \lambda
\]
\[
	\lambda = \tanh^{-1}\pfrac tx,\qquad a =\sqrt{x^2-t^2}.
\]
From our earlier equation 
\[
	x^2 -t^2 = a^2,
\]
we see that curves of constant $\lambda$ will yield a family of hyperbolas. Since $a^2 \ge 0$, these will be contrained to the $x\ge 0$ region, thus only covering half of the $x-t$ plane. Meanwhile, lines of constant $a$ take the form
\[
	t = \tanh(\lambda) x.
\]
For $-\infty < \lambda < \infty$, it follows that the above equation forms straight lines with slopes from $(-1,1)$. Note that these slopes specifically exclude $\{-1,1\}$ from the interval, and since they can only be reached in the limit of infinity, we consider these ``bad'' coordinates.
In total we will have something like this:
\figg[width=100mm]{hw2_5.pdf}
\item
% (d)
Recalling that
\[
	g_{\mu\nu} = \frac{\pd \xi^\alpha}{\pd x^\mu} \frac{ \pd\xi^\beta}{\pd x^\nu} \eta_{\alpha\beta},
\]
for our 2d case $(t,x) \to (\lambda,a)$ this becomes
\[
	g_{\mu\nu} =  -\frac{\pd t}{\pd x^\mu} \frac{ \pd t}{\pd x^\nu}+ \frac{\pd x}{\pd x^\mu} \frac{ \pd x}{\pd x^\nu}.
\]
Taking each component
\ba
	g_{00} &= -\pfrac{\pd t}{\pd \lambda}^2 + \pfrac{\pd x}{\pd \lambda}^2\\
	&= a^2(\sinh^2\lambda - \cosh^2\lambda) \\
	&= -a^2
\ea
\ba
	g_{01} = g_{10} &=   -\frac{\pd t}{\pd a} \frac{ \pd t}{\pd \lambda}+ \frac{\pd x}{\pd a} \frac{ \pd x}{\pd \lambda}\\
	& = -a\sinh\lambda\cosh\lambda + a\sinh\lambda \cosh\lambda\\
	&= 0
\ea
\ba
	g_{11} &= -\pfrac{\pd t}{\pd a}^2 + \pfrac{\pd x}{\pd a}^2\\
	&= -\sinh^2\lambda + \cosh^2\lambda \\
	&= 1.
\ea
Thus
\[
	g_{\mu\nu} = \bpm -a^2 & 0 \\ 0& 1 \epm.
\]
The Christoffel symbols are
\[
	\Gamma^0_{00} = \frac12 g^{00}\frac{\pd g_{00}}{\pd \lambda} = 0
\]
\[
	\Gamma^1_{11} = \frac12 g^{11}\frac{\pd g_{11}}{\pd a} = 0
\]
\[
	\Gamma^0_{01} = \frac12 g^{00} \frac{\pd g_{00}}{\pd a} = \frac1a
\]
\[
	\Gamma^1_{01} = \frac12 g^{11} \frac{\pd g_{10}}{\pd \lambda} = 0
\]
\[
	\Gamma^0_{11} = \frac12 g^{00}\blr{ 2\frac{\pd g_{01}}{\pd a} - \frac{\pd g_{11}}{\pd \lambda}} = 0
\]
\[
	\Gamma^1_{00} = \frac12 g^{11}\blr{ 2\frac{\pd g_{01}}{\pd \lambda} - \frac{\pd g_{00}}{\pd a}} = a.
\]
\eenum
% 6------------------------------------------------------------------------------------
\item
\[
	ds^2 = dx^2 + dy^2 + dz^2
\]
\[
	x = r\sin\theta\cos\phi,\qquad y = r\sin\theta\sin\phi,\qquad z=r\cos\theta
\]
\[
	dx = (\sin\theta\cos\phi)dr + (r\cos\theta\cos\phi)d\theta + (-r\sin\theta\sin\phi)d\phi
\]
\[
	dy = (\sin\theta\sin\phi)dr +(r\cos\theta\sin\phi)d\theta +(r\sin\theta\cos\phi)d\phi
\]
\[
	dz = \cos\theta dr + (-r\sin\theta)d\theta
\]
\ba
	dx^2+dy^2+dz^2 &= dr^2(\sin^2\theta+\cos^2\theta)+d\theta^2(r^2) + d\phi^2(r^2)\\
	& = dr^2+r^2d\theta^2 +r^2\sin^2\theta d\phi^2
\ea
Setting $dr =0$, we are left with only the angular part (at a fixed radius) $d\Omega^2$
\[
	g_{\mu\nu} = \bpm r^2 & 0 \\ 0 & r^2\sin^2\theta \epm.
\]
\eenum

\end{document}