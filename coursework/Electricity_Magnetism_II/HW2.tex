\documentclass[10pt,letterpaper]{article}
\usepackage{macroshw}

\title{\begin{spacing}{1.2}Electrodynamics II\\HW 2\end{spacing}}
\author{Matthew Phelps}
\date{Due: March 7}

\begin{document}
\maketitle

\benum

% 1------------------------------------------------------------------------------------
\item
A particle of charge $q$ and mass $m$ is moving in a uniform electric field $\vect E$. The vector of the initial
particle velocity $\vect v_0$ at the time $t=0$ is perpendicular to the electric field. Calculate as a function of time
the angular distribution and total energy of induced EM radiation.
\\ \\

% 2------------------------------------------------------------------------------------
\item
A particle of mass $m$ and charge $q$ is moving perpendicular to the uniform magnetic field $\vect B$.
The initial kinetic energy of this particle is $E_0$. Determine a time dependence of the particle energy
$E(t)$.
\\ \\
% 3------------------------------------------------------------------------------------
\item
Electron with the mass $m_e$ and charge $-e$ and proton with the mass $M_p$ and charge $+e$ attract
each other according to the Coulomb law and their relative motion is described by the elliptic trajectory:
\[	
	1+\varepsilon \cos\phi = \frac{a(1-\varepsilon^2)}{r}
\]
where $r$ is the radius of a relative motion and the semi-major axis $a$ and the eccentricity $\varepsilon$
are 
\[
	a= \frac{e^2}{2|E|},\quad \varepsilon = \sqrt{1-\frac{2|E|M^2}{\mu e^4}}.
\]
Here $E$ is the total energy of a relative motion $(E<0)$, $\mu$ is the reduced mass and $M$ is the
angular momentum. Calculate the power of EM radiation averaged over a full revolution in the ellipse. Answer
has to be presented as a function of $E$, $M$, $e$, $m_e$, and $M_p$. 
\\ \\

% 4------------------------------------------------------------------------------------
\item
Determine the differential effective cross-section for scattering of the linearly polarized EM wave by a charged
harmonic oscillator with the frequency of vibration $\omega_0$. The oscillator mass and charge 
are $q$ and $m$ respectively.
\\ \\

% 5------------------------------------------------------------------------------------
\item
Determine the differential total effective cross-section for scattering of EM waves by an electric dipole $\vect d$
which, mechanically, is a rigid rotor with the moment of inertia $J$. The frequency $\omega$ of the EM wave is
significantly larger than the frequency of a free rotation of the rotor and only the forced rotation under the 
action of the force $\vect d\times\vect E$ can be considered.

\eenum 


\end{document}