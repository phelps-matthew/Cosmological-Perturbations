\documentclass[10pt,letterpaper]{article}
\usepackage{macroshw}

\title{\begin{spacing}{1.2}Electrodynamics II\\HW 3\end{spacing}}
\author{Matthew Phelps}
\date{Due: April 6}

\begin{document}
\maketitle

\benum

% 1------------------------------------------------------------------------------------
\item
A particle of mass $m_1$ and energy $E_0$ collides elastically with a particle of mass $m_2$. The 
particle 2 was at rest before the collision in the Laboratory Frame (LF). The particle energies after collision
are $E_1$ and $E_2$. Calculate the LF scattering angles $\theta_1$ and $\theta_2$
\\ \\
In an elastic collision, both conservation and energy can be nicely expressed in terms of the invariant 4-vector
quantity
\[
	p_1^i +p_2^i = p_1^{i'} + p_2^{i'}.
\]
Now we may arrive at different forms of this invariant quantity by contracting it with a covariant vector. Contracting with $p_{1i}$ we have
\be\label{1}
	m_1^2 + p_{1i}p_2^i-p_{1i}p_1^{i'}-p_{1i}p_2^{i'} = 0
\ee
and contracting with $p_{2i}$ we have
\be\label{2}
	m_2^2 +p_{2i}p_1^i - p_{2i}p_1^{i'}-p_{2i}p_2^{i'} = 0.
\ee
Note the mass terms come from $p_ip^i = E^2/c^2 - \vect p\cdot\vect p = m^2$. Now, taking $m_2$ at rest,
the scalar products from \eqref 1 are
\[
	p_{1i}p_2^i = m_2E_0
\]
\[
	p_{1i}p_1^{i'} = E_0E_1 - \vect p_1\cdot\vect p_1' = E_0E_1-p_1p_1'\cos\theta_1
\]
\[
	p_{1i}p_2^{i'} = E_0E_2-p_1p_2'\cos\theta_2
\]

and the scalar products from \eqref 2 are 
\[
	p_{2i}p_1^i = m_2E_0
\]
\[
	 p_{2i}p_2^{i'} = m_2E_2
\]
\[
	p_{2i}p_1^{i'} = m_2E_1 .
\]
These form two equations
\[
	m_1^2+m_2E_0-E_0E_1+p_1p_1'\cos\theta_1 -E_0E_2+p_1p_2'\cos\theta_2= 0
\]
\[
	m_2(m_2+E_0-E_1-E_2) = 0
\]
I don't understand how Landau gets a $(-p_{2i}p_1^{i'})$ in equation \eqref 1  (13.2 in text).. 
\\ \\
I suppose I'll have to proceed differently.
\\ \\
Conservation laws
\[
	\vect p_1 = \vect p_2' +\vect p_1'
\]
\[
	E_0 +m_2c^2 = E_1+E_2.
\]
Now square these
\be\label{3}
	p_2'^2 = p_1^2+p_1'^2 -2p_1p_1'\cos\theta_1
\ee
\be\label{4}
	E_2^2 = E_0^2+E_1^2+m_2^2c^4-2E_0E_1 +2m_2c^2(E_0-E_1)
\ee
Subtract the difference of these squares
\[	
	(4)^2-c^2(3)^2 = E_2^2-c^2p_2'^2 = m_2^2c^4.
\]
Expand the left hand side
\[
	E_0^2+E_1^2+m_2^2c^5-2E_0E_1+2m_2c^2(E_0-E_1)-c^2p_1^2-c^2p_1'^2 +2p_1p_1'\cos\theta_1 = m_2^2c^4
\]	
which reduces to
\[
	2p_1p_1'\cos\theta_1 = -m_1^2c^4-m_1^2c^4+2E_0E_1-2m_2c^2(E_0-E_1).
\]
Solving for the angle
\[
	\cos\theta_1 = \frac{E_0E_1-m_2c^2(E_0-E_1)-m_1c^4}{p_1p_1'}.
\]
Lastly we may rid $p_1$ and $p_1'$ via the relations $p_i^2 = E_i^2/c^2 - m_i^2c^2$
to arrive at
\[
	\cos\theta_1 = \frac{(E_0+m_2c^2)(E_1-m_2c^2)+c^4(m_2^2-m_1^2)}{c^2[(E_0^2/c^2-m_1^2c^2)
	(E_1^2/c^2-m_1^2c^2)]^{1/2}}.
\]
Now we repeat the similar process for $\cos\theta_2$ but with $E_1^2$ on the LHS.
\[
	m_1c^4 = E_0^2+E_2^2+m_2^2c^4-2E_0E_2+2E_0m_2c^2-2E_2m_2c^2-p_1^2-p_2'^2+
	2p_1p_2'\cos\theta_2.
\]
Solving for $\cos\theta_2$ in its appropriate form, we have
\[
	\cos\theta_2 = \frac{(E_0+m_2c^2)(E_2-m_2c^2)}{c^2[(E_0^2/c^2-m_1^2c^2)(E_2^2/c^2 - m_2^2c^2)]}.
\]
\\ \\
% 2------------------------------------------------------------------------------------
\item
A disk of the radius $R_0$ is moving with the relativistic velocity $\vect v$ in a cold molecular gas. The molecular
rest mass is $m_0$ and the gas density is $n_0$. Find the pressure acting on the disk.
\\ \\
Take the disc to travel in the $x$-direction, $\vect v =v\vecth x$. In the rest frame of the disk, each gas particle collides and rebounds and the change in momentum is $2p_i$ or
\[
	p = \frac{2m_0v}{\sqrt{1-v^2/c^2}}.
\]
Given an area $A = \pi R_0^2$ the rate of collisions per unit time is
\[
	Z = vAn_0.
\]
The force on the disc can then be calculated by
\[
	F = Zp = 2m_0v\gamma v\pi R_0^2n_0
\]
where $\gamma = (1-v^2/c^2)^{-1/2}$. Now taking pressure as force per unit area, we then have the pressure
\[
	P = 2m\gamma v^2n_0.
\]
To be more general, we may give the velocity of the disk as measured by an observer moving with velocity $v'$
relative to the gas. This is given by the velocity addition 
\[
	v'' = \frac{v'+v}{1+\frac{vv'}{c^2}}.
\]
In this frame of reference $p = 2mv''\gamma$ and
\[
	P = 2m_0 \gamma v''^2 n_0.
\]
% 3------------------------------------------------------------------------------------
\item
The velocity vector of a particle is $\vect v'$ in the $S'$ frame. Calculate the absolute value of the velocity 
$|\vect v|$ of this particle in the frame $S$, moving with the velocity $\vect V$ with respect to the $S'$-frame. The 
answer has to include only given velocity vectors and the speed of light $c$. 
\\ \\
Let's assume that system $S$ moves relative to $S'$ with velocity $V$ along its $x$-axis. Then by 
the transformation of velocities we have each component of velocity in the $S$ frame
\[
	v_x  = \frac{v_x+V}{1+v_x'\frac{V}{c^2}}
\]
\[
	v_y = \frac{v_y'}{\gamma(1+v_x'\frac{V}{c^2})}
\]
\[
	v_z = \frac{v_z'}{\gamma(1+v_x'\frac{V}{c^2})}
\]
where $\gamma = (1-V^2/c^2)^{-1/2}$. Now find the magnitude via $v_x^2+v_y^2+v_z^2$. After some lengthy
algebra
\[
	v^2 = \frac{c^4(v_x'^2+v_y'^2+v_z'^2)+2v_x'Vc^4+V^2c^2(c^2-v_y'^2-v_z'^2)}{(c^2+v_x'V)^2}
\]
We may put this in a more general form by noting that (since taken in the $x$ direction)
\[
	v_x'V = \vect v'\cdot \vect V,
\]
\[
	(\vect v'\times \vect V)^2 = (\vect v'\cdot\vect v')(\vect V\cdot\vect V)-(\vect v'\cdot\vect V)(\vect V\cdot
	\vect v') = v'^2V^2-v_x'^2V^2 = V^2(v_y'^2+v_z'^2),
\]
and
\[
	(\vect v'+\vect V)^2 = v'^2+2v_x'V+V^2.
\]
Now, after algebraic manipulation, we may simplify the $v^2$ found earlier using these relations
\[
	v = \frac{1}{1+\frac{\vect v'\cdot\vect V}{c^2}}\blr{(\vect v'+\vect V)^2-(\vect V\times\vect v')^2}^{1/2}.
\]
\\ \\
% 4------------------------------------------------------------------------------------
\item
The para-positronium (a bound state of electron and positron) moves with a constant velocity $\vect V$ with respect
to the Laboratory Frame and decays with an emission of two photons. In the positronium rest frame, the angular
distribution of photons is isotropic.
\benum
\item
Calculate an angular distribution of photons in the Laboratory Frame.
\item
Show that the single-photon decay of a positronium is not allowed by the energy-momentum conservation. 
\\ \\
\eenum
\benum
\item
In the center of mass frame, we may describe the number of emitted photons detected 
in differential solid angle $d\Omega_0$ simply as the ratio over total solid angle, i.e.
\[
	dN = \frac{d\Omega_0}{4\pi}.
\]
This may also be expressed as 
\[
	dN = \frac{1}{2} |d(\cos\theta_0)|.
\]
To find the angular distribution in a different frame of reference, we must use the transformation of angles between these two coordinate systems
\[
	\cos\theta_0 = \frac{\cos\theta -\frac{v}{c}}{1-\frac{v}{c}\cos\theta}.
\]
We will specifically transform to the lab frame. Denoting $\beta = v/c$ we have
\ba
	d\cos\theta_0 &= \diff{\theta}\pfrac{\cos\theta-\beta}{1-\beta\cos\theta}\\
	&= \diff{\theta}\plr{\frac{-\sin\theta}{1-\beta\cos\theta}+\frac{\beta\sin\theta\cos\theta}{(1-\beta\cos\theta)^2}
	-\frac{\beta^2\sin\theta}{(1-\beta\cos\theta)^2}}d\theta\\
	&= \plr{\frac{\sin\theta-\beta\sin\theta\cos\theta+\beta\sin\theta\cos\theta-\beta^2\sin\theta}{
	(1-\beta\cos\theta)^2}}d\theta\\
	&= \frac{(1-\beta)^2\sin\theta}{(1-\beta\cos\theta)^2}d\theta
\ea
Now reutilizing the equation for particle number, but this time in the lab frame we have
\[
	dN = \frac{1}{2}|d(\cos\theta_0)|  = \frac{(1-\beta)^2\sin\theta}{2(1-\beta\cos\theta)^2} = 
	\frac{(1-\beta)^2\sin\theta}{4\pi(1-\beta\cos\theta)^2}d\Omega
\]
where $d\Omega = 2\pi \sin\theta\ d\theta$. 
\\ \\
\item
If we view the decay of a single photon in the center of mass frame, we have from the conservation of momentum
and energy
\[
	E_1=E_2\Rightarrow m_1c^2 = p_2c,\qquad p_1 = p_2.
\]
Index 1 denotes positronium and 2 denotes the photon.
But, in the c.o.m frame, $p_1 = 0$ and thus $p_2 = 0$. We have a violation because a photon cannot
have zero momentum in any frame of reference, i.e. it is massless and moves with speed $c$ in any frame.
We may continue on to see that its energy is also not conserved, because if it were to somehow have a zero momentum, then
\[
	E_2 = p_2c = \h\omega =0\ne E_1 = mc^2 >0.
\]

\eenum
% 5------------------------------------------------------------------------------------
\item
A mirror is moving with a velocity $V$ in the Laboratory Frame (LF). Determine the light reflection law
(the relation between incident and reflection angles), if the velocity vector is 
\benum
\item
Perpendicular to the mirror plane;
\item
Parallel to the mirror plane \\ \\
\eenum
\benum
\item
Let's orient our apparatus such that the mirror lies along the $\vecth y$ axis and the light moves along the $\vecth x$ axis. Denote $\alpha$ as the angle of incidence relative to the $\vecth x$ axis and $\theta$ as the 
angle of reflection relative to the $\vecth x$ axis. Lastly let $\vect c_i$ and $\vect c_f$ denote the initial
and final velocity vectors of light, respectively. \\ \\
Ok, in the lab frame 
\[
	c_i = c\cos\alpha \vecth x+c\sin\alpha\vecth y.
\]
Meanwhile we may transform this into the mirror frame via
\[
	c_{ix}' = \frac{c_{ix}+V}{1+\frac{c_{ix}V}{c^2}}
\]
\[
	c_{iy}' = \frac{c_{iy}\sqrt{1-\frac{V^2}{c^2}}}{1+\frac{c_{ix}V}{c^2}}.
\]
In the mirror frame, we have from conservation of momentum
\[
	c_{ix}' = -c_{fx}',\qquad c_{iy}'=c_{fy}'.
\]
Now, convert the $y$ component to the lab frame
\[
	c_{fy} = \frac{c_{fy}'\sqrt{1-V^2/c^2}}{1-\frac{c_{fx}'V}{c^2}} = c\sin\theta
\]
and relate these to each other via
\[
	c_{fy}' = c_{iy}' = \frac{c_{iy}\sqrt{1-V^2/c^2}}{1+\frac{c_{ix}V}{c^2}}.
\]
Now use the form with $\sin\theta$ to arrive at
\ba
	c\sin\theta &= \frac{c_{iy}\sqrt{1-V^2/c^2}}{1+\frac{c_{ix}V}{c^2}}\pfrac{
	\sqrt{1-V^2/c^2}}{1-\frac{c_{fx}'V}{c^2}}\\
	&= \frac{c_{iy}(1-V^2/c^2)}{\plr{1+\frac{c_{ix}V}{c^2}}\plr{1+\frac{c_{ix}'V}{c^2}}}.
\ea
Now use the relation
\[
	1+\frac{c_{ix}'V}{c^2} = 1+\frac{V}{c^2}\pfrac{c_{ix}+V}{1+\frac{c_{ix}V}{c^2}} =
	\pfrac{\beta\cos\alpha+\beta^2}{1+\beta\cos\alpha}
\]
to express $c\sin\theta$ as
\[
	c\sin\theta = \frac{c\sin\alpha(1-\beta^2)}{1+2\beta\cos\alpha+\beta^2}.
\]
Thus our reflection law in the lab frame is
\[
	\sin\theta = \frac{\sin\alpha(1-\beta^2)}{1+2\beta\cos\alpha+\beta^2}.
\]
\\ \\
\item
Now if the light is moving parallel to the mirror in the $\vecth y$ direction, we have the transformations
\[
	c_{ix}' =  \frac{c_{ix}\sqrt{1+V^2/c^2}}{1+\frac{c_{iy}V}{c^2}}
\]
\[
	c_{iy}' = \frac{c_{iy}+V}{1+\frac{c_{iy}V}{c^2}}.
\]
Our conservation of momentum then gives us
\[
	c_{ix}' = -c_{fx}',\qquad c_{iy}' = c_{fy}'.
\]
Bringing these together with the transformation equations
\[
	c_{fy} = \frac{-c_{fy}'+V}{1-\frac{c_{fy}'V}{c^2}} = \frac{-c_{iy}'+V}{1-\frac{c_{iy}'V}{c^2}}.
\]
Following the same procedure as before we have
\[
	c\sin\theta = \frac{-\pfrac{c\sin\alpha+V}{1+\beta\sin\alpha}+V}{1-\pfrac{\beta\sin\alpha+\beta^2}{1+
	\beta\sin\alpha}} = \frac{-c\sin\alpha+V\beta\sin\alpha}{1-\beta^2} = -c\sin\alpha
\]
Thus we finally have the reflection law in the lab frame
\[
	\sin\theta = -\sin\alpha.
\]
\eenum
\eenum 
\end{document}