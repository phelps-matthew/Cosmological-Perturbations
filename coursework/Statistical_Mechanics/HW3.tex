\documentclass[11pt,letterpaper]{article}
\usepackage{macroshw}

\title{\begin{spacing}{1.2}Statistical Mechanics\\HW 3\end{spacing}}
\author{Matthew Phelps}
\date{Due: Sept. 23}

\begin{document}
\maketitle

\benum
% 3.2  --------------------------------------------------------------------------------------------------------------------------------------------------------------------------------------
  	\item[\textbf{3.2}]
	\benum
		% (a)
		\item
		Suppose that in a gas-liquid phase transition the latent heat is a constant, that the volume of the liquid is negligible compared 
		to the volume of the gas, and that the gas may be assumed ideal. Find the liquid-gas coexistence curve.
		\\
		\\
		From the Clausius-Clapeyron equation we have
		\[
			\diff[p]{T} = \frac{\Delta q}{T\Delta v}
		\]
		where $\Delta q = T(s_{II}-s_I)$ is the latent heat and $\Delta v = v_{II}-v_I$ is the change in volume per particle under a phase
		transition from gas ($I$) to liquid ($II$). Assuming the latent heat is constant, we set
		\[	
			\Delta q = -C.
		\]
		The negative sign arises due to heat being released during the transition. 
		Further, if we assume the liquid volume is negligible in comparison to the gas volume $V_{II} \ll V_I$ we then have
		\[
			\Delta v = \frac{V_{II}}{N}-\frac{V_I}{N} \approx -\frac{V_I}{N}.
		\]
		Inserting these results into the C.C. equation
		\[
			\diff[p]{T} = \frac{CN}{TV_I}
		\]
		Since $V_I$ is the volume of the gas, which is assumed to be ideal, we may use the equation of state
		\[
			V_I = \frac{NkT}{p}
		\]
		and insert this into the C.C. equation to arrive at
		\[
			\diff[p]{T} = \frac{Cp}{kT^2}.
		\]
		Denoting the constant $\gamma \equiv C/k$, we see the relation
		\[
			\frac{dp}{p} = \gamma \frac{dT}{T^2}.
		\]
		Integrating both sides,
		\[
			\ln p = -\frac{\gamma}{T}+C_1
		\]
		or
		\[
			p(T) = C_2\exp\blr{-\frac{\gamma}{T}}. 
		\]
		\\
		% (b)
		\item
		Find vapor pressure data for water as a function of temperature from, e.g, the \emph{CRC Handbook of Physics and Chemistry} 
		(or some decent internet site). How good is the function form found in part (a)?
		\\
		\\
		Here is a graph of the vapor pressure vs. temperature for water taken from Wikipedia. For (most) values $T<\gamma$, our result
		 has the same exponential behavior as in the graph. It appears the critical point lies below $T=\gamma$. 
		\figg[width=150mm]{hw3_2.png}
		\eenum
		
		\phantom{}
		\phantom{}
		
	
	
% 3.3 ----------------------------------------------------------------------------------------------------------------------------------------------------------------------------------
	\item[\textbf{3.3}]
	When a gas is subject to an external potential energy (per particle) $\Phi(\vect r)$, it may happen that the potential
	varies little over the length scale in which the gas can find an equilibrium, and so the gas may be taken to behave
	locally as if there were no confining potential at all. The gas simply has some equilibrium properties that vary as a 
	function of position.
	
	\benum
		% (a)
		\item
		Sketch a local equilibrium argument showing that at fixed temperature the equilibrium condition is
		\[
			\mu(\vect r) +\Phi(\vect r) = const.
		\]
		Hint: The external potential creates forces on the molecules, so this is really a question of mechanical equilibrium
		between the forces from pressure and the form of the potential. 
		\\
		\\
		For a system at fixed temperature constrained to a volume with a set number of particles, the Helmholtz free energy
		$F$ will be at a minimum at thermodynamic equilibrium. If the volume is partitioned into two chambers $V_1$ and $V_2$
		then the condition for equilibrium becomes
		\[
			\mu_1 = \mu_2. 
		\]
		For $\mu_1\ne \mu_2$, there will be diffusion of particles across the barrier. However, one can establish equilibrium by
		applying an external potential equal to the difference between chemical potentials
		\[
			\mu_2-\mu_1 = -(\Phi_2-\Phi_1)
		\]
		or
		\[	
			\mu_2+\Phi_2 = \mu_1+\Phi_1.
		\]
		If we now take the chemical and external potentials as functions of position, then the condition for 
		local equilibrium may be stated as
		\[
			\mu(\vect r)+\Phi(\vect r) = C
		\]
		for all $\vect r$ in the local neighborhood of some point $\vect r_0$, with $C$ being a constant. As such
		\ba
			&\bigg( \mu(\vect r_0 +\epsilon\vecth r)+\Phi(\vect r_0+\epsilon\vecth r)\bigg)  - \bigg(\mu(\vect r_0)+\Phi(\vect r_0) \bigg)\\
			&= C-C \\
			&= 0
		\ea
		for $-a<\epsilon<a$ where $a$ is some parameter defining locality. Beyond the local range, the total potential of the gas will take 
		on a different constant value such that the equilibrium properties will vary overall as function of position. 
		\\
		\\
		% (b)
		\item
		Suppose the atmosphere is at constant temperature $T$, and is made out of molecules of mass $m$. How does
		the pressure vary as a function of height? (Expression for the ideal gas chemical potential was obtained in lectures,
		or in the previous exercise set).
		\\
		\\
		For the atmosphere to be in local equilibrium at constant temperature, we impose the equilibrium condition of part (a) namely
		\[
			\mu+mgz = C
		\]
		where $z$ denotes the altitude. Note that in the absence of gravity, the chemical potential of the atmosphere is independent
		of $z$. For an ideal gas, the chemical potential can be derived starting from
		\[
			\mu = \frac{G}{N} = \frac{1}{N}\plr{U-TS+PV}.
		\]
		The internal energy for an ideal gas is
		\[
			U = xNkT+C_1
		\]
		where $x=3/2$ or $x=5/2$ for a monatomic or diatomic gas respectively. For the ideal gas entropy we have
		\[
			S = Nk(\ln V+x\ln T)+C_2.
		\]
		For reasons I'm not yet sure of, we will take 
		\[
			C_2 = Nk\ln N
		\] 
		such that
		\ba
			S &= Nk\ln\pfrac{VT^x}{N} \\
			& = Nk\ln\pfrac{kT^{x+1}}{P}
		\ea
		Substituting these results into $\mu$ and using the ideal gas law we arrive at
		\[
			\mu = kT\blr{(x+1)+\ln\pfrac{p}{kT^{x+1}}}+C_1.
		\]
		Now imposing the equilibrium condition and solving for $P$ we have
		\[
			P(z) = kT^{x+1}\exp\blr{\frac{C_3}{kT}-(x+1)}\exp\blr{-\frac{mgz}{kT}}.
		\]
		Since $T$ and $x$ are constants, we may write this in a more suggestive form
		\[
			P(z) = P(0)\exp\blr{-\frac{z}{\zeta}}
		\]
		where $\zeta = kT/mg$ is the exponential decay constant. 
		\\
		\\
	\eenum


% 3.4 ---------------------------------------------------------------------------------------------------------------------------------------------------------------------
	\item[\textbf{3.4}]
	How hot does the air get in the pump when you try to put twice the atmospheric pressure in a bicycle tire at the ambient
	temperature of $20^\circ$C? Hint: Assume that the compression is reversible (is that a good assumption for a usual hand
	bicycle pump?) and takes place without exchange of heat with the environment (and this assumption?). Air is a diatomic gas.
	\\
	\\
	\\
	That the compression is reversible seems reasonable considering the slow rate at which pressure is changed from the limited 
	compression volume of the hand pump. If the hand pump is plastic, it may be reasonably insulated and thus exchange little
	heat with the environment as well. 
	\\
	\\
	As a reversible process $dQ = TdS$ and since no heat is exchanged $dS=0$. Alternatively, at equilbirium entropy is at a maximum. If 
	equilbrium is maintained under a transformation, the entropy cannot increase, thus $dS=0$. The change in entropy as the piston in
	the pump compresses the air is
	\[
		\Delta S = Nk\blr{\ln \frac{V}{V_0}+x\ln \frac{T}{T_0}} = 0.
	\]	
	If we assume the air is an ideal diatomic gas, we may express this as
	\[
		\ln\pfrac{P_0}{P} +\frac{5}{2} \ln \pfrac{T}{T_0} = 0
	\]
	or
	\ba
		T&= T_0\pfrac{P}{P_0}^{2/5}\\
		& = (273.15+20)\plr{2^{2/5}}\\
		& = 386.814\ \text K\\
		& = 113.66^\circ\text C
	\ea
	The air in the pump increases to a temperature of $113.66^\circ$C? 
	\\
	\\
	\\
% 3.5 ---------------------------------------------------------------------------------------------------------------------------------------------------------------------
	\item[\textbf{3.5}]
	Liquid quartz, if cooled slowly, crystallizes at a temperature $T_m$ and releases latent heat $\Delta q$. Under more rapid cooling 
	conditions the liquid is supercooled and becomes glassy.
	
	\benum
		% (a)
		\item
		As both phases of quartz are almost incompressible, there is no work input, and changes in internal energy satisfy 
		$dE = TdS+\mu dN$. Use the extensivity connection to obtain the expression for $\mu$ in terms of $E,T,S$ and $N$. 
		\\
		\\
		Starting with the differential form of $E$
		\[
			dE = TdS+\mu dN,
		\]
		we may construct a potential $F(T,N)$ by the Legendre transformation
		\[
			F = E-TS
		\]
		such that
		\[
			dF = -SdT+\mu dN.
		\]
		Being an extensive variable, the potential satisfies
		\[
			F(T,\lambda N) = \lambda F(T,N).
		\]
		Setting $\lambda = 1/N$, we have
		\[
			f(T) = \frac{1}{N} F(T,N).
		\]
		If we combine this with
		\[
			\plr{\pdiff[F]{N}}_T = \mu 
		\]
		we see that
		\[
			\mu = f(T) = \frac{1}{N}F(T,N)
		\]
		or
		\[
			\mu = \frac{1}{N}\plr{E-TS}.
		\]
		\\
		% (b)
		\item 
		The heat capacity of crystalline quartz is approximately $C_X = \alpha T^3$, while that of glassy quartz is roughly
		$C_G = \beta T$, where $\alpha$ and $\beta$ are constants.
		\\
		\\
		Assuming that the third law of thermodynamics applies to both crystalline and glass phases, calculate the entropies of the two
		phases at temperatures $T\le T_m$.
		\\
		\\
		For a reversible process
		\[
			dS = \frac{\delta q}{T}.
		\]
		and the heat capacity is
		\[
			C_i = \frac{\delta q}{dT},
		\]
		thus
		\[
			dS = \frac{C_i}{T}dT
		\]
		or
		\[
			S(T,i) = S(0,i) +\int_0^{T} dT'\ \frac{C_i}{T'}
		\]
		where $i$ is variable held constant. Since no work is done by this incompressible system, we need not bother with $i$. 
		For crystalline quartz,
		\[
			S(T) = S(0) + \alpha \frac{T^3}{3} 
		\]
		while for glassy quartz
		\[
			S(T) = S(0)+\beta T.
		\]
		Imposing the third law, that $S\to 0$ as $T\to 0$, we see that $S(0) =0$ for either phase. Thus
		\[
			S_X(T) = \alpha \frac{T^3}{3};\quad
			S_G(T) = \beta T
		\]
		\\
		% (c)
		\item
		At zero temperature, the local bonding structure is similar in glass and crystalline quartz, so that they have approximately
		the same internal energy $E_0$. Calculate the internal energies of both phases at temperatures $T\le T_m$. 
		\\
		\\
		From the Legendre transformation defined earlier, we may express the internal energy as
		\[
			E = TS+\mu N. 
		\]
		Inserting the appropriate entropies, we have
		\[
			E_X = \alpha \frac{T^4}{3}+\mu_X N
		\]
		\[
			E_G = \beta T^2 + \mu_G N.
		\]
		If the chemical potentials (bonding structure) are similar at $T=0$, we see that the energies are approximately equal 
		at $T=0$. If we keep $N$ fixed from $T=0 \to T_m$ and vary the temperature (entropy is a function of temperature), 
		we may write the energies as 
		\[
			E_X = \alpha \frac{T^4}{3}+E_0
		\]
		\[
			E_G = \beta T^2 + E_0.
		\]
		Also note that $N$ must be approximately the same for both phases to have equal energies at $T=0$. 
		\\
		\\
		% (d)
		\item
		Use the condition of thermal equilibrium between two phases to compute the equilibrium melting temperature $T_m$
		in terms of $\alpha$ and $\beta$. 
		\\
		\\
		At $T=T_m$ both phases may exist in equilibrium. At this constant temperature, our potential $F(T,N)$ defined earlier
		will be at a minimum for each phase individually. If we allow particles to move between phases, then 
		$dN_X = -dN_G = dN$ so that
		\[
			dF = dF_X+dF_G  = \blr{\plr{\pdiff[F_X]{N}}_T-\plr{\pdiff[F_G]{N}}_T}dN = 0.
		\]
		Therefore our equilibrium condition is
		\[
			\mu_X = \mu_G.
		\]
		Expressing $\mu$ as in part (a) we have 
		\[
			\frac{E_X-TS_X	}{N_X} = \frac{E_G-TS_G}{N_G}.
		\]	
		If we start with an equal number of $N$ particles before reaching melting temperature, then $N_X=N_G$ and 
		we can use the energy in its $E_0$ form 
		\[
			 \alpha \frac{T^4}{3}+E_0-\alpha \frac{T^4}{3} = \beta T^2 + E_0 -\beta T^2.
		\]
		This quantity seems to define a coexistence curve that holds for all $T$. To find the melting point, I guess
		we need to find the temperature at which energies are equal? 
		\\
		\\
		 \emph{How is one supposed to calculate the critical point?}
		 \\
		
		Using
		\[
			E_X(T) = E_G(T)
		\]
		we have for the melting point
		\[
			T^2_m = \frac{3\beta}{\alpha}.
		\]
		% (e)
		\item
		Compute the latent heat $\Delta q$ in terms of $\alpha$ and $\beta$.
		\\
		\\
		Latent heat may be calculated as
		\[
			\Delta q = T\plr{\frac{S_X}{N_X}-\frac{S_G}{N_G}}
		\]
		Once again assuming $N_X = N_G=N/2$ we have
		\[
			\Delta q = 0.
		\]
		This is the result for the phase transition between glassy and crystalline. However, I thought we are to find the 
		latent heat in transitioning from liquid to solid, which would require the entropy of liquid quartz?
		\\
		% (f)
		\item 
		Is the result in the previous part correct? If not, which of the steps leading to it is most likely to be incorrect?
		\\
		\\
		Part (e) suggests there is no heat required to transition in phase between glassy
		and crystalline quartz at the melting point. I am having trouble judging the validity of the results
		since they hinge upon the melting point of part (d). 
		
	\eenum
	 
\eenum
\end{document}