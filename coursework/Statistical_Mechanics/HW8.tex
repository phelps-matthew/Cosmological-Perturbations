\documentclass[11pt,letterpaper]{article}
\usepackage{macroshw}

\title{\begin{spacing}{1.2}Statistical Mechanics\\HW 8\end{spacing}}
\author{Matthew Phelps}
\date{Due: Nov. 2}

\begin{document}
\maketitle

\benum
% 8.1 ----------------------------------------------------------------------------------------------------------------------
  	\item[\textbf{8.1}]
	Find the chemical potential $\mu = \mu(n,T)$ and the entropy $S=Ns(n,T)$ of an ideal
	Fermi gas in the limit of a classical ideal gas (density $n\to 0$ and/or $T\to \infty$). There are
	situations in which the absolute value of entropy matters; this is one way getting it right.
	\emph{Hint: use the explicit form for $\mu$ if $z=z_0=n\lambda_{dB}^3\ll 1$ and keep only
	the largest order for $\mu$. Then calculate the form of $S$ via $U-TS+pV = G=\mu N$.} 
	\\
	\\
	For a Fermi gas the particle density is
	\[
		n = \frac{g}{\lambda^3}f_{3/2}(z)
	\]
	with
	\[
		z = 	e^{\beta\mu}.
	\]
	The density can be arranged and expanded as
	\[
		\frac{n\lambda^3}{g} \equiv z_0 =  f_{3/2}(z) = z- \frac{z^2}{2\sqrt 2}+\frac{z^3}{3\sqrt 3}+...
	\]
	We can invert the power series and instead express $z$ in term of $z_0$
	\[
		z = z_0+\frac{1}{2\sqrt 2}z_0^2+\frac{9-4\sqrt 3}{36}z_0^3+....
	\]
	In this form I think it is easier to see that for $z_0\ll 1$
	\[
		z\simeq z_0.
	\]
	Thus in the classical limit
	\[
		\mu = kT\ln\plr{\frac{n\lambda^3}{g}}.
	\]
	In much the same way, the energy may be calculated as
	\[
		U = \frac{3}{2}kTN\blr{1+\frac{1}{4\sqrt 2}z_0+...}
	\]
	with the classical limit of the energy being just the first term.
	\\
	\\
	From thermodynamics we know
	\[
		U = TS-pV+N\mu 
	\]
	so
	\ba
		s &= \frac{1}{T}\plr{\frac{U}{N}+\frac{p}{n}-\mu}\\
		& = k\blr{\frac{3}{2}-\ln\plr{\frac{n\lambda^3}{g}}}+\frac{p}{nT}
	\ea
	\[
		S = Ns(n,T) = N k\blr{\frac{3}{2}-\ln\plr{\frac{n\lambda^3}{g}}}+\frac{pV}{T}
	\]
	\\
% 8.2 -----------------------------------------------------------------------------------------------------------------------
	\item[\textbf{8.2}]
	For an isotropic 3D harmonic oscillator with angular frequency $\omega$ the energy levels are
	characterized by a triplet of non-negative integers $\vect n = (n_1,n_2,n_3)$, the energy
	being $\epsilon_n = \h\omega\plr{n_1+n_2+n_3+\frac{3}{2}}$. In this problem we always assume
	that the harmonic oscillator quantum $\h\omega$ is the smallest energy scale in the problem,
	so that the integers $n_i$ are large and anything that depends on these integers varies little
	with $n_i\to n_i+1$.
	\benum
		% (a)
		\item
		Argue that in this case the continuum approximation reads
		\[
			\sum_{\vect n} f(\vect n) \to \int d^3n\ f(\vect n)
		\]
		where the integral runs over the first octant (all Cartesian coordinates positive) of the 
		3D space.
		\\
		\\
		The distance between the neighboring quantized points is just $1$, so our lattice constant
		would be $(1)^3$. At large $n_i$ variation between $(n_i+1)-n_i\approx dn_i$ is infinitesimal and 
		the lattice becomes infinitely dense. This should justify our use of
		going from 
		\[
			\sum_{\vect n} f(\vect n) \to \int d^3n\ f(\vect n).
		\]
		\\
		% (b)
		\item
		Now some geometry: Show the volume in the first octant between the planes
		$x+y+z = k$ and $x+y+z =k+dk\quad (k>0)$ equals $dV = \frac{1}{2}k^2dk$.
		\\
		\\
		The plane in the first octant makes a triangle with sides all equal to $\sqrt 2k$. The area of 
		this plane is
		\[
			A = \frac{\sqrt 3}{2}k^2.
		\]
		The center of the plane is located by the normal vector with equal components
		\[
			\vect n = \plr{\frac{k}{3},\frac{k}{3},\frac{k}{3}}
		\]
		with magnitude $n = \frac{k}{\sqrt 3}$. Hence
		\[
			dn = \frac{1}{\sqrt 3}dk
		\]
		Now the volume can be formulated by
		\[
			dV = A\,dn = \frac{\sqrt 3}{2}k^2\plr{\frac{1}{\sqrt 3}dk}.
		\]
		Thus 
		\[
			dV = \frac{1}{2}k^2dk
		\]
		\\
		% (c)
		\item
		As usual, the density of energy eigenstates $D(\epsilon)$ is defined as the function of 
		energy $\epsilon$ such that for an arbitrary function of energy $g(\epsilon)$ we have
		\[
			\sum_i g(\epsilon_i) = \int_0^\infty d\epsilon\ D(\epsilon)g(\epsilon).
		\]
		Show that for the 3D harmonic oscillator the energy density, in fact, is 
		\[
			D(\epsilon) = \frac{\epsilon^2}{2(\h\omega)^3}
		\]
		\\
		\\
		We can reformulate the equation of a plane as
		\[
			\frac{\epsilon}{\h\omega} = \plr{n_1+\frac{1}{2}}
			+\plr{n_2+\frac{1}{2}}
			+\plr{n_3+\frac{1}{2}}
		\]
		However, we know that for $n_i$ large, we have $\plr{n_i+\frac{1}{2}}\approx n_i$
		so
		\[
			\frac{\epsilon}{\h\omega} = n_1+n_2+n_3.
		\]
		We proceed as in part (b) with $k = \frac{\epsilon}{\h\omega}$ and 
		$dk = \frac{d\epsilon}{\h\omega}$. 
		\[
			d^3n = \frac{1}{2}\pfrac{\epsilon}{\h\omega}^2 \frac{d\epsilon}{\h\omega}
		\]
		Thus
		\[
			D(\epsilon) = \frac{\epsilon^2}{2(\h\omega)^3}
		\] 
		\\
		\eenum
% 8.3 ------------------------------------------------------------------------------------------------------------------------
	\item[\textbf{8.3}]
	The sum-to-integral conversion from the last problem is obviously better the more states
	are involved in the sum. Suppose now that $D(\epsilon)\propto \epsilon^\alpha$ as
	$\epsilon\to 0$.
	
	\benum
		\item
            	% (a)
            	Argue that noninteracting bosons are liable to condense in a system only if the density
            	of states is characterized by an exponent $\alpha > 0$.
            	\\
		\\
		If the energy density goes as $1/\epsilon^\alpha$ for $\alpha > 0$, then as we approach the 
		ground state energy, the occupation number exponentially increases. This would even suggest 
		that at zero energy, we have an infinite number of states. This is clearly inconsistent, would 
		not allow us to separate the ground state occupation number from the rest of the integral
		term and would overshadow the effect when the fugacity approaches $z=1$.\\ \\ For the
		case that $\alpha = 0$, this suggests the density of states is constant, e.g. all
		equal occupation numbers and hence no condensate. 
		\\
		\\
            	\item
            	% (b)
            	Show that for a free massive particle in $D$ dimensions, $\alpha = (D-2)/2$ holds true.
            	\\
		\\
		For a free massive particle in $D$ dimensions, the energy is
		\[
			\epsilon_k = \frac{\h^2(\vect k\cdot\vect k)}{2m} = \frac{\h^2}{2m}\sum_i^D q_i^2
		\]
		with
		\[
			\vect k = \frac{2\pi}{L}\sum_i^D q_i\vecth e_i
		\]
		We also may express
		\[
			\sum_i g(\epsilon_i) = \sum_{\vect k} g(e_\vect k) \to \frac{L^D}{(2\pi)^D} \int d^Dk\ 
			g(\epsilon_{\vect k})
		\]
		The volume element in $D$ dimensions can be given in hyper-spherical coordinates as
		\[
			d^Dk = k^{D-1}dk\, d\Omega_{D-1}
		\]
		Integrating out the spherical dependence leave us with something like
		\[
			 ~ \int dk\ k^{D-1} g(\epsilon_{\vect k})
		\]
		Now we may use
		\[
			d\epsilon \sim k dk\quad\to\quad k^n\, dk \sim \epsilon^{(n-1)/2}\,d\epsilon
		\]
		so that 
		\[
			\int dk\ k^{D-1} g(\epsilon_{\vect k}) \sim \int d\epsilon\,
			\epsilon^{(D-2)/2} g(\epsilon).
		\]
		Thus $\alpha = (D-2)/2$ holds true. \\
		
            	\item
            	% (c)
            	The density of states is related to the number of energy eigenstates with energy less than
            	or equal to $\epsilon$, $N(\epsilon)$, by $D(\epsilon) = \frac{dN(\epsilon)}{d\epsilon}$.
            	On the basis of this observation, argue that for massive particles in a harmonic-oscillator in 
            	$D$ dimensions the exponent is $\alpha = D-1$.
            	\\
		\\
		The number of energy eigenstates is proportional to the energy volume under the shell of 
		energy less than $\epsilon$. Thus
		\[
			N \sim \epsilon^D \quad\to\quad D(\epsilon) = \frac{dN}{d\epsilon} \sim \epsilon^{D-1}
		\]
		\\
            	\item
            	% (d)
            	Do you have Bose-Einstein condensation in $D=1,2,3$ dimensions if the particles are free? 
		What if they are trapped in a harmonic potential well?
		\\
		\\
		According to answer (b), if $D=1,2$ then $\alpha = 0,-1/2$. Thus according to answer
		(a) which states $\alpha > 0$, we cannot have a Bose-Einstein condensate in two 
		or three dimensions. 
		\\
		\\
		If the particles are trapped in a harmonic potential, the ground state will never have zero
		energy, and thus we will never have an infinite nor negative ground state occupation number.
		\\
	\eenum
	
% 8.4 -------------------------------------------------------------------------------------------------------------------------
	\item[\textbf{8.4}]
		For the extremely relativistic ideal Fermi gas, one may write the dispersion relation 
		$\epsilon_k = \h c|\vect k|$. Find the Fermi energy, pressure, and average energy 
		per particle at zero temperature. 
		\\
		\\
		At zero temperature $\beta \to \infty$ and so the occupation number becomes
		\[
			\braket{n_i} = \theta(\mu-\epsilon_i).
		\]
		First we calculate the particle number as
		\ba
			N &= \frac{Vg}{((2\pi)^3}\int d^3k\ \theta(\mu-\epsilon_k) = \frac{Vg}{(2\pi)^3}
			\int_{\epsilon<\mu} d^3k \\
			& = \frac{Vg}{(2\pi)^3}\frac{4}{3\pi}k^3.
		\ea
		For these relativistic Fermions, the radius $k$ is
		\[	
			k = \frac{\mu}{\h c} 
		\]
		so $N$ becomes 
		\[
			N = \frac{Vg}{6\pi^2}\pfrac{\mu}{\h c}^3.
		\]
		Solving for $\mu$
		\[
			\mu = \h c k_F = \h c\pfrac{6\pi^2n}{g}^{1/3}
		\]
		So we have found
		\[
			\epsilon_f = \h c\pfrac{6\pi^2n}{g}^{1/3}.
		\]
		For the average energy per particle
		\ba
			\frac{\braket{E}}{\braket N} &= \frac{\int_{k<k_F} dk\ k^2 \epsilon_k}
			{\int_{k<k_F} dk\ k^2} 
			= \frac{\h c\int_{0}^{k_F} dk\ k^3 \epsilon_k}
			{\int_{0}^{k_F} dk\ k^2} = \frac{3}{4}\h c k_F = \frac{3}{4}\epsilon_F.
		\ea
		We can use $\braket E/\braket N$ and $N$ to find the average energy 
		(could have just directly integrated it as well)
		\[
			\braket{E} = \frac{Vg}{8\pi^2}\frac{\epsilon_F^4}{(\h c)^3}.
		\]
		The change in energy with respect to volume leads to the work done in expansion
		or contraction of the gas. Thus
		\[
			dE = dW = \frac{g}{8\pi^2}\frac{\epsilon_F^4}{(\h c)^3}dV = pdV.
		\]
		Hence the Fermi pressure should be
		\[
			p_F = \frac{g}{8\pi^2}\frac{\epsilon_F^4}{(\h c)^3}.
		\]
		
\eenum
\end{document}