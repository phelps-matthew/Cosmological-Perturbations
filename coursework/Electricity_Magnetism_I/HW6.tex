\documentclass[11pt,letterpaper]{article}
\usepackage{macroshw}

\title{\begin{spacing}{1.2}Electrodynamics I\\HW 6\end{spacing}}
\author{Matthew Phelps}
\date{Due: April 30}

\begin{document}
\maketitle

\benum
% #1 ------------------------------------------------------------------------------------------------------------------------------------------------------------------------------------
  	\item 
	A straight long wire and a ring of radius $a$ lie in the same plane. The distance between the wire and the ring center is $b$. Find
	the mutual inductance $L_{12}$ and interaction force $\vect F_{12}$, if the wire and ring currents are respectively $I_1$ and $I_2$.
	\\
	\\
	Ran out of time this week..
% #2 ----------------------------------------------------------------------------------------------------------------------------------------------------------------------------------
	\item
	Determine the trajectory of an electron moving in the uniform electric $\vect E = E_y\vecth e_y+E_z\vecth e_z$ and 
	magnetic  $\vect B = B\vecth e_z$ fields. Initial conditions at $t=0$: the electron radius vector $\vect r=0$, and the electron
	velocity $\vect v(t=0)$ is $\vect v_0 = v_{0x}\vecth e_x+v_{0z}\vecth e_z$. 
	\\
	\\
	We can use the Lorentz force law to determine the trajectory of the electron. The law goes as
	\[
		m\vect a= q\vect E+q(\vect v\times \vect B).
	\]
	Taking the cross product
	\[
		\vect v\times \vect B = v_yB\vecth x - v_xB\vecth y.
	\]
	Separating the components we have three differential equations
	\ba
		\ddot x &= \frac{q}{m}\dot yB \\
		\\
		\ddot y &= \frac{q}{m}(E_y-\dot xB) \\
		\\
		\ddot z &= \frac{q}{m}E_z.
	\ea
	The last equation is easily solved as 
	\[
		z(t) = \frac{q}{2m}E_zt^2+v_{z}(0)t+z(0)
	\]
	\[
		z(t) = \frac{E_zq}{2m}t^2+v_{0z}t
	\]
	The other two equations are coupled together, but we can use a substitution to solve them. Lets set
	\[
		u_1 = x+iy,\qquad u_2 = x-iy.
	\]
	In terms of these new variables, our first differential equation for $\ddot x$ becomes 
	\[
		\ddot{u_1}+\ddot{u_2} = \frac{qB}{m}i(\dot{u_2}-\dot{u_1})
	\]
	while the second differential equation for $\ddot{y}$ becomes
	\[
		i(\ddot{u_2}-\ddot{u_1}) = \frac{2q}{m}E_y - \frac{qB}{m}(\dot{u_1}+\dot{u_2}).
	\]
	Before we combine the equations, lets first multiply the $\ddot x$ equation by  a factor of $i$. Our two equations are then
	\[
		i(\ddot{u_1}+\ddot{u_2}) = \frac{qB}{m}(\dot{u_1}-\dot{u_2})
	\]
	\[
		i(\ddot{u_2}-\ddot{u_1}) = \frac{2q}{m}E_y - \frac{qB}{m}(\dot{u_1}+\dot{u_2}).
	\]
	Now we may add these equations together to get a differential equation of only one variable
	\[
		2i\ddot{u_2} = \frac{2q}{m}E_y+\frac{2qB}{m}(-\dot{u_2}).
	\]
	Simplifying
	\[
		\ddot{u_2} = \frac{q}{m}(-iE_y+iB\dot{u_2}).
	\]
	To make things easier, lets set $-\frac{q}{m}E_y \equiv a$ and $B\equiv b$ so that we have
	\[
		\ddot{u_2}-ib\dot{u_2}= -ia.
	\]
	As a inhomogeneous differential equation, we shall first find the homogeneous solution and then find the particular. For the
	equation 
	\[
		\ddot{u_2}-ib\dot{u_2} = 0
	\]
	the solution is easily solved to be 
	\[
		u_2(t) = Ae^{ibt}.
	\]
	Now in order to satisfy our inhomogeneous solution, we see that the differential equation is satisfied for the solution
	\[
		u_2(t) = Ae^{ibt} -iat.
	\]
	In terms of $x$ and $y$ we then have
	\[
		u_2(t)= x(t)-iy(t) = \cos(bt)+i\sin(bt) -iat.
	\]	
	Equating real and imaginary parts,
	\[
		x(t) = A\cos(bt)
	\]
	\[
		y(t) = -A\sin(bt)-at
	\]
	In terms of the original constants
	\[
		x(t) = A\cos(Bt)
	\]
	\[
		y(t) = -A\sin(Bt)+\frac{q}{m}E_yt
	\]
	\emph{I think something went wrong here, because the initial conditions cannot be satisfied as it is written. It seems that
	it should be as $x\propto \sin(Bt)$ and $y\propto\cos(Bt)$ but I am not sure where the mistake lies}
	\eenum
\end{document}