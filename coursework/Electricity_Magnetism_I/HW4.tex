\documentclass[11pt,letterpaper]{article}
\usepackage{macroshw}

\title{\begin{spacing}{1.2}Electrodynamics I\\HW 4\end{spacing}}
\author{Matthew Phelps}
\date{Due: April 7}

\begin{document}
\maketitle

\benum
% #1 ------------------------------------------------------------------------------------------------------------------------------------

	\item
	The center of a conducting sphere, carrying a charge $q$, lies on the plane boundary between two infinite homogeneous dielectrics with
	permittivities $\epsilon_1$ and $\epsilon_2$. Determine the potential $\phi$ of the electric field and the charge distribution $\sigma$ on 
	the sphere. 
	\\
	\\
	Lets orient the dielectrics such that we have $\epsilon_1$ for $z\ge 0$ and $\epsilon_2$ for $z\le 0$. Since our dielectric is isotropic and 
	uniform, then the induced polarization may be written as
	\[
		\vect P  = \epo \chi_e \vect E
	\]
	and the electric displacement is therefore
	\[
		\vect D = \epsilon_0 \vect E+\vect P = \epsilon \vect E=-\epsilon\del \Phi.
	\]
	Now we apply Gauss's law around the sphere:
	\[
		\oint \vect D\cdot d\vect S = q_f = q
	\]
	\[
		 \vect D_1 \int_0^{\frac{\pi}{2}}\int_0^{2\pi} d\theta\,  d\phi\, r^2\sin\theta+\vect D_2 \int_{\frac{\pi}{2}}^\pi \int_0^{2\pi} 
		d\theta\,  d\phi\, r^2\sin\theta = q
	\]
	\[
		\vect D_1 +\vect D_2 = \frac{q}{2\pi r^2}.
	\]
	Alternatively, we could solve for the electric field since $\epsilon \vect E = \vect D$ to find
	\[
		\vect E = \frac{q}{2\pi(\epsilon_1+\epsilon_2)r^2}
	\]
	and then separately we have
	\[
		\vect D_1 = \frac{\epsilon_1 q}{2\pi(\epsilon_1+\epsilon_2)r^2}
	\]
	\[
		\vect D_2 = \frac{\epsilon_2 q}{2\pi(\epsilon_1+\epsilon_2)r^2}.
	\]
	To find the potential, we can take the line integral setting $\Phi(\infty) = 0$:
	\[
		\Phi(r) = \int _r^\infty -\vect E\cdot d\vect l  = \int_r^\infty dr \, \frac{-q}{2\pi(\epsilon_1+\epsilon_2)r^2} =  
		\frac{1}{2\pi(\epsilon_1+\epsilon_2)}\frac{q}{r}.
	\]
	Of course, the potential within in the sphere takes the value of the potential at the surface thus
	\[
		\Phi(r) = \begin{cases}\ds \frac{1}{2\pi(\epsilon_1+\epsilon_2)}\frac{q}{R}&\quad(r\le R)\\ \\
		\ds \frac{1}{2\pi(\epsilon_1+\epsilon_2)}\frac{q}{r}&\quad (r \ge R)
		\end{cases}
	\]
	Before we find the charge distribution on the sphere, note that the surface charge at the plane interface of $\epsilon_1$ and $\epsilon_2$ 	is zero as the polarization is perpendicular to the normal
	\[
		\sigma_b = \vect P\cdot \vecth n = 0 .
	\]
	Now for the charge distribution on the sphere, we must find both the free and bound surface charge density. The total free 
	charge will simply be $q$, but there is additional bound charge at the interface of the sphere due to the polarization of the two 	
	dielectrics. Using $\vect D\cdot \vecth n = \sigma_f$ we have
	\[
		\sigma_f = \begin{cases} \ds \frac{\epsilon_2 }{2\pi(\epsilon_1+\epsilon_2)}\frac{q}{R^2}&\quad(z\le 0) \\ \\
		\ds \frac{\epsilon_1 }{2\pi(\epsilon_1+\epsilon_2)}\frac{q}{R^2}&\quad(z\ge 0)
		\end{cases}
	\]
	To find the bound charge we use $\vect P\cdot \vecth n = \sigma_b$. Calculating $\vect P$ as
	\[
		\vect P = \vect D -\epsilon_0\vect E
	\]
	we have
	\[
		\vect P_1 = \frac{\epsilon_1-\epo}{2\pi(\epsilon_1+\epsilon_2)}\frac{q}{r^2}
	\]
	\[
		\vect P_2 = \frac{\epsilon_2-\epo}{2\pi(\epsilon_1+\epsilon_2)}\frac{q}{r^2}.
	\]
	Therefore we find our bound charge distribution to be
	\[
		\sigma_b = \begin{cases} \ds \frac{-(\epsilon_2-\epo) }{2\pi(\epsilon_1+\epsilon_2)}\frac{q}{R^2}&\quad(z\le 0) \\ \\
		\ds \frac{-(\epsilon_1-\epo)}{2\pi(\epsilon_1+\epsilon_2)}\frac{q}{R^2}&\quad(z\ge 0)
		\end{cases}
	\]
	(notice that $\vecth n$ of the dielectric points in the $-\vecth r$ direction). Some important details to remark on is that 
	the total free charge surface density is equal to the free charge $q$ as expected:
	\[
			4\pi R^2 \sigma_f = q
	\]
	and that the total charge distribution is uniform on either hemisphere:
	\[
		\sigma_f+\sigma_b\quad (z\le 0)\  =\  \sigma_f+\sigma_b\quad(z\ge 0).
	\]
	This offers explanation as to why the potential is the same in either hemisphere. In summary, the total surface charge distribution
	on the sphere is
	\[
		\sigma = \sigma_f+\sigma_b =  \frac{\epo}{2\pi(\epsilon_1+\epsilon_2)}\frac{q}{r^2}.
	\]
% #2--------------------------------------------------------------------------------------------------------------------------------------

	\item 
	A spherical capacitor whose electrodes have radii $a$ and $b$ is filled with a dielectric whose permitivity is given by 
	$\epsilon(r) = \epsilon_c(a/r)^2$, where $r$ is the distance from the center and $\epsilon_c$ is constant. Calculate
	the capacitance of the system. 
	\\
	\\
	Using Gauss's law, the electric displacement outside the inner spherical shell can be given as 
	\[
		\oint \vect D\cdot d\vect S = q_f = q.
	\]
	Using $\vect D(r) = \epsilon(r)\vect E(r)$ we have
	\[
		\oint \vect \epsilon(r)\vect E(r)\cdot d\vect S = q.
	\]	
	Since the electric field is radially symmetric we have
	\[
		 \vect E = \frac{1}{4\pi r^2}\frac{q}{\epsilon(r)}\vecth r.
	\]
	To find the potential between the plates, we integrate 
	\[
		\int_b^a -\vect E\cdot d\vect l  = \Phi_b-\Phi_a
	\]
	\[
		-\frac{q}{4\pi a^2\epsilon_c} \int_b^a dr\, \frac{r^2}{r^2} = (b-a)\frac{q}{4\pi a^2\epsilon_c}
	\]
	Now using $C = q/\Phi$ we have
	\[
		C = \frac{4\pi a^2\epsilon_c}{b-a}
	\]
	\\
	\\
	
% #3 -------------------------------------------------------------------------------------------------------------------------------------------

	\item
	A point charge $q$ is placed at a distance $a$ from the plane, separating two infinite homogeneous dielectrics with permittivities 
	$\epsilon_1$ and $\epsilon_2$. Determine the potential $\phi$.
	\\
	Hint: The method of images may be used to construct potentials in both dielectrics.
	\\
	\\
	Within the dielectric mediums, we must find the laplace and poisson equations appropriate for a polarizable material, i.e.
	\[
		\del\cdot \vect D = \epsilon_1\del\cdot\vect E = -\epsilon_1\del^2\Phi = q\quad (z>0)
	\]
	\[
		\del\cdot \vect D = \epsilon_2\del\cdot\vect E = -\epsilon_2\del^2\Phi = q\quad (z<0)
	\]
	subject to the boundary conditions
	\[
		(\vect D_2-\vect D_1)\cdot \vecth n_{21} = \sigma_f 
	\]
	\[
		(\vect E_2-\vect E_1)\times \vecth n_{21} = 0.
	\]
	These boundary conditions follow from Gauss's law for $\vect D$ and the zero curl of $\vect E$. For this problem, such boundary 
	conditions can be met by placing image charges in the allowed (opposite) regions and solving the potential in each
	half space separately. The free surface charge density $\sigma_f$ should be zero for our configuration. Starting with the first region $	(z>0)$ the natural starting point is to place the image charge $q'$ at $z=-a$ to form a potential of
	\[
		\Phi_1(z>0) = \frac{1}{4\pi\epsilon_1}\plr{\frac{q}{R_1}+\frac{q'}{R_2}}
	\]
	where in cylindrical coordinates
	\[
		R_1 = [\rho^2+(a-z)^2]^{1/2};\quad R_2 = [\rho^2+(a+z)^2]^{1/2}.
	\]
	Now for $z<0$ where we have no charge, it is simplest to place a charge $q''$ at $z=a$ such that our potential is then
	\[
		\Phi_2(z<0) = \frac{1}{4\pi\epsilon_2}\frac{q''}{R_1}.
	\]
	To find the magnitudes of these charges, we must find the normal and tangental components of electric field and match our 
	boundary conditions. Computing the field for $z>0$,
	\ba
		\vect E_1 \times \vecth n_{21} = -\del_\rho\Phi_1&= -\elr{\pdiff{\rho}\blr{\frac{1}{4\pi\epsilon_1}
		\plr{\frac{q}{ [\rho^2+(a-z)^2]^{1/2}}+\frac{q'}{ [\rho^2+(a+z)^2]^{1/2}}}}}_{z=0}\vecth \rho\\
		& = \frac{2\rho}{4\pi\epsilon_1}\plr{\frac{q}{(\rho^2+a^2)^{3/2}}+\frac{q'}{(\rho^2+a^2)^{3/2}}}\vecth \rho\\
		& =  \frac{2\rho}{4\pi\epsilon_1}\plr{\frac{q+q'}{(\rho^2+a^2)^{3/2}}}\vecth \rho.
	\ea
	Similarly,
	\[
		\vect E_2 \times \vecth n_{21} =  \frac{\rho}{4\pi\epsilon_2}\plr{\frac{q''}{(\rho^2+a^2)^{3/2}}}.
	\]
	Now to find the tangental components,
	\[
		(\vect D_2-\vect D_1)\cdot \vecth n_{21} = (\epsilon_2 \vect E_2 - \epsilon_1 \vect E_1)\cdot \vecth n_{21} 
	\]
	we have
	\ba
		\epsilon_1\vect E_1\cdot \vecth n_{21} =  -\del_z\Phi_1&= -\elr{\pdiff{z}\blr{\frac{1}{4\pi\epsilon_1}
		\plr{\frac{q}{ [\rho^2+(a-z)^2]^{1/2}}+\frac{q'}{ [\rho^2+(a+z)^2]^{1/2}}}}}_{z=0}\vecth z\\
		&= \frac{1}{4\pi}
		\elr{\plr{\frac{q(a-z)}{ [\rho^2+(a-z)^2]^{3/2}}+\frac{q'(a+z)}{ [\rho^2+(a+z)^2]^{3/2}}}}_{z=0}\\
		&= \frac{1}{4\pi}\frac{a(q'-q)}{(\rho^2+a^2)^{3/2}}
	\ea
	Similarly, 
	\[
		\epsilon_2\vect E_2\cdot \vecth n_{21} = \frac{1}{4\pi}\frac{-aq''}{(\rho^2+a^2)^{3/2}}.
	\]
	Using these results in our boundary conditions we see that
	\[
		\frac{q''}{\epsilon_2} = \frac{q+q'}{\epsilon_1};\quad q'' = q-q'.
	\]
	Solving these two equations we find the image charges must be
	\[
		q' = -\pfrac{\epsilon_2-\epsilon_1}{\epsilon_2+\epsilon_1}q;\quad 
		q'' = \pfrac{2\epsilon_2}{\epsilon_2+\epsilon_1}q.
	\]
	Therefore the potential is
	\[
		\Phi(\rho,z) = \begin{cases}
		\ds \frac{1}{4\pi\epsilon_1}\plr{\frac{q}{ [\rho^2+(a-z)^2]^{1/2}}+\frac{-\pfrac{\epsilon_2-\epsilon_1}{\epsilon_2+\epsilon_1}
		q}{ [\rho^2+(a+z)^2]^{1/2}}}&\quad z\ge 0 \\ \\
		\ds  \frac{1}{4\pi\epsilon_2}\frac{\pfrac{2\epsilon_2}{\epsilon_2+\epsilon_1}q}{ [\rho^2+(a-z)^2]^{1/2}}&\quad z\le0
		\end{cases}
	\]
	At $z=0$ it can be shown that the potential is continuous ($\Phi_1 = \Phi_2$). 
% #4 -------------------------------------------------------------------------------------------------------------------------------------------

	\item
	An electric dipole of moment $p$ is placed in a homogeneous dielectric at the distance $z$ from the plane boundary of a semi-infinite
	conductor. The dielectric permittivity is $\epsilon$. Find the energy of interaction $U$ between dipole and induced charges.
	\\
	\\
	From question 3 we saw that the potential due to a point charge located above a plane interface of dielectrics $\epsilon_1$ and 
	$\epsilon_2$ is given as:
	\[
		\Phi(\rho,z) = \begin{cases}
		\ds \frac{1}{4\pi\epsilon_1}\plr{\frac{q}{ [\rho^2+(a-z)^2]^{1/2}}+\frac{-\pfrac{\epsilon_2-\epsilon_1}{\epsilon_2+\epsilon_1}
		q}{ [\rho^2+(a+z)^2]^{1/2}}}&\quad z\ge 0 \\ \\
		\ds  \frac{1}{4\pi\epsilon_2}\frac{\pfrac{2\epsilon_2}{\epsilon_2+\epsilon_1}q}{ [\rho^2+(a-z)^2]^{1/2}}&\quad z\le0
		\end{cases}
	\]
	To replace the dielectric of $\epsilon_2$ by a conductor and we should be able to take the limit
	\[
		\lim_{\epsilon_2\to\infty}\Phi(\rho,z) = \Phi(\rho,z)_{conductor}.
	\]
	If we note that
	\[
		\lim_{\epsilon_2\to\infty} \pfrac{\epsilon_2-\epsilon_1}{\epsilon_2+\epsilon_1} = 1
	\]
	and 
	\[
		\lim_{\epsilon_2\to\infty} \pfrac{\epsilon_2}{\epsilon_2+\epsilon_1} = 1
	\]
	then we can see that our potential becomes
	\[
		\Phi(\rho,z) = \begin{cases}
		\ds \frac{1}{4\pi\epsilon_1}\plr{\frac{q}{ [\rho^2+(a-z)^2]^{1/2}}+\frac{-q}{ [\rho^2+(a+z)^2]^{1/2}}}&\quad z\ge 0 \\ \\
		\ds  0 &\quad z\le0.
		\end{cases}
	\]
	This is exactly the potential for the common problem of a point charge and a grounded conductor! The only difference is the 
	$\epsilon_1$ substitution for the usual $\epsilon_0$. This means that we can continue all analysis as applied to a vacuum as long 
	as we remember to use $\epsilon_1$ appropriately. 
	\\
	\\
	Since this has been proven for the point charge, we can easily extend our result to that of a dipole. Let us assume we have a dipole
	located at a distance $z$ along the $z$-axis oriented at an arbitrary polar angle $\theta$. The potential and field of the
	dipole near a conducting plane (a problem well discussed in class) has an image dipole with a $\rho$ component in the 
	opposite direction. The field and potential can be solved, but it should not necessarily be needed for this problem. To find
	the energy of interaction, we can use Jackson eq. 4.26
	\[
		W_{12}= \frac{\vect p_1\cdot \vect p_2 - 3(\vecth n\cdot \vect p_1)(\vecth n\cdot \vect p_2)}
		{4\pi\epsilon_0|\vect x_1-\vect x_2|^3}
	\]
	where $\vecth n$ is in the direction $\vect x_1-\vect x_2$. Since this equation has been derived from the electric
	field of a dipole, we can in fact replace $\epsilon_0$ with $\epsilon_1$ to account for our dielectric. To compute the inner 
	products in the interaction energy equation, we first denote 
	\[ 
		\vect p_1 = p\cos\theta \vecth z + p\sin\theta \vecth\rho
	\]
	\[ 
		\vect p_2 = p\cos\theta \vecth z  -p\sin\theta \vecth\rho
	\]
	and observe that
	\[
		\vecth n= \vecth z
	\]
	\[
		|\vect x_1-\vect x_2| = 2z.
	\]
	Now we can carry on to calculating our the interaction energy
	\ba
		W_{12} &= \frac{1}{4\pi\epsilon_1}\frac{p^2(\cos^2\theta-\sin^2\theta)-3p^2\cos^2\theta}{8z^3}\\
		& = \frac{1}{4\pi\epsilon_1}\frac{-p^2}{8z^3}(2\cos^2\theta+\sin^2\theta)\\
		& =  \frac{1}{4\pi\epsilon_1}\frac{-p^2}{8z^3}(\cos^2\theta+1)\\
		& = \frac{1}{4\pi\epsilon_1}\frac{-p^2}{16z^3}(3+\cos(2\theta)).
	\ea
	The interaction energy is negative, as we may expect when we have two opposite charges (two pairs actually). In the limit 
	$z\to\infty$, the interaction energy vanishes. 
\eenum

\end{document}