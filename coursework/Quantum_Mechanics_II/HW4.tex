\documentclass[10pt,letterpaper]{article}
\usepackage{mymacros}
\usepackage{array}
\usepackage{booktabs}
\newcolumntype{L}{>{$}l<{$}}
\newcolumntype{C}{>{$}c<{$}}
\newcolumntype{R}{>{$}r<{$}}
\newcommand{\nm}[1]{\textnormal{#1}}


\title{\begin{spacing}{1.2}Quantum Mechanics II\\HW 4\end{spacing}}
\author{Matthew Phelps}
\date{Due: Oct. 7}

\begin{document}
\maketitle

\benum
% #1 -----------------------------------------------------------------------------------------------------------------------------------------------------------------
  	\item 
	Fine-structure energies in hydrogen
	\benum
		% (a)
		\item
		Verify that the spin-orbit splitting
		\[
			\Delta E_{SO} = \frac{(E_n^{Bohr})^2}{mc^2}\blr{\frac{n(j(j+1)-l(l+1)-\frac{3}{4})}{l(l+\frac{1}{2})
			(l+1)}}, \quad j = l \pm \frac{1}{2}
		\]
		and the relativistic 
		\[
			\Delta E_{rel} = -\frac{(E_n^{Bohr})^2}{mc^2}\blr{\frac{2n}{(l+\frac{1}{2})}-\frac{3}{2}}
		\]
		combine to give the net fine-structure correction
		\[
			\Delta E_{n,j}^{FS} = \frac{(E_n^{Bohr})^2}{2mc^2}\blr{3-\frac{4n}{(j+\frac{1}{2})}}
		\]
		so that (here $\vect R \equiv \frac{me^4}{2\h^2}$ is the ``Rydberg constant"):
		\be\label{1}
			E_{n,j}^{FS} = -\frac{\vect R}{n^2}\plr{1+\frac{\alpha^2}{n}\blr{\frac{1}{(j+\frac{1}{2})}-\frac{3}{4n}
			}}
		\ee
		\\
		\\
		First we note that
		\[
			 \frac{(E_n^{Bohr})^2}{mc^2} =\frac{1}{mc^2} \pfrac{-me^4}{2\h^2 n^2}^2 = -\frac{1}{2n^2}
			 \pfrac{e^2}{\h c}^2\pfrac{-me^4}{2\h^2 n^2} = -\frac{\alpha^2}{2}\frac{E_n^{Bohr}}{n^2} 
			 = \frac{\alpha^2}{2}\frac{\vect R}{n^4}.
		\]
		Next, we see that if we substitute the $j$'s in terms of $l$'s we have for the spin orbit correction
		\[
			\Delta E_{SO} = \frac{\alpha^2}{n}\frac{\vect R}{n^3} \blr{\frac{1}{2l(l+\frac{1}{2})
			(l+1)}}\begin{cases}l&\quad\text{for}\quad j = l+\frac{1}{2}\\-(l+1)&\quad\text{for}\quad j = l-\frac{1}{2}
			\end{cases}
		\]
			
		Now adding the spin-orbit and relativistic correction together
		\[
			\Delta E_{SO}+\Delta E_{rel} = -\frac{\vect R}{n^2}\frac{\alpha^2}{n}\blr{\frac{1}{(l+\frac{1}{2})}-\frac{3}{4n}
			-\frac{1}{2l(l+\frac{1}{2})(l+1)}\begin{cases}l\ ;\  j = l+
			\frac{1}{2}\\-(l+1)\ ;\  j = l-\frac{1}{2}
			\end{cases}}
		\]
		In this (improper) notation, only the last term in brackets is multiplied by the appropriate quantity corresponding
		to $j$. When we evaluate for $j = l+\frac{1}{2}$ we have
		\ba
			\Delta E^{FS}_{n,l} &= -\frac{\vect R}{n^2}\frac{\alpha^2}{n}\blr{\frac{1}{(l+\frac{1}{2})}-\frac{3}{4n}
			-\frac{1}{2(l+\frac{1}{2})(l+1)}} \\
			& = -\frac{\vect R}{n^2}\frac{\alpha^2}{n}\blr{-\frac{3}{4n}
			+\frac{1}{(l+1)}}  \\
			& = -\frac{\vect R}{n^2}\frac{\alpha^2}{n}\blr{-\frac{3}{4n}
			+\frac{1}{(j+\frac{1}{2})}}
		\ea
		while when we evaluate for $ j = l-\frac{1}{2}$ we have
		\ba
			\Delta E^{FS}_{n,l} &= -\frac{\vect R}{n^2}\frac{\alpha^2}{n}\blr{\frac{1}{(l+\frac{1}{2})}-\frac{3}{4n}
			-\frac{1}{2(l+\frac{1}{2})(l+1)}} \\
			& = -\frac{\vect R}{n^2}\frac{\alpha^2}{n}\blr{-\frac{3}{4n}
			+\frac{1}{l}}  \\
			& = -\frac{\vect R}{n^2}\frac{\alpha^2}{n}\blr{-\frac{3}{4n}
			+\frac{1}{(j+\frac{1}{2})}}.
		\ea
		We get the same answer! Thus the energy correction only depends on $j$ explicitly. Given
		the unperturbed energy 
		\[
			E^{Bohr}_{n} = -\frac{\vect R}{n^2}
		\]	
		 we thus confirm \eqref 1 by expressing the corrected fine structure energy levels as
		\[
		 	E_{n,j}^{FS} = -\frac{\vect R}{n^2}\plr{1+\frac{\alpha^2}{n}\blr{\frac{1}{(j+\frac{1}{2})}-\frac{3}{4n}
			}}.
		\]
		\\
		\\
		\item
		% (b) 
		Compute the $n=1$, $n=2$, $n=3$ hydrogen atom energies including fine-structure corrections, as
		given by formula \eqref 1. Use $\alpha = 1/137.035\,999\,074$. 
		\\
		
		\item
		% (c)
		The experimental values for hydrogen levels (labeled as $nl_j$, with standard notation $s : l = 0$,
		$p : l=1$, $d: l=2$), in units of the Rydberg are:
		\ba
			1s_{1/2}: -0.999 466 509\\
			2s_{1/2}: -0.249 867 761\\
			2p_{1/2}: -0.249 868 082\\
			2p_{3/2}: -0.249 864 748\\
			3s_{1/2}: -0.111 052 015\\
			3p_{1/2}: -0.111 052 111\\
			3p_{3/2}: -0.111 051 123\\
			3d_{3/2}: -0.111 051 124\\
			3d_{5/2}: -0.111 050 795\\
		\ea
		For each energy level compute the ratio of the experimental value to the fine-structure prediction 
		given by \eqref 1. Comment on this result. 
		\\
		
		\item
		% (d)
		Estimate the order of magnitude of the expected correction to the hydrogen energy levels if you use the 
		\textbf{reduced mass} of the electron [use $\frac{m_p}{m_e} = 1836.152\,672\,457$]. Use this correction to 
		recompute the hydrogen energy levels, and compare again with the experimental values. Comment. 
		\\
		\\
		The reduced mass $\mu$ of the electron is 
		\[
			\mu = \frac{m_pm_e}{m_p+m_e}.
		\]
		Denoting 
		\[
			\gamma \equiv \frac{m_p}{m_e} = 1836.152\,672\,457
		\]
		we may express the reduced mass as
		\[
			\mu = \pfrac{\gamma}{1+\gamma}m_e
		\]
		thus
		\[
			\vect R \to \pfrac{\gamma}{1+\gamma}\vect R
		\]
		and our overall energies are simply scaled by the factor $\pfrac{\gamma}{1+\gamma}$. To 
		estimate the the order of magnitude of this correction to the fine structure we
		take the difference between the two
		\[
			\frac{\gamma}{\gamma}-\frac{\gamma}{1+\gamma} = \frac{1}{1+\gamma} \approx \mathcal 
			O(10^{-3}).
		\]
		\\
		Below is 
		a table of experimental data compared to calculated energies and their respective ratios. 
		
		% Blindly copied a table format http://tex.stackexchange.com/questions/88929/vertical-table-lines-are-
		%discontinuous-with-booktabs
                \begin{table} [H]
                \centering
                \begin{tabular}{LCCCCC}
                
                \multicolumn{6}{c}{\large{Hydrogen Fine Structure Energy Levels}\makebox[0pt][l]{$^{\ast}$}} \\ 
                \midrule
                \multicolumn{1}{l}{$e^{(-)}$ Config.} &
                \multicolumn{1}{c}{Exp.}     &
                \multicolumn{1}{c}{F.S.} &
                \multicolumn{1}{c}{F.S.+R.M.}&
                \multicolumn{1}{c}{Exp./F.S.}&
                \multicolumn{1}{c}{Exp./(F.S.+R.M.)}
                \\
                \midrule
			1s_{1/2}& -0.999466509& -1.000013313& -0.999468985&0.999453203&0.999997523\\
			2s_{1/2}&	-0.249867761& -0.250004160& -0.249868078&0.999454412&0.999998732\\
			2p_{1/2}&	-0.249868082& -0.250004160& -0.249868078&0.999455696&1.000000017\\
			2p_{3/2}&	-0.249864748& -0.250000832 & -0.249864751&0.999455666&0.999999986\\
			3s_{1/2}&	-0.111052015&  -0.111112590& -0.111052109&0.999454829&0.999999150\\
			3p_{1/2}&	-0.111052111& -0.111112590& -0.111052109&0.999455693&1.000000014\\
			3p_{3/2}& 	-0.111051123& -0.111111604 & -0.111051124&0.999455672&0.999999992\\
			3d_{3/2}& 	-0.111051124&- 0.111111604& -0.111051124&0.999455681&1.000000001\\
			3d_{5/2}&	-0.111050795& -0.111111275 & -0.111050795&0.999455677&0.999999997\\
 		\midrule
		\multicolumn{6}{l}{ \footnotesize $^*$ Energies given in units of Ry.} \\
                \bottomrule
                \end{tabular}
                \end{table}
		Its possible that I need to adjust my the level of accuracy used in Mathematica in using the same level
		of supplied precision throughout the calculations. Nonetheless, we can still make many observations.
		\\
		\\
		Firstly (as is apparent in \eqref 1) we will have theoretical degeneracies across pairs of states due to the 
		dependence on $j$ and not $l$. For example the states $2s_{1/2}$ and $2p_{1/2}$ should have the
		same energy according to our corrections. However, when we look at the experimental data, we see 
		this is not the case. So we must be missing an additional correction that can lift such degeneracy. 
		Secondly, we see that the fine-structure correction is approximately accurate up to at least $\mathcal O
		(10^{-2})$ while the fine-structure plus reduced mass correction gets us to at least $\mathcal O
		(10^{-5})$. For the $3d$ states it even matches exactly up to the order the experimental data
		was given. Our estimate for the order of magnitude correction from the reduced mass appears to be
		correct - the difference between corrections is $\mathcal O(10^{-3})$. A look at the ratio for both sets 
		of corrections tends to show a trend toward higher accuracy 
		at higher energy levels. Perhaps the missing correction is proportional to something like $1/n$? The simple 
		correction of the reduced mass added a significant amount of accuracy to the energies. In general, it
		is remarkable how close these values are to the measured energies. 
		\\
	\eenum
% #2-------------------------------------------------------------------------------------------------------------------------------------------------------------
	\item
	Adiabatic Approximation
	\\
	\\
	Consider a Hamiltonian
	\[
		H = H(x,p,\lambda(t))
	\]
	that depends on the time through the time-dependent parameter $\lambda(t)$. Suppose that at each time $t$ the
	complete set of eigenstates of $H$ are known:
	\[
		H(x,p,\lambda(t))\psi_n(x,\lambda(t)) = E_n(\lambda(t))\psi_n(x,\lambda(t))
	\]
	\benum
		% (a)
		\item
		Write the full time-dependent wavefunction $\psi(x,t)$ as
		\[
			\psi(x,t) = \sum_n c_n(t)\exp\blr{-\frac{i}{h}\int_0^t E_n(t')\ dt'}\psi_n(x,\lambda(t))
		\]
		where $E_n(t) \equiv E_n(\lambda(t))$, and derive the following equation for the time evolution of the 
		coefficients:
		\[
			\dot{c_k}(t) = -\sum_n c_n(t)\exp\blr{\frac{i}{\h}\int_0^t (E_k(t')-E_n(t'))\ dt'}\int dx\ 
			\psi_k^*(x,\lambda(t))\dot{\psi}_n(x,\lambda(t))
		\]
		\\
		\\
		\\
		Lets denote 
		\[
			\theta_n(t) = -\frac{1}{\h}\int_0^t dt'\ E_n(t')
		\]
		and take it as given that all quantities are time-dependent herein-forth. Substituting the general time dependent 
		state $\psi(x,t)$ (in the eigen energy basis) into the time dependent Schrodinger equation
		\[
			i\h\pdiff{t}\psi = H\psi
		\]
		we have
		\[
			i\h\sum_n \blr{\dot c_ne^{i\theta_n}\psi_n+ic_n\dot\theta_ne^{i\theta_n}\psi_n+c_ne^{i\theta_n}\dot\psi_n}
			= \sum_n c_ne^{i\theta_n}E_n\psi_n.
		\]
		Factoring out the common exponential and evaluating $\dot\theta_n$ we obtain
		\[
			i\h\sum_n \blr{\dot c_n\psi_n-\frac{i}{\h}c_nE_n\psi_n+c_n\dot\psi_n}e^{i\theta_n} 
			= \sum_n c_nE_n\psi_ne^{i\theta_n}
		\]
		or
		\[
			\sum_n\dot c_n\psi_ne^{i\theta_n} = - \sum_n c_n\dot\psi_n e^{i\theta_n}. 
		\]
		Now we take the inner product with orthonormal energy state $\psi_k$ to obtain
		\[
			\dot c_ke^{i\theta_k} = -\sum_n c_n e^{i\theta_n} \int dx\ \psi^*_k\dot\psi_n 
		\]
		which may be rearranged as
		\be\label{2}
			\dot c_k = -\sum_n c_n e^{i(\theta_n-\theta_k)}\int dx\ \psi^*_k\dot\psi_n .
		\ee
		\\
		%(b)
		\item
		Use the result of (a) to show that to leading order, as $\dot\lambda\to0$, in the adiabatic approximation the 
		distribution of quantum states does not change in time.
		\\
		\\
		Given the time-derivative of energy eigenfunction coefficients, it may be possible to solve for them exactly
		if our Hamiltonian varies very slowly. Let us differentiate the energy eigenvalue equation
		\[
			\dot H\psi_n+H\dot\psi_n = \dot E_n\psi_n +E_n\dot\psi_n.
		\]
		Now take the inner product with eigenstate $\psi_k$ for $k\ne n$
		\[
			\int dx\  \psi_k^*\dot H\psi_n +\int dx\ \psi_k^* H\dot\psi_n = \int dx\  \dot E_n(0)+\int dx\ E_n\psi_k^*\dot
			\psi_n.
		\]
		Since our hamiltonian is hermitian, this leads to
		\[
			\int dx\ \psi_k^* \dot H\psi_n +\int dx\ E_k\psi_k^*\dot\psi_n =\int dx\  E_n\psi_k^*\dot\psi_n
		\]
		or
		\be \label{3}
			\int dx\ \frac{\psi_k^* \dot H \psi_n}{E_n-E_k} = \int dx\ \psi_k^*\dot\psi_n
		\ee
		This inner product term $\int dx\ \psi_k^*\dot\psi_n$ is also within the eq. for $\dot c_k$. Thus we can make the 
		substitution of \eqref 3 into \eqref 2 
		\[
			\dot c_k = -\sum_n c_n e^{i(\theta_n-\theta_k)}\int dx\ \frac{\psi_k^* \dot H \psi_n}{E_n-E_k}.
		\]
		Splitting the sum into two parts we obtain
		\be\label{4}
			\dot c_k = -c_k\int dx\ \psi_k^*\dot\psi_k - \sum_{n\ne k}c_n e^{i(\theta_n-\theta_k)}
			\int dx\ \frac{\psi_k^* \dot H \psi_n}{E_n-E_k}.
		\ee
		Now we look at the time derivative of $H$
		\[
			\diff[H]{t} = \pdiff[H]{x}\diff[x]{t}+\pdiff[H]{p}\diff[p]{t} + \pdiff[H]{\lambda}\diff[\lambda]{t}.
		\]
		Since $\dot x = \dot p=0$ in this Schrodinger picture, we see that
		\[
			\diff[H]{t} = \pdiff[H]{\lambda}\diff[\lambda]{t}.
		\]
		Thus taking $\dot\lambda \to 0$ leads to $\dot H \to 0 $ in which \eqref 4 becomes
		\[
			\dot c_k = -c_k\int dx\ \psi_k^*\dot\psi_k.
		\]
		This constitutes a basic differential equation of the form
		\[
			\diff[f(t)]{t} = -f(t)g(t)
		\]
		which has the solution
		\[
			f(t) = f(0)\exp\blr{-\int_0^t dt'\ g(t')}.
		\]
		Thus we may solve $c_k(t)$ as
		\[
			c_k(t) = c_k(0)e^{i\gamma_k(t)}
		\]
		with
		\[
			\gamma_k(t) = i\int_0^t dt'\ \psi_k^*\dot\psi_k.
		\]
		In fact, $\gamma_k(t)$ is real. This can be shown by the fact that the energy eigenstates are normalized
		\[
			0= \pdiff{t}\braket{\psi_k|\psi_k} = \braket{\dot\psi_k|\psi_k}+\braket{\psi_k|\dot\psi_k} = 2\mathfrak R[
			\braket{\dot\psi_k|\psi_k}].
		\]
		Therefore, under the adiabatic approximation any general state $\psi$ may be expressed as
		\[
			\psi(x,t) =  \sum_n c_n(0)e^{i(\theta_n+\gamma_n)}\psi_n(x,\lambda(t)) 
		\]
		which implies that the distribution of energy eigenstates over time only varies by a phase and 
		consequently the probability distribution is completely independent of time. 
		\\
		
		% (c)
		\item
		Show that for $k\ne n$:
		\[
			\int dx\ \psi_k^*(x,\lambda(t))\dot\psi_n(x,\lambda(t)) = \frac{1}{E_n(t)-E_k(t)}\int dx\ \psi_k^*(x\lambda(t))
			\plr{\pdiff[H]{t}}\psi_n(x,\lambda(t))
		\]
		and hence find the probability of transition to the $n^{th}$ unperturbed level at $t=+\infty$, for a
		charged linear harmonic oscillator, initially in the ground state at time $t=-\infty$, under the action
		of a homogeneous electric field $\mathcal E(t)$, where $\mathcal E(t) = \mathcal E_0e^{t^2/r^2}$.
		\\
		 \\
		 The relation for $k\ne n$ has been given in \eqref 3.
		 \\
		 \\
		We cannot use the adiabatic approximation here because the time variation of $H(t)$ does not vary slowly in 
		comparison to the time scale of our system. And even if we did, the probability of transition would be zero
		since eigenstates do not change in time. So instead, I will try a first order perturbation theory approach to give us at least
		a first order estimate of the probability of transition. Let us write a general time dependent state as
		\[
			\ket{\psi(t)} = \sum_n c_n(t)e^{-\frac{i}{\h}E_n^0t}\ket {n^0}.
		\]
		 Next we apply the Schrodinger equation and take the inner product with state $k^0$
		in the same vein as in part (a) to arrive at
		\be\label{5}
			\dot c_k = \frac{-i}{\h} \sum_n c_n(t) \braket{k^0|H(t)|n^0}e^{i\omega_{kn}t}
		\ee
		with
		\[
			\omega_{kn} = \frac{E_k^0-E_n^0}{\h}.
		\]
		For a zeroth order approximation in the time dependent Hamiltonian, we see from \eqref 5 that
		\[
			\dot c_k = 0
		\]
		which is the usual time independent system. The constants are fixed according to our imposed initial condition
		\[
			c_n(0) = \delta_{ni}
		\]
		for an initial state $\ket i$. To first order, we substitute the zeroth order coefficient into \eqref 5
		\[
			\dot c_n(t) =-\frac{i}{\h} \braket{k^0|H(t)|n^0}e^{i\omega_{kn}t}. 
		\]
		Along with the initial conditions, this equation may be solved as
		\be\label{6}
			c_n(t) = \delta_{nk}-\frac{i}{\h}\int_0^t dt'\ \braket{k^0|H(t')|n^0}e^{i\omega_{nk}t'}.
		\ee
		\\
		Now for the original problem, the coefficient of eigenstate $\ket n$ with the Hamiltonian
		\[
			H = \frac{p^2}{2m}+\frac{1}{2}m\omega^2 - q\mathcal E x e^{-t^2/r^2} 
		\]
		is
		\[
			c_n(\infty) = -\frac{i}{\h}\int_{-\infty}^{\infty} dt'\ \braket{0|-q\mathcal Exe^{-t'^2/r^2}|n} e^{i\omega_{n0}t'}.
		\]
		Being the friendly harmonic oscillator it is, we have the nice relation
		\[
			x = \sqrt{\frac{\h}{2m\omega}}(a+a^\dag)
		\]
		to see that the only state with non-zero probability of transmission is $\ket 1$. Thus our integral becomes
		\[
			c_1(\infty) = \frac{iq\mathcal E}{\h} \sqrt{\frac{\h}{2m\omega}} \int_{-\infty}^\infty dt'\ e^{-t'^2/r^2}e^{i\omega t'}.
		\]
		The integral is a Fourier transform and may be solved to give
		\[
			c_1(\infty) = \frac{iq\mathcal E}{\h} \sqrt{\frac{\h}{2m\omega}} \sqrt{\pi r^2}e^{-\omega^2 r^2/4}.
		\]
		Lastly we multiply by the complex conjugate to find the probability of transition:
		\[
			P_{0\to1} = \frac{q^2\mathcal E^2\pi r^2}{2m\omega \h}e^{-\omega^2r^2/2}.
		\]
		\\
		
	\eenum
% #3 ------------------------------------------------------------------------------------------------------------------------------------------------------------
	\item
	Landau-Zener Transition
	\\
	\\
	Consider a 2-level quantum system described by the 2$\times$2 matrix Hamiltoninan
	\[
		H(t) = \bpm E_-(t)&F\\F^*&E_+(t) \epm
	\]
	where $E_\pm(t) = \pm \frac{\nu}{2}t$, with $\nu>0$, and $F = const.$
	
	\benum
	% (a)
		\item
		Plot $E_\pm(t)$ and the instantaneous eigenvalues $\lambda_\pm(t)$ of $H(t)$ on the same graph, for various
		values of the parameters $F$ and $\nu$. Choose some parameters such that the curves $\lambda_\pm(t)$
		almost cross.
	\\
	\figg[width=100mm]{hw4_1.pdf}
	\figg[width=100mm]{hw4_2.pdf}
	\figg[width=100mm]{hw4_3.pdf}	
	\figg[width=100mm]{hw4_4.pdf}

	
	% (b)
		\item
		Write the wavefunction as
		\[
			\psi(t) = \bpm A(t)\exp\blr{-\frac{i}{\h}\int_{-\infty}^t dt'\ E_-(t')} \\
					B(t)\exp\blr{-\frac{i}{\h}\int_{-\infty}^t dt'\ E_+(t')}
				\epm
		\]
		and use the Schrodinger equation to derive the time evolution equations for the coefficients $A(t)$
		and $B(t)$.
		\\
		\\
		\\
		Using 
		\[
			i\h \pdiff{t}\ket{\psi(t)} = H\ket{\psi(t)}
		\]
		we form two differential equations for the coefficients of $\psi(t)$:
		\[
			i\h\plr{\dot A(t) -\frac{i}{\h}E_-(t)A(t)}e^{-\frac{i}{\h}\theta_a(t)} 
			= E_-(t)A(t)e^{-\frac{i}{\h}\theta_a(t)}+FB(t)e^{-\frac{i}{\h}\theta_b(t)}
		\]
		\[
			i\h\plr{\dot B(t) -\frac{i}{\h}E_+(t)B(t)}e^{-\frac{i}{\h}\theta_b(t)} 
			= F^*A(t)e^{-\frac{i}{\h}\theta_a(t)}+E_+(t)B(t)e^{-\frac{i}{\h}\theta_b(t)}
		\]
		where we have denoted
		\[
			\theta_a \equiv \int_{-\infty}^t dt'\ E_-(t),\quad \theta_b \equiv \int_{-\infty}^t dt'\ E_+(t).
		\]
		These two coupled differential equations may be simplified as
		\be\label{7}
			i\h\dot A(t)e^{-\frac{i}{\h}\theta_a(t)}=FB(t)e^{-\frac{i}{\h}\theta_b(t)} 
		\ee
		
		\be\label{8}
			i\h\dot B(t)e^{-\frac{i}{\h}\theta_b(t)}=F^*A(t)e^{-\frac{i}{\h}\theta_a(t)} 
		\ee
		If we like, we could take the derivative of either equation again and turn this into two decoupled
		second-order differential equations. But since this form is simple enough for Mathematica input, we 
		will leave it as it is. 
		\\
	% (c)
		\item
		Numerically integrate the equation for $B(t)$, with initial conditions
		\[
			B(-\infty)=1,\quad A(-\infty) = 0
		\]
		to obtain the probability $|B(+\infty)|^2$ that the system has made a transition. Plot $|B(t)|^2$ 
		and compare your final value, $|B(+\infty)|^2$, with the Landau-Zener formula
		\[	
			|B(+\infty)|^2 \approx \exp\blr{-\frac{2\pi|F|^2}{\nu\h}}
		\]
		Hint: Using NDsolve, start your numerical integration at some large negative value of $t$, and integrate
		out to some large positive value of $t$.
		\\
		\\
		\\
		Using the parameters $\nu = 1$ and $F = 0.2$ in units of $\h=1$ the Laudau-Zener forumla yields
		\[
			|B(+\infty)|^2_{LZ} = 0.777768
		\]
		while our numerical integration at large $t$ yields
		\[
			|B(+\infty)|^2_{Numerical} = 0.779318.
		\]
		This is an approximate error of
		\[
			\epsilon \approx 0.2 \%.
		\]
		\\
		Below is a plot of the probability amplitudes of both $A(t)$ and $B(t)$ where we can clearly see the probability of
		transition to the lower eigenstate. The system starts in an upper eigenstate in the infinite past and has a 
		considerable probability of transitioning to the lower eigenstate in the infinite future. Such is the
		behavior for the particular time dependent hamiltonian where $E_A(t)-E_B(t) \propto t$. 
		
		\figg[width=150mm]{hw4_5.pdf}
		
		\eenum
	
	
\eenum
\end{document}