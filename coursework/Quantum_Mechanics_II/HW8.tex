\documentclass[10pt,letterpaper]{article}
\usepackage{macroshw}

\title{\begin{spacing}{1.2}Quantum Mechanics II\\HW 8\end{spacing}}
\author{Matthew Phelps}
\date{Due: Nov. 11}

\begin{document}
\maketitle

\benum
% #1 -----------------------------------------------------------------------------------------------------------------------------------------------------------------
  	\item{Scattering from a radial delta shell}
	\\ \\
	Consider the potential for a spherical cavity of radius $R$, bounded by a thin wall of dimensionless opacity 
	$\Omega$:
	\[
		V(r) = \frac{\h^2}{2m}\frac{\Omega}{R}\delta(r-R)
	\]
	\benum
	% (a)
	\item
	Match the $s$-wave radial wavefunction at the delta function wall, and hence obtain an expression for 
	$\tan\delta_0$, or $\sin\delta_0$, in terms of the momentum $k$ and the parameters of the potential.
	% (b)
	\item
	From this matching also find an expression for the amplitude $A_0$ of the $s$-wave radial wavefunction
	inside the cavity, relative to the amplitude of the $s$-wave radial wavefunction outside the cavity.
	% (c)
	\item
	Plot $\delta_0(k)$, $\sigma_0(k)$, and $A_0(k)$ as functions of the dimensionless quantity $kR$, for 
	$\Omega = 4$, and again for $\Omega=10$.
	% (d)
	\item
	Comment on the physical interpretation of the features of these plots.
	\\
	\\
	\eenum

	
	\benum
	% (a)
	\item
	For $r<R$, the solution to the radial Schrodinger equation is that of a free particle with regular behavior at
	the origin
	\[
		R_l(r) = A_lj_l(kr)
	\]
	while for $r>R$ we again have a free particle but we include the Neumann function 
	\[
		R_l(r) = B_lj_l(kr)+C_ln_l(kr).
	\]
	At $r=R$, the wavefunction must be continuous
	\[
		A_lj_l(kR) = B_lj_l(kR)+C_ln_l(kR).
	\]
	For the other boundary condition, the delta-function will specify what the discontinuity of the wavefunction is. 
	Let us integrate the Schrodinger eq. with vanishing bounds around $R$
	\[
		-\frac{\h^2}{2m}\int_{R-\epsilon}^{R+\epsilon} dr'\ \difff{u_l}{*2r'} +
		\frac{\h^2}{2m} \int_{R-\epsilon}^{R+\epsilon} dr'\  \frac{\Omega}{R}\delta(r-R)u_l(r') = 0.
	\]
	The energy and centrifugal term vanish in the limit of $\epsilon\to 0$. From this equation we find
	\[
		\elr{\diff[u_l]{r}}_{R_+} - \elr{\diff[u_l]{r}}_{R_-} = \frac{\Omega}{R}u(R)
	\]
	or
	\[
		\elr{R\diff[R_l(r)]{r}}_{R_+}+R_l(R_+) - \elr{R\diff[R_l(r)]{r}}_{R_-}-R_l(R_-) = \frac{\Omega}{R}RR_l(R).
	\]
	Using the continuity of the wavefunction this leads to
	\[
		\elr{\diff[R_l]{r}}_{R_+} - \elr{\diff[R_l]{r}}_{R_-} = \frac{\Omega}{R}R_l(R).
	\]
	or
	\[
		 A_lkj'_l(kR) - B_lkj'_l(kR)-C_lkn'_l(kR) = \frac{\Omega}{R}\blr{B_lj_l(kR)+C_ln_l(kR)}.
	\]
	Ultimately, we want the relation between $C$ and $D$ to determine the phase shift. We can find this by first 
	eliminating $A_l$ and then solving for the ratio $\frac{C}{B}$
	\[
		\frac{B_lj_l(kR)+C_ln_l(kR)}{j_l(kR)} j'_l(kR) - B_lj'_l(kR)-C_ln'_l(kR) 
		= \frac{\Omega}{kR}\blr{B_lj_l(kR)+C_ln_l(kR)}.
	\]
	We find
	\[
		-\frac{C}{B} = \frac{\frac{\Omega}{kR}j_l(kR)}{\frac{\Omega}{kR}n_l(kR)+n'_l(kR)-n_l(kR)
		\frac{j'_l(kR)}{j_l(kR)}}.
	\]
	At this point we diverge from the general $l$ result and specifically restrict the problem to $l=0$. We 
	use the forms
	\[
		j_0 = \frac{\sin (kr)}{kr};\qquad n_0 =- \frac{\cos(kr)}{kr}
	\]
	to simplify the ratio as
	\[
		-\frac{C}{B} = \frac{\sin^2(kR)}{\frac{kR}{\Omega}-\cos(kR)\sin(kR)}.
	\]
	Now if we take the wavefunction for $r>R$ at $r\to \infty$ we  can show that it takes the form
	\[
		R(r) = \frac{(B^2+C^2)^{1/2}}{kr}\blr{\sin\plr{kr-\frac{(0)\pi}{2}+\delta_0}}
	\]
	with
	\[
		\tan\delta_0 = -\frac{C}{B}.
	\]
	Hence we have found $\delta_0$ in terms of the momentum and parameters of the potential
	\[
		\tan\delta_0 = \frac{\sin^2(kR)}{\frac{kR}{\Omega}-\cos(kR)\sin(kR)}.
	\]
	\\
	\\
	% (b)
	\item
	In the same way that we arrived at the form of the asymptotic radial wavefunction for $r>R$, $r\to\infty$, we 
	can take the solution of the free particle at the boundary $r=R$ and transform it in the same way:
	\[
		\frac{1}{kR}\blr{B\sin(kR)-C\cos(kR)}  = \frac{(B^2+C^2)^{1/2}}{kR}\blr{\sin\plr{kR+\delta_0}}.
	\]
	Now lets denote $D \equiv \sqrt{B^2+C^2}$ and use this in our first boundary condition equation
	\[
		A\sin(kR) = D\sin\plr{kR+\delta_0}
	\]
	hence
	\[
		\frac{A}{D} = \frac{\sin(kR+\delta_0)}{\sin(kR)}
	\]
	\\
	\\
	
	% (c)
	\item
	\[
		\boxed{
		\delta_0 = \tan^{-1} \plr{\frac{\sin^2(kR)}{\frac{kR}{\Omega}-\cos(kR)\sin(kR)}}
		}
	\]
	\\
	\figg[width=100mm]{8_11.pdf}
	\figg[width=100mm]{8_12.pdf}
	\phantom{} \phantom{} \phantom{} \phantom{} \phantom{} \phantom{} \phantom{} \phantom{} \phantom{} 
	\phantom{} \phantom{} \phantom{} \phantom{} 
	
	
	\[
		\boxed{
		\frac{A_0(k)}{D_0(k)} =  \frac{\sin(kR+\delta_0)}{\sin(kR)}
		}
	\]
	\\
	\figg[width=100mm]{8_13.pdf}
	\figg[width=100mm]{8_14.pdf}
	\phantom{} \phantom{} \phantom{} \phantom{} \phantom{} \phantom{} \phantom{} \phantom{} \phantom{} 
	\phantom{} \phantom{} \phantom{} \phantom{} 
	\[
		\boxed{
		\frac{\sigma_0}{R^2}  = \frac{4\pi}{(kR)^2} \sin^2(\delta_0)
		}
	\]
	\\
	\figg[width=100mm]{8_15.pdf}
	\figg[width=100mm]{8_16.pdf}
	
	% (d)
	\item
	Both the amplitude and phase-shift have periodic maximums and minimums, which correspond to resonant 
	energies of $kR$. The resonance is strongest at low energy - as we continue to increase the energy, the effect of 
	resonance becomes weak and we can see that the relative amplitude inside the well equilibrates to the amplitude 
	outside the well. Also, the s-wave cross section has the most significant contribution at low energy, which is what we 
	expect - as we increase the energy, higher angular momentum terms are required. In addition, increasing the 
	strength $\Omega$ of the potential increases the strength/effect of the resonance for the same given energy.
	This also shifts the $\sigma_0$ cross section energy dependence to larger values of $kR$.  
	\\
	\eenum
% 2 ----------------------------------------------------------------------------------------------------------------------------------------------------
	\item{Inverse square potential}
	\\ \\
	Consider the scattering problem with a radial potential
	\[
		V(r) = \frac{\alpha}{r^2}
	\]
	\benum
	% (a)
	\item 
	From the solution to the radial Schrodinger equation, write down an explicit expression for the phase shift 
	$\delta_l$. (Think about the form of the potential.)
	% (b)
	\item
	Show that in the large $l$ limit
	\[
		\delta_l \sim -\pfrac{\pi m \alpha}{2\h^2l}
	\]
	% (c)
	\item
	What is the momentum dependence of the total cross section?
	\\ \\ 
	\eenum
	
	\benum
	% (a)
	\item
	With $V=0$ the radial Schrodinger equation is
	\[
		\blr{\difff{}{*2r}+\frac{2}{r}\diff{r}-\frac{l(l+1)}{r^2}+k^2}R(r).
	\]
	With substitution $\rho = kr$ this becomes
	\[
		\difff{R}{*2\rho}+\frac{2}{\rho}\diff[R]{\rho}+\blr{1-\frac{l(l+1)}{\rho^2}}R = 0
	\]
	which is in fact the spherical Bessel differential equation whose solutions are the familiar $j_l(kr)$ 
	and $n_l(kr)$. Of course, none of this is new. 
	\\ \\
	Now add in the potential
	\[
		\difff{R}{*2\rho}+\frac{2}{\rho}\diff[R]{\rho}+\blr{1-\frac{l(l+1)}{\rho^2}-\frac{(2m\alpha/\h^2)}{\rho^2}}R = 0.
	\]
	With $s = \frac{2m\alpha}{\h^2}$ we can also write this as
	\[
		\difff{R}{*2\rho}+\frac{2}{\rho}\diff[R]{\rho}+\blr{1-\frac{s+l(l+1)}{\rho^2}}R = 0
	\]
	Using the fact that the spherical Bessel functions are not
	limited to integer values of the factor $l(l+1)$, this allows us to form a very similar differential equation
	whose solutions are in fact still spherical Bessel functions:
	\[
		\difff{R}{*2\rho}+\frac{2}{\rho}\diff[R]{\rho}+\blr{1-\frac{\lambda(\lambda+1)}{\rho^2}}R = 0
	\]
	where we have specifically defined
	\[
		\lambda(\lambda+1) \equiv s+l(l+1).
	\]
	Hence the solutions are
	\[
		R_l(r) = j_\lambda(kr)+ n_\lambda(kr)
	\]
	with (solving the quadratic eq. for lambda)
	\[
		\lambda = -\frac{1}{2}+\sqrt{\pfrac{2l+1}{2}^2+s}.
	\]
	The asymptotic form of the radial equation under the influence of the potential is
	\[
		R_l(r) \underset{r\to\infty}\longrightarrow \frac{\sin\plr{kr-\frac{\lambda\pi}{2}}}{kr}
	\]
	while the general form in terms of the phase shift is
	\[
		R_l(r) \underset{r\to\infty}\longrightarrow \frac{\sin\plr{kr-\frac{l\pi}{2}+\delta_l}}{kr}.
	\]
	To find $\delta_l$ we just have to match these
	\[
		-\frac{\lambda\pi}{2} = -\frac{l\pi}{2}+\delta_l
	\]
	so finally
	\[
		\delta_l = \frac{\pi}{2}\plr{l-\lambda} = \frac{\pi}{2}\plr{l+\frac{1}{2}-\sqrt{\pfrac{2l+1}{2}^2+\frac{2m\alpha}{\h^2}}}.
	\]
	\\
	\\
	
	% (b)
	\item
	For large $l$ we first have $l+\frac{1}{2}\approx l$ and thus
	\[
		\delta_l \approx \frac{\pi}{2}\plr{l-\sqrt{l^2+s}}.
	\]
	Inputting the binomial expansion
	\[
		\sqrt{l^2+s} = l\sqrt{\frac{s}{l^2}+1} \approx 1+\frac{s}{2l^2}
	\]
	we have
	\[
		\delta_l \approx \frac{\pi}{2}\plr{l-l\plr{1+\frac{s}{2l^2}}} = -\frac{\pi s}{4l}
	\]
	and therefore
	\[
		\delta_l \sim -\pfrac{\pi m \alpha}{2\h^2l}.
	\]
	\\
	\\
	
	% (c)
	\item
	With
	\[
		\sigma_l = \frac{4\pi}{k^2}(2l+1)\sin^2\delta_l
	\]
	the total cross section is
	\[
		\sigma = \sum_{l=0}^\infty \sigma_l .
	\]
	Interestingly, the phase shift does not depend on the energy. And so the momentum dependence is
	just $1/k^2$ as seen in 
	\[
		\sigma = \sum_{l=0}^\infty \sigma_l = \frac{1}{k^2}\sum_l f(l).
	\]
	For large $l$, $f(l)$ decreases in magnitude and thus makes little contribution to the total
	cross section. 
	\\
	\\
	\eenum 

% 3 ------------------------------------------------------------------------------------------------------------------------------------------------------
	\item{WKB Phase Shifts}
	\\ \\
	Consider a radial potential $V(r)$ and define $U(r) = \frac{2m}{\h^2}V(r)$. The WKB expression for the phase 
	shift is given by the difference between the WKB phases with and without the potential
	\[
		\delta_l^{WKB}(k) = \int_{r'}^{\infty} dr\ \sqrt{k^2-U(r)-\frac{l(l+1)}{r^2}} - \int_{r''}^{\infty} dr\ 
		\sqrt{k^2-\frac{l(l+1)}{r^2}}
	\]
	Here $r'$ and $r''$ are the relevant turning points with and without the potential.
	\benum
	% (a)
	\item
	Consider the case of scattering from a hard sphere of radius $a$. For the energy regime where 
	$k^^2>\frac{l(l+1)}{a^2}$, compute the WKB expression for the phase shift $\delta_l^{WKB(k)}$. 
	% (b)
	\item
	For $l=5$ and $l=10$, plot the contribution to the cross section from the $l^{th}$ partial wave, as a function of
	$k$ with $k>\sqrt\frac{l(l+1)}{a^2}$, using the WKB expression found in the previous part, and compare it on the
	same plot with the exact expression.
	% (c)
	\item
	Comment on the differences between the WKB and exact plots in the previous part, and explain the origin
	of the discrepancy. 
	\\
	\\
	\eenum
	
	\benum
	% (a)
	\item
 	As a hard sphere
	\[
		V(r) = \begin{cases} \infty &\qquad \text{for}\quad r\le a\\
						0 &\qquad \text{for}\quad r>a 
						\end{cases}
	\]
	For the particle under the potential influence, the turning point is simply $a$ and the potential vanishes
	beyond $a$, so that we have
	\[
		\int_{a}^{\infty} dr'\ \sqrt{k^2-\frac{l(l+1)}{r'^2}} .
	\]
	For the second integral, the turning point can be found as
	\[
		k^2 = \frac{l(l+1)}{r^2}
	\]
	or 
	\[
		r_0 = \sqrt{\frac{l(l+1)}{k^2}}.
	\]
	Altogether then
	\[
		\delta_l^{WKB}= \int_{a}^{\infty} dr'\ \sqrt{k^2-\frac{l(l+1)}{r'^2}}
		- \int_{r_0}^{\infty} dr''\ \sqrt{k^2-\frac{l(l+1)}{r''^2}}.
	\]
	Since we are in the energy regime $k^2 >\frac{l(l+1)}{a^2}$ this means $r_0 < a$ and we can write 
	the phase shift as
	\[
		\delta_l^{WKB} = -\int_{r_0}^a dr \sqrt{k^2-\frac{l(l+1)}{r'^2}} .
	\]
	Evaluating the integral we have
	\[
		\delta_l^{WKB} = \frac{\pi}{2}\sqrt{l(l+1)}-\sqrt{(ka)^2-l(l+1)}-l(l+1)
		\tan^{-1}\plr{\frac{\sqrt{l(l+1)}}{\sqrt{(ka)^2-l(l+1)}}}
	\]
	\\
	\\
	
	% (b)
	\item
	First let's compute the exact phase shift. For $r<a$, the wavefunction vanishes, while for $r>a$ it has the
	free particle form
	\[
		R_l(r) = A_lj_l(kr) + B_ln_l(kr).
	\]
	The boundary condition of continuity leads us to
	\[
		R_l(a) = 0
	\]
	or
	\[
		-\frac{B_l}{A_l} = \frac{j_l(ka)}{n_l(ka)}.
	\]
	As $r\to\infty$ the form is
	\[
		R_l(r)  \underset{r\to\infty}\longrightarrow \frac{(A^2+B^2)^{1/2}}{kr}
		\blr{\sin\plr{kr-\frac{l\pi}{2}+\tan^{-1}\pfrac{-B_l}{A_l}}}
	\]
	hence
	\[
		\delta_l = \tan^{-1}\pfrac{-B_l}{A_l} = \tan^{-1}\pfrac{j_l(ka)}{n_l(ka)}.
	\]
	\\
	\\
	The $l^{th}$ contribution to the total cross section is
	\[
		\sigma_l = \frac{4\pi}{k^2}(2l+1)\sin^2\delta_l.
	\]
	Both exactly and in the WKB approximation, we will plot $\sigma_l$ as a function of $k$ 
	for $l=2,5,10$. We set $a=1$ and look at energies $(ka)^2>l(l+1)$.
	\figg[width=100mm]{8_33.pdf}
	\figg[width=100mm]{8_31.pdf}
	\figg[width=100mm]{8_32.pdf}
	
	% (c)
	\item
	The WKB approximation is good for energies in which the potential varies slowly in comparison to the 
	wavelength of the particle $\lambda = 2\pi/k$ (this might be an oversimplification?). At the surface of the
	hard-sphere, the potential has a singular derivative and so the WKB fails to provide a reasonable 
	approximation near $r=a$. Consequently, the phase shift absorbs this discrepancy. 
	\\
	\\
	As we increase $l$, the angular momentum barrier slope increases and again the WKB becomes less valid. At the 
	same time, 
	higher energies are needed to increase the distance between $a$ and the angular momentum barrier, such
	that the phase shift can accumulate accurate contributions away from the spherical barrier at $a$. This is why we see 	the WKB being a good approximation at low values of $l$ with high energy. 
	\eenum
\eenum
\end{document}