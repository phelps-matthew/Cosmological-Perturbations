\documentclass[11pt,letterpaper]{article}
\usepackage{macroshw}

\title{\begin{spacing}{1.2}Quantum Mechanics II\\HW 1\end{spacing}}
\author{Matthew Phelps}
\date{Due: Sept. 9}

\begin{document}
\maketitle

\benum
% #1 --------------------------------------------------------------------------------------------------------------------------------------------------------------------------------------
  	\item 
	Consider the 1-dimensional double-well Hamiltonian
	\[
		H = -\frac{\h^2}{2m}\difff{}{*2x}+\frac{m\omega^2}{8a^2}(x^2-a^2)^2
	\]
	\benum
		\item 
		% (a)
		Show that an expansion about either one of the wells, in the limit of large separation between the wells, leads to a harmonic
		oscillator system of frequency $\omega$.
		\\
		\\
		First, we expand the potential $V$ about $a$ 
		\[
			V = \frac{m\omega^2}{2}\blr{(x-a)^2+\frac{(x-a)^3}{a}+\frac{(x-a)^4}{4a^2}}.
		\]
		In the limit of large separation between wells, we see that the last two terms tend toward zero for $x$ in the neighborhood of $a$. This 		leaves us with a Hamiltonian of 
		\[
			H = -\frac{\h^2}{2m}\difff{}{*2x}+\frac{m\omega^2}{2}(x-a)^2
		\]
		which represents the Hamiltonian of a harmonic oscillator centered around $a$. A similar result applies if we instead expand about 
		$-a$. 
		\\
		% (b)
		\item
		Make an educated guess for the form of the wavefunction for the ground state and first excited state, in the limit of large 
		separation between the wells. 
		\\
		\\
		Based on part (a), we see that for large separation between wells the potential takes the form of two harmonic oscillators; one at $a$
		and one at $-a$. As such, a reasonable guess for a ground state wave function would be two Gaussians centered at both $a$ and $-a$.
		As our potential is even, both even and odd solutions are permitted. Since the ground state is symmetric, we guess a solution of
		\[
			\ket{\Psi_0} = \braket{x-a|0}+\braket{x+a|0}
		\]
		where $\braket{x|n}$ represents the $n$th state of the harmonic oscillator. The next excited state will be an antisymmetric combination 
		of the two Gaussians
		\[
			\ket{\Psi_1} = \braket{x-a|0} - \braket{x+a|0}. 
		\]
		% (c)
		\item
		For each of these two wavefunctions, compute $\frac{\braket{\psi|H|\psi}}{\braket{\psi|\psi}}$
		\\
		\\
		First, for the approximate ground state energy (not yet taking the limit of large well separation) we have
		\[
			\braket{\Psi_0|H|\Psi_0} = \int_{-\infty}^\infty \Psi_0(x)\plr{-\frac{\h^2}{2m}\difff{}{*2x}+\frac{m\omega^2}{2}(x^2-a^2)^2}
			\Psi_0(x)\, dx
		\]
		where
		\[
			\Psi_0(x) = \pfrac{m\omega}{\pi\h}^{1/4}\exp\blr{-\frac{m\omega}{2\h}(x-a)^2}+
			 \pfrac{m\omega}{\pi\h}^{1/4}\exp\blr{-\frac{m\omega}{2\h}(x+a)^2}.
		\]
		Evaluating the integral we find
		\[
			\braket{\Psi_0|H|\Psi_0} = \h\omega+\frac{3}{16}\frac{\h^2}{a^2m}+\exp\plr{-\frac{a^2m\omega}{\h}}
			\blr{\frac{1}{4}\frac{\h}{\omega}+\frac{3}{16}\frac{\h^2}{a^2m}-\frac{3}{4}a^2m\omega^2}
		\]
		with norm squared
		\[
			\braket{\Psi_0|\Psi_0} = 2+2\exp\plr{\frac{a^2m\omega}{\h}}.
		\]
		Dividing the two, we arrive at our estimation of the ground state energy 
		\be\label{1}
			\frac{\braket{\Psi_0|H|\Psi_0}}{\braket{\Psi_0|\Psi_0}} = \frac{\h\omega}{2}\blr{1+\frac{3}{16}\frac{\h}{a^2m\omega}-\frac{3}{4}
			\plr{\frac{a^2m\omega}{\h}+1}\frac{1}{1+\exp\pfrac{a^2m\omega}{\h}}}.
		\ee
		\\
		
		For the first excited state, the relevant quantities are
		\ba
			&\Psi_1(x) = \pfrac{m\omega}{\pi\h}^{1/4}\exp\blr{-\frac{m\omega}{2\h}(x-a)^2}-
			 \pfrac{m\omega}{\pi\h}^{1/4}\exp\blr{-\frac{m\omega}{2\h}(x+a)^2}
			\\ \\
			&\braket{\Psi_1|H|\Psi_1} = \h\omega+\frac{3}{16}\frac{\h^2}{a^2m}-\exp\plr{-\frac{a^2m\omega}{\h}}
			\blr{\frac{1}{4}\frac{\h}{\omega}+\frac{3}{16}\frac{\h^2}{a^2m}-\frac{3}{4}a^2m\omega^2}
			\\ \\
			&\braket{\Psi_1|\Psi_1} = 2-2\exp\plr{\frac{a^2m\omega}{\h}}
		\ea
		thus
		\be\label{2}
			\frac{\braket{\Psi_1|H|\Psi_1}}{\braket{\Psi_1|\Psi_1}} = \frac{\h\omega}{2}\blr{1+\frac{3}{16}\frac{\h}{a^2m\omega}+\frac{3}{4}
			\plr{\frac{a^2m\omega}{\h}+1}\frac{1}{\exp\pfrac{a^2m\omega}{\h}-1}}.
		\ee
		\\	
		% (d)
		\item
		Deduce that in the limit of large separation between the wells the energies are approximately:
		\[
			E_0 \approx \frac{\h\omega}{2}\blr{1+\frac{3}{16}\frac{\h}{a^2m\omega}-\frac{3}{4}\frac{a^2m\omega}{\h}\exp
			\blr{-\frac{a^2m\omega}{\h}}+...}
		\]
		\[
			E_1 \approx \frac{\h\omega}{2}\blr{1+\frac{3}{16}\frac{\h}{a^2m\omega}+\frac{3}{4}\frac{a^2m\omega}{\h}\exp
			\blr{-\frac{a^2m\omega}{\h}}+...}
		\]
		Thus, the energy splitting is exponentially small in the limit of large separation between the wells.
		\\
		\\
		In the limit of large separation between wells
		\[
			\exp\pfrac{a^2m\omega}{\h} \gg 1
		\]
		such that \eqref{1} and $\eqref{2}$ respectively become 
		\[
			E_0 \approx \frac{\h\omega}{2}\blr{1+\frac{3}{16}\frac{\h}{a^2m\omega}-\frac{3}{4}\frac{a^2m\omega}{\h}\exp
			\blr{-\frac{a^2m\omega}{\h}}-\frac{3}{4}\exp\blr{-\frac{a^2m\omega}{\h}}}
		\]
		\[
			E_1 \approx \frac{\h\omega}{2}\blr{1+\frac{3}{16}\frac{\h}{a^2m\omega}+\frac{3}{4}\frac{a^2m\omega}{\h}\exp
			\blr{-\frac{a^2m\omega}{\h}}+\frac{3}{4}\exp\blr{-\frac{a^2m\omega}{\h}}}.
		\]
		\\
		\\
	\eenum
% #2 ----------------------------------------------------------------------------------------------------------------------------------------------------------------------------------
	\item
	Evaluate the integrals involved in the derivation of the variational approximation for the ground state energy of He. Specifically, evaluate
	the integrals involved in going from the first line to the second line of the formula (16.1.15) in Shankar.
	\\
	\\
	If we ignore the electron-electron repulsion we see that the helium Hamiltonian resembles that of two hydrogen Hamiltonians with 
	$Z=2$. Trying a wavefunction of the product of two $Z=2$ hydrogen ground states in the full helium hamiltonian, we find that the ground
	state energy is about -75eV. The actual energy of ground state of the He atom is -78.6eV. To better our approximation, we 
	allow $Z$ to be a free parameter. The motivation is that the effective potential each electron sees will be slightly less than $Z=2$, due 
	to the electron screening. As such, we will guess a solution of the ground state wavefunction as
	\[
		\psi(\vect r_1,\vect r_2) = \frac{Z^3}{\pi a_0^3}\exp\blr{\frac{-Z(r_1+r_2)}{a_0}}
	\]
	and we may express our (unchanged) Hamiltonian as
	\[
		H = \blr{-\frac{\h^2}{2m}\plr{\del_1^2+\del_2^2}-\frac{Ze^2}{r_1}-\frac{Ze^2}{r_2}}+\frac{(Z-2)e^2}{r_1}+\frac{(Z-2)e^2}{r_2}
		+\frac{e^2}{|\vect r_1-\vect r_2|}.
	\]
	In evaluating $\braket{\psi|H|\psi}$ we see that the bracketed term in our Hamiltonian is just twice the energy of the $Z=2$ hydrogen-like
	atom. Thus
	\ba
		\braket{\psi|H|\psi} &= \braket{\psi|H_0|\psi} + \braket{\psi|G+I|\psi}\\
		& = -Z^2\frac{e^2}{a_0} +\braket{\psi|G+I|\psi}.
	\ea
	where we have separated the remaining Hamiltonian into two parts
	\[
		I\equiv \braket{\psi|\frac{e^2}{|\vect r_1-\vect r_2|}|\psi}
	\]
	\[
		G\equiv \braket{\psi|\frac{(Z-2)e^2}{r_1}+\frac{(Z-2)e^2}{r_2}|\psi}.
	\]
	Evaluating $G$ first (as $\psi$ is real, we omit complex conjugates in inner products)
	\ba
		G =& \iint d^3\vect r_1d^3\vect r_2\ \psi_{100}(\vect r_1)\psi_{100}(\vect r_2)\frac{(Z-2)e^2}{r_1}\psi_{100}(\vect r_1)\psi_{100}(\vect r_2) 		\\
		&+  \iint d^3\vect r_1d^3\vect r_2\ \psi_{100}(\vect r_1)\psi_{100}(\vect r_2)\frac{(Z-2)e^2}{r_2}\psi_{100}(\vect r_1)\psi_{100}(\vect r_2)
	\ea
	where $\psi_{100}(\vect x)$ represents the ground state of the $Z$ hydrogen-like atom. Working on the first integral above we have
	\ba
		& \int d^3\vect r_1 \ \psi_{100}(\vect r_1)\frac{(Z-2)e^2}{r_1}\psi_{100}(\vect r_1)
		 \int d^3\vect r_2\ \psi_{100}(\vect r_2)\psi_{100}(\vect r_2)
		 \\
		 =& \int d^3\vect r_1 \ \psi_{100}(\vect r_1)\frac{(Z-2)e^2}{r_1}\psi_{100}(\vect r_1)
		 \\
		 =&(Z-2)e^2 \frac{4\pi Z^3}{\pi a_0^3}\int_0^\infty dr_1\ \exp\blr{\frac{-2Zr_1}{a_0}}r_1
		 \\
		 =&(Z-2)e^2 \frac{Z}{a_0}.
	\ea
	We can see that the second integral in $G$ is in fact the same integral, thus
	\[
		G = 2(Z-2)e^2 \frac{Z}{a_0}.
	\]
	For the interaction integral $I$, we need to compute
	\ba
		I=&\iint d^3\vect r_1d^3\vect r_2\ \psi_{100}(\vect r_1)\psi_{100}(\vect r_2)\frac{e^2}{|\vect r_1-\vect r_2|}\psi_{100}(\vect r_1)\psi_{100}
		(\vect r_2)
		\\
		=&\,e^2  \int d^3\vect r_2\ \psi_{100}^2(\vect r_2)\int d^3\vect r_1\ \psi^2_{100}(\vect r_1)\frac{1}{|\vect r_1-\vect r_2|}
	\ea
	Looking at the second integral, the potential can be expanded in terms of Legendre polynomials as
	\[
		\frac{1}{|\vect r_1-\vect r_2|} = \sum_l \frac{ r_1^l}{r_2^{l+1}}P_l(\cos\theta)\quad r_1<r_2.
	\]
	If we orient $\vect r_2$ to lie along the $z$-axis, we may express the integral as
	\ba
		&2\pi\frac{Z^3}{\pi a_0^3}\int_0^\pi \int_0^{r_2} d\theta\, dr_1\ r_1^2\sin\theta  \plr{\sum_l \frac{ r_1^l}{r_2^{l+1}}P_l(\cos\theta)}\exp
		\blr{\frac{-2Zr_1}{a_0}}
		\\
		+& 2\pi\frac{Z^3}{\pi a_0^3}\int_0^\pi \int_{r_2}^{\infty} d\theta\, dr_1\ r_1^2\sin\theta  \plr{\sum_l \frac{ r_2^l}{r_1^{l+1}}P_l(\cos\theta)}
		\exp\blr{\frac{-2Zr_1}		{a_0}}.
	\ea
	If we look at the orthonormality of the Legendre polynomials
	\[
		\int_0^\pi d\theta\ P_l(\cos\theta)P_m(\cos\theta) = \frac{2}{2l+1} \delta_{lm}
	\]
	and note that $P_0(\cos\theta) = 1$, we observe that only the $l=0$ term will survive in each integral. Thus
	\ba
		 \int d^3\vect r_1\ \psi^2_{100}(\vect r_1)\frac{1}{|\vect r_1-\vect r_2|} &= \frac{4Z^3}{ a_0^3}\plr{\int_0^{r_2} dr_1\  \frac{r_1^2}{r_2}
		 \exp\blr{\frac{-2Zr_1}{a_0}} + \int_{r_2}^{\infty} dr_1\ r_1\exp\blr{\frac{-2Zr_1}{a_0}}}
		 \\
		 & = \frac{1}{r_2}-\plr{\frac{1}{r_2}+\frac{Z}{a_0}}\exp\pfrac{-2Zr_2}{a_0}.
	\ea
	Lastly, we integrate with respect to to $\vect r_2$
	\ba
		I &= e^2\pfrac{4Z^3}{a_0^3}\int_0^\infty dr_2\ r_2^2 \exp\blr{\frac{-2Zr_2}{a_0}}\blr{\frac{1}{r_2}
		-\plr{\frac{1}{r_2}+\frac{Z}{a_0}}\exp\pfrac{-2Zr_2}{a_0}}
		\\
		& = \frac{5}{8}\frac{Ze^2}{a_0}
	\ea
	\\
	\\
	Altogether we have
	\[
		\braket{\psi|H|\psi} =  -Z^2\frac{e^2}{a_0}+2Z(Z-2)\frac{e^2}{a_0}+\frac{5}{8}\frac{Ze^2}{a_0}.
	\]
	Substituting
	\[
		\text{Ry} = \frac{e^2}{2a_0}
	\]
	we can express the approximate ground state energy of Helium as
	\[
		\braket{\psi|H|\psi}  = -2\text{Ry}\blr{4Z-Z^2-\frac{5}{8}Z}
	\]
	\\
	\\
	\\
% #3 ------------------------------------------------------------------------------------------------------------------------------------------------------------------------------
	\item
	Consider the hydrogen molecule ion problem with Hamiltonian
	\[
		H = -\frac{\h^2}{2m}\del^2-\frac{e^2}{r_1}-\frac{e^2}{r_2}
	\]
	where $r_1$ and $r_2$ are the distances of the electron from each nucleus, $R$ is the separation between the two nuclei. Thus, with 
	one nucleus at the origin and the other on the $z$-axis at $z=R$, we have in terms of spherical polar coordinates:
	$r_1=r$, and $r_2 = \sqrt{r^2+R^2-2rR\cos\theta}$, where $\theta$ is the angle to the $z$-axis. 
	\\
	\\
	For the ground and first-excited states choose trial wavefunction
	\[
		\psi_\pm = A(\psi_0(r_1)\pm\psi_0(r_2))
	\]
	where $\psi_0(r)$ is the normalized ground state wavefunction of hydrogen. 
	\benum
		% (a)
		\item
		Show that the overlap integral is
		\[
			I \equiv \braket{\psi_0(r_1)|\psi_0(r_2)} = \blr{1+\frac{R}{a}+\frac{1}{3}\pfrac{R}{a}^2}e^{-R/a}
		\]
		where $a$ is the Bohr radius.
		\\
		\\
		The normalized ground state wavefunction for hydrogen is 
		\[
			\psi_0(r) = \sqrt{\frac{1}{\pi a^3}}e^{-r/a}.
		\]
		Substituting the relevant quantities into the overlap integral we have
		\ba
			I &= \frac{1}{\pi a^3} \intasphere \exp\blr{-\frac{r}{a}}\exp\blr{-\frac{\sqrt{r^2+R^2-2rR\cos\theta}}{a}} 
			\\
			& = \frac{2}{a^3}\int_0^\pi \int_0^\infty dr\, d\theta\  r^2\sin\theta \exp\blr{-\frac{r}{a}}\exp\blr{-\frac{\sqrt{r^2+R^2-2rR\cos\theta}}{a}} 
			\\
			& = \frac{4}{a^3}\plr{e^{-R/a}\blr{\frac{a^3}{4}+\frac{1}{4}Ra^2+\frac{1}{12}R^2a}}
			\\
			& =  \blr{1+\frac{R}{a}+\frac{1}{3}\pfrac{R}{a}^2}e^{-R/a}
		\ea
		% (b)
		\item
		Show that the direct integral is 
		\[
			D \equiv a\braket{\psi_0(r_1)|\frac{1}{r_2}|\psi_0(r_1)} = \frac{a}{R}-\plr{1+\frac{a}{R}}e^{-2R/a}
		\]
		\\
		\\
		For the direct integral we have
		\[
			D = \frac{2}{a^2}\int_0^\pi \int_0^\infty d\theta\, dr\ r^2\sin\theta\frac{e^{-\frac{2r}{a}}}{\sqrt{r^2+R^2-2rR\cos\theta}}.
		\]
		For the angular part
		\ba
			\int_0^\pi d\theta\ \frac{\sin\theta}{\sqrt{r^2+R^2-2rR\cos\theta}} &= \elr{\frac{1}{rR}\sqrt{r^2+R^2-2rR\cos\theta}}_0^\pi
			\\
			& = \frac{1}{rR}\blr{(r+R)-|r-R|}.
		\ea
		Incorporating the absolute value, we now integrate over the radius
		\ba
			D &= \frac{2}{a^2}\blr{\int_0^R dr\  \pfrac{2}{R}r^2e^{-\frac{2r}{a}} +\int_R^\infty dr\ 2r\,e^{-\frac{2r}{a}}}\\
			& = \frac{a}{R}-\plr{1+\frac{a}{R}}e^{-2R/a}
		\ea
		% (c)
		\item
		Show that the exchange integral is
		\[
			X \equiv a\braket{\psi_0(r_1)|\frac{1}{r_1}|\psi_0(r_2)} = \plr{1+\frac{R}{a}}e^{-R/a}
		\]
		\\
		\\
		Now for the exchange integral
		\[	
			X = \frac{2}{a^2}\int_0^\pi \int_0^\infty dr\, d\theta\  r \exp\blr{-\frac{r}{a}}\exp\blr{-\frac{\sqrt{r^2+R^2-2rR\cos\theta}}{a}}.
		\]
		This is very similar to the overlap integral with a linear factor of $r$ in the integrand. Thus
		\ba
			X &= \frac{2}{a^2}\int_0^\pi \int_0^\infty dr\, d\theta\  r \exp\blr{-\frac{r}{a}}\exp\blr{-\frac{\sqrt{r^2+R^2-2rR\cos\theta}}{a}}
			\\
			& = \plr{1+\frac{R}{a}}e^{-R/a}
		\ea
		% (d)
		\item 
		Hence, including the nucleus-nucleus electrostatic energy $\frac{e^2}{R}$, show that the total energy expectation value
		for $\psi_\pm$ can be written as 
		\[
			\frac{\braket{\psi\pm|H_{total}|\psi\pm}}{\braket{\psi\pm|\psi\pm}}
			= |E_{Bohr}|
			\left\{-1+2\frac{a}{R}\blr{\frac{\plr{1+\frac{R}{a}}e^{-2R/a}\pm\plr{1-\frac{2}{3}\pfrac{R}{a}^2}e^{-R/a}}
			{1\pm\plr{1+\frac{R}{a}+\frac{1}{3}\pfrac{R}{a}^2}e^{-R/a}}}\right\}
		\]
		\\
		\\
		Let's start with the ground state wavefunction
		\[
			\psi_+ = A\blr{\psi_0(r_1)+\psi_0(r_2)}.
		\]
		We will denote this state as
		\[
			\ket{\psi_+} = A\plr{\ket{\psi_{01}}+\ket{\psi_{02}}}
		\]
		such that $\braket{r_1|\psi_{01}} = \psi_0(r_1)$ and $\braket{r_2|\psi_{02}} = \psi_0(r_2)$.\\
		For the normalization
		\ba
			\braket{\psi_+|\psi_+} &= A^2\plr{ 1+1+2\braket{\psi_{01}|\psi_{02}}} \\
			& =  2A^2\plr{1+I} \\
			& = 2A^2\plr{1+\blr{1+\frac{R}{a}+\frac{1}{3}\pfrac{R}{a}^2}e^{-R/a}}
		\ea
		Now for the Hamiltonian
		\ba
			\braket{\psi_+|H|\psi_+} &= A^2\plr{\bra{\psi_{01}}+\bra{\psi_{02}}}|-\frac{\h^2}{2m}\del^2-\frac{e^2}{r_1}-\frac{e^2}{r_2}+\frac{e^2}
			{R}|\plr{\ket{\psi_{01}}+\ket{\psi_{02}}}
			\\
			& = A^2\blr{2E_{Bohr}+2E_{Bohr}I + 4E_{Bohr}(D+X)}
			\\
			& = 2A^2|E_{Bohr}|\plr{-(1+I)-2(D+X)}.
		\ea
		Dividing by the normalization,
		\ba
			\frac{\braket{\psi_+|H|\psi_+}}{\braket{\psi_+|\psi_+}} &= |E_{Bohr}|\left\{-1-2\frac{(D+X)}{(1+I)}\right\}
			\\
			& = |E_{Bohr}|
			\left\{-1+2\frac{a}{R}\blr{\frac{\plr{1+\frac{R}{a}}e^{-2R/a}+\plr{1-\frac{2}{3}\pfrac{R}{a}^2}e^{-R/a}}
			{1+\plr{1+\frac{R}{a}+\frac{1}{3}\pfrac{R}{a}^2}e^{-R/a}}}\right\}
		\ea
		\\
		\\
		Now for the antisymmetric state
		\[
			\ket{\psi_-} = A\plr{\ket{\psi_{01}}-\ket{\psi_{02}}}
		\]
		we may use the same $I$, $D$, and $X$ integrals to form the expectation of energy as
		\[
			\braket{\psi_-|H|\psi_-} =  2A^2|E_{Bohr}|\plr{-(1-I)-2(D-X)}
		\]
		with normalization
		\[
			\braket{\psi_-|\psi_-} = 2A^2(1-I).
		\]
		Consequently we have for the asymmetric state 
		\ba
			\frac{\braket{\psi_-|H|\psi_-}}{\braket{\psi_-|\psi_-}} &= |E_{Bohr}|\left\{-1-2\frac{(D-X)}{(1-I)}\right\}
			\\
			& = |E_{Bohr}|
			\left\{-1+2\frac{a}{R}\blr{\frac{\plr{1+\frac{R}{a}}e^{-2R/a}-\plr{1-\frac{2}{3}\pfrac{R}{a}^2}e^{-R/a}}
			{1-\plr{1+\frac{R}{a}+\frac{1}{3}\pfrac{R}{a}^2}e^{-R/a}}}\right\}
		\ea
		Thus we arrive at the expression initially given 
		\[
			\frac{\braket{\psi\pm|H_{total}|\psi\pm}}{\braket{\psi\pm|\psi\pm}}
			= |E_{Bohr}|
			\left\{-1+2\frac{a}{R}\blr{\frac{\plr{1+\frac{R}{a}}e^{-2R/a}\pm\plr{1-\frac{2}{3}\pfrac{R}{a}^2}e^{-R/a}}
			{1\pm\plr{1+\frac{R}{a}+\frac{1}{3}\pfrac{R}{a}^2}e^{-R/a}}}\right\}
		\]
		where we should also include the contribution of the potential between protons, which goes as
		\[
			\frac{2e^2}{R}\plr{1\pm I} = |E_{Bohr}|\frac{4a}{R}(1\pm I)
		\]
		for the ground and first excited state respectively.
		\\
		\\
		% (e)
		\item
		Plot these two functions and deduce that the energy expectation value has a minimum for the plus state, but not for the minus state. 
		\\
		\\
		For the ground state we have 
		\[
			f(R/a) = \frac{\braket{\psi_+|H|\psi_+}}{\braket{\psi_+|\psi_+}}
		\]
		\figg[width=100mm]{hw1_1.pdf}
		while for the first excited state we have
		\[
			g(R/a) = \frac{\braket{\psi_-|H|\psi_-}}{\braket{\psi_-|\psi_-}}
		\]
		\figg[width=100mm]{hw1_2.pdf}
		As we can see, only the ground state has a minimum for the expectation energy. 
		
	
	\eenum 
\eenum
\end{document}