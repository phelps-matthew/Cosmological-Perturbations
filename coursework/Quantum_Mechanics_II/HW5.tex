\documentclass[10pt,letterpaper]{article}
\usepackage{macroshw}

\title{\begin{spacing}{1.2}Quantum Mechanics II\\HW 5\end{spacing}}
\author{Matthew Phelps}
\date{Due: Oct. 21}

\begin{document}
\maketitle

\benum
% #1 -----------------------------------------------------------------------------------------------------------------------------------------------------------------
  	\item{Coherent States}
	\\
	\\
	Define a coherent state $\ket \alpha$ as
	\[
		\ket \alpha = e^{\alpha a^\dag-\alpha^*a}\ket 0 = e^{-|\alpha|^2/2}e^{\alpha a^\dag}\ket 0
	\]
	where $[a,a^\dag] = 1$, and the vacuum state $\ket 0$ is such that $a\ket 0 = 0$. 
	
	\benum
	% (a)
	\item
	Using basic properties of the number operator and number eigenstates $\ket n$, verify that the coherent state
	$\ket \alpha$ is an eigenstate of the operator $a$: $a\ket\alpha = \alpha\ket\alpha$, and hence compute the
	expectation value of the number operator $N = a^\dag a$ in the coherent state $\ket\alpha$.
	\\
	\\
	The coherent state may be expressed as
	\[
		\ket\alpha = e^{-|\alpha|^2/2} \sum_n \frac{\alpha^n a^{\dag n}}{n!} \ket 0 
		=  e^{-|\alpha|^2/2} \sum_n \frac{\alpha^n}{\sqrt{n!}}\ket n.
	\]
	Now apply $a$
	\[
		a\ket\alpha =   e^{-|\alpha|^2/2}   \sum_n \alpha^{n+1}\frac{\sqrt{n+1}}{\sqrt{n+1!}}\ket n 
		= \alpha \plr{e^{-|\alpha|^2/2} \sum_n \frac{\alpha^n}{\sqrt{n!}}\ket n} = \alpha\ket\alpha.
	\]
	In retrospect, a simpler method would be the following:
	\ba
		a\ket\alpha &= e^{-|\alpha|^2/2}([a,e^{\alpha a^\dag}]+e^{\alpha a^\dag}a)\ket 0 \\
		& = e^{-|\alpha|^2/2}\alpha e^{\alpha a^\dag}\ket 0\\
		& = \alpha \ket \alpha
	\ea
	Now we are in a nice position to take the expectation value of $N = a^\dag a$ in state $\ket\alpha$
	\[
		\braket{\alpha|N|\alpha} = \alpha\alpha^*\braket{\alpha|\alpha} = |\alpha|^2.
	\]
	\\
	% (b)
	\item
	Evaluate the uncertainties $\delta x$ and $\delta p$ for a coherent state, using 
	\[
		x = \sqrt\frac{\h}{2m\omega}(a^\dag +a),\qquad p = i\sqrt\frac{m\h\omega}{2}(a^\dag-a)
	\]
	and show that this state saturates the lower bound from the uncertainty relation. 
	\\
	\\
	\\
	With dispersion $\Delta x \equiv x-\braket x$, the variance is 
	\[
		\braket{(\Delta x)^2} = \braket{x^2}-\braket{x}^2.
	\]
	For the first term
	\ba
		\braket{x^2} &= \frac{\h}{2m\omega}\braket{(a^\dag+a)^2} \\
		&=  \frac{\h}{2m\omega}\braket{(a^{\dag 2}+a^2+2N+1)} \\
		& =  \frac{\h}{2m\omega}\blr{(1+2|\alpha|^2)+\braket{a^2}^*+\braket{a^2}}\\
		& = \frac{\h}{2m\omega}\plr{1+2|\alpha|^2+\plr{\alpha^2}^*+\alpha^2}\\
		& = \frac{\h}{2m\omega}\blr{1+(\alpha+\alpha^*)^2}.
	\ea
	While the second term is
	\[
		\braket{x}^2 =  \frac{\h}{2m\omega}\braket{(a^\dag+a)}^2 = \frac{\h}{2m\omega}(\alpha+\alpha^*)^2.
	\]
	Thus
	\[
		\braket{(\Delta x)^2} =  \frac{\h}{2m\omega}.
	\]
	Similarly we can show for $p$ that we have
	\[
		\braket{(\Delta p)^2} = -\frac{m\h\omega}{2}\blr{(\alpha-\alpha^*)^2-1}+\frac{m\h\omega}{2}(\alpha-\alpha^*)^2
		= \frac{m\h\omega}{2}.
	\]
	Finally we can form the uncertainty relation
	\[
		\braket{(\Delta x)^2}\braket{(\Delta p)^2} = \frac{\h^2}{4}\quad\text{or}\quad
		\sigma_x\sigma_p = \frac{\h}{2}
	\]
	which meets the lower bound equality of the uncertainty relation. 
	\\
	\\
	% (c)
	\item
	Decompose the coherent state $\ket \alpha$ in the basis of eigenstates of the number operator and hence
	show that the probability that a measurement of the number operator yields the value $n$ for a system
	in a coherent state $\ket\alpha$ is given by the Poisson distribution: $P_n = \frac{|\alpha|^{2n}}{n!}e^{-|\alpha|^2}$.
	\\
	\\
	The decomposition is (a) was
	\[	
		\ket\alpha = e^{-|\alpha|^2/2} \sum_n \frac{\alpha^n}{\sqrt{n!}}\ket n.
	\]
	The probability of measuring $n$ is 
	\[
		|\braket{n|\alpha}|^2 = e^{-|\alpha|^2}\alpha^n(\alpha^n)^*\frac{1}{\sqrt{n!}}\frac{1}{\sqrt{n!}} 
	\]
	therefore
	\[
		P_n = \frac{|\alpha|^{2n}}{n!}e^{-|\alpha|^2}.
	\]
	\\
	% (d)
	\item
	Using the Baker-Campbell-Hausdorff formula, $e^{A+B} = e^{A}e^Be^{-\frac{1}{2}[A,B]}$ (which is valid if $[A,B]$
	commutes with both $A$ and $B$), show that
	\[
		e^{-\gamma^*a+\gamma a^\dag}\ket\alpha = e^{\frac{1}{2}(\alpha^*\gamma-\gamma^*\alpha)}
		\ket{\alpha+\gamma}
	\]
	\\
	\\
	Since $[a^\dag,a] = 1$, our commutation criteria for the BCH formula is satisfied
	\ba
		e^{-\gamma^*a+\gamma a^\dag}\ket\alpha &= e^{-\gamma^*a}e^{\gamma a^\dag}e^{\frac{1}{2}|\gamma|^2}
		\plr{e^{-|\alpha|^2/2}e^{\alpha a^\dag}\ket 0} \\
		& = e^{\frac{1}{2}(\alpha\gamma^*+\alpha^*\gamma)}e^{|\gamma|^2}e^{-\gamma^*a}e^{-|\alpha+\gamma|^2/2}
		e^{(\alpha+\gamma)a^\dag}\ket 0 \\
		& = e^{\frac{1}{2}(\alpha\gamma^*+\alpha^*\gamma)}e^{|\gamma|^2}e^{-\gamma^*a}\ket{\alpha+\gamma}\\
		& = e^{\frac{1}{2}(\alpha^*\gamma-\gamma^*\alpha)}\ket{\alpha+\gamma}
	\ea
	where the last step uses the eigen-relation $e^a\ket\alpha = e^\alpha\ket\alpha$. 
	\\
	\\
	\\
	\eenum
	
% 2 ------------------------------------------------------------------------------------------------------------------------------------------------------
	\item
	Time Evolution Operator for Forced Harmonic Oscillator, and Coherent States
	\\
	\benum
	
	% (a)
	\item
	Using the identity
	\be\label{1}
		\diff{t}e^{A(t)} = \plr{\dot A(t)+\frac{1}{2!}[A(t),\dot A(t)]+\frac{1}{3!}[A(t),[A(t),\dot A(t)]]+...}e^{A(t)}
	\ee
	choose
	\[
		A(t) = -\zeta^*(t)a+\zeta(t)a^\dag
	\]
	where $a$ and $a^\dag$ are the harmonic oscillator annihilation and creation operators, and show that
	\[
		\diff{t}e^{-\zeta^*a+\zeta a^\dag} = \plr{-\dot\zeta^*a+\dot\zeta a^\dag+\frac{1}{2}\plr{\zeta\dot\zeta^*-\zeta^*\dot
		\zeta}\mathbf{1}}e^{-\zeta^*a+\zeta a^\dag}
	\]
	\\
	\\
	For the derivative we have
	\[
		\dot A(t) = -\dot\zeta^*a+\dot\zeta a^\dag.
	\]
	First note that since $\zeta(t)$ are just parametrized complex numbers 
	\[
		[\zeta(t),\dot\zeta(t)] = 0.
	\]
	Now evaluating the commutator, 
	\ba
		[A(t),\dot A(t)] &= [-\zeta^*a+\zeta a^\dag,-\dot\zeta^*a+\dot\zeta a^\dag]\\
		& = \zeta^*\dot\zeta^*[a,a]+\zeta\dot\zeta[a^\dag,a^\dag]-\zeta^*\dot\zeta[a,a^\dag]-\zeta\dot\zeta^*
		[a^\dag,a]\\
		& = \mathbf 1(\zeta\dot\zeta^*-\zeta^*\dot\zeta).
	\ea
	From this result, we see that for any operator $B$
	\[
		[B,[A(t),\dot A(t)]] = 0. 
	\]
	Consequently all terms beyond $\frac{1}{2!}[A(t),\dot A(t)]$ in the parenthesis of \eqref 1 are zero. Thus we
	may express the derivative of our exponential operator as
	\[
		\diff{t}e^{-\zeta^*a+\zeta a^\dag} = \plr{-\dot\zeta^*a+\dot\zeta a^\dag+\frac{1}{2}\plr{\zeta\dot\zeta^*
		-\zeta^*\dot\zeta}\mathbf{1}}e^{-\zeta^*a+\zeta a^\dag}
	\]
	% (b)
	\item
	Hence show that for the forced oscillator with Hamiltonian
	\[
		H(t) = \h\omega\plr{a^\dag a+\frac{1}{2}}+\plr{f(t)a+f^*(t)a^\dag}
	\]
	the interaction picture evolution operator is
	\[
		U_I(t,t_0) = e^{i\beta(t,t_0)}e^{-\zeta^*(t,t_0)a+\zeta(t,t_0)a^\dag}
	\]
	where
	\[
		\zeta(t,t_0) = -\frac{i}{\h}\int_{t_0}^t dt'\ e^{i\omega t'}f^*(t')
	\]
	and $\beta(t,t_0)$ is a particular phase (find an explicit expression for this phase). 
	\\
	\\
	\\
	In the interaction picture, the time evolution operator is given by the differential equation
	\[
		i\h\diff[U_I]{t}  = H_I^1(t) U_I
	\]
	where $H_I^1(t)$ is the interaction Hamiltonian consisting of the ``rotated" time-dependent perturbation of the 
	original Hamiltonian
	\[
		H_I^1(t) = U^{0\dag}H^1(t)U^0
	\]
	and $U^0 = e^{-\frac{i}{\h}H^0t}$ is the time evolution operator of the unperturbed Hamiltonian. For our problem
	\[
		H^1(t) = f(t)a+f^*(t)a^\dag
	\] 
	and so the differential equation for $U_I(t)$ is
	\[
		i\h\diff[U_I(t,t_0)]{t}  = \exp\blr{-\frac{i}{\h}H^0 t}\plr{f(t)a+f^*(t)a^\dag }\exp\blr{\frac{i}{\h}H^0t} U_I(t,t_0).
	\]
	Taking a look at one particular term
	\[
		 \exp\blr{-\frac{i}{\h}H^0t}a\exp\blr{\frac{i}{\h}H^0t} 
	\]
	we see that this is the annihilation operator of the unperturbed harmonic oscillator in the Heisenberg picture. As
	such, it follows the Heisenberg equation of motion
	\[
		\diff[a]{t} = -\frac{i}{\h}[a,H^0] = -i\omega[a,N] = -i\omega a
	\]
	with solution
	\[
		a(t) = a(t_0)e^{-i\omega (t-t_0)}.
	\]
	If we allow the interaction picture and Schrodinger picture to coincide at $t=t_0$, $a(t_0)$ can be 
	interpreted just as time independent $a$. Similarly, we have
	\[
		a^\dag(t) = a^\dag e^{i\omega (t-t_0)}.
	\]
	Therefore we may write
	\[
		H_I^1(t) = f(t)e^{-i\omega (t-t_0)}a+f^*(t)e^{i\omega (t-t_0)}a^\dag
	\]
	and the differential equation for $U_I(t,t_0)$ becomes
	\[
		\diff[U_I(t,t_0)]{t} = -\frac{i}{\h} \blr{f(t)e^{-i\omega (t-t_0)}a+f^*(t)e^{i\omega (t-t_0)}a^\dag} U_I(t,t_0)
	\]
	Since $a$ and $a^\dag$ are stationary, we should be able to solve the differential equation as
	\[
		U_I(t,t_0) = \exp\blr{-\frac{i}{\h}\int_{t_0}^t dt'\ f(t)e^{-i\omega (t-t_0)}a+f^*(t)e^{i\omega (t-t_0)}a^\dag}
	\]
	We would like the evolution operator at $t_0$ to be $U(t_0,t_0) = \mathbf 1$, which can be achieved by the 
	exponential term if $f(t_0)$ is finite. Otherwise, we need the correct initial condition for $U(t_0,t_0)$.
	\\
	\\
	\emph{I am unsure how to proceed here. In one term I have an $e^{i\omega t_0}$ while for the other
	$e^{-i\omega t_0}$. Can't seem to be able to figure out this $\beta(t,t_0)$ phase}...
	\\
	\\
	% c)
	\item
	Hence, combine with the result of Q1 to show that if the system starts at initial time $t_0$ in a coherent state
	$\ket\alpha$, then the subsequent time evolution is
	\[
		\ket{\Psi(t)} = e^{i\tilde\beta(t,t_0)}\ket{\alpha+\zeta(t,t_0)}
	\]
	where $\tilde\beta(t,t_0)$ is a phase. 
	\\
	\\
	Comment: This means that the coherent state evolves in time in a very simple manner: just a multiplicative phase
	factor, and the complex parameters $\alpha$ characterizing the initial coherent state evolves as 
	$\alpha\to\alpha +\zeta(t,t_0)$. 
	\\
	\\
	\\
	If we now apply the time evolution operator in the interaction picture on the coherent state $\ket\alpha$
	\[
		U_I(t,t_0)\ket\alpha = e^{i\beta(t,t_0)}e^{-\zeta^*(t,t_0)a+\zeta(t,t_0)a^\dag}\ket{\alpha} 
	\]
	we see from 1(d) 
	\ba
		U_I(t,t_0)\ket\alpha &= e^{i\beta(t,t_0)}e^{\frac{1}{2}(\alpha^*\zeta(t,t_0)-\zeta(t,t_0)^*\alpha)}
		\ket{\alpha+\zeta(t,t_0)}\\
		& = e^{i\tilde\beta(t,t_0)}\ket{\alpha+\zeta(t,t_0)}
	\ea
	\eenum
	
% 3 ------------------------------------------------------------------------------------------------------------------------------------------------------
\item{Casimir Effect}
\\
\\
The Riemann zeta function is defined
\[
	\zeta(z) = \sum_{n=1}^\infty \frac{1}{n^z}
\]
and the sum converges for $\text{Re}(z)>1$. 
	
	\benum
	% (a)
	\item
	The Riemann zeta function has an analytic continuation in terms of the following integral representation:
	\[
		\zeta(z) = \frac{1}{\Gamma(z)}\blr{\int_0^\infty dt\ t^{z-1}e^{-t/2}\plr{\frac{1}{2\sinh(t/2)}-\frac{1}{t}
		+\frac{t}{24}-\frac{7t^3}{5760}}}-\frac{1}{\Gamma(z)}\int_0^\infty dt\ t^{z-1}e^{-t/2}\plr{-\frac{1}{t}
		+\frac{t}{24}-\frac{7t^3}{5760}}
	\]
	Use the expression for the gamma function, $\Gamma(z) = \int_0^\infty dt\ e^{-t}t^{z-1}$, and 
	$z\Gamma(z)=\Gamma(z+1)$, to evaluate the terms in the second integral, to obtain the following
	representation:
	\[
	\zeta(z) = \frac{1}{\Gamma(z)}\blr{\int_0^\infty dt\ t^{z-1}e^{-t/2}\plr{\frac{1}{2\sinh(t/2)}-\frac{1}{t}
		+\frac{t}{24}-\frac{7t^3}{5760}}}
		+\frac{2^{z-1}}{z-1}-\frac{z2^{z+1}}{24}+\frac{7z(z+1)(z+2)2^{z+3}}{5760}
	\]
	We can now evaluate the zeta function at $z=-3$, because the remaining integral is finite at
	$z=-3$ , and $1/\Gamma(-3) = 0$. Hence show that $\zeta(-3) = \frac{1}{120}$. Note how weird this 
	result looks if you look at the original series! 
	\\
	\\
	\\
	With substitution $2u = t$ we intregrate
	\[
		\int_0^\infty dt\ t^{z-1}e^{-t/2}\plr{-\frac{1}{t} +\frac{t}{24}-\frac{7t^3}{5760}} 
		= 2^z\int_0^\infty du\ u^{z-1}e^{-u}\plr{-\frac{1}{2u}+\frac{2u}{24}-\frac{56u^3}{5760}}.
	\]
	First term:
	\[
		-2^{z-1}\int_0^\infty du\ u^{z-2}e^{-u}= -\frac{2^{z-1}}{z-1}\Gamma(z)
	\] 	
	Second term: 
	\[
		\frac{2^{z+1}}{24}\int_0^\infty du\ u^{z}e^{-u}= \frac{z2^{z+1}}{24}\Gamma(z)
	\]
	Third term:
	\[
		-\frac{2^{z+3}(7)}{5760}\int_0^\infty du\ u^{z+2}e^{-u}= -\frac{7z(z+1)(z+2)2^{z+3}}{5760}\Gamma(z)
	\]
	Putting all these results together we arrive at
	\[
	\zeta(z) = \frac{1}{\Gamma(z)}\blr{\int_0^\infty dt\ t^{z-1}e^{-t/2}\plr{\frac{1}{2\sinh(t/2)}-\frac{1}{t}
		+\frac{t}{24}-\frac{7t^3}{5760}}}
		+\frac{2^{z-1}}{z-1}-\frac{z2^{z+1}}{24}+\frac{7z(z+1)(z+2)2^{z+3}}{5760}.
	\]
	Now we may evaluate this function at $\zeta(-3)$:
	\ba
		\zeta(-3) &= \frac{2^{-4}}{-4}-\frac{(-3)2^{-2}}{24}+\frac{7(-3)(-2)(-1)2^{0}}{5760}\\
		& = -\frac{1}{64}+\frac{1}{32}-\frac{21}{2880}\\
		& = \frac{1}{120}
	\ea
	\\
	\\
	% (b)
	\item
	The Casimir effect is a remarkable quantum mechanical phenomenon whereby two parallel mirrors, in vacuum,
	attract one another because the mirrors influence the allowed quantum vacuum fluctuations between the
	mirrors. The vacuum energy per unit surface area in the region between the mirrors is (the factor of 2 is
	for the two physical polarizations of the light field)
	\ba
		\frac{E}{A} &= 2\sum \frac{1}{2}\h\omega \\
		& = \h c \sum_{n=1}^\infty \int \frac{d^2k_\perp}{(2\pi)^2}\sqrt{\pfrac{n\pi}{L}^2+k^2_\perp}\\
		& = \h c\sum_{n=1}^\infty \int \frac{d^2k_\perp}{(2\pi)^2}\frac{1}{\Gamma(-1/2)}\int_0^\infty \frac{dt}			
		{t^{3/2}} e^{-t\plr{\pfrac{n\pi}{L}^2+k^2_\perp}}
	\ea
	Do the (Gaussian) $k_\perp$ integral, and then the (gamma function) $t$ integral, and use the result from 
	part (a) to show that the force per unit area is 
	\[
		\frac{F}{A} = -\frac{\h c\pi^2}{2L^4}\zeta(-3) = -\frac{\h c\pi^2}{240L^2}
	\]
	Note that this tiny force has been recently measured with high precision [see papers by Lamoreaux, and
	by Mohideen and Roy \href{a}{http://physics.aps.org/story/v2/st28} ], confirming both the dependence 	on the 
	mirror separation $L$, and the
	numerical coefficient (which is essentially given by $\zeta(-3)$; actually they used a slightly different geometry, 
	but it is only a small correction to this idealized parallel plate example). 
	\\
	\\
	\\
	Starting with the Gaussian integral over all space
	\[
		 \int d^2k_\perp e^{-tk^2_\perp} = \plr{\sqrt\frac{\pi}{t}}^2 = \frac{\pi}{t}
	\]
	and substituting we have the $t$ integral
	\[
		T = \pi \int_0^\infty dt\ t^{-5/2} e^{-t\pfrac{n\pi}{L}^2}
	\]
	With substitution $ u = t\pfrac{n\pi}{L}^2$ we have
	\ba
		T & = \pi \pfrac{L}{n\pi}^2\pfrac{n\pi}{L}^{5} \int _0^\infty du\ u^{-5/2}e^{-u} \\
		&= \pi \pfrac{n\pi}{L}^{3} \Gamma\plr{-\frac{3}{2}} \\
		& = \pi \pfrac{n\pi}{L}^{3} \frac{\Gamma\plr{-\frac{1}{2}}}{\plr{-\frac{3}{2}}}.
	\ea
	Substituting into original expression
	\[
		\frac{E}{A} = -\frac{\h c\pi^2}{6L^3}\sum_{n=1}^\infty n^{3} =  -\frac{\h c\pi^2}{6L^2} \zeta(-3).
	\]
	To find the force, we use $F = -\del V = -\diff[E]{L}$ so that we finally arrive at
	\[
		\frac{F}{A}= -\diff{L}\pfrac{E}{A} = -\frac{\h c\pi^2}{2L^4}\zeta(-3) = -\frac{\h c\pi^2}{240L^4}.
	\]
		
	\eenum
\eenum
\end{document}