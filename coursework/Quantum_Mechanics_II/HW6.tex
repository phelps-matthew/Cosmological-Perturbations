\documentclass[10pt,letterpaper]{article}
\usepackage{mymacros}

\title{\begin{spacing}{1.2}Quantum Mechanics II\\HW 6\end{spacing}}
\author{Matthew Phelps}
\date{Due: Oct. 28}

\begin{document}
\maketitle

\benum
% #1 -----------------------------------------------------------------------------------------------------------------------------------------------------------------
  	\item{One Dimensional Scattering Using Green's Functions}
	\\
	\\
	\benum
	% (a)
	\item
	Use a Fourier transform and contour integration to show that the Green's function of the one-dimensional
	Helmholtz operator, $\plr{\difff{}{*2x}+k^2}$ is 
	\[
		G_k(x,x') = -\frac{i}{2k}e^{ik|x-x'|}
	\]
	\\
	\\
	The Green's function operator is defined as
	\be\label{1}
		\plr{\del^2_x +k^2}G_k(x,x') = \delta(x-x').
	\ee
	If we apply a translation
	\[
		x\to \tilde x = x+a\qquad x' \to\tilde x' = x'+a
	\]
	eq. \eqref 1 becomes
	\[
		\plr{\del^2_{x}+k^2}G_k(\tilde x,\tilde x') = \delta(x-x').
	\]
	Since the Helmholtz operator and delta function are invariant under translation, we know that the solution to 
	$G_k(x,x')$ can be (but not strictly limited to) translationally invariant. Thus we assume a form for the 
	Green's function as
	\[
		G_k(x,x') = G_k(x-x') \equiv G_k(x).
	\]
	\\
	Now we do a Fourier transform of \eqref 1 to the $k'=p/\h$ space. This is motivated by the fact that
	the eigenfunctions of the Helmholtz operator are momentum eigenkets - thus by decomposing
	the Green's function in the momentum basis, we may easily act with the Helmholtz operator. 
	\[
		\frac{1}{\sqrt{2\pi}}\int_{-\infty}^{\infty} dx\ e^{-ik'x}\plr{\del^2_x +k^2}G_k(x) 
		= \frac{1}{\sqrt{2\pi}}\int_{-\infty}^{\infty} dx\ e^{-ik'x}\delta(x-x').
	\]
	Applying the hermitian differentiation operator to the left and noting that the remaining integral is just
	the Green's function in the momentum representation we have
	\[
		(k^2-k'^2)G_k(k') = \frac{1}{\sqrt{2\pi}}
	\]
	or
	\[
		G_k(k') = \frac{1}{\sqrt{2\pi}(k^2-k'^2)}.
	\]	
	Now we would like to revert back to the position basis, so we take the inverse Fourier transform
	\[
		G_k(x) = \frac{1}{2\pi}\int_{-\infty}^{\infty} dk'\ \frac{e^{ik'x}}{(k+k')(k-k')}.
	\]
	This integral is evaluated by Cauchy's integral formula
	\[
		\frac{1}{2\pi i} \oint dz\ \frac{f(z)}{(z-z_0)} = \begin{cases}f(z_0)\quad &z_0\ \text{within contour}\\
		0\quad\ &z_0\ \text{exterior to contour}\end{cases}
	\]
	
	There are two poles on the real axis at $k' = \pm k$. For $k'>0$ we will close above and only 
	enclose the pole at $k'=+k$
	\[
		G_k(x) = -\frac{1}{2\pi}\oint dk'\  \pfrac{e^{ik'x}}{k'+k}\frac{1}{k'-k} = -i\frac{e^{ikx}}{2k}.
	\]
	For $k'<0$ we close below only enclosing the pole $k'=-k$
	\[
		G_k(x) = \frac{1}{2\pi}\oint dk'\  \pfrac{e^{ik'x}}{k'-k}\frac{1}{k'+k} = -i\frac{e^{-ikx}}{2k}.
	\]
	As such, we may combine both results into an absolute value 
	\[
		G_k(x) = -\frac{i}{2k}e^{ik|x|}.
	\]
	Recalling that $x\equiv x-x'$ we finally have
	\[
		G_k(x,x')= -\frac{i}{2k}e^{ik|x-x'|}
	\]
	\\
	\\
	
	% (b)
	\item
	Use the result in (a) to write down the first Born approximation for the scattering amplitude in one-dimension,
	and hence show that the Born approximation for the reflection probability is
	\[
		R \approx \pfrac{m}{\h^2 k}^2\left|\int_{-\infty}^{\infty} dx'\ e^{2ikx'}V(x')\right|^2
	\]
	\\
	\\
	Given the Schrodinger equation
	\[
		\plr{\del_x^2+k^2}\psi = \frac{2m}{\h^2} V\psi
	\]
	we may use the Green's function in (a) to cast this into an integral equation for $\psi$
	\[
		\psi(x) = \frac{2m}{\h^2}\int_{-\infty}^\infty dx'\ G_k(|x-x'|)V(x')\psi(x').
	\]
	To be perfectly general, we must include the homogeneous solution $\psi_0$ to the Helmholtz equation. This will
	allow us to fit the appropriate boundary conditions
	\[
		\psi(x) = \psi_0(x) + \frac{2m}{\h^2}\int_{-\infty}^\infty dx'\ G_k(|x-x'|)V(x')\psi(x').
	\]
	We see that we can iterate this equation to expand $\psi$ as a perturbative series in powers of
	the potential $V$. In the limit that $V=0$, we expect our particle to propagate freely and thus
	\[
		\psi_0 = Ae^{ikx}\qquad\text{(incoming wave)}.
	\]
	Now, to zeroth order in $V$, $\psi(x) = \psi_0(x)$. Thus, to 1st order in $V$ we substitute the zeroth order
	$\psi(x)$ into the integral to obtain
	\[
		\psi(x) \approx Ae^{ikx}+\frac{2Am}{\h^2}\int_{-\infty}^\infty dx'\  G_k(|x-x'|)V(x')e^{ikx'}.
	\]
	Inputting the Green's function, we can write this equation in a more suitable form that clarifies the notion of reflected 
	and transmitted waves:
	\ba
		\psi(x) &\approx Ae^{ikx}-\frac{iAm}{\h^2k}\int_{-\infty}^x dx'\ e^{ik(x-x')}V(x')e^{ikx'}
		-\frac{iAm}{\h^2k}\int_{x}^\infty dx'\ e^{-ik(x-x')}V(x')e^{ikx'} \\
		& =  Ae^{ikx}-e^{ikx}\plr{\frac{iAm}{\h^2k}\int_{-\infty}^x dx'\ e^{-ikx'}V(x')e^{ikx'}}
		-e^{-ikx}\plr{\frac{iAm}{\h^2k}\int_{x}^\infty dx'\ e^{ikx'}V(x')e^{ikx'}}.
	\ea
	In the limit that $x\to\pm\infty$ (or far away from the potential) the remaining terms are either the transmitted or 	
	reflected waves. For the reflection probability, we take $x\to -\infty$ and compute ratio of incoming/reflected wave 
	probability amplitudes
	\ba
		R&\approx\pfrac{|A|^2\pfrac{m}{\h^2k}^2\left|\int_{-\infty}^\infty dx'\ e^{ikx'}V(x')e^{ikx'}\right|^2}{|A|^2}\\
		&= \pfrac{m}{\h^2 k}^2\left|\int_{-\infty}^{\infty} dx'\ e^{2ikx'}V(x')\right|^2
	\ea
	% (c
	\item
	Evaluate this Born approximation reflection probability for the two cases:
		\benum
		% (i)
		\item
		an attractive delta function well: $V(x) = -\lambda\delta(x)$
		\\
		\\
		For the attractive delta function 
		\[
			\left|\int_{-\infty}^{\infty} dx'\ e^{2ikx'}(-\lambda\delta(x))\right|^2 =\lambda^2
		\]
		hence
		\[
			R\approx \pfrac{\lambda m}{\h^2k}^2 = \frac{\lambda^2m}{2\h^2 E}.
		\]
		% (ii)
		\item
		a finite square well of depth $V_0$ and width $L$.
		\\
		\\
		For the finite square well
		\[
			\int_{-L/2}^{L/2} dx'\ e^{2ikx'}(-V_0) = -\frac{V_0}{k}\sin(kL)
		\]
		hence
		\[
			R\approx \pfrac{mV_0}{\h^2k^2}^2\sin^2(kL) = \pfrac{V_0}{2E}^2\sin^2(kL)
		\]

		\eenum
	
	% (d)
	\item
	Compare your answers in (c) with the exact answers [look them up in another QM book: e.g. Griffiths
	QM equations (2.141) and (2.169), in the second edition]. Comment on how good the Born approximation
	answers are, and in what physical region they are good.
	\\
	\\
	Attractive delta function potential:
	\[
		R = \frac{ \pfrac{\lambda^2m}{2\h^2 E}}{1+ \pfrac{\lambda^2m}{2\h^2 E}};
		\qquad R_{Born} = \frac{\lambda^2m}{2\h^2 E}
	\]
	The Born approximation overestimates the reflection probability by an error 
	\[
		\varepsilon \propto  R_{Born}.
	\]
	Strictly speaking, the criteria for a ``good" Born approximation is when the dimensionless ratio 
	\[
		\frac{\lambda^2m}{2\h^2 E}\ll 1.
	\]This delta potential 
	seems a little bit more complicated since the strength of the potential energy is related to not only $\lambda$,
	but also $m$ and $\h$. However we can certainly say that the approximation gets better as the energy gets 	
	higher.
	\\
	\\
	\\
	Finite square well: 
	\[
		R = \frac{\frac{V_0^2}{4E(E+V_0)}\sin^2\plr{kL\sqrt{1+\frac{V_0}{E}}}}{1
		+\frac{V_0^2}{4E(E+V_0)}\sin^2\plr{kL\sqrt{1+\frac{V_0}{E}}}};
		\qquad R_{Born} =  \frac{V_0^2}{4E^2}\sin^2(kL)
	\]
	Again, the Born approximation overestimates the true probability, but this time with an error proportional to
	\[
		\varepsilon \propto \frac{V_0}{E}. 
	\]
	If $E\gg V_0$ we see both the sine argument and its pre-factor will approach the Born approximation. 
	\\
	\\
	It appears that for both potentials, the Born approximation is most ideal for energies much greater than the 
	strength of the potential. 
	\eenum
	
	
% 2 ----------------------------------------------------------------------------------------------------------------------------------------------------
\item{Three Dimensional Born Approximation}
\\
	\benum
	% (a)
	\item 
	Compute the Born approximation for the following radially symmetric potentials:
		\benum
		\item 
		radial square well of depth $V_0$ and radius $a$.
		\item 
		Gaussian well: $V(r) = -V_0\exp-\frac{r^2}{a^2}$
		\item 
		$V(r) = -\frac{V_0}{\plr{1+\frac{r^2}{a^2}}^2}$
		\\
		\eenum
		
		In the Born approximation, the scattering amplitude may be approximated by
		\[
			|f(\theta,\phi)|^2 = \left|-\frac{m}{2\pi\h^2}\int d^3\vect r \ e^{-i\vect q\cdot\vect r}V(\vect r)\right|^2
		\]
		with momentum transfer
		\[
			\vect q = \vect k_f-\vect k_i;\qquad q = 2k\sin(\theta/2).
		\]
		For a spherically symmetric potential, this simplifies to 
		\[
			|f(\theta)|^2 = \left|-\frac{2m}{\h^2}\int_0^\infty dr\ r\frac{\sin(qr)}{q}V(r)\right|^2
		\]
		with $q$ containing the $\theta$ and $E$ dependence. 
		\\
		\benum
		
		% (i)
		\item 
		For the radial square well of depth $V_0$ radius $a$
		\ba
			f(\theta) &= \frac{2mV_0}{\h^2} \int_0^a dr\  r\frac{\sin(qr)}{q}\\
			& = \frac{2mV_0}{\h^2}\plr{\frac{\sin(aq)-aq\cos(aq)}{q^3}}
		\ea
		thus
		\[
			|f(\theta)|^2 = 4a^2\pfrac{mV_0a^2}{\h^2}^2\plr{\frac{\sin(aq)-aq\cos(aq)}{(qa)^3}}^2.
		\]
		\\
		
		% (ii)
		\item 
		For the Gaussian well
		\ba
			f(\theta) &= \frac{2mV_0}{\h^2} \int_0^\infty dr\  r\frac{\sin(qr)}{q}\exp{\pfrac{-r^2}{a^2}}\\
			& = \frac{2mV_0}{\h^2}\pfrac{\sqrt\pi a^3}{4}\exp\plr{-\frac{a^2q^2}{4}}\\
			& = \frac{2mV_0}{\h^2}\pfrac{\sqrt\pi a^3}{4}\exp\blr{-a^2k^2\sin(\theta/2)^2}
		\ea
		thus
		\[
			|f(\theta)|^2 = \frac{\pi a^2}{4}\pfrac{mV_0a^2}{\h^2}^2\exp\blr{-2a^2k^2\sin(\theta/2)^2}.
		\]
		\\
		% (iii)
		\item 
		For the last well
		\ba
			f(\theta) &= \frac{2mV_0}{\h^2} \int_0^\infty dr\  r\frac{\sin(qr)}{q}\plr{1+\frac{r^2}{a^2}}^{-2}\\
			& = \frac{2mV_0}{\h^2}\pfrac{\pi a^3}{4}\exp\plr{-a|q||}
		\ea
		thus
		\[
			|f(\theta)|^2 = \frac{\pi^2 a^2}{4}\pfrac{mV_0a^2}{\h^2}^2\exp\blr{-4ak|\sin(\theta/2)|}.
		\]
		\\
		\eenum
	
	% (b)
	\item
	In each of the cases in (a), write the leading low energy limit of the scattering amplitude.
	\\
	\\
	In the low energy limit, we may take $k\sim q \to 0$ before integrating such that
	\[
		f_{low}(\theta) = -\frac{2m}{\h^2}\int dr\ r^2 V(r).
	\]
	Or we may integrate and then take the resulting limit. If the strength of the potential is characterized by
	some length parameter $a$, then the low energy limit can also be taken for $ka\ll 1$. From this point of 
	view, the wavelength of the particle must be much larger than the effective range of the potential. 
	\\
	\benum
	\item
	For the radial square well, the leading order approximation is easiest to calculate by taking the integral
	\ba
		|f_{low}(\theta)|^2 &= \pfrac{2mV_0}{\h^2}^2\blr{\int_0^a dr\ r^2}^2\\
		& = \pfrac{2mV_0}{\h^2}^2\pfrac{a^3}{3}^2	
	\ea
	thus
	\[
		|f_{low}(\theta)|^2 = \frac{4a^2}{9}\pfrac{mV_0a^2}{\h^2}^2.
	\]
	\\
	
	\item
	For the Gaussian, we can simply use the Born approximation we found and take the argument in the exponential to 	
	zero. Thus
	\[
		|f_{low}(\theta)|^2 = \frac{\pi a^2}{4}\pfrac{mV_0a^2}{\h^2}^2.
	\]
	\\
	
	\item
	For the last well we again take the exponential argument to zero
	\[
		|f_{low}(\theta)|^2 = \frac{\pi^2 a^2}{4}\pfrac{mV_0a^2}{\h^2}^2.
	\]
	\\
	\eenum
	% (c)
	\item 
	In each of the cases in (a), make a 3d plot of $|f_k^{Born}(\theta,\phi)|^2$, as a function of both
	$k$ and $\theta$, and show that at large $k$ the scattering is concentrated at a small angle $\theta$. 
	What determines this small angle?
	\\
	\\
	Each well was plotted for three values of the dimensionless quantity $\alpha = ka$. The Mathematica
	function \emph{SphericalPlot3D} plots a radius equal to the magnitude of the function $|f(\theta,\phi)|^2$
	at each point $(\theta,\phi)$. 
	\\
	\\
	To gain an approximation to the forward cone of small angle $\theta$ at high scattering energies, we can look at
	the original integral
	\[
		|f(\theta)|^2 = \left|-\frac{2m}{\h^2}\int_0^\infty dr\ r^2\frac{\sin(qr)}{qr}V(r)\right|^2.
	\]
	We note that the integrand has a sinc function and so the only significant contributions to the sum come from
	the range in which the argument of the sinc function is less than $\pi$
	\[
		qr = 2kr\sin(\theta/2) < \pi.
	\]
	For large $k$, the only way to limit the argument to less than $\pi$ is if $r\ll1$ or $\sin(\theta/2)\ll 1$. For 
	$r\ll 1$ however, the linear factor of $r$ in the remaining part of the integrand (as well as the potenial) would 		
	dominate and so
	we would have little contribution. But for $\sin(\theta/2)\ll1$ or, perhaps more precisely $\sin(\theta/2)k \sim
	 \mathcal O(1)$ we may meet the desired inequality. To first order then
	 \[
	 	kr\theta < \pi.
	\]
	For a potential with effective range $a$, values of $r>a$ are insignificant due to $V(r)$ and thus we
	finally arrive at an approximation for the small forward cone:
	\[
		\theta < \frac{\pi}{ka}.
	\]
	\\
	\\
	\benum
	\item{Radial Square Well:\qquad $\ds\frac{|f(\theta)|^2}{a^2} \sim 4\plr{\frac{\sin(aq)-aq\cos(aq)}{(qa)^3}}^2$}
	\\
	\figg[width=100mm]{6_11.png}
	\figg[width=100mm]{6_12.png}
	\figg[width=100mm]{6_13.png}
	\phantom{}\phantom{}\phantom{}\phantom{}\phantom{}\phantom{}
	\item{Guassian Well:\qquad $\ds\frac{|f(\theta)|^2}{a^2} \sim 
	\frac{\pi}{4}\exp\blr{-2a^2k^2\sin(\theta/2)^2}$}
	\\
	\\
	\figg[width=100mm]{6_21.png}
	\figg[width=100mm]{6_22.png}
	\figg[width=100mm]{6_23.png}
	
	\phantom{}\phantom{}\phantom{}\phantom{}\phantom{}\phantom{}
	\item{Guassian-Like Well:\qquad $\ds\frac{|f(\theta)|^2}{a^2} \sim 
	\frac{\pi^2}{4}\exp\blr{-4ak|\sin(\theta/2)|}$}
	\\
	\\
	\figg[width=100mm]{6_31.png}
	\figg[width=100mm]{6_32.png}
	\figg[width=100mm]{6_33.png}
	\eenum
	
	% (d)
	\item
	In each of the cases in (a), state the dimensionless quantity that should be small for the low energy
	Born approximation to be good, and for the high energy Born approximation to be good. 
	\\
	\\
	For the low energy Born approximation, we see that in our various well calculations if  $ak\ll 1$, the low energy 		approximation approaches
	the Born approximation. At the same time however, if the quantity
	\[
		\frac{mV_0a^2}{\h^2}\ll1 
	\]
	than the low energy approximation is also good. If we are asking when the ``regular" Born approximation is good
	for low energies, we can answer this differently. Looking at the Lippman-Schwinger equation we can see that
	the Born approximation is good if the correction term is much smaller than the homogeneous term. Since the
	scattered term is largest at the origin, composing our condition at this point sets the upper bound:
	\[
		\frac{|\psi_{sc}(0)|^2}{|\psi_0(0)|^2} = |\psi_{sc}(0)|^2 \ll 1\quad \text{or}\quad |\psi_{sc}(0)| \ll 1.
	\]
	From
	\[
		\psi_{sc}(0) = \frac{2m}{4\pi \h^2} \int d^3\vect r'\ \frac{e^{ikr'}}{r'}V(\vect r')e^{-i\vect k_i\cdot\vect r'}
	\]
	we see that for a spherical potential $V(r)$ we may take $\vect k_i$ to lie along the $z$-direction in the integration
	and easily integrate out the $\theta '$ dependence exactly in the same fashion as we derived the scattering amplitude
	earlier. Hence
	\be\label{2}
		\frac{2m}{\h^2}\left|\int dr'\ e^{ikr'}\frac{\sin(kr')}{k}V(r')\right|\ll 1.
	\ee
	For low energy, $k\ll 1$ and so $e^{ikr}\approx 1$ and $\sin(kr)\approx kr$. The finite potential range $r_0$
	takes care of the values in which $r'\gg k$. This leads us to
	\[
		\frac{2m}{\h^2}\left|\int dr'\ V(r')\right|\ll 1.
	\]
	For a well of depth $V_0$ and range $a$ a reasonble upper bound on the integral is $\left|\int dr'\ r'V(r')\right| 
	\lesssim V_0a^2$ so finally we arrive at the result:
	\[
		\frac{mV_0a^2}{\h^2}\ll 1.
	\]
	\\
	\\
	For high energies we first rearrange the integrand term of \eqref 2
	\[
		e^{ikr'}\sin(kr') = \frac{e^{2ikr'}-1}{2i}
	\]
	and see that the exponential term oscillates rapidly, with no hope of stabilization. Thus we can drop the term
	from the integral (along with the factor of 2), leaving
	\[
		\frac{m}{\h^2 k}\left|\int dr'\ V(r')\right| \ll 1.
	\]
	With similar reasoning as before we have
	\[
		\left|\int dr'\ V(r')\right| \lesssim V_0a
	\] 
	and so the high energy criteria becomes
	\[
		\frac{mV_0a^2}{\h^2} \ll ka.
	\]
	\\
	\eenum

% 3 ------------------------------------------------------------------------------------------------------------------------------------------------------
	\item
	Consider an elastic scattering process as a transition from an initial and final state of momentum $\h \vect k$
	and $\h \vect k'$, respectively. \\ \\
	Use Fermi's Golden Rule, with the density of states for momentum states in the continuum to write the
	probability per unit time for scattering into the angle $d\Omega$, and show that you recover the Born 
	approximation expression.
	\\
	\\
	Given a periodic time dependent perturbation $H_1(t) = Ve^{i\omega t}$, Fermi's Golden Rule gives
	the transition rate from initial energy eigenstate $\ket i$ to final state $\ket f$ as
	\[
		R_{i\to f} = \frac{2\pi}{\h}|\braket{f|V|i}|^2\delta(\h\omega-(E_f-E_i)).
	\]
	In the context of this problem, our perturbation is time independent so $\omega = 0$. Since a
	detector is finite, we seek the range of momentum states $\h \vect k'$ that lie within the solid
	angle $d\Omega$ at a given $(\theta,\phi)$
	\[
		R_{i\to f\in d\Omega} = \sum_{\vect k'\in d\Omega} \frac{2\pi}{\h}
		|\braket{\vect k|V(\vect r)|\vect k'}|^2\delta(E_k-E_{k'})
	\]
	In the continuum this becomes an integral over $d^3\vect k'$
	\ba
		R_{i\to f} &=\frac{2\pi}{\h} \int_{\vect k'\in d\Omega} d^3\vect k'\
		|\braket{\vect k|V(\vect r)|\vect k'}|^2\delta(E_k-E_{k'})\\
		& = \frac{4\pi m}{\h^3} \int_0^\infty dk'\ d\Omega\ k'^2 |\braket{\vect k|V(\vect r)|\vect k'}|^2
		\delta(k^2-k'^2).
	\ea
	The integral is infinite because we are looking for all magnitudes of $k'$ who's direction lies in the differential
	solid angle $d\Omega$. However, from Fermi's golden rule, only those of which $k'^2=k^2$ filter through:
	\ba
		R_{i\to f}&=d\Omega\ \frac{2\pi m}{\h^3} \int_0^\infty d(k'^2)\  \sqrt{k'^2} |\braket{\vect k|V(\vect r)|\vect k'}|^2 
		\delta(k^2-k'^2)\\
		&= d\Omega\ \frac{2\pi mk}{\h^3}|\braket{\vect k|V(\vect r)|\vect k'}|^2 .
	\ea
	We can now use the incoming probability flux
	\[	
		j_{inc} =\frac{\h k}{m}\pfrac{1}{2\pi\h}^3
	\]
	to form the rate of detected probability in cross section $d\Omega$ 
	\ba
		\frac{R_{i\to f}}{j_{inc}} &= d\Omega\  (2\pi)^4m^2  |\braket{\vect k|V(\vect r)|\vect k'}|^2\\
		& = d\Omega\  (2\pi)^4m^2\h^2  |\braket{\vect p|V(\vect r)|\vect p'}|^2\\
		& = \frac{d\sigma}{d\Omega}d\Omega .
	\ea
	Minding some Fourier Transform constants, alas we recover the Born approximation (up to a phase constant)
	\ba
		\frac{d\sigma}{d\Omega} &=  (2\pi)^4m^2\h^2  |\braket{\vect p|V(\vect r)|\vect p'}|^2\\
		& = \left|\frac{m}{2\pi\h^2}\int d^3\vect r \ e^{-i\vect q\cdot\vect r}V(\vect r)\right|^2.
	\ea
\eenum
\end{document}