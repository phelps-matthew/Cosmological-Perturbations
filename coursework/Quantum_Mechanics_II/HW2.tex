\documentclass[11pt,letterpaper]{article}
\usepackage{macroshw}

\title{Quantum Mechanics II\\HW 2}
\author{Matthew Phelps}
\date{Due: Sept. 16}

\begin{document}
\maketitle

\benum
% #1 --------------------------------------------------------------------------------------------------------------------------------------------------------------------------------------
  	\item 
	\benum
		% (a)
		\item
		For the 1d Schrodinger equation with potential $V(x)$, use the standard WKB expansion
		\[
			\psi(x) = \exp\blr{\frac{i}{\h}(S_0+\h S_1+\h^2S_2+\h^3S_3+...)}
		\]
		to find expressions for $S_0$, $S_1$, $S_2$ and $S_3$, and show that both $S_1$ and $S_3$ are in fact total derivatives.
		\\
		\\
		\\
		Denoting $S(x)$ as
		\[
			S(x) = S_0(x)+\h S_1(x)+\h^2S_2(x)+h^3S_3(x)+...
		\]
		and $\psi(x)$ as
		\[
			\psi(x) = \exp\blr{\frac{i}{\h}S(x)}
		\]
		(no loss of generality for $S(x)$ complex), our TISE reads
		\be\label{1}
			-(S')^2+i\h S''+[p(x)]^2= 0
		\ee
		where
		\[
			p(x) \equiv \sqrt{E-V(x)}.
		\]
		If we now substitute the expansion of $S(x)$ into \eqref 1, keeping terms up to $\mathcal{O}(\h^3)$ we have
		\ba
			0= &[p(x)]^2-(S_0')^2 \\
			+&\h[-2S_0'S_1'+iS_0'']\\
			+&\h^2[-2S_0'S_2'+iS_1''-(S_1')^2]\\
			+&\h^3[-2S_0'S_3'-2S_1'S_2'+iS_2'']
		\ea
		As another power series in $\h$, we equate each ``coefficient" of $\h$ to zero identically. Thus 
		\ba
			&\phantom{}\ [p(x)]^2=(S_0')^2 \\
		 	\to \ & S_0(x) = \pm \int^x dx'\ p(x')
		\ea
		\ba
			&2S_0'S_1' = iS_0'' \\
			&S_1' = \frac{i}{2}\frac{p'}{p}\\
			\to\ &S_1' = \blr{\frac{i}{2}\ln p}'
		\ea 
		\ba
			&2S_0'S_2'= iS_1''-(S_1')^2\\
			&S_2' = \frac{1}{2S_0'}\blr{iS_1''-(S_1')^2}\\
			\to\ & S_2' = \frac{1}{4p}\blr{\frac{3}{2}\pfrac{p'}{p}^2-\frac{p''}{p}}
		\ea
		\ba
			& 2S_0'S_3'=-2S_1'S_2'+iS_2''\\
			& S_3' = \frac{1}{2S_0'}\blr{-2S_1'S_2'+iS_2''} \\
			& S_3' = \frac{3i}{4}\blr{\frac{p'p''}{p^4}-\frac{p'^3}{p^5}-\frac{1}{6}\frac{p'''}{p^3}}\\
			\to\ & S_3' = \frac{i}{16}\blr{\frac{3p'^2-2pp''}{p^4}}'
		\ea
		We see that $S_1$ and $S_3$ can be written in explicit form (in terms of $V(x)$ and its higher derivatives) while $S_0$ and $S_2$ 
		are given as integral forms.  
		\\
		\\
		% (b)
		\item
		Now make a different expansion, writing
		\[
			\psi(x) = A(x)\exp\blr{\frac{i}{h}W(x)}
		\]
		where the amplitude function $A(x)$ and phase function $W(x)$ are both real. Separate the Schrodinger equation into real and 
		imaginary parts to find an expression for the amplitude $A(x)$ in terms of the phase $W(x)$.
		\\
		\\
		\\
		Writing the TISE as
		\[
			\blr{\difff{}{*{2}x}+\pfrac{p(x)}{\h}^2}\psi = 0
		\]
		we substitute $\psi(x) = A(x)\exp\blr{\frac{i}{h}W(x)}$ to arrive at
		\[
			i\h[2A'W'+AW'']+[Ap^2-A(W')^2+\h^2A''] = 0.
		\]
		Now setting the imaginary and real parts equal to zero, we get two equations:
		\be\label{2}
			2A'W'+AW'' = 0
		\ee
		\be\label{3}
			Ap^2-AW'^2+\h^2A'' = 0
		\ee
		We can express \eqref 2 in terms of the total derivative
		\[
			\blr{\ln(A^2W')}'=0.
		\]
		Hence an equation for $A(x)$ in terms of $W(x)$ goes as
		\[
			A(x)= \frac{C}{\sqrt{W'(x)}}.
		\]
		\\
		\\
		
		% (c)
		\item
		Show that the resulting equation for $W(x)$ can be solved by an expansion of the form
		\[
			W(x) = W_0+\h^2W_2+\h^4W_4+...
		\]
		involving only terms with \emph{even} powers of $\h$. 
		\\
		\\
		\\
		First, we form the differential equation for $W(x)$ by substituting $A(x) = C/\sqrt{W'(x)}$ into \eqref 3
		\[
			\frac{p^2}{\sqrt{W'}}-(W')^{3/2}+\h^2\blr{\frac{3}{4}\frac{(W'')^2}{(W')^{5/2}}-\frac{W'''}{2(W')^{3/2}}} =0.
		\]
		We may simplify by multiplying through by $(W')^{5/2}$
		\be\label{3}
			p^2(W')^2-(W')^4+\h^2\blr{\frac{3}{4}(W'')^2-\frac{1}{2}W'W'''} = 0.
		\ee
		If we now substitute the even expansion of $W(x)$ up to $\mathcal O(\h^4)$ and equate each ``coefficient" of $\h$
		to zero, we have
		\ba
			\mathcal O(\h^0):\quad 
			&p^2W_0'^2 = W_0'^4 \\
			\to\ &W_0(x) = \int^x dx'\ p(x')
		\ea
		\ba
			\mathcal O(\h^2):\quad 
			&8p^2W_0'W_2'-16(W_0')^3W_2'+3(W_0')^2-2W_0'W_0''=0\\
			&-8p^3W_2'+3(p')^2-2pp''=0\\
			\to\ &W_2'(x) = \frac{1}{4p}\blr{\frac{3}{2}\pfrac{p'}{p}^2-\frac{p''}{p}}
		\ea
		\ba
			\mathcal O(\h^4):\quad 
			&4p^2(W_2')^2-24(W_0')^2(W_2')^2+8p^2W_0'W_4'-16(W_0')^3W_4'\\
			&\ +6W_0''W_2''-2W_2'W_0'''-2W_0'W_2'''=0\\
			&-8p^3W_2'+3(p')^2-2pp''=0\\
			\to\ &W_4'(x) = -\frac{297}{128}\frac{(p')^4}{p^7}+\frac{99}{32}\frac{(p')^2p''}{p^6}-\frac{13}{32}\frac{(p'')^2}{p^5}
			-\frac{5}{8}\frac{p'p^{(3)}}{p^5}+\frac{1}{16}\frac{p^{(4)}}{p^4}
		\ea
		\\
		If we desired, we could continue calculating even higher order terms of $\h^{2n}$. Thus we can see that an expansion of $W(x)$
		in even powers 	of $\h$ presents a solution to \eqref 3. 
		\\
		
		% (d)
		\item
		Show that your expressions for $W_0(x)$ and $W_2(x)$, and the corresponding expression for $A(x)$ to this order, are 
		consistent with your results for $S_0(x)$, $S_1(x)$, and $S_2(x)$ in (a), to $\mathcal O(\h^2)$. 
		\\
		\\
		\\
		The wavefunction can be expressed in terms of $A(x)$ and $W(x)$ as
		\ba
			\psi(x)& = A(x)\exp\blr{\frac{i}{h}W(x)}\\
			&\simeq \exp\blr{\frac{i}{h}(W_0+\h^2W_2)+\ln[(W_0'+\h^2W_2')^{-1/2}]}.
		\ea
		Alternatively, we may express it in terms of  $S(x)$ as
		\[
			\psi(x) \simeq \exp\blr{\frac{i}{\h}(S_0+\h S_1+\h^2S_2)}.
		\]
		For our results to be consistent, we must equate the arguments in the exponentials
		\[
			\frac{i}{h}(W_0+\h^2W_2)+\ln[(W_0'+\h^2W_2')^{-1/2}] = \frac{i}{\h}(S_0+\h S_1+\h^2S_2)
		\]
		or 
		\[
			F=G
		\]
		where $F$ denotes the LHS and $G$ the RHS. For $F$ we have
		\[
			F = \frac{i}{\h}W_0+ i\h W_2-\frac{1}{2}\blr{\ln(p+\h^2W_2')}
		\]
		while for $G$
		\[
			G = \frac{i}{\h}S_0 + i\h S_2 -\frac{1}{2}\ln p.
		\]
		If we note from earlier that $S_0 = W_0$ and $S_2 = W_2$ and then keep terms up to $\mathcal O(\h^2)$ such that
		\[
			\ln(p+\h^2W_2')\to \ln p.
		\]
		then we can conclude that $F=G$, thus showing consistency between results. 
		\\
		\\
		**As an aside, perhaps a better way to show the last step would be 
		\ba
			\ln(W_0'+\h^2W_2') &= \ln(W_0')+\ln\plr{1+\h^2\frac{W_2'}{W_0'}}\\
			&\approx \ln(W_0')+\h^2\frac{W_2'}{W_0'}+....
		\ea
		where, when we multiply both $F$ and $G$ by an overall $\h$ and take up to $\mathcal O(\h^2)$ only $\ln(p)$ remains.**
		\\
		\\
		\eenum
% #2 ----------------------------------------------------------------------------------------------------------------------------------------------------------------------------------
	\item
	\benum
		
		% (a)
		\item 
		Use WKB for bound states the estimate the energy for the Hamiltonian
		\[
			H = -\frac{\h^2}{2m}\difff{}{*{2}x}+\lambda x^{2N}
		\]
		where $\lambda > 0$ and $N$ is an integer. Comment on the limit $N\to \infty$. 
		\\
		\\
		\\
		For a bound state with two turning points, the WKB estimation for energy can used
		\[
			k\int_{x_1}^{x_2}dx\ \sqrt{1-\frac{V(x)}{E}} = \plr{n+\frac{1}{2}}\pi\quad (n=0,1,2,3,..)
		\]
		where
		\[
			k = \frac{\sqrt{2mE}}{\h}.
		\]
		For this problem, the two turning points are located at
		\[
			x = \pm \pfrac{E}{\lambda}^{\frac{1}{2N}}
		\]
		thus our energy estimation will be
		\[
			2k\int_0^{\pfrac{E}{\lambda}^{\frac{1}{2N}}} dx\ \sqrt{1-\frac{\lambda x^{2N}}{E}} = \plr{n+\frac{1}{2}}\pi.
		\]
		This yields energies
		\[
			E = \pfrac{\h^2 \pi}{2m}^{\frac{N}{N+1}}
			\pfrac{\Gamma\blr{\frac{1}{2}\plr{3+\frac{1}{N}}}}{\Gamma\blr{\frac{1}{2}\plr{2+\frac{1}{N}}}}^{\frac{2N}{N+1}}
			\lambda^{\frac{1}{N+1}}\plr{n+\frac{1}{2}}^{\frac{2N}{N+1}}	.	
		\]
		For $N=2$ and $\lambda = m\omega^2/2$ we get the familiar harmonic oscillator
		\[
			E = \h\omega\plr{n+\frac{1}{2}}. 
		\]
		In the limit of $N\to\infty$
		\[
			E = \frac{\h^2\pi^2n^2}{8m}\quad (n=1,2,3..)
		\]
		which is precisely the energy spectrum of the infinite square well with width $L=1$. 
		\\
		\\
		\\
		% (b)
		\item
		Use WKB to estimate the transmission probability for scattering from the potential
		\[
			V = -\lambda x^{2N}
		\]
		where $\lambda >0$ and $N$ is an integer, and for energies \textbf{below} the top of the barrier. 
		\\
		\\
		\\
		For a single barrier, the relation for the transmission probability is
		\[
			T = \exp(-\gamma)
		\]
		where
		\[
			\gamma \equiv \frac{2}{\h}\int_{x_1}^{x_2} \sqrt{2m(V(x)-E)}
		\]
		and $(x_2-x_1)$ represents the tunneling region. The points of entry/exit for this potential are located at
		\[
			x = \pm \pfrac{|E|}{\lambda}^{\frac{1}{2N}}
		\]
		thus
		\ba
			\gamma &= 4k\int_0^{\pfrac{|E|}{\lambda}^{\frac{1}{2n}}}dx\ \sqrt{1+\frac{\lambda x^{2N}}{E}}\\
			& = 2k\sqrt\pi\pfrac{|E|}{\lambda}^{\frac{1}{2N}}\frac{\Gamma\blr{\frac{1}{2}\plr{2+\frac{1}{N}}}}
			{\Gamma\blr{\frac{1}{2}\plr{3+\frac{1}{N}}}}
		\ea
		where
		\[
			k = \frac{\sqrt{2m|E|}}{\h}.
		\]
		Therefore the probability of transmission is
		\[
			T = \exp\blr{2k\sqrt\pi\pfrac{|E|}{\lambda}^{\frac{1}{2N}}\frac{\Gamma\blr{\frac{1}{2}\plr{2+\frac{1}{N}}}}
			{\Gamma\blr{\frac{1}{2}\plr{3+\frac{1}{N}}}}}.
		\]
		Since $\gamma \propto |E|$, the transmission probability increases as you approach the top of the barrier. 
		\\
		\\
		\\ \\ \\ \\ \\ \\ \\ \\ \\ \\ \\ \\
		
	\eenum
	
	
	% #3 ------------------------------------------------------------------------------------------------------------------------------------------------------------------------------
	\item
	\benum
	
		% (a)
		\item
		Using the data for Thorium in the attached table, make a plot of $\ln T_{1/2}$ versus $1/\sqrt E$
		\\
		\\
		\figg[width=150mm]{hw2_1.pdf}
		
		% (b)
		\item
		Your answer in (a) should show approximately linear behavior, over a huge range of energies. Use WKB to find a rough estimate for the 
		slope coefficient. [Note that thorium has $Z=90$, and the mass of an $\alpha$ particle is approximately 3724 Mev$/c^2$]. How good
		is the WKB estimate?
		\\
		\\
		We may use the tunneling probability to find half life of Thorium. In utilizing the WKB in three dimensions, we can use an effective 
		potential that includes the angular momentum barrier such that
		\[
			\gamma = \frac{2}{\h}\int_{r_0}^{r_1} dr\ \sqrt{2m(V_{eff}(r)-E)}
		\]
		where
		\[
			V_{eff}(r) = \frac{\h^2}{2m}\frac{(l+\frac{1}{2})^2}{r^2}+\frac{2(Z-2)e^2}{r}.
		\]
		Here we have used the Langer substitution along with a repulsive $(Z-2)$ Coulomb potential representing the nuclear force after
		emission of an alpha particle. Gamma then becomes
		\[
			\gamma = 2\int_{r_0}^{r_1} dr\ \frac{1}{r}\blr{\plr{l+\frac{1}{2}}^2+\frac{2m}{\h^2}\blr{2(Z-2)e^2r-Er^2}}^{1/2}.
		\]
		The first turning point may be approximated by
		\[
			r_0 \approxeq (1.0\  \text{fm})A^{1/3}
		\]
		with $A$ being the number of neutrons and protons. Then, $r_1$ may be found by solving for the positive root in the quadratic 
		equation for
		\[
			E = V_{eff}(r)
		\]
		If we now set 
		\[
			m = \text{3724 MeV}/c^2
		\]
		\[
			l=0
		\]
		\[
			Z = 90,
		\]
		we may get an estimation for $\gamma(E)$. Then, the lifetime can be found using
		\[
				\tau = \frac{1}{R}
		\]
		where
		\[
			R = \frac{\blr{2m(E+V_0)}^{1/2}}{2mr_0}\exp(-\gamma).
		\]
		Then we should be able to find the slope using
		\[
			\diff[]{\pfrac{1}{\sqrt E}}\plr{\ln[T_{1/2}(\tau(1/\sqrt E))]}.
		\]		
	\eenum
\eenum
\end{document}