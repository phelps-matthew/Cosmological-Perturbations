\documentclass[10pt,letterpaper]{article}
\usepackage{mymacros}

\title{\begin{spacing}{1.2}Quantum Mechanics II\\HW 7\end{spacing}}
\author{Matthew Phelps}
\date{Due: Nov. 4}

\begin{document}
\maketitle

\benum
% #1 -----------------------------------------------------------------------------------------------------------------------------------------------------------------
  	\item{Scattering from a radial square well potential}
	\\
	\\
	Consider the finite radial square well potential:
	\[
		V(r) = -V_0,\quad r<R;\quad V(r)=0,\quad r>R
	\]
	\benum
	% (a)
	\item
	Compute $\tan\delta_0$.
	\\
	\\
	To find the phase shift, we can solve the radial Schrodinger equation, take the solution $R_l(r)$ at $r\to \infty$ and 
	then identify the phase shift. We know that at $r\to\infty$ the form of our solution must be
	\be\label{1}
		R_l(r)\underset{r\to\infty}\longrightarrow \frac{A\sin\blr{kr-l\pi/2+\delta_l(k)}}{r}.
	\ee
	Here $\delta_l(k)$ is the potential dependent quantity that we seek to find, as it allows us to find the scattering
	amplitude. 
	\\
	\\
	Onwards to the problem, for $r<R$, the particle resides in potential $-V_0$
	and the solution to radial equation is that of a free particle with a shift in energy, namely:
	\[
		R_l(r) = Aj_l(\alpha r)+Bn_l(\alpha r)
	\]
	with
	\[
		\alpha = \sqrt\frac{2m(E+V_0)}{\h^2}.
	\]
	However, the Neumann function is singular at the origin and so we set $B_l =0$. Thus
	\[
		R_l(r) = Aj_l(\alpha r)\qquad r<R. 
	\]
	Outside the well, we have a free particle and can include the Neumann function 
	\[
		R_l(r) = Cj_l(kr)+Dn_l(kr)\qquad r>R
	\]
	where
	\[
		k = \sqrt\frac{2mE}{\h^2}. 
	\]
	Taking boundary conditions of continuity of the wavefunction and its derivative at $r=R$
	\[
		Aj_l(\alpha R) = Cj_l(kR)+Dn_l(kR)
	\]
	\[
		A\alpha j'_l(\alpha R) = Ckj'_l(kR)+Dkn'_l(kR).
	\]
	To find the form of $\delta_l$, we now take our solution at $r\to\infty$. The asymptotic forms of the sphereical
	Bessels are
	\[
		j_l(r) \to \frac{\sin(r-l\pi/2)}{r};\qquad n_l(r)\to -\frac{\cos(r-l\pi/2)}{r}.
	\]
	Therefore
	\ba
		R_l(r)&\underset{r\to\infty}\longrightarrow \frac{1}{kr}\blr{C \sin(r-l\pi/2)-D\cos(r-l\pi/2)}\\
		& = \frac{C}{kr}\blr{ \sin(r-l\pi/2)-\frac{D}{C}\cos(r-l\pi/2)}
	\ea
	We would like to put our asymptotic radial function in a form that resembles \eqref 1. One way we can do this is to 		assign $-\frac{C}{D} = \tan(\delta_l)$. The rationale is that the effect of the potential, represented by $\delta_l$,
	manifests itself as the boundary conditions and thus in $\frac{C}{D}$. Proceeding
	\ba
		R_l(r)\underset{r\to\infty}\longrightarrow &= \frac{C}{kr}\blr{ \sin(r-l\pi/2)+\tan\delta_l\cos(r-l\pi/2)}\\
		& = \frac{\sqrt{C^2+D^2}}{kr}\blr{\cos\delta_l\sin(r-l\pi/2)+\sin\delta_l \cos(r-l\pi/2)}\\
		& = \frac{\sqrt{C^2+D^2}}{kr}\blr{\sin\plr{kr-\frac{l\pi}{2}+\delta_l}}.
	\ea
	Two identities were needed in the above: $\frac{1}{\cos x}= \sqrt{\tan^2x+1}$ and $\sin x\cos y = \frac{1}{2}
	\plr{\sin(x+y)+\sin(x-y)}$. 
	
	Now we have the same form as \eqref 1 and all that remains is to find $\frac{D}{C}$.
	As we will see, it is the relation between $C$ and $D$ that we are after.  This is accomplished by the following
	series of painful-to-look-at algebraic steps:
	\ba
		& \frac{j_l(\alpha R)}{\alpha j'_l(\alpha R)} = \frac{Cj_l(kR)+Dn_l(kR)}{Ckj'_l(kR)+Dkn'_l(kR)} \\
		\to\quad& \plr{Ckj'_l(kR)+Dkn'_l(kR)}\pfrac{j_l(\alpha R)}{\alpha j'_l(\alpha R)} = Cj_l(kR)+Dn_l(kR) \\
		\to\quad&C\blr{ kj'_l(kR)\pfrac{j_l(\alpha R)}{\alpha j'_l(\alpha R)} -j_l(kR)}
		= D\blr{ n_l(kR)-kn'_l(kR)\pfrac{j_l(\alpha R)}{\alpha j'_l(\alpha R)}}\\
		\to \quad& \frac{D}{C} = \frac{kj'_l(kR)\pfrac{j_l(\alpha R)}{\alpha j'_l(\alpha R)} -j_l(kR)}
		{ n_l(kR)-kn'_l(kR)\pfrac{j_l(\alpha R)}{\alpha j'_l(kR)}} \\
		&= \frac{kj'_l(kR)j_l(\alpha R) -\alpha j_l(kR) j'_l(\alpha R)}
		{ \alpha n_l(kR) j'_l(\alpha R)-kn'_l(kR)j_l(\alpha R)}
	\ea
	\\
	\\
	Thus
	\be\label{2}
		\tan\delta_l = -\frac{D}{C} = \frac{kj_l(\alpha R)j'_l(kR) -\alpha j_l(kR) j'_l(\alpha R)}
		{ kj_l(\alpha R)n'_l(kR)-\alpha n_l(kR) j'_l(\alpha R)}.
	\ee
	To compute the S-wave we note
	\[
		j_0(x) = \frac{\sin x}{x};\qquad j_0'(x) = \frac{\cos x}{x}-\frac{\sin x}{x^2}
	\]
	\[
		n_0(x) = -\frac{\cos x}{x};\qquad n_0'(x) = \frac{\sin x}{x}+\frac{\cos x}{x^2}
	\]
	and after some algebraic manipulation we arrive at
	\[
		\tan\delta_0 = \frac{k\sin(\alpha R)\cot(kR)-\alpha \cos(\alpha R)}{k\sin(\alpha R)+\alpha 
		\cos(\alpha R)\cot(kR)}
	\]
	\\
	% (b)
	\item
	The scattering length $a_s$, and effective range, $r_0$, are \textbf{defined} from the low energy limit of the
	s-wave phase shift:
	\[
		k\cot\delta_0 \equiv -\frac{1}{a}+\frac{1}{2}r_0k^2+...,\quad k\to 0
	\]
	Find an expression for $a_s$ and for $r_0$, in terms of the momentum $k$ and the parameters of the 
	potential.
	\\
	\\
	\\
	From part (a) we have
	\[
		k\cot\delta_0 =  \frac{k^2\sin(\alpha R)+\alpha k
		\cos(\alpha R)\cot(kR)}{k\sin(\alpha R)\cot(kR)-\alpha \cos(\alpha R)}
	\]
	Using the fact that
	\[
		\alpha = \sqrt{k^2+\frac{2mV_0}{\h^2}}
	\]
	we may expand this function (using Mathematica) around $k=0$ up to second order. Denoting dimensionless 
	constant
	\[
		\beta= \sqrt{\frac{2mV_0R^2}{\h^2}}
	\]
	we have
	\[
		k\cot\delta_0 \approx -\frac{1}{R\plr{\frac{\tan\beta}{\beta}-1}}+\frac{1}{2}
		\blr{\frac{R\plr{\beta^2+\plr{\frac{3}{2}-3\beta^2}\sin(2\beta)+\beta(\beta^2-3)\cos(2\beta)}}
		{3\beta(\sin\beta-\beta\cos\beta)^2}}k^2.
	\]
	Hence
	\[
		a_s = R\plr{\frac{\tan\beta}{\beta}-1}
	\]
	\\
	\[
		r_0 = \frac{R\plr{\beta^2+\plr{\frac{3}{2}-3\beta^2}\sin(2\beta)+\beta(\beta^2-3)\cos(2\beta)}}
		{3\beta(\sin\beta-\beta\cos\beta)^2}.
	\]
	\\
	\\	
	% (c)
	\item
	Plot $a_s$ and $r_0$ as functions of the dimensionless ratio $\sqrt{2mV_0R^2/\h^2}$.
	\\
	\\
	\figg[width=100mm]{7_1.pdf}
	\phantom{}
	\figg[width=100mm]{7_2.pdf}
	% (d)
	\item 
	What is the physical meaning of the singularities in these plots?
	\\
	\\
	The singularities all lie at positions where the scattering length $a_s$ undergoes a sign change. The change in sign 		corresponds to the existence of a bound state in the potential well. These periodic singularities occur at values
	in which $\sin(\delta_0)$ is maximized in the scattering amplitude, i.e. resonant scattering
	\[
		\delta_0 = \plr{n+\frac{1}{2}}\frac{\pi}{2}.
	\]
	\phantom{}
	\\
	\\
	\\
	\phantom{}
\eenum
	
% 2 ----------------------------------------------------------------------------------------------------------------------------------------------------
	\item{Born and Eikonal Approximations}
	\\ \\
	Consider the radially symmetric potential 
	\[
		V(r) = -V_0e^{-r^2/r_0^2}
	\]
	\benum
	% (a)
	\item 
	Compute both the Born approximation and the Eikonal approximation for the scattering amplitude
	$f_k(\theta)$ for $V(r)$.
	\\
	\\
	\\
	For a spherically symmetric potential, the Eikonal approximation (good for large $k$) is given as
	\[
		f(\theta) = -ik\int_0^\infty db'\ b' J_0(kb'\theta)\blr{e^{i\vect\chi(b')}-1}
	\]
	where
	\[
		\chi(\vect b') = -\frac{m}{\h^2 k} \int_{-\infty}^{\infty} dz'\ V(\vect b,z').
	\]
	In terms of the impact parameter $b$ and $z$ we have $r^2 = b^2+z'^2$ and the phase integral goes as
	\ba
		\chi(b') &= \frac{mV_0}{\h^2 k}e^{-b^2/r_0^2} \int_{-\infty}^{\infty} dz'\ e^{-z'^2/r_0^2} \\
		& = \frac{mV_0}{\h^2 k}\sqrt\pi r_0 e^{-b^2/r_0^2} .
	\ea
	Looking at this double exponential
	\[
		\exp\blr{i\frac{mV_0}{\h^2 k}\sqrt\pi r_0\exp\plr{ e^{-b^2/r_0^2}}}
	\] 
	lets denote the dimensionless constants
	\[
		\gamma = \frac{mV_0}{\h^2}r_0^2
	\]
	\[
		y = \frac{b}{r_0} 
	\]
	\[
		\alpha = kr_0
	\]
	thus
	\[
		\exp\blr{i\sqrt\pi\gamma e^{-y^2}}.
	\]
	Our integral which we must integrate numerically is now
	\[
		f(\theta) = -ikr_0^2 \int_0^\infty dy\ y J_0(\alpha y \theta)\blr{\exp\blr{i\sqrt\pi\pfrac{\gamma}{\alpha} e^{-y^2}}-1}.
	\]
	or
	\be\label{3}
		\frac{|f(\theta)|}{r_0^2} = \alpha^2\left| 
		\int_0^\infty dy\ y J_0(\alpha y \theta)\blr{\exp\blr{i\sqrt\pi\pfrac{\gamma}{\alpha} e^{-y^2}}-1}\right|^2.
	\ee
	\\
	\\

	Born Approximation:
		\ba
			f(\theta) &= \frac{2mV_0}{\h^2} \int_0^\infty dr\  r\frac{\sin(qr)}{q}\exp{\pfrac{-r^2}{r_0^2}}\\
			& = \frac{2mV_0}{\h^2}\pfrac{\sqrt\pi r_0^3}{4}\exp\plr{-\frac{r_0^2q^2}{4}}\\
			& = \frac{2mV_0}{\h^2}\pfrac{\sqrt\pi r_0^3}{4}\exp\blr{-r_0^2k^2\sin(\theta/2)^2}
		\ea
		\[
			|f(\theta)|^2 = \frac{\pi r_0^2}{4}\pfrac{mV_0r_0^2}{\h^2}^2\exp\blr{-2r_0^2k^2\sin(\theta/2)^2}.
		\]
	In terms of dimensionless quantities
	\be\label{4}
		\frac{|f(\theta)|^2}{r_0^2} = \gamma^2\frac{\pi}{4}\exp\blr{-2\alpha^2\sin(\theta/2)^2}.
	\ee
	\\
	% (b)
	\item
	Compare plots of the Born and Eikonal expression $|f_k(\theta)|^2$ as functions of the angle $\theta$,
	at both low and high energies (choose some suitable values).
	\\
	\\
	We plot the dimensionless functions of \eqref 3 and \eqref 4 for various values of $\alpha = kr_0$ 
	and $\gamma = \frac{mV_0}{\h^2}r_0^2$. 
	\\
	\\
	Low energy:
	\figg[width=100mm]{7_41.pdf}
	\figg[width=100mm]{7_42.pdf}
	\figg[width=100mm]{7_43.pdf}
	\phantom{}
	Mid Energy:
	\figg[width=100mm]{7_44.pdf}
	\phantom{}
	High Energy:
	\figg[width=100mm]{7_45.pdf}
	\figg[width=100mm]{7_47.pdf}
	\figg[width=100mm]{7_46.pdf}
	\phantom{}
	We see that the Eikonal makes an excellent approximation to the Born approximation at high energies.
	\\
	\\ \\
	\eenum

% 3 ------------------------------------------------------------------------------------------------------------------------------------------------------
	\item{Finite Radial Well}
	Consider the finite radial square well potential:
	\[
		V(r) = -V_0,\quad r<a;\qquad V(r)=0,\quad r>a
	\]
	\benum
	% (a)
	\item
	Show that the exact phase shifts are given by
	\[
		\tan\delta_l = \frac{kj_l(\alpha a)j'_l(ka)-\alpha j_l(ka)j'_l(\alpha a)}{kj_l(\alpha a)n'_l(ka)
		-\alpha n_l(ka)j'_l(\alpha a)}
	\]
	where 
	\[
		k = \sqrt{2mE/ \h^2}; \qquad \alpha = \sqrt{2m(E+V_0)/\h^2}.
	\]
	\\
	\\
	This is given in \eqref 2 for $R=a$.
	\\
	\\
	% (b)
	\item 
	Compute the Born approximation cross-section as a numerical integral, and plot as a function of $k$. 
	\\
	\\
	Born approximation:
	\[
		|f(\theta)|^2 = \left|-\frac{2m}{\h^2}\int_0^\infty dr\ r\frac{\sin(qr)}{q}V(r)\right|^2
	\]
	with
	\[
		q = 2k\sin(\theta/2).
	\]
	For our potential
	\ba
		f(\theta) &= \frac{2mV_0}{\h^2} \int_0^a dr\  r\frac{\sin(qr)}{q}\\
		& = \frac{2mV_0}{\h^2}\plr{\frac{\sin(aq)-aq\cos(aq)}{q^3}}
	\ea
	thus
	\[
			\diff[\sigma]{\Omega} = 4a^2\pfrac{mV_0a^2}{\h^2}^2\plr{\frac{\sin(aq)-aq\cos(aq)}{(qa)^3}}^2.
	\]
	and the total cross section is
	\[
		\sigma = 8\pi a^2\pfrac{mV_0a^2}{\h^2}^2\int_0^\pi  d\theta\, \sin\theta 
		\plr{\frac{\sin(aq)-aq\cos(aq)}{(qa)^3}}^2.
	\]
	In our numerical integration, we will find the dimensionless quantity $F(ka) = F(\alpha)$
	\[
		\frac{\sigma}{8\pi a^2}\pfrac{mV_0a^2}{\h^2}^{-2} = F(\alpha) = 
		\int_0^\pi  d\theta\, \sin\theta \plr{\frac{\sin(2\alpha \sin(\theta/2))-\alpha\cos(2\alpha \sin(\theta/2))}{(2\alpha
		\sin(\theta/2))^3}}^2
	\]
	\figg[width=100mm]{7_22.pdf}
	\phantom{}
	% (c)
	\item
	Compare your Born approximation expression with a sum over the first few waves. How many partial waves do you 
	need to get good agreement, and in what region of $k$?
	\\
	\\
	The scattering amplitude is given in terms of partial waves as
	\[
		f(\theta) = \frac{1}{k}\sum_{l=0}^{\infty} (2l+1)e^{i\delta_l}\sin\delta_lP_l(\cos\theta)
	\]
	Perhaps what we could do is decompose the Born approximation for $f(\theta)$ in terms of Legendre polynomials
	and find for what values of $l$ give a good approximation. The Born approximation can
	be written as
	\[
		f(\theta,\alpha) =  \frac{2mV_0}{\h^2}\plr{\frac{\sin(aq)-aq\cos(aq)}{q^3}} =   \frac{2mV_0}{\h^2}g(\theta,\alpha)
	\]
	with 
	\[
		\alpha = ka.
	\]
	The decomposition is then
	\[
		f(\theta,\alpha) = \frac{2mV_0}{\h^2}\sum_{l=0}^{\infty} A_l(\alpha)P_l(\cos\theta)
	\]
	where
	\[
		A_l(\alpha)= \frac{2l+1}{2}\int_0^\pi d\theta\   \sin\theta\, g(\theta,\alpha) P_l(\cos\theta)
	\]
	If we look at a plot of $A_l(\alpha)$ for a sequence of $l$'s and different values of $\alpha$, we can come up
	with an estimate of how many partial waves are needed:
	\figg[width=100mm]{7_31.pdf}
	\figg[width=100mm]{7_32.pdf}
	\figg[width=100mm]{7_33.pdf}
	\figg[width=100mm]{7_34.pdf}
	\figg[width=100mm]{7_35.pdf}
	
	Assuming that the Born approximation is good, which is true when
	\[
		\frac{2m}{\h^2 k}\left|\int dr'\ e^{ikr'}\sin(kr')V(r')\right| \ll 1,
	\]
	we can see that in order for the partial waves to match the same $\theta$ behavior as the Born, we need an increasing
	number of $l$'s as the energy gets higher. Hence, the S-wave dominates at low energy and at higher energy (forward 
	scattering), we need higher orders of Legendre polynomials. Based on the graphs it appears that an approximation
	for a good value of $l_{max}$ is 
	\[
		l_{max} \approx  ka.
	\]
	Now that I'm looking at it, I see that this is the same relation given in Shankar. For a potential with range $r_0$, 
	particles at cylindrical radius $\rho > r_0$ are not affected. Given that the angular momentum is
	\[
		l\approx k\rho
	\]
	the condition for $l$ is then
	\[
		l_{max} = k\rho_{max} = kr_0.
	\]
	\\
	% (d)
	\item
	How does your answer depend on the depth of the well and width of the well?
	\\
	\\
	According the expansion of the Born approximation in terms of Legendre polynomials, the coefficients are
	scaled by $V_0$. Hence, for 
	\[
		\frac{2mV_0}{\h^2} \gg 1
	\]
	higher orders of Legendre polynomials are needed to provide a good approximation. Since we have been using 
	$\alpha = ka$, the same argument applies as in part (c); namely that as the width is increased, larger values
	of $l$ are needed for a proper estimate. 
	\eenum
\eenum
\end{document}