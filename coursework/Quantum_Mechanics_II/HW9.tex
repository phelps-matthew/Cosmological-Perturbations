\documentclass[10pt,letterpaper]{article}
\usepackage{macroshw}

\title{\begin{spacing}{1.2}Quantum Mechanics II\\HW 9\end{spacing}}
\author{Matthew Phelps}
\date{Due: Nov. 18}

\begin{document}
\maketitle

\benum
% #1 -----------------------------------------------------------------------------------------------------------------------------------------------------------------
  	 \item{Review of Lorentz transformation in relativistic notation}
	\\ \\
	A Lorentz boost from the unprimed inertial frame to the primed inertial frame, moving with velocity $\vect v$
	relative to the original frame, can be written as
	\[
		x_0' = \gamma\plr{x_0-\vec \beta\cdot \vec x},\qquad
		\vec x' = \vec x+\frac{(\gamma-1)}{\beta^2}(\vec\beta\cdot\vec x)\vec\beta-\gamma\vec\beta x_0
	\]	
	where $\vect \beta = \frac{\vec v}{c}$ and $\gamma = \frac{1}{\sqrt{1-\beta^2}}$.
	
	\benum
	% (a)
	\item
	Express this Lorentz transformation as a matrix equation
	\[
		x' = \Lambda x
	\]
	where $x$ is the 4-component column vector (and similarly for $x'$):
	\[
		\bpm x_0\\x_1\\x_2\\x_3 \epm
	\]
	Hence confirm simply that $x_\mu x^\mu$ is invariant under the Lorentz transformation.
	\\ \\
	First we compose the Lorentz transformation matrix - this can be done more or less by inspection. 
	Denoting $\vec\beta = (\beta_1,\beta_2,\beta_3)$
	\[
		\bpm x'_0\\ x'_1\\ x'_2 \\ x'_3 \epm = 
		\bpm
		\gamma & -\gamma\beta_1 & -\gamma\beta_2 &-\gamma\beta_3\\
		-\gamma\beta_1 & 1+\frac{(\gamma-1)}{\beta^2}\beta_1^2 & \frac{(\gamma-1)}{\beta^2}\beta_1\beta_2
		& \frac{(\gamma-1)}{\beta^2}\beta_1\beta_3 \\
		-\gamma\beta_2 & \frac{(\gamma-1)}{\beta^2}\beta_1\beta_2 &1+ \frac{(\gamma-1)}{\beta^2}\beta_2^2
		& \frac{(\gamma-1)}{\beta^2}\beta_2\beta_3 \\
		-\gamma\beta_3 & \frac{(\gamma-1)}{\beta^2}\beta_1\beta_3 & \frac{(\gamma-1)}{\beta^2}\beta_2\beta_3
		& 1+\frac{(\gamma-1)}{\beta^2}\beta_3^2
		\epm
		\bpm x_0 \\ x_1  \\ x_2\\ x_3 \epm
	\]
	In tensor notation, this corresponds to
	\[
		x'^\mu = \Lambda_\nu^\mu x^\nu. 
	\]
	To show that the inner product is invariant under the Lorentz boost we need to show that
	\[
		x'_\mu x'^\mu = x_\mu x^\mu. 
	\]
	This can be formulated as
	\ba
		g_{\mu\nu} x'^\nu x'^\mu & = g_{\mu\nu}x^\nu x^\mu \\
		g_{\mu\nu} \Lambda_\alpha ^\nu x^\alpha \Lambda_\beta^\mu x^\beta & = g_{\mu\nu}x^\nu x^\mu\\
		\plr{g_{\mu\nu}\Lambda_\alpha^\nu \Lambda_\beta^\mu} x^\alpha x^\beta &= g_{\mu\nu}x^\nu x^\mu.
	\ea
	This implies
	\ba
		g_{\mu\nu}\Lambda_\alpha^\nu \Lambda_\beta^\mu & = g_{\alpha\beta}\\
		\Lambda_\alpha^\nu \Lambda_{\nu\beta} &= g_{\alpha\beta} \\
		\plr{\Lambda_\alpha^\nu \Lambda_{\nu\beta}} g^{\beta\mu} &= g_{\alpha\beta}g^{\beta\mu}\\
		\Lambda_\alpha^\nu \Lambda_{\nu}^\mu &= \delta^\mu_\alpha
	\ea 
	Explicitly, 
	\[
		\sum_{\nu=0}^3 \Lambda_\alpha^\nu \Lambda_{\nu}^\mu = \delta^\mu_\alpha = 
		\Lambda_\alpha^0\Lambda_0^\mu + \Lambda_\alpha^1\Lambda_1^\mu +
		\Lambda_\alpha^2\Lambda_2^\mu + \Lambda_\alpha^3\Lambda_3^\mu.
	\]
	Lets test for $\alpha = \mu$. Since our Lorentz transformation matrix is symmetric,
	we are essentially summing the squares of each element in a column to see if they each
	equal unity
	\[
		\plr{\Lambda^0_\alpha}^2 + \plr{\Lambda^1_\alpha}^2 + \plr{\Lambda^2_\alpha}^2 +
		\plr{\Lambda^3_\alpha}^2  = 1.
	\]
	For the time column ($\alpha = 0$)
	\[
		\Lambda^\nu_0\Lambda^0_\nu = \gamma^2 +\gamma^2(1+\beta^2) = 1
	\]
	For the other columns we also have
	\[
		\Lambda^\nu_i\Lambda^i_\nu = 1.
	\]
	We can continue to show that the product between any two columns in the transformation matrix must 
	be zero unless they are the same column - this is due to the fact that a Lorentz boost 
	is unitary (and thus orthogonal). Suffice to say, Lorentz boosts preserve the inner product and space-time 
	interval, which is what we expect between two inertial frames of reference. 
	\\
	\\
	% (b)
	\item
	Construct the Lorentz transformation matrix for two successive boosts, of velocities $\vec v$ and $\vec u$,
	each aligned along the $x^1$ axis. Hence derive the relativistic velocity addition formula:
	\[
		w = \frac{u+v}{1+\frac{uv}{c^2}}
	\]
	\\
	For a boost along the $x^1$ direction, we have $\vec\beta = (\beta,0,0)$ and our transformation
	matrix goes as 
	\[
		\Lambda^\mu_\nu = \bpm \gamma & -\gamma \beta & 0 & 0
		\\ -\gamma\beta &\gamma & 0 & 0 \\
		0& 0& 1& 0 \\ 0 & 0& 0&1\epm
	\]
	As expected, only the $x^0$ and $x^1$ coordinates mix. We focus only on the transformation 
	between these two components from hereon. With two successive boosts we have
	\[
		\Lambda_\beta^\alpha \Lambda_\nu^\beta x^\nu = x''^\mu = 
		\bpm \gamma_v &-\gamma_v\beta_v \\ -\gamma_v\beta_v & \gamma_v \epm
		\bpm \gamma_u &-\gamma_u\beta_u \\ -\gamma_u\beta_u & \gamma_u \epm
		\bpm x^0 \\ x^1 \epm
	\]
	From here we make the substitutions
	\[
		\sinh \theta_v = \gamma_v\beta_v, \qquad \cosh\theta_v = \gamma_v
	\]
	and similarly for velocity $u$. This becomes
	\ba
		x''^\mu &= 
		\bpm \cosh\theta_v & -\sinh\theta_v  \\ -\sinh\theta_v & \cosh\theta_v \epm
		\bpm \cosh\theta_u & -\sinh\theta_u  \\ -\sinh\theta_u & \cosh\theta_u \epm
		\bpm x^0 \\ x^1 \epm \\
		&= \bpm \cosh(\theta_v+\theta_u) & -\sinh(\theta_v+\theta_u) \\
		-\sinh(\theta_v+\theta_u) & \cos(\theta_v+\theta_u)\epm \bpm x^0 \\ x^1 \epm
	\ea
	We see that this is equivalent to a single boost of ``hyperbolic" velocity
	\[
		\theta_w = \theta_v+\theta_u.
	\]
	We can convert this to the actual relation between velocities by inverting our hyperbolic 
	substitutions
	\ba
		\tanh\theta_w &= \frac{\tanh\theta_v+\tanh\theta_u}{1+\tanh\theta_v\tanh\theta_u}\\
		& = \frac{\beta_v+\beta_u}{1+\beta_v\beta_u} = \beta_w
	\ea
	hence
	\[
		w = \frac{v+u}{1+\frac{vu}{c^2}}.
	\]
	\\
	\\
	% (c)
	\item
	Recalling that the field strength $F^{\mu\nu}$ is a tensor, compute the effect on the electric and
	magnetic fields of a Lorentz transformation corresponding to a boost of velocity $\vec v$ along 
	the $x^1$ axis. 
	\\
	\\
	Under a Lorentz boost, the tensor transforms according to
	\[
		F'^{\mu\nu} = \Lambda_\alpha^\mu F^{\alpha\beta}\Lambda_\beta^\nu
	\]
	Using the boost found earlier, this becomes, in matrix form
	\ba
		F'^{\mu\nu} &= \bpm 
		\gamma & -\gamma \beta & 0 & 0 \\
		-\gamma\beta &\gamma & 0 & 0 \\
		0 & 0& 1& 0\\
		0 & 0& 0& 1 \epm
		\bpm 
		0& E_1&E_2&E_3 \\
		-E_1& 0 & B_3 & -B_2 \\
		-E_2 & -B_3 & 0 & B_1 \\
		-E_3 & B_2 &-B_1& 0
		\epm 
		\bpm 
		\gamma & -\gamma \beta & 0 & 0 \\
		-\gamma\beta &\gamma & 0 & 0 \\
		0 & 0& 1& 0\\
		0 & 0& 0& 1 \epm \\
		& = \bpm 
		0 & E_1 &\gamma(E_2-\beta B_3) & \gamma(E_3+\beta B_2) \\
		-E_1 &0 &\gamma(B_3-\beta E_2) & -\gamma (\beta E_3+B_2) \\
		\gamma(\beta B_3-E_2)&\gamma(\beta E_2-B_3) & 0 & B_1 \\
		-\gamma(E_3 +\beta B_2) & \gamma(\beta E_3+B_2) & -B_1 & 0
		\epm
	\ea
	We find that the EM fields in the $x^1$ direction are unaffected - but components in the plane
		orthogonal to the $x^1$-axis are mixed.
		\\
			
	\eenum
% 2 ----------------------------------------------------------------------------------------------------------------------------------------------------
	\item{Dirac matrices, Spin Matrices and Relativistic Total Angular Momentum }
	\\ 
	\benum
	% (a)
	\item 
	Use the anti-commutation properties of the Pauli matrices to verify that the Dirac matrices really
	do satisfy the anti-commutation relations:
	\[
		\{ \gamma_\mu,\gamma_\nu \} = 2g_{\mu\nu}
	\]
	\\
	\\
	Let's separate the space and time indices. With $\mu = \nu = 0$ we have
	\[
		\{ \gamma_0,\gamma_0 \} = 2\gamma_0^2 = 2\mathds 1.
	\]
	For $\nu = 0$ we note that
	\[
		\gamma_i \gamma_0 = \bpm 0 & \sigma_i \\ -\sigma_i & 0 \epm
		 \bpm \mathds 1 & 0 \\ 0 & -\mathds 1\epm  = \bpm 0&-\sigma_i\\ -\sigma_i &0\epm
		 = -\gamma_0\gamma_i
	\]
	Therefore
	\[
		\{ \gamma_i,\gamma_0 \} = 2\delta_{i 0} \mathds 1.
	\]
	Now for the space indices $i$, $j$,
	\[
		\gamma_i\gamma_j = 
		\bpm 0 & \sigma_i \\ -\sigma_i & 0 \epm
		\bpm 0 & \sigma_j \\ -\sigma_j & 0 \epm
		= \bpm -\sigma_i\sigma_j & 0\\ 0 & -\sigma_i\sigma_j \epm
	\]
	hence
	\[
		\{ \gamma_i,\gamma_j \} =- \bpm \{ \sigma_i,\sigma_j\} & 0 \\ 0& \{\sigma_i,\sigma_j\}  \epm.
	\]
	Using the anti-commutation property of the Pauli matrices,
	\[
		\{ \sigma_i,\sigma_j \} = 2\delta_{ij}\mathds 1
	\]
	we have
	\[
		\{ \gamma_i,\gamma_j \} =
		-2 \delta_{ij}\mathds 1.
	\]
	With our results,
	\[
		\{ \gamma_i,\gamma_0 \} = 2\delta_{i 0} \mathds 1;\qquad 
		\{ \gamma_i,\gamma_j \} = -2\delta_{i j} \mathds 1,
	\]
	we see that we get back our Minkowski metric with a factor of 2
	\[
		\{ \gamma_\mu,\gamma_\nu \} = 2g_{\mu\nu}\mathds 1.
	\]
	\\ \\
	% (b)
	\item
	The relativistic spin matrices are defined as $\sigma_{\mu\nu} \equiv \frac{i}{2}[ \gamma_\mu
	,\gamma_\nu]$. Treat separately the cases where the space-time index is temporal or spatial, and 
	compute $\sigma_{0i}$ and $\sigma_{ij}$. 
	\\
	\\
	Most of the previous results apply here. Using $\gamma_0\gamma_i = -\gamma_i\gamma_0$, 
	\[
		\sigma_{0i} = \frac{i}{2}[\gamma_0,\gamma_i] = i\gamma_0\gamma_i = \sigma_i
		\bpm 0 & \mathds 1 \\ \mathds 1 & 0\epm.
	\]
	For $\sigma_{ij}$, 
	\[
		\sigma_{ij} = \frac{i}{2}\blr{ - \bpm \sigma_i\sigma_j & 0\\ 0 & \sigma_i\sigma_j \epm
		+
		 \bpm \sigma_j\sigma_i & 0\\ 0 & \sigma_j\sigma_i \epm}
		 = -\frac{i}{2}\bpm [\sigma_i,\sigma_j] &0 \\ 0 & [\sigma_i,\sigma_j] \epm. 
	\]
	With the Pauli commutation relations
	\[
		[\sigma_i,\sigma_j] = 2i\epsilon_{ijk}\sigma_k
	\]
	this becomes
	\[
		\sigma_{ij} = \bpm \epsilon_{ijk}\sigma_k & 0 \\ 0 & \epsilon_{ijk} \sigma_k \epm.
	\]
	
	
	% (c)
	\item
	The total 	relativistic angular momentum is defined to be
	\[
		\vec J = \vec L +\frac{\h}{2}\vec \Sigma
	\]
	where
	\[
		\vec \Sigma = \bpm \vec\sigma&0\\0&\vec\sigma \epm
	\]
	Show using the Dirac equation for an electron in a radially symmetric scalar potential, 
	$\phi(r)$, that $\vec J$ is conserved.
	\\ 
	\\
	\\
	The Dirac equation for a particle in a spherically symmetric potential $\phi(r)$ can be expressed as
	\[
		i\h \pdiff[\psi]{t} = (c\vec \alpha \cdot \vec p +\vec \beta mc^2+\phi(r))\psi
	\]
	or as
	\[
		i\h \pdiff[\psi]{t} = (c\gamma^0\gamma^i p_i +\gamma^0 mc^2+\phi(r))\psi.
	\]
	In this form, the Schrodinger-like Hamiltonian can be seen to be
	\[
		H = c\gamma^0\gamma^i p_i +\gamma^0 mc^2+\phi(r).
	\]
	The Hamiltonian is a $4\times4$ matrix. To determine if $\vec J$ is conserved, it must commute 
	with the Hamiltonian. We note that the Dirac matrices belong to a group of tensor operators and therefore
	only act on the spinor components; similar reasoning applies the momentum and position operators. 
	\\
	\\
	First lets analyze the potential part of the Hamiltonian. We know, either from other problems like the hydrogen
	atom or by explicit 
	calculation, that the orbital angular momentum operator commutes with any function of $r$
	\[
		[L_i,r] \sim [L_i, (x^2+y^2+z^2)] = 0.
	\]
	In addition, the spin angular momentum components are the identity in position space, so they too commute:
	\[
		[\Sigma_i,\phi(r)] = 0.
	\]
	Therefore the total angular momentum commutes with the potential
	\[
		[J_i,\phi(r)] = 0.
	\]
	Next we look at $\gamma^0$ with $\vec J$. The relevant commutator is
	\[
		\frac{\h}{2}mc^2[\gamma^0,\vec\Sigma] = \frac{\h}{2}mc^2
		\blr{\bpm \mathds 1 &0 \\ 0& -\mathds 1 \epm 
		\bpm \vec\sigma &0 \\ 0 &\vec\sigma \epm-\bpm \vec\sigma &0 \\ 0 &\vec\sigma \epm
		\bpm \mathds 1 &0 \\ 0& -\mathds 1 \epm  } =0  .
	\]
	Lastly, we must look at the $c\gamma^0\gamma^i p_i $ (or $c\vec \alpha \cdot \vec p$) term with $\vec J$. Lets 
	commute with the 
	$z$ component of spin angular momentum
	\ba
		[c\vec \alpha \cdot \vec p , \frac{\h}{2}\Sigma_3] &= \frac{\h c}{2} [\alpha_ip^i, \Sigma_3] \\
		& =  \frac{\h c}{2}\plr{ p_1[\alpha_1,\Sigma_3]+p_2[\alpha_2,\Sigma_3]+p_3[\alpha_3,\Sigma_3]}.
	\ea
	Individually, the commutators are
	\ba
		[\alpha_1,\Sigma_3] &= \blr{\bpm 0 &\sigma_1 \\ \sigma_1 & 0 \epm
		\bpm \sigma_3 & 0 \\ 0 & \sigma_3 \epm-
		\bpm \sigma_3 & 0 \\ 0 & \sigma_3 \epm \bpm 0 &\sigma_1 \\ \sigma_1 & 0 \epm} \\
		& = \bpm 0&[\sigma_1,\sigma_3] \\ [\sigma_1,\sigma_3] &0 \epm \\
		& = -2i\alpha_2.
	\ea
	Hence
	\[
		[\alpha_2,\Sigma_3] = 2i\alpha_1
	\]
	and
	\[
		[\alpha_3,\Sigma_3] = 0.
	\]
	In general we then have
	\[
		[\alpha_i,\Sigma_j] = \epsilon_{ijk} 2i \alpha_k.
	\]
	Now we can evaluate
	\ba
		[c\vec \alpha \cdot \vec p , \frac{\h}{2}\Sigma_3] = &
		 i\h c\plr{ p_1[\alpha_1,\Sigma_3]+p_2[\alpha_2,\Sigma_3]+p_3[\alpha_3,\Sigma_3]}\\
		 &= i\h c \plr{\alpha_1p_2-\alpha_2p_1} \\
		 & = i\h c(\vec \alpha\times \vec p)_3
	\ea
	Therefore the more general result is
	\be\label{1}
		[c\vec \alpha \cdot \vec p , \frac{\h}{2}\Sigma_i] =  i\h c(\vec \alpha\times \vec p)_i.
	\ee
	\\
	
	Lastly we evaluate 
	\[
		[c\vec \alpha \cdot \vec p , L_i] = c[\alpha_i p^i,L_j].
	\]
	Looking at the case for $L_3$ 
	\ba
		[\alpha_1p_1+\alpha_2p_2+\alpha_3p_3,L_3] &= 
		[\alpha_1p_1+\alpha_2p_2+\alpha_3p_3,(x_3p_2-x_2p_3)] \\
		& = \alpha_1p_2[p_1,x_1]-\alpha_2p_1[p_2,x_2] \\
		& = -i\h (\alpha_1p_2-\alpha_2p_1)\\
		& =-i\h(\vec\alpha\times\vec p)_3
	\ea
	So in general we have
	\be\label{2}
		[c\vec \alpha \cdot \vec p , L_i] = -i\h c(\vec\alpha\times\vec p)_i.
	\ee
	Since \eqref 1 and \eqref 2 are the only non-zero contributions to the commutator with the Hamiltonian,
	and because they are exactly equal and opposite, we conclude that the total angular momentum
	commutes with the Hamiltonian
	\[
		[H,\vect J] =0
	\]
	and thus is a conserved quantity. 
	\\ \\
	
	\eenum
	
% 3 ------------------------------------------------------------------------------------------------------------------------------------------------------
	\item{4-momentum in an electromagnetic field}
	\\ \\
	Defining the kinetic 4-momentum
	\[
		\pi_\mu \equiv c\plr{ p_\mu +\frac{e}{c}A_\mu}
	\]
	use the quantum mechanical relation $p_\mu \to i\h \pdiff{x^\mu}$ to commute the operator
	\[
		\plr{\gamma^\mu \pi_\mu}^2
	\]
	Express your answer in terms of the Klein-Gordan part, $\pi_\mu\pi^\mu$, and parts linear
	in the electric and magnetic field.
	\\
	\\
	\\
	As the kinetic momentum operator is separate from the gamma spin matrices, we of course
	have
	\[
		[\gamma^\mu,\pi_\nu] = 0
	\]
	and so we may write the product as 
	\ba
		\plr{\gamma^\mu \pi_\mu}^2 &= \gamma^\mu\gamma^\nu\pi_\mu\pi_\nu\\
		& = \plr{\frac{1}{2}\{\gamma^\mu,\gamma^\nu\}+\frac{1}{2}[\gamma^\mu,\gamma^\nu]}\pi_\mu\pi_\nu \\
		& = \plr{\frac{1}{2}(2g^{\mu\nu}\mathds 1) +\frac{1}{2}[\gamma^\mu,\gamma^\nu]}\pi_\mu\pi_\nu \\
		& = \frac{1}{2}[\gamma^\mu,\gamma^\nu]\plr{\pi_\mu\pi_\nu} + \pi^\nu\pi_\nu\\
		& =  \frac{1}{4}[\gamma^\mu,\gamma^\nu]\plr{\{ \pi_\mu,\pi_\nu\} +
		[\pi_\mu,\pi_\nu]}+ \pi^\nu\pi_\nu\tag{3}
	\ea
	Lets first take the anti-commutator
	\[
		 \frac{1}{4}[\gamma^\mu,\gamma^\nu]\{ \pi_\mu,\pi_\nu\} .
	\]
	Naturally, the commutator is antisymmetric while the anti-commutator is symmetric. Thus 
	as we sum over all $\mu$, $\nu$, it will vanish (in the case where $\mu = \nu$, the 
	commutator of course also vanishes). For the kinetic momentum commutator
	\ba
		[\pi_\mu,\pi_\nu] &= i\h ce\plr{ \blr{\pdiff{x^\mu},A_\nu}+\blr{\pdiff{x^\nu},A_\mu}}\\
		& = i\h c e\plr{\pdiff{x^\mu}A_\nu-\pdiff{x^\nu}A_\mu} \\
		& = i\h c e F_{\mu\nu}.
	\ea
	We may now use this result to form 
	\[
		 \frac{1}{4}[\gamma^\mu,\gamma^\nu]\{ \pi_\mu,\pi_\nu\}  = \frac{ i\h c e }{4}
		 [\gamma^\mu,\gamma^\nu]F_{\mu\nu}.
	\]
	In the case where $\mu = 0$, $\nu\ne0$ we have
	\[
		[\gamma^0,\gamma^\nu] = 2\sigma_\nu \bpm 0 &\mathds 1 \\ \mathds 1&0\epm
	\]
	so 
	\ba
		 \frac{ i\h c e }{4}
		 [\gamma^0,\gamma^\nu]F_{0\nu} &= \frac{i\h c e}{2}\sigma_\nu F_{0\nu}
		 \bpm 0 &\mathds 1 \\ \mathds 1&0\epm
	\ea
	The components of $F_{0\nu}$ are just the components of $\vec E$ so we may rewrite 
	this as 
	\be
		 \frac{ i\h c e }{4}
		 [\gamma^0,\gamma^\nu]F_{0\nu} =  \frac{ i\h c e }{2}\vec \sigma\cdot \vec E
		 \bpm 0 &\mathds 1 \\ \mathds 1&0\epm \tag{4}
	\ee
	Similarly, for $\mu \ne 0$, $\nu = 0$, we have the same result due to antisymmetry
	in the commutator canceling antisymmetry in the EM tensor. Lastly, 
	for $\mu$, $\nu = 1,2,3$ we know that
	\[
		[\gamma^i,\gamma^j] = 2i\epsilon_{ijk}\sigma_k \bpm \mathds 1&0 \\ 0&\mathds 1\epm
	\]
	so
	\[
		 \frac{ i\h c e }{4}
		 [\gamma^i,\gamma^j]F_{ij} = -\frac{ \h c e }{2}\epsilon_{ijk}\sigma_kF_{ij}
		 \bpm \mathds 1&0 \\ 0&\mathds 1\epm.
	\]
	The off diagonal elements of $F_{ij}$ for $i$, $j = 1,2,3$ are precisely the components of the 
	magnetic field so we have
	\be
		 \frac{ i\h c e }{4}
		 [\gamma^i,\gamma^j]F_{ij} = \h c e (\vec\sigma\cdot\vec B)
		 \bpm \mathds 1&0 \\ 0&\mathds 1\epm\tag{5}
	\ee
	Combining $(3)$, $(4)$, and $(5)$ we finally have 
	\[
		\plr{\gamma^\mu \pi_\mu}^2 = \pi_\mu\pi^\mu +\h e c 
		\bpm \vec\sigma\cdot\vec B & i\vec \sigma\cdot \vec E \\
		i\vec\sigma\cdot\vec E & \vec\sigma\cdot\vec B \epm.
	\]
\eenum
\end{document}