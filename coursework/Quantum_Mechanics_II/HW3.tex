\documentclass[10pt,letterpaper]{article}
\usepackage{macroshw}

\title{\begin{spacing}{1.2}Quantum Mechanics II\\HW 3\end{spacing}}
\author{Matthew Phelps}
\date{Due: Sept. 30}

\begin{document}
\maketitle

\benum
% #1 -----------------------------------------------------------------------------------------------------------------------------------------------------------------
  	\item 
	Radial WKB and perturbation theory
	\benum
		% (a)
		\item
		Show that with the ``Langer substitution", the radial WKB quantization condition yields the exact bound state spectrum for
		the hydrogen atom. \\
		Hint: Write the WKB integral in terms of the turning points, and then only at the end, after you have done
		the integral, substitute the actual expressions for the turning points.
		\\
		\\
		Using the Langer substitution, the effective potential for the hydrogen atom is
		\[
			V_{eff} = -\frac{e^2}{r}+\pfrac{\h^2}{2m}\frac{(l+\frac{1}{2})^2}{r^2}.
		\]
		Thus our bound state WKB energy estimate
		\[
			\int_{r_0}^{r_1} dr\ \sqrt{\frac{2m}{\h^2}(E-V)}
		\]
		becomes
		\[
			\int_{r_0}^{r_1} dr\ \sqrt{\frac{2m}{\h^2}\plr{E+\frac{e^2}{r}}-\frac{(l+\frac{1}{2})^2}{r^2}}.
		\]
		We then rescale $r$ by the wavevector $k$ such that 
		\[
			\rho  = \sqrt{\frac{-2mE}{\h^2}}r = kr
		\]
		in which our integral becomes
		\[
			\int_{\rho_0}^{\rho_1} d\rho\ \sqrt{-1+\frac{2me^2}{\h^2k\rho}-\frac{(l+\frac{1}{2})^2}{\rho^2}}.
		\]
		We are left with the dimensionless parameter which may be expressed in the relevant scales as
		\[
			\frac{me^2}{\h^2 k} = \frac{1}{ak} = \frac{1}{\alpha}\pfrac{\lambda}{\lambda_C}.
		\]
		with $a$, $\lambda$, $\lambda_C$ and $\alpha \equiv \frac{1}{137}$ being the Bohr radius, de Broglie wavelength, Compton 
		wavelength, and fine structure constant respectively. For now we will express our parameter in terms of $k$ and $a$. We
		may further simplify this integral by putting the terms under the radical in factored quadratic form
		\[
			\int_{\rho_0}^{\rho_1} d\rho\ \frac{1}{\rho}\sqrt{-(\rho-\rho_0)(\rho-\rho_1)}.
		\]
		Since we have essentially factored the solution to $E=V_{eff}$, the zeros of the quadratic are precisely our turning points
		\[
			\rho_{0,1} = \frac{1}{ak}\plr{1\pm \sqrt{1-\plr{ak(l+\frac{1}{2})}^2}}.
		\]
		The solution to the integral is
		\[
			\int_{\rho_0}^{\rho_1} d\rho\ \frac{1}{\rho}\sqrt{-(\rho-\rho_0)(\rho-\rho_1)} = \frac{\pi}{2}\plr{\rho_0+\rho_1
			-2\sqrt{\rho_0\rho_1}}
		\]
		thus according the WKB energy quantization,
		\[
			\frac{1}{2}\plr{\rho_0+\rho_1-2\sqrt{\rho_0\rho_1}} = n'+\frac{1}{2}.
		\]
		Substituting the appropriate expressions for $\rho_0$ and $\rho_1$, we arrive at
		\[
			\frac{1}{ak}- \plr{l+\frac{1}{2}}= n'+\frac{1}{2}
		\]
		or
		\[
			E=\frac{-\h^2}{2ma^2(n'+l+1)^2} \qquad l=0,1,2,3..\quad n'=0,1,2,3..
		\]
		Denoting the principle quantum number
		\[
			n = n'+l+1
		\]
		we retain the full hydrogen energy spectrum
		\[
			E=\frac{-\h^2}{2ma^2n^2} \qquad n=1,2,3..
		\]
		\\
		\\
		% (b)
		\item
		Consider a radial potential 
		\[
			V = -\frac{e^2}{r}+V_{int}(r)
		\]
		Using the fact that at large $n$ the radial hydrogenic wavefunctions behave like [check this for yourself with some plots!]
		\[
			R_{n,l}(r) \approx \frac{\sqrt 2J_{2l+1}\plr{2\sqrt 2\sqrt{r/a}}}{a^{3/2}n^{3/2}\sqrt{r/a}}
		\]
		show that the first-order shift in the energy can be written in the form
		\[
			\Delta E_N^{(1)}=\frac{e^2}{a}\frac{\delta}{n^3}
		\]
		where $\delta$ is some integral transform of $V_{int}$. Hence show that to this order the energy levels can be
		written as an ``effective shift in n":
		\[
			E_n\approx -\frac{e^2}{2a(n+\delta)^2}. 
		\]
		Comment: Note that Mathematica has a slightly unusual convention for writing the Laguerre polynomials so that the
		hydrogenic radial wavefunctions are (in dimensionless radial coordinates)
		\[
			R_{n,l}(r) = \frac{2^{l+1}e^{-\frac{r}{n}}\sqrt{\frac{(-l+n-1)!}{(l+n)!}}\pfrac{r}{n}^l\ LaguerreL[n-l-1,2l+1,(2r)/n]}{n^2}
		\]
		It's worth checking that your version of Mathematica has this same convention: check the first few with the radial 
		wavefunctions from a QM book. 
		\\
		\\
		\\
		To calculate the first order energy correction of the $n$th state, we use
		\[
			\Delta E_n^1 = \braket{n^0|H^1|n^0}
		\]
		where $H^0\ket{n^0} = E_n^0\ket{n^0}$ denote the energy eigenstates. For our hydrogen atom the first 
		order energy correction is
		\ba
			\Delta E_n^1 &= \braket{n^0|V(r)|n^0} \\
			& = \int_S d\Omega\ |Y^l_m(\theta,\phi)|^2  \int_0^\infty dr\ r^2 |R_{n,l}(r)|^2\,V(r) \\
			& = \int d\rho\ |R_{n,l}(\rho)|^2 \\
			& = \frac{2}{a^3n^3}\int_0^\infty dr\ r^2 \frac{\blr{J_{2l+1}(2\sqrt 2\sqrt\frac{r}{a})}^2}{r/a}V(r)\\
			& = \frac{e^2}{an^3}\int_0^\infty d\rho\ \rho \blr{J_{2l+1}\plr{2\sqrt 2\sqrt\rho}}^2\pfrac{2a}{e^2}V(\rho)\\
			& = \frac{e^2\delta}{an^3}
		\ea
		where $\delta$ is a dimensionless integral of $V_{int}(r)$ in dimensionless variable $\rho = r/a$. Adding
		this correction to unperturbed hydrogen eigenstates, we have
		\ba
			E_n &\approx E_n^0+E_n^1\\
			& = -\frac{e^2}{2an^2}+\frac{e^2\delta}{an^3}\\
			& = -\frac{e^2}{2a}\plr{\frac{1}{n^2}-\frac{2\delta}{n^3}}\\
			& = -\frac{e^2}{2a}\plr{\frac{1-2\frac{\delta}{n}}{n^2}}\\
			& \approx -\frac{e^2}{2a}\pfrac{1}{n^2\plr{1+\frac{\delta}{n}}^2}\\
			& = -\frac{e^2}{2a(n+\delta)^2}
		\ea
		Thus a first order approximation to the large $n$ energy spectrum of hydrogen can be calculated by 
		\[
			-\frac{e^2}{2a(n+\delta)^2}
		\]
		where $\delta$ is a dimensionless integral transform of $V_{int}(r)$. 
		\\
		\\
		\emph{What integral transform is this exactly? Looks similar to a Hankel transform.}
		\\
		\\
	\eenum
	
	
% #2-----------------------------------------------------------------------------------------------------------------------------------------------------------------
	\item
	Consider a constant electric field $\mathcal E$ applied to a hydrogen atom in the $n=3$ energy level.
	
	\benum
		% (a)
		\item 
		List the relevant states for first-order perturbation theory and use symmetry arguments to show that only a small number
		of non-zero matrix elements are needed.
		\\
		\\
		Our task is to evaluate the matrix elements of the degenerate states of in the $n=3$ energy level of hydrogen
 		which are of the form
 		\[
			\braket{3l'm'|H'|3lm} = \braket{3l'm'|e\mathcal{E}z|3lm}
		\]
		where $\mathcal E$ denotes the strength of the electric field. We do not account for relativistic corrections nor 
		spin orbit coupling here. For the $n=3$ level, there is $3^2 = 9$ fold degeneracy and thus the first order
		perturbation matrix would contain a total of $81$ elements. 
		\\
		\\ 
		First off, the matrix is Hermitian and so we only need to calculate $n(n+1)/2 = 45$ elements. Next, since the 
		perturbation $H'$ is a tensor operator
		\[
			H' = e\mathcal Ez \sim T_{q=0}^{(k=1)}
		\]
		we may use the Wigner-Eckart theorem
		\[
			\braket{3l'm'|T_{q=0}^{(k=1)}|3lm} = \braket{lk;mq|jk;j'm'}\frac{\braket{3l'||T^{(k)}|3l}}{\sqrt{2l+1}}
		\]
		to conclude that the Clebsch-Gordan coefficients vanish unless
		\[
			l' = l-1,l,l+1\quad \text{and} \quad m' = m.
		\]
		In addition, since the eigenstates of hydrogen have definite parity and the perturbation $H'$ is of odd parity,
		we deduce that matrix elements of the same $l=l'$ must vanish. Thus, the criteria for non-vanishing matrix
		elements is then
		\[
			\braket{3l'm'|e\mathcal{E}z|3lm} = 0\quad\text{unless}\quad 
			\begin{cases} l'=l\pm1\\m'=m \end{cases}
		\]
		This leaves us with
		\[
			\braket{300|H'|310},\quad \braket{31(-1)|H'|32(-1)},\quad \braket{310|H'|320},\quad \braket{311|H'|321}
		\]
		\\
		% (b)
		\item
		Compute the non-zero matrix elements, and hence find the first-order Stark shifts.
		\\
		\\
		Evaluating these elements,
		\ba
			\braket{300|H'|310} &= e\mathcal E a\plr{-3\sqrt 6}
			\\
			 \braket{31(-1)|H'|32(-1)} &=  e\mathcal E a\plr{-\frac{9}{2}}
			 \\
			 \braket{310|H'|320} & =  e\mathcal E a\plr{-3\sqrt 3}
			 \\
			  \braket{311|H'|321} & =  e\mathcal E a\plr{-\frac{9}{2}}
		\ea
		Forming these elements into a matrix equation and solving for the eigenvalues yields the first order 
		energy corrections of 
		\[
			E_3 = E_3^0 \pm 9 e\mathcal E a \quad \text{and}\quad E_3 = E_3^0\pm \frac{9}{2}e\mathcal E a.
		\]
		Thus the $n=3$ hydrogen energy level is split into four perturbed energies under application of an external
		electric field. 
		\\
		\\
		% (c)
		\item
		What remaining degeneracies are there?
		\\
		\\
		In applying the electric field, we have broken the ``theta" symmetry but the perturbation still commutes with 
		$L_z$ and thus possesses azimuthal symmetry. Therefore, we expect extra degeneracies to exist due to this
		symmetry. Would it be correct to say the degeneracy in $l$ has been lifted, but our system still is 
		degenerate in $m$? 
		\\
	\eenum
	
	
% #3 ------------------------------------------------------------------------------------------------------------------------------------------------------------
	\item
	Second order Stark shift
	\benum
	
		% (a)
		\item
		Consider the following expression for the second-order Stark shift of the hydrogen ground state energy level, coming from
		summing over all bound states other than the ground state:
		\[
			\Delta E_1^{(2)} = \sum_{(n,l,m)\ne(1,0,0)} \frac{|\braket{n,l,m|e\mathcal E z|1,0,0}|^2}{E_1-E_n}
		\]
		Without doing any computation, explain the \textbf{sign} of this second-order shift.
		\\
		\\
		As the modulus squared of a complex number is real and positive, we are left to analyze the denominator, which
		must be negative since we are summing over states with energy greater than the ground state. Thus 
		our second order energy shift is negative. 
		\\
		
		% (b)
		\item
		Use scaling and symmetries to reduce this expression to an expression of the form
		\[
			\Delta E_1^{(2)} = E_1^{\text{Bohr}}\times\text{(dimensionless measure of strength of the applied field)}
			\times \sum_{n=2}^\infty F(n)
		\]
		for some numbers $F(n)$, given by dimensionless integrals involving the hydrogenic radial wavefunctions. Explain
		the physical meaning of the dimensionless measure of the strength of the applied field that arises in your expression.
		\\
		\\
		We could use the same selection rules and parity as in question 2 to deduce that only the $l=1$ $m=0$ elements survive. Or,
		alternatively, we may see that
		\[
			z = rcos\theta = r\sqrt{\frac{4\pi}{3}}Y_0^1(\theta,\phi)
		\]
		so we integrate the angular dependence
		\[
			\sqrt{\frac{4\pi}{3}}\int d\Omega\ Y_m^lY_0^1Y_0^0 = \sqrt{\frac{4\pi}{3}}\pfrac{1}{\sqrt{4\pi}}
			\int d\Omega\ Y_m^lY_0^1=  \pfrac{1}{\sqrt 3} \delta_{m,0}\delta_{l,1}
		\]
		which leads us to the same conclusion. As a result, the square modulus of the matrix elements goes as
		\[
			|\braket{n,l,m|e\mathcal E z|1,0,0}|^2 = \frac{1}{3}(e\mathcal E a)^2
			\blr{\int_0^\infty d\rho\ \rho^3 R_{n,1}(\rho)R_{1,0}(\rho)}^2
		\]
		where $\rho$ and $R_{n,l}(\rho)$ are dimensionless. If we now look at the denominator in the summation
		of our second order energy correction, we have
		\[
			E_1 - E_n = E_1\plr{1-\frac{1}{n^2}} = E_1\pfrac{n^2-1}{n^2}.
		\]
		Putting all this together, the energy correction is
		\[
			\Delta E_1^{(2)} = \frac{1}{3}(e\mathcal E a)^2\frac{1}{E_1}\sum_{n=2}^\infty
			\pfrac{n^2}{n^2-1}\blr{\int_0^\infty d\rho\ \rho^3 R_{n,1}(\rho)R_{1,0}(\rho)}^2.
		\]
		This may be rearranged as
		\[
			\Delta E_1^{(2)} = E_1^{(0)}\pfrac{\mathcal E a^2}{e}^2 \frac{4}{3}
			\sum_{n=2}^\infty F(n)
		\]
		where of course
		\[
			F(n) = \pfrac{n^2}{n^2-1}\blr{\int_0^\infty d\rho\ \rho^3 R_{n,1}(\rho)R_{1,0}(\rho)}^2.
		\]
		The dimensionless strength of the applied field
		\[
			\frac{\mathcal E}{e/a^2}
		\]
		sets a natural scale relative to the electric field produced by a proton at the Bohr radius $a$. This is the same
		magnitude of the field acting upon an electron in the hydrogen atom (assuming the electron is at $a$, which it is 
		on average).  
		\\
		% (c)
		\item
		The numbers $F(n)$ are difficult to compute in closed-form, but are simple to compute numerically: for example
		use the NIntegrate command in mathematica.\\
		Evaluate $\sum_{n=2}^{\infty} F(n)$ numerically by summing up to $n=20$, and compare with the exact answer
		which we derived in class.
		\\
		\\
		Utilizing mathematica we have
		\[
			\sum_{n=2}^{20} F(n) = 2.74.
		\]
		Substituting this into our formula for $\Delta E_1^{(2)}$
		\[
			\Delta E_1^{(2)} \approxeq E_1^{(0)}\pfrac{\mathcal E a^2}{e}^2(3.66).
		\]
		In class we derived this as 
		\[
			\Delta E_1^{(2)} = E_1^{(0)}\pfrac{\mathcal E a^2}{e}^2\pfrac{9}{2}.
		\]
		\\
		 % (d)
		 \item
		 Why is there a mis-match, and what would you need to do to correct it [you don't need to do this, just explain briefly what should 
		 be done]?
		 \\
		 \\
		 As a numerical solution, we may continue to approximate the second order energy shift to higher accuracy by summing over 
		 more terms. However, since this is an infinite summation we will never reach the exact result. Instead, we may use the first
		 order equation in the perturbation
		 \[
		 	(H^{(0)}-E^{(0)}_1)\ket{\psi^{(1)}_1} = (E^{(1)}_1-H^{(1)})\ket{\psi^{(0)}_1}
		\]
		to solve for $\ket{\psi^{(1)}}$ in which we can find the second order energy from
		\[
			E^{(2)} = \braket{\psi^{(0)}_1|H^{(1)}|\psi^{(1)}_1}.
		\]
		In order to solve for $\ket{\psi^{(1)}_1}$ in this case, we may make an educated guess of something similar to
		\[
			\psi^{(1)}_1(r) = e^{-r/a}\cos\theta f(r),
		\]
		then substitute this into the modified radial equation to finally solve for $f(r)$. 
		\eenum
\eenum
\end{document}