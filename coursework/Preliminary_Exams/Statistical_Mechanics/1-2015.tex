\documentclass[10pt,letterpaper]{article}
\usepackage{macroshw}

\title{\begin{spacing}{1.2}Stat Mech Winter 2015\end{spacing}}
\author{Matthew Phelps}
\date{}

\begin{document}
\maketitle

\benum
% #1 --------------------------------------------------------------------------------------------------------------------------------------------------------
  	 \item
	You discover a new state of matter that has equation of state and internal energy given by
	\[
		p = A\frac{T^3}{V},\qquad U = BT^n\ln\frac{V}V_0 +f(T)
	\]
	where $A$, $B$, $n$, $V_0$ are constants while $f(T)$ only depends on the temperature. Find a
	numerical value for $n$ and find a relation between the constants $B$ and $A$. \\
	Hint: Find a Legendre transformation of the entropy $S(U,V)$ to a function that only depends on
	$T$ and $V$. \\ \\
	
	We note that we have $p(T,V)$ and $U(T,V)$. From the first law we have $U(S,V)$. Ultimately we seek to 
	relate $U$ to $p$. This can be achieved via some sort of Maxwell relation that derives $U$ and $p$ with
	respect to $T$ and $V$. Hence we need some thermodynamic quantity that is a function of $(T,V)$ with 
	$f(U(T,V),T,V)$ and $g(p(T,V),T,V)$ as the coefficients in the total derivative. From the first law, this suggests
	we need to legendre transform the entropy, as this will remove the entropy out of the picture. Thus
	we seek the transformation $S(U,V) \to W(T,V)$.
	\[
		dS(U,V) = \frac{dU}{T}+pdV
	\]
	Do the transformation
	\[
		W = S-\frac{U}{T}
	\]
	then
	\[
		dW(T,V) = dS-\frac{dU}{T}+\frac{U}{T^2}dT = \frac{U}{T^2}dT+pdV.
	\]
	Now we can use the maxwell relation
	\[
		\plr{\pdiff[p]{T}}_V = \plr{\pdiff[\pfrac{U}{T^2}]{V}}_T
	\]
	hence
	\[
		\frac{3AT^2}{V} = \frac{BT^{n-2}}{V}
	\]
	and since $A$ and $B$ are only constants
	\[
		3A = B;\qquad n=4.
	\]
	\\ \\ 
% #2 ------------------------------------------------------------------------------------------------------------------------------------------------------
	\item
	Consider a two-dimensional gas of noninteracting fermions with spin $s$ with energy dispersion given by
	the usual relativistic expression for a particle of mass $m$
	\[
		\epsilon(\vect k) = \sqrt{(mc^2)^2+(\h c|\vect k|)^2}
	\]
	The gas consists of $N$ particles in an area $A$ and has number density $n = \frac{N}{A}$. 
	\benum
	% (a)
	\item
	Find an expression for the Fermi wave vector $k_F$ for this gas as a function of number density $n$.
	
	% (b)
	\item
	Show the expression for the total energy of the gas at $T=0$. Show that contains the (purely algebraic)
	expression
	\[
		\int_{0}^{x_F} dx\ x\sqrt{x^2+1}.
	\]
	What is $x_F$?
	
	% (c)
	\item
	Consider the nonrelativistic limit $\h k_f \ll mc$. Find the lowest order terms in $x$ to obtain an expression
	for the total energy valid in this limit. Express your answer in terms of $N$ and $A$ and show that it has 
	the form $E = N(mc^2+\epsilon_{NR})$ where $\epsilon{NR}$ is the energy density of a nonrelativistic gas 
	of fermions. 
	
	% (d)
	\item
	Now consider the ultrarelativistic limit $\h k_F \gg mc$. Obtain an expression for the total energy valid
	in this limit.
	\\ \\
	\eenum 
	
	\benum
	%(a)
	\item
	Fermi quantities are always given at $T=0$. The fermi energy represents the difference in energy between the 
	lowest and highest energy of occupied states. At $T=0$, the occupation for fermions
	\[
		\braket{n_{\vect k}} = \frac{1}{e^{\beta(\ep_{\vect k} -\mu)} +1}
	\]
	becomes
	\[
		\theta(\mu-\epsilon_\vect k).
	\]
	The energy which equals the chemical potential is the fermi energy
	\[
		\mu(T=0) = \ep_F.
	\]
	We find the fermi wave vector by finding the particle number. Here I assume the states are quantized the
	same as in the nonrelativistic case $\vect k = \frac{2\pi}{L}(n_1\vecth x_1 +n_2\vecth x_2)$ where the
	$n$'s take on integer values. Hence we integrate over these values in the continuation
	\[
		\braket{N} = g\pfrac{L}{2\pi}^2\int_{\ep_\vect k < \ep_F} d^2k = \frac{gL^2}{2\pi} \int_0^{k_F} kdk.
	\]
	Our upper bound is the particular wavevector $|\vect k| = k_F$ such that $\ep(\vect k) = \ep_F$. As such
	\[
		n = \frac{gL^2}{4\pi}k_F^2
	\]
	\[
		k_F = \pfrac{4\pi n}{g}^{1/2}.
	\]
	\\ \\
	
	% (b)
	\item
	To find the total energy, we integrate over all states
	\[
		\frac{gL^2}{2\pi} \int_0^{k_F} dk\  k\sqrt{(mc^2)^2+(\h c k)^2} = \frac{g}{2\pi}\pfrac{mcL}{\h}^2(mc^2)
		\int_0^{x_F} dx\ x\sqrt{x^2+1}
	\]
	where $x = \frac{\h k}{mc}$ and so $x_F = \frac{\h k_F}{mc}$. 
	\\ \\
	
	\item %(c)
	For the nonrelativistic case $x\ll 1$, we may approximate the integral as
	\ba
		\frac{gL^2}{2\pi}\pfrac{mc}{\h}^2mc^2\int_0^{x_F} dx\ x\plr{1+\frac{1}2 x^2}
		& = \frac{gL^2}{2\pi}\pfrac{mc}{\h}^2mc^2\plr{\frac{x_F^2}{2}+\frac{x_F^4}{8}} \\
		& = \frac{gL^2}{2\pi}\pfrac{mc}{\h}^2mc^2\blr{\frac{1}{2}\pfrac{\h k_F}{mc}^2+\frac{1}{8}\pfrac{\h k_F}{mc}^4
		} \\
		& = N\pfrac{mc}{\h}^2mc^2\blr{ \pfrac{\h}{mc}^2+\pfrac{\h}{mc}^4\pfrac{k_F}{2}^2} \\
		& = N\plr{mc^2+\frac{\h^2 k_F^2}{4m}}.
	\ea
	This last factor is indeed the nonrelativistic total energy per particle as can be shown by
	\[
		\frac{U}{N} = \frac{\frac{gL^2}{2\pi} \int_0^{k_F} dk\ k\pfrac{\h^2k^2}{2m}}{\frac{gL^2}{4\pi}k_F^2} =
		\frac{\h^2 k_F^2}{4m} = \frac{\ep_F}{2}
	\] \\ \\
	
	\item
	% (d)
	For the ultrarelativistic case, we would like to use $x \gg 1$. The limits in this problem rely on the fermi
	wavevector, not just $\vect k$. So really this is more strict than the usual relativistic limit. We shall proceed
	with the usual $x\gg 1$ since I'm not sure how to fufill the other approximation for small values of $k$. At
	any rate, we may approximate the integrand as
	\[
		x\sqrt{x^2+1} = x^2\sqrt{\frac{1}{x^2}+1} \approx x^2\plr{1+\frac{1}{2x^2}} = \frac{1}{2}+x^2
	\]
	so our integral becomes
	\ba
		U &= \frac{gL^2}{2\pi}\pfrac{mc}{\h}^2mc^2\int_0^{x_F} dx\ \frac{1}{2}+x^2 \\
		& =  \frac{gL^2}{2\pi}\pfrac{mc}{\h}^2mc^2\plr{\frac{x_F}{2}+\frac{x_F^3}{3}}
	\ea
	I don't think the simplification brings anything meaningful, so I leave as is. \\
	*Interestingly, if we evaluate the integral in (b) and then take the limit as $x_F \gg 1$, we
	obtain the same result as the approximation we did in (d). Perhaps this means $\h k \gg mc $ is a 
	good approximation at the origin as well?* \\ \\ \\
	\eenum
% #3 ---------------------------------------------------------------------------------------------------------------------------------------------------
	\item
	Consider the transverse modes of a quantum mechanical string of length $L$ and uniform mass
	per unit length $\mu$ stretched with tension $F$ between two fixed points. The string is in thermal
	equilibrium at temperature $T$. The transverse size of the string is negligible compared to its length,
	and the effects of gravity on its motion are negligible. Consider the transverse displacement $\vect y$
	of the string, small compared to $L$, such that is obeys the linear equation of motion
	\[
		\mu \difff{\vect y}{*2t} = F\difff{\vect y}{*2x}
	\]
	The solutions to this wave equations are of the form
	\[
		\vect y(x,t) = (a_k \vect e_1 +b_k\vecth e_2)\sin(kx-\omega t)
	\]
	where the sum is over all wavenumber $k$ values that obey the boundary conditions of the string, 
	and $\omega = vk$ where $v = \sqrt{F/\mu}$ is the transverse velocity on the string.
	\benum
	% (a)
	\item
	What is the energy of a single quantum excitation in a mode with wavenumber $k$?
	\item
	% (b)
	Write down the grand canonical partition function for the system.
	% (c)
	\item
	What is the heat capacity $C$ of the system in the limit $kT\gg \h\omega_1$ is the frequency of the 
	fundamental mode?
	\\ \\ 
	Hint: You may find the following integral to be useful:
	\[
		\int_0^\infty dx\ \frac{x}{e^x-1} = \frac{\pi^2}{6}.
	\]
	\\ \\
	\eenum
	
	\benum
	% (a)
	\item
	In some fashion, we are working with phonons here, which are to be massless bosons. As massless particles,
	their relativistic energy is given as
	\[
		E = pv
	\]
	where $v = \sqrt{\frac{F}{\mu}}$. If the phonons are quantized as harmonic oscillators, then the energy is 
	\[
		E = \h\omega
	\]
	where $\frac{\omega}{k} = v$. From the boundary conditions, $k$ is quantized and we end up at
	\[
		E = \h \omega = \h v\frac{\pi}{L}n\qquad n=0,1,2,3,....
	\]
	This is not exactly clear. On the other hand, we can Fourier transform the wave equation, treating 
	$\vect y(x,t)$ as a quantum mechanical wavefunction
	\[
		\int_{-\infty}^{\infty} dx\ e^{-ixp/\h}\pfrac{p}{\h}^2 \vect y(x,t) =-\frac{1}{v^2\h^2} 
		\int_{-\infty}^{\infty} dx\ e^{-ixp\/h}\plr{i\h\pdiff{t}}^2 \vect y(x,t)
	\]
	Here we have turned the differential operator into $p$. Note that we (should) also be able to $i\h\pdiff{t}\psi
	= E\psi$. Thus we can write everything as
	\[
		\pfrac{p}{\h}^2 \tilde{\vect y}(p,t) = -\pfrac{E}{v\h}^2\tilde{\vect y}(p,t).
	\]
	And we can deduce that
	\[
		E = pv = \h k v = \h v \frac{\pi}{L} n\qquad n=0,1,2,3,....
	\]
	where boundary conditions on $\lambda$ have been used. This is very sloppy admittedly. The energy of a single
	quantum excitation would be $E = \h v\frac{\pi}{L}$.
	\\ \\
	
	%(b)
	\item
	The wavevector $k$, given by integers, defines a single mode of oscillation. Each mode can support 
	multiple ``excitations" i.e. phonons. This is quite analogous to photons in a cavity. In this case, we have two 
	possible polarizations from the transverse vectors. The partition function of a single mode (and single polarization) 	is
	\[
		Z_\vect k = \sum_{n=0}^\infty e^{-\beta \h\omega_k n} = \frac{1}{1-e^{-\beta \omega_k}}. 
	\]
	I think the most straightforward way to find the grand canonical partition function is to use it in the form of
	\[
		\mathcal Z = \prod_i \sum_{n_i} e^{-\beta n_i(\ep_i-\mu)}
	\]
	where we sum over all possible occupation numbers $n_i$. For this boson case, this becomes
	\[
		\mathcal Z = \prod_i \frac{1}{1-e^{-\beta(\epsilon_i-\mu)}}.
	\]
	In this form, we have a product of individual grand partition functions corresponding to single energy states
	indexed by $i$. But since we have two transverse modes, we must not forget to add in the degeneracy 
	factor, thus
	\[
		\mathcal Z = \prod_i \pfrac{1}{1-e^{-\beta(\epsilon_i-\mu)}}^2.
	\]
	The single particle energy eigenstates $\ep_i$ are simply the different modes $\h \omega n$. Thus
	\[
		\prod_{n}^\infty \pfrac{1}{1-e^{-\beta(\h\omega n-\mu)}}^2
	\]
	In this form we may nicely compute $\Omega = -kT\ln \mathcal Z$. **Importantly, we must take care to
	note that the number of phonons is not conserved. They can easily be absorbed or emitted (by the counter
	part to the electron).** Thus $\mu$ is zero and we should have
	\[
		\mathcal Z = \prod_{n}^\infty \pfrac{1}{1-e^{-\beta(\h\omega n)}}^2.
	\]
	Since we have ignored the zero-point energy $\frac{1}2 \h\omega$, there should really be no discontinuity at the 	ground state.
	\\ \\
	
	% (c)
	\item
	It would appear everything so far mimics what we typically expect for phonons trapped in some sort of
	cavity. Based on the form of the partition function the occupancy should be same as the typical boson 
	occupancy, apart from the degeneracy factor
	\[
		\braket{n_k} = \frac{2}{e^{\beta(\ep_k -\mu)}-1}.
	\]
	We proceed to compute the energy
	\[
		U = \sum_k \ep_k\braket{n_k} = \sum_{n=0}^{\infty} \frac{2\h\omega_1 n}{e^{\beta(\h\omega_1 n-\mu)}-1}
	\]
	where $\omega_1 = v\frac{\pi}{L}$. Another interesting way to derive the energy is to note that
	\[
		\mathcal Z = \tr(e^{-\beta(H}) = \sum_{N,k} \braket{N,k|e^{\beta H_k}|N,k}  = \braket{e^{-\beta H}}
	\]
	and thus
	\[
		-\pdiff{\beta} \ln \mathcal Z = \sum_{N,k} \frac{E_k e^{-\beta E_k}}{\mathcal Z} = \braket{H}.
	\]
	Now applying this general ($\mu =0$) relation to our specific partition function,
	\[
		-\pdiff{\beta} \plr{\ln \mathcal Z} = -\pdiff{\beta}\plr{\sum_n 2\ln\pfrac{1}{1-e^{-\beta(\h\omega_1 n)}}}
		= \sum_{n=0}^{\infty} \frac{2\h\omega_1 n}{e^{\beta(\h\omega_1 n-\mu)}-1}.
	\]
	This is consistent, but note that it did not advance us towards calculating the energy in the end. To do so,
	we move to the continuum (valid as $L\to \infty$, thermodynamic limit)
	\[
		U = \int_0^\infty dn\ \frac{2\h\omega_1 n}{e^{\beta(\h\omega_1 n)}-1} = \frac{1}{\beta^2\h\omega_1}
		\int_0^\infty dx\ \frac{x}{e^x-1} = \frac{(kT)^2}{\h\omega_1}\frac{2\pi^2}{3}.
	\]
	To find the heat capacity
	\[
		C = \diff[U]{T} = \frac{k^2 T}{\h\omega_1}\frac{\pi^2}{3}
	\]
	In the limit $kT\gg \h\omega_1$ we mustn't ignore our integral. In this regime we have
	\[
		 \frac{1}{\beta^2\h\omega_1} \int_0^\infty dx\ \frac{x}{(1+x)-1} = \infty
	\]
	I'm not sure at this point. Typically with phonons, we have a maximum Debye mode that corresponds 
	to a minimum wavelength that is twice the distance between atoms. But with this quantum mechanical string
	I am not sure. In the event we knew the Debye frequency, we should replace the integral with
	\[
		U = \frac{1}{\beta^2\h\omega_1} \int_0^{x_D} dx\ \frac{x}{(1+x)-1} 
		 = \frac{(kT)^2}{\h\omega_1}\pfrac{T_D}{T}^2 = \frac{k^2T_D^2}{\h\omega_1}.
	\]
	\\
	\\
	\eenum 
% #4 ---------------------------------------------------------------------------------------------------------------------------------------------------
	\item
	A refrigerator consists of a system of a large number $N$ of distinguishable atoms placed in a variable
	uniform external magnetic field of magnitude $B$. The atoms have a total electronic spin of $J$ and
	maximum magnetic moment $\mu_{max} = \mu_0J$. The closed thermal cycle works as follows.
	\benum
	\item
	The magnetic field is ramped up adiabatically until the temperature reaches $T_2$ at field $B_2$.
	\item
	The system is brought into thermal contact with the hot reservoir at temperature $T_2$, and the 
	magnetic field is ramped up further to maximum field $B_{max}$ expelling $Q_2$ as heat.
	\item
	The system is removed from thermal contact with the hot reservoir and the magnetic field is
	ramped adiabatically down until the temperature reaches $T_1$ at field $B_1$.
	\item
	The system is brought into thermal contact with the cold side at temperature $T_1$, and 
	the magnetic field is ramped down to $B_{min}$, absorbing heat $Q_4$ at heat.
	\item
	The system is removed from thermal contact with the cold reservoir and step 1 resumes.
	\\ \\
	\eenum

	The atoms interact with the external field but not with one another. The interaction of one atom with
	the external field is given by $H_{int} = -\mu_0 m B$ where $m$ labels the projection of the atom's spin
	onto the magnetic field axis.
	
	
	\benum
	% (a)
	\item
	What is the entropy of the system in the limit $B\to 0$ at fixed temperature? Hint: Use the expression
	$S = -k\sum_i P_i\log P_i$ ($P_i$ is the probability for configuration $i$), and make a general argument
	based on degeneracy of states. If you attempt to compute a closed form expression for $S$ at finite
	$B$ and take the limit of large field, you will find the calculation to cumbersome. 
	
	% (b)
	\item
	What is the entropy of the system in the limit of strong field $B$ at fixed temperature? What is the field
	strength $B_0$ that distinguishes the strong and weak field limits?
	
	%(c)
	\item Suppose field $B_1 \ll B_0$ and $B_2\gg B_0$. During steps 1 and 3 the magnetic field $B$ is
	changing when the system is not in thermal contact with either reservoir, so $\delta S = 0$
	during those steps. Does this contradict the result from parts (a) and (b)? If not, what condition must
	be satisfied?
	
	% (d)
	\item
	Derive the maximum theoretical efficiency that can be obtained using this refrigerator cycle, defined 
	as $\epsilon = \frac{Q_2-Q_4}{Q_2}$. 
	\\ \\
	\eenum
	
	\benum
	% (a)
	\item
	We can denote a configuration of the system with a specified energy as $E = -\mu_0 S_i B$ where 
	$\mu_0S_i$ is the total magnetic moment for the entire system. We expect each energy configuration
	to be degenerate, and so the partition function (fixed $N$) then becomes
	\[
		Z = \sum_i g_i e^{\beta\mu_0 S_i B} = \sum_{S_i=-NJ}^{NJ} g_i e^{\beta \mu_0 S_i B}.
	\]
	Note that our energy can vary from all spins aligned or anti-aligned with the magnetic field axis. If we
	now form the entropy
	\ba
		S &= -k\tr(\rho\ln\rho) = \sum_i p_i\ln(p_i) = \frac{\sum_i g_ie^{\beta\mu_0S_iB}\ln(e^{\beta\mu_0S_iB}/Z)}{Z}\\
		& = \frac{-k}{Z}\sum_i g_i e^{\beta\mu_0S_iB}\plr{\beta\mu_0S_iB-\ln Z} 
	\ea
	In the limit $B\to 0$, we see that $Z = \sum g_i = M$, the total number of distinct states. Our entropy
	becomes in this limit
	\[
		\lim_{B\to 0}S = \frac{k}{Z}M\ln M = k\ln M.
	\]
	
	% (b)
	\item
	As we take the limit to $B\to \infty$
	\[
		e^{\beta\mu_0S_iB} \begin{cases} 0&\qquad S_i < 0 \\ 1&\qquad S_i = 0 \\ \infty &\qquad S_i >0
		\end{cases}.
	\]
	In the large $B$ limit, any probability contribution from a net anti-aligned spin decays toward zero, 
	and configuration with a net positive spin increase exponentially. However, I was not able to evaluate this
	limit. 
	
	% (c)
	\item
	
	% (d)
	\item
	The efficiency is given by the ratio of heat taken away from the low temperature reservoir to the 
	work input.
	\[
		W = (T_2-T_1)(S_2-S_1).
	\]
	The heat absorbed is $Q_1 = T_1\Delta S$
	and thus
	\[
		\eta = \frac{Q_1}{W} = \frac{T_1}{T_2-T_1}
	\]
	\eenum
	\eenum
\end{document}