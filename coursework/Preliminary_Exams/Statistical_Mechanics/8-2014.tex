\documentclass[10pt,letterpaper]{article}
\usepackage{macroshw}

\title{\begin{spacing}{1.2}Stat Mech Summer 2014\end{spacing}}
\author{Matthew Phelps}
\date{}

\begin{document}
\maketitle

\benum
% #1 --------------------------------------------------------------------------------------------------------------------------------------------------------
  	 \item
	Consider a system of $N$ non-interacting diatomic molecules at temperature $T$ inside a volume $V$. 
	Let the Hamiltonian function of a single molecule be given by
	\[
		H(\vect p_1,\vect p_2, \vect r_1,\vect r_2) = \frac{\vect p_1^2+\vect p_2^2}{2m} 
		+\alpha \frac{ |\vect r_1-\vect r_2|^2}{2}
	\]
	
	\benum
	%(a)
	\item
	Derive the expression for the classical canonical partition function, and show that it is
	of the form $Z(T,V,N) = c_N V^N(kT)^{9N/2}$ where $c_N$ is a function of $N$. 
	% (b)
	\item
	Derive an equation of state of the form $f(p,T,V,N) = 0$.
	% (c)
	\item
	Find the heat capacity at constant volume $C_v(T,V,N)$ \\ \\ \\
	\eenum
	
	\benum
	% (a)
	\item 
	For the classical continuous partition function, we must integrate over the phase space density. Since
	the particles are non-interacting, we have the factorization
	\[
		Z = (Z_1)^N.
	\]
	In addition, the momentum and spatial dependence in the hamiltonian factorizes (when used in
	the Boltzmann factor $e^{-\beta H}$, so
	\[
		Z_1 = \frac{1}{h^6}\iiint_{-\infty}^{\infty} d^3p_1d^3p_2\  e^{-\beta\pfrac{\vect p_1^2+\vect p_2^2}{2m}}
		\iiint_{-L/2}^{L/2}d^3q_1d^3q_2\  e^{-\beta\alpha\pfrac{|\vect r_1-\vect r_2|^2}{2}}
	\]
	For the purpose of this problem, the quantization of the momentum eigenvalues $k=\frac{2\pi}{L}n$
	subject to boundary conditions $\psi(x) = \psi(x+L)$ has been done in a box centered at the origin. This
	way our position integral runs over even bounds, which allow us to evaluate the Gaussian at infinity. 
	\\ \\
	For the momentum portion, we need only evaluate a single instance of 
	\[
		\iiint_{-\infty}^{\infty} d^3p\ e^{-\frac{\beta}{2m}\vect p^2} = \pfrac{2m\pi}{\beta}^{3/2}.
	\]
	Now, for the spring potential, the Guassian at finite limits gives us a error function (not useful). The only 
	realistic way of getting a sensible answer here is to take $L\to \infty$
	\[
		\lim_{L\to\infty} \iiint_{-L/2}^{L/2}d^3q_1d^3q_2\  e^{-\beta\alpha\pfrac{|\vect r_1-\vect r_2|^2}{2}}
	\]
	The only relevant integral here is
	\[
		\iint_{-\infty}^\infty dq_1dq_2\ e^{-\frac{\beta\alpha}{2}(q_1-q_2)^2} = \sqrt{\frac{2\pi}{\beta\alpha}}
	\]
	Even when you are integrating over another variable (such as $q_2$), the result will never
	depend on $q_2$. You would think it would have to if you break up the Gaussian as
	\[
		e^{-\frac{\beta\alpha}{2}(x_1^2-2x_1x_2)}e^{-\frac{\beta\alpha}{2}x_2^2}
	\]
	yet the result comes out exactly the same. I guess the point is the integral will always be 
	$\sqrt{\frac{2\pi}{\beta\alpha}}$ irrespective of any possible $q_2$ value. Anyway, the 
	position integrals are now
	\[
		\lim_{L\to\infty} \iiint_{-L/2}^{L/2}d^3q_1d^3q_2\  e^{-\beta\alpha\pfrac{|\vect r_1-\vect r_2|^2}{2}}
	 	= \pfrac{2\pi}{\beta\alpha}^{3/2}V
	\]
	The single particle partition function is then
	\[
		Z_1 = \frac{1}{h^6}V\pfrac{2m\pi}{\beta}^{3}\pfrac{2\pi}{\beta\alpha}^{3/2}
		= \frac{V}{h^6}\pfrac{(2\pi)^3m^2}{\alpha}^{3/2}\pfrac{1}{\beta}^{9/2}
	\]
	I am not sure on this, but it is my belief that we should divide by $2!$. The reasoning is that we should
	not be able to distinguish between a diatomic molecule that has been rotated by $\pi$ about the difference
	vector $|\vect r_1-\vect r_2|$. With $Z_1$ in hand, we raise it to the $N$ power for $N$ molecules
	and also divide by $N!$ since the diatomic molecule are indistinguishable. Hence we
	finally arrive at
	\[
		Z = \frac{1}{h^{6N}2^NN!}\pfrac{(2\pi)^3m^2}{\alpha}^{3N/2} V^N (kT)^{9N/2}.
	\]
	\\ \\
	
	% (b)
	\item
	We use the thermodynamic connection
	\[
		F = -kT\ln Z.
	\]	
	From this, we should be able to find pressure as a function of $T,V,N$
	\[
		p = -\plr{\pdiff[F]{V}}_T.
	\]
	Let's denote 
	\[
		\gamma^{3N/2} =\pfrac{(2\pi)^3m^2}{2^{2/3}h^4\alpha}^{3N/2}
	\]
	then
	\[
		Z = \frac{\gamma^{3N/2}}{N!}V^N(kT)^{9N/2}
	\]
	\[
		-kT\ln Z= -kT\blr{\frac{3N}{2}\ln\gamma - N\ln N +N+N\ln V +\frac{9N}{2}\ln(kT)}
	\]
	\[
		p = \frac{NkT}{V}.
	\]
	or
	\[
		f(p,T,V,N) = p-\frac{NkT}{V} = 0.
	\]
	\\ \\
	
	% (c)
	\item
	We may find the internal energy by 
	\[
		U = -\pdiff{\beta}\ln Z.
	\]
	\[
		U = -\pdiff{\beta}\plr{-\frac{9N}{2}\ln{\beta}} = \frac{9N}{2}kT
	\]
	then
	\[
		C_V = \frac{9N}{2}k.
	\]
	\\ \\
	\eenum 

% #2 ------------------------------------------------------------------------------------------------------------------------------------------------------
	\item
	Find the probability distribution $P(\omega_1,\omega_2,\omega_3)$ for the angular velocities $\omega_i$,
	$i=1,2,3$ in 3D rotation of polyatomic molecules in a classical ideal gas at the temperature $T$. The
	respective principal moments of inertia of this molecule are $I_1$, $I_2$, $I_3$. From the 
	distribution $P$, determine the mean squares of angular velocity $\braket{\vect\omega^2}$
	and angular momentum $\braket{\vect L^2}$ of a molecule.
	\\ \\ \\
	The probability distribution of a single particle, in this classical case, is just 
	\[
		\frac{e^{-\beta H}}{Z}
	\]
	used in the form
	\[
		\braket{A} = \frac{1}{h^3}\iiint d^3pd^3q\ \frac{e^{-\beta H(p,q)}}{Z}A(p,q).
	\]
	The hamiltonian is that due to the kinetic energy of rotation
	\[
		H = \frac{1}{2}I\omega^2 = \frac{L^2}{2I}.
	\]
	In 3D we have an inertia tensor with principle moments of inertia (all lying orthogonal to each other) and
	we can thus formulate the hamiltonian as
	\[
		H = \frac{L_1^2}{2I_1}+\frac{L_2^2}{2I_2}+\frac{L_3^2}{2I_3}.
	\]
	I would think we should be able to transform from a phase space in $p,q$ to $L,q$ so that
	\ba
		Z &= \frac{1}{h^3}\iiint_{-\infty}^{\infty} dL_1dL_2dL_3 \int d\Omega \ e^{-\beta H}\\
		& = \frac{4\pi}{h^3}\prod_{i=1}^3 \int_{-\infty}^{\infty} dL_1\ e^{-\beta \frac{L_1^2}{2I_1}}\\
		& = \frac{4\pi}{h^3}\prod_{i=1}^3 \sqrt{\frac{2 \pi I_i}{\beta}} \\
		& = \frac{4\pi}{h^3}\pfrac{2\pi}{\beta}^{3/2}\sqrt{I_1I_2I_3}.
	\ea
	To find the mean square of angular velocity, we note
	\[
		\omega^2 = \sum_i \pfrac{L_i}{I_i}^2.
	\]
	Compute the following
	\ba
		\braket{\omega_1^2} &= \frac{1}{h^3}\iiint dL_1dL_2dL_3\ \frac{e^{-\beta H} }{Z}\pfrac{L_1}{I_1}^2 \\
		&=  \frac{4\pi}{h^3}\pfrac{2\pi}{\beta}\sqrt{I_2I_3}\frac{1}{ZI_1^2}\int_{-\infty}^{\infty}dL_1\ 
		e^{-\beta\pfrac{L_1^2}{2I_1}}L_1^2
	\ea 
	I won't evaluate the rest, but the calculation is clear. \\ \\
	
	
% #3 ---------------------------------------------------------------------------------------------------------------------------------------------------
	\item
	Consider an ideal Fermi gas of particles with two interconverting species 1 and 2, such as
	two hyperfine states in an atom. Suppose it is possible to effect the conversion in such a way that
	an atom picks up an added energy $\Delta \ep$ when it moves from species 1 to species 2, and loses
	energy $\Delta \ep$ in the reverse transition. This could happen, say, if an off-resonant microwave field 
	is transferring atoms between the species. Write the total density of the gas $2\rho$, and denote
	the densities of the individual species by $\rho_{1,2} = \rho\mp \frac{1}{2}\Delta\rho$.
	
	\benum
	% (a)
	\item
	Given the global temperature and pressure, and the energy difference, the equilibrium condition for
	the two species is $\mu_1 +\Delta \epsilon = \mu_2$. Why?
	
	%(b)
	\item
	Show that for a small energy difference $\Delta \ep$ and at zero temperature, the density
	difference equals
	\[
		\Delta \rho = \frac{3\rho}{\ep_F}\Delta \ep
	\]
	where $\ep_F$ is the Fermi energy of a single species at the density $\rho$.
	\\ \\
	\eenum
	
	\benum
	% (a)
	\item
	Ultimately, this comes down to just arguing that the change in free energy as we move from one species to another
	must increase by $\Delta \ep$. Typically we try to make an equilibrium argument as 
	\[
		dG(p,T)_{eq} = 0
	\]
	\[
		dN_1 = -dN_2 \equiv dN
	\]
	\[
		dG = dG_1+dG_2 = dN(\mu_1-\mu_2) = 0.
	\]
	As we go from species $1\to 2$ the change in Gibbs energy is
	\[
		\Delta G_{1\to 2} = -\mu_1 +\mu_2+\Delta \ep
	\]
	and as we go from $2\to 1$
	\[
		\Delta G_{2\to 1}= \mu_1-\mu_2-\Delta \ep
	\]
	and if $\Delta G = 0$ for both cases it must be that
	\[
		\mu_1 = \mu_2+\Delta \ep.
	\]
	\\ \\
	
	% (b)
	\item
	From the last equation we are looking for $\rho_2-\rho_1$. The density in each case may is found by integrating up
	to the fully occupied fermi energy $\ep_F$
	\[
		\rho_1 = \frac{g}{6\pi^2}\pfrac{2m}{\h^2}^{3/2}\ep_F^{3/2}
	\]
	\[
		\rho_2 = \frac{g}{6\pi^2}\pfrac{2m}{\h^2}^{3/2}(\ep_F+\Delta\ep)^{3/2}
	\]
	The fermi energy for species two has the additional energy term. Now we simply take the difference 
	for $\Delta \ep$ small and denote $\rho_1 \approx \rho$. We expand the small energy term 
	using the binomial expansion. We end up with
	\[
		\Delta\rho = \frac{3\rho}{2\ep_F}\Delta\ep
	\]
	\\ \\
	\eenum
	
% #4 ---------------------------------------------------------------------------------------------------------------------------------------------------
	\item
	Collective elementary excitations called \emph{spin waves} or \emph{magnons} determine the low-T
	specific heat of a spin system that has undergone the ferromagnetic phase transition. For a small wave 
	vector $\vect k$ the dispersion relation of magnons is $\omega(k) \propto \vect k^2$. What is the temperature
	dependence of the heat capacity of magnons at low temperatures? \\
	Hint: As usual in three dimensions, at least at low energies the density of states as a function
	of the wave number $k =|\vect k|$ is proportional to $k^2$. 
	\\ \\
	It seems that excitations of a particular field are mediated by \emph{massless} bosons. This means that
	these bosonic excitations probably do not conserve particle number and hence we set the
	chemical potential to zero. In this event, the occupations are
	\[
		\braket{n_\vect k} = \frac{1}{e^{\beta \ep_k-1}}
	\]
	Let's define the dispersion relation as $\ep(k) = \alpha k^2$ then we can find the energy as
	\ba
		U & = \sum_\vect k \braket{n_\vect k}\ep(k) \to 4\pi\alpha\pfrac{L}{2\pi}^3
		 \int_0^\infty dk\ \frac{k^4}{e^{\alpha\beta k^2}-1}
	\ea
	In the low temperature approximation, $\beta \to \infty$ and so it seems justified that we should
	be able to integrate over the full range of wavevectors. We need to convert this to a dimensionless
	integral to determine the $T$ dependence
	\[
		U = \gamma/2 \pfrac{1}{\alpha\beta}^{5/2}  \int_0^\infty d(\alpha\beta k^2)\ \frac{\alpha\beta k^2\sqrt{\alpha
		\beta k^2}}{e^{\alpha\beta k^2}-1}
	\]
	and so we have
	\[
		U \propto T^{5/2}
	\]
	and
	\[
		C_V \propto T^{3/2}.
	\]
	
	\eenum
\end{document}