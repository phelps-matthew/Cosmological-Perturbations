\documentclass[10pt,letterpaper]{article}
\usepackage{macroshw}

\title{\begin{spacing}{1.2}Stat Mech Summer 2015\end{spacing}}
\author{Matthew Phelps}
\date{}

\begin{document}
\maketitle

\benum
% #1 --------------------------------------------------------------------------------------------------------------------------------------------------------
  	 \item
	 Molecular oxygen is a paramagnetic gas. Each molecule has a spin angular momentum $\vec S$ 
	 with $|\vect S|=1$. For a dilute gas (with N molecules in volume V) immersed in a magnetic field 
	 $\vect B = B\vecth z$, we can write the magnetic contribution to the Hamiltonian as
	 \[
	 	H_{mag} = -\gamma \sum_{i=1}^N (S_{iz}B)
	\]

	\benum
	% (a)
	\item
	If this (lattice) gas is at temperature $T$, calculate the probabilities for $S_{iz}$ to have values
	$+1$, $-1$, and $0$.
	
	% (b)
	\item
	Calculate (i) the magnetization $\vect m = \vect M/V$ (average magnetic moment per unit volume)
	and (ii) the entropy at temperature $T$ and $B\ne 0$	.
	% (c)
	\item
	For $B\ne 0$, find the limiting values of the entropy and magnetization as (i) $T\to 0$ and (ii) $T\to \infty$. 
	Give a physical explanation for each of the limiting values you obtain. \\ \\
	\eenum
	
	\benum
	% (a)
	\item
	Not sure what $|\vect S|=1$ means, because this suggests $\vect S^2 = 1$ in which 
	\[
		\vect S^2\ket{s,m_z} = s(s+1)\ket{s,m_z} = 1\ket{s,m_z}
	\]
	which means $s = -1/2 \pm \sqrt 5/2$. Anyway, lets assume $s=1$ (Bosons). As a lattice gas,
	each particle is distinguishable. The partition function factors as the partition function of a single particle,
	$\zeta$
	\[
		Z_N = \zeta^N.
	\]
	\[
		\zeta = \sum_i e^{-\beta E_i} = \sum_{S_z=\pm1,0} e^{\beta\gamma B S_z} = 1+2\cosh(\beta\gamma B)
	\]
	We are looking for the expectation value of spin for a single particle. Noting that $N$ is fixed, the canonical
	density operator is
	\[
		\rho = \frac{e^{-\beta H}}{Z} = \sum_i p_i \ket i\bra i
	\]
	\[
		\tr(\rho) = \sum_i p_i = \frac{1}{Z}\tr(e^{-\beta H}) = \sum_i \frac{e^{-\beta E_i}}{Z}
	\]	
	hence
	\[	
		p_i = \frac{e^{-\beta E_i}}{Z}
	\]
	where $Z = \zeta$ above and so
	\[
		p(s_z=1) = \frac{e^{\beta\gamma B}}{1+2\cosh(\beta\gamma B)}
	\]
	\[
		p(s_z=-1) = \frac{e^{-\beta\gamma B}}{1+2\cosh(\beta\gamma B)}
	\]
	\[
		p(s_z=0) = \frac{1}{1+2\cosh(\beta\gamma B)}
	\]
	\\ \\
	% (b)
	\item
	The magnetic moment is $\mu = \gamma S_z$. We seek to find the average magnetic moment
	\[
		\vect M = \braket{M_z} .
	\]
	Since the particles are independent and distinguishable, the expectation of total magnetic moment
	is a sum of expectation values of individual moments
	\[
		\braket{M_z} = N\braket{m_{iz}} = N\gamma\braket{s_{iz}}.
	\]
	This becomes
	\[
		\braket{M_z} = N \gamma \frac{\plr{-e^{-\beta \gamma B}+e^{\beta\gamma B}}}{\zeta}
	\]
	thus
	\be\label{1}
		\vect m = n\gamma  \pfrac{2\sinh(\beta\gamma B)}{1+2\cosh(\beta\gamma B)}
	\ee
	**As an aside, 
	\[
		\braket{M_z} = \tr(\rho M_z) = \frac{\gamma}{Z_N} \sum_{\{n\}} e^{-\beta E_n}\plr{\sum_i s_{iz}}
	\]
	Here we are summing over all possible sets of $\{ n\} = \{ n_0,n_1,..,n_n\}$ where $n_i = \{-1,0,1\}$. 
	The particles are distinguishable, so states with the same energy are treated distinctly. 
	Noting that $\braket{S_z} = -\frac{1}{\gamma B}
	\braket{E}$, we will simply take the expectation value of $\braket{E}$ instead and divide at the end.
	 Anyway, when we do this we have
	\ba
		\braket{E} &=  \frac{1}{Z_N} \sum_{\{n\}} E_ne^{-\beta E_{n}}  \\
		&= \frac{\plr{ 
		(NE_0)e^{-\beta NE_0}+((N-1)E_0+E_1)e^{(N-1)E_0+E_1} + ... }}{Z_N} \\
		& = \frac{N\braket{E_i}Z_N}{Z_N} = N\braket{E_i}
	\ea
	Basically, we can factor the partition function out of the summation and everything reduces 
	down to simply finding the one particle expectation value (with corresponding one particle partition function)
	as we initially expected.**
	\\ \\
	To calculate the entropy, we have the thermodynamics connection
	\[
		F  = -kT\ln Z
	\]
	and so
	\[
		S = -\pdiff[F]{T}.
	\]
	Actually, the Von Neumann entropy is the more fundamental connection (coming from $S = \tr(\rho\ln\rho)$), but
	$F$ is easy to work with. At any rate, we can show that
	\[
		S = k\ln Z_N + \frac{\braket{H}}{T}.
	\]
	The energy is 
	\[
		\braket{E} = N\braket{E_i} =  -B\braket{M_z}
	\]
	and the log of the partition function is
	\[
		\ln Z_N = N\ln \zeta = N\ln(1+2\cosh(\beta\gamma B))
	\]
	and thus
	\be\label{2}
		S = kN\ln(1+2\cosh(\beta\gamma B))-\frac{B\braket{M_z}}{T}.
	\ee
	\\ \\
	% (c)
	\item 
	For $T\to\infty$, $\beta \to 0$ and $T\to 0$, $\beta \to \infty$. For the magnetization \eqref 1
	\[
		\lim_{T\to 0} \vect m = n\gamma
	\]
	\[
		\lim_{T\to \infty} \vect m = 0 .
	\]
	At high enough temperatures, the magnetic spins are randomly orientated, averaging to a net spin of zero. At 
	low temperatures, we approach maximal magnetization with all spins pointing up (why up and not down, 
	symmetry breaking due to nonzero B?). 
	\\ \\
	For the entropy,
	\[
		\lim_{T\to 0} S = kN\ln(1+e^\infty)-NkB\gamma(\infty) \approx \infty - \infty = 0
	\]
	\[
		\lim_{T\to\infty} S = kN\ln(3)
	\]
	From the third law of thermodynamics, we must have $S\to0$ as $T\to 0$. As the temperature goes to infinity,
	we approach a maximum finite value of $kN\ln(3)$. This is the single particle entropy of N particles, 
	independent of the magnetic field. 
	\\ \\
	\eenum
	
% #2 ------------------------------------------------------------------------------------------------------------------------------------------------------
	\item
	Consider a classical ideal gas with fixed particle number $N$ such that $pV = NkT$ with a constant
	heat capacity with fixed volume $C_V = Nk\eta$, where $\eta$ can be considered to be constant (for
	the temperatures $T$ we are interested in). Use Maxwell relations etc. to find the heat capacity at constant
	pressure $C_p$. Then show that in that case the entropy can be written as
	\[
		S = Nk\ln{\frac{V}{N}} + f(T)+\text{const}.,
	\]
	where $f(T)$ is a function of temperature and "const." does not depend on either temperature of volume.
	What is $f(T)$? Show that for an adiabatic process, both $VT^\eta$ and $pV^\eta$ are constant. Here
	$\gamma = \frac{C_p}{C_V}$. 
	\\ \\ \\
	The heat capacity at constant pressure
	\[
		C_p = T\plr{\pdiff[S]{T}}_p
	\]
	suggests the form $S(T,p)$. However, we know from our equation of state that we may express this as
	$S(T,p(T,V))$. And so with the chain rule we may relate $C_p$ to $C_V$ by
	\[
		\plr{\pdiff[S]{T}}_V = \plr{\pdiff[S]{T}}_p+\plr{\pdiff[S]{p}}_T\plr{\pdiff[p]{T}}_V.
	\]
	From the Gibbs maxwell relation
	\[
		-\plr{\pdiff[S]{p}}_T = \plr{\pdiff[V]{T}}_p
	\]
	we then have
	\[
		\plr{\pdiff[S]{T}}_V = \plr{\pdiff[S]{T}}_p- \plr{\pdiff[V]{T}}_p\plr{\pdiff[p]{T}}_V.
	\]
	Thus
	\ba
		C_V &= C_p-T \plr{\pdiff[V]{T}}_p\plr{\pdiff[p]{T}}_V \\
		&= C_p - \frac{TNk}{p}\pfrac{Nk}{V} \\
		& = C_p - \frac{TN^2k^2}{pV} \\
		& = C_p - Nk.
	\ea
	So
	\ba
		C_p &= C_V + Nk \\
		& = Nk(1+\eta).
	\ea
	 \\
	For the entropy, we use its two partial derivatives: one from the heat capacity
	\[
		\plr{\pdiff[S]{T}}_V = \frac{C_V}{T}
	\]
	and one from the Helmholtz maxwell equation 
	\[
		\plr{\pdiff[S]{V}}_T = \plr{\pdiff[p]{T}}_V = \frac{Nk}{V}.
	\]
	Hence we have a path independent function
	\[
		dS = \plr{\pdiff[S]{T}}_V dT + \plr{\pdiff[S]{V}}_T dV
	\]
	and we integrate
	\ba
		S(T,V)  &= \int dT\ \frac{C_V}{T} + \int dV\ \frac{Nk}{V}\\ 
		& = C_V\ln T +Nk\ln V + C
	\ea
	where $C$ is a volume and temp. independent constant. We may use the constant to put this into the form
	\[
		S = Nk\plr{\ln\frac{V}{N}+\eta \ln T} + C'.
	\]
	Therefore $f(T)= C_V\ln T$. 
	\\ \\
	In an adiabatic process, $\delta Q = 0$. If we may assume the process is irreversible, then $\delta Q = TdS$
	and thus $dS = 0 \to S = \text{const}$. Then we may use the form of entropy we found earlier
	\[
		S = Nk\plr{\ln V-\eta\ln T} + C = K
	\]
	and so
	\[
		\ln(VT^\eta) = K
	\]
	or
	\[
		VT^\eta = K.
	\]
	For the other relation, we can use the eq. of state to get to
	\[
		p\eta V^{\eta+1} = K
	\]
	in which we can raise to the $\frac{1}{\eta}$ and use the ratio of heat capacities 
	\[
		\frac{C_p}{C_V} = \frac{\eta+1}{\eta}
	\]
	to arrive at
	\[
		pV^\gamma = K.
	\]
	However, if the process is not reversible, then we must find another way to relate $p$ to $V$ and $V$ to $T$. This
	can be accomplished through the energy (in an adiabatic process)
	\[
		dU = dW = -pdV.
	\]
	Firstly, in an ideal gas we can find the two derivatives for $dU$ and integrate to find the energy:
	\[
		\plr{\pdiff[U]{T}}_V = C_V
	\]
	\[
		\plr{\pdiff[U]{V}}_T = T\plr{\pdiff[S]{V}}_T-p.
	\]
	For the second partial, the Helmholtz maxwell eq. gives 
	\[
		\plr{\pdiff[S]{V}}_T = \plr{\pdiff[p]{T}}_V = \frac{Nk}{V}.
	\]
	Thus for $U(T,V)$ 
	\[
		dU = C_VdT-(0)dV
	\]
	or
	\[
		U = C_VT +K.
	\]
	Okay, now we return to the adiabatic process $dU = \delta W$ and form
	\[
		C_VdT = -pdV.
	\]
	From the eq. of state
	\[
		NkdT = Vdp+pdV.
	\]
	Thus we find the relation
	\[
		\eta(Vdp+pdV) = -pdV
	\]
	or
	\[
		-\frac{(1+\eta)}{\eta}\frac{dV}{V} = \frac{dp}{p}.
	\]
	Integrating and exponentiating
	\[
		V^{-\frac{(1+\eta)}{\eta}}p^{-1} = K
	\]
	or
	\[
		pV^\gamma = K.
	\]
	We proceed by the same steps as before to show that $VT^\eta = K$. 
	\\ \\
	
% #3 ---------------------------------------------------------------------------------------------------------------------------------------------------
	\item
	Consider an ideal gas of photons in 2-dimensions confined to an area $A$. The total energy is obtained by using 
	the harmonic oscilator quantization (without the zero-point energy) as $\sum_{\vect k} n_{\vect k}
	\h\omega_{\vect k}$ where $n_\vect k =0$, $1$, $2....$ (being the number of photons); $\omega_{\vect k}$ is
	the angular frequency of an oscillator associated with the radiation mode (polarization index has been 
	surpressed for clarity). The partition function $Z$ is given by
	\[
		\ln Z = -2\sum_{\vect k}\ln\blr{1-\exp\plr{-\beta\h\omega_{\vect k}}}.
	\]
	
	\benum
	% (a)
	\item
	Find the average occupation number $\braket{n_\vect k}$ and an integral expression in $\omega$
	for the internal energy $U(T)$ at temperature $T$. Hence obtain the Plank's distribution (or radiation law)
	of energy density $\rho(\omega, T)$ in 2-dimensions. How does $\rho(\omega,T)$ behave as
	$\omega \to 0$ and $\omega \to \infty$ at fixed $T$?
	\item
	% (b)
	Evaluate the temperature dependence of the specific heat and find the pressure of this photon gas in terms
	of $U$. (You may express your answers in terms of the integrals $C_n = \int_0^\infty dx\ \frac{x^n}{e^x-1}$.)
	\\ \\
	\eenum
	
	\benum
	% (a)
	\item
	To find the expectation of the occupation number (in each mode), first we use the partition function of 
	a single mode
	\ba
		Z_{\vect k} &= \sum_{n=0}^\infty e^{\beta \h\omega_{\vect k}n} \\
		& = \frac{1}{1-e^{\beta \h\omega_{\vect k}}}.
	\ea
	Then the probability for a single mode to contain $n$ particles is
	\[	
		P(n) = \frac{e^{-\beta \h\omega_{\vect k}n}}{Z_{\vect k}}.
	\]
	Each oscillator is independent and thus the total partition factorizes. Each mode can have 2 different 
	polarizations as well. Since each system is independent, average quantities of the entire system can 
	be found by summing the averages of the subsystems (modes in this case). Back to expectation 
	of particle number in a given mode, we have
	\ba
		\braket{n_\vect k} &= \sum_{n=0}^\infty nP(n) = \frac{1}{Z_\vect k}\sum_n ne^{-\beta \h\omega_{\vect k}n}
		\\
		& = -\frac{1}{Z_\vect k}\diff{x}\plr{ \sum_n e^{-nx}} \\
		& =  -\frac{1}{Z_\vect k}\diff{x} \pfrac{1}{1-e^{-x}} \\
		& = \frac{1}{Z_\vect k}\frac{e^{-x}}{(1-e^{-x})^2} \\
		& = \frac{e^{-x}}{1-e^{-x}} \\
		& = \frac{1}{e^{\beta \h\omega_{\vect{k}}}-1}
	\ea
	Looks just like the occupancy for Bosons with $\mu = 0$. However, we must recall that from the two 
	polarization modes we have
	\[
		\braket{n_\vect k} = \frac{2}{e^{\beta \h\omega_{\vect{k}}}-1}.
	\]
	* Alternatively, we could compute the expectation number for a system of two polarizations
	\[
		Z = Z_{\vect k}^2
	\]
	and sum over all possible combinations $\{n_i\}: \sum_i n_i =n$ for a given $n$ (which turns out to be $(n+1)$). 
	We arrive at the same answer with this combined polarization system.*
	\\ \\
	**Also much simpler to take $-\pdiff{\beta}\ln Z = U$, then proceed with integration.**
	\\ \\
	The internal energy can be found by summing the average thermal energy per mode
	\[
		\braket{\epsilon_\vect k} = \braket{n_\vect k}\h\omega_\vect k
	\]
	\[
		U = \sum_{\vect k} \braket{\epsilon_\vect k}.
	\]
	As for the vector $\vect k$ itself, it is what we typically expect from box normalization (in 2D)
	\[
		\vect k = \frac{2\pi}{L}\plr{n_1 \vecth x_1 + n_2 \vecth x_2}\qquad n_1,n_2 = 0,1,2,...
	\]
	Negative values are not given as they produce modes that are not independent? We integrate
	over $\vec k$ space to find the internal energy
	\[
		\sum_{\vect k} = \frac{1}{4}\int d^2n = \frac{\pi}{2} \int_0^\infty dn\ n.
	\]
	With $\omega_\vect k = c|\vect k| = \frac{2\pi c}{L}|\vect n|$ and $d\omega = \frac{2\pi c}{L} dn$
	\[
		2\h \pfrac{L}{2\pi c}^2\pfrac{\pi}{2} \int d\omega \frac{\omega^2}{e^{\beta\omega\h}-1} = 
		\frac{A\h}{4\pi c^2} \int_0^\infty d\omega \frac{\omega^2}{e^{\beta\omega\h}-1} 
	\]
	The energy density as a function of frequency $U/V = \int d\omega \rho(\omega,T)$ is 
	\[
		\rho(\omega,T) = \frac{\h}{4\pi c^2}\frac{\omega^2}{e^{\beta\omega\h}-1}.
	\]
	For fixed $\beta$, the energy density at zero oscillation is
	\[
		\lim_{\omega\to 0} \rho(\omega,T) = \frac{k^2 T^2}{2\pi \h}
	\]
	where L'Hopitals has been used twice. As the frequency becomes infinite
	\[
		\lim_{\omega\to\infty} \rho(\omega,T) = 0.
	\]
	\\ \\
	
	% (b)
	\item
	Using
	\[
		\pdiff{T} = -\frac{k}{\beta^2}\pdiff{\beta}
	\]
	we take the derivative of $U$ with respect to $T$ to find the heat capacity. First we need $U$ in a more
	appropriate form
	\ba
		U &= \frac{A\h}{4\pi c^2} \int_0^\infty d\omega \frac{\omega^2}{e^{\beta\omega\h}-1}  \\
		&= \frac{A\h}{4\pi c^2}\pfrac{1}{\beta\h}^3 \int_0^\infty dx\ \frac{x^2}{e^x-1} \\
		& = \frac{(kT)^3A}{4\pi (\h c)^2} C_2 \\
		& = AT^3\gamma
	\ea
	Now we may easily take the derivative with respect to $T$:
	\[
		C_V = 3AT^2\gamma
	\]
	Not sure how to express this in terms of $C_n$. \\ \\
	We may find the pressure as
	\[
		F = -kT\ln Z
	\]
	\[
		-\plr{\pdiff[F]{V}}_T = p.
	\]
	It may be slightly more simply to just calculate $F$ first
	\[
		F = 2kT\sum_\vect k \ln(1-e^{-\beta\omega_\vect k\h})
	\]
	which upon continuation becomes
	\[
		F = \frac{kTL^2}{4\pi c^2}\int_0^\infty d\omega\ \omega\ln(1-e^{-\beta\omega\h}).
	\]
	Converting to dimensionless quantity $x = \beta \omega \h$
	\[
		F= \frac{kTL^2}{4\pi c^2}\pfrac{1}{\beta\h}^2 \int_0^\infty dx\ x \ln(1-e^{-x}).
	\]
	Under integration by parts, the surface term vanishes and we are left with
	\[
		F= -\frac{k^3T^3L^2}{8\pi \h ^2c^2} \int_0^\infty dx \frac{x^2}{e^x-1} = -\frac{k^3T^3L^2}{8\pi \h ^2c^2}C_2
		= -\frac{1}{2}AT^3\gamma = -\frac{1}{2}U.
	\]
	Now we use the relation for pressure
	\[
		p = \frac{T^3}{2}\gamma = \frac{U}{2A}.
	\]
	\\ \\ \\
	
	\eenum 
% #4 ---------------------------------------------------------------------------------------------------------------------------------------------------
	\item
	A ferromagnetic XY model consists of unit classical spins, $\vect S_i = (S_i^x,S_i^y,S_i^z) =
	(\cos\phi_i,\sin\phi_i,0)$ such that $|\vect S_i|=1$ on a three dimensional cude lattice with $i$ labeling
	the site. The spins can point in any direction in the $xy$ plane. The Hamiltonian with nearest neighbor 
	interactions is given by
	\[
		H = -\frac{1}{2} J\sum_{i,l}\vect S_i\cdot \vect S_{i+1} - \vect h\cdot \sum_i \vect S_i,
	\]
	where $\vect h$ is a field pointing the in the $x$-direction (i.e. $\vect h = (h,0,0)$) and with $i$ running
	over all sites and $l$ running over the six nearest neighbors.
	
	
	\benum
	% (a)
	\item
	For the noninteracting case $J=0$, calculate the susceptibility $\chi = \plr{\pdiff[m_{ix}]{h}}_{h=0}$ per
	spin showing that the average $\vect m_i = \braket{\vect S_i}$ reduces to a component $m_{ix}$ along
	the $x$-axis. 
	\\
	\emph{Hint: While it is hard to evaluate the integral expression for $\vect m(\vect h)$, it should
	be easy to evaluate its derivative at $h=0$.} \\
	(In order to not get confused with the nomenclature, rename $\vect m$ for this $J=0$ case as $\vect m_0$,
	and $\chi$ as $\chi_0$ before part (b).)
	
	% (b)
	\item
	Use your result in (a) to calculate the (critical) transition temperature in the ferromagnetic case using the 
	standard "mean field theory" for $h=0$ and $J\ne 0$. In this regard, find $H_i$ such that (in mean field theory)
	\[
		H = \sum_i H_i = -\vect h'(\vect m_0)\cdot \sum_i \vect S_i + const.
	\]
	What is $\vect h'(\vect m)$? 
	\\
	Hint: In order to find the critical temperature, repeat the procedure from part (a) to find $\chi$, but 
	self-consistently replace $\vect h'(\vect m_0)$ by $\vect h'(\vect m)$. The find $\chi = \plr{
	\pdiff[m_{ix}]{h}}_{h=0}$ and calculate the critical temperature $T_C$ where $\chi \to \infty$. 
	\\ \\ 
	\eenum
	
	\benum
	% (a)
	\item
	To find the magnetization $\braket{\vect S_i}$ at $J=0$, we take the usual Maxwell Boltzmann distribution
	function. However, since the microstates are continuous, we have a probability density function
	\[
		\rho(\vect S_i) = e^{\beta h\cos\phi_i}/Z
	\]
	where 
	\[
		Z = \int_{-\pi}^\pi d\phi\ e^{\beta h\cos\phi}.
	\]
	Thus to find the expectation value we integrate
	\[
		\braket{\vect S_i} = \int_{-\pi}^\pi \frac{e^{\beta h\cos\phi_i}}{Z}(\cos\phi \vecth x_1 +\sin\phi \vecth x_2).
	\]
	The projection of the average magnetization along the $\vecth x_2$ direction is zero from 
	integrating an odd overall function over even bounds. Hence
	\[
		\braket{\vect S_i} = \braket{S^x_i} = \int_{-\pi}^\pi d\phi\ \frac{e^{\beta h\cos\phi}}Z \cos\phi.
	\]
	Now we find the susceptibility. Denote the integral without $Z$ as $A$. Then
	\[
		\chi = AZ^{-1}
	\]
	\[
		\plr{\pdiff{h}\braket{S_i^x}}_{h=0} = \pfrac{A'Z-AZ'}{Z^2}_{h=0} 
	\]
	\[
		\elr{A}_{h=0} = \int_{-\pi}^\pi d\phi \ \cos\phi = 0
	\]
	\[
		\elr{Z}_{h=0} = 2\pi
	\]
	\[
		\chi = \pfrac{A'}{2\pi}_{h=0}
	\]
	\[
		\elr{A'}_{h=0} = \beta \int_{-\pi}^\pi d\phi \ \cos^2(\phi) = \frac{\beta}{2} \int_{-\pi}^\pi d\phi \ (1+\cos(2\phi))
		= \beta\pi.
	\]
	All together then
	\[
		\chi_0 = \frac{\beta}{2};\qquad \vect m_0 = \int_{-\pi}^\pi d\phi\ \frac{e^{\beta h\cos\phi}}Z \cos\phi.
	\]
	\\ \\
	
	% (b)
	\item
	Applying the mean field theory to only the interaction term (keeping in mind that all variables are vectors)
	\[
		\delta s_i = s_i-m;\qquad s_i = \delta s_i+m
	\]
	we have
	\ba
		\sum_{i,l} 	s_is_{i+l} &= \sum_{i,l} (\delta s_i+m)(\delta s_{i+l}+m) \\
		& \approx \sum_{i,l} m^2 +m(s_i-m)+m(s_{i+l}-m) \\
		& = \sum_{i,l} -m^2 + m(s_i+s_{i+l})
	\ea
	This summation is over the $q=6$ nearest neighbors. Hence, there are $qN$ nearest neighbor pairs
	in the lattice, but a single bond is counted twice for each pair and so we must divide the result by two
	\[
		 \sum_{i,l} -m^2+2ms_i. = -\frac{m^2Nq}{2} +qm \sum_i s_i.
	\]
	Now adding back the rest of the noninteracting factors
	\[
		H_{int}^{MF} = -Jq\vect m \cdot \sum_i \vect s_i+\pfrac{JNq}{2}\vect m^2.
	\]
	Hence 
	\[
		\vect h'(\vect m_0) = 6J\vect m.
	\]
	I'm assuming $\vect m_0$ is at $h=0$ in this case? Very confusing.
	\\ \\
	In the mean field theory, the total hamiltonian is
	\[
		H^{MF} = -(6J\vect m+\vect h)\cdot  \sum_i \vect s_i + \pfrac{JNq}{2}\vect m^2
	\]
	The constant term in the Hamiltonian should be able to be ignored. This leaves us with the same 
	Hamiltonian as in part (a), but with an effective applied field that includes the effect of magnetization.
	I am not sure why, but I am going to proceed by inputting the effective field into the magnetization. I assume
	that the magnetization is the same as in part (a) in the sense that it lies only along the $x$-direction and
	tends to zero as $h\to 0$. 
	\\ \\
	In this regard, we have the magnetization
	\[
		m_x = \frac{\int_{-\pi}^{\pi} d\phi\ e^{\beta(h+6Jm_x)\cos\phi}\cos\phi}{
			\int_{-\pi}^{\pi} d\phi\ e^{\beta(h+6Jm_x)\cos\phi}}
	\]
	Denoting the numerator as $A$ and denominator as $Z$ (the partition funciton), the susceptibility is then
	\[
		\elr{\pdiff[m_x]{h}}_{h=0}=\chi = \frac{A'Z-AZ'}{Z^2}
	\]
	\[
		\elr{A}_{h=0} = \int_{-\pi}^\pi d\phi\ \cos\phi = 0
	\]
	\[
		\elr{Z}_{h=0} = 2\pi
	\]
	 \ba
	 	\elr{A'}_{h=0} &= \elr{\int_{-\pi}^\pi d\phi\ e^{\beta(h+6Jm_x)\cos\phi}\cos^2(\phi)
		\beta(1+6J\pdiff[m_x]{h})}_{h=0} \\
		 &=  \beta(1+6J\chi) \int_{-\pi}^\pi d\phi\ \cos^2(\phi) \\
		 & = \beta\pi(1+6J\chi).
	\ea
	Thus we have
	\[
		\chi = \frac{\beta(1+6J\chi)}{4\pi}
	\]
	or
	\[
		\chi = \frac{1}{4\pi/\beta-6J}.
	\]
	For $\chi\to\infty$
	\[
		T_C = \frac{6J}{4\pi k}.
	\]
	\eenum
	\eenum
\end{document}