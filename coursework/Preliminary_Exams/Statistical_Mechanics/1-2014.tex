\documentclass[10pt,letterpaper]{article}
\usepackage{macroshw}

\title{\begin{spacing}{1.2}Stat Mech Winter 2014\end{spacing}}
\author{Matthew Phelps}
\date{}

\begin{document}
\maketitle

\benum
% #1 --------------------------------------------------------------------------------------------------------------------------------------------------------
  	 \item
	Consider a classical ideal gas of $N$ particles in three dimensions confined to an external
	potential $V(r) = K(r/r_0)^\alpha$, where $K>0$, $\alpha >0$, and $r_0>0$ are constants.
	\benum
	% (a)
	\item
	Show that the heat capacity is $C = \plr{\frac{3}{2}+\frac{3}{2}}Nk$.
	% (b)
	\item
	Suppose the shape of the parameter of the potential $\alpha$ may be varied. Why is it that
	the standard heat capacity of the free ideal gas $C_V = \frac{3}{2}Nk$ occurs in the limit
	$\alpha \to \infty$?
	\\ \\
	\eenum
	
	\benum
	%(a)
	\item
	The easiest way to find the heat capacity in this case is to use the overall method of
	\[
		C_V = k\beta^2\plr{\pdifff{}{*2\beta}}_V\ln Z
	\]
	Hence to find $Z$
	\ba
		Z_1 &= \frac{4\pi}{h^3}\int_{\infty}^{\infty}d^3p\int_0^\infty dr\ r^2
		e^{-\beta\pfrac{p^2}{2m}}e^{-\beta K\pfrac{r}{r_0}^\alpha} \\
		& = \frac{4\pi}{h^3}\pfrac{2\pi m}{\beta}^{3/2} \int_0^\infty dr\ r^2e^{-\beta K\pfrac{r}{r_0}^\alpha}
	\ea
	To extract the temperature dependence, we must convert this to a dimensionless integral
	using $x = (\beta K)^{1/\alpha}\frac{r}{r_0}$
	\ba
		\int_0^\infty dr\ r^2e^{-\beta K\pfrac{r}{r_0}^\alpha} &= 
		(\beta K)^{-3/\alpha}r_0^3 \int_0^\infty d\pfrac{(\beta K)^{1/\alpha}r}{r_0}\pfrac{(\beta K)^{1/\alpha }r}{r_0}^2
		e^{\pfrac{(\beta K)^{1/\alpha }r}{r_0}^\alpha}.
	\ea
	If we denote the dimensionless integral by $I$, then we have
	\[
		Z_1 = \frac{4\pi}{h^3}(2\pi m)^{3/2}\beta^{-(3/2+3/\alpha)}K^{-3/\alpha}r_0^3
	\]
	and so
	\[
		\ln Z = N\ln Z_1 = N\ln\gamma -N\plr{\frac{3}{2}+\frac{3}{\alpha}} \ln (\beta)+\ln r_0^3
	\]
	\[
		k\beta^2\pdifff{}{2*\beta} \ln Z = k\beta^2\plr{\frac{3}{2}+\frac{3}{\alpha}}\frac{1}{\beta^2}
		= \plr{\frac{3}{2}+\frac{3}{\alpha}}Nk.
	\]
	\\ \\
	%(b)
	\item
	If we look at the potential
	\[
		\lim_{\alpha\to\infty} V(r) = \begin{cases} 0& r<r_0 \\ \infty & r>r_0 \end{cases}
	\]
	and so we end up confined in a bound box of dimension $r_0^3$. Hence we recover the usual heat
	capacity.
	\\ 
	\eenum

% #2 ------------------------------------------------------------------------------------------------------------------------------------------------------
	\item
	\benum
	% (a)
	\item
	For the two dimensional non-relativistic Bose gas with zero spin, calculate the chemical potential
	as a function of temperature and (area) density.
	% (b)
	\item
	Do we have critical density and do we need to add Bose-Einstein condensate at low temperatures, as
	in the case of the three dimensional Bose gas?
	\\ \\
	\eenum
	\benum
	% (a)
	\item
	Dealing with Bosons, we are used to knowing the grand partition function and occupancy. A 
	way to extract the chemical potential is to calculate $N$ for this 2D gas. This can be done by
	integration. Remember that
	\[
		\braket{n_\vect k}_B = \frac{1}{e^{\beta(\ep_k-\mu)}-1}
	\]
	so in the limit of continuation
	\[
		N = \frac{L^2}{2\pi} \int_0^\infty dk\ \frac{k}{e^{\beta(\ep_k-\mu)}-1}
	\]
	Since it is nonrelativistic, the dispersion relation is
	\[
		\ep(\vect k) = \frac{\h^2 |\vect k|^2}{2m}
	\]
	and
	\[
		 d\ep = dk  \frac{\h^2}{m} k
	\]
	\[
		k = \sqrt{\frac{2m \ep}{\h^2}}
	\]
	so
	\[
		k\ dk = \frac{m}{\h^2} d\ep.
	\]
	While we are here, we might as well generalize this for $D$ dimension for part (b). The density relations are
	\[
		k^{D-1} dk = k^{D-2}\ \frac{m}{\h^2} d\ep = \pfrac{2m\ep}{\h^2}^{(D-2)/2}\frac{m}{h^2}\ d\ep
		= \frac{1}{2}\pfrac{2m}{\h^2}^{D/2}\ep^{(D-2)/2}\ d\ep.
	\]
	Back to the 2D integral
	\ba
		\int_0^\infty dk\ \frac{k}{e^{\beta(\ep_k-\mu)}-1} &= \frac{m}{\h^2} 
		\int_0^\infty d\ep\ \frac{1}{e^{\beta(\ep-\mu)}-1}\\
		& = \frac{m}{\h^2\beta} \int_{-\beta \mu}^{\infty} d(\beta(\ep-\mu)) \frac{1}{e^{\beta(\ep-\mu)}-1}
		\\ 
		& =  \frac{m}{\h^2\beta} \int_{-\beta \mu}^{\infty} d(\beta(\ep-\mu)) \frac{e^{-\beta(\ep-\mu)}}
		{1-e^{-\beta(\ep-\mu)}} \\
		& = \frac{m}{\h^2\beta} \int_{-\beta \mu}^{\infty} d(\beta(\ep-\mu))\ \pdiff{(\beta(\ep-\mu))} 
		\ln(1-e^{-\beta(\ep-\mu)}) \\
		& =  \frac{m}{\h^2\beta} \elr{\ln(1-e^{-\beta(\ep-\mu)})}_{-\beta\mu}^\infty \\
		& = -\frac{m}{\h^2\beta} \ln\plr{1-e^{\beta\mu}} 
	\ea
	Gathering the rest of the constants we have
	\[
		N = -\frac{L^2}{2\pi} \pfrac{m}{\h^2\beta} \ln(1-e^{\beta\mu})
	\]
	solving for $\mu$
	\[
		\mu =kT\ln\plr{1-  e^{-\frac{2\pi \h^2\beta n}{m}}}
	\]
	This suggests that $-\infty < \mu < 0$, which, upon further reading, is not crazy. This means the system actually 
	loses free energy as another particle input i.e., you have to do work to extract a particle out of the system. This
	is characteristic to bosons. With fermions, it takes work to input a particle. 
	\\ \\
	% (b)
	\item 
	For a fixed temperature, the critical density is defined as the density at which 
	\[
		N_E(T,\mu=\ep_0) = N_C
	\]
	In this notation, it is the density of excited states (given by the integral) at $\mu = \ep_0$ where
	$\ep_0 = 0$ in our case. It is the maximum density of the excited states. However, as we let $\mu \to \ep_0$
	the ground state
	\[
		\braket{n_0} = \frac{1}{e^{\beta(\ep_0-\mu)}-1}
	\]
	becomes macroscopically large. If we assume that the chemical potential calculated in part (a) is 
	the density of the excited states, then the critical density is given when $\mu = 0$ which leads to
	\[
		\ln\plr{1-\exp\plr{-\frac{2\pi\h^2 n}{kT}}} = 0.
	\]
	For a fixed temperature, this happens when
	\[
		n\to\infty.
	\]
	Hence we never approach a finite critical density and no condensate is formed. At low temperatures,
	the excited density still does not converge (along with the ground state occupancy). Since both
	quantities are always divergent as we let $\mu \to 0$, there is no need to separate the contributions.
	Additionally, the excited state integral naturally includes the ground state and together they 
	are not bounded by a particular density. 
	\\ \\
	\eenum
			
% #3 ---------------------------------------------------------------------------------------------------------------------------------------------------
	\item
	For a non-realtivistic ideal degenerate ($T=0$) Fermi gas, find the average relative velocity
	$|\vect v_1-\vect v_2|$ of the two particles.
	\\ \\
	At low temperature, states are occupied up to the fermi energy $\ep_F$. 
	\\ \\
	
% #4 ---------------------------------------------------------------------------------------------------------------------------------------------------
	\item
	The dynamics of the vibrational normal modes of a solid made of $N$ atoms can be approximated in 
	terms of uncoupled harmonic oscillators by the Hamiltonian 
	\[
		H = \sum_{j=1}^{3N} \frac{p_j^2}{2m} +\frac{1}{2}m\omega_j^2x_j^2.
	\]
	\benum
	% (a)
	\item
	Calculate the canonical partition function $Z(T,N)$ of the system, determine its
	internal energy $U$, and show that it can be written as 
	\[
		U(T,N) = \int_0^\infty \frac{1}{2}\h\omega \coth\plr{\frac{1}{2}\beta\h\omega}\sigma(\omega)
		d\omega
	\]
	with the normal-mode vibrational frequency distribution $\sigma(\omega) = \sum_{j=1}^{3N}
	\delta(\omega-\omega_j)$.
	% (b)
	\item
	Consider the Einstein model of a solid where $\sigma(\omega) =\sum_{j=1}^{3N}
	\delta(\omega-\omega_j)$ with $\omega_j =\omega_E \ \forall\  j$. Derive the expression for 
	the heat capacity $C$. Show that $C$ satisfies the Dulong-Petit law for $T\gg T_E$, and 
	vanishes exponentially for $T\ll T_E$ where $T_E = \h\omega_E/k_B$.
	% (c)
	\item
	Consider the Debye model which assumes $\sigma(\omega) = 9N\omega^2/\omega_D^2$ if
	$\omega \le \omega_D$ and zero otherwise, where the value of $\omega_D$ is fixed
	by the normalization condition $\int_0^\infty \sigma(\omega)\ d\omega = 3N$. Derive the expression
	for the heat capacity $C$. Show that $C$ satisfies the Dulong-Petit law for $T\gg T_D$, and
	vanishes like $C=const\times T^n$ for $T\ll T_D$ where $T_D = \h\omega_D/k_B$.
	\\
	Remark: your calculation should determine the power $n$.
	\\
	Hint: you may encounter an integral of the type $\int_0^\infty dx\ x^4 e^x/(e^x-1)^2 = \frac{4\pi^4}{15}$.
	% (d)
	\item
	What does the third law of thermodynamics imply for the heat capacity? Do the models of Einstein and Debye
	discussed in part (b) and (c) satisfy the third law of thermodynamics?
	\\ \\
	\eenum
	\eenum 
\end{document}