\documentclass[11pt,letterpaper]{article}
\usepackage{macroshw}


\title{Classical Mechanics Preliminary Examination Problems}
\author{Matthew Phelps}
\date{Last Updated: \today}

\begin{document}
\maketitle

% 2015 Winter------------------------------------------------------------------------------------------------------------------------------------------------------------------------
\subsection*{2015 Winter}
\phantom{}
	\benum
	% 1
		\item
		A system in uniform gravity, consisting of two masses $m$, each carrying a charge $q$, is arranged as in the figure, that is, the arms of 
		length $l$ with the masses can move in $\theta$ and $\phi$ directions independently. (The interaction between the charges is
		$V_{12} = \frac{q^2}{r_{12}}$, where $r_{12}$ is the distance between the masses.) Find the Lagrangian and the equilibrium
		(the assumption that $\phi_{1,0} -\phi_{2,0} =\pi$ and $\theta_{1,0} = \theta_{2,0} \equiv \theta$ can be made from symmetry without 
		proof). Show that the equilibrium condition is
		\\
		\\
		\\
		\[
			\delta S = \delta \int_{t_1}^{t_2} \mathcal{L}(q_i,\dot{q_i})\, dt =0 
		\]
		\\
		\\
		\\
		\[
			\frac{4mgl^2}{q^2}\tan^3\theta = 1+\tan^2\theta.
		\]
		(You do not need to solve this for $\theta$ explicitly.) Then show that for the potential we have for the deviations from the equilibrium 
		position $\{\delta\theta_1,\delta\theta_2,\delta\phi_1,\delta\phi_2\} \equiv \{ \eta_i\}$
		\[
			\elr{\pdifff{V}{\eta_i,\eta_j}}_\text{equilibrium} = \bpm u+v&v&0&0\\v&u+v&0&0\\0&0&w&-w\\0&0&-w&w\epm.
		\]
		What are the constants $u$, $v$, and $w$?
		\\
		Then show the eigenmodes (i.e. frequencies and eigenvectors) for small deviations from the equilibrium.

		\figg{15W1.png}
		
		You can use (any or none of)
		\ba
			\sin(x+\delta x) &\approx \sin x+\delta x\cos x-\frac{1}{2}(\delta x)^2\sin x \\
			\cos(x+\delta x) &\approx \cos x-\delta x\sin x-\frac{1}{2}(\delta x)^2\cos x \\
			\tan(x+\delta x) &\approx \tan x+\delta x(\tan^2 x+1)+(\delta x)^2(\tan^3(x)+\tan x) \\
			\frac{1}{r_{12}} &\approx \frac{1}{2l\sin\theta}\plr{1+\mathcal O(\delta\theta_i,\delta\phi_i)+\frac{1}{8}\plr{\plr{\frac{2}{\sin^2\theta}
			-1}(\delta\theta_1+\delta\theta_2)^2+(\delta\phi_1-\delta\phi_2)^2}}.
		\ea
		\\ \\
		From cartesian to polar
		\ba
			T &= \frac{p^2}{2m}\\ 
			&= \frac{1}{2}m(x_1^2+x_2^2+x_3^2)\\
			& = \frac{1}{2}m(\dot r^2+r^2\dot\theta^2+r^2\sin^2\theta\dot\phi^2)
		\ea
		Therefore our Lagrangian for the system takes the form
		\be\label{1}
			\mathcal{L} = \frac{1}{2}ml^2(\dot\theta_1^2+\dot\theta_2^2+\sin^2\theta_1\dot\phi_1^2+\sin^2\theta_2\dot\phi_2^2)
			+mgl\cos\theta_1+mgl\cos\theta_2-\frac{q^2}{r_{12}}.
		\ee
		A system is defined to be in equilibrium when all the generalized forces vanish. In our Lagrangian formulation then, this amounts to
		\[
			Q_j \equiv -\pdiff[V]{q_j} = 0.
		\]
		We can use this condition to prove that $\theta_1 = \theta_2 \equiv \theta$ and $\phi_1-\phi_2 = \pi$. 
		At equilibrium, the deviations from equilibrium are of course zero
		\[
			\delta\theta_1,\delta\theta_2,\delta\phi_1,\delta\phi_2=0
		\]
		so that the potential may be written as (using $r_{12}$ defined below)
		\[
			V = -2mgl\cos\theta+\frac{q^2}{2l\sin\theta}.
		\]	
		Now imposing the equilibrium condition
		\[
			-\pdiff[V]{\theta} = 0
		\]
		\[
			2mgl\sin\theta =\frac{q^2}{2l}\frac{\cos\theta}{\sin^2\theta}\to \frac{4mgl^2}{q^2}\tan^3\theta = 1+\tan^2\theta
		\]
		To find the matrix elements, we first take the potential term from the Lagrangian 
		\[
			V = -mgl\cos\theta_1-mgl\cos\theta_2+\frac{q^2}{r_{12}}
		\]
		where $r_{12}$ is given already as
		\[
			\frac{1}{r_{12}} \approx \frac{1}{2l\sin\theta}\plr{1+\mathcal O(\delta\theta_i,\delta\phi_i)+\frac{1}{8}\plr{\plr{\frac{2}{\sin^2\theta
			}-1}(\delta\theta_1+\delta\theta_2)^2+(\delta\phi_1-\delta\phi_2)^2}}.
		\]
		Note that $\theta$ here is the equilibrium angle. Also note that the change of variable from $q_i$ to $\eta_i$ does not affect the 
		derivative, i.e. $\pdiff[\eta_i]{q_i} = 1$. For the first two terms in the potential, the only non-vanishing elements are 
		$V_{11}$ and $V_{22}$
		\[
			\elr{\pdifff{V}{*2\theta_1}}_\theta = \elr{\pdifff{V}{*2\theta_2}}_\theta= mgl\cos\theta.
		\]
		Therefore, 
		\[
			u = mgl\cos\theta.
		\]
		We can carry on calculating the remaining part of $V_{11}$ to find $v$. 
		\ba
			V_{11} &= mgl\cos\theta +\frac{q^2}{2l\sin\theta}\pfrac{2}{8}\plr{\frac{2}{\sin^2\theta}-1}\\
			& = mgl\cos\theta+\frac{q^2}{8l\sin\theta}\plr{\frac{2}{\sin^2\theta}-1}.
		\ea
		Thus,
		\[
			v = \frac{q^2}{8l\sin\theta}\plr{\frac{2}{\sin^2\theta}-1}.
		\]
		To find $w$ note that we will end up with the same term as for $v$. Thus altogether we have
		\[
			u = mgl\cos\theta;\quad v=w=\frac{q^2}{8l\sin\theta}\plr{\frac{2}{\sin^2\theta}-1}.
		\]
		\\
		\\
		We have essentially computed a first approximation to $V$ as
		\[
			V = \frac{1}{2}V_{ij}\eta_i\eta_j.
		\]
		It will be useful to similarly expand the potential portion of the Lagrangian. We can express this as
		\[
			T = \frac{1}{2}m_{ij}\dot q_i\dot q_j = \frac{1}{2}m_{ij}\dot\eta_i\dot\eta_j.
		\]
		In general the $m_{ij}$ are dependent on position (like $\sin\theta_1$ in this problem), but if we expand $T$ as a taylor series
		and take the lowest non-vanishing term we have
		\[
			m_{ij} = m_{ij}(q_{01},q_{02},..).
		\]
		Thus the $m_{ij}$ is a matrix of coefficients evaluated at equilibrium denoted by $T_{ij}$. We then have
		\[
			T = \frac{1}{2}T_{ij}\dot\eta_i\dot\eta_j
		\]
		where
		\[
			T_{ij} = ml^2\bpm 1&0&0&0\\ 0&1&0&0\\0&0&\sin^2\theta&0\\0&0&0&\sin^2\theta \epm.
		\]
		In these forms the Lagrangian may be expressed as
		\[
			\mathcal L = \frac{1}{2}(T_{ij}\dot\eta_i\dot\eta_j-V_{ij}\eta_i\eta_j).
		\]
		Guessing an exponential solution $\eta_i = a_{i}e^{-i\omega t}$ leads to the determinant equation
		\[
			(V-\omega^2 T)\vect a = 0
		\]
		which have non-zero solutions only if the determinant vanishes
		\[
			|V-\omega^2 T| =0.
		\]
		As a matrix we have
		\[
			\bvm  (u+v)-\omega^2&v&0&0\\v&(u+v)-\omega^2&0&0\\0&0&v-\omega^2\sin^2\theta&-v\\0&0&-v&v
			-\omega^2\sin^2\theta \evm.
		\]
		Breaking up into smaller determinants we find
		\ba
			 &[(u+v)-\omega^2]^2[(v-\omega^2\sin^2\theta)^2-v^2]-v^2[(v-\omega^2\sin^2\theta)^2-v^2]\\
			& =\{ [(u+v)-\omega^2]^2-v^2\}[(v-\omega^2\sin^2\theta)^2-v^2].
		\ea
		Honestly, this is about as good as it gets in factored form. From here we basically have to use some inspection to find the form of 
		$\omega$. From this $4\times 4$ determinant, there are four possible solutions for $\omega^2$:
		\ba
			\omega^2 & = 0\\
				& = u\\
				& = u+2v\\
				& = \frac{2v}{\sin^2\theta}.
		\ea
		Now for the eigenvectors
		\[
			(V-\omega_i^2T)\vect a_i = 0.
		\]
		As a homogeneous system, we will have an infinite number of solutions. Therefore, we can solve for $n-1 = 3$ variables in terms of 
		one component, or we may impose some sort of normalization. This normalization is a bit tricky, admittedly. The easiest scenario is 
		that for nondegenerate roots, which is what we have here. For the $k$th eigenvalue $\omega_k^2 = \lambda_k$ we first assume 
		(this can be shown Goldstein p. 242) that $\vect a_k$ and $\lambda_k$ are real. As such, $\vect a_k^\dag = \vect a_k^T$. Our 
		eigenvalue equation is then
		\[
			\vect V\vect a_k = \lambda_k\vect T\vect a_k.
		\]
		We next form a new transposed equation of the eigenvalue equation and continue to manipulate.
		\[
			\vect a_l^T\vect V = \lambda_l\vect a_l^T\vect T
		\]
		\[
			\vect a_l^T\vect V\vect a_k =\lambda_l\vect a_l^T\vect T\vect a_k
		\]
		\[
			\lambda_k \vect a_l^T\vect T\vect a_k = \lambda_l\vect a_l^T\vect T\vect a_k
		\]
		\[
			(\lambda_k-\lambda_l)\vect a_l^T\vect T\vect a_k = 0.
		\]
		These series of steps have been achieved by the reality and symmetric properties of $\vect V$ and $\vect T$. Now we can see
		that for distinct eigenvalues, we must have
		\be\label{2}
			\vect a_l^T\vect T\vect a_k = 0. 
		\ee
		Now at this point we choose to impose the normalization condition such that
		\be\label{3}
			\vect a_k^T\vect T\vect a_k = 1.
		\ee
		If we combine both the above equations, we simply have
		\[
			\vect A^T \vect T\vect A = 1
		\]
		where $\vect A$ is a matrix composed of the column vectors $\vect a_k$. Point is, our normalization condition that fixes
		our eigenvector components is given by \eqref 3
		\[
			a_{1k}^2+a_{2k}^2+\sin^2\theta(a_{3k}^2+a_{4k}^2) = 1.
		\]
		Proceeding to the eigenvectors themselves, we solve the following equations
		\[
			((u+v)-\omega_k^2)a_{1k}+va_{2k}=0
		\]
		\[
			va_{1k}+((u+v)-\omega_k^2)a_{2k}=0
		\]
		\[
			(v-\omega_k^2\sin^2\theta)a_{3k}-va_{4k} = 0
		\]
		\[
			-va_{3k}+(v-\omega_k^2\sin^2\theta)a_{4k} = 0.
		\]
		Starting with the first frequency $\omega_1 = 0$ we find that
		\[
			a_{11} = a_{21}=0,\quad a_{31} = a_{41} = \frac{1}{\sqrt 2\sin\theta}
		\]
		The first two components may only satisfy our system of linear equations if $a_{11} = a_{21} = 0$. The arbitrariness of 
		$a_{31} = a_{41}$ is dealt with by the normalization condition. For $\omega^2 = u$ we find
		\[
			a_{12} = -a_{22} = \frac{-1}{\sqrt 2},\quad a_{32} = -a_{42} = 0. 
		\]
		For $\omega^2 = u+2v$ we have
		\[
			a_{13} = a_{23} = \frac{1}{\sqrt 2},\quad a_{33} = -a_{43} = 0.
		\]
		Lastly for $\omega^2 = \frac{2v}{\sin^2\theta}$ we have
		\[
			a_{14}=a_{24} = 0,\quad a_{34} = -a_{44}= \frac{-1}{\sqrt 2\sin\theta}.
		\]
		Thus in total the following frequencies and vectors for displacement from equilibruim
		\[
			\omega_1^2 = 0,\quad \vect a_1 = \frac{1}{\sqrt 2\sin\theta} \bpm 0\\0\\1\\1 \epm
		\]
		\[
			\omega_2^2 = u,\quad \vect a_2 = \frac{1}{\sqrt 2}\bpm 1\\-1\\0\\0\epm
		\]
		\[
			\omega_3^2 = u+2v,\quad \vect a_3 = \frac{1}{\sqrt 2} \bpm 1\\1\\0\\0 \epm
		\]
		\[
			\omega_4^2 = \frac{2v}{\sin^2\theta},\quad \vect a_4 =  \bpm 0\\0\\1\\-1 \epm.
		\]
		Stopping for a moment to analyze what these motions mean, we find they are in agreement with the types of oscillations we
		intuitively expect.
		\\
		\\
	%2
		\item
		A top in a uniform gravitational field consists of a homogeneous, circular disc of radius $r$, and a massless axle of length $r/2$
		perpendicular to the disc through its center. The axle is connected by a joint to a point $A$ on the periphery of a horizontal circular
		merry-go-round of radius $R$, so that the axle of the top can pivot with negligible friction in a vertical plane containing the axle of the
		merry-go-round. The merry-go-round has the constant angular velocity $\Omega$ about its axle, and the top has the constant angular
		velocity $\omega$, relative to the merry-go-round, about its axle. Find the magnitude of the angle $\alpha$ between the axle of the top
		and the verticle.
		
		\figg[width=80mm]{15W2.png}
		
		A few hints: This problem might be easiest if one uses (and compares) a top-fixed and a space-fixed coordinate system. How do they 
		convert (especially since one is time dependent, the other is not)? Show during your calculation that the absolute value of the torque
		on the top is given by 
		\[
			|\vect N| = \frac{r}{2}mg\sin\alpha+\frac{r}{2}m\Omega^2(R+\frac{r}{2}\sin\alpha)\cos\alpha.
		\]
		Since you are looking for the equilibrium angle $\alpha$, it is fine to set $\dot\alpha = 0$ throughout the problem. 
		\\
		\\
		The hint seemed to derail me. Take the lagrangian of the system - center of mass plus rotation about center of mass. Rotation
		about it is simple. Center of mass motion is most easily decomposed in cartesian form. In the end, you find your equation of motion
		\[
			\frac{1}{4}m\ddot{\alpha} = \frac{r}{2}mg\sin\alpha+\frac{r}{2}m\Omega^2(R+\frac{r}{2}\sin\alpha)\cos\alpha
		\]
		The LHS represents the generalized torque for variable $\alpha$. Setting this net torque to zero allows one to find the
		equilibrium angle. Earlier I found equilibrium by setting the generalized forces
		\[
			-\pdiff[V]{q_i} = Q_i = 0.
		\]
		Here we essentially just set $\dot\alpha = 0$. Have not gotten to the bottom of this disparity yet. 
	% 3
		\item A uniform rigid rod $AB$ of mass $M$ and length $L$ is free to rotate smoothly (in a vertical plane $\chi$) about a
		horizontal axis through $A$ (fixed).
		\benum
			% (a)
			\item
			When the rod is hanging vertically at rest with $B$ below $A$, a bullet of mass $m$, moving horizontally in $\chi$ with
			speed $v$ perpendicular to the rod, hits the rod and get embedded in it at $B$. What is the \emph{initial angular speed} 
			of the rod immediately after this impact? Briefly justify your answer explaining why your approach in valid.
			\\
			\\
			First note that this must be an inelastic collision as some kinetic energy must be lost to the deformation of the bullet/rod. 
			Had this been a perfectly elastic collision, the bullet would transfer all its K.E. to the rod and thus be static post-collision. 
			Two objects sticking to one another after collision represents a maximally inelastic collision. Being inelastic, we cannot use 
			the conservation of energy in determining the initial speed $\omega$. In fact, momentum is not conserverd either, as there
			exists an external force at the hinge that acts to resist linear motion. Such an external force applies no torque on our system since 
			$\vect r_{hinge} = 0$. However, gravity \emph{will} induce a net torque on our system as soon as the rod begins to pivot. Thus we 
			may conclude that initially angular momentum is conserved, but as soon as the rod begins to pivot
			\[
				\frac{d\vect L}{dt}\ne 0.
			\]
			To calculate the initial angular velocity of the particle, we relate the angular momentum just before collision to that just 
			after
			\[
				lmv = I\omega.
			\]
			Choosing a coordinate system in which $\vect \omega = \omega\vecth x$ with origin at the pivot, we have
			\[
				I_{zz} = 0
			\]
			\[
				I_{xx}=I_{yy}= \int_{-l}^0 \frac{M}{l}z^2 = \frac{1}{3}Ml^2.
			\]
			We must also remember to add the contribution due to the bullet, 
			\[
				I = \frac{1}{3}Ml^2+ml^2
			\]
			so the angular momentum is 
			\[
				\vect L = \plr{\frac{1}{3}Ml^2+ml^2}\omega\vecth x.
			\]
			Equating the initial momentum to the angular momentum of the system immediately after impact
			\[
				lmv = I\omega
			\]
			\[
				\omega = \frac{lmv}{I}
			\]
			with $I$ given as above. 
			% (b)
			\item
			If the impact is \emph{barley sufficient} to make the rod horizontal, find the \emph{initial speed $v$} of the bullet, 
			before the impact (in terms of $l$ and $m$). Again, justify your answer explaining why your approach is valid.
			\\
			\\
			Gravity, being a conservative force means that the total mechanical energy of the system ($U=T+V$) is conserved. This
			is true only if the pivot is frictionless of course. Operating on that assumption, we may equate the initial kinetic energy of the 
			system to that of the difference in gravitational potential in order to find the speed $v$ that makes the rod horizontal. The 
			potential due to gravity is 
			\[
				V = m_i\vect r_i\cdot\vect g.
			\] 
			With the definition of the location of the center of mass being 
			\[
				\vect R = \frac{\sum_i m_i\vect r_i}{\sum_i m_i}
			\]
			we have
			\[
				V = M\vect R\cdot \vect g.
			\]
			The location of the center of mass along the direction of the rod is simply
			\[	
				r' = \frac{M\pfrac{l}{2}+ml}{M+m}.
			\]
			Defining the gravitational potential of the system immediately after collision as $V=0$, the gravitational potential 
			when the rod is horizontal is then
			\[
				V = (M+m)r'g.
			\]
			Finally equating the potential to the kinetic energy,
			\[
				\frac{1}{2}I\omega^2 = (M+m)r'g
			\]
			substituting $\omega$
			\[
				\frac{1}{2}Il^2m^2v^2 = (M+m)r'g. 
			\]
			We shall leave $v$ in this implicit form.
			\\
			\\ 
			\item
			% (c)
			If the hinge at $C$ is not smooth, are your answers to (a) and (b) still valid? Explain briefly. (No calculations here.)
			\\
			\\
			With friction present at the hinge, this introduces an external torque acting on the system as the rod rotates. In the limit
			that the frictional torque becomes very large, we know the rod does not move. Thus, its initial angular velocity $\omega
			\approx 0$. In addition, the torque friction causes a loss in energy so conservation of mechanical energy cannot be used. 
			Therefore, the answers to (a) and (b) no longer remain valid. 
		\eenum
		
	% 4
		\item Three identical objects, each of mass $m$, are connected by springs of spring constant $k$ as shown in the
		figure. The motion is confined to one dimension. At $t=0$, the masses are at rest at their equilibrium positions. Mass $A$
		is subjected to a force of $F = f\cos(\omega t)$, $t>0$. Calculate the motion of mass $C$. All surfaces are frictionless.
		
		\figg[width = 100mm]{15W4.png}
		
		First for the homogeneous system without the driving force,
		\[
			V = \frac{1}{2}k\blr{(\eta_2-\eta_1)^2+(\eta_3-\eta_2)^2}
		\]
		and
		\[
			T = \frac{1}{2}m(\dot\eta_1^2+\dot\eta_2^2+\dot\eta_3^2). 
		\]
		As we can see, $V$ is already in its quadratic form, so its first order approximation is exact. In other words
		\[
			V \approx \frac{1}{2}\elr{\plr{\pdifff{V}{q_i,q_j}}}_{eq}\eta_i\eta_j = V.
		\]
		In definiing
		\[
			T = \frac{1}{2}T_{ij}\dot\eta_1\dot\eta_2, \quad
			V = \frac{1}{2}V_{ij}\eta_1\eta_2
		\]
		we see that
		\[
			T_{ij} = \bpm m&0&0\\0&m&0\\0&0&m \epm
		\]
		\[
			V_{ij} = \bpm k&-k&0\\-k&2k&-k\\0&-k&k \epm.
		\]
		Now our Lagrangian takes the form 
		\[
			\mathcal{L} = \frac{1}{2}\plr{T_{ij}\dot\eta_i\dot\eta_j-V_{ij}\eta_i\eta_j}
		\]
		with an equation of motion as
		\[
			\frac{1}{2}T_{ij}\ddot{\eta_j}+\frac{1}{2}V_{ij}\eta_j = 0. 
		\]
		If we guess a solution of the form
		\[
			\eta_i = a_ie^{i\omega t}
		\]
		then we find
		\[
			(V_{ij}-T_{ij}\omega^2)a_i  = 0.
		\]
		This is our homogeneous set of linear equations, which can only have a nontrivial solution (must not be invertible) if
		\[
			|V_{ij}-\omega^2T_{ij}| = 0. 
		\]
		Solving for the roots of the secular equation, we find the roots of
		\[
			\omega_1 = 0,\quad \omega_2 = \sqrt\frac{k}{m},\quad \omega_3 = \sqrt{\frac{3k}{m}}.
		\]
		We may solve for the eigenvectors of each frequency. We will either need to choose a normalization condition, or express
		two components in terms of one component (set one component constant in other words). Doing such, we find
		\[
			\omega_1 = 0,\quad C\bpm 1\\1\\1 \epm
		\]
		\[
			\omega_2 = \sqrt\frac{k}{m},\quad C\bpm 1\\0\\-1 \epm
		\]
		\[
			\omega_3 = \sqrt\frac{3k}{m},\quad \bpm 1\\-2\\1 \epm .
		\]
		The position of each mass over time may be given as a linear combination of the homogeneous solutions
		\[
			\bpm \eta_1(t)\\ \eta_2(t) \\ \eta_3(t) \epm  = C_1 \bpm 1\\1\\1 \epm e^{i\omega_1t}
			+ C_2 \bpm 1\\0\\-1 \epm e^{i\omega_2 t}
			+ C_3 \bpm 1\\-2\\1 \epm e^{i\omega_3t}
		\]
		where $C_1$, $C_2$, and $C_3$ represent the initial conditions at $t=0$. These scale factors are complex, and thus may be 
		decomposed into real and imaginary parts. 
	\eenum
	
% 2012 Winter ------------------------------------------------------------------------------------------------------------------------------------------------------------------------
\phantom{}
\subsection*{2012 Winter}
\phantom{}
	\benum
	% 1
		\item
		Let us define $D = \frac 12(xp+px)$, where $x$ is the position operator and $p$ the momentum operator in one dimension.
		\benum
			% (a)
			\item
			Calculate $[D,x^m]$ and $[D,p^n]$ where $m$ and $n$ are integers.
			\\
			\\
	
			% (b)
			\item
			Consider the Hamiltonian operator $\ds H = \frac{p^2}{2m}+V(x)$ with the potential $V(x) = \alpha x^\beta$ where $\alpha$ and 
			$\beta$ are real non-zero constants. Calculate $U(\lambda)HU^\dag(\lambda)$ with $U(\lambda) = \exp(i\lambda D/\h)$.
	
				
			% (c)
			\item
			There is a value for $\beta$ in the potential $V(x) = \alpha x^\beta$ for which the Hamiltonian in part (b) transforms as 
			$U(\lambda)HU^\dag(\lambda) = f(\lambda)H$. What is the function $f(\lambda)$?
			\\
			\\
			We can easily see that for $\beta = -2$ we could express
			\[
				U(\lambda)HU^\dag(\lambda) = \exp(-2\lambda)H
			\]
			and thus
			\[
				f(\lambda) = \exp(-2\lambda).
			\]
			\\
			\\
			Hints: Recall the identity for two non-commuting linear operators $A$ and $B$:
			\[
				\exp(\lambda A)B\exp(-\lambda A) = B+\frac{\lambda^1}{1!}[A,B]+\frac{\lambda^2}{2!}[A,[A,B]]+\frac{\lambda^3}{3!}
				[A,[A,[A,B]]]+...
			\]
			You may do the mathematics formally, ignoring issues such as the precise definitions and domains of various operators.
		\eenum
	% 2
		\item
		In this astrophysics exercise denote the mass, radius and rotation angular velocity of the Earth by $M_e$, $R_e$, and 
		$\omega_e$. A small asteroid with mass $m\ll M_e$ strikes the assumedly perfectly spherical Earth at the co-latitude (polar angle) 
		$\theta$. Assume the sole effect of the asteroid is to deposit its mass at the point of impact. This breaks the full rotation symmetry of
		the Earth, and thus fixes the direction for one principle axis of rotation.
		\benum
			% (a)
			\item
			Show that the new moments of inertia are $I_1 = I_2 = \frac{2}{5}M_eR_e^2+mR_e^2+\mathcal O(m^2)$, 
			$I_3 = \frac{2}{5}M_eR_e^2$.
			\\ \\
			For an object there exists an orientation of body fixes axes such that the inertia tensor is in diagonal form. The eigenvalues or
			elements of this diagonal matrix are the principle moments of inertia (not to be confused with the moment of inertia, which is 
			defined as $I = \vect \omega\cdot \vect I\cdot \omega$). Given an origin and orientation of the principle axes, the principle 
			moments of inertia can be calculated by using
			\[
				I_{\alpha\beta} = \int \rho(\vect r)(\delta_{\alpha\beta}r^2-r_\alpha r_\beta)dV.
			\]
			For the case at hand, we have a sphere which possesses full rotational symmetry. Thus $I_1 = I_2 = I_3$. With the origin of 
			the body axes placed at the center of the sphere, we may calculate $I_i = \frac{2}{5}M_e R_e^2$. When the asteroid impacts 
			the Earth, it deposits a mass $m$ located at some point $\vect r_0$ where $|\vect r_0| = R_e$. Having broke the rotational 
			symmetry, the body axes we defined prior no longer serve as the principle axes. We must choose a new orientation of body 
			axes for $\vect I$ to be in diagonal form. A simple choice is to place the body $z$-axis through the point of impact such that 
			we have symmetry always lying in the $x$-$y$ plane, i.e. $I_1 = I_2$. The new principle moments of inertia are calculated by 
			adding the contribution due to the asteroid. Since the asteroid mass is located at $(x,y) = (0,0)$, the added contribution can
			be calculated to be $I' = mR_e^2$. Thus we have shown that the new principle moments of inertia are
			\[
				I_1 = I_2 = \frac{2}{5}M_e R_e^2+mR_e^2
			\]
			\[
				I_3 = \frac{2}{5}M_eR_e^2.
			\]
			\\
			Choosing these new principle axes, however, means that the axis of rotation $\vect \omega$ no longer lies along a principle 
			axis. Therefore, as the body rotates about $\vect \omega$ the moments of inertia (from the space frame) change. In order to 
			conserve angular momentum, $\vect \omega$ itself must also change accordingly. What we find is that in the body frame, 
			$\vect \omega$ and $\vect L$ precess around the principle $z$-axis at a precession rate $\Omega$. In the space fixed frame, we 
			see that $\vect \omega$ and the $z$-axis of the body frame precess around the angular momentum vector $\vect L$. If the angular 
			momentum were to remain in the same direction as before the collision, then the body $z$-axis would precess about
			the north pole with the precession frequency. 
			\\ \\
			The result is that the axis of the Earth starts precessing about an axis that goes through the center of the Earth and the point 
			of the impact.\\
			% (b)
			\item
			Find the angular frequency of the precession.
			\\
			\\
			To quantify the above discussion, we use Euler's equations of motion for a rigid body. Starting with 
			\[
				\vect N = \frac{d\vect L}{dt} 
			\]
			we can express the change of a vector as seen in the space frame to that in body frame which is rotating by
			\[
				\plr{\frac{d\vect L}{dt}}_s = \plr{\frac{d\vect L}{dt}}_b +\vect\omega \times \vect L.
			\]
			Taking the body axis as the principle axis and expanding the equations, we arrive at Euler's equations of motion
			\[
				I_1\dot{\omega_1}-\omega_2\omega_3(I_2-I_3) = N_1
			\]
			\[
				I_2\dot{\omega_2}-\omega_3\omega_1(I_3-I_1) = N_2
			\]
			\[
				I_3\dot{\omega_3}-\omega_1\omega_2(I_1-I_2) = N_3.
			\]
			Note that $\vect \omega$ and $\vect L$ are taken with respect to the body frame. For our problem there are no torques 
			applied and $I_1=I_2$, thus we have
			\[
				I_1\dot{\omega_1} = (I_1-I_3)\omega_3\omega_2
			\]
			\[
				I_1\dot{\omega_2}= (I_3-I_1)\omega_3\omega_1
			\]
			\[
				I_3\dot{\omega_3} = 0. 
			\]
			First we see that $\omega_3 = \omega_e\cos\theta$. If we define
			\[
				\Omega = \omega_3\frac{I_3-I_1}{I_1}
			\]
			then we can express the last two EOM as
			\[
				\dot{\omega_1}  = -\Omega \omega_2
			\]
			\[
				\dot{\omega_2} = \Omega\omega_1. 
			\]
			Applying another time derivative
			\[
				\ddot{\omega_1} = -\Omega\dot{\omega_2} = -\Omega^2\omega_1
			\]
			we find
			\[
				\omega_1 = A\cos(\Omega t).
			\]
			where $A$ is the amplitude of $\omega_1$ at $t=0$.
			Using this result we find 
			\[
				\omega_2 = A\sin(\Omega t).
			\]
			We conclude that the vector $\omega_1\vecth x +\omega_2 \vecth y$ has constant magnitude and precesses around the
			$\vecth z$ with frequency $\Omega$. Expressed in terms of the initial problem, the rate of precession is
			\[
				\Omega = \omega_e\cos\theta\plr{\frac{mR_e^2}{\frac{2}{5}M_eR_e^2+mR_e^2}}. 
			\]
		\eenum
	% 3
		\item
		A particle of mass $m$ and electric charge $q$ is constrained to move in a tightly confining ring of radius $R$; call the remaining 
		coordinate along the ring $x$. The motion along $x$ is free, i.e., there are no forces acting on the particle in the direction $x$.
		Determine:
		\benum
			% (a)
			\item
			Eigenvalues and eigenfunctions of energy.
			% (b)
			\item
			The maximum value of the electric current $I$ in the first excited state.
			\\
			\\
			Hint: The current density of a quantum particle is $\ds\vect j = \frac{i\h q}{2m}(\psi\del\psi^*-\psi\del\psi)$.
		\eenum	
	% 4
		\item
		Problem
		\benum
			% (a)
			\item
			
			% (b)
			\item
			% (c)
			\item
		\eenum
	% 5
		\item
		Consider the Hamiltonian
		\[
			H = E_1\ket 1\bra 1+E_2\ket 2\bra 2+V\ket 2\bra 1 +V^*\ket 1\bra 2
		\]
		with $|V|\ll |E_2-E_1|$.
		\benum
			% (a)
			\item
			Find the eigenvalues of energy and the corresponding normalized eigenstates up to the lowest nontrivial order in the
			strength of the perturbation $V$. Denote these by $E'_1$, $\ket{1'}$ and $E'_2$, $\ket{2'}$, with $E'_1\to E_1$ as $V\to 0$ 
			and so on.
			% (b)
			\item
			Suppose we are studying transitions from yet another state $\ket g$ to the states $\ket 1$ and $\ket 2$ governed by the operator
			$D$, and have the known transition matrix elements $\braket{1|D|g}=d$, $\braket{2|D|g}=0$. At this level the transition $g\to 2$ is 
			evidently forbidden. However, the perturbation $V$ leads to a small admixture of the original state $\ket 1$ in the state $\ket{2'}$. 
			Thus a transition that to an observer unaware of the existence of the perturbation $V$ might seem to be $g\to 2$ is possible after 
			all. Find the corresponding matrix element $\braket{2'|D|g}$. 
		\eenum
	\eenum
	
% 2014 Winter ------------------------------------------------------------------------------------------------------------------------------------------------------------------------
\phantom{}
\subsection*{2014 Winter}
\phantom{}
	\benum
	% 1
		\item
		Let $H = H_{kin} + V(\vect x)$ be a single-particle Hamiltonian operator with $H_{kin} = \frac{\vect p^2}{2m}$ and $m$ the
		mass of the particle. Consider the operator $D = \frac{1}{2}(\vect x\cdot\vect p +\vect p\cdot\vect x)$.
		\benum
			% (a)
			\item
			Calculate the commutators $[D, \vect x]$ and $[D,\vect p]$.
			\\
			% (b)
			\item
			Calculate $[D, F(\vect x)]$ and $[D,G(\vect p)]$ where $F$ and $G$ are differentiable functions. You may want to
			work out the commutators in position or momentum space.
			\\
			% (c)
			\item
			Let $H\ket{E_i} = E_i\ket{E_i}$. Calculate $[D,H]$ and prove that $2\braket{E_i|H_{kin}|E_i} = 
			\braket{E_i|\vect x\cdot\del V(\vect x)|E_i}$, which is the quantum mechanical virial theorem. 
		\eenum
	\eenum
\end{document}