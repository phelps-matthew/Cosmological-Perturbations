\documentclass[11pt,letterpaper]{article}
\usepackage{macroshw}


\title{Electricity and Magnetism Preliminary Examination Problems}
\author{Matthew Phelps}
\date{Last Updated: \today}

\begin{document}
\maketitle

% 2015 Winter------------------------------------------------------------------------------------------------------------------------------------------------------------------------
\subsection*{2015 Winter}
\phantom{}
	\benum
	% 1
		\item
		Consider a point charge $q$ inside a conducting grounded sphere of radius $R$ at the distance $a<R$ from the center of the
		sphere.
		
		\benum
			% (a)
			\item
			Determine the Greens function $G_D(\vect x,\vect x')$ for this Dirichlet-type problem. 
			\\
			\\
			Dirichlet-type because we are specifying the potential on a closed surface. The generalized integral relation for the potential
			is derived from Green's identities 
			\[
				\Phi(\vect x) = \frac{1}{4\pi\epo}\int_Vd^3x'\, \rho(\vect x')G(\vect x,\vect x') + \frac{1}{4\pi}\oint_S da'\, 
				\blr{G(\vect x,\vect x')\pdiff[\Phi]{n'}-\Phi(\vect x')\pdiff[G(\vect x,\vect x')]{n'}}.
			\]
			With 
			\[
				G(\vect x,\vect x') = \frac{1}{|\vect x-\vect x'|}+F(\vect x,\vect x'),
			\] 
			where $G(\vect x,\vect x')$ is a Green's function, i.e. satisfying
			\[
				\del^2G(\vect x,\vect x') = -4\pi\delta(\vect x-\vect x'),
			\]
			we need to find the form of $F(\vect x,\vect x')$ such that $G(\vect x,\vect x') = 0$ on the surface. This will cancel the
			first term in the surface integral and allow us to compute the potential with the specified boundary condition. For the sphere,
			such a boundary condition can be simulated by placing an image charge along the position ray. If we denote $\vecth n'$ in the 
			direction of the position of the real charge and $\vecth n$ in the direction of an arbitrary point of observation, the then potential
			will be
			\[
				\Phi(\vect x) = \frac{1}{4\pi\epo}\blr{\frac{q}{|x\vecth n-a\vecth n'|}+\frac{q'}{|x\vecth n - b\vecth n'|}}.
			\]
			At position $|\vect x| = R$, the potential must vanish thus
			\[
				\Phi(\vect x = R) = \frac{1}{4\pi\epo}\blr{\frac{q}{|R\vecth n-a\vecth n'|}+\frac{q'}{|R\vecth n - b\vecth n'|}} = 0.
			\]
			Written in a clearer form,
			\[
				\Phi(\vect x = R) = \frac{1}{4\pi\epo}\blr{\frac{q/R}{|\vecth n-a/R\vecth n'|}+\frac{q'/b}{|R/b\vecth n - \vecth n'|}} = 0
			\]
			we can now see that we must have
			\[
				\frac{q}{R} = \frac{q'}{b}
			\]
			\[
				\frac{a}{R} = \frac{R}{b}
			\]
			or
			\[
				q' = -q\frac{R}{a},\quad b = \frac{R^2}{a}.
			\]
			Therefore our potential is then
			\[
				\Phi(\vect x) = \frac{q}{4\pi\epo}	\blr{\frac{1}{|\vect x-\vect x'|}-\frac{R/a}{|\vect x-\frac{R^2}{a^2}\vect x'|}}
			\]
			where $\vect x' = a\vecth n'$. Such an equation has the value of zero for $|\vect x| = R$. To form the Green's function,
			note that we can view $\vect x'$ as a variable. It may represent the location of a unit point charge. While this charge may
			have been fixed inside the sphere for our potential, in the Green's function, it may vary anywhere inside the volume. Thus 
			we can easily generalize our potential to the Green's function as
			\[
					G(\vect x,\vect x') = \frac{1}{|\vect x-\vect x'|}-\frac{R/x'}{|\vect x-\frac{R^2}{x'^2}\vect x'|}.
			\]
			$G(\vect x,\vect x')$ can be shown to be symmetric and thus we are free to exchange our interpretation of observation/source
			point location. 
			\\
			% (b)
			\item
			Derive the potential $\phi(\vect x)$ and the electric field $\vect E(\vect x)$ inside the sphere.
			\\
			\\
			The potential has been given as
			\[
				\Phi(\vect x) = \frac{q}{4\pi\epo}	\blr{\frac{1}{|\vect x-\vect x'|}-\frac{R/a}{|\vect x-\frac{R^2}{a^2}\vect x'|}}.
			\]
			To find the electric field, we may take the derivative. We know the electric field must be perpendicular to the conducting
			surface, so we only need to find the derivative with respect to the radial component
			\[
				 E(r) = -\del\cdot\vecth r = -\pdiff[\Phi(\vect x)]{r}.
			\]
			We may represent the magnitude of the difference between two vectors using the law of cosines
			\[
				\frac{1}{|\vect x-\vect x'|} = \frac{1}{\plr{r^2+a^2-2ra\cos\theta}^{1/2}}
			\]
			where $\vect x' = a\vecth x'$ and $\vect x = r\vecth x$, and $\theta$ denotes the angle between the two vectors.  Similarly,
			\[
				\frac{R/a}{|\vect x-\frac{R^2}{a^2}\vect x'|} = \frac{1}{\plr{r^2+\frac{R^4}{a^2}-2r\pfrac{R^2}{a}\cos\theta}^{1/2}}.
			\]
			We know our problem also posses azimuthal symmetry, so we may define $\vecth x' = \vecth z$ such that $\theta$
			represents the polar angle. Then we may take the derivative in polar coordinates. We should expect to have a component
			in the $\vecth \theta$ direction due to the lack of polar symmetry in the location of the point charge. Derivatives 
			can be taken in the usual way. 
			
				
			% (c)
			\item
			Determine the induced charge density per unit area, $\sigma(\vect x)$, on the sphere, and compute the total induced
			charge $Q_{ind} = \int_S\sigma(\vect x)\, dA$ where $S$ denotes the area of the sphere.
			\\
			\\
			Since the conductor is grounded and the potential at infinity should be zero, we should be able to discern that 
			the electric field is zero outside the sphere. There are alternate (and more precise) ways of arriving to the same conclusion. 
			Nonetheless, we may then imagine a Gaussian surface enclosing a small area on the conductor. We find that the 
			surface charge density must be given as
			\[
				\vect E\cdot d\vect S = \ \frac{\sigma}{\epsilon_0}dS = \frac{q_{enc}}{\epo}.
			\]
			Thus the charge density at the surface is proportional to the radial component of the electric field
			\[
				\sigma(\theta) = -\epo\elr{\pdiff[\Phi]{r}}_{r=R}.
			\]
			I won't do out the derivative explicitly. Now to find the total induced charge, we simply integrate the surface charge density
			over the area of a sphere. Starting with
			\[	
				dS = R^2\sin\theta\, d\theta\, d\phi
			\]
			we have
			\[	
				q_{ind} = 2\pi R^2 \int_0^\pi d\theta\, \sin(\theta) \sigma(\theta) = -q
			\]
			\eenum
			
			Hint: Use the convention $\Delta_xG_D(\vect x,\vect x') = -4\pi\delta^{(3)}(\vect x-\vect x')$, where the solution of a 
			Dirichlet-type problem is given by 
			
			\[
				\phi(\vect x) = \frac{1}{4\pi\epo}\int_V\rho(\vect x')G_D(\vect x,\vect x')\, d^3x' 
				- \frac{1}{4\pi}\int_S\phi(\vect x')\pdiff[G_D(\vect x,\vect x')]{n'}\,dA
			\]
	%2
		\item
		Consider a homogeneous and isotropic conductor with conductivity $\sigma$, permittivity $\epsilon$, permeability $\mu$,
		and volume $V_{cond}$. Let a small volume $V_0 \ll V_{cond}$ inside the conductor contain a charge distribution 
		$\rho_0(\vect x)\ne 0$ at the time $t=0$
		\\
		\benum
			\item
			% (a)
			Derive the expression $\rho(\vect x,t)$ describing the charge distribution inside the volume $V_0$ for times $t>0$,
			assuming that $\rho(\vect x,t)$ varies with time sufficiently slowly to neglect retardation effects.
			\\
			
			\item
			% (b)
			Determine the current $I(t)$ flowing out of the volume $V_0$. Calculate the total charge 
			$Q = \int_0^\infty dt\, I(t) $ that flows out of the volume $V_0$. Where does $Q$ end up "after" $t\to\infty$?
			\\
			
			\item 
			% (c)
			Using the quantities specified in the problem, find the dimensionless ration $x\ll 1$ which justifies that 
			"things vary with time slowly" in part (a).
			\\
		\eenum
		
		
	% 3
		\item Two long cylindrical conductors of radius $a_1$ and $a_2$ are parallel and separated by a distance $d$ which
		is large compared to $a_1$, $a_2$. Show that the capacitance per unit length is given by 
		\[
			C\approxeq \frac{\pi\epo}{\ln(d/a)}\quad\text{where}\quad a = (a_1a_2)^{1/2}
		\]
		
		
	% 4
		\item Three identical objects, each of mass $m$, are connected by springs of spring constant $k$ as shown in the
		figure. The motion is confined to one dimension. At $t=0$, the masses are at rest at their equilibrium positions. Mass $A$
		is subjected to a force of $F = f\cos(\omega t)$, $t>0$. Calculate the motion of mass $C$. All surfaces are frictionless.
		
		
	\eenum
	
% 2012 Winter ------------------------------------------------------------------------------------------------------------------------------------------------------------------------
\phantom{}
\subsection*{2012 Winter}
\phantom{}
	\benum
	% 1
		\item
		Let us define $D = \frac 12(xp+px)$, where $x$ is the position operator and $p$ the momentum operator in one dimension.
		\benum
			% (a)
			\item
			Calculate $[D,x^m]$ and $[D,p^n]$ where $m$ and $n$ are integers.
			\\
			\\
	
			% (b)
			\item
			Consider the Hamiltonian operator $\ds H = \frac{p^2}{2m}+V(x)$ with the potential $V(x) = \alpha x^\beta$ where $\alpha$ and 
			$\beta$ are real non-zero constants. Calculate $U(\lambda)HU^\dag(\lambda)$ with $U(\lambda) = \exp(i\lambda D/\h)$.
	
				
			% (c)
			\item
			There is a value for $\beta$ in the potential $V(x) = \alpha x^\beta$ for which the Hamiltonian in part (b) transforms as 
			$U(\lambda)HU^\dag(\lambda) = f(\lambda)H$. What is the function $f(\lambda)$?
			\\
			\\
			We can easily see that for $\beta = -2$ we could express
			\[
				U(\lambda)HU^\dag(\lambda) = \exp(-2\lambda)H
			\]
			and thus
			\[
				f(\lambda) = \exp(-2\lambda).
			\]
			\\
			\\
			Hints: Recall the identity for two non-commuting linear operators $A$ and $B$:
			\[
				\exp(\lambda A)B\exp(-\lambda A) = B+\frac{\lambda^1}{1!}[A,B]+\frac{\lambda^2}{2!}[A,[A,B]]+\frac{\lambda^3}{3!}
				[A,[A,[A,B]]]+...
			\]
			You may do the mathematics formally, ignoring issues such as the precise definitions and domains of various operators.
		\eenum
	% 2
		\item
		In this astrophysics exercise denote the mass, radius and rotation angular velocity of the Earth by $M_e$, $R_e$, and 
		$\omega_e$. A small asteroid with mass $m\ll M_e$ strikes the assumedly perfectly spherical Earth at the co-latitude (polar angle) 
		$\theta$. Assume the sole effect of the asteroid is to deposit its mass at the point of impact. This breaks the full rotation symmetry of
		the Earth, and thus fixes the direction for one principle axis of rotation.
		\benum
			% (a)
			\item
			Show that the new moments of inertia are $I_1 = I_2 = \frac{2}{5}M_eR_e^2+mR_e^2+\mathcal O(m^2)$, 
			$I_3 = \frac{2}{5}M_eR_e^2$.
			\\ \\
			For an object there exists an orientation of body fixes axes such that the inertia tensor is in diagonal form. The eigenvalues or
			elements of this diagonal matrix are the principle moments of inertia (not to be confused with the moment of inertia, which is 
			defined as $I = \vect \omega\cdot \vect I\cdot \omega$). Given an origin and orientation of the principle axes, the principle 
			moments of inertia can be calculated by using
			\[
				I_{\alpha\beta} = \int \rho(\vect r)(\delta_{\alpha\beta}r^2-r_\alpha r_\beta)dV.
			\]
			For the case at hand, we have a sphere which possesses full rotational symmetry. Thus $I_1 = I_2 = I_3$. With the origin of 
			the body axes placed at the center of the sphere, we may calculate $I_i = \frac{2}{5}M_e R_e^2$. When the asteroid impacts 
			the Earth, it deposits a mass $m$ located at some point $\vect r_0$ where $|\vect r_0| = R_e$. Having broke the rotational 
			symmetry, the body axes we defined prior no longer serve as the principle axes. We must choose a new orientation of body 
			axes for $\vect I$ to be in diagonal form. A simple choice is to place the body $z$-axis through the point of impact such that 
			we have symmetry always lying in the $x$-$y$ plane, i.e. $I_1 = I_2$. The new principle moments of inertia are calculated by 
			adding the contribution due to the asteroid. Since the asteroid mass is located at $(x,y) = (0,0)$, the added contribution can
			be calculated to be $I' = mR_e^2$. Thus we have shown that the new principle moments of inertia are
			\[
				I_1 = I_2 = \frac{2}{5}M_e R_e^2+mR_e^2
			\]
			\[
				I_3 = \frac{2}{5}M_eR_e^2.
			\]
			\\
			Choosing these new principle axes, however, means that the axis of rotation $\vect \omega$ no longer lies along a principle 
			axis. Therefore, as the body rotates about $\vect \omega$ the moments of inertia (from the space frame) change. In order to 
			conserve angular momentum, $\vect \omega$ itself must also change accordingly. What we find is that in the body frame, 
			$\vect \omega$ and $\vect L$ precess around the principle $z$-axis at a precession rate $\Omega$. In the space fixed frame, we 
			see that $\vect \omega$ and the $z$-axis of the body frame precess around the angular momentum vector $\vect L$. If the angular 
			momentum were to remain in the same direction as before the collision, then the body $z$-axis would precess about
			the north pole with the precession frequency. 
			\\ \\
			The result is that the axis of the Earth starts precessing about an axis that goes through the center of the Earth and the point 
			of the impact.\\
			% (b)
			\item
			Find the angular frequency of the precession.
			\\
			\\
			To quantify the above discussion, we use Euler's equations of motion for a rigid body. Starting with 
			\[
				\vect N = \frac{d\vect L}{dt} 
			\]
			we can express the change of a vector as seen in the space frame to that in body frame which is rotating by
			\[
				\plr{\frac{d\vect L}{dt}}_s = \plr{\frac{d\vect L}{dt}}_b +\vect\omega \times \vect L.
			\]
			Taking the body axis as the principle axis and expanding the equations, we arrive at Euler's equations of motion
			\[
				I_1\dot{\omega_1}-\omega_2\omega_3(I_2-I_3) = N_1
			\]
			\[
				I_2\dot{\omega_2}-\omega_3\omega_1(I_3-I_1) = N_2
			\]
			\[
				I_3\dot{\omega_3}-\omega_1\omega_2(I_1-I_2) = N_3.
			\]
			Note that $\vect \omega$ and $\vect L$ are taken with respect to the body frame. For our problem there are no torques 
			applied and $I_1=I_2$, thus we have
			\[
				I_1\dot{\omega_1} = (I_1-I_3)\omega_3\omega_2
			\]
			\[
				I_1\dot{\omega_2}= (I_3-I_1)\omega_3\omega_1
			\]
			\[
				I_3\dot{\omega_3} = 0. 
			\]
			First we see that $\omega_3 = \omega_e\cos\theta$. If we define
			\[
				\Omega = \omega_3\frac{I_3-I_1}{I_1}
			\]
			then we can express the last two EOM as
			\[
				\dot{\omega_1}  = -\Omega \omega_2
			\]
			\[
				\dot{\omega_2} = \Omega\omega_1. 
			\]
			Applying another time derivative
			\[
				\ddot{\omega_1} = -\Omega\dot{\omega_2} = -\Omega^2\omega_1
			\]
			we find
			\[
				\omega_1 = A\cos(\Omega t).
			\]
			where $A$ is the amplitude of $\omega_1$ at $t=0$.
			Using this result we find 
			\[
				\omega_2 = A\sin(\Omega t).
			\]
			We conclude that the vector $\omega_1\vecth x +\omega_2 \vecth y$ has constant magnitude and precesses around the
			$\vecth z$ with frequency $\Omega$. Expressed in terms of the initial problem, the rate of precession is
			\[
				\Omega = \omega_e\cos\theta\plr{\frac{mR_e^2}{\frac{2}{5}M_eR_e^2+mR_e^2}}. 
			\]
		\eenum
	% 3
		\item
		A particle of mass $m$ and electric charge $q$ is constrained to move in a tightly confining ring of radius $R$; call the remaining 
		coordinate along the ring $x$. The motion along $x$ is free, i.e., there are no forces acting on the particle in the direction $x$.
		Determine:
		\benum
			% (a)
			\item
			Eigenvalues and eigenfunctions of energy.
			% (b)
			\item
			The maximum value of the electric current $I$ in the first excited state.
			\\
			\\
			Hint: The current density of a quantum particle is $\ds\vect j = \frac{i\h q}{2m}(\psi\del\psi^*-\psi\del\psi)$.
		\eenum	
	% 4
		\item
		Problem
		\benum
			% (a)
			\item
			
			% (b)
			\item
			% (c)
			\item
		\eenum
	% 5
		\item
		Consider the Hamiltonian
		\[
			H = E_1\ket 1\bra 1+E_2\ket 2\bra 2+V\ket 2\bra 1 +V^*\ket 1\bra 2
		\]
		with $|V|\ll |E_2-E_1|$.
		\benum
			% (a)
			\item
			Find the eigenvalues of energy and the corresponding normalized eigenstates up to the lowest nontrivial order in the
			strength of the perturbation $V$. Denote these by $E'_1$, $\ket{1'}$ and $E'_2$, $\ket{2'}$, with $E'_1\to E_1$ as $V\to 0$ 
			and so on.
			% (b)
			\item
			Suppose we are studying transitions from yet another state $\ket g$ to the states $\ket 1$ and $\ket 2$ governed by the operator
			$D$, and have the known transition matrix elements $\braket{1|D|g}=d$, $\braket{2|D|g}=0$. At this level the transition $g\to 2$ is 
			evidently forbidden. However, the perturbation $V$ leads to a small admixture of the original state $\ket 1$ in the state $\ket{2'}$. 
			Thus a transition that to an observer unaware of the existence of the perturbation $V$ might seem to be $g\to 2$ is possible after 
			all. Find the corresponding matrix element $\braket{2'|D|g}$. 
		\eenum
	\eenum
	
% 2014 Winter ------------------------------------------------------------------------------------------------------------------------------------------------------------------------
\phantom{}
\subsection*{2014 Winter}
\phantom{}
	\benum
	% 1
		\item
		Let $H = H_{kin} + V(\vect x)$ be a single-particle Hamiltonian operator with $H_{kin} = \frac{\vect p^2}{2m}$ and $m$ the
		mass of the particle. Consider the operator $D = \frac{1}{2}(\vect x\cdot\vect p +\vect p\cdot\vect x)$.
		\benum
			% (a)
			\item
			Calculate the commutators $[D, \vect x]$ and $[D,\vect p]$.
			\\
			% (b)
			\item
			Calculate $[D, F(\vect x)]$ and $[D,G(\vect p)]$ where $F$ and $G$ are differentiable functions. You may want to
			work out the commutators in position or momentum space.
			\\
			% (c)
			\item
			Let $H\ket{E_i} = E_i\ket{E_i}$. Calculate $[D,H]$ and prove that $2\braket{E_i|H_{kin}|E_i} = 
			\braket{E_i|\vect x\cdot\del V(\vect x)|E_i}$, which is the quantum mechanical virial theorem. 
		\eenum
	\eenum
\end{document}