\documentclass[10pt,letterpaper]{article}
\usepackage[textwidth=7in, top=1in,textheight=9in]{geometry}
\usepackage[fleqn]{mathtools} 
\usepackage{amssymb}
 \usepackage{braket}
\newcommand{\vect}[1]{\mathbf{#1}}
\newcommand{\vecth}[1]{\hat{\mathbf{#1}}}
%\numberwithin{equation}{subsection}
\title{Coordinate Transformations RW $k<0$ v7}
\date{}
\begin{document} 
\maketitle
\noindent 
For $K<1$ FRW cosmology with $L^2 a^2 = t^2+d^2$, the line element takes the form
\begin{align}
ds^2 &=  dt^2 - a(t)^2 \left(  \frac{dr^2}{1+r^2/L^2} + r^2 d\theta^2 + r^2\sin^2\theta d\phi^2 \right)\nonumber\\
&= d^2 \left[ du^2 - (1+u^2)\left( \frac{dv^2}{1+v^2} + v^2 d\Omega^2\right)\right] ,
\end{align}
where we have introduced
\begin{equation}
u = \frac{t}{d},\qquad v = \frac{r}{L}.
\end{equation}
\section*{Original Coordinates}
Transformations and Asymptotics:
\begin{equation}
p' = \frac{u}{(1+u^2)^{1/2}+(1+v^2)^{1/2}},\qquad r' = \frac{v}{(1+u^2)^{1/2}+(1+v^2)^{1/2}}
\end{equation}
\begin{equation}
u^2 = \frac{4 p'^2}{(1-(p'+r')^2)(1-(p'-r')^2)},\qquad v =\left(\frac{r'}{p'}\right)u
\end{equation}
\begin{equation}
\Omega^2(p',r') = \frac{4 L^2 a^2}{(1-(p'+r')^2)(1-(p'-r')^2)} = d^2(1+u^2)\left[ (1+u^2)^{1/2}+(1+v^2)^{1/2}\right]^2
\end{equation}
%%%%%%%%%%%%%%%%%%
\subsection*{Null Trajectory}
In the $u$, $v$ geometry, the condition for null separation (at fixed angle) is $u=v$. Inspection of coordinate transformation (3-5) shows the leading order ($u\gg 1$) contributions for null separation:
\begin{equation}
p' \sim 1,\qquad r' \sim 1,\qquad \Omega^2 \sim u^4.
\end{equation}
\begin{align}
\frac{\partial p'}{\partial t} & \sim  \frac{1}{u}\qquad
\frac{\partial p'}{\partial r}  \sim 	\frac{1}{u},\qquad
\frac{\partial r'}{\partial t}  \sim \frac{1}{u}\qquad
\frac{\partial r'}{\partial r}  \sim  \frac{1}{u}.
\end{align}
The leading behavior for the full $K_{\mu\nu}^{(cm)}$ behaves as
\begin{align}
K^{(cm)}_{00} &\sim u^2 \nonumber\\
K^{(cm)}_{01} &\sim  u^2\nonumber\\
K^{(cm)}_{02} &\sim  u^3\nonumber\\
K^{(cm)}_{03} &\sim   u^3\nonumber\\
K^{(cm)}_{11} &\sim  u^2\nonumber\\
K^{(cm)}_{22} &\sim  u^4\nonumber\\
K^{(cm)}_{33} &\sim  u^4\nonumber\\
K^{(cm)}_{12} &\sim u^3\nonumber\\
K^{(cm)}_{13} &\sim u^3\nonumber\\
K^{(cm)}_{23} &\sim u^4
\end{align}

The purely angular sector of this result coincides with the null configuration given in PRD 2012.
%%%%%%%%%%%%%%%%%%%%%%%%%%%%%%%%%%
\subsection*{Timelike Trajectory}
For coordinate separations which are timelike we take $u\gg v$. In order to find the leading contribution in $u$, we will effectively take $v$ to be finite on the order $\mathcal O(1)$,  and take $u\gg1$. These results yield a leading behavior of:
\begin{equation}
p'\sim 1,\qquad r'\sim \frac{1}{u},\qquad \Omega^2\sim u^4. 
\end{equation}
\begin{align}
\frac{\partial p'}{\partial t} & \sim  \frac{1}{u^2}\qquad
\frac{\partial p'}{\partial r}  \sim 	\frac{1}{u},\qquad
\frac{\partial r'}{\partial t}  \sim \frac{1}{u^2}\qquad
\frac{\partial r'}{\partial r}  \sim  \frac{1}{u}.
\end{align}
The leading behavior for the full $K_{\mu\nu}^{(cm)}$ behaves as
\begin{align}
K^{(cm)}_{00} &\sim 1 \nonumber\\
K^{(cm)}_{01} &\sim  u\nonumber\\
K^{(cm)}_{02} &\sim  u\nonumber\\
K^{(cm)}_{03} &\sim   u\nonumber\\
K^{(cm)}_{11} &\sim  u^2\nonumber\\
K^{(cm)}_{22} &\sim  u^2\nonumber\\
K^{(cm)}_{33} &\sim  u^2\nonumber\\
K^{(cm)}_{12} &\sim u^2\nonumber\\
K^{(cm)}_{13} &\sim u^2\nonumber\\
K^{(cm)}_{23} &\sim u^2
\end{align}
%%%%%%%%%%%%%%%%%%%%%%%%%%%%%%%
\section*{Email Comment}
The difference between the revised Appendix B and the results above resides only in the timelike $K_{t\theta}$ and $K_{t\phi}$ components. In the notation of Appendix B, when transforming from Cartesian to polar, a prefactor analgous to (B7) should be included for these modes, i.e.
\begin{equation}
K_{t'\theta} = \frac{\partial x^\alpha}{\partial \theta} K_{t'\alpha} = r'\cos\theta\cos\phi K_{t'x'}+r'\cos\theta\sin\phi K_{t'y'} - r\sin\theta K_{t'z'} \sim \frac{1}{t}.
\end{equation}
If such a prefactor is included, then $K_{t\phi}$ and $K_{t\theta}$ have an overall supression of $1/{t^3}$, and then when multiplied by $p'\Omega^2$ behave in total as $\sim t$. 
\\ \\
Concerning the difference between PRD 2012 and the results above, regarding $k_{\theta\theta}$ in particular,  we note that eq. (114)  in PRD has solution
\begin{equation}
k_{\theta\theta} \propto r'p' e^{iq(r'-p')}.
\end{equation}
However, in solving for $k_{\theta\theta}$ in APM (via coordinate transformation from the flat $\Box^2 k_{\mu\nu}=0$) we found solutions to obey
\begin{equation}
k_{\theta\theta} \propto r'^2 p' e^{iq(r'-p')}. 
\end{equation}
This additional factor of $r'$ differentiates PRD and APM. In null configuations $r'\sim 1$, and thus the two results agree asymptotically. However, for lightlike configurations, $r' \sim 1/t$ and thus the leading angular sector behavior in PRD will behave as $t^3$ while in APM as $t^2$. 
\\ \\
It still remains for me to figure out a). why the synchronous condition would yield a different $r'$ dependence and b). why it appears the asymptotic behavior differs when working in the new coordinates system ($\Omega(T,R)$). Insight into the latter is expected to be found in the gauging procedure for each coordinate system 
\end{document}