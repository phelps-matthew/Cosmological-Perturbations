\documentclass[10pt,letterpaper]{article}
\usepackage[textwidth=7in, top=1in,textheight=9in]{geometry}
\usepackage[fleqn]{mathtools} 
\usepackage{amssymb}
\newcommand{\vect}[1]{\mathbf{#1}}
\newcommand{\vecth}[1]{\hat{\mathbf{#1}}}
%\numberwithin{equation}{subsection}
\title{Coordinate Transformations RW $k<0$ v1 }
\date{}
\begin{document}
\maketitle
\noindent 
\section*{Roberston Walker Metric}
We may form a 3-space of constant curvature by embedding within a flat 4-space, just as we may embed a 2-sphere or 2 dimensional hyperbola (or also a flat plane) within 3 dimensional space. Constraining to a space of constant curvature, we have
\begin{equation}
\vect x^2 +z^2 = C^2.
\end{equation}
Here $C^2$ represents the degree and sign of curvature, with dimension of length $C \sim [L]$. For $C^2$ positive, we have a bound 3-sphere, while for $C^2 =0$, we have unbound Euclidean geometry, and for $C^2 <0$ we have an unbound hyperbolic geometry. Constructing the flat 4-space line element,
\begin{equation}
ds^2 = d\vect x^2 + dz^2.
\end{equation}
Taking the differential of (1) allows us to relate $dz$ to the three space variables $\vect x$ via
\begin{equation}
dz^2 = \frac{ (\vect x\cdot d \vect x)^2}{C^2-\vect x^2}
\end{equation}
Substituting into the line element we have
\begin{equation}
	ds^2 = d\vect x^2 +   \frac{ (\vect x\cdot d \vect x)^2}{C^2-\vect x^2}
\end{equation}
Adopting polar coordinates, this becomes
\begin{equation}
 ds^2 = \frac{dr^2}{1-r^2/C^2} + r^2 d\Omega^2
\end{equation}
With the above general form for a maximally symmetric 3-space with constant curvature, we may form the invariant spacetime interval as 
\begin{equation}
ds^2 = dt^2 - a(t)^2 \left(  \frac{dr^2}{1-r^2/C^2} + r^2 d\theta^2 + r^2\sin^2\theta d\phi^2 \right)
\end{equation}
where $a(t)$ is an arbitrary function of time to be set by dynamics. Worth noting is that if we rescale $r' = r/|C|$, radial distances will be dimensionless and $a_{rescaled}(t) = a(t)/|C|$ will have dimension of $[L]$. Such a rescaling is necessary for the metric convention in which $\frac{dr^2}{1-Kr^2}$ for $K \in [-1,0,1]$. However, cosmological convention utilizes a dimensionless $a(t)$, thus we leave in the form of $r^2/C^2$. 
\\ \\
By a coordinate transformation upon $t$ via
\begin{equation}
\tau = \int \frac{dt}{a(t)},
\end{equation}
we may express (6) in terms of conformal time $\tau$ as 
\begin{equation}
ds^2 = a^2(\tau) \left ( d\tau^2 - \frac{dr^2}{1-r^2/C^2} + r^2 d\Omega^2 \right)
\end{equation}
\section*{RW to Conformal to Flat Form}
\subsection*{First Transformation}
As the first step towards bringing the metric to conformal-flat form for $C^2<0$, we introduce curvature magnitude $L^2 = -C^2$ (an inherently positive quantity) and we make coordinate transformations 
\begin{equation}
p = \frac{\tau}{L},\qquad \sinh \chi = \frac{r}{
L},
\end{equation}
which take the line element of (8) into
\begin{equation}
 ds^2 = L^2 a^2(p) \left( dp^2 - d\chi^2 - \sinh^2\chi d\Omega^2\right).
\end{equation}
In this form, all length dimension lies within $L^2$. 
\subsection*{Second Transformation (Alternative)}
To finally bring (10) to the flat form, we make coordinate substitutions
\begin{equation}
T = e^{p}\cosh \chi,\qquad R = e^{p}\sinh \chi. 
\end{equation}
It is convenient to introduce a somewhat 'light-like' coordinate defined by
\begin{equation}
X^2 \equiv T^2 - R^2.
\end{equation}
The coordinate relation for the time coordinate $p(T,R)$ is in fact only a function of $X^2$, viz.
\begin{equation}
e^{2p} = X^2,\qquad p = \frac12 \ln(X^2).
\end{equation}
For the radial coordinate $\chi(T,R)$ we have the relations
\begin{equation}
\sinh \chi = \frac{R}{X},\qquad \cosh \chi = \frac{T}{X}.
\end{equation}
Though not as useful, we may invert (14) to find $\chi(T,R)$ as
\begin{equation}
\chi = \ln \left( \frac{T+R}{X}\right)
\end{equation}
To aid in determining the differentials, we note
\begin{equation}
dX = \frac{\partial X}{\partial T}dT + \frac{\partial X}{\partial R} dR= \frac{TdT - RdR}{X}.
\end{equation}
We first determine $dp$:
\begin{equation}
dp =\frac{ T}{X^2}dT - \frac{R}{X^2}dR.
\end{equation}
To find $d\chi$, we differentiate $\sinh \chi$:
\begin{align}
d(\sinh \chi) = \cosh \chi d\chi &= \frac{dR}{X} -\frac{ R }{X^3}(TdT-RdR)\\
\frac{T}{X}d\chi &= \frac{dR}{X} -\frac{TR}{X^{3}}dT + \frac{R^2}{X^{3}} dR,
\end{align}
hence
\begin{equation}
d\chi =\frac{dR}{T} - \frac{R}{X^2}dT +\frac{R^2}{TX^2} dR.
\end{equation}
After repeated usage of $X^2 = T^2- R^2$, we find the coordinate relation between infinitesimals
\begin{equation}
dp^2-d\chi^2 = \frac{1}{X^2}\left( dT^2 - dR^2\right).
\end{equation}
Finally, with $\sinh^2 \chi = \frac{R^2}{X^2}$, we may write the line element in these new coordinates:
\begin{equation}
ds^2 = L^2\frac{a^2(X)}{X^2} \left( dT^2 - dR^2 - R^2 d\Omega^2\right)
\end{equation}
\section*{Conformal Flat to RW Coordinates}
\subsection*{Conformal Factor}
We note that the conformal factor in the flat $T,R$ coordinates is only a function of $X^2 = T^2-R^2$. The factor is simply
\begin{equation}
\Omega(X)^2 = L^2\frac{a^2(X)}{X^2}
\end{equation}
where 
\begin{equation}
a(X)=a\left(\frac12 \ln(X^2)\right).
\end{equation}
The relation of the conformal factor to the $p$, $\chi$ geometry is simple,
\begin{equation}
\Omega^2(X) \equiv \Omega^2(p,\chi) = L^2 a^2(p)e^{-2p}.
\end{equation}
Interestingly, it is a function entirely of time coordinate $p$. 
We may bring this to the comoving RW form by successive transformations
\begin{equation}
p = \frac{\tau}{L},\qquad \tau = \int \frac{a(t)}{dt},
\end{equation}
in which the conformal factor becomes
\begin{equation}
\Omega^2(X) \equiv \Omega^2(t) = L^2 a^2(t) \exp\left[{-\frac{2}{L^2}\int\frac{dt}{a(t)}}\right]
\end{equation}
\subsection*{Two Step Transformation}
From the relations
\begin{equation}
T = e^{p}\cosh \chi,\qquad R = e^{p}\sinh \chi
\end{equation}
and
\begin{equation}
p = \frac{\tau}{L},\qquad \sinh \chi = \frac{r}{L}
\end{equation}
we see that we could enact a coordinate transformation from conformal time ($\tau$) RW geometry
\begin{equation}
ds^2 = a^2(\tau) \left ( d\tau^2 - \frac{dr^2}{1+r^2/L^2} + r^2 d\Omega^2 \right)
\end{equation}
to conformal to flat (polar) geometry
\begin{equation}
ds^2 = L^2\frac{a^2(X)}{X^2} \left( dT^2 - dR^2 - R^2 d\Omega^2\right)
\end{equation}
via the effective transformation
\begin{equation}
T = \exp\left(\frac{\tau}{L}\right)\left( 1+ \left(\frac{r}{L}\right)^2\right)^{1/2},\qquad R = \exp\left(\frac{\tau}{L}\right)\frac{r}{L},\qquad X^2 \equiv T^2-R^2 = \exp\left(\frac{2\tau}{L}\right)
\end{equation}
\subsection*{One Step Transformation}
Lastly, we may substitute the transformation of $\tau$ viz
\begin{equation}
\tau = \int\frac{dt}{a(t)},
\end{equation}
to finally bring us to comoving coordinates. That is, via coordinate transformation
\begin{equation}
T = \exp\left(\frac{1}{L}\int\frac{dt}{a(t)}\right)\left( 1+ \left(\frac{r}{L}\right)^2\right),\qquad R = \exp\left(\frac{1}{L}\int\frac{dt}{a(t)}\right)\frac{r}{L},\qquad X^2 \equiv T^2-R^2 = \exp\left(\frac{2}{L}\int\frac{dt}{a(t)}\right)
\end{equation}
we may transform from comoving coordinates 
\begin{equation}
ds^2 = dt^2 - a(t)^2 \left(  \frac{dr^2}{1+r^2/L^2} + r^2 d\theta^2 + r^2\sin^2\theta d\phi^2 \right)
\end{equation}
to conformal flat (polar) coordinates
\begin{equation}
ds^2 = L^2\frac{a^2(X)}{X^2} \left( dT^2 - dR^2 - R^2 d\Omega^2\right).
\end{equation}
When $a(t)$ is specified apriori via a dynamics, exponential factors will simplify, especially for a $\tau$ which behaves logarithmically. For example, in the early universe radiation era, we have determined $\tau$ as
\begin{equation}
\tau = L \int_0^t \frac{dt}{(d^2+t^2)^{1/2}} = L\  \text{arcsinh} \left(\frac{t}{d}\right).
\end{equation}
This is equivalent to 
\begin{equation}
\tau =L \ln \left( \frac{t}{d} + \sqrt{\left(\frac{t}{d}\right)^2 + 1}\right)
\end{equation}
in which our exponential calculates to 
\begin{equation}
\exp\left(\frac{1}{L}\int\frac{dt}{a(t)}\right) = \frac{t}{d} + \sqrt{\left(\frac{t}{d}\right)^2 + 1}.
\end{equation}
In the (conformal) early universe then, the conformal factor $\Omega(X)$ goes as
\begin{align}
\Omega^2(X) &= L^2 a^2(t) \exp\left[{-\frac{2}{L^2}\int\frac{dt}{a(t)}}\right] \\
&= (d^2+t^2)\left(\frac{t}{d} + \sqrt{\left(\frac{t}{d}\right)^2 + 1}\right)^{-2}
\end{align}
The flat space coordinate transformations $T$ and $R$ then are specified as
\begin{align}
T = \left(\frac{t}{d} + \sqrt{\left(\frac{t}{d}\right)^2 + 1}\right)\left( 1+ \left(\frac{r}{L}\right)^2\right)^{1/2},\qquad R =\left(\frac{t}{d} + \sqrt{\left(\frac{t}{d}\right)^2 + 1}\right)\frac{r}{L}
\end{align}
\begin{align}
X^2 \equiv T^2-R^2 =\left(\frac{t}{d} + \sqrt{\left(\frac{t}{d}\right)^2 + 1}\right)^2
\end{align}
\begin{equation}
a^2(X) = \frac{d^2}{L^2} \frac{(X^2+1)^2}{4X^2}
\end{equation}
\begin{equation}
\Omega^2(X) = L^2 \frac{a^2(X)}{X^2} = \left[ \frac{d}{2}\left( 1+ \frac{1}{X^2}\right)\right]^2
\end{equation}
\begin{equation}
\Omega(X) = \frac{d}{2}(1+X^{-2})
\end{equation}
\newpage
\section*{Cartesian to Polar}
\subsection*{Transformation Matrices}
\begin{equation}
\renewcommand*{\arraystretch}{1.5}
\begin{pmatrix}
dx\\dy\\dz
\end{pmatrix}
=\begin{pmatrix}\frac{\partial x}{\partial r}&\frac{\partial x}{\partial \theta}&\frac{\partial x}{\partial \phi}\\ \frac{\partial y}{\partial r}&\frac{\partial y}{\partial \theta}&\frac{\partial y}{\partial \phi}\\
\frac{\partial z}{\partial r}&\frac{\partial z}{\partial \theta}&\frac{\partial z}{\partial \phi} \end{pmatrix}
\begin{pmatrix}dr\\d\theta\\d\phi\end{pmatrix}
= \begin{pmatrix}
\sin\theta\cos\phi&  r\cos\theta\cos\phi&-r\sin\theta\sin\phi\\
\sin\theta\sin\phi &r\cos\theta\sin\phi & r\sin\theta\cos\phi \\
\cos\theta&-r\sin\theta&0
\end{pmatrix}
\begin{pmatrix}dr\\d\theta\\d\phi\end{pmatrix}
\end{equation}
\begin{equation}
\renewcommand*{\arraystretch}{1.5}
\begin{pmatrix}dr\\d\theta\\d\phi\end{pmatrix}
=
\begin{pmatrix}
\sin\theta\cos\phi&\sin\theta\sin\phi&\cos\theta \\
\frac{\cos\theta\cos\phi}{r}&\frac{\cos\theta\sin\phi}{r}&-\frac{\sin\theta}{r}\\
-\frac{\sin\phi}{r\sin\theta}&\frac{\cos\phi}{r\sin\theta}&0
\end{pmatrix}
\begin{pmatrix}dx\\dy\\dz\end{pmatrix}
\end{equation}
\subsection*{Time-Time}
\begin{equation}
K'_{00} = K_{00}
\end{equation}
\subsection*{Time-Space}
\begin{equation}
K'_{0i} = \frac{\partial x^j}{\partial x'^i}K_{0j}
\end{equation}
\begin{equation}
\renewcommand*{\arraystretch}{1.5}
\begin{pmatrix}K'_{01}\\ K'_{02}\\K'_{03} \end{pmatrix}
=
\begin{pmatrix}\frac{\partial x^1}{\partial x'^1}&\frac{\partial x^2}{\partial x'^1}&\frac{\partial x^3}{\partial x'^1}\\ \frac{\partial x^1}{\partial x'^2}&\frac{\partial x^2}{\partial x'^2}&\frac{\partial x^3}{\partial x'^2}\\
\frac{\partial x^1}{\partial x'^3}&\frac{\partial x^3}{\partial x'^1}&\frac{\partial x^3}{\partial x'^3} \end{pmatrix}
\begin{pmatrix}K_{01}\\ K_{02}\\ K_{03} \end{pmatrix}
\end{equation}
\begin{equation}
K'_{01} = K_{01} \sin (\theta ) \cos (\phi )+K_{02} \sin (\theta ) \sin (\phi )+K_{03} \cos (\theta )
\end{equation}
\begin{equation}
K'_{02} = K_{01} r \cos (\theta ) \cos (\phi )+K_{02} r \cos (\theta ) \sin (\phi )-K_{03} r \sin (\theta )
\end{equation}
\begin{equation}
K'_{03} =-K_{01} r \sin (\theta ) \sin (\phi )+K_{02} r \sin (\theta ) \cos (\phi )
\end{equation}
\subsection*{Space-Space}
\begin{equation}
K'_{ij} = \frac{\partial x^k}{\partial x'^i}K_{kl}\frac{\partial x^l}{\partial x'^j}
\end{equation}
\begin{equation}
\renewcommand*{\arraystretch}{1.5}
\begin{pmatrix}K'_{11}&K'_{12}&K'_{13}\\K'_{21}&K'_{22}&K'_{23}\\K'_{31}&K'_{32}&K'_{33} \end{pmatrix}=
\begin{pmatrix}\frac{\partial x^1}{\partial x'^1}&\frac{\partial x^2}{\partial x'^1}&\frac{\partial x^3}{\partial x'^1}\\ \frac{\partial x^1}{\partial x'^2}&\frac{\partial x^2}{\partial x'^2}&\frac{\partial x^3}{\partial x'^2}\\
\frac{\partial x^1}{\partial x'^3}&\frac{\partial x^3}{\partial x'^1}&\frac{\partial x^3}{\partial x'^3} \end{pmatrix}
\begin{pmatrix}K_{11}&K_{12}&K_{13}\\K_{21}&K_{22}&K_{23}\\K_{31}&K_{32}&K_{33} \end{pmatrix}
\begin{pmatrix}\frac{\partial x^1}{\partial x'^1}&\frac{\partial x^2}{\partial x'^1}&\frac{\partial x^3}{\partial x'^1}\\ \frac{\partial x^1}{\partial x'^2}&\frac{\partial x^2}{\partial x'^2}&\frac{\partial x^3}{\partial x'^2}\\
\frac{\partial x^1}{\partial x'^3}&\frac{\partial x^3}{\partial x'^1}&\frac{\partial x^3}{\partial x'^3} \end{pmatrix}^T
\end{equation}
\\
Example:
\begin{align}
K'_{11} &= K_{11} \sin ^2(\theta ) \cos ^2(\phi )+K_{12} \sin ^2(\theta ) \sin (2 \phi )+K_{13} \sin (2 \theta ) \cos (\phi )+K_{22} \sin ^2(\theta ) \sin ^2(\phi )\nonumber\\
\quad&+K_{23} \sin (2 \theta ) \sin (\phi )+K_{33} \cos ^2(\theta )
\end{align}
\begin{align}
K'_{22} &= K_{11} r^2 \cos ^2(\theta ) \cos ^2(\phi )+K_{12} r^2 \cos ^2(\theta ) \sin (2 \phi )-K_{13} r^2 \sin (2 \theta ) \cos (\phi )+K_{22} r^2 \cos ^2(\theta ) \sin ^2(\phi )\nonumber\\
&\quad -K_{23} r^2 \sin (2 \theta ) \sin (\phi )+K_{33} r^2 \sin ^2(\theta )
\end{align}
\begin{align}
K'_{33} &=K_{11} r^2 \sin ^2(\theta ) \sin ^2(\phi )-2 K_{12} r^2 \sin ^2(\theta ) \sin (\phi ) \cos (\phi )+K_{22} r^2 \sin ^2(\theta ) \cos ^2(\phi )
\end{align}
\begin{align}
K'_{12} &=K_{11} r \sin (\theta ) \cos (\theta ) \cos ^2(\phi )+K_{12} r \sin (\theta ) \cos (\theta ) \sin (2 \phi )+K_{13} r \cos (2 \theta ) \cos (\phi )+K_{22} r \sin (\theta ) \cos (\theta ) \sin ^2(\phi )\nonumber\\
&\quad+K_{23} r \cos (2 \theta ) \sin (\phi )-K_{33} r \sin (\theta ) \cos (\theta )
\end{align}
\begin{align}
K'_{13} &=-K_{11} r \sin ^2(\theta ) \sin (\phi ) \cos (\phi )+K_{12} r \sin ^2(\theta ) \cos (2 \phi )-K_{13} r \sin (\theta ) \cos (\theta ) \sin (\phi )+K_{22} r \sin ^2(\theta ) \sin (\phi ) \cos (\phi )\nonumber\\
&\quad+K_{23} r \sin (\theta ) \cos (\theta ) \cos (\phi )
\end{align}
\begin{align}
K'_{23} &=-K_{11} r^2 \sin (\theta ) \cos (\theta ) \sin (\phi ) \cos (\phi )+K_{12} r^2 \sin (\theta ) \cos (\theta ) \cos (2 \phi )+K_{13} r^2 \sin ^2(\theta ) \sin (\phi )+K_{22} r^2 \sin (\theta ) \cos (\theta ) \sin (\phi ) \cos (\phi )\nonumber\\
&\quad-K_{23} r^2 \sin ^2(\theta ) \cos (\phi )
\end{align}
\section*{Early Universe (Radiation Era)}
In the conformal to Minkowski coordinate system of
\begin{equation}
ds^2 = \Omega^2(X^2)( dT^2 - dx^2 - dy^2 - dz^2),\qquad\qquad X^2 = T^2-(x^2+y^2+z^2)
\end{equation}
when we impose the transverse gauge,
solutions to conformal gravity $\delta W_{\mu\nu} = 0$ are found to obey
\begin{equation}
\frac{1}{2}\Omega^{-2}\Box^2 k_{\mu\nu} = 0
\end{equation}
where $\Omega^2 k_{\mu\nu} = K_{\mu\nu}$. Upon performing residual gauge transformation to eliminate gauge degrees of freedom, the general solution to (64) for a given $k$-mode is then
\begin{equation}
k_{\mu\nu} = 
 \begin{pmatrix}0&0&0&0\\0&A_{11}&A_{12}&0\\0&A_{12}&-A_{11}&0\\0&0&0&0\end{pmatrix}e^{ikx} + \begin{pmatrix}
0&B_{01}&B_{02}&0\\B_{01}&B_{11}&B_{12}&0\\B_{02}&B_{12}&-B_{11}&0\\0&0&0&0  \end{pmatrix}T e^{ikx} 
\end{equation}
Since we will soon find that $T \sim t$, with $t$ the comoving time, we see the $B_{\mu\nu}$ are leading order in $t$. Hence for early universe fluctuations, the leading $t$ order solution to $K_{\mu\nu}$ will be
\begin{equation}
K_{\mu\nu} \sim \Omega^2 TB_{\mu\nu}e^{ikx},
\end{equation}
where 
\begin{equation}
B_{22}= -B_{11},\qquad B_{0\mu} = B_{33} = 0
\end{equation}
\subsection*{Conformal Minkoski to Polar RW Comoving}
In going from the geometry of 
\begin{equation}
ds^2 = \Omega^2(dT^2 - dx^2-dy^2-dz^2)
\end{equation}
to
\begin{equation}
ds^2 = \Omega^2 (dT^2-dR^2 - R^2 d\Omega^2),
\end{equation}
we utilize the Cartesian to polar conversions given in the Appendix. Denoting the polar coordinate system as $x^{(P)}$, we find, after imposing the transverse and residual relations, the following:
\begin{align}
K^{(P)}_{00} &= 0\nonumber\\
K^{(P)}_{01} &= K_{01} \sin (\theta ) \cos (\phi )+K_{02} \sin (\theta ) \sin (\phi )\nonumber\\
K^{(P)}_{02} &= K_{01} r \cos (\theta ) \cos (\phi )+K_{02} r \cos (\theta ) \sin (\phi )\nonumber\\
K^{(P)}_{03} &= -K_{01} r \sin (\theta ) \sin (\phi )+K_{02} r \sin (\theta ) \cos (\phi )\nonumber\\
K^{(P)}_{11} &= K_{11} \sin ^2(\theta ) \cos (2 \phi )+K_{12} \sin ^2(\theta ) \sin (2 \phi )\nonumber\\
K^{(P)}_{22} &= K_{11} r^2 \cos ^2(\theta ) \cos (2 \phi )+K_{12} r^2 \cos ^2(\theta ) \sin (2 \phi )\nonumber\\
K^{(P)}_{33} &= -K_{11} r^2 \sin ^2(\theta ) \cos (2 \phi )-2 K_{12} r^2 \sin ^2(\theta ) \sin (\phi ) \cos (\phi )\nonumber\\
K^{(P)}_{12} &= \frac{1}{2} K_{11} r \sin (2 \theta ) \cos (2 \phi )+K_{12} r \sin (\theta ) \cos (\theta ) \sin (2 \phi )\nonumber\\
K^{(P)}_{13} &=-2 K_{11} r \sin ^2(\theta ) \sin (\phi ) \cos (\phi )+ K_{12} r \sin ^2(\theta ) \cos (2 \phi )\nonumber\\
K^{(P)}_{23} &= -2 K_{11} r^2 \sin (\theta ) \cos (\theta ) \sin (\phi ) \cos (\phi )+K_{12} r^2 \sin (\theta ) \cos (\theta ) \cos (2 \phi )
\end{align}

\begin{equation}
K'_{\mu\nu}(t,r,\theta,\phi) = \frac{\partial x^\alpha}{\partial x'^\mu}\frac{\partial x^\beta}{\partial x'^\nu} K_{\alpha\beta}(T,R,\theta,\phi)
\end{equation}
\begin{equation}
J_{\mu\nu} = \frac{\partial x^\nu}{\partial x'^\mu},\qquad\text{where}\quad x(T,R,\theta,\phi)\quad x'(t,r,\theta,\phi)
\end{equation}
\begin{equation}
\renewcommand*{\arraystretch}{1.5}
J_{\mu\nu} = 
\begin{pmatrix}
\frac{\partial T}{\partial t}&\frac{\partial R}{\partial t}&0&0\\
\frac{\partial T}{\partial r}&\frac{\partial R}{\partial r}&0&0\\
0&0&1&0\\
0&0&0&1
\end{pmatrix}
\end{equation}
\begin{equation}
K^{(cm)}_{\mu\nu} = \frac{\partial x_{(P)}^k}{\partial x_{(cm)}^i}K^{(P)}_{kl}\frac{\partial x_{(P)}^l}{\partial x_{(cm)}^j}
\end{equation}
\begin{equation}
\renewcommand*{\arraystretch}{1.5}
\begin{pmatrix}
K^{(cm)}_{00}&K^{(cm)}_{01}&K^{(cm)}_{02}&K^{(cm)}_{03}\\
K^{(cm)}_{10}&K^{(cm)}_{11}&K^{(cm)}_{12}&K^{(cm)}_{13}\\
K^{(cm)}_{20}&K^{(cm)}_{21}&K^{(cm)}_{22}&K^{(cm)}_{23}\\
K^{(cm)}_{30}&K^{(cm)}_{31}&K^{(cm)}_{32}&K^{(cm)}_{33} \end{pmatrix}
=
\begin{pmatrix}
\frac{\partial T}{\partial t}&\frac{\partial R}{\partial t}&0&0\\
\frac{\partial T}{\partial r}&\frac{\partial R}{\partial r}&0&0\\
0&0&1&0\\
0&0&0&1
\end{pmatrix}
\begin{pmatrix}
K^{(P)}_{00}&K^{(P)}_{01}&K^{(P)}_{02}&K^{(P)}_{03}\\
K^{(P)}_{10}&K^{(P)}_{11}&K^{(P)}_{12}&K^{(P)}_{13}\\
K^{(P)}_{20}&K^{(P)}_{21}&K^{(P)}_{22}&K^{(P)}_{23}\\
K^{(P)}_{30}&K^{(P)}_{31}&K^{(P)}_{32}&K^{(P)}_{33} 
\end{pmatrix}
\begin{pmatrix}
\frac{\partial T}{\partial t}&\frac{\partial R}{\partial t}&0&0\\
\frac{\partial T}{\partial r}&\frac{\partial R}{\partial r}&0&0\\
0&0&1&0\\
0&0&0&1
\end{pmatrix}^T
\end{equation}
\begin{align}
K^{(cm)}_{00} &= \left(\frac{\partial T}{\partial t}\right)^2 K^{(P)}_{00} + 2 \frac{\partial T}{\partial t} \frac{\partial R}{\partial t} K^{(P)}_{01} + \left(\frac{\partial R}{\partial t}\right)^2 K^{(P)}_{11}\nonumber\\
K^{(cm)}_{01} &= \frac{\partial T}{\partial t} \frac{\partial T}{\partial r} K^{(P)}_{00}+\frac{\partial T}{\partial t} \frac{\partial R}{\partial r} K^{(P)}_{01}+\frac{\partial R}{\partial t} \frac{\partial T}{\partial r} K^{(P)}_{01}+\frac{\partial R}{\partial t} \frac{\partial R}{\partial r} K^{(P)}_{11} \nonumber\\
K^{(cm)}_{02} &= \frac{\partial T}{\partial t} K^{(P)}_{02}+\frac{\partial R}{\partial t} K^{(P)}_{12} \nonumber\\
K^{(cm)}_{03} &= \frac{\partial T}{\partial t} K^{(P)}_{03}+\frac{\partial R}{\partial t} K^{(P)}_{13} \nonumber\\
K^{(cm)}_{11} &= \left(\frac{\partial T}{\partial r}\right)^2 K^{(P)}_{00}+2 \frac{\partial T}{\partial r} \frac{\partial R}{\partial r} K^{(P)}_{01}+\left(\frac{\partial R}{\partial r}\right)^2 K^{(P)}_{11} \nonumber\\
K^{(cm)}_{22} &= K^{(P)}_{22}\nonumber\\
K^{(cm)}_{33} &= K^{(P)}_{33}\nonumber\\
K^{(cm)}_{12} &= \frac{\partial T}{\partial r} K^{(P)}_{02}+\frac{\partial R}{\partial r} K^{(P)}_{12} \nonumber\\
K^{(cm)}_{13} &= \frac{\partial T}{\partial r} K^{(P)}_{03}+\frac{\partial R}{\partial r} K^{(P)}_{13} \nonumber\\
K^{(cm)}_{23} &= K^{(P)}_{23}\nonumber\\
\end{align}

\begin{equation}
\frac{\partial T}{\partial t}=
\end{equation}
\begin{equation}
\frac{\partial R}{\partial t}=
\end{equation}
\begin{equation}
\frac{\partial T}{\partial r}=
\end{equation}
\begin{equation}
\frac{\partial R}{\partial r}= \frac1L \left( \frac{t}{d} + \sqrt{1+ \frac{t^2}{d^2}}\right)
\end{equation}

\end{document}