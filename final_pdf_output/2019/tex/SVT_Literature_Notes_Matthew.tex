\documentclass[10pt,letterpaper]{article}
\usepackage[textwidth=7in, top=1in,textheight=9in]{geometry}
\usepackage[fleqn]{mathtools} 
\usepackage{amssymb,braket,hyperref,xcolor}
\hypersetup{colorlinks, linkcolor={blue!50!black}, citecolor={red!50!black}, urlcolor={blue!80!black}}
\usepackage[title]{appendix}
\usepackage[sorting=none]{biblatex}
\numberwithin{equation}{section}
\setlength{\parindent}{0pt}
\title{SVT in Literature}
\date{}
\begin{document} 
\maketitle
\noindent 
%%%%%%%%%%%%%%%%%%%%%%%%%%%%%%%%%%
\subsection{Weinberg: Cosmology (2008)}
 Uses specific gauge choices, harmonic eigenfuction decomposition, RW metric directly (not conformal to flat)

\subsection{Ellis, Maartens: Relativistic Cosmology (2012)}
SVT formalism is closest to matching APM. Uses too many extra gauge invariant combinations. 
\\ \\
For scalar solutions, see (10.76) and (10.8.1) for $\nu =1$. 
\\ \\
For vector and tensor equations, see (10.89)-(10.91)

\subsection{Bardeen: Gauge Invariant Cosmological Perturbations (1980)}
Uses harmonic decomposition. 
\\ \\
For scalar equations, see (4.3), (4.4), and (4.9)
\\ \\
For vectors and tensor, see (4.12)-(4.14)

\subsection{Kodama, Sasaki: Cosmological Perturbation Theory (1984)}
Uses harmonic decomposition. Large and confusing number of gauge invariant quantities.
\\ \\
For scalar equations, see (4.2a-c)
\\ \\
For vectors see (4.10a) and (4.10b)
\\ \\
For tensors, see (4.15)
 and tensor, see (4.12)-(4.14)
 
\subsection{Maggiore: Gravitational Waves (2018)}
Formalism similar to Ellis, close to APM. 
\\ \\
For scalars see (19.2)-(19.4) and "master" eq. (19.26). For analytic radiation solution, see (19.127) (with initial conditions)
\\ \\
For tensors, see (19.213). For analytic radiation solution with initial conditions, see (19.232)
%%%%%%%%%%%%%%%%%%%%%%%%%%%%%%%%%%
\end{document}