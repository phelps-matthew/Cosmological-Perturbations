\documentclass[10pt,letterpaper]{article}
\usepackage[textwidth=7in, top=1in,textheight=9in]{geometry}
\usepackage[fleqn]{mathtools} 
\usepackage{amssymb}
 \usepackage{braket}
\newcommand{\vect}[1]{\mathbf{#1}}
\newcommand{\vecth}[1]{\hat{\mathbf{#1}}}
%\numberwithin{equation}{subsection}
\title{Coordinate Transformations RW $k<0$ v8}
\date{}
\begin{document} 
\maketitle
\noindent 
%%%%%%%%%%%%%%%%%%%%%%%%%%%%%%%%%
\section*{Summary}
Since planes waves in the $(p',r')$ coordinate system behave much differently than those in the $(T,R)$ coordinate system, their asymptotic behavior may vary. After accounting for the leading $u$ behavior in the expansion of $e^{ik(R\cos\theta-T)}$ in the null and timelike configurations, all results are found to agree between both coordinate systems. However, there still seems to remain a mismatch with PRD 2012. This is summarized in the Plane Wave Expansion section. Null and timelike calculations are summarized below.
\\ \\
\begin{minipage}[t]{0.45\textwidth}
\textbf{Null}
\begin{align}
K^{(cm)}_{00} &\sim u^2 \nonumber\\
K^{(cm)}_{01} &\sim  u^2\nonumber\\
K^{(cm)}_{02} &\sim  u^3\nonumber\\
K^{(cm)}_{03} &\sim   u^3\nonumber\\
K^{(cm)}_{11} &\sim  u^2\nonumber\\
K^{(cm)}_{22} &\sim  u^4\nonumber\\
K^{(cm)}_{33} &\sim  u^4\nonumber\\
K^{(cm)}_{12} &\sim u^3\nonumber\\
K^{(cm)}_{13} &\sim u^3\nonumber\\
K^{(cm)}_{23} &\sim u^4
\end{align}
\end{minipage}
\begin{minipage}[t]{0.45\textwidth}
\textbf{Timelike}
\begin{align}
K^{(cm)}_{00} &\sim 1 \nonumber\\
K^{(cm)}_{01} &\sim  u\nonumber\\
K^{(cm)}_{02} &\sim  u\nonumber\\
K^{(cm)}_{03} &\sim   u\nonumber\\
K^{(cm)}_{11} &\sim  u^2\nonumber\\
K^{(cm)}_{22} &\sim  u^2\nonumber\\
K^{(cm)}_{33} &\sim  u^2\nonumber\\
K^{(cm)}_{12} &\sim u^2\nonumber\\
K^{(cm)}_{13} &\sim u^2\nonumber\\
K^{(cm)}_{23} &\sim u^2
\end{align}
\end{minipage}
%%%%%%%%%%%%%%%%%%%%%%%%%%%%%%%%%%
\section*{Notation/Background}
For $K<1$ FRW cosmology with $L^2 a^2 = t^2+d^2$, the line element takes the form
\begin{align}
ds^2 &=  dt^2 - a(t)^2 \left(  \frac{dr^2}{1+r^2/L^2} + r^2 d\theta^2 + r^2\sin^2\theta d\phi^2 \right)\nonumber\\
&= d^2 \left[ du^2 - (1+u^2)\left( \frac{dv^2}{1+v^2} + v^2 d\Omega^2\right)\right] ,
\end{align}
where we have introduced
\begin{equation}
u = \frac{t}{d},\qquad v = \frac{r}{L}.
\end{equation}
%%%%%%%%%%%%%%%%%%%%%%%%%%%%%%%%%%%
\section*{Original Coordinates $(p',r')$}
Transformations and Asymptotics:
\begin{equation}
p' = \frac{u}{(1+u^2)^{1/2}+(1+v^2)^{1/2}},\qquad r' = \frac{v}{(1+u^2)^{1/2}+(1+v^2)^{1/2}}
\end{equation}
\begin{equation}
u^2 = \frac{4 p'^2}{(1-(p'+r')^2)(1-(p'-r')^2)},\qquad v =\left(\frac{r'}{p'}\right)u
\end{equation}
\begin{equation}
\Omega^2(p',r') = \frac{4 L^2 a^2}{(1-(p'+r')^2)(1-(p'-r')^2)} = d^2(1+u^2)\left[ (1+u^2)^{1/2}+(1+v^2)^{1/2}\right]^2
\end{equation}
\begin{equation}
r'\cos\theta-p' = \frac{v\cos\theta -u}{\sqrt{1+u^2}+\sqrt{1+v^2}}
\end{equation}
%%%%%%%%%%%%%%%%%%
\subsection*{Null Trajectory}
In the $u$, $v$ geometry, the condition for null separation (at fixed angle) is $u=v$. Inspection of coordinate transformation (5-8) shows the leading order ($u\gg 1$) contributions for null separation:
\begin{equation}
p' \sim 1,\qquad r' \sim 1,\qquad \Omega^2 \sim u^4.
\end{equation}
\begin{align}
\frac{\partial p'}{\partial t} & \sim  \frac{1}{u}\qquad
\frac{\partial p'}{\partial r}  \sim 	\frac{1}{u},\qquad
\frac{\partial r'}{\partial t}  \sim \frac{1}{u}\qquad
\frac{\partial r'}{\partial r}  \sim  \frac{1}{u}.
\end{align}
\begin{equation}
\exp[{ik(r'\cos\theta-p')} ]\sim 1
\end{equation}
The leading behavior for the full $K_{\mu\nu}^{(cm)}$ behaves as
\begin{align}
K^{(cm)}_{00} &\sim u^2 \nonumber\\
K^{(cm)}_{01} &\sim  u^2\nonumber\\
K^{(cm)}_{02} &\sim  u^3\nonumber\\
K^{(cm)}_{03} &\sim   u^3\nonumber\\
K^{(cm)}_{11} &\sim  u^2\nonumber\\
K^{(cm)}_{22} &\sim  u^4\nonumber\\
K^{(cm)}_{33} &\sim  u^4\nonumber\\
K^{(cm)}_{12} &\sim u^3\nonumber\\
K^{(cm)}_{13} &\sim u^3\nonumber\\
K^{(cm)}_{23} &\sim u^4
\end{align}

The purely angular sector of this result coincides with the null configuration given in PRD 2012.
%%%%%%%%%%%%%%%%%
\subsection*{Timelike Trajectory}
For coordinate separations which are timelike, $u\gg v$. In order to find the leading contribution in $u$, we take both $u\gg v$ and $u\gg1$. These results yield a leading behavior of:
\begin{equation}
p'\sim 1,\qquad r'\sim \frac{1}{u},\qquad \Omega^2\sim u^4. 
\end{equation}
\begin{align}
\frac{\partial p'}{\partial t} & \sim  \frac{1}{u^2}\qquad
\frac{\partial p'}{\partial r}  \sim 	\frac{1}{u},\qquad
\frac{\partial r'}{\partial t}  \sim \frac{1}{u^2}\qquad
\frac{\partial r'}{\partial r}  \sim  \frac{1}{u}.
\end{align}
\begin{equation}
\exp[{ik(r'\cos\theta-p')} ]\sim 1
\end{equation}
The leading behavior for the full $K_{\mu\nu}^{(cm)}$ behaves as
\begin{align}
K^{(cm)}_{00} &\sim 1 \nonumber\\
K^{(cm)}_{01} &\sim  u\nonumber\\
K^{(cm)}_{02} &\sim  u\nonumber\\
K^{(cm)}_{03} &\sim   u\nonumber\\
K^{(cm)}_{11} &\sim  u^2\nonumber\\
K^{(cm)}_{22} &\sim  u^2\nonumber\\
K^{(cm)}_{33} &\sim  u^2\nonumber\\
K^{(cm)}_{12} &\sim u^2\nonumber\\
K^{(cm)}_{13} &\sim u^2\nonumber\\
K^{(cm)}_{23} &\sim u^2
\end{align}
%%%%%%%%%%%%%%%%%%%%%%%%%%%%%%%
\section*{New Coordinates $(T,R)$}
Transformations and Asymptotics:
\begin{align}
T = \left[u+(1+u^2)^{1/2}\right]( 1+v^2)^{1/2},\qquad R = \left[u+(1+u^2)^{1/2}\right]v
\end{align}
\begin{equation}
\Omega^2(T,R) = \frac{L^2 a^2}{T^2-R^2} = d^2\frac{(1+u^2)}{(u+(1+u^2)^{1/2})^2}
\end{equation}
\begin{equation}
R\cos\theta - T = (u+\sqrt{1+u^2})(v\cos\theta - \sqrt{1+v^2}) \sim 2u(v\cos\theta-\sqrt{1+v^2})
\end{equation}
%%%%%%%%%%%%%%%%%%
\subsection*{Null Trajectory}
In the $u$, $v$ geometry, the condition for null separation (at fixed angle) is $u=v$. Inspection of coordinate transformation (17-19) shows the leading order ($u\gg 1$) contributions for null separation:
\begin{equation}
T \sim u^2,\qquad R \sim u^2,\qquad \Omega^2 \sim 1. 
\end{equation}
\begin{align}
\frac{\partial T}{\partial t} & \sim  u,\qquad
\frac{\partial T}{\partial r}  \sim 	u,\qquad
\frac{\partial R}{\partial t}  \sim u,\qquad
\frac{\partial R}{\partial r}  \sim  u.
\end{align}
\begin{equation}
\exp[R\cos\theta-T] \sim \frac{1}{u^2}
\end{equation}
The leading behavior for the full $K_{\mu\nu}^{(cm)}$ behaves as
\begin{align}
K^{(cm)}_{00} &\sim u^2 \nonumber\\
K^{(cm)}_{01} &\sim  u^2\nonumber\\
K^{(cm)}_{02} &\sim  u^3\nonumber\\
K^{(cm)}_{03} &\sim   u^3\nonumber\\
K^{(cm)}_{11} &\sim  u^2\nonumber\\
K^{(cm)}_{22} &\sim  u^4\nonumber\\
K^{(cm)}_{33} &\sim  u^4\nonumber\\
K^{(cm)}_{12} &\sim u^3\nonumber\\
K^{(cm)}_{13} &\sim u^3\nonumber\\
K^{(cm)}_{23} &\sim u^4
\end{align}

The purely angular sector of this result coincides with the null configuration given in PRD 2012.
%%%%%%%%%%%%%%%%%%%%
\subsection*{Timelike Trajectory}
For coordinate separations which are timelike, $u\gg v$. In order to find the leading contribution in $u$, we take both $u\gg v$ and $u\gg1$. These results yield a leading behavior of:
\begin{equation}
T\sim u,\qquad R\sim u,\qquad \Omega^2\sim 1. 
\end{equation}
\begin{align}
\frac{\partial T}{\partial t} & \sim  1,\qquad
\frac{\partial T}{\partial r}  \sim  u,\qquad
\frac{\partial R}{\partial t}  \sim 1,\qquad
\frac{\partial R}{\partial r}  \sim  u.
\end{align}
\begin{equation}
\exp[R\cos\theta-T] \sim \frac{1}{u}
\end{equation}
The leading behavior for the full $K_{\mu\nu}^{(cm)}$ behaves as
\begin{align}
K^{(cm)}_{00} &\sim 1 \nonumber\\
K^{(cm)}_{01} &\sim  u\nonumber\\
K^{(cm)}_{02} &\sim  u\nonumber\\
K^{(cm)}_{03} &\sim   u\nonumber\\
K^{(cm)}_{11} &\sim  u^2\nonumber\\
K^{(cm)}_{22} &\sim  u^2\nonumber\\
K^{(cm)}_{33} &\sim  u^2\nonumber\\
K^{(cm)}_{12} &\sim u^2\nonumber\\
K^{(cm)}_{13} &\sim u^2\nonumber\\
K^{(cm)}_{23} &\sim u^2
\end{align}
%%%%%%%%%%%%%%%%%%%%%%%%%%%%%%%%%
\section*{Plane-Wave Expansions}
The plane wave $e^{ik(Z-T)}$ behaves much differently from $e^{ik(z'-p')}$. In order to extract the asymptotic behavior as $u\to \infty$, we utilize the plane wave expansion in terms of Spherical Bessels, viz (30). 
%%%%%%%%%%%%%%%
\subsection*{Spherical Bessels}
\begin{equation}
e^{ikr\cos\theta} = \sum_{n=0}^\infty (2n+1)i^n j_n(kr)P_n(\cos\theta)
\end{equation}
where $j_n$ are the spherical Bessels and $P_n$ are the Legendre polynomials. The asymptotic form for $j_n(z)$ is (NIST 10.52.3)
\begin{equation}
j_n(z) = z^{-1}\sin(z-\frac12 n\pi)+e^{\mathbb I(z)}\mathcal O(z^{-2}).
\end{equation}
Hence for $r\to\infty$, we have
\begin{equation}
e^{ikz} \to \sum_{n=0}^{\infty}(2n+1)i^nP_n(\cos\theta)\left( \frac{\sin(kr-\frac12 n\pi)}{kr}\right)\sim \frac{1}{kr}
\end{equation}
since the Legendre polynomials and $\sin$ function are bounded between $[-1,1]$. 
%%%%%%%%%%%%%%%%
\subsection*{Null}
Here we express the phase in terms of $u$ and $v$, set $u=v$ for the null condition, and find the asymptotic behavior for $u\gg 1$:
\begin{equation}
r'\cos\theta-p' = \frac{1}{2}\left( \frac{u(\cos\theta-1)}{\sqrt{1+u^2}}\right) \sim \frac{1}{2}(\cos\theta-1)
\end{equation}
\begin{equation}
\implies \exp[{ik(r'\cos\theta-p')} ]\approx\exp[ik(\tfrac12\cos\theta -\tfrac12)] \sim 1
\end{equation}
\\ 
\begin{equation}
R\cos\theta-T = (u+\sqrt{1+u^2})(u\cos\theta-\sqrt{1+u^2})\sim 2u^2(\cos\theta-1)
\end{equation}
\begin{align}
\implies \exp[{ik(R\cos\theta-T)} ]&\approx \exp[-2ik u^2]\exp[2ik u^2\cos\theta]\\
&\sim \exp[-2ik u^2] 
 \sum_{n=0}^{\infty}(2n+1)i^nP_n(\cos\theta)\left( \frac{\sin(2ku^2-\frac12 n\pi)}{2ku^2}\right)\\
&\sim \frac{1}{u^2}
\end{align}
%%%%%%%%%%%%%%%
\subsection*{Timelike}
Here we express the phase in terms of $u$ and $v$, hold $v$ finite, and take $u\gg 1$ and $u\gg v$:
\begin{equation}
r'\cos\theta-p' = \frac{v\cos\theta -u}{\sqrt{1+u^2}+\sqrt{1+v^2}}\sim \frac{v\cos\theta-u}{u}\sim -1
\end{equation}
\begin{equation}
\implies \exp[ik(r'\cos\theta -p')] \approx \exp[-ik] \sim 1
\end{equation}
\\
\begin{equation}
R\cos\theta - T = (u+\sqrt{1+u^2})(v\cos\theta - \sqrt{1+v^2}) \sim 2u(v\cos\theta-\sqrt{1+v^2})
\end{equation}
\begin{align}
\implies \exp[R\cos\theta-T]&\approx \exp[ -2iku\sqrt{1+v^2}]\exp[2ik uv\cos\theta]\\
&\sim  \exp[-2ik u\sqrt{1+v^2}] 
 \sum_{n=0}^{\infty}(2n+1)i^nP_n(\cos\theta)\left( \frac{\sin(2kuv-\frac12 n\pi)}{2kuv}\right)\\
&\sim \frac{1}{u}
\end{align}
\subsection*{PRD 2012}
\textbf{Null}:\\
$u=v$ implies $p'=r'$ and thus
\begin{equation}
\Omega^2 p'r' e^{ik(r'-p')} = \Omega^2 p'r' \sim u^4
\end{equation}
\\
\textbf{Timelike}:\\
In the timelike configuration, $r'\sim \frac1u$, and so large $u$ implies small $r'$. Thus we have
\begin{equation}
\Omega^2 p'r' e^{ikr'}e^{-ikp'} \sim u^4  e^{-ikp'} r'^2 \left( \frac{e^{ikr'}}{r'}\right ).
\end{equation}
We note that as $r'\to 0$,
\begin{equation}
 \frac{e^{ikr'}}{r'} \sim \frac{1}{r'},
\end{equation}
and thus it would seem 
\begin{equation}
\Omega^2 p'r' e^{ik(r'-p')}\approx \Omega^2 p' r'\sim u^3.
\end{equation}
Asymptotically, $e^{ikr'\cos\theta}$ behaves as $\sin(kr')/r'$. However, since $r'$ is not small, we cannot use the asymptotic form of the spherical Bessels, so it not clear for me how to arrive at $\sim u^2$ in PRD.
\end{document}