\documentclass[10pt,letterpaper]{article}
\usepackage[textwidth=7in, top=1in,textheight=9in]{geometry}
\usepackage[fleqn]{mathtools} 
\usepackage{amssymb}

\title{RW Projections}
\date{}
\begin{document}
\maketitle
\noindent 
\section*{Projection Method}
Via the 3+1 splitting we may express a general $T_{\mu\nu}$ as
\begin{equation}
T_{\mu\nu} = (\rho+p)U_\nu U_\mu + pg_{\mu\nu} +U_\mu q_\nu + U_\nu q_\mu + \pi_{\mu\nu}.
\end{equation}
Then we may deconstruct the curved space $q_i$ as $q_i = Q_i + \nabla_i Q$, a procedure indicated below for a rank 1 tensor. Then, it only remains to decompose the spatial (traceless) $\pi_{ij}$ in terms of curved space projectors. Since the RW background is maximally symmetric in the underyling 3-space, it suggests the possibility to construct longitudinal and transverse components based solely on the 3-space covariant derivatives. More precisely, spatial covariant derivatives $\tilde\nabla_i$ are defined solely with respect to the 3-space constant curvature background, $\gamma_{ij}$. In polar coordinates for example, this would be
\begin{equation}
\gamma_{ij}dx^i dx^j = \frac{dr^2}{1-kr^2}+r^2d\Omega^2.
\end{equation}
Upon conformally transforming with a time dependent $\Omega(t)$, the transverse and longitudinal components preserve their decomposed structure. Below is an attempt at constructing projectors within a maximally symmetric space, first by looking at rank 1 tensors, then flat space rank 2 tensors, then finally constant curvature rank 2 tensors.
\section*{Longitudinal Decomposition}
First we posit the form of the longitudinal component, project out any transverse components, then integrate to solve in terms of the original tensor. 
\\ 
\subsection*{Rank 1 Tensor In Curved Space}
For a rank 1 tensor $A_\nu$ we express the longitudinal component in terms of a derivative onto a scalar $A$
\begin{equation}
A_\nu^L = \nabla_\nu A.
\end{equation}
Now project out the transverse component, 
\begin{equation}
\nabla^\nu A_\nu = \nabla_\nu \nabla^\nu A.
\end{equation}
Solving for $A$, we have
\begin{equation}
A = \int d^Dx'\sqrt{g}\  D(x,x') \nabla^\mu A_\mu,
\end{equation}
where we have introduced the curved space propogator
\begin{equation}
\nabla_\mu \nabla^\mu D(x,x') = g^{-1/2} \delta^D(x-x').
\end{equation}
Thus we have
\begin{equation}
A_\nu^L = \nabla_\nu \int d^Dx'\sqrt{g}\ D(x,x') \nabla^\mu A_\mu,
\end{equation}
and the transverse component is just the remaining part,
\begin{equation}
A_\nu^T = A_\nu - A_\nu^L.
\end{equation}
Lastly, we may construct a longitudinal projector $\Pi_{\mu\nu}^L$,
\begin{equation}
\Pi_{\mu\nu}^L = \nabla_\mu \int d^Dx'\sqrt{g}\  D(x,x') \nabla_\nu
\end{equation}
\subsection*{Rank 2 Tensor In Minkowski Space}
For a rank 2 tensor (in Minkowski background), we posit
\begin{equation}
h^L_{\mu\nu} = \partial_\mu V_\nu + \partial_\nu V_\mu,
\end{equation}
where $V^{\mu}$ remains to be determined in terms of $h^{\mu\nu}$.
Now project out the transverse components of $h^{\mu\nu}$, noting $h^{\mu\nu}_T$ can make no contribution,
\begin{equation}
\partial_\nu h^{\mu\nu} = \partial_\nu \partial^\mu V^\nu + \partial_\nu \partial^\nu V^\mu,
\end{equation}
\begin{equation}
\partial_\mu \partial_\nu h^{\mu\nu} = \partial_\mu\partial_\nu \partial^\mu V^\nu + \partial_\mu\partial_\nu \partial^\nu V^\mu
= 2 \partial_\mu \partial^\mu \partial_\nu V^\nu.
\end{equation}
From $\partial_\mu\partial_\nu h^{\mu\nu}$, solve for $\partial_\nu V^\nu$,
\begin{equation}
\partial_\nu V^\nu = \frac12 \int d^3y D(x-y) \partial_\sigma\partial_\rho h^{\sigma\rho}
= \frac12\left[ \partial_\sigma \int d^3y D(x-y)\partial_\rho h^{\sigma\rho}
+ \int dS_\sigma D(x-y) \partial_\rho h^{\sigma\rho}\right],
\end{equation}
where we use the flat space propagator
\begin{equation}
\partial_\nu \partial^\nu D(x-x') = \delta(x-x').
\end{equation}
Insert $\partial_\nu V^\nu$ back into $\partial_\nu h^{\mu\nu}$
\begin{equation}
\partial_\nu h^{\mu\nu} =  \frac12 \partial^\mu\left[ \partial_\sigma \int d^3y D(x-y)\partial_\rho h^{\sigma\rho}
+ \int dS_\sigma D(x-y) \partial_\rho h^{\sigma\rho}\right] + \partial_\nu \partial^\nu V^\mu,
\end{equation}
and solve for $V^\mu$,
\begin{align}
V^\mu &= \int d^3y D(x-y)\partial_\sigma h^{\sigma\mu} -  \frac12\int d^3y D(x-y) \partial^\mu \left[ \partial_\sigma \int d^3z D(y-z)\partial_\rho h^{\sigma\rho}
+ \int dS_\sigma D(y-z) \partial_\rho h^{\sigma\rho}\right]\\
&=  \int d^3y D(x-y)\partial_\sigma h^{\sigma\mu} -  \frac12
\partial^\mu \int d^3y D(x-y)\left[ \partial_\sigma \int d^3z D(y-z)\partial_\rho h^{\sigma\rho}
+ \int dS_\sigma D(y-z) \partial_\rho h^{\sigma\rho}\right]\\
&\qquad
-\frac12 \int dS^\mu D(x-y)\left[ \partial_\sigma \int d^3z D(y-z)\partial_\rho h^{\sigma\rho}
+ \int dS_\sigma D(y-z) \partial_\rho h^{\sigma\rho}\right].
\end{align}
Dropping surface terms, $V^\mu$ takes the form
\begin{align}
V^\mu &= \int d^3y D(x-y)\partial_\sigma h^{\sigma\mu} -  \frac12
\partial^\mu \int d^3y D(x-y) \partial_\sigma \int d^3z D(y-z)\partial_\rho h^{\sigma\rho}.
\end{align}
Now using $V^\mu$, we can  construct $h^{\mu\nu}_L = \partial^\mu V^\nu + \partial^\nu V^\mu$, 
\begin{align}
h^{\mu\nu}_L &= \partial^\mu \int d^3y D(x-y)\partial_\sigma h^{\sigma\nu} + \partial^\nu \int d^3y D(x-y)\partial_\sigma h^{\sigma\mu} 
\\
&\qquad -  
\partial^\mu\partial^\nu \int d^3y D(x-y) \partial_\sigma \int d^3z D(y-z)\partial_\rho h^{\sigma\rho}
\end{align}
Lastly, we can express this in terms of the longitudinal projector
\begin{align}
L_{\mu\nu\sigma\rho} &= \partial_\mu \int d^4x'\ D(x-x') \eta_{\nu\rho}\partial_\sigma + \partial_\nu \int d^4x'\ D(x-x') \eta_{\mu\sigma}\partial_\tau
\\
&\qquad - \partial_\nu\partial_\mu \int d^4x'\ D(x-x') \partial_\sigma \int d^4x''\ D(x-x'') \partial_\rho.
\end{align}
\subsection*{Transverse and Longitudinal Decomposition: $h_{\mu\nu} = h^L_{\mu\nu}+h^T_{\mu\nu}$}
In a maximally symmetric space of constant curvature, we have the curvature relations
\begin{equation}
R_{\lambda\mu\nu\kappa} = k(g_{\mu\nu}g_{\lambda\kappa} - g_{\lambda\nu}g_{\mu\kappa}),
\qquad R_{\mu\nu} = -(D-1)k g_{\mu\nu},\qquad  R = -D(D-1)k.
\end{equation}
It is convenient to express the curvature tensors in terms of $R$, via
\begin{equation}
\qquad R_{\mu\nu} = \frac{R}{D}g_{\mu\nu},\qquad \nabla_\mu R = 0.
\end{equation}
We posit the longitudinal component of $h^{\mu\nu}$ may be expressed as derivatives onto vectors,
\begin{equation}
h^{\mu\nu}_L = \nabla^\mu V^\nu  + \nabla^\nu V^\mu,
\end{equation}
where $V^{\mu}$ remains to be determined in terms of $h^{\mu\nu}$.
Now project out the transverse components of $h^{\mu\nu}$,
\begin{equation}
\nabla_\nu h^{\mu\nu} = \nabla_\nu \nabla^\mu V^\nu + \nabla_\nu \nabla^\nu V^\mu
= \left(\nabla_\nu\nabla^\nu - \frac{R}{D}\right)V^\mu + \nabla^\mu \nabla_\nu V^\nu 
\end{equation}
\begin{align}
\nabla_\mu\nabla_\nu h^{\mu\nu} &= \nabla_\mu\nabla_\nu( \nabla^\mu V^\nu + \nabla^\nu V^\mu)
\nonumber\\
& = 
 2 \nabla_\mu \nabla^\mu \nabla_\nu V^\nu - 2(\nabla^\mu R_{\mu\nu})V^\nu - 2 R_{\mu\nu} \nabla^\mu V^\nu
\nonumber\\
&
=  2\left(
\nabla_\mu \nabla^\mu - \frac{R}{D}\right) \nabla_\nu V^\nu.
\end{align}
From $\nabla_\mu\nabla_\nu h^{\mu\nu}$, solve for $\nabla_\nu V^\nu$
\begin{equation}
\nabla_\nu V^\nu = \frac12 \int d^Dx' \sqrt{g}\ D(x,x') \nabla_\sigma\nabla_\rho h^{\sigma\rho},
\end{equation}
where we have introduced the curved space scalar propagator
\begin{equation}
\left( \nabla_\nu \nabla^\nu -\frac{R}{D} \right)D(x,x') = g^{-1/2} \delta^D(x-x').
\end{equation}
Now insert $\nabla_\nu V^\nu$ back into $\nabla_\nu h^{\mu\nu}$
\begin{align}
\left(\nabla_\nu\nabla^\nu - \frac{R}{D}\right)V^\mu&= \nabla_\nu h^{\mu\nu} -\nabla^\mu \nabla_\nu V^\nu 
\nonumber\\
&=  \nabla_\nu h^{\mu\nu} - \frac12 \nabla^\mu  \int d^Dx' \sqrt{g}\ D(x,x') \nabla_\sigma\nabla_\rho h^{\sigma\rho}.
\end{align}
Solving for $V^\mu$,
\begin{equation}
V^{\mu} =   \int d^Dx' \sqrt{g}\ D(x,x') \nabla_\sigma h^{\mu\sigma} - \frac12
  \int d^Dx' \sqrt{g}\ D(x,x')\nabla^\mu   \int d^Dx'' \sqrt{g}\ D(x',x'') \nabla_\sigma\nabla_\rho h^{\sigma\rho}.
\end{equation}
Performing integration by parts and dropping the surface integrals (an action whos validity needs investigation), we can bring $V^\mu$ to the form
\begin{equation}
V^{\mu} =   \int d^Dx' \sqrt{g}\ D(x,x') \nabla_\sigma h^{\mu\sigma} - \frac12\nabla^\mu 
  \int d^Dx' \sqrt{g}\ D(x,x')\nabla_\sigma   \int d^Dx'' \sqrt{g}\ D(x',x'') \nabla_\rho h^{\sigma\rho}.
\end{equation}
Now we can construct the longitudinal tensor $h^{\mu\nu}_L = \nabla^\mu V^\nu + \nabla^\nu V^\mu$, 
\begin{align}
  h^{\mu\nu}_L&=\nabla^\mu \int d^Dx' \sqrt{g}\ D(x,x')\nabla_\sigma h^{\sigma\nu} + \nabla^\nu \int d^Dx' \sqrt{g}\  D(x,x')\nabla_\sigma h^{\sigma\mu} 
\\
&\qquad -  
 \nabla^\mu\nabla^\nu \int d^Dx'\sqrt{g}\  D(x,x') \nabla_\sigma \int d^Dx'' \sqrt{g}\ D(x',x'')\nabla_\rho h^{\sigma\rho}.
\end{align}
To verify, let us confirm $\nabla_\nu h^{\mu\nu}_L = \nabla_\nu h^{\mu\nu}$,
\begin{align}
\nabla_\nu h^{\mu\nu}_L &= \nabla_\nu \nabla^\mu \int d^Dx' \sqrt{g}\ D(x,x')\nabla_\sigma h^{\sigma\nu}
+ \nabla_\sigma h^{\sigma\mu} + \frac{R}{D}  \int d^Dx' \sqrt{g}\  D(x,x')\nabla_\sigma h^{\sigma\mu} 
\\
&\qquad - \nabla_\nu \nabla^\mu \nabla^\nu \int d^Dx'\sqrt{g}\  D(x,x') \nabla_\sigma \int d^Dx'' \sqrt{g}\ D(x',x'')\nabla_\rho h^{\sigma\rho}.
\end{align}
Noting the commutation relation
\begin{equation}
\nabla_\nu \nabla^\mu \nabla^\nu f(x) = \nabla^\mu\left[\left( \nabla_\nu \nabla^\nu - \frac{R}{D}\right)f(x)\right]
\end{equation}
we can express the longitudinal tensor as
\begin{align}
\nabla_\nu h^{\mu\nu}_L &= \nabla_\nu \nabla^\mu \int d^Dx' \sqrt{g}\ D(x,x')\nabla_\sigma h^{\sigma\nu}
+ \nabla_\sigma h^{\sigma\mu} + \frac{R}{D}  \int d^Dx' \sqrt{g}\  D(x,x')\nabla_\sigma h^{\sigma\mu} 
\nonumber
\\ &\qquad 
- \nabla^\mu \nabla_\sigma \int d^Dx' \sqrt{g}\ D(x,x')\nabla_\rho h^{\sigma\rho}.
\end{align}
Taking another commutation relation
\begin{equation}
\nabla^\mu \nabla_\sigma A^\sigma(x) = \nabla_\sigma\nabla^\mu A^\sigma(x) + \frac{R}{D}A^\mu(x),
\end{equation}
we are finally left with
\begin{equation}
\nabla_\nu h^{\mu\nu}_L = \nabla_\nu h^{\mu\nu}.
\end{equation}
Lastly, we cast the longitudinal component into the form a projector
\begin{align}
L_{\mu\nu\sigma\rho} &= \nabla_\mu \int d^Dx' \sqrt g\ D(x,x') g_{\sigma\nu}\nabla_\rho 
+ \nabla_\nu \int d^Dx' \sqrt g\ D(x,x') g_{\sigma\mu}\nabla_\rho 
\nonumber\\
&\qquad - \nabla_\mu\nabla_\nu \int d^Dx'\sqrt{g}\  D(x,x') \nabla_\sigma \int d^Dx'' \sqrt{g}\ D(x',x'')\nabla_\rho. 
\end{align}
It follows that the transverse projector is just what remains,
\begin{align}
T_{\mu\nu\sigma\rho} &= g_{\mu\sigma}g_{\nu\rho}- \nabla_\mu \int d^Dx' \sqrt g\ D(x,x') g_{\sigma\nu}\nabla_\rho 
- \nabla_\nu \int d^Dx' \sqrt g\ D(x,x') g_{\sigma\mu}\nabla_\rho 
\nonumber\\
&\qquad + \nabla_\mu\nabla_\nu \int d^Dx'\sqrt{g}\  D(x,x') \nabla_\sigma \int d^Dx'' \sqrt{g}\ D(x',x'')\nabla_\rho. 
\end{align}
\emph{Still need to confirm if the above actually behave as projectors, i.e. $L_{\mu\nu\sigma\rho}L^{\sigma\rho}{}_{\alpha\beta} = L_{\mu\nu\alpha\beta}$, etc.}
\subsection*{Traceless Transverse and Traceless Longitudinal Decomposition: : $h_{\mu\nu} = h^{L\theta}_{\mu\nu}+h^{T\theta}_{\mu\nu}+h^{tr}_{\mu\nu}$}
Following C.93 in \emph{Brane Gravity}, we may construct the traceless longitudinal component via
\begin{align}
h_{\mu\nu}^{L\theta} &= h_{\mu\nu}^L - \frac{1}{D-1} g_{\mu\nu} g^{\sigma\tau}h^L_{\sigma\tau} +\frac{1}{D-1}
\left[ \nabla_\mu\nabla_\nu- g_{\mu\nu}\frac{R}{D(D-1)}\right] \int d^Dx' \sqrt{g}\ F(x,x')g^{\sigma\tau}h_{\sigma\tau}^L ,
\end{align}
where we have introduced another scalar propogator obeying
\begin{equation}
\left( \nabla_\rho \nabla^\rho - \frac{R}{D-1}\right)F(x,x') = g^{-1/2} \delta^D (x-x').
\end{equation}
As written, the tensor $h_{\mu\nu}^{L\theta}$ obeys
\begin{align}
g^{\mu\nu}h_{\mu\nu}^{L\theta} = 0,\qquad \nabla^\nu h_{\mu\nu}^{L\theta} = \nabla^\nu h_{\mu\nu}^{L}.
\end{align}
With the analogous decomposition following for $h_{\mu\nu}^{T\theta}$ taking the form
\begin{align}
h_{\mu\nu}^{T\theta} &= h_{\mu\nu}^T - \frac{1}{D-1} g_{\mu\nu} g^{\sigma\tau}h^T_{\sigma\tau} +\frac{1}{D-1}
\left[ \nabla_\mu\nabla_\nu- g_{\mu\nu}\frac{R}{D(D-1)}\right] \int d^Dx' \sqrt{g}\ F(x,x')g^{\sigma\tau}h_{\sigma\tau}^T ,
\end{align}
we may construct the full $h_{\mu\nu}$ by taking their sum:
\begin{align}
h_{\mu\nu}^{T\theta}+h_{\mu\nu}^{L\theta}&= h_{\mu\nu} - \frac{1}{D-1} g_{\mu\nu} g^{\sigma\tau}h_{\sigma\tau} +\frac{1}{D-1}
\left[ \nabla_\mu\nabla_\nu- g_{\mu\nu}\frac{R}{D(D-1)}\right] \int d^Dx' \sqrt{g}\ F(x,x')g^{\sigma\tau}h_{\sigma\tau}.
\end{align}
Hence the full $h_{\mu\nu}$ takes the form
\begin{align}
h_{\mu\nu}&= h_{\mu\nu}^{T\theta}+h_{\mu\nu}^{L\theta} + \frac{1}{D-1} g_{\mu\nu} g^{\sigma\tau}h_{\sigma\tau} -\frac{1}{D-1}
\left[ \nabla_\mu\nabla_\nu- g_{\mu\nu}\frac{R}{D(D-1)}\right] \int d^Dx' \sqrt{g}\ F(x,x')g^{\sigma\tau}h_{\sigma\tau}
\nonumber\\
&\equiv h_{\mu\nu}^{T\theta}+h_{\mu\nu}^{L\theta}+h^{tr}_{\mu\nu}.
\end{align}
\subsection*{The SVT Basis}
Given the form for $h_{\mu\nu}^{L\theta}$, unlike the flat space case, I was unable to construct a vector $V_\mu$ such that
\begin{equation}
h_{\mu\nu}^{L\theta} = \nabla_\mu V_\nu + \nabla_\nu V_\mu -\frac{2}{D}g_{\mu\nu} \nabla^\sigma V_\sigma. 
\end{equation}
However, this intermediate step, though useful, is not required for an SVT decomposition.
First, let us note the relation
\begin{align}
h_{\mu\nu}^{L\theta} + h^{tr}_{\mu\nu} &= h_{\mu\nu}^L+ \frac{1}{D-1} g_{\mu\nu} g^{\sigma\tau}(h_{\sigma\tau}-h^L_{\sigma\tau})
\nonumber\\
&\quad -\frac{1}{D-1}
\left[ \nabla_\mu\nabla_\nu- g_{\mu\nu}\frac{R}{D(D-1)}\right] \int d^Dx' \sqrt{g}\ F(x,x')g^{\sigma\tau}(h_{\sigma\tau}-h_{\sigma\tau}^L)
\end{align}
Next, let us introduce the vector
\begin{equation}
W_{\mu} = \int d^Dx' \sqrt{g}\ D(x,x')\nabla^\sigma h_{\sigma\mu},
\end{equation}
whereby the longitudinal component (ref) may be expressed as
\begin{equation}
h_{\mu\nu}^L = \nabla_\mu W_\nu + \nabla_\nu W_\mu - \nabla_\mu\nabla_\nu \int d^Dx' \sqrt{g}\ D(x,x')\nabla^\sigma W_\sigma,
\end{equation}
with a trace obeying
\begin{equation}
g^{\mu\nu}h_{\mu\nu}^{L} = \nabla^\sigma W_\sigma - \frac{R}{D} \int d^Dx' \sqrt{g}\ D(x,x')\nabla^\sigma W_\sigma.
\end{equation}
Now we elect to decompose $W_{\mu}$ into its transverse and longitudinal components viz. 
\begin{align}
W_\mu &= W_\mu^T + \nabla_\mu W,\qquad W = \int d^Dx' \sqrt{g}\ A(x,x')\nabla^\sigma W_\sigma, 
\qquad \nabla_\rho \nabla^\rho W = \nabla^\sigma W_\sigma,
\end{align}
where we have introduced the scalar propagator which obeys
\begin{equation}
 \nabla_\rho \nabla^\rho A(x,x') = g^{-1/2}\delta^D(x-x').
\end{equation}
In the scalar vector basis, $h^L_{\mu\nu}$ takes the form
\begin{equation}
h_{\mu\nu}^L =\nabla_\mu W^T_\nu + \nabla_\nu W^T_\mu + \nabla_\mu\nabla_\nu\left( 2W-\int d^Dx' \sqrt{g}\ D(x,x')\nabla_\rho \nabla^\rho W\right),
\end{equation}
with trace
\begin{equation}
g^{\mu\nu}h_{\mu\nu}^L = \nabla_\rho \nabla^\rho W - \frac{R}{D}\int d^Dx' \sqrt g\ D(x,x') \nabla_\rho \nabla^\rho W.
\end{equation}
For compactness, let us define the scalar
\begin{align}
M(x)  &= g^{\mu\nu}h_{\mu\nu} - g^{\mu\nu}h_{\mu\nu}^L 
\nonumber\\
&= g^{\sigma\tau}h_{\sigma\tau} -  \nabla_\rho \nabla^\rho W + \frac{R}{D}\int d^Dx' \sqrt g\ D(x,x') \nabla_\rho \nabla^\rho W
\nonumber\\
&= g^{\sigma\tau}h_{\sigma\tau} - \nabla^\sigma \int d^Dx' \sqrt{g}\ D(x,x') \nabla^\rho h_{\sigma\rho}
 +\frac{R}{D}\int d^Dx' \sqrt g\ D(x,x') \nabla^\sigma \int d^Dx'' \sqrt{g}\ D(x',x'') \nabla^\rho h_{\sigma\rho}.
\end{align}
Now we can express (ref) in terms of scalars and vectors as
\begin{align}
h_{\mu\nu}^{L\theta} + h^{tr}_{\mu\nu} &=\nabla_\mu W^T_\nu + \nabla_\nu W^T_\mu
\nonumber\\
 &\quad+ \nabla_\mu\nabla_\nu \left[ 2W- \int d^Dx' \sqrt{g}\ D(x,x')\nabla_\rho \nabla^\rho W - \frac{1}{D-1} \int  d^Dx' \sqrt{g}\ F(x,x') M(x')\right]
\nonumber\\
&\quad +\frac{1}{D-1} g_{\mu\nu}\left[ M(x) + \frac{R}{D(D-1)} \int   d^Dx' \sqrt{g}\ F(x,x') M(x') \right].
\end{align}
The full $h_{\mu\nu}$ then may be written as
\begin{align}
h_{\mu\nu} &=h_{\mu\nu}^{T\theta} + \nabla_\mu W^T_\nu + \nabla_\nu W^T_\mu
\nonumber\\
 &\quad+ \nabla_\mu\nabla_\nu \left[ 2W- \int d^Dx' \sqrt{g}\ D(x,x')\nabla_\rho \nabla^\rho W - \frac{1}{D-1} \int  d^Dx' \sqrt{g}\ F(x,x') M(x')\right]
\nonumber\\
&\quad +\frac{1}{D-1} g_{\mu\nu}\left[ M(x) + \frac{R}{D(D-1)} \int   d^Dx' \sqrt{g}\ F(x,x') M(x') \right].
\end{align}
With the two scalars and the transverse vector
\begin{align}
M(x) &=  g^{\sigma\tau}h_{\sigma\tau} - \nabla^\sigma \int d^Dx' \sqrt{g}\ D(x,x') \nabla^\rho h_{\sigma\rho}
 +\frac{R}{D}\int d^Dx' \sqrt g\ D(x,x') \nabla^\sigma \int d^Dx'' \sqrt{g}\ D(x',x'') \nabla^\rho h_{\sigma\rho}
\nonumber\\
W(x) &=  \int d^Dx' \sqrt{g}\ A(x,x') \nabla^\sigma  \int   d^Dx'' \sqrt{g}\ D(x',x'') \nabla^\rho h_{\sigma\rho}
\nonumber\\
W^T_\mu &= \int d^Dx' \sqrt{g}\ D(x,x')\nabla^\sigma h_{\sigma\mu} - \nabla_\mu \int d^Dx' \sqrt{g}\ A(x,x')
\nabla^\sigma \int d^Dx'' \sqrt{g}\ D(x',x'') \nabla^\rho h_{\sigma\rho},
\end{align}
upon defining
\begin{align}
2\psi &= -\frac{1}{(D-1)}\left[ M(x) + \frac{R}{D(D-1)} \int   d^Dx' \sqrt{g}\ F(x,x') M(x') \right]
\nonumber\\
2E&= 2W(x)- \int d^Dx' \sqrt{g}\ D(x,x')\nabla_\rho \nabla^\rho W(x') - \frac{1}{D-1} \int  d^Dx' \sqrt{g}\ F(x,x') M(x')
\nonumber\\
E_{\mu}&= W_{\mu}^T
\nonumber\\
2E_{\mu\nu} &= h_{\mu\nu}^{T\theta},
\end{align}
the tensor takes the SVT form
\begin{equation}
h_{\mu\nu} = -2 g_{\mu\nu}\psi + 2\nabla_\mu \nabla_\nu E + \nabla_\mu E_\nu +\nabla_\nu E_\mu + 2E_{\mu\nu}.
\end{equation}
If we restrict to flat space, we have the following simplifications:
\begin{align}
R &= 0,\qquad A(x,x') = D(x,x') = F(x,x'),\qquad M(x) =  g^{\sigma\tau}h_{\sigma\tau} - \nabla^\sigma \int d^Dx' \sqrt{g}\ D(x,x') \nabla^\rho h_{\sigma\rho}
\nonumber\\
\sqrt g &= 1,\qquad W(x) = \int d^Dx' \sqrt{g}\ D(x,x') \nabla^\sigma  \int   d^Dx'' \sqrt{g}\ D(x',x'') \nabla^\rho h_{\sigma\rho}.
\end{align}
According to (ref 63), the SVT components would then be reduce to
\begin{align}
2\psi &= -\frac{1}{(D-1)}\left[  g^{\sigma\tau}h_{\sigma\tau} - \nabla^\sigma \int d^Dx' \ D(x,x') \nabla^\rho h_{\sigma\rho} \right]
\nonumber\\
2E&= \frac{D}{D-1} \int d^Dx'\ D(x,x') \nabla^\sigma  \int   d^Dx' \ D(x,x') \nabla^\rho h_{\sigma\rho}
 - \frac{1}{D-1} \int  d^Dx' \ D(x,x') g^{\sigma\tau}h_{\sigma\tau}
\nonumber\\
E_{\mu}&=  \int d^Dx' \ D(x,x')\nabla^\sigma h_{\sigma\mu} - \nabla_\mu \int d^Dx' \ D(x,x')
\nabla^\sigma \int d^Dx'' \ D(x',x'') \nabla^\rho h_{\sigma\rho}
\nonumber\\
2E_{\mu\nu} &= h_{\mu\nu}^{T\theta}.
\end{align}
Follwing an integration by parts on $E$ and $\psi$, the above equates to our prior paper results. 
\subsection*{Traceless $\pi_{\mu\nu}$ Decomposition}
After the 3+1 splitting of $T_{\mu\nu}$, we are left with a traceless $\pi_{\mu\nu}$ of which we would like to decompose into scalars, vectors tensors. Taking $\pi_{\mu\nu}$ to be of the same SVT form as $h_{\mu\nu}$, namely
\begin{equation}
\pi_{\mu\nu} = -2 g_{\mu\nu}\psi + 2\nabla_\mu \nabla_\nu E + \nabla_\mu E_\nu +\nabla_\nu E_\mu + 2E_{\mu\nu}.
\end{equation}
From the tracelessness of $\pi_{\mu\nu}$ it follows
\begin{equation}
2D\psi = 2 \nabla_\rho \nabla^\rho E
\end{equation}
(expressing $\psi$ and $E$ in their projected integral form, the above holds identically when $g^{\mu\nu}\pi_{\mu\nu}=0$, as anticipated). 
Substituting
\begin{equation}
\psi = \frac{1}{D} \nabla_\rho \nabla^\rho E, 
\end{equation}
the tensor becomes
\begin{equation}
\pi_{\mu\nu} = -\frac{2}{D}g_{\mu\nu}\nabla_\rho \nabla^\rho E + 2\nabla_\mu \nabla_\nu E + \nabla_\mu E_\nu +\nabla_\nu E_\mu + 2E_{\mu\nu}.
\end{equation}
Finally, upon defining
\begin{equation}
\pi = E,\qquad \pi_\mu = E_\mu,\qquad 2E_{\mu\nu}=\pi_{\mu\nu}^{T\theta},
\end{equation}
we may write $\pi_{\mu\nu}$ in the desired form
\begin{equation}
\pi_{\mu\nu} = -\frac{2}{D}g_{\mu\nu}\nabla_\rho \nabla^\rho \pi + 2\nabla_\mu \nabla_\nu \pi + \nabla_\mu \pi_\nu +\nabla_\nu \pi_\mu + \pi^{T\theta}_{\mu\nu}.
\end{equation}
For reference, the components in their projected form are
\begin{align}
2\pi &= 2W(x)- \int d^Dx' \sqrt{g}\ D(x,x')\nabla_\rho \nabla^\rho W(x') - \frac{1}{D-1} \int  d^Dx' \sqrt{g}\ F(x,x') M(x')
\nonumber\\
\pi_{\mu}&=  \int d^Dx' \sqrt{g}\ D(x,x')\nabla^\sigma h_{\sigma\mu} - \nabla_\mu \int d^Dx' \sqrt{g}\ A(x,x')
\nabla^\sigma \int d^Dx'' \sqrt{g}\ D(x',x'') \nabla^\rho h_{\sigma\rho},
\end{align}
where
\begin{align}
M(x) &= - \nabla^\sigma \int d^Dx' \sqrt{g}\ D(x,x') \nabla^\rho h_{\sigma\rho}
 +\frac{R}{D}\int d^Dx' \sqrt g\ D(x,x') \nabla^\sigma \int d^Dx'' \sqrt{g}\ D(x',x'') \nabla^\rho h_{\sigma\rho}
\nonumber\\
W(x) &=  \int d^Dx' \sqrt{g}\ A(x,x') \nabla^\sigma  \int   d^Dx'' \sqrt{g}\ D(x',x'') \nabla^\rho h_{\sigma\rho}.
\end{align}
%---------------------------------------------------------------------------------------------------------------------------------------------------------------------------%
\newpage
the longitudinal traceless component is expressed as
\begin{align}
h_{\mu\nu}^{L\theta} &= \nabla_\mu W_\nu + \nabla_\nu W_\mu 
\nonumber\\
 &\quad+ \nabla_\mu\nabla_\nu \bigg[ - \int d^Dx' \sqrt{g}\ D(x,x')\nabla^\sigma W_\sigma
\nonumber\\
&\qquad\qquad
+ \frac{1}{D-1} \int d^Dx' \sqrt{g}\ F(x,x') \left( \nabla^\sigma W_\sigma-\frac{R}{D}\int d^Dx' \sqrt{g}\ D(x,x')\nabla^\sigma W_\sigma\right)\bigg]
\nonumber\\
&\quad + \frac{g_{\mu\nu}}{D-1}\bigg[ - \nabla^\sigma W_\sigma - \frac{R}{D} \int d^Dx' \sqrt{g}\ D(x,x')\nabla^\sigma W_\sigma 
\nonumber \\
&\qquad\qquad
- \frac{R}{D(D-1)} \int d^Dx' \sqrt{g}\ F(x,x') \left( \nabla^\sigma W_\sigma-\frac{R}{D}\int d^Dx' \sqrt{g}\ D(x,x')\nabla^\sigma W_\sigma\right)\bigg].
\end{align}
At this point, we may elect to decompose the $W_{\mu}$ into its transverse and longitudinal components viz. 
\begin{align}
W_\mu &= W_\mu^T + \nabla_\mu W,\qquad 
\nabla_\rho \nabla^\rho W = \nabla^\sigma W_\sigma =\nabla^\sigma \int d^Dx'' \sqrt{g}\ (x',x'') \nabla^\rho h_{\sigma\rho}.
\end{align}
Now $h_{\mu\nu}^{L\theta}$ may be expressed in terms of transverse vectors $W^T_{\mu}$ and scalars $W$,
\begin{align}
h_{\mu\nu}^{L\theta} &= \nabla_\mu W_\nu^T + \nabla_\nu W_\mu^T  
\nonumber\\
&\quad + \nabla_\mu\nabla_\nu \bigg[ 2W- \int d^Dx' \sqrt{g}\ D(x,x')\nabla_\rho \nabla^\rho W
\nonumber\\
&\qquad \qquad + \frac{1}{D-1} \int d^Dx' \sqrt{g}\ F(x,x') \left( \nabla_\rho \nabla^\rho W-\frac{R}{D}\int d^Dx' \sqrt{g}\ D(x,x')\nabla_\rho \nabla^\rho W\right)\bigg]
\nonumber\\
&\quad + \frac{g_{\mu\nu}}{D-1}\bigg[ -\nabla_\rho \nabla^\rho W - \frac{R}{D} \int d^Dx' \sqrt{g}\ D(x,x')\nabla_\rho \nabla^\rho W 
\nonumber \\
&\qquad\qquad 
- \frac{R}{D(D-1)} \int d^Dx' \sqrt{g}\ F(x,x') \left(\nabla_\rho \nabla^\rho W-\frac{R}{D}\int d^Dx' \sqrt{g}\ D(x,x')\nabla_\rho \nabla^\rho W\right)\bigg].
\end{align}
\subsection*{Remark}
We are at somewhat of an impasse since 1) The vector decomposition involved a propagator different from the tensor decomposition and 2) We have not fully decomposed $h_{\mu\nu}^L$ in terms of scalars and vectors, since the integral relation still remains. 
\begin{align}
h_{\mu\nu}^{L\theta} &= h_{\mu\nu}^L - \frac{1}{D-1} g_{\mu\nu} g^{\sigma\tau}h^L_{\sigma\tau} +\frac{1}{D-1}
\left[ \nabla_\mu\nabla_\nu + g_{\mu\nu}\frac{R}{D} - g_{\mu\nu}\frac{R}{D-1}\right] \int d^Dx' \sqrt{g}\ F(x,x')g^{\sigma\tau}h_{\sigma\tau}^L 
\end{align}
asl;dkfjsl;kdfj
\begin{align}
h_{\mu\nu}^{L\theta} &= h_{\mu\nu}^L - \frac{1}{D-1} g_{\mu\nu} g^{\sigma\tau}h^L_{\sigma\tau} +\frac{1}{D-1}
\left[ \nabla_\mu\nabla_\nu- g_{\mu\nu}\frac{R}{D(D-1)}\right] \int d^Dx' \sqrt{g}\ F(x,x')g^{\sigma\tau}h_{\sigma\tau}^L 
\end{align}
\begin{align}
  h^{\mu\nu}_L&=\nabla^\mu \int d^Dx' \sqrt{g}\ D(x,x')\nabla_\sigma h^{\sigma\nu} + \nabla^\nu \int d^Dx' \sqrt{g}\  D(x,x')\nabla_\sigma h^{\sigma\mu} 
\\
&\qquad -  
 \nabla^\nu\nabla^\mu \int d^Dx'\sqrt{g}\  D(x,x') \nabla_\sigma \int d^Dx'' \sqrt{g}\ D(x',x'')\nabla_\rho h^{\sigma\rho},
\end{align}
where
\begin{equation}
\left( \nabla_\nu \nabla^\nu -\frac{R}{D} \right)D(x,x') = g^{-1/2} \delta^D(x-x'),\qquad \left( \nabla_\nu \nabla^\nu -\frac{R}{D-1} \right)F(x,x') = g^{-1/2} \delta^F(x-x').
\end{equation}
\begin{equation}
\nabla_\mu \nabla^\mu \nabla^\nu \phi = \nabla^\nu \nabla_\mu \nabla^\mu \phi - \frac{R}{D}\nabla^\nu \phi,
\quad \nabla_\nu \nabla^\mu \nabla^\nu \phi = \nabla^\mu \nabla_\nu \nabla^\nu \phi - \frac{R}{D}\nabla^\mu \phi,\quad
\nabla_\nu \nabla^\sigma W_{\sigma} = \nabla^\sigma \nabla_\nu W_{\sigma} + \frac{R}{D}W_{\nu}
\end{equation}
\begin{align}
\nabla^\nu h_{\mu\nu}^L &= 
\nabla^\nu\left[ \nabla_\mu W_\nu + \nabla_\nu W_\mu - \nabla_\mu\nabla_\nu
 \int d^Dx'\sqrt{g}\  D(x,x') \nabla_\sigma W^\sigma\right]
\nonumber\\
&=
\left(\nabla_\alpha \nabla^\alpha - \frac{R}{D}\right)W_\mu
\nonumber\\
&= \nabla^\nu h_{\mu\nu}
\end{align}
\begin{equation}
\nabla^\nu h_{\mu\nu}^L = \left(\nabla_\alpha \nabla^\alpha - \frac{R}{D}\right)W_\mu,
\qquad \nabla^\mu \nabla^\nu h_{\mu\nu}^L = \left(\nabla_\alpha \nabla^\alpha - 2\frac{R}{D}\right)\nabla^\sigma W_\sigma
\end{equation}
\begin{align}
g^{\mu\nu}h_{\mu\nu}^L = \nabla^\sigma W_{\sigma} - \frac{R}{D} \int dx' \sqrt{g}\ D(x,x') \nabla^\sigma W_\sigma
\end{align}
\begin{equation}
\nabla^\nu h_{\mu\nu}^L =  \left(\nabla_\nu\nabla^\nu - \frac{R}{D}\right)V_\mu + \nabla_\mu \nabla_\nu V^\nu,
\qquad \nabla^\mu \nabla^\nu h_{\mu\nu}^L = 2\left(\nabla_\alpha \nabla^\alpha - \frac{R}{D}\right)\nabla_\nu V^\nu,
\qquad g^{\mu\nu}h_{\mu\nu}^L = 2 \nabla^\sigma V_\sigma
\end{equation}
\begin{align}
h_{\mu\nu}^{L\theta} &= \nabla_\mu V_\nu +\nabla_\nu V_\mu - \frac{2}{D-1} g_{\mu\nu} \nabla^\sigma V_\sigma +\frac{2}{D-1}
\left[ \nabla_\mu\nabla_\nu- g_{\mu\nu}\frac{R}{D(D-1)}\right] \int d^Dx' \sqrt{g}\ F(x,x')\nabla^\sigma V_\sigma
\end{align}
\begin{align}
h_{\mu\nu}^{L\theta} &=  \nabla_\mu \int d^Dx' \sqrt{g}\ D(x,x')\nabla^\sigma h_{\sigma\nu} + \nabla_\nu \int d^Dx' \sqrt{g}\  D(x,x')\nabla^\sigma h_{\sigma\mu} 
\\
&\qquad -  
 \nabla_\mu\nabla_\nu \int d^Dx'\sqrt{g}\  D(x,x') \nabla_\sigma \int d^Dx'' \sqrt{g}\ D(x',x'')\nabla_\rho h^{\sigma\rho},
\nonumber\\
&\qquad - \frac{1}{D-1} g_{\mu\nu} g^{\sigma\tau}h^L_{\sigma\tau} +\frac{1}{D-1}
\left[ \nabla_\mu\nabla_\nu- g_{\mu\nu}\frac{R}{D(D-1)}\right] \int d^Dx' \sqrt{g}\ F(x,x')g^{\sigma\tau}h_{\sigma\tau}^L 
\end{align}
\begin{align}
h_{\mu\nu}^{L\theta} &=  \nabla_\mu \int d^Dx' \sqrt{g}\ D(x,x')\nabla^\sigma h_{\sigma\nu} + \nabla_\nu \int d^Dx' \sqrt{g}\  D(x,x')\nabla^\sigma h_{\sigma\mu} 
\\
&\qquad -  
 \nabla_\mu\nabla_\nu \int d^Dx'\sqrt{g}\  D(x,x') \nabla_\sigma \int d^Dx'' \sqrt{g}\ D(x',x'')\nabla_\rho h^{\sigma\rho},
\nonumber\\
&\qquad - \frac{1}{D-1} g_{\mu\nu} g^{\sigma\tau}h^L_{\sigma\tau} +\frac{1}{D-1}
 \nabla_\mu\nabla_\nu  \int d^Dx' \sqrt{g}\ D(x,x')g^{\sigma\tau}h_{\sigma\tau}^L 
\end{align}
\newpage
\begin{equation}
h_{\mu\nu}^{L\theta} = \nabla_\mu V_\nu +\nabla_\nu V_\mu + \alpha g_{\mu\nu} \nabla^\sigma V_\sigma + \beta R g_{\mu\nu} \nabla^\sigma V_\sigma
\end{equation}
\begin{equation}
g^{\mu\nu}h_{\mu\nu}^{L\theta} = (2+\alpha D +\beta R D)\nabla^\sigma V_\sigma
\end{equation}
\begin{align}
\nabla^\nu h_{\mu\nu}^{L\theta} &= \nabla^\sigma \nabla_\sigma V_\mu + \nabla_\mu \nabla^\sigma V_\sigma - \frac{R}{D}V_\mu +\alpha \nabla_\mu \nabla^\sigma V_\sigma + \beta R \nabla_\mu \nabla^\sigma V_{\sigma}
\nonumber\\
&= ( 1+\alpha +\beta R) \nabla_\mu\nabla^\sigma V_\sigma + \left(\nabla^\sigma \nabla_\sigma- \frac{R}{D}\right) V_\mu
\end{align}
\begin{align}
\nabla^\mu \nabla^\nu h_{\mu\nu}^{L\theta} &= 2\nabla^\sigma \nabla_\sigma \nabla^\mu V_\mu -2\frac RD \nabla^\mu V_\mu  + \alpha \nabla^\sigma \nabla_\sigma \nabla^\mu  V_\mu + \beta R \nabla^\sigma \nabla_\sigma \nabla^\mu V_\mu
\nonumber\\
&= \left[ \left( 2 + \alpha +\beta R\right) \nabla_\sigma \nabla^\sigma -2\frac{R}{D}\right] \nabla^\mu V_{\mu}
\end{align}
\begin{equation}
\nabla^\mu V_{\mu} = \int dx' \sqrt{g}\ D(x,x') \nabla^\sigma \nabla^\tau h_{\sigma\tau}
\end{equation}
where
\begin{equation}
\left[ \left( 2 + \alpha +\beta R\right) \nabla_\sigma \nabla^\sigma -2\frac{R}{D}\right]D(x,x') = \sqrt g \delta(x-x')
\end{equation}
Substitute this into $\nabla^\nu h_{\mu\nu}^{L\theta}$,
\begin{equation}
h_{\mu\nu}^{L\theta} = \nabla_\mu V_\nu +\nabla_\nu V_\mu + \alpha g_{\mu\nu} \nabla^\sigma V_\sigma + \beta R g_{\mu\nu} \nabla^\sigma V_\sigma
+\left[ \gamma \nabla_\mu \nabla_\nu + \rho g_{\mu\nu} + \kappa R g_{\mu\nu} \right] \int D(x,x') \nabla^\sigma V_\sigma
\end{equation}
\begin{equation}
\nabla_\nu \nabla^\nu D(x,x') - A(x) D(x,x') = \sqrt g \delta^D(x-x')
\end{equation}
\begin{align}
\nabla^\mu \nabla^\nu h_{\mu\nu}^{L\theta} &= 2\left( \nabla_\rho \nabla^\rho- \frac{R}{D}\right) \nabla^\sigma V_\sigma
+(\alpha+\beta R) \nabla_\rho \nabla^\rho \nabla^\sigma V_\sigma + \gamma \left(\nabla_\rho \nabla^\rho-\frac{R}{D}\right) \nabla^\sigma V_\sigma 
\nonumber \\ 
&\qquad 
+\gamma\left(\nabla_\rho \nabla^\rho-\frac{R}{D}\right)  \left( A(x) \int D(x,x') \nabla^\sigma V_\sigma \right) 
\nonumber\\
&\qquad
+(\rho + \kappa R) \nabla^\sigma V_\sigma + (\rho+\kappa R) \left( A(x) \int D(x,x')\nabla^\sigma V_\sigma\right)
\nonumber\\
&= \left[ \left(\nabla_\rho \nabla^\rho - \frac{R}{D}\right)(2+\gamma) + (\alpha +\beta R)\nabla_\rho \nabla^\rho + \rho +\kappa R\right]\nabla^\sigma V_\sigma
\nonumber\\
&\qquad + \left[\gamma\left(\nabla_\rho \nabla^\rho -\frac{R}{D}\right) + \rho+\kappa R\right]  A(x) \int D(x,x')\nabla^\sigma V_\sigma
\nonumber\\
&= \left[ (\alpha +\beta R +2+\gamma)\nabla_\rho \nabla^\rho -(2+\gamma)\frac{R}{D} + \rho+\kappa R+\gamma A(x)\right] \nabla^\sigma V_\sigma
\nonumber \\
&\qquad +\left[  \left(\rho +\kappa R -\gamma \frac{R}{D} + \gamma A(x)\right) A(x) + \gamma \nabla_\rho \nabla^\rho A(x)\right] \int D(x,x')\nabla^\sigma V_\sigma
\end{align}
\begin{align}
g^{\mu\nu}h_{\mu\nu}^{L\theta} &= \left[ 2 + D(\alpha +\beta R) + \gamma\right] \nabla^\sigma V_\sigma 
+\gamma A(x) \int D(x,x') \nabla^\sigma V_\sigma + D(\rho +\kappa R)\int D(x,x') \nabla^\sigma V_\sigma
\end{align}
\begin{align}
\nabla^\mu \nabla^\nu h_{\mu\nu}^{L\theta} &= \left[
(\alpha + 2 +\gamma) \nabla_\rho \nabla^\rho + R \left( \frac{-2 - \gamma}{D} +\kappa + \gamma q \right) +\rho+\gamma p\right] \nabla^\sigma V_\sigma
\nonumber\\
&\qquad + \left\{ p(\rho +\gamma p) + R\left[ p(\kappa-\frac{\gamma}{D} + \gamma q)+q(\rho+\gamma p)\right] + R^2 q\left( k - \frac{\gamma}{D} + \gamma q\right)\right\} \int D(x,x') \nabla^\sigma V_\sigma
\end{align}
Hence we require
\begin{equation}
2+\gamma + D(\alpha +\beta R) = 0,\qquad \gamma A(x) + D(\rho +\kappa R) = 0.
\end{equation}
Taking $A(x) = p+qR$, the conditions are then (holding for each power of $R$), 
\begin{equation}
2+\gamma + D\alpha = 0,\qquad D\beta R = 0,\qquad \gamma p + D\rho = 0,\qquad \gamma q + D\kappa = 0
\end{equation}
For convenience, we would like the integral relation in $\nabla^\mu \nabla^\nu h_{\mu\nu}^{L\theta}$ to vanish, and thus we set
\begin{equation}
0= p(\rho +\gamma p) + R\left[ p(\kappa-\frac{\gamma}{D} + \gamma q)+q(\rho+\gamma p)\right] + R^2 q\left( k - \frac{\gamma}{D} + \gamma q\right).
\end{equation}
Hence all together we have
\begin{align}
2+\gamma + D\alpha = 0,\qquad \beta =0,\qquad \gamma p + D\rho = 0,\qquad \gamma q + D\kappa = 0
\end{align}
\begin{align}
p(\rho +\gamma p) = 0,\qquad  p(\kappa-\frac{\gamma}{D} + \gamma q)+q(\rho+\gamma p)=0,\qquad 
q\left( \kappa - \frac{\gamma}{D} + \gamma q\right)=0
\end{align}
Six equations, six unknowns. Start with the relations
\begin{equation}
\kappa = -\gamma \frac{q}{D},\qquad \rho = -\gamma \frac{p}{D},
\end{equation}
which leads to 
\begin{equation}
\gamma p \left( p - \frac{p}{D}\right) = 0
\end{equation}
Now either $p=0$, $\gamma=0$, or $p -\frac pD = 0$. 
\\ \\
Helpful covariant commutations (within maximally symmetric space):
\begin{equation}
\nabla^\nu \nabla_\mu V_\nu = \nabla_\mu \nabla^\nu V_\nu - \frac{R}{D}V_\mu,\qquad
\nabla^\mu \nabla_\rho \nabla^\rho V_\mu = \nabla_\rho \nabla^\rho \nabla^\mu V_\mu - \frac{R}{D}\nabla^\sigma V_\sigma.
\end{equation}
Let us posit $h_{\mu\nu}^{L\theta}$ to be of the follwing form (see A.1)
\begin{equation}
h_{\mu\nu}^{L\theta} = \nabla_\mu V_\nu +\nabla_\nu V_\mu -\frac{2}{D} g_{\mu\nu} \nabla^\sigma V_\sigma.
\end{equation}
It follows that 
\begin{equation}
 g^{\mu\nu} h_{\mu\nu}^{L\theta} = 0,
\end{equation}
\begin{equation}
\nabla^\nu h_{\mu\nu}^{L\theta} = \nabla^\nu h_{\mu\nu}= \left( \nabla_\rho \nabla^\rho - \frac{R}{D}\right) V_\mu + \frac{D-2}{D} \nabla_\mu \nabla^\sigma V_\sigma,
\end{equation}
\begin{align}
\nabla^\mu \nabla^\nu h_{\mu\nu}^{L\theta} &=\nabla^\mu \nabla^\nu h_{\mu\nu}= 2\left(\frac{D-1}{D}\nabla_\rho \nabla^\rho - \frac{R}{D}\right) \nabla^\sigma V_\sigma 
\nonumber
\\
\to& \quad \frac{D}{2(D-1)}\nabla^\mu \nabla^\nu h_{\mu\nu}  = \left( \nabla_\rho \nabla^\rho - \frac{R}{D-1}\right)\nabla^\sigma V_\sigma,
\end{align}
where we have imposed $\nabla^\nu h_{\mu\nu}^{L\theta} = \nabla^\nu h_{\mu\nu}$, $\nabla^\mu \nabla^\nu h_{\mu\nu}^{L\theta} =\nabla^\mu \nabla^\nu h_{\mu\nu}$.
Now introduce a scalar propagator $D(x,x')$ which obeys
\begin{equation}
\left( \nabla_\rho \nabla^\rho - \frac{R}{D-1}\right)D(x,x') = g^{-1/2} \delta^D (x-x'),
\end{equation}
and solve for $\nabla^\sigma V_\sigma$, viz.
\begin{equation}
\nabla^\sigma V_\sigma = \frac{D}{2(D-1)} \int d^Dx' \sqrt{g}\ D(x,x')\nabla^\sigma \nabla^\tau h_{\sigma\tau}.
\end{equation}
Next, substitute $\nabla^\sigma V_\sigma$ into $\nabla^\nu h_{\mu\nu}^{L\theta}$, to yield
\begin{equation}
\nabla^\nu h_{\mu\nu}=  \left( \nabla_\rho \nabla^\rho - \frac{R}{D}\right) V_\mu
+ \frac{D-2}{2(D-1)} \nabla_\mu  \int d^Dx' \sqrt{g}\ D(x,x')\nabla^\sigma \nabla^\tau h_{\sigma\tau}.
\end{equation}
or
\begin{equation}
  \left( \nabla_\rho \nabla^\rho - \frac{R}{D}\right) V_\mu = \nabla^\nu h_{\mu\nu}- \frac{D-2}{2(D-1)} \nabla_\mu  \int d^Dx' \sqrt{g}\ D(x,x')\nabla^\sigma \nabla^\tau h_{\sigma\tau}.
\end{equation}
Introduce another scalar propogator $F(x,x')$, which obeys
\begin{equation}
\left( \nabla_\rho \nabla^\rho - \frac{R}{D}\right)F(x,x') = g^{-1/2} \delta^D(x-x'),
\end{equation}
whereby $V_\mu$ is solved as
\begin{equation}
V_\mu =   \int d^Dx' \sqrt{g}\ F(x,x') \nabla^\nu h_{\mu\nu}- \frac{D-2}{2(D-1)}  \int d^Dx' \sqrt{g}\ F(x,x') \nabla_\mu^{x'}  \int d^Dx'' \sqrt{g}\ D(x',x'')\nabla^\sigma \nabla^\tau h_{\sigma\tau}.
\end{equation}
Let us now introduce a tensor $h_{\mu\nu}^{tr}$, to facilitate expressing the entire $h_{\mu\nu}$ as
\begin{equation}
h_{\mu\nu} = h_{\mu\nu}^{L\theta} + h_{\mu\nu}^{T\theta} + h_{\mu\nu}^{tr}.
\end{equation}
For $h_{\mu\nu}$ to take this form, such a $h_{\mu\nu}^{tr}$ must obey
\begin{equation}
g^{\mu\nu}h_{\mu\nu}^{tr} = g^{\mu\nu}h_{\mu\nu},\qquad \nabla^\nu h_{\mu\nu}^{tr} = 0.
\end{equation}
With $h^{L\theta}_{\mu\nu}$ already obeying $\nabla^\nu h_{\mu\nu}^{L\theta} = \nabla^\nu h_{\mu\nu}$, $g^{\mu\nu}h_{\mu\nu}^{L\theta} = 0$, 
we see that $h^{T\theta}_{\mu\nu} = h_{\mu\nu} - h^{L\theta}_{\mu\nu} - h^{tr}_{\mu\nu}$ will be transverse and traceless as desired. As constructed in (C.93),
the tensor that satisfies our requirements is
\begin{equation}
h_{\mu\nu}^{tr} = \frac{1}{D-1} g_{\mu\nu} g^{\sigma\tau}h_{\sigma\tau}  - \frac{1}{D-1}\left( \nabla_\mu\nabla_\nu - \frac{1}{D(D-1)}g_{\mu\nu}R\right)
 \int d^Dx' \sqrt{g}\ D(x,x') g^{\sigma\tau}h_{\sigma\tau}.
\end{equation}
Consequently, we may express the entire $h_{\mu\nu}$ as
\begin{align}
h_{\mu\nu} &= h^{T\theta}_{\mu\nu} + \nabla_\mu V_\nu +\nabla_\nu V_\mu -\frac{2}{D} g_{\mu\nu} \nabla^\sigma V_\sigma+
 \frac{1}{D-1} g_{\mu\nu} g^{\sigma\tau}h_{\sigma\tau}
\nonumber\\
&\qquad - \frac{1}{D-1}\left( \nabla_\mu\nabla_\nu - \frac{1}{D(D-1)}g_{\mu\nu}R\right)
 \int d^Dx' \sqrt{g}\ D(x,x') g^{\sigma\tau}h_{\sigma\tau}.
\end{align}
To match the desired form for SVT decomposition, we will need to decompose the vectors $V_{\mu}$ into transverse and longitudinal components (denoted here as $W_\mu$ and $W$). This is achieved by introducing the scalar propagator
\begin{equation}
 \nabla_\rho \nabla^\rho A(x,x') = g^{-1/2} \delta^D(x-x'),
\end{equation}
whereby $V_\mu$ is deconstructed as
\begin{equation}
V_{\mu} = W_{\mu} + \nabla_\mu W,
\end{equation}
with
\begin{equation}
W = \int d^Dx' \sqrt{g}\ A(x,x') \nabla^\sigma V_\sigma,\qquad W_{\mu} = V_{\mu} - \nabla_\mu W.
\end{equation}
The full $h_{\mu\nu}$ then takes the form
\begin{align}
h_{\mu\nu} &= h^{T\theta}_{\mu\nu} +\nabla_\mu W_\nu +\nabla_\nu W_\mu +2\nabla_\mu\nabla_\nu W - \frac{2}{D}g_{\mu\nu} \nabla_\sigma \nabla^\sigma W +
 \frac{1}{D-1} g_{\mu\nu} g^{\sigma\tau}h_{\sigma\tau}
\nonumber\\
&\qquad - \frac{1}{D-1}\left( \nabla_\mu\nabla_\nu - \frac{1}{D(D-1)}g_{\mu\nu}R\right)
 \int d^Dx' \sqrt{g}\ D(x,x') g^{\sigma\tau}h_{\sigma\tau}.
\end{align}
Upon defining
\begin{align}
2\psi &= \frac{2}{D} \nabla_\sigma \nabla^\sigma W - \frac{1}{D-1}g^{\sigma\tau}h_{\sigma\tau} - \frac{R}{D(D-1)^2} \int d^Dx' \sqrt{g}\ D(x,x') g^{\sigma\tau}h_{\sigma\tau}
\nonumber\\
2E &= 2W - \frac{1}{D-1}  \int d^Dx' \sqrt{g}\ D(x,x') g^{\sigma\tau}h_{\sigma\tau}
\nonumber\\
E_\mu &= W_\mu
\nonumber\\
2 E_{\mu\nu} &= h^{T\theta}_{\mu\nu},
\end{align}
$h_{\mu\nu}$ may be written in the SVT form
\begin{equation}
h_{\mu\nu} = -2\psi g_{\mu\nu} +2 \nabla_\mu \nabla_\nu E + \nabla_\mu E_\nu + \nabla_\nu E_\mu + 2E_{\mu\nu}.
\end{equation}
\begin{equation}
V_\mu =   \int d^Dx' \sqrt{g}\ F(x,x') \nabla^\nu h_{\mu\nu}- \frac{D-2}{2(D-1)}  \int d^Dx' \sqrt{g}\ F(x,x') \nabla_\mu^{x'}  \int d^Dx'' \sqrt{g}\ D(x',x'')\nabla^\sigma \nabla^\tau h_{\sigma\tau}.
\end{equation}
\begin{equation}
\nabla^\sigma V_{\sigma} = 
\end{equation}
\\ \\
\begin{equation}
V_\mu = W_\mu + \nabla_\mu W,\qquad h_{\mu\nu} = \nabla_\mu W_\nu + \nabla_\nu W_\mu + 2 \nabla_\mu \nabla_\nu W
\end{equation}
\begin{equation}
\nabla^\nu h_{\mu\nu} = \nabla_\sigma \nabla^\sigma W_{\mu} + 2\nabla_\sigma \nabla^\sigma \nabla_\mu W
\end{equation}
\begin{equation}
\nabla^\mu \nabla^\nu h_{\mu\nu} = 2\nabla_\rho \nabla^\rho \nabla_\sigma \nabla^\sigma W
\end{equation}
\begin{equation}
\nabla_\sigma \nabla^\sigma W = \frac12 \int D(x,x') \nabla^\mu \nabla^\nu h_{\mu\nu} = \nabla^\sigma V_\sigma
\end{equation}
\begin{equation}
\nabla_\sigma \nabla^\sigma W_\mu = \nabla^\nu h_{\mu\nu} - 2\nabla_\mu \nabla_\sigma \nabla^\sigma W
\end{equation}
\begin{equation}
W_\mu = \int D(x,x')\nabla^\nu h_{\mu\nu} - \int D(x,x') \nabla_\mu \int D(x',x'')\nabla^\mu \nabla^\nu h_{\mu\nu}
\end{equation}
\begin{align}
V_{\mu} &= W_\mu + \nabla_\mu W
\\
&=  \int D(x,x')\nabla^\sigma h_{\mu\sigma} - \int D(x,x') \nabla_\mu \int D(x',x'')\nabla^\sigma \nabla^\rho h_{\sigma\rho} + \frac{1}{2}\nabla_\mu 
\int D(x,x') \int D(x',x'') \nabla^\sigma \nabla^\rho h_{\sigma\rho}
\end{align}
\\ \\
\begin{equation}
V_\mu =   \int d^Dx' \sqrt{g}\ F(x,x') \nabla^\nu h_{\mu\nu}- \frac{D-2}{2(D-1)}  \nabla_\mu \int d^Dx' \sqrt{g}\ F(x,x') \nabla^\sigma  \int d^Dx'' \sqrt{g}\ D(x',x'') \nabla^\tau h_{\sigma\tau}.
\end{equation}
\begin{equation}
\nabla^\nu h_{\mu\nu}^{L\theta} = \nabla^\nu h_{\mu\nu}= \left( \nabla_\rho \nabla^\rho - \frac{R}{D}\right) V_\mu + \frac{D-2}{D} \nabla_\mu \nabla^\sigma V_\sigma,
\end{equation}
\begin{align}
\nabla^\mu \nabla^\nu h_{\mu\nu}^{L\theta} &=\nabla^\mu \nabla^\nu h_{\mu\nu}= 2\left(\frac{D-1}{D}\nabla_\rho \nabla^\rho - \frac{R}{D}\right) \nabla^\sigma V_\sigma 
\nonumber
\\
\to& \quad \frac{D}{2(D-1)}\nabla^\mu \nabla^\nu h_{\mu\nu}  = \left( \nabla_\rho \nabla^\rho - \frac{R}{D-1}\right)\nabla^\sigma V_\sigma,
\end{align}
\begin{equation}
\nabla^\sigma V_\sigma = \nabla^\sigma \int F(x,x') \nabla^\rho h_{\sigma\rho} - \frac{D-2}{2(D-1)} \nabla^\sigma \int D(x,x') \nabla^\tau h_{\sigma\tau}
+\frac{D-2}{2(D-1)} \frac{R}{D} \int F(x,x') \nabla^\sigma \int D(x',x'') \nabla^\tau h_{\sigma\tau}
\end{equation}
\begin{align}
\left( \nabla_\rho \nabla^\rho - \frac{R}{D}\right) V_\mu 
= \nabla^\nu h_{\mu\nu} - \frac{D-2}{2(D-1)}\nabla_\mu \nabla^\sigma \int d^Dx' \sqrt{g}\ D(x,x') \nabla^\tau h_{\sigma\tau}.
\end{align}
\\ \\
\begin{equation}
V^{\mu} =   \int d^Dx' \sqrt{g}\ D(x,x') \nabla_\sigma h^{\mu\sigma} - \frac12\nabla^\mu 
  \int d^Dx' \sqrt{g}\ D(x,x')\nabla_\sigma   \int d^Dx'' \sqrt{g}\ D(x',x'') \nabla_\rho h^{\sigma\rho}.
\end{equation}
Now we can construct the longitudinal tensor $h^{\mu\nu}_L = \nabla^\mu V^\nu + \nabla^\nu V^\mu$, 
\begin{align}
  h^{\mu\nu}_L&=\nabla^\mu \int d^Dx' \sqrt{g}\ D(x,x')\nabla_\sigma h^{\sigma\nu} + \nabla^\nu \int d^Dx' \sqrt{g}\  D(x,x')\nabla_\sigma h^{\sigma\mu} 
\\
&\qquad -\nabla^\mu\nabla^\nu \int d^Dx'\sqrt{g}\  D(x,x') \nabla_\sigma \int d^Dx'' \sqrt{g}\ D(x',x'')\nabla_\rho h^{\sigma\rho}.
\end{align}
\begin{equation}
W^{\mu} = \int d^Dx' \sqrt{g}\ D(x,x')\nabla_\sigma h^{\sigma\mu}
\end{equation}
\begin{equation}
h_{\mu\nu}^L = \nabla_\mu W_\nu + \nabla_\nu W_\mu - \nabla_\mu\nabla_\nu \int d^Dx' \sqrt{g}\ D(x,x')\nabla^\sigma W_\sigma
\end{equation}
\begin{align}
h_{\mu\nu}^{L\theta} &= h_{\mu\nu}^L - \frac{1}{D-1} g_{\mu\nu} g^{\sigma\rho}h_{\sigma\rho}^L
+ \frac{1}{D-1}\left[ \nabla_\mu\nabla_\nu - \frac{1}{D(D-1)}Rg_{\mu\nu}\right]\int d^Dx' \sqrt{g}\ F(x,x')g^{\sigma\rho}h_{\sigma\rho}^L
\end{align}
\begin{equation}
g^{\mu\nu}h_{\mu\nu}^{L} = \nabla^\sigma W_\sigma - \frac{R}{D} \int d^Dx' \sqrt{g}\ D(x,x')\nabla^\sigma W_\sigma
\end{equation}
\begin{align}
h_{\mu\nu}^{L\theta} &= \nabla_\mu W_\nu + \nabla_\nu W_\mu 
\nonumber\\
\quad &+ \nabla_\mu\nabla_\nu \left[ - \int d^Dx' \sqrt{g}\ D(x,x')\nabla^\sigma W_\sigma + \frac{1}{D-1} \int d^Dx' \sqrt{g}\ F(x,x') \left( \nabla^\sigma W_\sigma-\frac{R}{D}\int d^Dx' \sqrt{g}\ D(x,x')\nabla^\sigma W_\sigma\right)\right]
\nonumber\\
&\qquad + \frac{g_{\mu\nu}}{D-1}\bigg[ - \nabla^\sigma W_\sigma - \frac{R}{D} \int d^Dx' \sqrt{g}\ D(x,x')\nabla^\sigma W_\sigma 
\nonumber \\
&\qquad
- \frac{1}{D(D-1)}R \int d^Dx' \sqrt{g}\ F(x,x') \left( \nabla^\sigma W_\sigma-\frac{R}{D}\int d^Dx' \sqrt{g}\ D(x,x')\nabla^\sigma W_\sigma\right)\bigg]
\nonumber
\end{align}
\begin{align}
h_{\mu\nu}^{L\theta} &=\nabla_\mu W_\nu + \nabla_\nu W_\mu - \nabla_\mu\nabla_\nu \int d^Dx' \sqrt{g}\ D(x,x')\nabla^\sigma W_\sigma
\nonumber\\
&\qquad - \frac{1}{D-1} g_{\mu\nu} \left( g^{\sigma\rho}h_{\sigma\rho}^L + \frac{1}{D(D-1)}R
\int d^Dx' \sqrt{g}\ F(x,x')g^{\sigma\rho}h_{\sigma\rho}^L\right)
\nonumber\\
&\qquad
+ \frac{1}{D-1}\nabla_\mu\nabla_\nu \int d^Dx' \sqrt{g}\ F(x,x')g^{\sigma\rho}h_{\sigma\rho}^L
\end{align}
\begin{equation}
V_\mu =  W_\mu- \frac{D-2}{2(D-1)}  \int d^Dx' \sqrt{g}\ D(x,x') \nabla_\mu^{x'}  \int d^Dx'' \sqrt{g}\ F(x',x'')\nabla^\sigma \nabla^\tau h_{\sigma\tau}.
\end{equation}
\begin{equation}
W_\mu =  V_\mu+ \frac{D-2}{2(D-1)}  \int d^Dx' \sqrt{g}\ D(x,x') \nabla_\mu^{x'}  \int d^Dx'' \sqrt{g}\ F(x',x'')\nabla^\sigma \nabla^\tau h_{\sigma\tau}.
\end{equation}
\begin{equation}
V_\mu =   \int d^Dx' \sqrt{g}\ F(x,x') \nabla^\nu h_{\mu\nu}- \frac{D-2}{2(D-1)}  \int d^Dx' \sqrt{g}\ F(x,x') \nabla_\mu^{x'}  \int d^Dx'' \sqrt{g}\ D(x',x'')\nabla^\sigma \nabla^\tau h_{\sigma\tau}.
\end{equation}
\newpage
\begin{align}
h_{\mu\nu}^{L\theta} &= \nabla_\mu W_\nu + \nabla_\nu W_\mu 
\nonumber\\
\quad &+ \nabla_\mu\nabla_\nu \left[ - \int d^Dx' \sqrt{g}\ D(x,x')\nabla^\sigma W_\sigma + \frac{1}{D-1} \int d^Dx' \sqrt{g}\ F(x,x') \left( \nabla^\sigma W_\sigma-\frac{R}{D}\int d^Dx' \sqrt{g}\ D(x,x')\nabla^\sigma W_\sigma\right)\right]
\nonumber\\
&\qquad + \frac{g_{\mu\nu}}{D-1}\bigg[ - \nabla^\sigma W_\sigma - \frac{R}{D} \int d^Dx' \sqrt{g}\ D(x,x')\nabla^\sigma W_\sigma 
\nonumber \\
&\qquad
- \frac{1}{D(D-1)}R \int d^Dx' \sqrt{g}\ F(x,x') \left( \nabla^\sigma W_\sigma-\frac{R}{D}\int d^Dx' \sqrt{g}\ D(x,x')\nabla^\sigma W_\sigma\right)\bigg]
\nonumber
\end{align}
\begin{align}
W_\mu &= W_\mu^T + \nabla_\mu W\\
&=W_\mu^T + \nabla_\mu \int d^Dx' \sqrt{g}\ A(x,x')\nabla^\sigma \int d^Dx'' \sqrt{g}\ (x',x'') \nabla^\rho h_{\sigma\rho}
\end{align}
\begin{equation}
\nabla_\rho \nabla^\rho W = \nabla^\sigma W_\sigma =\nabla^\sigma \int d^Dx'' \sqrt{g}\ (x',x'') \nabla^\rho h_{\sigma\rho}
\end{equation}
\begin{align}
h_{\mu\nu}^{L\theta} &= \nabla_\mu W_\nu^T + \nabla_\nu W_\mu^T  
\nonumber\\
\quad &+ \nabla_\mu\nabla_\nu \bigg[ 2W- \int d^Dx' \sqrt{g}\ D(x,x')\nabla_\rho \nabla^\rho W
\nonumber\\
&\qquad  + \frac{1}{D-1} \int d^Dx' \sqrt{g}\ F(x,x') \left( \nabla_\rho \nabla^\rho W-\frac{R}{D}\int d^Dx' \sqrt{g}\ D(x,x')\nabla_\rho \nabla^\rho W\right)\bigg]
\nonumber\\
&\qquad + \frac{g_{\mu\nu}}{D-1}\bigg[ -\nabla_\rho \nabla^\rho W - \frac{R}{D} \int d^Dx' \sqrt{g}\ D(x,x')\nabla_\rho \nabla^\rho W 
\nonumber \\
&\qquad
- \frac{1}{D(D-1)}R \int d^Dx' \sqrt{g}\ F(x,x') \left(\nabla_\rho \nabla^\rho W-\frac{R}{D}\int d^Dx' \sqrt{g}\ D(x,x')\nabla_\rho \nabla^\rho W\right)\bigg]
\nonumber
\end{align}
Upon defining
\begin{align}
2\psi &=  \frac{1}{D-1}\bigg[ \nabla_\rho \nabla^\rho W + \frac{R}{D} \int d^Dx' \sqrt{g}\ D(x,x')\nabla_\rho \nabla^\rho W 
\nonumber \\
&\qquad
+ \frac{1}{D(D-1)}R \int d^Dx' \sqrt{g}\ F(x,x') \left(\nabla_\rho \nabla^\rho W-\frac{R}{D}\int d^Dx' \sqrt{g}\ D(x,x')\nabla_\rho \nabla^\rho W\right)\bigg]
\nonumber\\
2E &= 2W - \frac{1}{D-1}  \int d^Dx' \sqrt{g}\ D(x,x') g^{\sigma\tau}h_{\sigma\tau}
\nonumber\\
E_\mu &= W_\mu
\nonumber\\
2 E_{\mu\nu} &= h^{T\theta}_{\mu\nu},
\end{align}
\end{document}