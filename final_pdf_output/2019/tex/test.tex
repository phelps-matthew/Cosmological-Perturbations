\documentclass[10pt,letterpaper]{article}
\usepackage[textwidth=7in, top=1in,textheight=9in]{geometry}
\usepackage[fleqn]{mathtools} 
\usepackage{amssymb,braket,hyperref}
\usepackage[title]{appendix}
\numberwithin{equation}{section}
\setlength{\parindent}{0pt}
\title{3-Space Einstein Tensor Gauge Dependence v2}
\date{}
\begin{document} 
\maketitle
\noindent 
%%%%%%%%%%%%%%%%%%%%%%%%%%%%%%%%%
\section{Covariant Form $\delta G_{ij} = \delta T_{ij}$} 
\label{S1}
Within the geometry of 
\begin{eqnarray}
ds^2 = (g_{ij}^{(0)} + h_{ij})dx^i dx^j 
\end{eqnarray}
with maximally symmetric background
\begin{eqnarray}
g_{ij}^{(0)} = \begin{pmatrix} \frac{1}{1-kr^2} &0&0\\ 0&r^2&0\\0&0&r^2\sin^2\theta\end{pmatrix}
\end{eqnarray}
assume the metric perturbation can be (covariant) SVT decomposed as
\begin{eqnarray}
h_{ij} = -2 g_{ij}\psi + 2\nabla_i\nabla_j E + \nabla_i E_j + \nabla_j E_i + 2E_{ij},
\end{eqnarray}
with 3-trace
\begin{eqnarray}
h = -6 \psi + 2\nabla^a\nabla_a E.
\end{eqnarray}
The three dimensional Einstein background field equations take the form $G_{ij}^{(0)} = -T_{ij}^{(0)}$. Since the background is maximally symmetric, the solution to the zeroth order Einstein equations yields energy momentum tensor $T_{ij}^{(0)} = \Lambda g_{ij}^{(0)} = k g_{ij}^{(0)}$.
\\ \\
The perturbed Einstein equations then take the form,
\begin{eqnarray}
\delta G_{ij} &=& -\delta T_{ij}\\
&=& k h_{ij} 
\end{eqnarray}
Evaluating the Einstein tensor in terms of (3) yields
\begin{eqnarray}
\delta G_{ij}=\tfrac{1}{2} \nabla_{a}\nabla^{a}h_{ij}
 -  \tfrac{1}{2} g_{ij} \nabla_{a}\nabla^{a}h
 + \tfrac{1}{2} g_{ij} \nabla_{b}\nabla_{a}h^{ab}
 -  \tfrac{1}{2} \nabla_{i}\nabla_{a}h_{j}{}^{a}
 -  \tfrac{1}{2} \nabla_{j}\nabla_{a}h_{i}{}^{a}
 + \tfrac{1}{2} \nabla_{j}\nabla_{i}h,
\end{eqnarray}
which takes the SVT form
\begin{eqnarray}
\delta G_{ij}=\nabla_{a}\nabla^{a}E_{ij}
 + g_{ij} \nabla_{a}\nabla^{a}\psi
 + k \nabla_{i}E_{j}
 + k \nabla_{j}E_{i}
 + 2 k \nabla_{j}\nabla_{i}E
 -  \nabla_{j}\nabla_{i}\psi.
\label{dGsvt}
\end{eqnarray}
Composing the field equation $\delta G_{\mu\nu} =-\delta T_{\mu\nu}$ yields
\begin{eqnarray}
&&\nabla_{a}\nabla^{a}E_{ij}
 +g_{ij} \nabla_{a}\nabla^{a}\psi
 + k \nabla_{i}E_{j}
 + k \nabla_{j}E_{i}
 + 2 k \nabla_{j}\nabla_{i}E
 -  \nabla_{j}\nabla_{i}\psi =
\nonumber \\
&& k (-2 g_{ij}\psi + 2\nabla_i\nabla_j E + \nabla_i E_j + \nabla_j E_i + 2E_{ij}),
\end{eqnarray}
which may be simplified as
\begin{eqnarray}
\boxed{
(\nabla_a \nabla^a-2k)E_{ij} + g_{ij}\nabla_a \nabla^a \psi - \nabla_j\nabla_i \psi+2k g_{ij}\psi = 0}.
\label{eqref1}
\end{eqnarray}
\begin{eqnarray}
\boxed{
(\nabla_a \nabla^a + 3k)\psi = 0}.
\end{eqnarray}
Utilizing the trace above, we see that 
Under gauge transformation 
\begin{eqnarray}
\bar\psi &=&\psi
\nonumber\\
\bar E &=& E-L
\nonumber\\
\bar E_i &=& E_i - L_i
\nonumber\\
\bar E_{ij} &=& E_{ij}
\end{eqnarray}
\begin{eqnarray}
\nabla^i \nabla^j h_{ij} = -2\nabla^i \nabla_i \psi + 2\nabla^i\nabla_i \nabla^j\nabla_j E + 2k \nabla_i \nabla^i E
\end{eqnarray}
\begin{eqnarray}
\nabla^j \delta G_{ij} =  -2k\nabla_i \psi + k(\nabla^a\nabla_a + 2k)E_i + 2k \nabla^a\nabla_a \nabla_i E 
\end{eqnarray}
\begin{eqnarray}
\nabla^i \nabla^j \delta G_{ij} = -2k\nabla^a \nabla_a \psi + 2k \nabla^a\nabla_a(\nabla^b \nabla_b + 2k)E
\end{eqnarray}
%%%%%%%%%%%%%%%%%%%%%%%%%


\section{Conformal to Flat $\delta G_{ij} = \delta T_{ij}$}
The 3-space of constant curvature can be expressed in the conformal flat form as in \eqref{k>0cf}

\begin{eqnarray}
ds^2 &=& \Omega^2(\rho)\left( d\rho^2 + \rho^2 d\Omega^2\right)
\nonumber\\
&=& \frac{4}{\left(1+k \rho^2\right)^2}\left( d\rho^2 + \rho^2 d\Omega^2\right)
\label{cfbg}
\end{eqnarray}

Within the above geometry, the perturbed Einstein tensor takes the form (with $\tilde \nabla$ denoting flat space derivative)

\begin{eqnarray}
\delta G_{ij}=&&g_{ij} \tilde{\nabla}_{a}\tilde{\nabla}^{a}\psi
 + 2 g_{ij} \Omega^{-1} \tilde{\nabla}^{a}\Omega \tilde{\nabla}_{b}\tilde{\nabla}^{b}\tilde{\nabla}_{a}E
 - 2 g_{ij} \Omega^{-2} \tilde{\nabla}^{a}\Omega \tilde{\nabla}_{b}\tilde{\nabla}_{a}E \tilde{\nabla}^{b}\Omega\nonumber\\
&& + 4 g_{ij} \Omega^{-1} \tilde{\nabla}_{b}\tilde{\nabla}_{a}\Omega \tilde{\nabla}^{b}\tilde{\nabla}^{a}E
 + 2 \Omega^{-1} \tilde{\nabla}_{i}\Omega \tilde{\nabla}_{j}\psi
 + 2 \Omega^{-1} \tilde{\nabla}_{i}\psi \tilde{\nabla}_{j}\Omega
 - 4 \Omega^{-1} \tilde{\nabla}_{a}\tilde{\nabla}^{a}\Omega \tilde{\nabla}_{j}\tilde{\nabla}_{i}E
\nonumber\\
&& + 2 \Omega^{-2} \tilde{\nabla}_{a}\Omega \tilde{\nabla}^{a}\Omega \tilde{\nabla}_{j}\tilde{\nabla}_{i}E
 -  \tilde{\nabla}_{j}\tilde{\nabla}_{i}\psi
 - 2 \Omega^{-1} \tilde{\nabla}^{a}\Omega \tilde{\nabla}_{j}\tilde{\nabla}_{i}\tilde{\nabla}_{a}E
\nonumber\\ \nonumber\\
&& +2 g_{ij} \Omega^{-1} \tilde{\nabla}^{a}\Omega \tilde{\nabla}_{b}\tilde{\nabla}^{b}E_{a}
 - 2 g_{ij} \Omega^{-2} \tilde{\nabla}_{a}\Omega \tilde{\nabla}_{b}\Omega \tilde{\nabla}^{b}E^{a}
 + 4 g_{ij} \Omega^{-1} \tilde{\nabla}_{b}\tilde{\nabla}_{a}\Omega \tilde{\nabla}^{b}E^{a}
\nonumber\\
&& - 2 \Omega^{-1} \tilde{\nabla}_{a}\tilde{\nabla}^{a}\Omega \tilde{\nabla}_{i}E_{j}
 + \Omega^{-2} \tilde{\nabla}_{a}\Omega \tilde{\nabla}^{a}\Omega \tilde{\nabla}_{i}E_{j}
 - 2 \Omega^{-1} \tilde{\nabla}_{a}\tilde{\nabla}^{a}\Omega \tilde{\nabla}_{j}E_{i}
\nonumber\\
&& + \Omega^{-2} \tilde{\nabla}_{a}\Omega \tilde{\nabla}^{a}\Omega \tilde{\nabla}_{j}E_{i}
 - 2 \Omega^{-1} \tilde{\nabla}^{a}\Omega \tilde{\nabla}_{j}\tilde{\nabla}_{i}E_{a}
\nonumber \\ \nonumber\\
&&+\tilde{\nabla}_{a}\tilde{\nabla}^{a}E_{ij}
 - 4 E_{ij} \Omega^{-1} \tilde{\nabla}_{a}\tilde{\nabla}^{a}\Omega
 + 2 \Omega^{-1} \tilde{\nabla}_{a}E_{ij} \tilde{\nabla}^{a}\Omega
 + 2 E_{ij} \Omega^{-2} \tilde{\nabla}_{a}\Omega \tilde{\nabla}^{a}\Omega\nonumber\\
&& + 4 E^{ab} g_{ij} \Omega^{-1} \tilde{\nabla}_{b}\tilde{\nabla}_{a}\Omega
 - 2 E_{ab} g_{ij} \Omega^{-2} \tilde{\nabla}^{a}\Omega \tilde{\nabla}^{b}\Omega
 - 2 \Omega^{-1} \tilde{\nabla}^{a}\Omega \tilde{\nabla}_{i}E_{ja}
 - 2 \Omega^{-1} \tilde{\nabla}^{a}\Omega \tilde{\nabla}_{j}E_{ia}
\end{eqnarray}
with energy momentum tensor
\begin{eqnarray}
\delta T_{ij} = k\Omega^2 h_{ij}= k \Omega^2 (-2 g_{ij}\psi + 2\tilde\nabla_i\tilde\nabla_j E + \tilde\nabla_i E_j + \nabla_j E_i + 2E_{ij})
\end{eqnarray}
%%%%%%%%%%%%%%%%%%%%%%%%%%%%%%%%%%%%%%%%%%%%%%%%%%%%%%
\newpage
\begin{appendices}
\section{Conformal to Flat Maximal 3-Space}
%%%%%%%%%%%%%%
\subsection{$k<0$}
The 3-space of constant curvature can be expressed in the conformal flat form (using $-k = 1/L^2$) as
\begin{eqnarray}
ds^2 &=& \Omega^2(\rho)\left( d\rho^2 + \rho^2 d\Omega^2\right)\\
&=& \frac{4}{\left(1-\rho^2/L^2\right)^2}\left( d\rho^2 + \rho^2 d\Omega^2\right)\\
&=& \frac{dr^2}{1+r^2/L^2} + r^2 d\Omega^2
\end{eqnarray}
The relevant transformations are:
\begin{eqnarray}
\rho(r) &=& \frac{r}{1+\left(1+r^2/L^2\right)^{1/2}},\qquad \Omega^2(r) = \left(1+\left[1+r^2/L^2\right)^{1/2}\right]^2
\nonumber\\
r(\rho) &=& \frac{2\rho}{1-\rho^2/L^2},\qquad \Omega^2(\rho) = \frac{4}{\left(1-\rho^2/L^2\right)^2}
\end{eqnarray}

%%%%%%%%%
\subsection{$k>0$}
Now instead we set $k = 1/L^2$ to express the line element as
\begin{eqnarray}
ds^2 &=& \Omega^2(\rho)\left( d\rho^2 + \rho^2 d\Omega^2\right)\\
&=& \frac{4}{\left(1+\rho^2/L^2\right)^2}\left( d\rho^2 + \rho^2 d\Omega^2\right)
\label{k>0cf}\\
&=& \frac{dr^2}{1-r^2/L^2} + r^2 d\Omega^2
\end{eqnarray}
The relevant transformations are:
\begin{eqnarray}
\rho(r) &=& \frac{r}{1+\left(1-r^2/L^2\right)^{1/2}},\qquad \Omega^2(r) = \left[1+\left(1-r^2/L^2\right)^{1/2}\right]
\nonumber\\
r(\rho) &=& \frac{2\rho}{1+\rho^2/L^2},\qquad \Omega^2(\rho) = \frac{4}{\left(1+\rho^2/L^2\right)^2}
\end{eqnarray}
After calculation, we see that solutions to positive/negative geometries are affected by $L^2 \to - L^2$. This is not quite the case in 4D comoving RW, where we must make use of trigonometric and hyperbolic transformations depending on the sign of the curvature.

%%%%%%%%%%%%%%
\section{$\delta G_{ij}$ Under Conformal Transformation}
Under general conformal transformation $g_{ij}\to \Omega^2(x)g_{ij}$, the  Einstein tensor transforms as
\begin{eqnarray}
G_{ij} &\to& G_{ij} + S_{ij}
\nonumber\\
&=& G_{ij} +
\Omega^{-1}\left( -2g_{ij}\tilde\nabla^a \tilde\nabla_a \Omega + 2\tilde\nabla_i\tilde \nabla_j \Omega\right) +
\Omega^{-2}\left( g_{ij} \tilde\nabla_a \Omega \tilde\nabla^a \Omega - 4\tilde \nabla_i \Omega \tilde\nabla_j \Omega\right).
\end{eqnarray}
Perturbing the above to first order yields the transformation of $\delta G_{ij}$:
\begin{eqnarray}
\delta G_{ij} \to \delta G_{ij} + \delta S_{ij},
\label{dGsum}
\end{eqnarray}
where
\begin{eqnarray}
\delta S_{ij}&=&-2 h_{ij} \Omega^{-1} \tilde{\nabla}_{a}\tilde{\nabla}^{a}\Omega
 -  g_{ij} \Omega^{-1} \tilde{\nabla}_{a}\Omega \tilde{\nabla}^{a}h
 + \Omega^{-1} \tilde{\nabla}_{a}h_{ij} \tilde{\nabla}^{a}\Omega
 + h_{ij} \Omega^{-2} \tilde{\nabla}_{a}\Omega \tilde{\nabla}^{a}\Omega\nonumber\\
&& + 2 g_{ij} \Omega^{-1} \tilde{\nabla}^{a}\Omega \tilde{\nabla}_{b}h_{a}{}^{b}
 + 2 g_{ij} h^{ab} \Omega^{-1} \tilde{\nabla}_{b}\tilde{\nabla}_{a}\Omega
 -  g_{ij} h_{ab} \Omega^{-2} \tilde{\nabla}^{a}\Omega \tilde{\nabla}^{b}\Omega
 -  \Omega^{-1} \tilde{\nabla}^{a}\Omega \tilde{\nabla}_{i}h_{ja}\nonumber\\
&&-  \Omega^{-1} \tilde{\nabla}^{a}\Omega \tilde{\nabla}_{j}h_{ia}.
\end{eqnarray}

In the conformal to flat metric , $\delta G_{ij}$ as defined by  takes the same form as with $k=0$,  and $\delta S_{ij}$ evaluates to

\begin{eqnarray}
\delta S_{ij}=&&2 g_{ij} \Omega^{-1} \tilde{\nabla}^{a}\Omega \tilde{\nabla}_{b}\tilde{\nabla}^{b}\tilde{\nabla}_{a}E
 - 2 g_{ij} \Omega^{-2} \tilde{\nabla}^{a}\Omega \tilde{\nabla}_{b}\tilde{\nabla}_{a}E \tilde{\nabla}^{b}\Omega
 + 4 g_{ij} \Omega^{-1} \tilde{\nabla}_{b}\tilde{\nabla}_{a}\Omega \tilde{\nabla}^{b}\tilde{\nabla}^{a}E
\nonumber\\
&& + 2 \Omega^{-1} \tilde{\nabla}_{i}\Omega \tilde{\nabla}_{j}\psi
 + 2 \Omega^{-1} \tilde{\nabla}_{i}\psi \tilde{\nabla}_{j}\Omega
 - 4 \Omega^{-1} \tilde{\nabla}_{a}\tilde{\nabla}^{a}\Omega \tilde{\nabla}_{j}\tilde{\nabla}_{i}E
 + 2 \Omega^{-2} \tilde{\nabla}_{a}\Omega \tilde{\nabla}^{a}\Omega \tilde{\nabla}_{j}\tilde{\nabla}_{i}E
\nonumber\\
&& - 2 \Omega^{-1} \tilde{\nabla}^{a}\Omega \tilde{\nabla}_{j}\tilde{\nabla}_{i}\tilde{\nabla}_{a}E
\nonumber\\
&&+2 g_{ij} \Omega^{-1} \tilde{\nabla}^{a}\Omega \tilde{\nabla}_{b}\tilde{\nabla}^{b}E_{a}
 - 2 g_{ij} \Omega^{-2} \tilde{\nabla}_{a}\Omega \tilde{\nabla}_{b}\Omega \tilde{\nabla}^{b}E^{a}
 + 4 g_{ij} \Omega^{-1} \tilde{\nabla}_{b}\tilde{\nabla}_{a}\Omega \tilde{\nabla}^{b}E^{a}
\nonumber\\
&& - 2 \Omega^{-1} \tilde{\nabla}_{a}\tilde{\nabla}^{a}\Omega \tilde{\nabla}_{i}E_{j}
 + \Omega^{-2} \tilde{\nabla}_{a}\Omega \tilde{\nabla}^{a}\Omega \tilde{\nabla}_{i}E_{j}
 - 2 \Omega^{-1} \tilde{\nabla}_{a}\tilde{\nabla}^{a}\Omega \tilde{\nabla}_{j}E_{i}
\nonumber\\
&& + \Omega^{-2} \tilde{\nabla}_{a}\Omega \tilde{\nabla}^{a}\Omega \tilde{\nabla}_{j}E_{i}
 - 2 \Omega^{-1} \tilde{\nabla}^{a}\Omega \tilde{\nabla}_{j}\tilde{\nabla}_{i}E_{a}
\nonumber\\
&&-4 E_{ij} \Omega^{-1} \tilde{\nabla}_{a}\tilde{\nabla}^{a}\Omega
 + 2 \Omega^{-1} \tilde{\nabla}_{a}E_{ij} \tilde{\nabla}^{a}\Omega
 + 2 E_{ij} \Omega^{-2} \tilde{\nabla}_{a}\Omega \tilde{\nabla}^{a}\Omega
 + 4 E^{ab} g_{ij} \Omega^{-1} \tilde{\nabla}_{b}\tilde{\nabla}_{a}\Omega\nonumber\\
&& - 2 E_{ab} g_{ij} \Omega^{-2} \tilde{\nabla}^{a}\Omega \tilde{\nabla}^{b}\Omega
 - 2 \Omega^{-1} \tilde{\nabla}^{a}\Omega \tilde{\nabla}_{i}E_{ja}
 - 2 \Omega^{-1} \tilde{\nabla}^{a}\Omega \tilde{\nabla}_{j}E_{ia}.
\label{dSsvt}
\end{eqnarray}

%%%%%%%%%%%%%%
\section{RW Christoffels and Geometric Tensors}
\begin{eqnarray}
R_{ijkl} = k(g_{jk}g_{il}-g_{ik}g_{jl}),\qquad R_{ij} = -2kg_{ij},\qquad R = -6k
\end{eqnarray}
\begin{eqnarray}
[\nabla_i,\nabla_j]V_k = V_m R^m{}_{kij}= k (g_{ki}g^{m}{}_j - g^m{}_{i}g_{kj})V_m = k(g_{ik}V_j - g_{jk}V_i)
\label{covcom}
\end{eqnarray}
Christoffels for 
\begin{eqnarray}
ds^2 = g_{ij}dx^idx^j = \left( \frac{dr^2}{1-kr^2} + r^2 d\theta^2 + r^2\sin^2\theta d\phi^2\right):
\end{eqnarray}
\begin{eqnarray}
\Gamma^r_{rr} &=& \frac{kr}{1-kr^2},\qquad \Gamma^r_{\theta\theta} = -r(1-kr^2),\qquad \Gamma^r_{\phi\phi} = -r(1-kr^2)\sin^2\theta
\nonumber\\
\Gamma^\theta_{r\theta} &=& \Gamma^{\phi}_{r\phi} = \frac{1}{r},\qquad \Gamma^{\theta}_{\phi\phi} = -\sin\theta\cos\theta, \qquad \Gamma^{\phi}_{\theta\phi} = \cot\theta,
\end{eqnarray}
with all others zero. 
\end{appendices}
%%%%%%%%%%%%%%%%%%%%%%%%%%%%%%%%%