\documentclass[10pt,letterpaper]{article}
\usepackage[textwidth=7in, top=1in,textheight=9in]{geometry}
\usepackage[fleqn]{mathtools} 
\usepackage{amssymb,braket,hyperref,xcolor}
\hypersetup{colorlinks, linkcolor={blue!50!black}, citecolor={red!50!black}, urlcolor={blue!80!black}}
\usepackage[title]{appendix}
\numberwithin{equation}{section}
\setlength{\parindent}{0pt}
\title{Gravitational Invariants}
\date{}
\begin{document} 
\maketitle
\noindent 
%%%%%%%%%%%%%%%%%%%%%%%%%%%%%%%%%%%%%%%%%%%%%%%%%%%%%%%%
\section{Summary}
%%%%%%%%%%%%%%%%%%%%%%%%%%%%%%%%%%%%%%%%%%%%%%%%%%%%%%%%
In a Minkowski background, the two gravitational gauge invariants are $\delta R$ and $\delta R_{\mu\nu}$. With the Bianchi identities, this yields 6 independent gauge invariants, which are taken as $\delta G$ and $\delta G_{\mu\nu}^{T\theta}$.
\\ \\
In a dS$_4$, $\delta R$, $\delta R_{\mu\nu}$, and $\delta G_{\mu\nu}$ are not gauge invariant. However, we may construct a gravitational gauge invariant $\Delta_{\mu\nu} = \delta G_{\mu\nu} - 3kh_{\mu\nu}$. Being conserved, the 6 components are analogously $\Delta$ and $\Delta_{\mu\nu}^{T\theta}$. 
\\ \\
By virtue of $\delta G_{\mu\nu} = \delta T_{\mu\nu}$, Einstein gravity does not impose any equation of motion upon the gravitational gauge invariants - it merely equates gravitational gauge invariants to matter gauge invariants.
\\ \\
In conformal gravity, the gravitational invariants are dynamic. In a Minkowski background, the gravitational invariant obeys
\begin{eqnarray}
\delta W_{\mu\nu} &=& \nabla^2 \delta G_{\mu\nu}^{T\theta}
\nonumber\\ \nonumber\\
\to \nabla^2 \delta G_{\mu\nu}^{T\theta} &=& \delta T_{\mu\nu}
\end{eqnarray}
In a dS$_4$ background, we have determined
\begin{eqnarray}
\delta W_{\mu\nu} &=& (\nabla^2-4k) \Delta_{\mu\nu}^{T\theta}
\nonumber\\ \nonumber\\
\to (\nabla^2-4k) \Delta_{\mu\nu}^{T\theta} &=& \delta T_{\mu\nu}
\end{eqnarray}


%%%%%%%%%%%%%%%%%%%%%%%%%%%%%%%%%%
\section{Minkowski}
%%%%%%%%%%%%%%%%%%%%%%%%%%%%%%%%%%

\begin{eqnarray}
ds^2 &=& (\eta_{\mu\nu} + h_{\mu\nu})dx^\mu dx^\nu
\nonumber\\ \nonumber\\
\delta W_{\mu\nu}&=&\tfrac{1}{2} \nabla_{\beta}\nabla^{\beta}\nabla_{\alpha}\nabla^{\alpha}h_{\mu \nu}
-  \tfrac{1}{6} g_{\mu \nu} \nabla_{\beta}\nabla^{\beta}\nabla_{\alpha}\nabla^{\alpha}h
+ \tfrac{1}{6} g_{\mu \nu} \nabla_{\gamma}\nabla^{\gamma}\nabla_{\beta}\nabla_{\alpha}h^{\alpha \beta}
-  \tfrac{1}{2} \nabla_{\mu}\nabla_{\beta}\nabla^{\beta}\nabla_{\alpha}h_{\nu}{}^{\alpha}
\nonumber\\
&& -  \tfrac{1}{2} \nabla_{\nu}\nabla_{\beta}\nabla^{\beta}\nabla_{\alpha}h_{\mu}{}^{\alpha}
+ \tfrac{1}{6} \nabla_{\nu}\nabla_{\mu}\nabla_{\alpha}\nabla^{\alpha}h
+ \tfrac{1}{3} \nabla_{\nu}\nabla_{\mu}\nabla_{\beta}\nabla_{\alpha}h^{\alpha \beta}
\nonumber\\\nonumber\\
\delta G_{\mu\nu}&=&\tfrac{1}{2} \nabla_{\alpha}\nabla^{\alpha}h_{\mu \nu}
-  \tfrac{1}{2} g_{\mu \nu} \nabla_{\alpha}\nabla^{\alpha}h
+ \tfrac{1}{2} g_{\mu \nu} \nabla_{\beta}\nabla_{\alpha}h^{\alpha \beta}
-  \tfrac{1}{2} \nabla_{\mu}\nabla_{\alpha}h_{\nu}{}^{\alpha}
-  \tfrac{1}{2} \nabla_{\nu}\nabla_{\alpha}h_{\mu}{}^{\alpha}
+ \tfrac{1}{2} \nabla_{\nu}\nabla_{\mu}h
\nonumber\\\nonumber\\
\delta G &=&  \nabla^\alpha \nabla^\beta h_{\alpha\beta} - \nabla_\alpha\nabla^\alpha h
\nonumber\\\nonumber\\
\delta G^{T\theta}_{\mu\nu} &=& \delta G_{\mu\nu} - \frac{1}{3}g_{\mu\nu}\delta G + \frac{1}{3}\nabla_\mu\nabla_\nu \int D \delta G
\nonumber\\ \nonumber\\
\nabla^2\delta  G^{T\theta}_{\mu\nu} &=& \nabla^2 \delta G_{\mu\nu} + \frac{1}{3}\left[ 
\nabla_\mu\nabla_\nu - g_{\mu\nu}\nabla^2\right]\delta G
\nonumber\\ \nonumber\\
\nabla^2 \delta G_{\mu\nu}^{T\theta} 
&=& \tfrac{1}{2} \nabla_{\beta}\nabla^{\beta}\nabla_{\alpha}\nabla^{\alpha}h_{\mu \nu}
-  \tfrac{1}{6} g_{\mu \nu} \nabla_{\beta}\nabla^{\beta}\nabla_{\alpha}\nabla^{\alpha}h
+ \tfrac{1}{6} g_{\mu \nu} \nabla_{\gamma}\nabla^{\gamma}\nabla_{\beta}\nabla_{\alpha}h^{\alpha \beta}
-  \tfrac{1}{2} \nabla_{\mu}\nabla_{\beta}\nabla^{\beta}\nabla_{\alpha}h_{\nu}{}^{\alpha}
\nonumber\\
&& -  \tfrac{1}{2} \nabla_{\nu}\nabla_{\beta}\nabla^{\beta}\nabla_{\alpha}h_{\mu}{}^{\alpha}
+ \tfrac{1}{6} \nabla_{\nu}\nabla_{\mu}\nabla_{\alpha}\nabla^{\alpha}h
+ \tfrac{1}{3} \nabla_{\nu}\nabla_{\mu}\nabla_{\beta}\nabla_{\alpha}h^{\alpha \beta}
\nonumber\\
&=& \delta W_{\mu\nu}
\end{eqnarray}
%
%
%%%%%%%%%%%%%%%%%%%%%%%%%%%%%%%%
\subsection{Gauge Transformation}
%%%%%%%%%%%%%%%%%%%%%%%%%%%%%%%%
Under $x^\mu \to x'^{\mu} = x^\mu -\epsilon^\mu(x)$,

\begin{eqnarray}
\delta \bar W_{\mu\nu} &=& \delta W_{\mu\nu} + W^{(0)}_{\rho\mu}g^{\lambda\rho} \nabla_\nu \epsilon_\lambda + W^{(0)}_{\rho\nu}g^{\lambda\rho} \nabla_\mu \epsilon_\lambda + \epsilon^\lambda\nabla_\lambda W^{(0)}_{\mu\nu} 
\nonumber\\
&=&\delta W_{\mu\nu}
\nonumber\\\nonumber\\
\delta \bar G_{\mu\nu} &=& \delta G_{\mu\nu} + G^{(0)}_{\rho\mu}g^{\lambda\rho} \nabla_\nu \epsilon_\lambda + G^{(0)}_{\rho\nu}g^{\lambda\rho} \nabla_\mu \epsilon_\lambda + \epsilon^\lambda\nabla_\lambda G^{(0)}_{\mu\nu} 
\nonumber\\
&=&\delta G_{\mu\nu}
\end{eqnarray}

%
%
%%%%%%%%%%%%%%%%%%%%%%%%%%%%%%%%%%
\section{dS${}_4$}
%%%%%%%%%%%%%%%%%%%%%%%%%%%%%%%%%%
\begin{eqnarray}
G^{(0)}_{\mu\nu} &=& 3kg_{\mu\nu}
\nonumber\\
R^{(0)}_{\lambda\mu\nu\kappa} &=& k(g_{\mu\nu}g_{\lambda\kappa}-g_{\lambda\nu}g_{\mu\kappa})
\nonumber\\
R^{(0)}_{\mu\kappa} &=& -3k g_{\mu\kappa} = \frac{R}{D}g_{\mu\kappa}
\nonumber\\
R^{(0)}&=& -12 k
\nonumber\\ \nonumber\\
ds^2 &=& (g_{\mu\nu} + h_{\mu\nu})dx^\mu dx^\nu
\nonumber\\ \nonumber\\
\delta W_{\mu\nu}&=&4 k^2 h_{\mu \nu}
-  k^2 g_{\mu \nu} h
- 3 k \nabla_{\alpha}\nabla^{\alpha}h_{\mu \nu}
+ \tfrac{1}{2} k g_{\mu \nu} \nabla_{\alpha}\nabla^{\alpha}h
+ k g_{\mu \nu} \nabla_{\beta}\nabla_{\alpha}h^{\alpha \beta}\nonumber\\
&& + \tfrac{1}{2} \nabla_{\beta}\nabla^{\beta}\nabla_{\alpha}\nabla^{\alpha}h_{\mu \nu}
-  \tfrac{1}{6} g_{\mu \nu} \nabla_{\beta}\nabla^{\beta}\nabla_{\alpha}\nabla^{\alpha}h
+ \tfrac{1}{6} g_{\mu \nu} \nabla_{\gamma}\nabla^{\gamma}\nabla_{\beta}\nabla_{\alpha}h^{\alpha \beta}
-  \tfrac{1}{2} k \nabla_{\mu}\nabla_{\alpha}h_{\nu}{}^{\alpha}\nonumber\\
&& -  \tfrac{1}{2} \nabla_{\mu}\nabla_{\beta}\nabla^{\beta}\nabla_{\alpha}h_{\nu}{}^{\alpha}
-  \tfrac{1}{2} k \nabla_{\nu}\nabla_{\alpha}h_{\mu}{}^{\alpha}
-  \tfrac{1}{2} \nabla_{\nu}\nabla_{\beta}\nabla^{\beta}\nabla_{\alpha}h_{\mu}{}^{\alpha}
+ k \nabla_{\nu}\nabla_{\mu}h\nonumber\\
&& + \tfrac{1}{6} \nabla_{\nu}\nabla_{\mu}\nabla_{\alpha}\nabla^{\alpha}h
+ \tfrac{1}{3} \nabla_{\nu}\nabla_{\mu}\nabla_{\beta}\nabla_{\alpha}h^{\alpha \beta}.
\nonumber\\ \nonumber\\
\delta G_{\mu\nu}&=& 2 k h_{\mu \nu}
-  \tfrac{1}{2} k g_{\mu \nu} h
+ \tfrac{1}{2} \nabla_{\alpha}\nabla^{\alpha}h_{\mu \nu}
-  \tfrac{1}{2} g_{\mu \nu} \nabla_{\alpha}\nabla^{\alpha}h
+ \tfrac{1}{2} g_{\mu \nu} \nabla_{\beta}\nabla_{\alpha}h^{\alpha \beta}
-  \tfrac{1}{2} \nabla_{\mu}\nabla_{\alpha}h_{\nu}{}^{\alpha}\nonumber\\
&& -  \tfrac{1}{2} \nabla_{\nu}\nabla_{\alpha}h_{\mu}{}^{\alpha}
+ \tfrac{1}{2} \nabla_{\nu}\nabla_{\mu}h
\nonumber\\ \nonumber\\
\delta G &=&  \nabla^\alpha \nabla^\beta h_{\alpha\beta} - \nabla_\alpha\nabla^\alpha h
\nonumber\\\nonumber\\
\Delta_{\mu\nu} &=& \delta G_{\mu\nu}-3k h_{\mu\nu}
\nonumber\\ \nonumber\\
\Delta &=& \delta G-3k h
\nonumber\\\nonumber\\
\Delta^{T\theta}_{\mu\nu} &=&  \Delta_{\mu\nu} - \frac{1}{3}g_{\mu\nu} \Delta + \frac{1}{3}\left( \nabla_\mu\nabla_\nu +kg_{\mu\nu}\right)\int D \Delta
\nonumber\\ \nonumber\\
(\nabla^2 -4k)\Delta^{T\theta}_{\mu\nu} &=& (\nabla^2 -4k)\Delta_{\mu\nu} +
\frac{1}{3}\left[ \nabla_\mu\nabla_\nu + 3k g_{\mu\nu} - g_{\mu\nu}\nabla^2\right] \Delta
\nonumber\\ \nonumber\\
(\nabla^2 -4k)\Delta^{T\theta}_{\mu\nu} &=& 
4 k^2 h_{\mu \nu}
-  k^2 g_{\mu \nu} h
- 3 k \nabla_{\alpha}\nabla^{\alpha}h_{\mu \nu}
+ \tfrac{1}{2} k g_{\mu \nu} \nabla_{\alpha}\nabla^{\alpha}h
+ k g_{\mu \nu} \nabla_{\beta}\nabla_{\alpha}h^{\alpha \beta}\nonumber\\
&& + \tfrac{1}{2} \nabla_{\beta}\nabla^{\beta}\nabla_{\alpha}\nabla^{\alpha}h_{\mu \nu}
-  \tfrac{1}{6} g_{\mu \nu} \nabla_{\beta}\nabla^{\beta}\nabla_{\alpha}\nabla^{\alpha}h
+ \tfrac{1}{6} g_{\mu \nu} \nabla_{\gamma}\nabla^{\gamma}\nabla_{\beta}\nabla_{\alpha}h^{\alpha \beta}
-  \tfrac{1}{2} k \nabla_{\mu}\nabla_{\alpha}h_{\nu}{}^{\alpha}\nonumber\\
&& -  \tfrac{1}{2} \nabla_{\mu}\nabla_{\beta}\nabla^{\beta}\nabla_{\alpha}h_{\nu}{}^{\alpha}
-  \tfrac{1}{2} k \nabla_{\nu}\nabla_{\alpha}h_{\mu}{}^{\alpha}
-  \tfrac{1}{2} \nabla_{\nu}\nabla_{\beta}\nabla^{\beta}\nabla_{\alpha}h_{\mu}{}^{\alpha}
+ k \nabla_{\nu}\nabla_{\mu}h\nonumber\\
&& + \tfrac{1}{6} \nabla_{\nu}\nabla_{\mu}\nabla_{\alpha}\nabla^{\alpha}h
+ \tfrac{1}{3} \nabla_{\nu}\nabla_{\mu}\nabla_{\beta}\nabla_{\alpha}h^{\alpha \beta}
\nonumber\\
&=& \delta W_{\mu\nu}
\end{eqnarray}
%
%
%%%%%%%%%%%%%%%%%%%%%%%%%%%%%%%%
\subsection{Gauge Transformation}
%%%%%%%%%%%%%%%%%%%%%%%%%%%%%%%%
Background:
\begin{eqnarray}
G^{(0)}_{\mu\nu} &=& 3kg_{\mu\nu}
\end{eqnarray}

Gravitational Invariant:
\begin{eqnarray}
\Delta_{\mu\nu} &=& \delta G_{\mu\nu} - 3kh_{\mu\nu}
\end{eqnarray}

Under $x^\mu \to x'^{\mu} = x^\mu -\epsilon^\mu(x)$,

\begin{eqnarray}
\delta \bar W_{\mu\nu} &=& \delta W_{\mu\nu} + W^{(0)}_{\rho\mu}g^{\lambda\rho} \nabla_\nu \epsilon_\lambda + W^{(0)}_{\rho\nu}g^{\lambda\rho} \nabla_\mu \epsilon_\lambda + \epsilon^\lambda\nabla_\lambda W^{(0)}_{\mu\nu} 
\nonumber\\
&=&0
\nonumber\\\nonumber\\
\delta \bar G_{\mu\nu} &=& \delta G_{\mu\nu} + G^{(0)}_{\rho\mu}g^{\lambda\rho} \nabla_\nu \epsilon_\lambda + G^{(0)}_{\rho\nu}g^{\lambda\rho} \nabla_\mu \epsilon_\lambda + \epsilon^\lambda\nabla_\lambda G^{(0)}_{\mu\nu} 
\nonumber\\
&=&\delta G_{\mu\nu} + 3k(\nabla_\nu\epsilon_\mu + \nabla_\mu\epsilon_\nu)
\nonumber\\\nonumber\\
\bar\Delta _{\mu\nu} &=& \delta G_{\mu\nu} + 3k(\nabla_\nu\epsilon_\mu+\nabla_\mu\epsilon_\nu) -3kh_{\mu\nu} - 3k(\nabla_\nu\epsilon_\mu+\nabla_\mu\epsilon_\nu) 
\nonumber\\
&=& \Delta_{\mu\nu} 
\end{eqnarray}
%
%
%%%%%%%%%%%%%%%%%%%%%%%%%%%%%%%%
\section{Conformal to Flat}
%%%%%%%%%%%%%%%%%%%%%%%%%%%%%%%%
\begin{eqnarray}
ds^2 &=& (\tilde g_{\mu\nu} + \delta \tilde g_{\mu\nu})dx^\mu dx^\nu
\nonumber\\
&=&\Omega^2(x)(\eta_{\mu\nu} + h_{\mu\nu})dx^\mu dx^\nu
\nonumber\\ \nonumber\\
\delta \tilde W_{\mu\nu} &=& \Omega^{-2} \delta W_{\mu\nu}
\nonumber\\
&=& \Omega^{-2}\nabla^2 \delta G^{T\theta}_{\mu\nu}
\end{eqnarray}
To be continued.
%%%%%%%%%%%%%%%%%%%%%%%%%%%%%%%%%%%
\section{Conformal to Flat}
%%%%%%%%%%%%%%%%%%%%%%%%%%%%%%%%%%%
In a conformal metric $\tilde g_{\mu\nu} = \Omega^2(x) g_{\mu\nu}$, the Einstein tensor transforms as
\begin{eqnarray}
\tilde G_{\mu\nu}(\tilde g_{\mu\nu}) &=& G_{\mu\nu}(g_{\mu\nu}) + S_{\mu\nu}(g_{\mu\nu})
\nonumber\\
&=& G_{\mu\nu}+\Omega^{-1}\left( -2 g_{\mu\nu}\nabla_\alpha \nabla^\alpha \Omega + 2\nabla_\mu \nabla_\nu \Omega\right) +
\Omega^{-2}\left(  g_{\mu\nu} \nabla_\alpha \Omega \nabla^\alpha \Omega - 4 \nabla_\mu \Omega \nabla_\nu \Omega\right)
\nonumber\\ \nonumber\\
\tilde S_{\mu\nu}(\tilde g_{\mu\nu}) &=& \Omega^{-1}\left( -2 \tilde g_{\mu\nu}\tilde\nabla_\alpha \tilde\nabla^\alpha \Omega + 2\tilde\nabla_\mu\tilde\nabla_\nu \Omega\right) + 3\Omega^{-2} \tilde g_{\mu\nu} \tilde\nabla_\alpha \Omega \tilde\nabla^\alpha \Omega
\nonumber\\
&=& \Omega^{-1}\left( -2 g_{\mu\nu}\nabla_\alpha \nabla^\alpha \Omega + 2\nabla_\mu \nabla_\nu \Omega\right) +
\Omega^{-2}\left(  g_{\mu\nu} \nabla_\alpha \Omega \nabla^\alpha \Omega - 4 \nabla_\mu \Omega \nabla_\nu \Omega\right)
\nonumber\\ \nonumber\\
ds^2 &=& \Omega^2(x)( g_{\mu\nu} + h_{\mu\nu})
\nonumber\\ \nonumber\\
\delta \tilde G_{\mu\nu} &=& \delta G_{\mu\nu} + \delta S_{\mu\nu}
\nonumber\\
&=& \delta G_{\mu\nu} -2 h_{\mu \nu} \Omega^{-1} \nabla_{\alpha}\nabla^{\alpha}\Omega
+ \Omega^{-1} \nabla_{\alpha}\Omega \nabla^{\alpha}h_{\mu \nu}
-   g_{\mu \nu} \Omega^{-1} \nabla_{\alpha}\Omega \nabla^{\alpha}h
+ h_{\mu \nu} \Omega^{-2} \nabla_{\alpha}\Omega \nabla^{\alpha}\Omega\nonumber\\
&& + 2  g_{\mu \nu} \Omega^{-1} \nabla_{\alpha}\Omega \nabla_{\beta}h^{\alpha \beta}
-   g_{\mu \nu} h^{\alpha \beta} \Omega^{-2} \nabla_{\alpha}\Omega \nabla_{\beta}\Omega
+ 2  g_{\mu \nu} h_{\alpha \beta} \Omega^{-1} \nabla^{\beta}\nabla^{\alpha}\Omega\nonumber\\
&& -  \Omega^{-1} \nabla_{\alpha}\Omega \nabla_{\mu}h_{\nu}{}^{\alpha}
-  \Omega^{-1} \nabla_{\alpha}\Omega \nabla_{\nu}h_{\mu}{}^{\alpha}.
\end{eqnarray}
Note that in the transformation of $G_{\mu\nu}$, all curvature tensors ($R_{\mu\nu}$, $R$) are contained within $G_{\mu\nu}$ and not $S_{\mu\nu}$. 
\begin{eqnarray}
ds^2 &=& \Omega^2(x)(\eta_{\mu\nu} + h_{\mu\nu})dx^\mu dx^\nu
\nonumber\\ \nonumber\\
\delta \tilde W_{\mu\nu}&=&\Omega^{-2} \delta W_{\mu\nu}
\nonumber\\
&=& \Omega^{-2} \nabla^2 \delta G_{\mu\nu}^{T\theta}
\end{eqnarray}

\begin{eqnarray}
G_{\mu\nu} &=& U_\mu U_\nu (-2H^2 + 2\dot H)+ g_{\mu\nu}(H^2+2\dot H)
\end{eqnarray}
\begin{eqnarray}
\Delta_\epsilon G_{\mu\nu} &=& G^\lambda{}_\mu \nabla_\nu \epsilon_\lambda + G^\lambda{}_\nu \nabla_\mu \epsilon_\lambda + \epsilon^\lambda\nabla_\lambda G_{\mu\nu} 
\end{eqnarray}
\begin{eqnarray}
\nabla_\mu \epsilon_\lambda &=& \Omega^2 \nabla_\mu f_\lambda + \Omega(f_\lambda \nabla_\mu - f_\mu \nabla_\lambda + g_{\mu\lambda}f^\rho \nabla_\rho)\Omega
\end{eqnarray}
\begin{eqnarray}
\tilde \nabla^\mu \tilde G_{\mu\nu} &=& \Omega^{-2}\nabla^\mu \tilde G_{\mu\nu} +\Omega^{-3}\left( 
2\tilde G_{\mu\nu}\nabla^\mu \Omega - \tilde G \nabla_\nu \Omega\right)
\nonumber\\
&=& \Omega^{-2}\nabla^\mu  G_{\mu\nu} +\Omega^{-3}\left( 
2 G_{\mu\nu}\nabla^\mu \Omega -  G \nabla_\nu \Omega\right)
+\Omega^{-2}\nabla^\mu  S_{\mu\nu} +\Omega^{-3}\left( 
2 S_{\mu\nu}\nabla^\mu \Omega - S \nabla_\nu \Omega\right)
\end{eqnarray}
Taking $G_{\mu\nu} = 0$,
\begin{eqnarray}
\tilde \nabla^\mu \tilde G_{\mu\nu} &=& \Omega^{-2}\nabla^\mu  S_{\mu\nu} +\Omega^{-3}\left( 
2 S_{\mu\nu}\nabla^\mu \Omega - S \nabla_\nu \Omega\right)
\nonumber\\
&=&0
\nonumber\\
\to  \quad\nabla^\mu S_{\mu\nu} &=& \Omega^{-1}\left( S\nabla_\nu \Omega - 2S_{\mu\nu}\nabla^\mu \Omega\right) 
\end{eqnarray}
\begin{eqnarray}
\tilde T_{\mu\nu} &=& \tilde S_{\mu\nu}
\nonumber\\
\tilde U_{\mu}\tilde U_{\nu}(\tilde\rho+\tilde p) + \tilde g_{\mu\nu} \tilde p &=&  \Omega^{-1}\left( -2 \tilde g_{\mu\nu}\tilde\nabla_\alpha \tilde\nabla^\alpha \Omega + 2\tilde\nabla_\mu\tilde\nabla_\nu \Omega\right) + 3\Omega^{-2} \tilde g_{\mu\nu} \tilde\nabla_\alpha \Omega \tilde\nabla^\alpha \Omega
\end{eqnarray}
\begin{eqnarray}
\Delta^{T\theta}_{\mu\nu} &=& \Delta_{\mu\nu} -\frac13 \tilde g_{\mu\nu}\Delta +\frac13 \left[ \tilde\nabla_\mu\tilde\nabla_\nu + \Omega^{-1}(\tilde\nabla_\mu\Omega \tilde\nabla_\nu + \tilde\nabla_\nu\Omega \tilde\nabla_\mu) -\Omega^{-1}\tilde g_{\mu\nu} \tilde\nabla^\alpha \Omega \tilde\nabla_\alpha \right]\int g^{1/2} D(x,x') \Omega^2 \Delta
\end{eqnarray}
\begin{eqnarray}
\left( \Omega^2 \tilde\nabla_\alpha \tilde\nabla^\alpha - 2\Omega \nabla^\alpha \Omega \tilde\nabla_\alpha \right) D(x,x') &=&g^{-1/2}\delta^4(x-x')
\end{eqnarray}
\begin{eqnarray}
\nabla^\mu \delta G_{\mu\nu}^{T\theta} &=&0
\nonumber\\
\Rightarrow\quad  \Omega^2 \tilde\nabla^\mu \Delta_{\mu\nu}^{T\theta} - 2\Omega \tilde\nabla^\mu \Omega \Delta_{\mu\nu} &=& 0
\end{eqnarray}
%%%%%%%%%%%%%%%%%%%%%%%%%%%%%%%%%%
%
%
%%%%%%%%%%%%%%%%%%%%%%%%%%%%%%%%%%%%%%%%%%%%
\section{ Conformal Laplacian}
%%%%%%%%%%%%%%%%%%%%%%%%%%%%%%%%%%%%%%%%%%%%
\begin{eqnarray}
ds^2 &=& g_{\mu\nu} dx^\mu dx^\nu = \Omega^2(x) \tilde g_{\mu\nu} dx^\mu dx^\nu
\\ \nonumber\\
A \nabla_\alpha \nabla^\alpha \chi + B R^\alpha{}_\alpha \chi
&=& B R^\alpha{}_\alpha \chi \Omega^{-2} + A \Omega^{-2} \nabla_{\alpha }\nabla^{\alpha }\chi + 6 B \chi \Omega^{-3} \nabla_{\alpha }\nabla^{\alpha }\Omega + 2 A \Omega^{-3} \nabla_{\alpha }\Omega \nabla^{\alpha }\chi 
\nonumber\\ \\
\Omega^{F}\left[A \tilde\nabla_\alpha\tilde\nabla^\alpha( \Omega^H \chi) + B R^\alpha{}_\alpha (\Omega^H \chi)\right]
&=& B R^\alpha{}_\alpha \chi \Omega^{F + H}
+ A \Omega^{F + H} \nabla_{\alpha }\nabla^{\alpha }\chi
+ A H \chi \Omega^{-1 + F + H} \nabla_{\alpha }\nabla^{\alpha }\Omega
\nonumber\\
&&
+ 2 A H \Omega^{-1 + F + H} \nabla_{\alpha }\Omega \nabla^{\alpha }\chi -  A H \chi \Omega^{-2 + F + H} \nabla_{\alpha }\Omega \nabla^{\alpha }\Omega
\nonumber\\
&&
+ A H^2 \chi \Omega^{-2 + F + H} \nabla_{\alpha }\Omega \nabla^{\alpha }\Omega
\\ \nonumber\\
\implies && H=1,\qquad F=-3,\qquad A=6B
\\ \nonumber\\
6 \nabla_\alpha \nabla^\alpha \chi + R^\alpha{}_\alpha \chi
&=& \Omega^{-3} \left[ 6 \tilde \nabla_\alpha \tilde \nabla^\alpha (\Omega\chi) + R^\alpha{}_\alpha (\Omega \chi) \right]
\end{eqnarray}
In dimension $D$,
\begin{eqnarray}
\frac{4(D-1)}{D-2} \nabla_\alpha\nabla^\alpha \chi + R^\alpha{}_\alpha \chi
= \Omega^{-\left(\frac{D+2}{2}\right)}\left[
\frac{4(D-1)}{D-2}\tilde\nabla_\alpha\tilde\nabla^\alpha \left( \Omega^{\frac{D-2}{2}}\chi\right)
+R^\alpha{}_\alpha \left( \Omega^{\frac{D-2}{2}}\chi\right)
\right]
\end{eqnarray}
\newpage
\begin{appendices}
%
%
%
%%%%%%%%%%%%%%%%%%%%%%%%%%%%%%%%%%%%%%%%%%%%%%%%%%%%%%
\section{$h_{\mu\nu}^{T\theta}$}
%%%%%%%%%%%%%%%%%%%%%%%%%%%%%%%%%%%%%%%%%%%%%%%%%%%%%%

%%%%%%%%%%%%%%%%%%%%%%%%%%%%%%%%%%%%%%%%%%%%%%%%%%%%%%
\subsection{Minkowski}
%%%%%%%%%%%%%%%%%%%%%%%%%%%%%%%%%%%%%%%%%%%%%%%%%%%%%%
\begin{eqnarray}
h_{\mu\nu} &=& h_{\mu\nu}^{T\theta} + \nabla_\mu W_\nu + \nabla_\nu W_\mu - \frac{g_{\mu\nu}}{D-1}(\nabla^\sigma W_\sigma - h)
\nonumber\\
&& +\frac{2-D}{D-1} \nabla_\mu\nabla_\nu  \int D \nabla^\sigma W_\sigma
-\frac{1}{D-1}\nabla_\mu\nabla_\nu  \int D h
\label{decompmin}
\end{eqnarray}
with scalar Green's function
\begin{eqnarray}
 \nabla^\sigma \nabla_\sigma D(x,x') &=& \delta^4(x-x').
\end{eqnarray}
Taking the trace of \eqref{decompmin}, we find
\begin{eqnarray}
h &=& h.
\end{eqnarray}
As for the transverse component we find  a condition upon vector $W_\nu$
\begin{eqnarray}
\nabla^\sigma h_{\nu\sigma}&=& \nabla^\sigma\nabla_\sigma W_\nu.
\end{eqnarray}
The particular integral solution for $W_\nu$ is
\begin{eqnarray}
W_\nu &=& \int D \nabla^\sigma h_{\mu\sigma}.
\end{eqnarray}
If decompose a $T_{\mu\nu}$ that is apriori transverse, then with $W_\mu =0$ the decomposition reduces to
\begin{eqnarray}
T^{T\theta}_{\mu\nu} &=& T_{\mu\nu} - \frac{g_{\mu\nu}}{D-1}T + \frac{1}{D-1}\nabla_\mu\nabla_\nu \int D T
\label{consmin}
\end{eqnarray}
To bring into a local form, we apply the box operator
\begin{eqnarray}
\nabla^2 T^{T\theta}_{\mu\nu} &=& \nabla^2 T_{\mu\nu} + \frac{1}{D-1}\left[ 
\nabla_\mu\nabla_\nu - g_{\mu\nu}\nabla^2\right]T
\label{boxconsmin}
\end{eqnarray}
%
%
%


%%%%%%%%%%%%%%%%%%%%%%%%%%%%%%%%%%%%%%%%%%%%%%%%%%%%%%
\subsection{Maximally Symmetric}
%%%%%%%%%%%%%%%%%%%%%%%%%%%%%%%%%%%%%%%%%%%%%%%%%%%%%%
Curvature Tensors:
\begin{eqnarray}
R_{\lambda\mu\nu\kappa} &=& k(g_{\mu\nu}g_{\lambda\kappa}-g_{\lambda\nu}g_{\mu\kappa})
\nonumber\\
R_{\mu\kappa} &=& k(1-D)g_{\mu\kappa} = \frac{R}{D}g_{\mu\kappa}
\nonumber\\
R&=& kD(1-D) 
\end{eqnarray}
Covariant Commutation:
\begin{eqnarray}
[\nabla^\sigma \nabla_\nu] W_\sigma &=& -R_{\nu}{}^\sigma W_\sigma = -\frac{R}{D}W_\nu
\nonumber\\
{[}\nabla^\mu \nabla_\mu, \nabla_\nu] V &=& -R_{\nu}{}^\mu \nabla_\mu V = -\frac{R}{D}\nabla_\nu V
\nonumber\\
{[}\nabla^2,\nabla_\mu\nabla_\nu]V &=& \frac{2 g_{\mu\nu}R}{D(D-1)}\nabla^2 V - \frac{2R}{D-1}\nabla_\mu\nabla_\nu V
\label{comms}
\end{eqnarray}
Decomposition:
\begin{eqnarray}
h_{\mu\nu} &=& h_{\mu\nu}^{T\theta} + \nabla_\mu W_\nu + \nabla_\nu W_\mu - \frac{g_{\mu\nu}}{D-1}(\nabla^\sigma W_\sigma - h)
\nonumber\\
&& +\frac{2-D}{D-1}\left( \nabla_\mu\nabla_\nu -\frac{ g_{\mu\nu}R}{D(D-1)}\right) \int D \nabla^\sigma W_\sigma
-\frac{1}{D-1}\left( \nabla_\mu\nabla_\nu -\frac{g_{\mu\nu}R}{D(D-1)}\right) \int D h
\label{decomp}
\end{eqnarray}
with scalar Green's function
\begin{eqnarray}
\left( \nabla^\sigma \nabla_\sigma  - \frac{R}{D-1}\right) D(x,x') &=& g^{-1/2} \delta^4(x-x').
\end{eqnarray}
Taking the trace of \eqref{decomp}, we find
\begin{eqnarray}
h &=& h.
\end{eqnarray}
As for the transverse component we find, upon applying covariant commutations \eqref{comms}, a condition upon vector $W_\nu$
\begin{eqnarray}
\nabla^\sigma h_{\nu\sigma}&=& \nabla^\sigma\nabla_\sigma W_\nu.
\end{eqnarray}
With the box operator mixing indices of $W_\nu$, the particular integral solution for $W_\nu$ involves a bi-tensor Green's function $F_{\sigma\rho'}$ which obeys
\begin{eqnarray}
\nabla^\alpha\nabla_\alpha F_{\sigma\rho'}(x,x') &=& g_{\sigma\rho'}g^{-1/2} \delta^4(x-x')
\end{eqnarray}
\begin{eqnarray}
W_\nu &=& \int F_\nu{}^{\rho'}\nabla^{\sigma'}h_{\rho'\sigma'}.
\end{eqnarray}
If a tensor $T_{\mu\nu}$ is apriori transverse, then we again may set $W_\mu =0$ to find for a conserved tensor, the decomposition
\begin{eqnarray}
T^{T\theta}_{\mu\nu} &=&  T_{\mu\nu} - \frac{g_{\mu\nu}}{D-1} T + \frac{1}{D-1}\left( \nabla_\mu\nabla_\nu - \frac{g_{\mu\nu}R}{D(D-1)}\right)\int D T.
\label{Ttt1}
\end{eqnarray}
We see that to retain transversality, we cannot simply just extract the trace in a trivial way.  
\\ \\
To form a second order equation for $T^{T\theta}_{\mu\nu}$ that is absent of the non-local integral, we need to apply a specific box operator. Acting upon a scalar $V$, the desired operator is given below with commutation relation
\begin{eqnarray}
\left( \nabla^2 + \frac{R}{D-1}\right) \left(\nabla_\mu\nabla_\nu - \frac{Rg_{\mu\nu}}{D(D-1)}\right)V &=& \left(\nabla_\mu\nabla_\nu + \frac{Rg_{\mu\nu}}{D(D-1)}\right)\left( \nabla^2 - \frac{R}{D-1}\right)V,
\end{eqnarray}
which may be verified using \eqref{comms}.
\\ \\
Now applying this operator to \eqref{Ttt1}, we find
\begin{eqnarray}
\left( \nabla^2 + \frac{R}{D-1}\right) T^{T\theta}_{\mu\nu} &=& \left( \nabla^2 + \frac{R}{D-1}\right) T_{\mu\nu} - \frac{g_{\mu\nu}}{D-1}\left( \nabla^2 + \frac{R}{D-1}\right)T
\nonumber\\
&&
+ \frac{1}{D-1}\left( \nabla_\mu\nabla_\nu + \frac{g_{\mu\nu}R}{D(D-1)}\right)T.
\end{eqnarray}
Expressed in terms of curvature constant $R = -kD(D-1)$, the above becomes
\begin{eqnarray}
(\nabla^2 -Dk)T^{T\theta}_{\mu\nu} &=& (\nabla^2 -Dk)T_{\mu\nu} +
\frac{1}{D-1}\left[ \nabla_\mu\nabla_\nu + (D-1)k g_{\mu\nu} - g_{\mu\nu}\nabla^2\right] T.
\label{Ttt2}
\end{eqnarray}
\end{appendices}
\end{document}