\documentclass[10pt,letterpaper]{article}
\usepackage[textwidth=7in, top=1in,textheight=9in]{geometry}
\usepackage[fleqn]{mathtools} 
\usepackage{amssymb}

\title{Bach External SVT}
\date{}
\begin{document}
\maketitle
\noindent 
Via orthogonal projection to the four velocity $U^\mu$, we may decompose a rank 2 $T_{\mu\nu}$ as
\begin{equation}
T_{\mu\nu} = (\rho+p)U_\mu U_\nu + p g_{\mu\nu} + U_\mu q_\nu + U_\nu q_\mu + \pi_{\mu\nu}
\end{equation}
where
\begin{equation}
	U^\mu q_{\mu} = 0,\qquad U^\nu \pi_{\mu\nu} = 0,\qquad \pi_{\mu\nu} = \pi_{\nu\mu},\qquad g^{\mu\nu}\pi_{\mu\nu} =U^\mu U^\nu \pi_{\mu\nu} = 0.
\end{equation}
We evaluate within a Minkowski background $g^{(0)}_{\mu\nu} = \eta_{\mu\nu}$. 
 \\ \\
Given $ T_{0i} = -q_i$, let us decompose the $q_i$ into longitudinal and transverse parts by introducing the scalar
\begin{equation}
Q = \int d^3y\ D(x-y)\tilde\nabla^i q_i.
\end{equation}
Now we can form the transverse piece as
\begin{equation}
q_i -  \tilde\nabla_i Q = Q_i,
\end{equation}
with it following that $\tilde\nabla^i Q_i = 0$. Additionally, we may decompose the 5 component $\pi_{\mu\nu}$ into a transverse traceless $\pi_{ij}$, a divergenceless $\pi_i$, and a scalar $\pi$ as
\begin{equation}
	\pi_{ij} = -\frac{2}{3} \delta_{ij}\tilde\nabla^k \tilde\nabla_k \pi  + 2\tilde\nabla_i\tilde\nabla_j \pi + \tilde\nabla_i \pi_j + \tilde\nabla_j \pi_i + \pi_{ij}^{T\theta}.
\end{equation}
Now $ T_{\mu\nu}$ can be expressed in the SVT form as
\begin{align}
 T_{00}  &= \rho,
\nonumber\\	
 T_{0i} &= -Q_i - \tilde\nabla_i Q,
\nonumber\\	
 T_{ij}  &= \delta_{ij}  p -\frac{2}{3} \delta_{ij}\tilde\nabla^k \tilde\nabla_k \pi + 2\tilde\nabla_i\tilde\nabla_j \pi + \tilde\nabla_i \pi_j + \tilde\nabla_j \pi_i + \pi_{ij}^{T\theta}.
\end{align} 
Such a $ T_{\mu\nu}$ must be covariantly conserved and thus must obey the four conditions
\begin{align}
-\partial_t\rho = &{} \tilde\nabla_i \tilde\nabla^i Q\\
0 = &{} \partial_t (Q^i + \tilde\nabla^i Q) + \tilde\nabla^i  p +\frac43 \tilde\nabla^i \tilde\nabla^k \tilde\nabla_k \pi + \tilde\nabla_k \tilde\nabla^k \pi^i.
\end{align}
In conformal gravity, such an energy momentum tensor must also be traceless and as such must obey
\begin{equation}
\rho = 3p.
\end{equation}
From the first condition, we may express $Q$ in terms of $\rho$ as
\begin{equation}
Q = -\int d^3y D^3(\mathbf x-\mathbf y) \partial_t  \rho.
\end{equation}
We may extract a scalar condition from the second transverse condition, which takes the form
\begin{equation}
0 = \tilde\nabla_a\tilde\nabla^a (\partial_t Q +  p + \frac43 \tilde\nabla_b\tilde\nabla^b \pi).
\end{equation}
This allows expression of $\pi$ as
\begin{equation}
\pi = \frac34 \int d^3y\ D(x-y) \left[ \int d^3z\ D(y-z) \partial_t ^2 \rho - p\right].
\end{equation}
Substitution of $\pi$ back into the transverse condition then yields a vector condition
\begin{equation}
0=\partial_t Q_i + \tilde\nabla_a\tilde\nabla^a \pi_i,
\end{equation}
from which we may solve $\pi_i$ as
\begin{equation}
\pi_i = -\int d^3y\ D(x-y)\partial_t Q_i.
\end{equation}
In total, we may express a $\delta T_{\mu\nu}$ in terms of $\rho$, $Q_i$ and $\pi_{ij}^{T\theta}$, totaling 5 components:
\begin{align}
 \delta T_{00}  &= \rho,
\nonumber\\	
 \delta T_{0i} &= -Q_i + \tilde\nabla_i \int d^3y D^3(\mathbf x-\mathbf y) \partial_t  \rho,
\nonumber\\	
\delta T_{ij}  &= \frac12 \left( \delta_{ij}\rho - \tilde\nabla_i \tilde\nabla_j \int d^3y\ D(x-y) \rho\right) - \frac12 \int d^3y\ D(x-y) \delta_{ij} \partial_t^2 \rho 
\nonumber\\
&\quad {}+  \frac32
\tilde\nabla_i\tilde\nabla_j  \int d^3y\ D(x-y) \int d^3z\ D(y-z) \partial_t^2 \rho - \tilde\nabla_i  \int d^3y\ D(x-y) \partial_t Q_j
\nonumber \\
&\quad{}
- \tilde\nabla_j  \int d^3y\ D(x-y) \partial_t Q_i + \pi_{ij}^{T\theta}.
\end{align} 
Likewise we may express a general $\delta W_{\mu\nu}$ in terms of the barred quantities 
\begin{align}
 \delta W_{00}  &= \bar \rho,
\nonumber\\	
 \delta W_{0i} &= -\bar Q_i + \tilde\nabla_i \int d^3y D^3(\mathbf x-\mathbf y) \partial_t  \bar \rho,
\nonumber\\	
 \delta W_{ij}  &= \frac12 \left( \delta_{ij}\bar \rho - \tilde\nabla_i \tilde\nabla_j \int d^3y\ D(x-y) \bar \rho\right) - \frac12 \int d^3y\ D(x-y) \delta_{ij} \partial_t^2 \bar \rho 
\nonumber\\
&\quad {}+  \frac32
\tilde\nabla_i\tilde\nabla_j  \int d^3y\ D(x-y) \int d^3z\ D(y-z) \partial_t^2 \bar \rho - \tilde\nabla_i  \int d^3y\ D(x-y) \partial_t \bar Q_j
\nonumber \\
&\quad{}
- \tilde\nabla_j  \int d^3y\ D(x-y) \partial_t \bar Q_i + \bar \pi_{ij}^{T\theta}.
\end{align} 
Solving for $\delta W_{00} =\delta T_{00}$ fixes $\rho$, and $\delta W_{0i} = \delta T_{0i}$ fixes $Q_i$ viz.
\begin{equation}
\bar\rho = \rho,\qquad
\bar Q_i = Q_i. 
\end{equation}
It then follows that these terms mutually cancel within $\delta W_{ij} = \delta T_{ij}$, leaving the remaing expression
\begin{equation}
\bar \pi_{ij}^{T\theta} = \pi_{ij}^{T\theta}.
\end{equation}
Thus we can express the entire $\delta W_{\mu\nu} = \delta T_{\mu\nu}$ field equation in terms of irreducible SO(3) equations as
\begin{align}
\bar \rho& = \rho
\nonumber\\
\bar Q_i &= Q_i
\nonumber\\
\bar \pi_{ij}^{T\theta} &= \pi_{ij}^{T\theta}.
\end{align}
\\ \\
We can try to express the above SVT relations in terms of the actual tensor components. Recall the flat 3+1 projector
\begin{equation}
P_{\mu\nu} = \eta_{\mu\nu}+U_{\mu}U_{\nu},\qquad U_{\mu} = -\delta^0_\mu,\qquad U^\mu = \delta^\mu_0.
\end{equation}
In terms of the the flat space projectors, the splitting of the 3+1 components goes as
\begin{equation}
\rho = U^\sigma U^\tau T_{\sigma\tau} = T_{00} ,\qquad q_{i} = -P_i{}^\sigma U^\tau T_{\sigma\tau} = -T_{0i},\qquad
\pi_{\mu\nu} = \left[ \frac12 P_\mu{}^\sigma P_\nu{}^\tau + \frac12 P_\nu{}^\sigma P_\mu{}^\tau - \frac13 P_{\mu\nu}P^{\sigma\tau}\right]T_{\sigma\tau},
\end{equation}
in which it follows 
\begin{equation}
\pi_{ij} = T_{ij} -\frac13 \delta_{ij} \delta^{kl}T_{kl}.
\end{equation}
We recall the definition of $Q_i$ as
\begin{equation}
Q_i = q_i - \tilde\nabla_i \int d^3y\ D(x-y)\tilde\nabla^i q_i.
\end{equation}
This may be alternatively expressed as
\begin{equation}
Q_i = -T_{0i} + \tilde\nabla_i \int d^3y\ D(x-y)\tilde\nabla^j T_{0j}
\end{equation}
Noting that $\pi_{ij}$ is already traceless by construction, we may project out its transverse part and define $\pi^{T\theta}_{ij}$ as
\begin{align}
\pi_{ij}^{T\theta} &= \pi_{ij} - \tilde\nabla_i \int d^3y\ D(x-y) \tilde\nabla^k \pi_{jk} - \tilde\nabla_j \int d^3y\ D(x-y) \tilde\nabla^k \pi_{ik}
\nonumber\\
&\qquad
+\tilde\nabla_i\tilde\nabla_j \int d^3y\ D(x-y) \tilde\nabla_k\ \int d^3z\ D(y-z) \tilde\nabla_l \pi^{kl}.
\end{align}
Substituting in $\pi_{ij} = T_{ij} -\frac13 \delta_{ij} \delta^{kl}T_{kl}$, we have
\begin{align}
\pi_{ij}^{T\theta} &=\left(T_{ij} -\frac13 \delta_{ij} \delta^{kl}T_{kl}\right) - \tilde\nabla_i \int d^3y\ D(x-y) \tilde\nabla^k \left(T_{jk} -\frac13 \delta_{jk} \delta^{mn}T_{mn}\right)
\nonumber\\
&\qquad
 - \tilde\nabla_j \int d^3y\ D(x-y) \tilde\nabla^k \left(T_{ik} -\frac13 \delta_{ik} \delta^{mn}T_{mn}\right)
\nonumber\\
&\qquad
+\tilde\nabla_i\tilde\nabla_j \int d^3y\ D(x-y) \tilde\nabla_k\ \int d^3z\ D(y-z) \tilde\nabla_l \left(T^{kl} -\frac13 \delta^{kl} \delta^{mn}T_{mn}\right).
\end{align}
In total, we may express relations (19) explicitly in terms of the components of the tensors as the following:
\begin{align}
 \bar \rho - \rho &= \delta W_{00} - \delta T_{00}\\
 \bar Q_i - Q_i & = -(\delta W_{0i}-\delta T_{0i}) + \tilde\nabla_i \int d^3y\ D(x-y)\tilde\nabla^j( \delta W_{0j}-\delta T_{0j})
\\
\bar \pi^{T\theta}_{ij} - \pi_{ij}^{T\theta}&= \left[\delta W_{ij} -\delta T_{ij} -\frac13 \delta_{ij} \delta^{kl}\left(\delta W_{kl}-\delta T_{kl}\right)\right] 
\nonumber\\
&- \tilde\nabla_i \int d^3y\ D(x-y) \tilde\nabla^k \left[\delta W_{jk}-\delta T_{jk} 
-\frac13 \delta_{jk} \delta^{mn}\left(\delta W_{mn}-\delta T_{mn}\right)\right]
\nonumber\\
&\qquad
 - \tilde\nabla_j \int d^3y\ D(x-y) \tilde\nabla^k \left[\delta W_{ij} -\delta T_{ik} -\frac13 \delta_{ik} \delta^{mn}\left(\delta W_{mn} -\delta T_{mn}\right)\right]
\nonumber\\
&\qquad
+\tilde\nabla_i\tilde\nabla_j \int d^3y\ D(x-y) \tilde\nabla_k\ \int d^3z\ D(y-z) \tilde\nabla_l \left[\delta W_{kl}-\delta T^{kl} -\frac13 \delta^{kl} \delta^{mn}\left(\delta W_{mn}-\delta T_{mn}\right)\right].
\end{align}
Now we explicitly evaluate $\delta W_{\mu\nu}$ in terms of its SVT metric components (see appendix for reference). The scalar portion $\rho$ takes the form
\begin{equation}
\rho =  -\frac{2}{3} \tilde{\nabla}_a\tilde{\nabla}^a\tilde{\nabla}_b\tilde{\nabla}^b (\phi + \psi +\dot{B}-\ddot{E}).
\end{equation}
The two component tranverse vector $Q_i$ evaluates to
\begin{align}
Q_i &= -\delta W_{0i} + \tilde\nabla_i \int d^3y\ D(x-y)\tilde\nabla^j \delta W_{0j}
\nonumber\\
&=  \frac{2}{3}\tilde{\nabla}_i\tilde{\nabla}_a\tilde{\nabla}^a\partial_t(\phi +\psi +\dot{B}-\ddot{E})
	-\frac{1}{2}\left[ \tilde{\nabla}_a\tilde{\nabla}^a\left(\tilde{\nabla}_b\tilde{\nabla}^b-\partial_t^2\right)(B_i - \dot{E}_i) \right]
\nonumber\\
&\quad-  \tilde\nabla_i \int d^3y\ D(x-y) \left[\frac{2}{3} \tilde{\nabla}_a\tilde{\nabla}^a\tilde{\nabla}_b\tilde{\nabla}^b \partial_t (\phi + \psi +\dot{B}-\ddot{E})\right].
\end{align}
Let us denote the scalar quantity
\begin{equation}
\Psi = \tilde{\nabla}_a\tilde{\nabla}^a (\phi +\psi +\dot{B}-\ddot{E}),
\end{equation}
then we may rewrite $Q_i$ in the simpler form
\begin{equation}
Q_i=-\frac{1}{2}\left[ \tilde{\nabla}_a\tilde{\nabla}^a\left(\tilde{\nabla}_b\tilde{\nabla}^b-\partial_t^2\right)(B_i - \dot{E}_i) \right]
+ \frac23\tilde\nabla_i \partial_t \left( \Psi -  \int d^3y\ D(x-y) \tilde\nabla_a\tilde\nabla^a \Psi\right).
\end{equation}
Looking at the form for $\Psi$, we recall that we can decompose any scalar into longitudinal and transverse components as
\begin{align}
\phi(x) &= \int d^3y\ D(x-y) \tilde\nabla_a\tilde\nabla^a \phi(y) + 
\int dS_a\left[ \phi(y)\tilde\nabla^a D(x-y) - D(x-y) \tilde\nabla^a \phi(y)\right]
\nonumber\\
&= \phi^L(x) + \phi^T(x).
\end{align}
Through the above decomposition, the only $\tilde\nabla_a\tilde\nabla^a \phi^L$ that vanishes is one for which $\phi^L$ itself vanish. Additionally, the transverse $\phi$ identically obeys $\tilde\nabla_a\tilde\nabla^a \phi^T = 0$. 
\\ \\
Upon analyzing $Q_i$, we see that in fact it is only $\Psi^T$ that contributes. We can show that $\Psi^T$ can be expressed solely as a surface integral (as given in (34)) if we perform an integration by parts. 
\\ \\
Moreover, looking back at our equation for $\rho$, we note that this may be expressed as
\begin{equation}
\rho  = -\frac23\tilde\nabla_a\tilde\nabla^a \Psi^L.
\end{equation}
Hence taking $\Psi = \tilde{\nabla}_a\tilde{\nabla}^a (\phi +\psi +\dot{B}-\ddot{E})$, we may express the scalar and vector equations as
\begin{align}
\rho &= -\frac23 \tilde\nabla_a\tilde\nabla^a \Psi^L
\\
Q_i&=-\frac{1}{2}\left[ \tilde{\nabla}_a\tilde{\nabla}^a\left(\tilde{\nabla}_b\tilde{\nabla}^b-\partial_t^2\right)(B_i - \dot{E}_i) \right]
+ \frac23\tilde\nabla_i \partial_t \Psi^T
\end{align}
................................
\\ Incomplete (Tensor Sector)\\
................................
\\ \\
\begin{align}
\pi_{ij}^{T\theta} &=\left(\delta W_{ij} -\frac13 \delta_{ij} \delta^{kl}\delta W_{kl}\right) - \tilde\nabla_i \int d^3y\ D(x-y) \tilde\nabla^k \left(\delta W_{jk} -\frac13 \delta_{jk} \delta^{mn}\delta W_{mn}\right)
\nonumber\\
&\qquad
 - \tilde\nabla_j \int d^3y\ D(x-y) \tilde\nabla^k \left(\delta W_{ik} -\frac13 \delta_{ik} \delta^{mn}\delta W_{mn}\right)
\nonumber\\
&\qquad
+\tilde\nabla_i\tilde\nabla_j \int d^3y\ D(x-y) \tilde\nabla_k\ \int d^3z\ D(y-z) \tilde\nabla_l \left(\delta W^{kl} -\frac13 \delta^{kl} \delta^{mn}\delta W_{mn}\right).
\end{align}
\begin{equation}
 \delta W_{ij}^{(S)} = \frac{1}{3}\bigg{[} \delta_{ij}\left(\partial_t^2-\tilde\nabla_a\tilde\nabla^a\right)\Psi +\tilde{\nabla}_i\tilde{\nabla}_j \left(\tilde{\nabla}_a\tilde{\nabla}^a-3\partial_t^2\right)\tilde\nabla^{-2}\Psi\bigg]
\end{equation}
For the transverse tensor contribution, the trace is
\begin{align}
\delta^{ij}\delta W_{ij} = -\frac23 \tilde\nabla_a\tilde\nabla^a \Psi.
\end{align}
\begin{equation}
\left(\delta W_{ij}^{(S)} -\frac13 \delta_{ij} \delta^{kl}\delta W^{(S)}_{kl}\right) = \frac13\left[
\delta_{ij}\left( \partial_t^2 - \frac13 \tilde\nabla_a\tilde\nabla^a\right)\Psi +\tilde{\nabla}_i\tilde{\nabla}_j \left(\tilde{\nabla}_a\tilde{\nabla}^a-3\partial_t^2\right)\tilde\nabla^{-2}\Psi\right]
\end{equation}
\begin{equation}
\tilde\nabla^i\left(\delta W_{ij}^{(S)} -\frac13 \delta_{ij} \delta^{kl}\delta W^{(S)}_{kl}\right) = \frac23 \left(\frac13 \tilde\nabla_a\tilde\nabla^a - \partial_t^2\right)\tilde\nabla_j \Psi
\end{equation}
\begin{align}
\pi_{ij}^{T\theta} &= \delta W_{ij}^{(V)}+\delta W_{ij}^{(T)} + \frac13\left[
\delta_{ij}\left( \partial_t^2 - \frac13 \tilde\nabla_a\tilde\nabla^a\right)\Psi +\tilde{\nabla}_i\tilde{\nabla}_j \left(\tilde{\nabla}_a\tilde{\nabla}^a-3\partial_t^2\right)\tilde\nabla^{-2}\Psi\right]
\nonumber
\\ &\quad  - \frac 23 \tilde\nabla_i \int d^3y\ D(x-y)  \left(\frac13 \tilde\nabla_a\tilde\nabla^a - \partial_t^2\right)\tilde\nabla_j \Psi
\nonumber \\
&\quad  - \frac 23 \tilde\nabla_j \int d^3y\ D(x-y)  \left(\frac13 \tilde\nabla_a\tilde\nabla^a - \partial_t^2\right)\tilde\nabla_i \Psi
\nonumber\\
&\quad  
+\frac23 \tilde\nabla_i\tilde\nabla_j \int d^3y\ D(x-y) \tilde\nabla_k\ \int d^3z\ D(y-z)  \left(\frac13 \tilde\nabla_a\tilde\nabla^a - \partial_t^2\right)\tilde\nabla^k \Psi
\end{align}
Nonzero vector contributions also present in integrals. 
\newpage
\section*{Appendix}
\begin{eqnarray}
\delta W_{00}  &=& -\frac{2}{3} \tilde{\nabla}_a\tilde{\nabla}^a\tilde{\nabla}_b\tilde{\nabla}^b (\phi + \psi +\dot{B}-\ddot{E}),
\nonumber\\	
\delta W_{0i} &=&  -\frac{2}{3}\tilde{\nabla}_i\tilde{\nabla}_a\tilde{\nabla}^a\partial_t(\phi +\psi +\dot{B}-\ddot{E})
	+\frac{1}{2}\left[ \tilde{\nabla}_a\tilde{\nabla}^a\left(\tilde{\nabla}_b\tilde{\nabla}^b-\partial_t^2\right)(B_i - \dot{E}_i) \right],
\nonumber\\	
\delta W_{ij}  &=& \frac{1}{3}\bigg{[} \delta_{ij}\left(\partial_t^2-\tilde\nabla_a\tilde\nabla^a\right)\tilde{\nabla}_b\tilde{\nabla}^b (\phi+ \psi+\dot{B}-\ddot{E}) +\tilde{\nabla}_i\tilde{\nabla}_j \left(\tilde{\nabla}_a\tilde{\nabla}^a-3\partial_t^2\right)(\phi + \psi +\dot{B}-\ddot{E})\bigg]
\nonumber\\
&&+\frac{1}{2}\left[ \tilde{\nabla}_{a}\tilde{\nabla}^a \tilde{\nabla}_i   \partial_t(B_j - \dot{E}_j)+\tilde{\nabla}_{a}\tilde{\nabla}^a \tilde{\nabla}_j \partial_t(B_i - \dot{E}_i) - \tilde{\nabla}_i\partial_t^3(B_j - \dot{E}_j)-\tilde{\nabla}_j\partial_t^3(B_i - \dot{E}_i)\right]
\nonumber\\
&&+\left[\tilde{\nabla}_a\tilde{\nabla}^a-\partial_t^2\right]^2E_{ij}.
\end{eqnarray}
\end{document}