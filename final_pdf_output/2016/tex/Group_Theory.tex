\documentclass[10pt,letterpaper]{article}
\usepackage{mymacros}

\title{Group Theory Notes}
\author{}
\date{}

\begin{document}
\maketitle

\noindent A group is defined as a set of elements that 
\\
1. Are closed under composition (multiplication)
\\
2. Contain all inverses\\
3. Associative\\
4. Existence of identity\\
\\
If a group is parameterized by continuous parameters (such as $\theta$ in 2D rotations), then this is a Lie group. If two elements of a Lie group commute, the group is said to be abelian. Otherwise, non-abelian.
\\ \\
Infinitesimal transformations can be found by taking derivatives with respect to the continuous parameters. These infinitesimal generators form a group as well, and if elements under commutation lie within the vector space spanned by the group, then they are said for form a Lie algebra. \\
\\ 
Most often we take groups to be transformations acting on a vector space $V$. On such group is the general linear group
which consists of all \emph{invertible} linear operators acting on a space $V$
\[
	GL(V) \subset \mathcal L(V).
\] 
For a vector space with scalar field $C$ and dimension $n$, then $GL(V)$ may be represented by a subset of invertable
$n\times n$ matrices denoted
\[
	GL(n,C) \subset M_n(C).
\]
For $C=\mathbb C, \mathbb R$, we have the complex general linear group and real general linear group in $n$ dimensions respectively.\\
\\ 
If our vector space is equipped with a non-degenerate hermitian form (defined via an ``inner product"), there exist an important subset of $GL(V)$ called isometries, Isom$(V)$ which by definition preserve the inner product. For $T\in$ Isom$(V)$, 
\[
	(Tu,Tw) = (u,w)\  \forall u,w\in V.
\]
These form a group.
\\ \\
We can show that for a vector space in $n$ dimensions with an inner product space ($(v,v) >0\ \forall v\ne 0$), $T$ consists of all $n\times n$ matrices such that
\[
	T^{-1} = T^\dag
\]
where for real vector spaces this simplifies to $T^{-1} = T^T$. These groups correspond to $U(n)$ and $O(n)$
respectively. If we further restrict to matrices with determinant equal to unity, then we have $SU(n)$ and $SO(n)$
which are important subgroups. 
\\ **Real vector space nd hermitian forms are symmetric, called a metric**
\\
\\
For the Minkowski metric, the group Isom$(V)$ can be shown to be transformations that satisfy
\[
	T_\mu{}^\rho T_\nu{}^\sigma \eta_{\rho\sigma} = \eta_{\mu \nu}
\]
i.e. Lorentz transformations denoted as $O(n-1,1)$. The restricted Lorentz group in four dimensions
$SO(3,1)_\sigma$ is the restriction that for $A\in O(3,1)$
\[
	|A| = 1,\quad A_{11}>0.
\]
This preserves the orientation of space and time. Also, any element $A\in SO(3,1)_\sigma$ can be expressed as a rotation and a Lorentz boost $R'L$ where
\[
	R' = \bpm R&0\\0&1\epm
\]
for $R \in SO(3)$. 
 If we add the parity and time reversal transformations to the $SO(3,1)_\sigma$ group, we recover the full $O(3,1)$ group. 
\end{document}